\sect{चतुर्थः प्रश्नः}
\setcounter{anuvakam}{0}
\dnsub{तैत्तिरीयब्राह्मणे द्वितीयाष्टके चतुर्थः प्रपाठकः}

%2.4.1.1
जुष्टो॒ दमू॑ना॒ अति॑थिर्दुरो॒णे। इ॒मन्नो॑ य॒ज्ञमुप॑ याहि वि॒द्वान्। विश्वा॑ अग्नेऽभि॒युजो॑ वि॒हत्य॑। श॒त्रू॒य॒तामा भ॑रा॒ भोज॑नानि। अग्ने॒ शर्ध॑ मह॒ते सौभ॑गाय। तव॑ द्यु॒म्नान्यु॑त्त॒मानि॑ सन्तु। सञ्जास्प॒त्य सु॒यम॒मा कृ॑णुष्व। श॒त्रू॒य॒ताम॒भि ति॑ष्ठा॒ महा सि। अग्ने॒ यो नो॒ऽभितो॒ जन॑। वृको॒ वारो॒ जिघासति॥१॥

%2.4.1.2
तास्त्वं वृ॑त्रहं जहि। वस्व॒स्मभ्य॒मा भ॑र। अग्ने॒ यो नो॑ऽभि॒दास॑ति। स॒मा॒नो यश्च॒ निष्ट्य॑। इ॒ध्मस्ये॑व प्र॒क्षाय॑तः। मा तस्योच्छे॑षि॒ किञ्च॒न। त्वमि॑न्द्राभि॒भूर॑सि। दे॒वो विज्ञा॑तवीर्यः। वृ॒त्र॒हा पु॑रु॒चेत॑नः। अप॒ प्राच॑ इन्द्र॒ विश्वा अ॒मित्रान्॑॥२॥

%2.4.1.3
अपापा॑चो अभिभूते नुदस्व। अपोदी॑चो॒ अप॑शूराध॒रा च॑ ऊ॒रौ। यथा॒ तव॒ शर्म॒न्मदे॑म। तमिन्द्रं॑ वाजयामसि। म॒हे वृ॒त्राय॒ हन्त॑वे। स वृषा॑ वृष॒भो भु॑वत्। यु॒जे रथ॑ङ्ग॒वेष॑ण॒ हरि॑भ्याम्। उप॒ ब्रह्मा॑णि जुजुषा॒णम॑स्थुः। विबा॑धिष्टा॒स्य रोद॑सी महि॒त्वा। इन्द्रो॑ वृ॒त्राण्य॑प्र॒तीज॑घ॒न्वान्॥३॥

%2.4.1.4
ह॒व्य॒वाह॑मभिमाति॒षाहम्। र॒क्षो॒हणं॒ पृत॑नासु जि॒ष्णुम्। ज्योति॑ष्मन्त॒न्दीद्य॑तं॒ पुर॑न्धिम्। अ॒ग्नि स्वि॑ष्ट॒कृत॒मा हु॑वेम। स्वि॑ष्टमग्ने अ॒भि तत्पृ॑णाहि। विश्वा॑ देव॒ पृत॑ना अ॒भि ष्य। उ॒रुन्न॒ पन्थां प्रदि॒शन्विभा॑हि। ज्योति॑ष्मद्धेह्य॒जर॑न्न॒ आयु॑। त्वाम॑ग्ने ह॒विष्म॑न्तः। दे॒वं मर्ता॑स ईडते॥४॥

%2.4.1.5
मन्ये त्वा जा॒तवे॑दसम्। स ह॒व्या व॑क्ष्यानु॒षक्। विश्वा॑नि नो दु॒र्गहा॑ जातवेदः। सिन्धु॒न्न ना॒वा दु॑रि॒ताऽति॑ पर्‌षि। अग्ने॑ अत्रि॒वन्मन॑सा गृणा॒नः। अ॒स्माकं॑ बोध्यवि॒ता त॒नूनाम्। पू॒षा गा अन्वे॑तु नः। पू॒षा र॑क्ष॒त्वर्व॑तः। पू॒षा वाज सनोतु नः। पू॒षेमा आशा॒ अनु॑वेद॒ सर्वा॥५॥

%2.4.1.6
सो अ॒स्मा अभ॑यतमेन नेषत्। स्व॒स्ति॒दा अघृ॑णि॒ सर्व॑वीरः। अप्र॑युच्छन्पु॒र ए॑तु॒ प्रजा॒नन्। त्वम॑ग्ने स॒प्रथा॑ असि। जुष्टो॒ होता॒ वरेण्यः। त्वया॑ य॒ज्ञं वित॑न्वते। अ॒ग्नी रक्षासि सेधति। शु॒क्रशो॑चि॒रम॑र्त्यः। शुचि॑ पाव॒क ईड्य॑। अग्ने॒ रक्षा॑ णो॒ अह॑सः॥६॥

%2.4.1.7
प्रति॑ ष्म देव॒ रीष॑तः। तपि॑ष्ठैर॒जरो॑ दह। अग्ने॒ हसि॒ न्य॑त्रिणम्। दीद्य॒न्मर्त्ये॒ष्वा। स्वे क्षये॑ शुचिव्रत। आ वा॑त वाहि भेष॒जम्। वि वा॑त वाहि॒ यद्रप॑। त्व हि वि॒श्वभे॑षजः। दे॒वानान्दू॒त ईय॑से। द्वावि॒मौ वातौ॑ वातः॥७॥

%2.4.1.8
आ सिन्धो॒रा प॑रा॒वत॑। दक्षं॑ मे अ॒न्य आ॒वातु॑। परा॒न्यो वा॑तु॒ यद्रप॑। यद॒दो वा॑त ते गृ॒हे। अ॒मृत॑स्य नि॒धिर्\mbox{}हि॒तः। ततो॑ नो देहि जी॒वसे। ततो॑ नो धेहि भेष॒जम्। ततो॑ नो॒ मह॒ आव॑ह। वात॒ आवा॑तु भेष॒जम्। श॒म्भूर्म॑यो॒भूर्नो॑ हृ॒दे॥८॥

%2.4.1.9
प्र ण॒ आयूषि तारिषत्। त्वम॑ग्ने अ॒यासि॑। अ॒या सन्मन॑सा हि॒तः। अ॒या सन् ह॒व्यमू॑हिषे। अ॒या नो॑ धेहि भेष॒जम्। इ॒ष्टो अ॒ग्निराहु॑तः। स्वाहा॑कृतः पिपर्तु नः। स्व॒गा दे॒वेभ्य॑ इ॒दन्नम॑। कामो॑ भू॒तस्य॒ भव्य॑स्य। स॒म्राडेको॒ विरा॑जति॥९॥

%2.4.1.10
स इ॒दं प्रति॑ पप्रथे। ऋ॒तूनुत्सृ॑जते व॒शी। काम॒स्तदग्रे॒ सम॑वर्त॒ताधि॑। मन॑सो॒ रेत॑ प्रथ॒मं यदासीत्। स॒तो बन्धु॒मस॑ति॒ निर॑विन्दन्। हृ॒दि प्र॒तीष्या॑ क॒वयो॑ मनी॒षा। त्वया॑ मन्यो स॒रथ॑मारु॒जन्त॑। हर्‌ष॑माणासो धृष॒ता म॑रुत्वः। ति॒ग्मेष॑व॒ आयु॑धा स॒शिशा॑नाः। उप॒ प्रय॑न्ति॒ नरो॑ अ॒ग्निरू॑पाः॥१०॥

%2.4.1.11
म॒न्युर्भगो॑ म॒न्युरे॒वास॑ दे॒वः। म॒न्युर्‌होता॒ वरु॑णो वि॒श्ववे॑दाः। म॒न्युं विश॑ ईडते देव॒यन्ती। पा॒हि नो॑ मन्यो॒ तप॑सा॒ श्रमे॑ण। त्वम॑ग्ने व्रत॒भृच्छुचि॑। दे॒वा आसा॑दया इ॒ह। अग्ने॑ ह॒व्याय॒ वोढ॑वे। व्र॒तानुबिभ्र॑द्व्रत॒पा अदाभ्यः। यजा॑ नो दे॒वा अ॒जर॑ सु॒वीर॑। दध॒द्रत्ना॑नि सुविदा॒नो अ॑ग्ने। गो॒पा॒य नो॑ जी॒वसे॑ जातवेदः॥११॥\anuvakamend[जिघासत्य॒मित्राञ्जघ॒न्वानी॑डते॒ सर्वा॒ अह॑सो वातो हृ॒दे रा॑जत्य॒ग्निरू॑पाः सुविदा॒नो अ॑ग्न॒ एकं च]

%2.4.2.1
चक्षु॑षो हेते॒ मन॑सो हेते। वाचो॑ हेते॒ ब्रह्म॑णो हेते। यो मा॑ऽघा॒युर॑भि॒दास॑ति। तम॑ग्ने मे॒न्या मे॒निं कृ॑णु। यो मा॒ चक्षु॑षा॒ यो मन॑सा। यो वा॒चा ब्रह्म॑णाऽघा॒युर॑भि॒दास॑ति। तयाऽग्ने॒ त्वं मे॒न्या। अ॒मुम॑मे॒निं कृ॑णु। यत्किञ्चा॒सौ मन॑सा॒ यच्च॑ वा॒चा। य॒ज्ञैर्जु॒होति॒ यजु॑षा ह॒विर्भि॑॥१२॥

%2.4.2.2
तन्मृ॒त्युर्निर्\mbox{}ऋ॑त्या संविदा॒नः। पु॒रादि॒ष्टादाहु॑तीरस्य हन्तु। या॒तु॒धाना॒ निर्\mbox{}ऋ॑ति॒रादु॒रक्ष॑। ते अ॑स्य घ्न॒न्त्वनृ॑तेन स॒त्यम्। इन्द्रे॑षिता॒ आज्य॑मस्य मथ्नन्तु। मा तत्समृ॑द्धि॒ यद॒सौ क॒रोति॑। हन्मि॑ ते॒ऽहं कृ॒त ह॒विः। यो मे॑ घो॒रमची॑कृतः। अपाञ्चौ त उ॒भौ बा॒हू। अप॑नह्याम्या॒स्यम्॥१३॥

%2.4.2.3
अप॑ नह्यामि ते बा॒हू। अप॑ नह्याम्या॒स्यम्। अ॒ग्नेर्दे॒वस्य॒ ब्रह्म॑णा। सर्व॑न्तेऽवधिषं कृ॒तम्। पु॒राऽमुष्य॑ वषट्का॒रात्। य॒ज्ञन्दे॒वेषु॑ नस्कृधि। स्वि॑ष्टम॒स्माकं॑ भूयात्। माऽस्मान्प्राप॒न्नरा॑तयः। अन्ति॑ दू॒रे स॒तो अ॑ग्ने। भ्रातृ॑व्यस्याभि॒दास॑तः॥१४॥

%2.4.2.4
व॒ष॒ट्का॒रेण॒ वज्रे॑ण। कृ॒त्या ह॑न्मि कृ॒ताम॒हम्। यो मा॒ नक्त॒न्दिवा॑ सा॒यम्। प्रा॒तश्चाह्नो॑ नि॒पीय॑ति। अ॒द्या तमि॑न्द्र॒ वज्रे॑ण। भातृ॑व्यं पादयामसि। इन्द्र॑स्य गृ॒हो॑ऽसि॒ तन्त्वा। प्रप॑द्ये॒ सगु॒ साश्व॑। स॒ह यन्मे॒ अस्ति॒ तेन॑। ईडे॑ अ॒ग्निं वि॑प॒श्चितम्॥१५॥

%2.4.2.5
गि॒रा य॒ज्ञस्य॒ साध॑नम्। श्रु॒ष्टी॒वान॑न्धि॒तावा॑नम्। अग्ने॑ श॒केम॑ ते व॒यम्। यमं॑ दे॒वस्य॑ वा॒जिन॑। अति॒ द्वेषासि तरेम। अव॑तं मा॒ सम॑नसौ॒ समो॑कसौ। सचे॑तसौ॒ सरे॑तसौ। उ॒भौ माम॑वतञ्जातवेदसौ। शि॒वौ भ॑वतम॒द्य न॑। स्व॒यं कृ॑ण्वा॒नः सु॒गमप्र॑यावम्॥१६॥

%2.4.2.6
ति॒ग्मशृ॑ङ्गो वृष॒भः शोशु॑चानः। प्र॒त्न स॒धस्थ॒मनु॒ पश्य॑मानः। आ तन्तु॑म॒ग्निर्दि॒व्यन्त॑तान। त्वन्न॒स्तन्तु॑रु॒त सेतु॑रग्ने। त्वं पन्था॑ भवसि देव॒यान॑। त्वयाऽग्ने पृ॒ष्ठं व॒यमारु॑हेम। अथा॑ दे॒वैः स॑ध॒मादं॑ मदेम। उदु॑त्त॒मं मु॑मुग्धि नः। वि पाशं॑ मध्य॒मञ्चृ॑त। अवा॑ध॒मानि॑ जी॒वसे॥१७॥

%2.4.2.7
व॒य सो॑म व्र॒ते तव॑। मन॑स्त॒नूषु॒ बिभ्र॑तः। प्र॒जाव॑न्तो अशीमहि। इ॒न्द्रा॒णी दे॒वी सु॒भगा॑ सु॒पत्नी। उदशे॑न पति॒विद्ये॑ जिगाय। त्रि॒शद॑स्या ज॒घन॒य्योँज॑नानि। उ॒पस्थ॒ इन्द्र॒ स्थवि॑रं बिभर्ति। सेना॑ ह॒ नाम॑ पृथि॒वी ध॑नञ्ज॒या। वि॒श्वव्य॑चा॒ अदि॑ति॒ सूर्य॑त्वक्। इ॒न्द्रा॒णी दे॒वी प्रा॒सहा॒ ददा॑ना॥१८॥

%2.4.2.8
सा नो॑ दे॒वी सु॒हवा॒ शर्म॑ यच्छतु। आत्वा॑ऽहार्‌षम॒न्तर॑भूः। ध्रु॒वस्ति॒ष्ठावि॑चाचलिः। विश॑स्त्वा॒ सर्वा॑ वाञ्छन्तु। मा त्वद्रा॒ष्ट्रमधि॑ भ्रशत्। ध्रु॒वा द्यौर्ध्रु॒वा पृ॑थि॒वी। ध्रु॒वं विश्व॑मि॒दञ्जग॑त्। ध्रु॒वा ह॒ पर्व॑ता इ॒मे। ध्रु॒वो राजा॑ वि॒शाम॒यम्। इ॒हैवैधि॒ मा व्य॑थिष्ठाः॥१९॥

%2.4.2.9
पर्व॑त इ॒वावि॑चाचलिः। इन्द्र॑ इवे॒ह ध्रु॒वस्ति॑ष्ठ। इ॒ह रा॒ष्ट्रमु॑ धारय। अ॒भिति॑ष्ठ पृतन्य॒तः। अध॑रे सन्तु॒ शत्र॑वः। इन्द्र॑ इव वृत्र॒हा ति॑ष्ठ। अ॒पः क्षेत्रा॑णि स॒ञ्जय\sn{}। इन्द्र॑ एणमदीधरत्। ध्रु॒वन्ध्रु॒वेण॑ ह॒विषा। तस्मै॑ दे॒वा अधि॑ब्रवन्। अ॒यं च॒ ब्रह्म॑ण॒स्पति॑॥२०॥\anuvakamend[ह॒विर्भि॑रा॒स्य॑मभि॒ दास॑तो विप॒श्चित॒मप्र॑यावञ्जी॒वसे॒ ददा॑ना व्यथिष्ठा ब्रव॒न्नेकं च]

%2.4.3.1
जुष्टी॑ नरो॒ ब्रह्म॑णा वः पितृ॒णाम्। अक्ष॑मव्यय॒न्न किला॑रिषाथ। यच्छक्व॑रीषु बृह॒ता रवे॑ण। इन्द्रे॒ शुष्म॒मद॑धाथा वसिष्ठाः। पा॒व॒का न॒ सर॑स्वती। वाजे॑भिर्वा॒जिनी॑वती। य॒ज्ञं व॑ष्टु धि॒या व॑सुः। सर॑स्वत्य॒भिनो॑ नेषि॒ वस्य॑। मा प॑स्फरी॒ पय॑सा॒ मा न॒ आध॑क्। जु॒षस्व॑ नः स॒ख्या॑ वे॒श्या॑ च॥२१॥

%2.4.3.2
मा त्वक्षेत्रा॒ण्यर॑णानि गन्म। वृ॒ञ्जे ह॒विर्नम॑सा ब॒र्॒हिर॒ग्नौ। अया॑मि॒ स्रुग्घृ॒तव॑ती सुवृ॒क्तिः। अम्य॑क्षि॒ सद्म॒ सद॑ने पृथि॒व्याः। अश्रा॑यि य॒ज्ञः सूर्ये॒ न चक्षु॑। इ॒हार्वाञ्च॒मति॑ ह्वये। इन्द्रं॒ जैत्रा॑य॒ जेत॑वे। अ॒स्माक॑मस्तु॒ केव॑लः। अ॒र्वाञ्च॒मिन्द्र॑म॒मुतो॑ हवामहे। यो गो॒जिद्ध॑न॒जिद॑श्व॒जिद्यः॥२२॥

%2.4.3.3
इ॒मन्नो॑ य॒ज्ञं वि॑ह॒वे जु॑षस्व। अ॒स्य कु॑र्मो हरिवो मे॒दिन॑न्त्वा। असं॑मृष्टो जायसे मातृ॒वोः शुचि॑। म॒न्द्रः क॒विरुद॑तिष्ठो॒ विव॑स्वतः। घृ॒तेन॑ त्वा वर्धयन्नग्न आहुत। धू॒मस्ते॑ के॒तुर॑भवद्दि॒वि श्रि॒तः। अ॒ग्निरग्रे प्रथ॒मो दे॒वता॑नाम्। संया॑तानामुत्त॒मो विष्णु॑रासीत्। यज॑मानाय परि॒गृह्य॑ दे॒वान्। दी॒क्षये॒द ह॒विरा ग॑च्छतन्नः॥२३॥

%2.4.3.4
अ॒ग्निश्च॑ विष्णो॒ तप॑ उत्त॒मं म॒हः। दी॒क्षा॒पा॒लेभ्यो॒ऽवन॑त॒ हि श॑क्रा। विश्वैर्दे॒वैर्य॒ज्ञियै संविदा॒नौ। दी॒क्षाम॒स्मै यज॑मानाय धत्तम्। प्र तद्विष्णु॑ स्तवते वी॒र्या॑य। मृ॒गो न भी॒मः कु॑च॒रो गि॑रि॒ष्ठाः। यस्यो॒रुषु॑ त्रि॒षु वि॒क्रम॑णेषु। अधि॑ क्षि॒यन्ति॒ भुव॑नानि॒ विश्वा। नूमर्तो॑ दयते सनि॒ष्यन् यः। विष्ण॑व उरुगा॒याय॒ दाश॑त्॥२४॥

%2.4.3.5
प्र यः स॒त्राचा॒ मन॑सा॒ यजा॑तै। ए॒ताव॑न्त॒न्नर्य॑मा॒ विवा॑सात्। विच॑क्रमे पृथि॒वीमे॒ष ए॒ताम्। क्षेत्रा॑य॒ विष्णु॒र्मनु॑षे दश॒स्यन्। ध्रु॒वासो॑ अस्य की॒रयो॒ जना॑सः। उ॒रु॒क्षि॒ति सु॒जनि॑मा चकार। त्रिर्दे॒वः पृ॑थि॒वीमे॒ष ए॒ताम्। विच॑क्रमे श॒तर्च॑सं महि॒त्वा। प्र विष्णु॑रस्तु त॒वस॒स्तवी॑यान्। त्वे॒ष ह्य॑स्य॒ स्थवि॑रस्य॒ नाम॑॥२५॥

%2.4.3.6
होता॑रञ्चि॒त्रर॑थमध्व॒रस्य॑। य॒ज्ञस्य॑यज्ञस्य के॒तु रुश॑न्तम्। प्रत्य॑र्धिन्दे॒वस्य॑देवस्य म॒ह्ना। श्रि॒या त्व॑ग्निमति॑थिं॒ जना॑नाम्। आ नो॒ विश्वा॑भिरू॒तिभि॑ स॒जोषा। ब्रह्म॑ जुषा॒णो ह॑र्यश्व याहि। वरी॑वृज॒त्स्थवि॑रेभिः सुशिप्र। अ॒स्मे दध॒द्वृष॑ण॒ शुष्म॑मिन्द्र। इन्द्र॑ सुव॒र्॒षा ज॒नय॒न्नहा॑नि। जि॒गायो॒शिग्भि॒ पृत॑ना अभि॒ श्रीः॥२६॥

%2.4.3.7
प्रारो॑चय॒न्मन॑वे के॒तुमह्नाम्। अवि॑न्द॒ज्ज्योति॑र्बृह॒ते रणा॑य। अश्वि॑ना॒वव॑से॒ निह्व॑ये वाम्। आ नू॒नं या॑त सुकृ॒ताय॑ विप्रा। प्रा॒त॒र्यु॒क्तेन॑ सु॒वृता॒ रथे॑न। उ॒पाग॑च्छत॒मव॒साग॑तन्नः। अ॒वि॒ष्टन्धी॒ष्वश्वि॑ना न आ॒सु। प्र॒जाव॒द्रेतो॒ अह्र॑यन्नो अस्तु। आवान्तो॒के तन॑ये॒ तूतु॑जानाः। सु॒रत्ना॑सो दे॒ववी॑तिङ्गमेम॥२७॥

%2.4.3.8
त्व सो॑म॒ क्रतु॑भिः सु॒क्रतु॑र्भूः। त्वदन्दक्षै सु॒दक्षो॑ वि॒श्ववे॑दाः। त्वं वृषा॑ वृष॒त्वेभि॑र्महि॒त्वा। द्यु॒म्नेभि॑र्द्यु॒म्न्य॑भवो नृ॒चक्षा। अषा॑ढय्युँ॒त्सु पृत॑नासु॒ पप्रिम्। सु॒व॒र्॒षाम॒प्स्वां वृ॒जन॑स्य गो॒पाम्। भ॒रे॒षु॒जा सु॑क्षि॒ति सु॒श्रव॑सम्। जय॑न्त॒न्त्वामनु॑ मदेम सोम। भवा॑ मि॒त्रो न शेव्यो॑ घृ॒तासु॑तिः। विभू॑तद्युम्न एव॒ या उ॑ स॒प्रथा॥२८॥

%2.4.3.9
अधा॑ ते विष्णो वि॒दुषा॑ चि॒दृध्य॑। स्तोमो॑ य॒ज्ञस्य॒ राध्यो॑ ह॒विष्म॑तः। यः पू॒र्व्याय॑ वे॒धसे॒ नवी॑यसे। सु॒मज्जा॑नये॒ विष्ण॑वे॒ ददा॑शति। यो जा॒तम॒स्य म॑ह॒तो म॒हि ब्रवात्। सेदु॒ श्रवो॑भिर्यु॒ज्यं॑         चिद॒भ्य॑सत्। तमु॑ स्तोतारः पू॒र्व्यं यथा॑ वि॒द ऋ॒तस्य॑। गर्भ ह॒विषा॑ पिपर्तन। आऽस्य॑ जा॒नन्तो॒ नाम॑ चिद्विवक्तन। बृ॒हत्ते॑ विष्णो सुम॒तिं भ॑जामहे॥२९॥

%2.4.3.10
इ॒मा धा॒ना घृ॑त॒स्नुव॑। हरी॑ इ॒होप॑वक्षतः। इन्द्र सु॒खत॑मे॒ रथे। ए॒ष ब्र॒ह्मा प्रते॑म॒हे। वि॒दथे॑ शसिष॒ हरी। य ऋ॒त्विय॒ प्रते॑ वन्वे। व॒नुषो॑ हर्य॒तं मदम्। इन्द्रो॒ नाम॑ घृ॒तन्नयः। हरि॑भि॒श्चारु॒ सेच॑ते। श्रु॒तो ग॒ण आ त्वा॑ विशन्तु॥३०॥

%2.4.3.11
हरि॑वर्पस॒ङ्गिर॑। आच॑र्‌षणि॒प्रा वृ॑ष॒भो जना॑नाम्। राजा॑ कृष्टी॒नां पु॑रुहू॒त इन्द्र॑। स्तु॒तश्र॑व॒स्यन्नव॒सोप॑म॒द्रिक्। यु॒क्त्वा हरी॒ वृष॒णायाह्य॒र्वाङ्। प्र यत्सिन्ध॑वः प्रस॒वं यदाय\sn{}। आप॑ समु॒द्र र॒थ्ये॑व जग्मुः। अत॑श्चि॒दिन्द्र॒ सद॑सो॒ वरी॑यान्। यदी॒ सोम॑ पृ॒णति॑ दु॒ग्धो अ॒शुः। ह्वया॑मसि॒ त्वेन्द्र॑ या॒ह्य॑र्वाङ्॥३१॥

%2.4.3.12
अर॑न्ते॒ सोम॑स्त॒नुवे॑ भवाति। शत॑क्रतो मा॒दय॑स्वा सु॒तेषु॑। प्रास्मा अ॑व॒ पृत॑नासु॒ प्रयु॒त्सु। इन्द्रा॑य॒ सोमा प्र॒दिवो॒ विदा॑नाः। ऋ॒भुर्येभि॒र्वृष॑पर्वा॒ विहा॑याः। प्र॒य॒म्यमा॑णा॒न्प्रति॒ षू गृ॑भाय। इन्द्र॒ पिब॒ वृष॑धूतस्य॒ वृष्ण॑। अहे॑डमान॒ उप॑याहि य॒ज्ञम्। तुभ्यं॑ पवन्त॒ इन्द॑वः सु॒तास॑। गावो॒ न व॑ज्रिन्त्स्व॒मोको॒ अच्छ॑॥३२॥

%2.4.3.13
इन्द्रा ग॑हि प्रथ॒मो य॒ज्ञिया॑नाम्। या ते॑ का॒कुत्सुकृ॑ता॒ या वरि॑ष्ठा। यया॒ शश्व॒त्पिब॑सि॒ मध्व॑ ऊ॒र्मिम्। तया॑ पाहि॒ प्र ते॑ अध्व॒र्युर॑स्थात्। सन्ते॒ वज्रो॑ वर्ततामिन्द्र ग॒व्युः। प्रा॒त॒र्युजा॒ वि बो॑धय। अश्वि॑ना॒वेह ग॑च्छतम्। अ॒स्य सोम॑स्य पी॒तये। प्रा॒त॒र्यावा॑णा प्रथ॒मा य॑जध्वम्। पु॒रा गृध्रा॒दर॑रुषः पिबाथः। प्रा॒तर्\mbox{}हि य॒ज्ञम॒श्विना॒ दधा॑ते। प्रशसन्ति क॒वय॑ पूर्व॒भाज॑। प्रा॒तर्य॑जध्वम॒श्विना॑ हिनोत। न सा॒यम॑स्ति देव॒या अजु॑ष्टम्। उ॒तान्यो अ॒स्मद्य॑जते॒ विचा॑यः। पूर्व॑ पूर्वो॒ यज॑मानो॒ वनी॑यान्॥३३॥\anuvakamend[चा॒श्व॒जिद्यो ग॑च्छतन्नो॒ दाश॒न्नामा॑भि॒श्रीर्ग॑मेम स॒प्रथा॑ भजामहे विशन्तु या॒ह्य॑र्वाङच्छ॑ पिबाथ॒ष्षट्च॑]

%2.4.4.1
न॒क्तं॒ जा॒ताऽस्यो॑षधे। रामे॒ कृष्णे॒ असि॑क्नि च। इ॒द र॑जनि रजय। कि॒लासं॑ पलि॒तं च॒ यत्। कि॒लास॑ञ्च पलि॒तं च॑। निरि॒तो ना॑शया॒ पृष॑त्। आ न॒ स्वो अ॑श्ञुतां॒ वर्ण॑। परा श्वे॒तानि॑ पातय। असि॑तन्ते नि॒लय॑नम्। आ॒स्थान॒मसि॑त॒न्तव॑॥३४॥

%2.4.4.2
असि॑क्नियस्योषधे। निरि॒तो ना॑शया॒ पृष॑त्। अ॒स्थि॒जस्य॑ कि॒लास॑स्य। त॒नू॒जस्य॑ च॒ यत्त्व॒चि। कृ॒त्यया॑ कृ॒तस्य॒ ब्रह्म॑णा। लक्ष्म॑ श्वे॒तम॑नीनशम्। सरू॑पा॒ नाम॑ ते मा॒ता। सरू॑पो॒ नाम॑ ते पि॒ता। सरू॑पाऽस्योषधे॒ सा। सरू॑पमि॒दं कृ॑धि॥३५॥

%2.4.4.3
शु॒न हु॑वेम म॒घवा॑न॒मिन्द्रम्। अ॒स्मिन्भरे॒ नृत॑मं॒ वाज॑सातौ। शृ॒ण्वन्त॑मु॒ग्रमू॒तये॑ स॒मत्सु॑। घ्नन्तं॑ वृ॒त्राणि॑ स॒ञ्जितं॒ धना॑नाम्। धू॒नु॒थ द्यां पर्व॑तान्दा॒शुषे॒ वसु॑। नि वो॒ वना॑ जिहते॒ याम॑ नो भि॒या। को॒पय॑थ पृथि॒वीं पृ॑श्ञिमातरः। यु॒धे यदु॑ग्रा॒ पृष॑ती॒रयु॑ग्ध्वम्। प्रवे॑पयन्ति॒ पर्व॑तान्। विवि॑ञ्चन्ति॒ वन॒स्पतीन्॑॥३६॥

%2.4.4.4
प्रोवा॑रत मरुतो दु॒र्मदा॑ इव। देवा॑स॒ सर्व॑या वि॒शा। पु॒रु॒त्रा हि स॒दृङ्ङसि॑। विशो॒ विश्वा॒ अनु॑ प्र॒भु। स॒मत्सु॑ त्वा हवामहे। स॒मत्स्व॒ग्निमव॑से। वा॒ज॒यन्तो॑ हवामहे। वाजे॑षु चि॒त्ररा॑धसम्। सङ्ग॑च्छध्व॒ संव॑दध्वम्। सव्वोँ॒ मनासि जानताम्॥३७॥

%2.4.4.5
दे॒वा भा॒गं यथा॒ पूर्वे। स॒ञ्जा॒ना॒ना उ॒पास॑त। स॒मा॒नो मन्त्र॒ समि॑तिः समा॒नी। स॒मा॒नं मन॑ स॒ह चि॒त्तमे॑षाम्। स॒मा॒नङ्केतो॑ अ॒भि स र॑भध्वम्। सं॒ज्ञाने॑न वो ह॒विषा॑ यजामः। स॒मा॒नी व॒ आकू॑तिः। स॒मा॒ना हृद॑यानि वः। स॒मा॒नम॑स्तु वो॒ मन॑। यथा॑ व॒ सुस॒हास॑ति॥३८॥

%2.4.4.6
सं॒ज्ञान॑न्न॒ स्वैः। सं॒ज्ञान॒मर॑णैः। सं॒ज्ञान॑मश्विना यु॒वम्। इ॒हास्मासु॒ निय॑च्छतम्। सं॒ज्ञानं॑ मे॒ बृह॒स्पति॑। सं॒ज्ञान सवि॒ता क॑रत्। सं॒ज्ञान॑मश्विना यु॒वम्। इ॒ह मह्यं॒ नि य॑च्छतम्। उप॑ च्छा॒यामि॑व॒ घृणे। अग॑न्म॒ शर्म॑ ते व॒यम्॥३९॥

%2.4.4.7
अग्ने॒ हिर॑ण्यसन्दृशः। अद॑ब्धेभिः सवितः पा॒युभि॒ष्ट्वम्। शि॒वेभि॑र॒द्य परि॑पाहि नो॒ गयम्। हिर॑ण्यजिह्वः सुवि॒ताय॒ नव्य॑से। रक्षा॒ माकि॑र्नो अ॒घशस ईशत। मदे॑मदे॒ हि नो॑ द॒दुः। यू॒था गवा॑मृजु॒क्रतु॑। सङ्गृ॑भाय पु॒रूश॒ता। उ॒भ॒या ह॒स्त्या वसु॑। शि॒शी॒हि रा॒य आ भ॑र॥४०॥

%2.4.4.8
शिप्रि॑न्वाजानां पते। शची॑व॒स्तव॑ द॒सना। आ तू न॑ इन्द्र भाजय। गोष्वश्वे॑षु शु॒भ्रुषु॑। स॒हस्रे॑षु तुवीमघ। यद्दे॑वा देव॒हेड॑नम्। देवा॑सश्चकृ॒मा व॒यम्। आदि॑त्या॒स्तस्मान्मा यू॒यम्। ऋ॒तस्य॒र्तेन॑ मुञ्चत। ऋ॒तस्य॒र्तेना॑दित्याः॥४१॥

%2.4.4.9
यज॑त्रा मु॒ञ्चते॒ह मा। य॒ज्ञैर्वो॑ यज्ञवाहसः। आ॒शिक्ष॑न्तो॒ न शे॑किम। मेद॑स्वता॒ यज॑मानाः। स्रु॒चाऽऽज्ये॑न॒ जुह्व॑तः। अ॒का॒मा वो॑ विश्वेदेवाः। शिक्ष॑न्तो॒ नोप॑ शेकिम। यदि॒ दिवा॒ यदि॒ नक्तम्। एन॑ एन॒स्योक॑रत्। भू॒तं मा॒ तस्मा॒द्भव्यं॑ च॥४२॥

%2.4.4.10
द्रु॒प॒दादि॑व मुञ्चतु। द्रु॒प॒दादि॒वेन्मु॑मुचा॒नः। स्वि॒न्नः स्ना॒त्वी मला॑दिव। पू॒तं प॒वित्रे॑णे॒वाज्यम्। विश्वे॑ मुञ्चन्तु॒ मैन॑सः। उद्व॒यन्तम॑स॒स्परि॑। पश्य॑न्तो॒ ज्योति॒रुत्त॑रम्। दे॒वन्दे॑व॒त्रा सूर्यम्। अग॑न्म॒ ज्योति॑रुत्त॒मम्॥४३॥\anuvakamend[तव॑ कृधि॒ वन॒स्पतीञ्जानता॒मस॑ति व॒यं भ॑रादित्याश्च॒ नव॑ च]

%2.4.5.1
वृषा॒सो अ॒शुः प॑वते ह॒विष्मा॒न्त्सोम॑। इन्द्र॑स्य भा॒ग ऋ॑त॒युः श॒तायु॑। स मा॒ वृषा॑णं वृष॒भं कृ॑णोतु। प्रि॒यं वि॒शा सर्व॑वीर सु॒वीरम्। कस्य॒ वृषा॑ सु॒ते सचा। नि॒युत्वान्वृष॒भो र॑णत्। वृ॒त्र॒हा सोम॑पीतये। यस्ते॑ शृङ्ग वृषोनपात्। प्रण॑पात्कुण्ड॒पाय्य॑। न्य॑स्मिन्दध्र॒ आ मन॑॥४४॥

%2.4.5.2
त स॒ध्रीची॑रू॒तयो॒ वृष्णि॑यानि। पौस्या॑नि नि॒युत॑ सश्चु॒रिन्द्रम्। स॒मु॒द्रन्न सिन्ध॑व उ॒क्थशु॑ष्माः। उ॒रु॒व्यच॑स॒ङ्गिर॒ आ वि॑शन्ति। इन्द्रा॑य॒ गिरो॒ अनि॑शितसर्गाः। अ॒पः प्रैर॑य॒न्त्सग॑रस्य बु॒ध्नात्। यो अक्षे॑णेव च॒क्रिया॒ शची॑भिः। विष्व॑क्त॒स्तम्भ॑ पृथि॒वीमु॒त द्याम्। अक्षो॑दय॒च्छव॑सा॒ क्षाम॑बु॒ध्नम्। वार्णवा॑त॒स्तवि॑षीभि॒रिन्द्र॑॥४५॥

%2.4.5.3
दृ॒ढान्यौघ्नादु॒शमा॑न॒ ओज॑। अवा॑भिनत्क॒कुभ॒ पर्व॑तानाम्। आ नो॑ अग्ने सुके॒तुना। र॒यिं वि॒श्वायु॑पोषसम्। मा॒र्डी॒कन्धे॑हि जी॒वसे। त्व सो॑म म॒हे भगम्। त्वं यून॑ ऋताय॒ते। दक्ष॑न्दधासि जी॒वसे। रथ॑य्युँञ्जते म॒रुत॑ शु॒भे सु॒गम्। सूरो॒ न मि॑त्रावरुणा॒ गवि॑ष्टिषु॥४६॥

%2.4.5.4
रजासि चि॒त्रा विच॑रन्ति त॒न्यव॑। दि॒वः स॑म्राजा॒ पय॑सा न उक्षतम्। वाच॒ सुमि॑त्रावरुणा॒विरा॑वतीम्। प॒र्जन्य॑श्चि॒त्रां व॑दति॒ त्विषी॑मतीम्। अ॒भ्रा व॑सत मरुतः सुमा॒यया। द्यां व॑र्‌षयतमरु॒णाम॑रे॒पसम्। अयु॑क्त स॒प्त शु॒न्ध्युव॑। सूरो॒ रथ॑स्य न॒प्त्रिय॑। ताभि॑र्याति॒ स्वयु॑क्तिभिः। वहि॑ष्ठेभिर्\mbox{}वि॒हर॑न् यासि॒ तन्तुम्॥४७॥

%2.4.5.5
अ॒व॒व्यय॒न्नसि॑तन्देव॒ वस्व॑। दवि॑ध्वतो र॒श्मय॒ सूर्य॑स्य। चर्मे॒वावा॑धु॒स्तमो॑ अ॒प्स्व॑न्तः। प॒र्जन्या॑य॒ प्र गा॑यत। दि॒वस्पु॒त्राय॑ मी॒ढुषे। स नो॑ य॒वस॑मिच्छतु। अच्छा॑ वद त॒वस॑ङ्गी॒र्भिरा॒भिः। स्तु॒हि प॒र्जन्य॒न्नम॒साऽऽवि॑वास। कनि॑क्रदद्वृष॒भो जी॒रदा॑नुः। रेतो॑ दधा॒त्वोष॑धीषु॒ गर्भम्॥४८॥

%2.4.5.6
यो गर्भ॒मोष॑धीनाम्। गवां कृ॒णोत्यर्व॑ताम्। प॒र्जन्य॑ पुरु॒षीणाम्। तस्मा॒ इदा॒स्ये॑ ह॒विः। जू॒होता॒ मधु॑मत्तमम्। इडान्नः सं॒यत॑ङ्करत्। ति॒स्रो यद॑ग्ने श॒रद॒स्त्वामित्। शुचि॑ङ्घृ॒तेन॒ शुच॑यः सप॒र्यन्। नामा॑नि चिद्दधिरे य॒ज्ञिया॑नि। असू॑दयन्त त॒नुव॒ सुजा॑ताः॥४९॥

%2.4.5.7
इन्द्र॑श्च नः शुनासीरौ। इ॒मं य॒ज्ञं मि॑मिक्षतम्। गर्भ॑न्धत्त स्व॒स्तये। ययो॑रि॒दं विश्वं॒ भुव॑नमा वि॒वेश॑। ययो॑रान॒न्दो निहि॑तो॒ मह॑श्च। शुना॑सीरावृ॒तुभि॑ संविदा॒नौ। इन्द्र॑वन्तौ ह॒विरि॒दं जु॑षेथाम्। आघा॒ये अ॒ग्निमि॑न्ध॒ते। स्तृ॒णन्ति॑ ब॒र्॒हिरा॑नु॒षक्। येषा॒मिन्द्रो॒ युवा॒ सखा। अग्न॒ इन्द्र॑श्च मे॒दिना। ह॒थो वृ॒त्राण्य॑प्र॒ति। यु॒व हि वृ॑त्र॒हन्त॑मा। याभ्या॒ सुव॒रज॑य॒न्नग्र॑ ए॒व। यावा॑तस्थ॒तुर्भुव॑नस्य॒ मध्ये। प्रच॑र्‌ष॒णी वृ॑षणा॒ वज्र॑बाहू। अ॒ग्नी इन्द्रा॑वृत्र॒हणा॑ हुवे वाम्॥५०॥\anuvakamend[मन॒ इन्द्रो॒ गवि॑ष्टिषु॒ तन्तु॒ङ्गर्भ॒ सुजा॑ता॒ सखा॑ स॒प्त च॑]

%2.4.6.1
उ॒त न॑ प्रि॒या प्रि॒यासु॑। स॒प्तस्वसा॒ सुजु॑ष्टा। सर॑स्वती॒ स्तोम्या॑ऽभूत्। इ॒मा जुह्वा॑नायु॒ष्मदा नमो॑भिः। प्रति॒ स्तोम सरस्वति जुषस्व। तव॒ शर्म॑न्प्रि॒यत॑मे॒ दधा॑नाः। उप॑स्थेयाम शर॒णन्न वृ॒क्षम्। त्रिणि॑ प॒दा विच॑क्रमे। विष्णु॑र्गो॒पा अदाभ्यः। ततो॒ धर्मा॑णि धा॒रय\sn{}॥५१॥

%2.4.6.2
तद॑स्य प्रि॒यम॒भि पाथो॑ अश्याम्। नरो॒ यत्र॑ देव॒यवो॒ मद॑न्ति। उ॒रु॒क्र॒मस्य॒ स हि बन्धु॑रि॒त्था। विष्णो प॒दे प॑र॒मे मध्व॒ उत्स॑। क्र॒त्वा॒दा अ॑स्थु॒ श्रेष्ठ॑। अ॒द्य त्वा॑ व॒न्वन्त्सु॒रेक्णा। मर्त॑ आनाश सुवृ॒क्तिम्। इ॒मा ब्र॑ह्म ब्रह्मवाह। प्रि॒या त॒ आ ब॒र्॒हिः सी॑द। वी॒हि सू॑र पुरो॒डाशम्॥५२॥

%2.4.6.3
उप॑ नः सू॒नवो॒ गिर॑। शृ॒ण्वन्त्व॒मृत॑स्य॒ ये। सु॒मृ॒डी॒का भ॑वन्तु नः। अ॒द्या नो॑ देव सवितः। प्र॒जाव॑त्सावी॒ सौभ॑गम्। परा॑ दु॒ष्वप्नि॑य सुव। विश्वा॑नि देव सवितः। दु॒रि॒तानि॒ परा॑ सुव। यद्भ॒द्रन्तन्म॒ आ सु॑व। शुचि॑म॒र्कैर्बृह॒स्पतिम्॥५३॥

%2.4.6.4
अ॒ध्व॒रेषु॑ नमस्यत। अ॒ना॒म्योज॒ आ च॑के। या धा॒रय॑न्त दे॒वा सु॒दक्षा॒ दक्ष॑पितारा। अ॒सु॒र्या॑य॒ प्रम॑हसा। स इत् क्षेति॒ सुधि॑त॒ ओक॑सि॒ स्वे। तस्मा॒ इडा॑ पिन्वते विश्व॒दानी। तस्मै॒ विश॑ स्व॒यमे॒वान॑मन्ति। यस्मि॑न्ब्र॒ह्मा राज॑नि॒ पूर्व॒ एति॑। सकू॑तिमिन्द्र॒ सच्यु॑तिम्। सच्यु॑तिञ्ज॒घन॑च्युतिम्॥५४॥

%2.4.6.5
क॒नात्का॒भान्न॒ आ भ॑र। प्र॒य॒प्स्यन्नि॑व स॒क्थ्यौ। वि न॑ इन्द्र॒ मृधो॑ जहि। कनी॑खुनदिव सा॒पय\sn{}। अ॒भि न॒ सुष्टु॑तिन्नय। प्र॒जाप॑तिः स्त्रि॒यां यश॑। मु॒ष्कयो॑रदधा॒त्सपम्। काम॑स्य॒ तृप्ति॑मान॒न्दम्। तस्याग्ने भाजये॒ह मा। मोद॑ प्रमो॒द आ॑न॒न्दः॥५५॥

%2.4.6.6
मु॒ष्कयो॒र्निहि॑त॒ सप॑। सृ॒त्वेव॒ काम॑स्य तृप्याणि। दक्षि॑णानां प्रतिग्र॒हे। मन॑सश्चि॒त्तमाकू॑तिम्। वा॒चः स॒त्यम॑शीमहि। प॒शू॒ना रू॒पमन्न॑स्य। यश॒ श्रीः श्र॑यतां॒ मयि॑। यथा॒ऽहम॒स्या अतृ॑प स्त्रि॒यै पुमान्॑। यथा॒ स्त्री तृप्य॑ति पु॒सि प्रि॒ये प्रि॒या। ए॒वं भग॑स्य तृप्याणि॥५६॥

%2.4.6.7
य॒ज्ञस्य॒ काम्य॑ प्रि॒यः। ददा॒मीत्य॒ग्निर्व॑दति। तथेति॑ वा॒युरा॑ह॒ तत्। हन्तेति॑ स॒त्यञ्च॒न्द्रमा। आ॒दि॒त्यः स॒त्यमोमिति॑। आप॒स्तत्स॒त्यमा भ॑रन्। यशो॑ य॒ज्ञस्य॒ दक्षि॑णाम्। अ॒सौ मे॒ काम॒ समृ॑द्ध्यताम्। न हि स्पश॒मवि॑दन्न॒न्यम॒स्मात्। वै॒श्वा॒न॒रात्पु॑रए॒तार॑म॒ग्नेः॥५७॥

%2.4.6.8
अथे॑ममन्थन्न॒मृत॒ममू॑राः। वै॒श्वा॒न॒रङ्क्षेत्र॒जित्या॑य दे॒वाः। येषा॑मि॒मे पूर्वे॒ अर्मा॑स॒ आस\sn{}। अ॒यू॒पाः सद्म॒ विभृ॑ता पु॒रूणि॑। वैश्वा॑नर॒ त्वया॒ ते नु॒त्ताः। पृ॒थि॒वीम॒न्याम॒भित॑स्थु॒र्जना॑सः। पृ॒थि॒वीं मा॒तरं॑ म॒हीम्। अ॒न्तरि॑क्ष॒मुप॑ ब्रुवे। बृ॒ह॒तीमू॒तये॒ दिवम्। विश्वं॑ बिभर्ति पृथि॒वी॥५८॥

%2.4.6.9
अ॒न्तरि॑क्षं॒ वि प॑प्रथे। दु॒हे द्यौर्बृ॑ह॒ती पय॑। न ता न॑शन्ति॒ न द॑भाति॒ तस्क॑रः। नैना॑ अमि॒त्रो व्यथि॒राद॑धर्‌षति। दे॒वाश्च॒ याभि॒र्यज॑ते॒ ददा॑ति च। ज्योगित्ताभि॑ सचते॒ गोप॑तिः स॒ह। न ता अर्वा॑ रे॒णुक॑काटो अश्ञुते। न सस्कृत॒त्रमुप॑ यन्ति॒ ता अ॒भि। उ॒रु॒गा॒यमभ॑य॒न्तस्य॒ ता अनु॑। गावो॒ मर्त्य॑स्य॒ वि च॑रन्ति॒ यज्व॑नः॥५९॥

%2.4.6.10
रात्री॒ व्य॑ख्यदाय॒ती। पु॒रु॒त्रा दे॒व्य॑क्षभि॑। विश्वा॒ अधि॒ श्रियो॑ऽधित। उप॑ ते॒ गा इ॒वाक॑रम्। वृ॒णी॒ष्व दु॑हितर्दिवः। रात्री॒ स्तोम॒न्न जि॒ग्युषी। दे॒वीं वाच॑मजनयन्त दे॒वाः। तां वि॒श्वरू॑पाः प॒शवो॑ वदन्ति। सा नो॑ म॒न्द्रेष॒मूर्ज॒न्दुहा॑ना। धे॒नुर्वाग॒स्मानुप॒ सुष्टु॒तैतु॑॥६०॥

%2.4.6.11
यद्वाग्वद॑न्त्यविचेत॒नानि॑। राष्ट्री॑ दे॒वानान्निष॒साद॑ म॒न्द्रा। चत॑स्र॒ ऊर्ज॑न्दुदुहे॒ पयासि। क्व॑ स्विदस्याः पर॒मं ज॑गाम। गौ॒री मि॑माय सलि॒लानि॒ तक्ष॑ती। एक॑पदी द्वि॒पदी॒ सा चतु॑ष्पदी। अ॒ष्टाप॑दी॒ नव॑पदी बभू॒वुषी। स॒हस्राक्षरा पर॒मे व्यो॑मन्। तस्या समु॒द्रा अधि॒ विक्ष॑रन्ति। तेन॑ जीवन्ति प्र॒दिश॒श्चत॑स्रः॥६१॥

%2.4.6.12
तत॑ क्षरत्य॒क्षरम्। तद्विश्व॒मुप॑ जीवति। इन्द्रा॒सूरा॑ ज॒नय॑न्वि॒श्वक॑र्मा। म॒रुत्वा अस्तु ग॒णवान्त्सजा॒तवान्॑। अ॒स्य स्नु॒षा श्वशु॑रस्य॒ प्रशि॑ष्टिम्। स॒पत्ना॒ वाचं॒ मन॑सा॒ उपा॑सताम्। इन्द्र॒ सूरो॑ अतर॒द्रजासि। स्नु॒षा स॒पत्ना॒ श्वशु॑रो॒ऽयम॑स्तु। अ॒य शत्रूञ्जयतु॒ जर्‌हृ॑षाणः। अ॒यं वां॑ जयतु॒ वाज॑सातौ। अ॒ग्निः क्ष॑त्र॒भृदनि॑भृष्ट॒मोज॑। स॒ह॒स्रियो॑ दीप्यता॒मप्र॑युच्छन्। वि॒भ्राज॑मानः समिधा॒न उ॒ग्रः। आऽन्तरि॑क्षमरुह॒दग॒न्द्याम्॥६२॥\anuvakamend[धा॒रय॑न्पुरो॒डाशं॒ बृह॒स्पतिं॑ ज॒घन॑च्युतिमान॒न्दो भग॑स्य तृप्याण्य॒ग्नेः पृ॑थि॒वी यज्व॑न एतु प्र॒दिश॒श्चत॑स्रो॒ वाज॑सातौ च॒त्वारि॑ च]

%2.4.7.1
वृषाऽस्य॒शुर्वृ॑ष॒भाय॑ गृह्यसे। वृषा॒ऽयमु॒ग्रो नृ॒चक्ष॑से। दि॒व्यः क॑र्म॒ण्यो॑ हि॒तो बृ॒हन्नाम॑। वृ॒ष॒भस्य॒ या क॒कुत्। वि॒षू॒वान् वि॑ष्णो भवतु। अ॒यं यो मा॑म॒को वृषा। अथो॒ इन्द्र॑ इव दे॒वेभ्य॑। वि ब्र॑वीतु॒ जनेभ्यः। आयु॑ष्मन्तं॒ वर्च॑स्वन्तम्। अथो॒ अधि॑पतिं वि॒शाम्॥६३॥

%2.4.7.2
अ॒स्याः पृ॑थि॒व्या अध्य॑क्षम्। इ॒ममि॑न्द्र वृष॒भं कृ॑णु। यः सु॒शृङ्ग॑ सुवृष॒भः। क॒ल्याणो॒ द्रोण॒ आहि॑तः। कार्‌षी॑वल प्रगाणेन। वृ॒ष॒भेण॑ यजामहे। वृ॒ष॒भेण॒ यज॑मानाः। अक्रू॑रेणेव स॒र्पिषा। मृध॑श्च॒ सर्वा॒ इन्द्रे॑ण। पृत॑नाश्च जयामसि॥६४॥

%2.4.7.3
यस्या॒यमृ॑ष॒भो ह॒विः। इन्द्रा॑य परिणी॒यते। जया॑ति॒ शत्रु॑मा॒यन्तम्। अथो॑ हन्ति पृतन्य॒तः। नृ॒णामह॑ प्र॒णीरस॑त्। अग्र॑ उद्भिन्द॒ताम॑सत्। इन्द्र॒ शुष्म॑न्त॒नुवा॒ मेर॑यस्व। नी॒चा विश्वा॑ अ॒भिति॑ष्ठा॒भिमा॑तीः। नि शृ॑णीह्याबा॒धं यो॒ नो॒ अस्ति॑। उ॒रुन्नो॑ लो॒कं कृ॑णुहि जीरदानो॥६५॥

%2.4.7.4
प्रेह्य॒भि प्रेहि॒ प्र भ॑रा॒ सह॑स्व। मा विवे॑नो॒ वि शृ॑णुष्वा॒ जने॑षु। उदी॑डि॒तो वृ॑षभ॒ तिष्ठ॒ शुष्मै। इन्द्र॒ शत्रून्पु॒रो अ॒स्माक॑ युध्य। अग्ने॒ जेता॒ त्वं ज॑य। शत्रून्त्सहस॒ ओज॑सा। वि शत्रू॒न्॒ विमृधो॑ नुद। ए॒तन्ते॒ स्तोम॑न्तुविजात॒ विप्र॑। रथ॒न्न धीर॒ स्वपा॑ अतक्षम्। यदीद॑ग्ने॒ प्रति॒त्वन्दे॑व॒ हर्या॥६६॥

%2.4.7.5
सुव॑र्वतीर॒प ए॑ना जयेम। यो घृ॒तेना॒भिमा॑नितः। इन्द्र॒ जैत्रा॑य जज्ञिषे। स न॒ सङ्का॑सु पारय। पृ॒त॒ना॒साह्ये॑षु च। इन्द्रो॑ जिगाय पृथि॒वीम्। अ॒न्तरि॑क्ष॒ सुव॑र्म॒हत्। वृ॒त्र॒हा पु॑रु॒चेत॑नः। इन्द्रो॑ जिगाय॒ सह॑सा॒ सहासि। इन्द्रो॑ जिगाय॒ पृत॑नानि॒ विश्वा॥६७॥

%2.4.7.6
इन्द्रो॑ जा॒तो वि पुरो॑ रुरोज। स न॑ पर॒स्पा वरि॑वः कृणोतु। अ॒यं कृ॒त्नुरगृ॑भीतः। वि॒श्व॒जिदु॒द्भिदित्सोम॑। ऋषि॒र्विप्र॒ काव्ये॑न। वा॒युर॑ग्रे॒गा य॑ज्ञ॒प्रीः। सा॒कङ्ग॒न्मन॑सा य॒ज्ञम्। शि॒वो नि॒युद्भि॑ शि॒वाभि॑। वायो॑ शु॒क्रो अ॑यामि ते। मध्वो॒ अग्र॒न्दिवि॑ष्टिषु॥६८॥

%2.4.7.7
आ या॑हि॒ सोम॑ पीतये। स्वा॒रु॒हो दे॑व नि॒युत्व॑ता। इ॒ममि॑न्द्र वर्धय क्ष॒त्रिया॑णाम्। अ॒यं वि॒शां वि॒श्पति॑रस्तु॒ राजा। अ॒स्मा इ॑न्द्र॒ महि॒ वर्चासि धेहि। अ॒व॒र्चस॑ङ्कणुहि॒ शत्रु॑मस्य। इ॒ममा भ॑ज॒ ग्रामे॒ अश्वे॑षु॒ गोषु॑। निर॒मुं भ॑ज॒ यो॑ऽमित्रो॑ अस्य। वर्ष्म॑न् क्ष॒त्रस्य॑ क॒कुभि॑ श्रयस्व। ततो॑ न उ॒ग्रो वि भ॑जा॒ वसू॑नि॥६९॥

%2.4.7.8
अ॒स्मे द्या॑वापृथिवी॒ भूरि॑ वा॒मम्। सन्दु॑हाथाङ्घर्म॒दुघे॑व धे॒नुः। अ॒य राजा प्रि॒य इन्द्र॑स्य भूयात्। प्रि॒यो गवा॒मोष॑धीनामु॒तापाम्। यु॒नज्मि॑ त उत्त॒राव॑न्त॒मिन्द्रम्। येन॒ जया॑सि॒ न परा॒ जया॑सै। स त्वा॑ऽकरेकवृष॒भ स्वानाम्। अथो॑ राजन्नुत्त॒मं मा॑न॒वानाम्। उत्त॑र॒स्त्वमध॑रे ते स॒पत्ना। एक॑वृषा॒ इन्द्र॑सखा जिगी॒वान्॥७०॥

%2.4.7.9
विश्वा॒ आशा॒ पृत॑नाः स॒ञ्जयं॒ जय\sn{}। अ॒भि ति॑ष्ठ शत्रूय॒तः स॑हस्व। तुभ्यं॑ भरन्ति क्षि॒तयो॑ यविष्ठ। ब॒लिम॑ग्ने॒ अन्ति॑त॒ ओत दू॒रात्। आ भन्दि॑ष्ठस्य सुम॒तिञ्चि॑किद्धि। बृ॒हत्ते॑ अग्ने॒ महि॒ शर्म॑ भ॒द्रम्। यो दे॒ह्यो अन॑मयद्वध॒स्नैः। यो अर्य॑पत्नीरु॒षस॑श्च॒कार॑। स नि॒रुध्या॒ नहु॑षो य॒ह्वो अ॒ग्निः। विश॑श्चक्रे बलि॒हृत॒ सहो॑भिः॥७१॥

%2.4.7.10
प्र स॒द्यो अ॑ग्ने॒ अत्येष्य॒न्यान्। आ॒विर्यस्मै॒ चारु॑तरो ब॒भूथ॑। ई॒डेन्यो॑ वपु॒ष्यो॑ वि॒भावा। प्रि॒यो वि॒शामति॑थि॒र्मानु॑षीणाम्। ब्रह्म॑ज्येष्ठा वी॒र्या॑ सम्भृ॑तानि। ब्रह्माग्रे॒ ज्येष्ठ॒न्दिव॒मा त॑तान। ऋ॒तस्य॒ ब्रह्म॑ प्रथ॒मोत ज॑ज्ञे। तेना॑र्‌हति॒ ब्रह्म॑णा॒ स्पर्धि॑तु॒ङ्कः। ब्रह्म॒ स्रुचो॑ घृ॒तव॑तीः। ब्रह्म॑णा॒ स्वर॑वो मि॒ताः॥७२॥

%2.4.7.11
ब्रह्म॑ य॒ज्ञस्य॒ तन्त॑वः। ऋ॒त्विजो॒ ये ह॑वि॒ष्कृत॑। शृङ्गा॑णी॒वेच्छृ॒ङ्गिणा॒ सन्द॑दृश्रिरे। च॒षाल॑वन्त॒ स्वर॑वः पृथि॒व्याम्। ते दे॒वास॒ स्वर॑वस्तस्थि॒वास॑। नम॒ सखि॑भ्यः स॒न्नान्माऽव॑गात। अ॒भि॒भूर॒ग्निर॑तर॒द्रजासि। स्पृधो॑ वि॒हत्य॒ पृत॑ना अभि॒श्रीः। जु॒षा॒णो म॒ आहु॑तिं मामहिष्ट। ह॒त्वा स॒पत्ना॒न्॒ वरि॑वस्करन्नः। ईशा॑नन्त्वा॒ भुव॑नानामभि॒श्रियम्। स्तौम्य॑ग्न उरु॒कृत सु॒वीरम्। ह॒विर्जु॑षा॒णः स॒पत्ना अभि॒भूर॑सि। ज॒हि शत्रू॒ रप॒ मृधो॑ नुदस्व॥७३॥\anuvakamend[वि॒शां ज॑यामसि जीरदानो॒ हर्या॒ विश्वा॒ दिवि॑ष्टिषु॒ वसू॑नि जिगी॒वान्त्सहो॑भिर्मि॒ता न॑श्च॒त्वारि॑ च]

%2.4.8.1
स प्र॑त्न॒वन्नवी॑यसा। अग्ने द्यु॒म्नेन॑ सं॒यता। बृ॒हत्त॑तन्थ भा॒नुना। नव॒न्नु स्तोम॑म॒ग्नये। दि॒वः श्ये॒नाय॑ जीजनम्। वसो कु॒विद्व॒नाति॑ नः। स्वा॒रु॒हा यस्य॒ श्रियो॑ दृ॒शे। र॒यिर्वी॒रव॑तो यथा। अग्रे॑ य॒ज्ञस्य॒ चेत॑तः। अदाभ्यः पुरए॒ता॥७४॥

%2.4.8.2
अ॒ग्निर्वि॒शां मानु॑षीणाम्। तूर्णी॒ रथ॒ सदा॒ नव॑। नव॒ सोमा॑य वा॒जिने। आज्यं॒ पय॑सोऽजनि। जुष्ट॒ शुचि॑तमं॒ वसु॑। नव सोम जुषस्व नः। पी॒यूष॑स्ये॒ह तृ॑प्णुहि। यस्ते॑ भा॒ग ऋ॒ता व॒यम्। नव॑स्य सोम ते व॒यम्। आ सु॑म॒तिं वृ॑णीमहे॥७५॥

%2.4.8.3
स नो॑ रास्व सह॒स्रिण॑। नव ह॒विर्जु॑षस्व नः। ऋ॒तुभि॑ सोम॒ भूत॑मम्। तद॒ङ्ग प्रति॑हर्य नः। राजन्त्सोम स्व॒स्तये। नव॒स्तोम॒न्नव ह॒विः। इ॒न्द्रा॒ग्निभ्यां॒ नि वे॑दय। तज्जु॑षेता॒ सचे॑तसा। शुचि॒न्नु स्तोम॒न्नव॑जातम॒द्य। इन्द्राग्नी वृत्रहणा जु॒षेथाम्॥७६॥

%2.4.8.4
उ॒भा हि वा सु॒हवा॒ जोह॑वीमि। ता वाज स॒द्य उ॑श॒ते धेष्ठा। अ॒ग्निरिन्द्रो॒ नव॑स्य नः। अ॒स्य ह॒व्यस्य॑ तृप्यताम्। इ॒ह दे॒वौ स॑ह॒स्रिणौ। य॒ज्ञन्न॒ आ हि गच्छ॑ताम्। वसु॑मन्त सुव॒र्विदम्। अ॒स्य ह॒व्यस्य॑ तृप्यताम्। अ॒ग्निरिन्द्रो॒ नव॑स्य नः। विश्वान्दे॒वास्त॑र्पयत॥७७॥

%2.4.8.5
ह॒विषो॒ऽस्य नव॑स्य नः। सु॒व॒र्विदो॒ हि ज॑ज्ञि॒रे। एदं ब॒र्॒हिः सु॒ष्टरी॑मा॒ नवे॑न। अ॒यं य॒ज्ञो यज॑मानस्य भा॒गः। अ॒यं ब॑भूव॒ भुव॑नस्य॒ गर्भ॑। विश्वे॑ दे॒वा इ॒दम॒द्याग॑मिष्ठाः। इ॒मे नु द्यावा॑पृथि॒वी स॒मीची। त॒न्वा॒ने य॒ज्ञं पु॑रु॒पेश॑सन्धि॒या। आऽस्मै॑ पृणीतां॒ भुव॑नानि॒ विश्वा। प्र॒जां पुष्टि॑म॒मृत॒न्नवे॑न॥७८॥

%2.4.8.6
इ॒मे धे॒नू अ॒मृतं॒ ये दु॒हाते। पय॑स्वत्युत्त॒रामे॑तु॒ पुष्टि॑। इ॒मं य॒ज्ञं जु॒षमा॑णे॒ नवे॑न। स॒मीची॒ द्यावा॑पृथि॒वी घृ॒ताची। यवि॑ष्ठो हव्य॒वाह॑नः। चि॒त्रभा॑नुर्घु॒तासु॑तिः। नव॑जातो॒ वि रो॑चसे। अग्ने॒ तत्ते॑ महित्व॒नम्। त्वम॑ग्ने दे॒वताभ्यः। भा॒गे दे॑व॒ न मी॑यसे॥७९॥

%2.4.8.7
स ए॑ना वि॒द्वान् य॑क्ष्यसि। नव॒ स्तोमं॑ जुषस्व नः। अ॒ग्निः प्र॑थ॒मः प्राश्ञा॑तु। स हि वेद॒ यथा॑ ह॒विः। शि॒वा अ॒स्मभ्य॒मोष॑धीः। कृ॒णोतु॑ वि॒श्वच॑र्‌षणिः। भ॒द्रान्न॒ श्रेय॒ सम॑नैष्ट देवाः। त्वया॑ऽव॒सेन॒ सम॑शीमहि त्वा। स नो॑ मयो॒भूः पि॑तो॒ आ वि॑शस्व। शन्तो॒काय॑ त॒नुवे स्यो॒नः। ए॒तमु॒ त्यं मधु॑ना॒ संयु॑तं॒ यवम्। सर॑स्वत्या॒ अधि॑म॒नाव॑चर्कृषुः। इन्द्र॑ आसी॒त्सीर॑पतिः श॒तक्र॑तुः। की॒नाशा॑ आसन्म॒रुत॑ सु॒दान॑वः॥८०॥\anuvakamend[पु॒र॒ए॒ता वृ॑णीमहे जु॒षेथान्तर्पयता॒मृत॒न्नवे॑न मीयसे स्यो॒नश्च॒त्वारि॑ च]




\prashnaend{जुष्ट॒श्चक्षु॑षो॒ जुष्टी॑नरो नक्तञ्जा॒ता वृषा॒स उ॒त नो॒ वृषाऽस्य॒शुः सप्र॑त्न॒वद॒ष्टौ॥८॥}{जुष्टो॑ म॒न्युर्भगो॒ जुष्टी॑ नरो॒ हरि॑वर्पस॒ङ्गिर॒ शिप्रि॑न्वाजानामु॒त न॑ प्रि॒या यद्वाग्वद॑न्ती॒ विश्वा॒ आशा॒ अशी॑तिः॥८०॥}{जुष्ट॑ सु॒दान॑वः॥}{हरि॑ ओम्॥}{इति श्रीकृष्णयजुर्वेदीयतैत्तिरीयब्राह्मणे द्वितीयाष्टके चतुर्थः प्रपाठकः समाप्तः॥}
\clearpage
