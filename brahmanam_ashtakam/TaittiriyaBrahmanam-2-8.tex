\sect{अष्टमः प्रश्नः}
\setcounter{anuvakam}{0}
\dnsub{तैत्तिरीयब्राह्मणे द्वितीयाष्टके अष्टमः प्रपाठकः}

%2.8.1.1
पीवोन्ना रयि॒वृध॑ सुमे॒धाः। श्वे॒तः सि॑षक्ति नि॒युता॑मभि॒श्रीः। ते वा॒यवे॒ सम॑नसो॒ वित॑स्थुः। विश्वेन्नर॑ स्वप॒त्यानि॑ चक्रुः। रा॒येऽनु यञ्ज॒ज्ञतू॒ रोद॑सी उ॒भे। रा॒ये दे॒वी धि॒षणा॑ धाति दे॒वम्। अधा॑ वा॒युन्नि॒युत॑ सश्चत॒ स्वाः। उ॒त श्वे॒तं वसु॑धितिन्निरे॒के। आ वा॑यो॒ प्र याभि॑। प्र वा॒युमच्छा॑ बृह॒ती म॑नी॒षा॥१॥

%2.8.1.2
बृ॒हद्र॑यिं वि॒श्ववा॑रा रथ॒प्राम्। द्यु॒तद्या॑मा नि॒युत॒ पत्य॑मानः। क॒विः क॒विमि॑यक्षसि प्रयज्यो। आ नो॑ नि॒युद्भि॑ श॒तिनी॑भिरध्व॒रम्। स॒ह॒स्रिणी॑भि॒रुप॑ याहि य॒ज्ञम्। वायो॑ अ॒स्मिन् ह॒विषि॑ मादयस्व। यू॒यं पा॑त स्व॒स्तिभि॒ सदा॑ नः। प्रजा॑पते॒ न त्वदे॒तान्य॒न्यः। विश्वा॑ जा॒तानि॒ परि॒ ता ब॑भूव। यत्का॑मास्ते जुहु॒मस्तन्नो॑ अस्तु॥२॥

%2.8.1.3
व॒य स्या॑म॒ पत॑यो रयी॒णाम्। र॒यी॒णां पतिं॑ यज॒तं बृ॒हन्तम्। अ॒स्मिन्भरे॒ नृत॑मं॒ वाज॑सातौ। प्र॒जाप॑तिं प्रथम॒जामृ॒तस्य॑। यजा॑म दे॒वमधि॑ नो ब्रवीतु। प्रजा॑पते॒ त्वन्नि॑धि॒पाः पु॑रा॒णः। दे॒वानां पि॒ता ज॑नि॒ता प्र॒जानाम्। पति॒र्विश्व॑स्य॒ जग॑तः पर॒स्पाः। ह॒विर्नो॑ देव विह॒वे जु॑षस्व। तवे॒मे लो॒काः प्र॒दिशो॒ दिश॑श्च॥३॥

%2.8.1.4
प॒रा॒वतो॑ नि॒वत॑ उ॒द्वत॑श्च। प्रजा॑पते विश्व॒सृज्जी॒वध॑न्य इ॒दन्नो॑ देव। प्रति॑हर्य ह॒व्यम्। प्र॒जाप॑तिं प्रथ॒मं य॒ज्ञिया॑नाम्। दे॒वाना॒मग्रे॑ यज॒तं य॑जध्वम्। स नो॑ ददातु॒ द्रवि॑ण सु॒वीर्यम्। रा॒यस्पोषं॒ वि ष्य॑तु॒ नाभि॑म॒स्मे। यो रा॒य ईशे॑ शतदा॒य उ॒क्थ्य॑। यः प॑शू॒ना र॑क्षि॒ता विष्ठि॑तानाम्। प्र॒जाप॑तिः प्रथम॒जा ऋ॒तस्य॑॥४॥

%2.8.1.5
स॒हस्र॑धामा जुषता ह॒विर्न॑। सोमा॑पूषणे॒मौ दे॒वौ। सोमा॑पूषणा॒ रज॑सो वि॒मानम्। स॒प्तच॑क्र॒ रथ॒मवि॑श्वमिन्वम्। वि॒षू॒वृतं॒ मन॑सा यु॒ज्यमा॑नम्। तञ्जि॑न्वथो वृषणा॒ पञ्च॑रश्मिम्। दि॒व्य॑न्यः सद॑नञ्च॒क्र उ॒च्चा। पृ॒थि॒व्याम॒न्यो अध्य॒न्तरि॑क्षे। ताव॒स्मभ्यं॑ पुरु॒वारं॑ पुरु॒क्षुम्। रा॒यस्पोषं॒ विष्य॑ता॒न्नाभि॑म॒स्मे॥५॥

%2.8.1.6
धियं॑ पू॒षा जि॑न्वतु विश्वमि॒न्वः। र॒यि सोमो॑ रयि॒पति॑र्दधातु। अव॑तु दे॒व्यदि॑तिरन॒र्वा। बृ॒हद्व॑देम वि॒दथे॑ सु॒वीरा। विश्वान्य॒न्यो भुव॑ना ज॒जान॑। विश्व॑म॒न्यो अ॑भि॒चक्षा॑ण एति। सोमा॑पूषणा॒वव॑त॒न्धियं॑ मे। यु॒वभ्यां॒ विश्वा॒ पृत॑ना जयेम। उदु॑त्त॒मं व॑रु॒णास्त॑भ्ना॒द्द्याम्। यत्किञ्चे॒दङ्कि॑त॒वास॑। अव॑ ते॒ हेड॒स्तत्त्वा॑ यामि। आ॒दि॒त्याना॒मव॑सा॒ न द॑क्षि॒णा। धा॒रय॑न्त आदि॒त्यास॑स्ति॒स्रो भूमीर्धारयन्। य॒ज्ञो दे॒वाना॒ शुचि॑र॒पः॥६॥\anuvakamend[म॒नी॒षाऽस्तु॑ च॒र्तस्या॒स्मे कि॑त॒वास॑श्च॒त्वारि॑ च]

%2.8.2.1
ते शु॒क्रास॒ शुच॑यो रश्मि॒वन्त॑। सीद॑न्नादि॒त्या अधि॑ ब॒र्॒हिषि॑ प्रि॒ये। कामे॑न दे॒वाः स॒रथ॑न्दि॒वो न॑। आ यान्तु य॒ज्ञमुप॑ नो जुषा॒णाः। ते सू॒नवो॒ अदि॑तेः पीव॒सामिषम्। घृ॒तं पिन्व॒त्प्रति॑हर्यन्नृते॒जाः। प्र य॒ज्ञिया॒ यज॑मानाय येमुरे। आ॒दि॒त्याः कामं॑ पितु॒मन्त॑म॒स्मे। आ न॑ पु॒त्रा अदि॑तेर्यान्तु य॒ज्ञम्। आ॒दि॒त्यास॑ प॒थिभि॑र्देव॒यानै ॥७॥

%2.8.2.2
अ॒स्मे काम॑न्दा॒शुषे॑ स॒न्नम॑न्तः। पुरो॒डाश॑ङ्घृ॒तव॑न्तं जुषन्ताम्। स्क॒भा॒यत॒ निर्\mbox{}ऋ॑ति॒ सेध॒ताम॑तिम्। प्र र॒श्मिभि॒र्यत॑माना अमृध्राः। आदि॑त्या॒ काम॒ प्रय॑तां॒ वष॑ट्कृतिम्। जु॒षध्व॑न्नो ह॒व्यदा॑तिं यजत्राः। आ॒दि॒त्यान्काम॒मव॑से हुवेम। ये भू॒तानि॑ ज॒नय॑न्तो विचि॒ख्युः। सीद॑न्तु पु॒त्रा अदि॑तेरु॒पस्थम्। स्ती॒र्णं ब॒र्॒हिर्\mbox{}ह॑वि॒रद्या॑य दे॒वाः॥८॥

%2.8.2.3
स्ती॒र्णं ब॒र्॒हिः सी॑दता य॒ज्ञे अ॒स्मिन्। ध्रा॒जाः सेध॑न्तो॒ अम॑तिन्दु॒रेवाम्। अ॒स्मभ्यं॑ पुत्रा अदिते॒ प्र यसत। आदि॑त्या॒ काम॑ ह॒विषो॑ जुषा॒णाः। अग्ने॒ नय॑ सु॒पथा॑ रा॒ये अ॒स्मान्। विश्वा॑नि देव व॒युना॑नि वि॒द्वान्। यु॒यो॒ध्य॑स्मज्जु॑हुरा॒णमेन॑। भूयि॑ष्ठान्ते॒ नम॑ उक्तिं विधेम। प्र व॑ शु॒क्राय॑ भा॒नवे॑ भरध्वम्। ह॒व्यं म॒तिञ्चा॒ग्नये॒ सुपू॑तम्॥९॥

%2.8.2.4
यो दैव्या॑नि॒ मानु॑षा ज॒नूषि॑। अ॒न्तर्विश्वा॑नि वि॒द्मना॒ जिगा॑ति। अच्छा॒ गिरो॑ म॒तयो॑ देव॒यन्ती। अ॒ग्निं य॑न्ति॒ द्रवि॑णं॒ भिक्ष॑माणाः। सु॒स॒न्दृश सु॒प्रती॑क॒ स्वञ्चम्। ह॒व्य॒वाह॑मर॒तिं मानु॑षाणाम्। अग्ने॒ त्वम॒स्मद्यु॑यो॒ध्यमी॑वाः। अन॑ग्नित्रा अ॒भ्य॑मन्त कृ॒ष्टीः। पुन॑र॒स्मभ्य सुवि॒ताय॑ देव। क्षां विश्वे॑भिर॒जरे॑भिर्यजत्र॥१०॥

%2.8.2.5
अग्ने॒ त्वं पा॑रया॒ नव्यो॑ अ॒स्मान्। स्व॒स्तिभि॒रति॑ दु॒र्गाणि॒ विश्वा। पूश्च॑ पृ॒थ्वी ब॑हु॒ला न॑ उ॒र्वी। भवा॑ तो॒काय॒ तन॑याय॒ शय्योँः। प्रका॑रवो मन॒ना व॒च्यमा॑नाः। दे॒व॒द्रीचीन्नयथ देव॒यन्त॑। द॒क्षि॒णा॒वाड्वा॒जिनी॒ प्राच्ये॑ति। ह॒विर्भर॑न्त्य॒ग्नये॑ घृ॒ताची। इन्द्र॒न्नरो॑ यु॒जे रथम्। ज॒गृ॒भ्णाते॒ दक्षि॑णमिन्द्र॒ हस्तम्॥११॥

%2.8.2.6
व॒सू॒यवो॑ वसुपते॒ वसू॑नाम्। वि॒द्मा हि त्वा॒ गोप॑ति शूर॒ गोनाम्। अ॒स्मभ्यं॑ चित्रं वृष॑ण र॒यिन्दा। तवे॒दं विश्व॑म॒भित॑ पश॒व्यम्। यत्पश्य॑सि॒ चक्ष॑सा॒ सूर्य॑स्य। गवा॑मसि॒ गोप॑ति॒रेक॑ इन्द्र। भ॒क्षी॒महि॑ ते॒ प्रय॑तस्य॒ वस्व॑। समि॑न्द्र णो॒ मन॑सा नेषि॒ गोभि॑। स सू॒रिभि॑र्मघव॒न्त्स स्व॒स्त्या। सं ब्रह्म॑णा दे॒वकृ॑तं॒ यदस्ति॑॥१२॥

%2.8.2.7
सन्दे॒वाना सुम॒त्या य॒ज्ञिया॑नाम्। आ॒राच्छत्रु॒मप॑ बाधस्व दू॒रम्। उ॒ग्रो यः शम्ब॑ पुरुहूत॒ तेन॑। अ॒स्मे धे॑हि॒ यव॑म॒द्गोम॑दिन्द्र। कृ॒धीधियं॑ जरि॒त्रे वाज॑रत्नाम्। आ वे॒धस॒ स हि शुचि॑। बृह॒स्पति॑ प्रथ॒मञ्जाय॑मानः। म॒हो ज्योति॑षः पर॒मे व्यो॑मन्। स॒प्तास्य॑स्तुविजा॒तो रवे॑ण। वि स॒प्तर॑श्मिरधम॒त्तमासि॥१३॥

%2.8.2.8
बृह॒स्पति॒ सम॑जय॒द्वसू॑नि। म॒हो व्र॒जान्गोम॑तो दे॒व ए॒षः। अ॒पः सिषा॑स॒न्त्सुव॒रप्र॑तीत्तः। बृह॒स्पति॒र्॒हन्त्य॒मित्र॑म॒र्कैः। बृह॑स्पते॒ पर्ये॒वा पि॒त्रे। आ नो॑ दि॒वः पावी॑रवी। इ॒मा जुह्वा॑ना॒ यस्ते॒ स्तन॑। सर॑स्वत्य॒भि नो॑ नेषि। इ॒य शुष्मे॑भिर्बिस॒खा इ॑वारुजत्। सानु॑ गिरी॒णान्त॑वि॒षेभि॑रू॒र्मिभि॑। पा॒रा॒व॒द॒घ्नीमव॑से सुवृ॒क्तिभि॑। सर॑स्वती॒मा वि॑वासेम धी॒तिभि॑॥१४॥\anuvakamend[दे॒व॒यानैर्दे॒वाः सुपू॑तं यजत्र॒ हस्त॒मस्ति॒ तमास्यू॒र्मिभि॒र्द्वे च॑]

%2.8.3.1
सोमो॑ धे॒नु सोमो॒ अर्व॑न्तमा॒शुम्। सोमो॑ वी॒रङ्क॑र्म॒ण्य॑न्ददातु। सा॒द॒न्यं॑ विद॒थ्य स॒भेयम्। पि॒तु॒ श्रव॑ण॒य्योँ ददा॑शदस्मै। अषा॑ढय्युँ॒त्सु त्व सो॑म॒ क्रतु॑भिः। या ते॒ धामा॑नि ह॒विषा॒ यज॑न्ति। त्वमि॒मा ओष॑धीः सोम॒ विश्वा। त्वम॒पो अ॑जनय॒स्त्वङ्गाः। त्वमात॑तन्थो॒र्व॑न्तरि॑क्षम्। त्वञ्ज्योति॑षा॒ वि तमो॑ ववर्थ॥१५॥

%2.8.3.2
या ते॒ धामा॑नि दि॒वि या पृ॑थि॒व्याम्। या पर्व॑ते॒ष्वोष॑धीष्व॒प्सु। तेभि॑र्नो॒ विश्वै सु॒मना॒ अहे॑डन्। राजन्त्सोम॒ प्रति॑ ह॒व्या गृ॑भाय। विष्णो॒र्नुक॒न्तद॑स्य प्रि॒यम्। प्र तद्विष्णु॑। प॒रो मात्र॑या त॒नुवा॑ वृधान। न ते॑ महि॒त्वमन्व॑श्ञुवन्ति। उ॒भे ते॑ विद्म॒ रज॑सी पृथि॒व्या विष्णो॑ देव॒ त्वम्। प॒र॒मस्य॑ वित्से॥१६॥

%2.8.3.3
विच॑क्रमे॒ त्रिर्दे॒वः। आ ते॑ म॒हो यो जा॒त ए॒व। अ॒भि गो॒त्राणि॑। आभि॒ स्पृधो॑ मिथ॒तीररि॑षण्यन्। अ॒मित्र॑स्य व्यथया म॒न्युमि॑न्द्र। आभि॒र्विश्वा॑ अभि॒युजो॒ विषू॑चीः। आर्या॑य॒ विशोव॑तारी॒र्दासी। अ॒य शृ॑ण्वे॒ अध॒ जय॑न्नु॒त घ्नन्। अ॒यमु॒त प्र कृ॑णुते यु॒धा गाः। य॒दा स॒त्यं कृ॑णु॒ते म॒न्युमिन्द्र॑॥१७॥

%2.8.3.4
विश्व॑न्दृ॒ढं भ॑यत॒ एज॑दस्मात्। अनु॑ स्व॒धाम॑क्षर॒न्नापो॑ अस्य। अव॑र्धत॒ मध्य॒ आ ना॒व्या॑नाम्। स॒ध्री॒चीने॑न॒ मन॑सा॒ तमि॑न्द्र॒ ओजि॑ष्ठेन। हन्म॑नाहन्न॒भिद्यून्। म॒रुत्व॑न्तं वृष॒भं वा॑वृधा॒नम्। अक॑वारिन्दि॒व्य शा॒समिन्द्रम्। वि॒श्वा॒साह॒मव॑से॒ नूत॑नाय। उ॒ग्र स॑हो॒दामि॒ह त हु॑वेम। जनि॑ष्ठा उ॒ग्रः सह॑से तु॒राय॑॥१८॥

%2.8.3.5
म॒न्द्र ओजि॑ष्ठो बहु॒लाभि॑मानः। अव॑र्ध॒न्निन्द्रं॑ म॒रुत॑श्चि॒दत्र॑। मा॒ता यद्वी॒रन्द॒धन॒द्धनि॑ष्ठा। क्व॑स्यावो॑ मरुतः स्व॒धाऽऽसीत्। यन्मामेक स॒मध॑त्ताहि॒हत्ये। अ॒ह ह्यु॑ग्रस्त॑वि॒षस्तुवि॑ष्मान्। विश्व॑स्य॒ शत्रो॒रन॑मं वध॒स्नैः। वृ॒त्रस्य॑ त्वा श्व॒सथा॒ दीष॑माणाः। विश्वे॑ दे॒वा अ॑जहु॒र्ये सखा॑यः। म॒रुद्भि॑रिन्द्र स॒ख्यन्ते॑ अस्तु॥१९॥

%2.8.3.6
अथे॒मा विश्वा॒ पृत॑ना जयासि। वधीं वृ॒त्रं म॑रुत इन्द्रि॒येण॑। स्वेन॒ भामे॑न तवि॒षो ब॑भू॒वान्। अ॒हमे॒ता मन॑वे वि॒श्वश्च॑न्द्राः। सु॒गा अ॒पश्च॑कर॒ वज्र॑बाहुः। स यो वृषा॒ वृष्णि॑येभि॒ समो॑काः। म॒हो दि॒वः पृ॑थि॒व्याश्च॑ स॒म्राट्। स॒ती॒नस॑त्वा॒ हव्यो॒ भरे॑षु। म॒रुत्वान्नो भव॒त्विन्द्र॑ ऊ॒ती। इन्द्रो॑ वृ॒त्रम॑तरद्वृत्र॒तूर्ये॥२०॥

%2.8.3.7
अ॒ना॒धृ॒ष्यो म॒घवा॒ शूर॒ इन्द्र॑। अन्वे॑नं॒ विशो॑ अमदन्त पू॒र्वीः। अ॒य राजा॒ जग॑तश्चर्\mbox{}षणी॒नाम्। स ए॒व वी॒रः स उ॑ वी॒र्या॑वान्। स ए॑करा॒जो जग॑तः पर॒स्पाः। य॒दा वृ॒त्रमत॑र॒च्छूर॒ इन्द्र॑। अथा॑भवद्दमि॒ताभिक्र॑तूनाम्। इन्द्रो॑ य॒ज्ञं व॒र्धय॑न्वि॒श्ववे॑दाः। पु॒रो॒डाश॑स्य जुषता ह॒विर्न॑। वृ॒त्रन्ती॒र्त्वा दा॑न॒वं वज्र॑बाहुः॥२१॥

%2.8.3.8
दिशो॑ऽदृहद्दृहि॒ता दृह॑णेन। इ॒मं य॒ज्ञं व॒र्धय॑न्वि॒श्ववे॑दाः। पु॒रो॒डाशं॒ प्रति॑ गृभ्णा॒त्विन्द्र॑। य॒दा वृ॒त्रमत॑र॒च्छूर॒ इन्द्र॑। अथै॑करा॒जो अ॑भव॒ज्जना॑नाम्। इन्द्रो॑ दे॒वाञ्छ॑म्बर॒हत्य॑ आवत्। इन्द्रो॑ दे॒वाना॑मभवत्पुरो॒गाः। इन्द्रो॑ य॒ज्ञे ह॒विषा॑ वावृधा॒नः। वृ॒त्र॒तूर्नो॒ अभ॑य॒ शर्म॑ यसत्। यः स॒प्त सिन्धू॒ रद॑धात्पृथि॒व्याम्। यः स॒प्त लो॒कानकृ॑णो॒द्दिश॑श्च। इन्द्रो॑ ह॒विष्मा॒न्त्सग॑णो म॒रुद्भि॑। वृ॒त्र॒तूर्नो॑ य॒ज्ञमि॒होप॑ यासत्॥२२॥\anuvakamend[व॒व॒र्थ॒ वि॒त्स॒ इन्द्र॑स्तु॒रायास्तु वृत्र॒तूर्ये॒ वज्र॑बाहुः पृथि॒व्यान्त्रीणि॑ च]

%2.8.4.1
इन्द्र॒स्तर॑स्वानभिमाति॒होग्रः। हिर॑ण्यवाशीरिषि॒रः सु॑व॒र्॒षाः। तस्य॑ व॒य सु॑म॒तौ य॒ज्ञिय॑स्य। अपि॑ भ॒द्रे सौ॑मन॒से स्या॑म। हिर॑ण्यवर्णो॒ अभ॑यङ्कृणोतु। अ॒भि॒मा॒ति॒हेन्द्र॒ पृत॑नासु जि॒ष्णुः। स न॒ शर्म॑ त्रि॒वरू॑थं॒ वि यसत्। यू॒यं पा॑त स्व॒स्तिभि॒ सदा॑ नः। इन्द्र स्तुहि व॒ज्रिण॒ स्तोम॑पृष्ठम्। पु॒रो॒डाश॑स्य जुषता ह॒विर्न॑॥२३॥

%2.8.4.2
ह॒त्वाभिमा॑ती॒ पृत॑ना॒ सह॑स्वान्। अथाभ॑यङ्कृणुहि वि॒श्वतो॑ नः। स्तु॒हि शूरं॑ व॒ज्रिण॒मप्र॑तीत्तम्। अ॒भि॒मा॒ति॒हनं॑ पुरुहू॒तमिन्द्रम्। य एक॒ इच्छ॒तप॑ति॒र्जने॑षु। तस्मा॒ इन्द्रा॑य ह॒विरा जु॑होत। इन्द्रो॑ दे॒वाना॑मधि॒पाः पु॒रोहि॑तः। दि॒शां पति॑रभवद्वा॒जिनी॑वान्। अ॒भि॒मा॒ति॒हा त॑वि॒षस्तुवि॑ष्मान्। अ॒स्मभ्यं॑ चित्रं वृष॑ण र॒यिन्दात्॥२४॥

%2.8.4.3
य इ॒मे द्यावा॑पृथि॒वी म॑हि॒त्वा। बले॒नादृहदभिमाति॒हेन्द्र॑। स नो॑ ह॒विः प्रति॑ गृभ्णातु रा॒तये। दे॒वानान्दे॒वो नि॑धि॒पा नो॑ अव्यात्। अन॑वस्ते॒ रथं॒ वृष्णे॒ यत्ते। इन्द्र॑स्य॒ नु वी॒र्याण्यह॒न्नहिम्। इन्द्रो॑ या॒तोऽव॑सितस्य॒ राजा। शम॑स्य च शृ॒ङ्गिणो॒ वज्र॑बाहुः। सेदु॒ राजा क्षेति चर्\mbox{}षणी॒नाम्। अ॒रान्न ने॒मिः परि॒ ता ब॑भूव॥२५॥

%2.8.4.4
अ॒भि सि॒ध्मो अ॑जिगादस्य॒ शत्रून्॑। विति॒ग्मेन॑ वृष॒भेणा॒ पुरो॑भेत्। सं वज्रे॑णासृजद्वृ॒त्रमिन्द्र॑। प्र स्वां म॒तिम॑तिर॒च्छाश॑दानः। विष्णुं॑ दे॒वं वरु॑णमू॒तये॒ भगम्। मेद॑सा दे॒वा व॒पया॑ यजध्वम्। ता नो॑ य॒ज्ञमाग॑तं वि॒श्वधे॑ना। प्र॒जाव॑द॒स्मे द्रवि॑णे॒ह ध॑त्तम्। मेद॑सा दे॒वा व॒पया॑ यजध्वम्। विष्णुं॑ च दे॒वं वरु॑णं च रा॒तिम्॥२६॥

%2.8.4.5
ता नो॒ अमी॑वा अप॒ बाध॑मानौ। इ॒मं य॒ज्ञं जु॒षमा॑णा॒वुपेतम्। विष्णू॑वरुणा यु॒वम॑ध्व॒राय॑ नः। वि॒शे जना॑य॒ महि॒ शर्म॑ यच्छतम्। दी॒र्घप्र॑यज्ज्यू ह॒विषा॑ वृधा॒ना। ज्योति॒षाऽरा॑तीर्दहत॒न्तमासि। ययो॒रोज॑सा स्कभि॒ता रजासि। वी॒र्ये॑भिर्वी॒रत॑मा॒ शवि॑ष्ठा। याऽपत्ये॑ ते॒ अप्र॑तीत्ता॒ सहो॑भिः। विष्णू॑ अग॒न्वरु॑णा पू॒र्वहू॑तौ॥२७॥

%2.8.4.6
विष्णू॑वरुणावभिशस्ति॒पावाम्। दे॒वा य॑जन्त ह॒विषा॑ घृ॒तेन॑। अपामी॑वा सेधत र॒क्षस॑श्च। अथा॑धत्तं॒ यज॑मानाय॒ शय्योँः। अ॒हो॒मुचा॑ वृष॒भा सु॒प्रतूर्ती। दे॒वानान्दे॒वत॑मा॒ शचि॑ष्ठा। विष्णू॑वरुणा॒ प्रति॑हर्यतन्नः। इ॒दन्नरा॒ प्रय॑तमू॒तये॑ ह॒विः। म॒ही नु द्यावा॑पृथि॒वी इ॒ह ज्येष्ठे। रु॒चा भ॑वता शु॒चय॑द्भिर॒र्कैः॥२८॥

%2.8.4.7
यत्सीं॒ वरि॑ष्ठे बृह॒ती वि॑मि॒न्वन्। नृ॒वद्भ्यो॒क्षा प॑प्रथा॒नेभि॒रेवै। प्रपूर्व॒जे पि॒तरा॒ नव्य॑सीभिः। गी॒र्भिः कृ॑णुध्व॒ सद॑ने ऋ॒तस्य॑। आ नो द्यावापृथिवी॒ दैव्ये॑न। जने॑न यातं॒ महि॑ वां॒ वरू॑थम्। स इत्स्वपा॒ भुव॑नेष्वास। य इ॒मे द्यावा॑पृथि॒वी ज॒जान॑। उ॒र्वी ग॑भी॒रे रज॑सी सु॒मेके। अ॒व॒शे धीर॒ शच्या॒ समै॑रत्॥२९॥

%2.8.4.8
भूरि॒न्द्वे अच॑रन्ती॒ चर॑न्तम्। प॒द्वन्त॒ङ्गर्भ॑म॒पदी॑दधाते। नित्य॒न्न सू॒नुं पि॒त्रोरु॒पस्थे। तं पि॑पृत रोदसी सत्य॒वाचम्। इ॒दन्द्या॑वापृथिवी स॒त्यम॑स्तु। पित॒र्मात॒र्यदि॒होप॑ ब्रु॒वे वाम्। भू॒तन्दे॒वाना॑मव॒मे अवो॑भिः। विद्यामे॒षं वृ॒जनं॑ जी॒रदा॑नुम्। उ॒र्वी पृ॒थ्वी ब॑हु॒ले दू॒रे अ॑न्ते। उप॑ ब्रुवे॒ नम॑सा य॒ज्ञे अ॒स्मिन्। दधा॑ते॒ ये सु॒भगे॑ सु॒प्रतूर्ती। द्यावा॒ रक्ष॑तं पृथि॒वी नो॒ अभ्वात्। या जा॒ता ओष॑ध॒योऽति॒ विश्वा परि॒ष्ठाः। या ओष॑धय॒ सोम॑राज्ञीरश्वाव॒ती सो॑मव॒तीम्। ओष॑धी॒रिति॑ मातरो॒ऽन्या वो॑ अ॒न्याम॑वतु॥३०॥\anuvakamend[ह॒विर्नो॑ दाद्भभूव रा॒तिं पू॒र्वहू॑ताव॒र्कैरै॑रद॒स्मिन्पञ्च॑ च]

%2.8.5.1
शुचि॒न्नु स्तोम॒ श्ञथ॑द्वृ॒त्रम्। उ॒भा वा॑मिन्द्राग्नी॒ प्र च॑र्\mbox{}ष॒णिभ्य॑। आ वृ॑त्रहणा गी॒र्भिर्विप्र॑। ब्रह्म॑णस्पते॒ त्वम॒स्य य॒न्ता। सू॒क्तस्य॑ बोधि॒ तन॑यं च जिन्व। विश्व॒न्तद्भ॒द्रं यद॒वन्ति॑ दे॒वाः। बृ॒हद्व॑देम वि॒दथे॑ सु॒वीरा। स ई स॒त्येभि॒ सखि॑भिः शु॒चद्भि॑। गोधा॑यसं॒ विध॑न॒सैर॑तर्दत्। ब्रह्म॑ण॒स्पति॒र्वृष॑भिर्व॒राहै॥३१॥

%2.8.5.2
घ॒र्मस्वे॑देभि॒र्द्रवि॑ण॒व्व्याँ॑नट्। ब्रह्म॑ण॒स्पते॑रभवद्यथाव॒शम्। स॒त्यो म॒न्युर्महि॒ कर्मा॑ करिष्य॒तः। यो गा उ॒दाज॒त्स दि॒वे वि चा॑भजत्। म॒हीव॑ री॒तिः शव॑सा सर॒त्पृथ॑क्। इन्धा॑नो अ॒ग्निं व॑नवद्वनुष्य॒तः। कृ॒तब्र॑ह्मा शूशुवद्रा॒तह॑व्य॒ इत्। जा॒तेन॑ जा॒तमति॒सृत्प्र सृसते। यं य॒य्युँज॑ङ्कृणु॒ते ब्रह्म॑ण॒स्पति॑। ब्रह्म॑णस्पते सु॒यम॑स्य वि॒श्वहा॥३२॥

%2.8.5.3
रा॒यः स्या॑म र॒थ्यो॑ विव॑स्वतः। वी॒रेषु॑ वी॒रा उप॑पृङ्ग्धि न॒स्त्वम्। यदीशा॑नो॒ ब्रह्म॑णा॒ वेषि॑ मे॒ हवम्। स इज्जने॑न॒ स वि॒शा स जन्म॑ना। स पु॒त्रैर्वाजं॑ भरते॒ धना॒ नृभि॑। दे॒वानां॒ यः पि॒तर॑मा॒ विवा॑सति। श्र॒द्धाम॑ना ह॒विषा॒ ब्रह्म॑ण॒स्पतिम्। यास्ते॑ पूष॒न्नावो॑ अ॒न्तः। शु॒क्रन्ते॑ अ॒न्यत्पू॒षेमा आशा। प्रप॑थे प॒थाम॑जनिष्ट पू॒षा ॥३३॥

%2.8.5.4
प्रप॑थे दि॒वः प्रप॑थे पृथि॒व्याः। उ॒भे अ॒भि प्रि॒यत॑मे स॒धस्थे। आ च॒ परा॑ च चरति प्रजा॒नन्। पू॒षा सु॒बन्धु॑र्दि॒व आ पृ॑थि॒व्याः। इ॒डस्पति॑र्म॒घवा॑ द॒स्मव॑र्चाः। तन्दे॒वासो॒ अद॑दुः सू॒र्यायै। कामे॑न कृ॒तन्त॒वस॒ स्वञ्चम्। अ॒जाऽश्व॑ पशु॒पा वाज॑बस्त्यः। धि॒यं॒ जि॒न्वो विश्वे॒ भुव॑ने॒ अर्पि॑तः। अष्ट्रां पू॒षा शि॑थि॒रामु॒द्वरी॑वृजत्॥३४॥

%2.8.5.5
स॒ञ्चक्षा॑णो॒ भुव॑ना दे॒व ई॑यते। शुची॑ वो ह॒व्या म॑रुत॒ शुची॑नाम्। शुचि हिनोम्यध्व॒र शुचि॑भ्यः। ऋ॒तेन॑ स॒त्यमृत॒साप॑ आयन्। शुचि॑जन्मान॒ शुच॑यः पाव॒काः। प्र चि॒त्रम॒र्कं गृ॑ण॒ते तु॒राय॑। मारु॑ताय॒ स्वत॑वसे भरध्वम्। ये सहासि॒ सह॑सा॒ सह॑न्ते। रेज॑ते अग्ने पृथि॒वी म॒खेभ्य॑। असे॒ष्वा म॑रुतः खा॒दयो॑ वः॥३५॥

%2.8.5.6
वक्ष॑ सुरु॒क्मा उप॑ शिश्रिया॒णाः। वि वि॒द्युतो॒ न वृ॒ष्टिभी॑ रुचा॒नाः। अनु॑ स्व॒धामायु॑धै॒र्यच्छ॑मानाः। या व॒ शर्म॑ शशमा॒नाय॒ सन्ति॑। त्रि॒धातू॑नि दा॒शुषे॑ यच्छ॒ताधि॑। अ॒स्मभ्य॒न्तानि॑ मरुतो॒ विय॑न्त। र॒यिन्नो॑ धत्त वृषणः सु॒वीरम्। इ॒मे तु॒रं म॒रुतो॑ रामयन्ति। इ॒मे सह॒ सह॑स॒ आ न॑मन्ति। इ॒मे शसं॑वनुष्य॒तो नि पान्ति॥३६॥

%2.8.5.7
गु॒रुद्वेषो॒ अर॑रुषे दधन्ति। अ॒रा इ॒वेदच॑रमा॒ अहे॑व। प्रप्र॑ जायन्ते॒ अक॑वा॒ महो॑भिः। पृश्ञे प्रु॒त्रा उ॑प॒मासो॒ रभि॑ष्ठाः। स्वया॑ म॒त्या म॒रुत॒ सं मि॑मिक्षुः। अनु॑ ते दायि म॒ह इ॑न्द्रि॒याय॑। स॒त्रा ते॒ विश्व॒मनु॑ वृत्र॒हत्ये। अनु॑ क्ष॒त्रमनु॒ सहो॑ यजत्र। इन्द्र॑ दे॒वेभि॒रनु॑ ते नृ॒षह्ये। य इन्द्र॒ शुष्मो॑ मघवन्ते॒ अस्ति॑॥३७॥

%2.8.5.8
शिक्षा॒ सखि॑भ्यः पुरुहूत॒ नृभ्य॑। त्व हि दृ॒ढा म॑घव॒न्विचे॑ताः। अपा॑वृधि॒ परि॑वृति॒न्न राध॑। इन्द्रो॒ राजा॒ जग॑तश्चर्‌षणी॒नाम्। अ॒धि॒क्षमि॒ विषु॑रूपं॒ यदस्ति॑। ततो॑ ददातु दा॒शुषे॒ वसू॑नि। चोद॒द्राध॒ उप॑स्तुतश्चिद॒र्वाक्। तमु॑ष्टुहि॒ यो अ॒भिभूत्योजाः। व॒न्वन्नवा॑तः पुरुहू॒त इन्द्र॑। अषा॑ढमु॒ग्र सह॑मानमा॒भिः॥३८॥

%2.8.5.9
गी॒र्भिर्व॑र्ध वृष॒भञ्च॑र्\mbox{}षणी॒नाम्। स्थू॒रस्य॑ रा॒यो बृ॑ह॒तो य ईशे। तमु॑ ष्टवाम वि॒दथे॒ष्विन्द्रम्। यो वा॒युना॒ जय॑ति॒ गोम॑तीषु। प्र धृ॑ष्णु॒या न॑यति॒ वस्यो॒ अच्छ॑। आ ते॒ शुष्मो॑ वृष॒भ ए॑तु प॒श्चात्। ओत्त॒राद॑ध॒रागा पु॒रस्तात्। आ वि॒श्वतो॑ अ॒भिसमेत्व॒र्वाङ्। इन्द्र॑ द्यु॒म्न सुव॑र्वद्धेह्य॒स्मे॥३९॥\anuvakamend[व॒राहैर्वि॒श्वहा॑ऽजनिष्ट पू॒षोद्वरी॑वृजत्खा॒दयो॑ वः पा॒न्त्यस्त्या॒भिर्नव॑ च]

%2.8.6.1
आ दे॒वो या॑तु सवि॒ता सु॒रत्न॑। अ॒न्त॒रि॒क्ष॒प्रा वह॑मानो॒ अश्वै। हस्ते॒ दधा॑नो॒ नर्या॑ पु॒रूणि॑। नि॒वे॒शयं॑ च प्रसु॒वं च॒ भूम॑। अ॒भीवृ॑त॒ङ्कृश॑नैर्वि॒श्वरू॑पम्। हिर॑ण्यशम्यं यज॒तो बृ॒हन्तम्। आस्था॒द्रथ सवि॒ता चि॒त्रभा॑नुः। कृ॒ष्णा रजा सि॒ तवि॑षी॒न्दधा॑नः। सघा॑ नो दे॒वः स॑वि॒ता स॒वाय॑। आ सा॑विष॒द्वसु॑पति॒र्वसू॑नि॥४०॥

%2.8.6.2
वि॒श्रय॑माणो॒ अम॑तिमुरू॒चीम्। म॒र्त॒भोज॑न॒मध॑रासतेन। विजनाञ्छ्या॒वाः शि॑ति॒पादो॑ अख्यन्। रथ॒ हिर॑ण्यप्रउगं॒ वह॑न्तः। शश्व॒द्दिश॑ सवि॒तुर्दैव्य॑स्य। उ॒पस्थे॒ विश्वा॒ भुव॑नानि तस्थुः। वि सु॑प॒र्णो अ॒न्तरि॑क्षाण्यख्यत्। ग॒भी॒रवे॑पा॒ असु॑रः सुनी॒थः। क्वे॑दानी॒ सूर्य॒ कश्चि॑केत। क॒त॒मान्द्या र॒श्मिर॒स्या त॑तान॥४१॥

%2.8.6.3
भग॒न्धियं॑ वा॒जय॑न्त॒ पुर॑न्धिम्। नरा॒शसो॒ ग्नास्पति॑र्नो अव्यात्। आ ये वा॒मस्य॑ सङ्ग॒थे र॑यी॒णाम्। प्रि॒या दे॒वस्य॑ सवि॒तुः स्या॑म। आ नो॒ विश्वे॒ अस्क्रा॑गमन्तु दे॒वाः। मि॒त्रो अ॑र्य॒मा वरु॑णः स॒जोषा। भुव॒न्॒ यथा॑ नो॒ विश्वे॑ वृ॒धास॑। करन्त्सु॒षाहा॑ विथु॒रन्न शव॑। शन्नो॑ दे॒वा वि॒श्वदे॑वा भवन्तु। श सर॑स्वती स॒ह धी॒भिर॑स्तु॥४२॥

%2.8.6.4
शम॑भि॒षाच॒ शमु॑ राति॒षाच॑। शन्नो॑ दि॒व्याः पार्थि॑वा॒ शन्नो॒ अप्या। ये स॑वि॒तुः स॒त्यस॑वस्य॒ विश्वे। मि॒त्रस्य॑ व्र॒ते वरु॑णस्य दे॒वाः। ते सौभ॑गं वी॒रव॒द्गोम॒दप्न॑। दधा॑तन॒ द्रवि॑णञ्चि॒त्रम॒स्मे। अग्ने॑ या॒हि दू॒त्यं॑ वारि॑षेण्यः। दे॒वा अच्छा ब्रह्म॒कृता॑ ग॒णेन॑। सर॑स्वतीं म॒रुतो॑ अ॒श्विना॒पः। य॒क्षि॒ दे॒वान्र॑त्न॒धेया॑य॒ विश्वान्॑॥४३॥

%2.8.6.5
द्यौः पि॑त॒ पृथि॑वि॒ मात॒रध्रु॑क्। अग्ने भ्रातर्वसवो मृ॒डता॑ नः। विश्व॑ आदित्या अदिते स॒जोषा। अ॒स्मभ्य॒ शर्म॑ बहु॒लं वि य॑न्त। विश्वे॑ देवाः शृणु॒तेम हवं॑ मे। ये अ॒न्तरि॑क्षे॒ य उप॒ द्यवि॒ ष्ठ। ये अ॑ग्निजि॒ह्वा उ॒त वा॒ यज॑त्राः। आ॒सद्या॒स्मिन्ब॒र्॒हिषि॑ मादयध्वम्। आ वां मित्रावरुणा ह॒व्यजु॑ष्टिम्। नम॑सा देवा॒वव॑साववृत्याम्॥४४॥

%2.8.6.6
अ॒स्माकं॒ ब्रह्म॒ पृत॑नासु सह्या अ॒स्माकम्। वृ॒ष्टिर्दि॒व्या सु॑पा॒रा। यु॒वं वस्त्रा॑णि पीव॒सा व॑साथे। यु॒वोरच्छि॑द्रा॒ मन्त॑वो ह॒ सर्गा। अवा॑तिरत॒मनृ॑तानि॒ विश्वा। ऋ॒तेन॑ मित्रावरुणा सचेथे। तत्सु वां मित्रावरुणा महि॒त्वम्। ई॒र्मा त॒स्थुषी॒रह॑भिर्दुदुह्रे। विश्वा पिन्वथ॒ स्वस॑रस्य॒ धेना। अनु॑ वा॒मेक॑ प॒विरा व॑वर्ति॥४५॥

%2.8.6.7
यद्बहि॑ष्ठ॒न्नाति॒ विदे॑ सुदानू। अच्छि॑द्र॒ शर्म॒ भुव॑नस्य गोपा। ततो॑ नो मित्रावरुणाववीष्टम्। सिषा॑सन्तो जी(जि?)गि॒वास॑ स्याम। आ नो मित्रावरुणा ह॒व्यदा॑तिम्। घृ॒तैर्गव्यू॑तिमुक्षत॒मिडा॑भिः। प्रति॑ वा॒मत्र॒ वर॒मा जना॑य। पृ॒णी॒तमु॒द्नो दि॒व्यस्य॒ चारो। प्र बा॒हवा॑ सिसृतञ्जी॒वसे॑ नः। आ नो॒ गव्यू॑तिमुक्षतङ्घृ॒तेन॑॥४६॥

%2.8.6.8
आ नो॒ जने श्रवयतय्युँवाना। श्रु॒तं मे॑ मित्रावरुणा॒ हवे॒मा। इ॒मा रु॒द्राय॑ स्थि॒रध॑न्वने॒ गिर॑। क्षि॒प्रेष॑वे दे॒वाय॑ स्व॒धाम्ने। अषा॑ढाय॒ सह॑मानाय मी॒ढुषे। ति॒ग्मायु॑धाय भरता शृ॒णोत॑न। त्वाद॑त्तेभी रुद्र॒ शन्त॑मेभिः। श॒त हिमा॑ अशीय भेष॒जेभि॑। व्य॑स्मद्द्वेषो॑ वित॒रव्व्यँह॑। व्यमी॑वाश्चातयस्वा॒ विषू॑चीः॥४७॥

%2.8.6.9
अर्\mbox{}ह॑न्बिभर्\mbox{}षि॒ मा न॑स्तो॒के। आ ते॑ पितर्मरुता सु॒म्नमे॑तु। मा न॒ सूर्य॑स्य स॒न्दृशो॑ युयोथाः। अ॒भि नो॑ वी॒रो अर्व॑ति क्षमेत। प्र जा॑येमहि रुद्र प्र॒जाभि॑। ए॒वा ब॑भ्रो वृषभ चेकितान। यथा॑ देव॒ न हृ॑णी॒षे न हसि॑। हा॒व॒न॒श्रूर्नो॑ रुद्रे॒ह बो॑धि। बृ॒हद्व॑देम वि॒दथे॑ सु॒वीरा। परि॑ णो रु॒द्रस्य॑ हे॒तिः स्तु॒हि श्रु॒तम्। मीढु॑ष्ट॒मार्\mbox{}ह॑न्बिभर्\mbox{}षि। त्वम॑ग्ने रु॒द्र आ वो॒ राजा॑नम्॥४८॥\anuvakamend[वसू॑नि ततानास्तु॒ विश्वान्॑ ववृत्यां ववर्ति घृ॒तेन॒ विषू॑चीः श्रु॒तन्द्वे च॑]

%2.8.7.1
सूर्यो॑ दे॒वीमु॒षस॒ रोच॑माना॒मर्य॑। न योषा॑म॒भ्ये॑ति प॒श्चात्। यत्रा॒ नरो॑ देव॒यन्तो॑ यु॒गानि॑। वि॒त॒न्वते॒ प्रति॑ भ॒द्राय॑ भ॒द्रम्। भ॒द्रा अश्वा॑ ह॒रित॒ सूर्य॑स्य। चि॒त्रा एद॑ग्वा अनु॒माद्या॑सः। न॒म॒स्यन्तो॑ दि॒व आ पृ॒ष्ठम॑स्थुः। परि॒ द्यावा॑पृथि॒वी य॑न्ति स॒द्यः। तत्सूर्य॑स्य देव॒त्वन्तन्म॑हि॒त्वम्। म॒ध्या कर्तो॒र्वित॑त॒ सञ्ज॑भार॥४९॥

%2.8.7.2
य॒देदयु॑क्त ह॒रित॑ स॒धस्थात्। आद्रात्री॒ वास॑स्तनुते सि॒मस्मै। तन्मि॒त्रस्य॒ वरु॑णस्याभि॒चक्षे। सूर्यो॑ रू॒पं कृ॑णुते॒ द्योरु॒पस्थे। अ॒न॒न्तम॒न्यद्रुश॑दस्य॒ पाज॑। कृ॒ष्णम॒न्यद्ध॒रित॒ सं भ॑रन्ति। अ॒द्या दे॑वा॒ उदि॑ता॒ सूर्य॑स्य। निरह॑सः पिपृ॒तान्निर॑व॒द्यात्। तन्नो॑ मि॒त्रो वरु॑णो मामहन्ताम्। अदि॑ति॒ सिन्धु॑ पृथि॒वी उ॒त द्यौः॥५०॥

%2.8.7.3
दि॒वो रु॒क्म उ॑रु॒चक्षा॒ उदे॑ति। दू॒रे अ॑र्थस्त॒रणि॒र्भ्राज॑मानः। नू॒नञ्जना॒ सूर्ये॑ण॒ प्रसू॑ताः। आयन्नर्था॑नि कृ॒णव॒न्नपासि। शन्नो॑ भव॒ चक्ष॑सा॒ शन्नो॒ अह्ना। शं भा॒नुना॒ श हि॒मा शङ्घृ॒णेन॑। यथा॒ शम॒स्मै शमस॑द्दुरो॒णे। तत्सूर्य॒ द्रवि॑णन्धे॒हि चि॒त्रम्। चि॒त्रन्दे॒वाना॒मुद॑गा॒दनी॑कम्। चक्षु॑र्मि॒त्रस्य॒ वरु॑णस्या॒ग्नेः॥५१॥

%2.8.7.4
आप्रा॒ द्यावा॑पृथि॒वी अ॒न्तरि॑क्षम्। सूर्य॑ आ॒त्मा जग॑तस्त॒स्थुष॑श्च। त्वष्टा॒ दध॒त्तन्न॑स्तु॒रीपम्। त्वष्टा॑ वी॒रं पि॒शङ्ग॑रूपः। दशे॒मन्त्वष्टु॑र्जनयन्त॒ गर्भम्। अत॑न्द्रासो युव॒तयो॒ बिभ॑र्त्रम्। ति॒ग्मानी॑क॒ स्वय॑शस॒ञ्जने॑षु। वि॒रोच॑मानं॒ परि॑षीन्नयन्ति। आविष्ट्यो॑ वर्धते॒ चारु॑रासु। जि॒ह्माना॑मू॒र्ध्वस्वय॑शा उ॒पस्थे॥५२॥

%2.8.7.5
उ॒भे त्वष्टु॑र्बिभ्यतु॒र्जाय॑मानात्। प्र॒तीची॑ सि॒हं प्रति॑जोषयेते। मि॒त्रो जना॒न्प्र स मि॑त्र। अ॒यं मि॒त्रो न॑म॒स्य॑ सु॒शेव॑। राजा॑ सुक्ष॒त्रो अ॑जनिष्ट वे॒धाः। तस्य॑ व॒य सु॑म॒तौ य॒ज्ञिय॑स्य। अपि॑ भ॒द्रे सौ॑मन॒से स्या॑म। अ॒न॒मी॒वास॒ इड॑या॒ मद॑न्तः। मि॒तज्म॑वो॒ वरि॑म॒न्ना पृ॑थि॒व्याः। आ॒दि॒त्यस्य॑ व्र॒तमु॑प॒क्ष्यन्त॑॥५३॥

%2.8.7.6
व॒यं मि॒त्रस्य॑ सुम॒तौ स्या॑म। मि॒त्रन्न ई शिम्या॒ गोषु॑ ग॒व्यव॑त्। स्वा॒धियो॑ वि॒दथे॑ अ॒प्स्वजी॑जनन्। अरे॑जयता॒ रोद॑सी॒ पाज॑सा गि॒रा। प्रति॑ प्रि॒यं य॑ज॒तञ्ज॒नुषा॒मव॑। म॒हा आ॑दि॒त्यो नम॑सोप॒सद्य॑। या॒त॒यज्ज॑नो गृण॒ते सु॒शेव॑। तस्मा॑ ए॒तत्पन्य॑तमाय॒ जुष्टम्। अ॒ग्नौ मि॒त्राय॑ ह॒विरा जु॑होत। आ वा॒ रथो॒ रोद॑सी बद्बधा॒नः॥५४॥

%2.8.7.7
हि॒र॒ण्ययो॒ वृष॑भिर्या॒त्वश्वै। घृ॒तव॑र्तनिः प॒विभी॑रुचा॒नः। इ॒षाव्वोँ॒ढा नृ॒पति॑र्वा॒जिनी॑वान्। स प॑प्रथा॒नो अ॒भि पञ्च॒ भूम॑। त्रि॒व॒न्धु॒रो मन॒साया॑तु यु॒क्तः। विशो॒ येन॒ गच्छ॑थो देव॒यन्ती। कुत्रा॑ चि॒द्याम॑मश्विना॒ दधा॑ना। स्वश्वा॑ य॒शसाऽऽया॑तम॒र्वाक्। दस्रा॑ नि॒धिं मधु॑मन्तं पिबाथः। वि वा॒ रथो॑ व॒ध्वा॑ याद॑मानः॥५५॥

%2.8.7.8
अन्तान्दि॒वो बा॑धते वर्त॒निभ्याम्। यु॒वोः श्रियं॒ परि॒ योषा॑वृणीत। सूरो॑ दुहि॒ता परि॑तक्मियायाम्। यद्दे॑व॒यन्त॒मव॑थ॒ शची॑भिः। परि॑घ्र॒ सवां॒ मना॑वां॒ वयो॑गाम्। यो ह॒स्यवा रथिरा॒वस्त॑ उ॒स्राः। रथो॑ युजा॒नः प॑रि॒याति॑ व॒र्तिः। तेन॑ न॒ शँय्योरु॒षसो॒ व्यु॑ष्टौ। न्य॑श्विना वहतं य॒ज्ञे अ॒स्मिन्। यु॒वं भु॒ज्युमव॑विद्ध समु॒द्रे॥५६॥

%2.8.7.9
उदू॑हथु॒रर्ण॑सो॒ अस्रि॑धानैः। प॒त॒त्रिभि॑रश्र॒मैर॑व्य॒थिभि॑। द॒सना॑भिरश्विना पा॒रय॑न्ता। अग्नी॑षोमा॒ यो अ॒द्य वाम्। इ॒दं वच॑ सप॒र्यति॑। तस्मै॑ धत्त सु॒वीर्यम्। गवां॒ पोष॒ स्वश्वि॑यम्। यो अ॒ग्नीषोमा॑ ह॒विषा॑ सप॒र्यात्। दे॒व॒द्रीचा॒ मन॑सा॒ यो घृ॒तेन॑। तस्य॑ व्र॒त र॑क्षतं पा॒तमह॑सः॥५७॥

%2.8.7.10
वि॒शे जना॑य॒ महि॒ शर्म॑ यच्छतम्। अग्नी॑षोमा॒ य आहु॑तिम्। यो वा॒न्दाशाद्ध॒विष्कृ॑तिम्। स प्र॒जया॑ सु॒वीर्यम्। विश्व॒मायु॒र्व्य॑श्ञवत्। अग्नी॑षोमा॒ चेति॒ तद्वी॒र्यं॑ वाम्। यदमु॑ष्णीतमव॒सं प॒णिङ्गोः। अवा॑तिरतं॒ प्रथ॑यस्य॒ शेष॑। अवि॑न्दतं॒ ज्योति॒रेकं॑ ब॒हुभ्य॑। अग्नी॑षोमावि॒म सु मेऽग्नी॑षोमा ह॒विष॒ प्रस्थि॑तस्य॥५८॥\anuvakamend[ज॒भा॒र॒ द्यौर॒ग्नेरु॒पस्थ॑ उप॒क्ष्यन्तो॑ बद्बधा॒नो व॒ध्वा॑ याद॑मानः समु॒द्रेऽह॑स॒ प्रस्थि॑तस्य]

%2.8.8.1
अ॒हम॑स्मि प्रथम॒जा ऋ॒तस्य॑। पूर्वं॑ दे॒वेभ्यो॑ अ॒मृत॑स्य॒ नाभि॑। यो मा॒ ददा॑ति॒ स इदे॒वमावा। अ॒हमन्न॒मन्न॑न॒दन्त॑मद्मि। पूर्व॑म॒ग्नेरपि॑ दह॒त्यन्नम्। य॒त्तौ हा॑साते अहमुत्त॒रेषु॑। व्यात्त॑मस्य प॒शव॑ सु॒जम्भम्। पश्य॑न्ति॒ धीरा॒ प्रच॑रन्ति॒ पाका। जहाम्य॒न्यन्न ज॑हाम्य॒न्यम्। अ॒हमन्नं॒ वश॒मिच्च॑रामि॥५९॥

%2.8.8.2
स॒मा॒नमर्थं॒ पर्ये॑मि भु॒ञ्जत्। को मामन्नं॑ मनु॒ष्यो॑ दयेत। परा॑के॒ अन्न॒न्निहि॑तं लो॒क ए॒तत्। विश्वैर्दे॒वैः पि॒तृभि॑र्गु॒प्तमन्नम्। यद॒द्यते॑ लु॒प्यते॒ यत्प॑रो॒प्यते। श॒त॒त॒मी सा त॒नूर्मे॑ बभूव। म॒हान्तौ॑ च॒रू स॑कृद्दु॒ग्धेन॑ पप्रौ। दिवं॑ च॒ पृश्ञि॑ पृथि॒वीं च॑ सा॒कम्। तत्सं॒पिब॑न्तो॒ न मि॑नन्ति वे॒धस॑। नैतद्भूयो॒ भव॑ति॒ नो कनी॑यः॥६०॥

%2.8.8.3
अन्नं॑ प्रा॒णमन्न॑मपा॒नमा॑हुः। अन्नं॑ मृ॒त्युन्तमु॑ जी॒वातु॑माहुः। अन्नं॑ ब्र॒ह्माणो॑ ज॒रसं॑  वदन्ति। अन्न॑माहुः प्र॒जन॑नं प्र॒जानाम्। मोघ॒मन्नं॑ विन्दते॒ अप्र॑चेताः। स॒त्यं ब्र॑वीमि व॒ध इत्स तस्य॑। नार्य॒मणं॒ पुष्य॑ति॒ नो सखा॑यम्। केव॑लाघो भवति केवला॒दी। अ॒हं मे॒घः स्त॒नय॒न्वऱ़्ष॑न्नस्मि। माम॑दन्त्य॒हम॑द्म्य॒न्यान्॥६१॥

%2.8.8.4
अ॒ह सद॒मृतो॑ भवामि। मदा॑दि॒त्या अधि॒ सर्वे॑ तपन्ति। दे॒वीं वाच॑मजनयन्त॒ यद्वाग्वद॑न्ती। अ॒न॒न्तामन्ता॒दधि॒ निर्मि॑तां म॒हीम्। यस्यान्दे॒वा अ॑दधु॒र्भोज॑नानि। एकाक्षरां द्वि॒पदा॒ षट्प॑दां च। वाचं॑ दे॒वा उप॑ जीवन्ति॒ विश्वे। वाचं॑ दे॒वा उप॑ जीवन्ति॒ विश्वे। वाच॑ङ्गन्ध॒र्वाः प॒शवो॑ मनु॒ष्या। वा॒चीमा विश्वा॒ भुव॑ना॒न्यर्पि॑ता॥६२॥

%2.8.8.5
सा नो॒ हवं॑ जुषता॒मिन्द्र॑पत्नी। वाग॒क्षरं॑ प्रथम॒जा ऋ॒तस्य॑। वेदा॑नां मा॒ताऽमृत॑स्य॒ नाभि॑। सा नो॑ जुषा॒णोप॑ य॒ज्ञमागात्। अव॑न्ती दे॒वी सु॒हवा॑ मे अस्तु। यामृष॑यो मन्त्र॒कृतो॑ मनी॒षिण॑। अ॒न्वैच्छं॑ दे॒वास्तप॑सा॒ श्रमे॑ण। तान्दे॒वीं वाच ह॒विषा॑ यजामहे। सा नो॑ दधातु सुकृ॒तस्य॑ लो॒के। च॒त्वारि॒ वाक्परि॑मिता प॒दानि॑॥६३॥

%2.8.8.6
तानि॑ विदुर्ब्राह्म॒णा ये म॑नी॒षिण॑। गुहा॒ त्रीणि॒ निहि॑ता॒ नेङ्ग॑यन्ति। तु॒रीयं॑ वा॒चो म॑नु॒ष्या॑ वदन्ति। श्र॒द्धया॒ऽग्निः समि॑ध्यते। श्र॒द्धया॑ विन्दते ह॒विः। श्र॒द्धां भग॑स्य मू॒र्धनि॑। वच॒सा वे॑दयामसि। प्रि॒य श्र॑द्धे॒ दद॑तः। प्रि॒य श्र॑द्धे॒ दिदा॑सतः। प्रि॒यं भो॒जेषु॒ यज्व॑सु॥६४॥

%2.8.8.7
इ॒दं म॑ उदि॒तं कृ॑धि। यथा॑ दे॒वा असु॑रेषु। श्र॒द्धामु॒ग्रेषु॑ चक्रि॒रे। ए॒वं भो॒जेषु॒ यज्व॑सु। अ॒स्माक॑मुदि॒तं कृ॑धि। श्र॒द्धान्दे॑वा॒ यज॑मानाः। वा॒युगो॑पा॒ उपा॑सते। श्र॒द्धा हृ॑द॒य्य॑याऽऽकूत्या। श्र॒द्धया॑ हूयते ह॒विः। श्र॒द्धां प्रा॒तर्\mbox{}ह॑वामहे॥६५॥

%2.8.8.8
श्र॒द्धां म॒ध्यन्दि॑नं॒ परि॑। श्र॒द्धा सूर्य॑स्य नि॒म्रुचि॑। श्रद्धे॒ श्रद्धा॑पये॒ह मा। श्र॒द्धा दे॒वानधि॑ वस्ते। श्र॒द्धा विश्व॑मि॒दञ्जग॑त्। श्र॒द्धाङ्काम॑स्य मा॒तरम्। ह॒विषा॑ वर्धयामसि। ब्रह्म॑ जज्ञा॒नं प्र॑थ॒मं पु॒रस्तात्। वि सी॑म॒तः सु॒रुचो॑ वे॒न आ॑वः। स बु॒ध्निया॑ उप॒ मा अ॑स्य वि॒ष्ठाः॥६६॥

%2.8.8.9
स॒तश्च॒ योनि॒मस॑तश्च॒ विव॑। पि॒ता वि॒राजा॑मृष॒भो र॑यी॒णाम्। अ॒न्तरि॑क्षं वि॒श्वरू॑प॒ आवि॑वेश। तम॒र्कैर॒भ्य॑र्चन्ति व॒त्सम्। ब्रह्म॒ सन्तं॒ ब्रह्म॑णा व॒र्धय॑न्तः। ब्रह्म॑ दे॒वान॑जनयत्। ब्रह्म॒ विश्व॑मि॒दञ्जग॑त्। ब्रह्म॑णः क्ष॒त्रन्निर्मि॑तम्। ब्रह्म॑ ब्राह्म॒ण आ॒त्मना। अ॒न्तर॑स्मिन्नि॒मे लो॒काः॥६७॥

%2.8.8.10
अ॒न्तर्विश्व॑मि॒दञ्जग॑त्। ब्रह्मै॒व भू॒तानां॒ ज्येष्ठम्। तेन॒ को॑ऽर्\mbox{}हति॒ स्पर्धि॑तुम्। ब्रह्म॑न्दे॒वास्त्रय॑स्त्रिशत्। ब्रह्म॑न्निन्द्रप्रजाप॒ती। ब्रह्म॑न् ह॒ विश्वा॑ भू॒तानि॑। ना॒वीवा॒न्तः स॒माहि॑ता। चत॑स्र॒ आशा॒ प्रच॑रन्त्व॒ग्नय॑। इ॒मन्नो॑ य॒ज्ञन्न॑यतु प्रजा॒नन्। घृ॒तं पिन्व॑न्न॒जर सु॒वीरम्॥६८॥

%2.8.8.12
ब्रह्म॑ स॒मिद्भ॑व॒त्याहु॑तीनाम्। आ गावो॑ अग्मन्नु॒त भ॒द्रम॑क्रन्। सीद॑न्तु गो॒ष्ठे र॒णय॑न्त्व॒स्मे। प्र॒जाव॑तीः पुरु॒रूपा॑ इ॒ह स्युः। इन्द्रा॑य पू॒र्वीरु॒षसो॒ दुहा॑नाः। इन्द्रो॒ यज्व॑ने पृण॒ते च॑ शिक्षति। उपेद्द॑दाति॒ न स्वं मु॑षायति। भूयो॑भूयो र॒यमिद॑स्य व॒र्धय\sn{}। अभि॑न्ने खि॒ल्ले नि द॑धाति देव॒युम्। न ता न॑शन्ति॒ न ता अर्वा॥६९॥

%2.8.8.13
गावो॒ भगो॒ गाव॒ इन्द्रो॑ मे अच्छात्। गाव॒ सोम॑स्य प्रथ॒मस्य॑ भ॒क्षः। इ॒मा या गाव॒ सज॑नास॒ इन्द्र॑। इ॒च्छामीद्धृ॒दा मन॑सा चि॒दिन्द्रम्। यू॒यङ्गा॑वो मेदयथा कृ॒शञ्चि॑त्। अ॒श्ली॒लञ्चि॑त्कृणुथा सु॒प्रती॑कम्। भ॒द्रङ्गृ॒हं कृ॑णुथ भद्रवाचः। बृ॒हद्वो॒ वय॑ उच्यते स॒भासु॑। प्र॒जाव॑तीः सू॒यव॑स रि॒शन्ती। शु॒द्धा अ॒पः सु॑प्रपा॒णे पिब॑न्तीः। मा व॑ स्ते॒न ई॑शत॒ माऽघशसः। परि॑ वो हे॒ती रु॒द्रस्य॑ वृञ्ज्यात्। उपे॒दमु॑प॒पर्च॑नम्। आ॒सु गोषूप॑पृच्यताम्। उप॑र्\mbox{}ष॒भस्य॒ रेत॑सि। उपेन्द्र॒ तव॑ वी॒र्ये॥७०॥\anuvakamend[च॒रा॒मि॒ कनी॑यो॒ऽन्यानर्पि॑ता प॒दानि॒ यज्व॑सु हवामहे वि॒ष्ठा लो॒काः सु॒वीर॒मर्वा॒ पिब॑न्ती॒ष्षट्च॑]

%2.8.9.1
ता सूर्याचन्द्र॒मसा॑ विश्व॒भृत्त॑मा म॒हत्। तेजो॒ वसु॑मद्राजतो दि॒वि। सामात्माना चरतः सामचा॒रिणा। ययोर्व्र॒तन्न म॒मे जातु॑ दे॒वयो। उ॒भावन्तौ॒ परि॑ यात॒ अर्म्या। दि॒वो न र॒श्मी स्त॑नु॒तो व्य॑र्ण॒वे। उ॒भा भु॑व॒न्ती भुव॑ना क॒विक्र॑तू। सूर्या॒ न च॒न्द्रा च॑रतो ह॒ताम॑ती। पती द्यु॒मद्वि॑श्व॒विदा॑ उ॒भा दि॒वः। सूर्या॑ उ॒भा च॒न्द्रम॑सा विचक्ष॒णा॥७१॥

%2.8.9.2
वि॒श्ववा॑रा वरिवो॒भा वरेण्या। ता नो॑ऽवतं मति॒मन्ता॒ महि॑व्रता। वि॒श्व॒वप॑री प्र॒तर॑णा तर॒न्ता। सु॒व॒र्विदा॑ दृ॒शये॒ भूरि॑रश्मी। सूर्या॒ हि च॒न्द्रा वसु॑ त्वे॒षद॑र्शता। म॒न॒स्विनो॒भानु॑चर॒तोनु॒ सन्दिवम्। अ॒स्य श्रवो॑ न॒द्य॑ स॒प्त बि॑भ्रति। द्यावा॒ क्षामा॑ पृथि॒वी द॑र्\mbox{}श॒तं वपु॑। अ॒स्मे सूर्याचन्द्र॒मसा॑ऽभि॒चक्षे। श्र॒द्धेकमि॑न्द्र चरतो विचर्तु॒रम्॥७२॥

%2.8.9.3
पू॒र्वा॒प॒रञ्च॑रतो मा॒ययै॒तौ। शिशू॒ क्रीड॑न्तौ॒ परि॑ यातो अध्व॒रम्। विश्वान्य॒न्यो भुव॑नाऽभि॒ चष्टे। ऋ॒तून॒न्यो वि॒दध॑ज्जायते॒ पुन॑। हिर॑ण्यवर्णा॒ शुच॑यः पाव॒का यासा॒ राजा। यासान्दे॒वाः शि॒वेन॑ मा॒ चक्षु॑षा पश्यत। आपो॑ भ॒द्रा आदित्प॑श्यामि। नास॑दासी॒न्नो सदा॑सीत्त॒दानीम्। नासी॒द्रजो॒ नो व्यो॑मा प॒रो यत्। किमाव॑रीव॒ कुह॒ कस्य॒ शर्म\sn{}॥७३॥

%2.8.9.4
अम्भ॒ किमा॑सी॒द्गह॑नङ्गभी॒रम्। न मृ॒त्युर॒मृत॒न्तर्\mbox{}हि॒ न। रात्रि॑या॒ अह्न॑ आसीत्प्रके॒तः। आनी॑दवा॒त स्व॒धया॒ तदेकम्। तस्माद्धा॒न्यन्न प॒रः किञ्च॒नास॑। तम॑ आसी॒त्तम॑सा गू॒ढमग्रे प्रके॒तम्। स॒लि॒ल सर्व॑मा इ॒दम्। तु॒च्छेना॒भ्वपि॑हितं॒ यदासीत्। तम॑स॒स्तन्म॑हि॒ना जा॑य॒तैकम्। काम॒स्तदग्रे॒ सम॑वर्त॒ताधि॑॥७४॥

%2.8.9.5
मन॑सो॒ रेत॑ प्रथ॒मं यदासीत्। स॒तो बन्धु॒मस॑ति॒ निर॑विन्दन्। हृ॒दि प्र॒तीष्या॑ क॒वयो॑ मनी॒षा। ति॒र॒श्चीनो॒ वित॑तो र॒श्मिरे॑षाम्। अ॒धः स्वि॑दा॒सी ३ दु॒परि॑ स्विदासी ३ त्। रे॒तो॒धा आ॑सन्महि॒मान॑ आसन्। स्व॒धा अ॒वस्ता॒त्प्रय॑तिः प॒रस्तात्। को अ॒द्धा वे॑द॒ क इ॒ह प्र वो॑चत्। कुत॒ आजा॑ता॒ कुत॑ इ॒यं विसृ॑ष्टिः। अ॒र्वाग्दे॒वा अ॒स्य वि॒सर्ज॑नाय॥७५॥

%2.8.9.6
अथा॒ को वे॑द॒ यत॑ आब॒भूव॑। इ॒यं विसृ॑ष्टि॒र्यत॑ आब॒भूव॑। यदि॑ वा द॒धे यदि॑ वा॒ न। यो अ॒स्याध्य॑क्षः पर॒मे व्यो॑मन्। सो अ॒ङ्ग वे॑द॒ यदि॑ वा॒ न वेद॑। किस्वि॒द्वन॒ङ्क उ॒ स वृ॒क्ष आ॑सीत्। यतो॒ द्यावा॑पृथि॒वी नि॑ष्टत॒क्षुः। मनी॑षिणो॒ मन॑सा पृ॒च्छतेदु॒तत्। यद॒ध्यति॑ष्ठ॒द्भुव॑नानि धा॒रय\sn{}। ब्रह्म॒ वनं॒ ब्रह्म॒ स वृ॒क्ष आ॑सीत्॥७६॥

%2.8.9.7
यतो॒ द्यावा॑पृथि॒वी नि॑ष्टत॒क्षुः। मनी॑षिणो॒ मन॑सा॒ विब्र॑वीमि वः। ब्रह्मा॒ध्यति॑ष्ठ॒द्भुव॑नानि धा॒रय\sn{}। प्रा॒तर॒ग्निं प्रा॒तरिन्द्र हवामहे। प्रा॒तर्मि॒त्रावरु॑णा प्रा॒तर॒श्विना। प्रा॒तर्भगं॑ पू॒षणं॒ ब्रह्म॑ण॒स्पतिम्। प्रा॒तः सोम॑मु॒त रु॒द्र हु॑वेम। प्रा॒त॒र्जितं॒ भग॑मु॒ग्र हु॑वेम। व॒यं पु॒त्रमदि॑ते॒र्यो वि॑ध॒र्ता। आ॒ध्रश्चि॒द्यं मन्य॑मानस्तु॒रश्चि॑त्॥७७॥

%2.8.9.8
राजा॑ चि॒द्यं भगं॑ भ॒क्षीत्याह॑। भग॒ प्रणे॑त॒र्भग॒ सत्य॑राधः। भगे॒मान्धिय॒मुद॑व॒ दद॑न्नः। भग॒ प्र णो॑ जनय॒ गोभि॒रश्वै। भग॒ प्र नृभि॑र्नृ॒वन्त॑ स्याम। उ॒तेदानीं॒ भग॑वन्तः स्याम। उ॒त प्रपि॒त्व उ॒त मध्ये॒ अह्नाम्। उ॒तोदि॑ता मघव॒न्त्सूर्य॑स्य। व॒यन्दे॒वाना सुम॒तौ स्या॑म। भग॑ ए॒व भग॑वा अस्तु देवाः॥७८॥

%2.8.9.9
तेन॑ व॒यं भग॑वन्तः स्याम। तन्त्वा॑ भग॒ सर्व॒ इज्जो॑हवीमि। स नो॑ भग पुरए॒ता भ॑वे॒ह। सम॑ध्व॒रायो॒षसो॑ नमन्त। द॒धि॒क्रावे॑व॒ शुच॑ये प॒दाय॑। अ॒र्वा॒ची॒नं व॑सु॒विदं॒ भग॑न्नः। रथ॑मि॒वाश्वा॑ वा॒जिन॒ आव॑हन्तु। अश्वा॑वती॒र्गोम॑तीर्न उ॒षास॑। वी॒रव॑ती॒ सद॑मुच्छन्तु भ॒द्राः। घृ॒तन्दुहा॑ना वि॒श्वत॒ प्रपी॑नाः। यू॒यं पा॑त स्व॒स्तिभि॒ सदा॑ नः॥७९॥\anuvakamend[वि॒च॒क्ष॒णा वि॑चर्तु॒र शर्म॒न्नधि॑ वि॒सर्ज॑नाय॒ ब्रह्म॒ वनं॒ ब्रह्म॒ स वृ॒क्ष आ॑सीत्तु॒रश्चि॑द्देवा॒ प्रपी॑ना॒ एकं च]
\prashnaend{पीवोन्ना॒न्ते शु॒क्रास॒ सोमो॑ धे॒नुमिन्द्र॒स्तर॑स्वा॒ञ्छुचि॒मा दे॒वो या॑तु॒ सूर्यो॑ दे॒वीम॒हम॑स्मि॒ ता सूर्याचन्द्र॒मसा॒ नव॑॥९॥}{पीवोन्ना॒मग्ने॒ त्वं पा॑रयानाधृ॒ष्यः शुचि॒न्नु वि॒श्रय॑माणो दि॒वो रु॒क्मोऽन्नं॑ प्रा॒णमन्न॒न्ता सूर्याचन्द्र॒मसा॒ नव॑सप्ततिः॥७९॥}{पीवोन्नाय्यूँ॒यं पा॑त स्व॒स्तिभि॒ सदा॑ नः॥}{हरि॑ ओम्॥}{इति श्रीकृष्णयजुर्वेदीयतैत्तिरीयब्राह्मणे द्वितीयाष्टके अष्टमः प्रपाठकः समाप्तः॥}
\clearpage
