\sect{चतुर्थः प्रश्नः}
\setcounter{anuvakam}{0}
\dnsub{तैत्तिरीयब्राह्मणे प्रथमाष्टके चतुर्थः प्रपाठकः}

%1.4.1.1
उ॒भये॒ वा ए॒ते प्र॒जाप॑ते॒रध्य॑सृज्यन्त। दे॒वाश्चासु॑राश्च। तान्न व्य॑जानात्। इ॒मेऽन्य इ॒मेऽन्य इति॑। स दे॒वान॒शून॑करोत्। तान॒भ्य॑षुणोत्। तान्प॒वित्रे॑णापुनात्। तान्प॒रस्तात्प॒वित्र॑स्य॒ व्य॑गृह्णात्। ते ग्रहा॑ अभवन्। तद्ग्रहा॑णां ग्रह॒त्वम्॥१॥

%1.4.1.2
दे॒वता॒ वा ए॒ता यज॑मानस्य गृ॒हे गृ॑ह्यन्ते। यद्ग्रहा। वि॒दुरे॑नं दे॒वाः। यस्यै॒वं वि॒दुष॑ ए॒ते ग्रहा॑ गृ॒ह्यन्ते। ए॒षा वै सोम॒स्याहु॑तिः। यदु॑पा॒शुः। सोमे॑न दे॒वास्त॑र्पया॒णीति॒ खलु॒ वै सोमे॑न यजते। यदु॑पा॒शुं जु॒होति॑। सोमे॑नै॒व तद्दे॒वास्त॑र्पयति। यद्ग्रहां जु॒होति॑॥२॥

%1.4.1.3
दे॒वा ए॒व तद्दे॒वान्ग॑च्छन्ति। यच्च॑म॒सां जु॒होति॑। तेनै॒वानु॑रूपेण॒ यज॑मानः सुव॒र्गं लो॒कमे॑ति। किं न्वे॑तदग्र॑ आसी॒दित्या॑हुः। यत्पात्रा॒णीति॑। इ॒यं वा ए॒तदग्र॑ आसीत्। मृ॒न्मया॑नि॒ वा ए॒तान्या॑सन्। तैर्दे॒वा न व्या॒वृत॑मगच्छन्। त ए॒तानि॑ दारु॒मया॑णि॒ पात्राण्यपश्यन्। तान्य॑कुर्वत॥३॥

%1.4.1.4
तैर्वै ते व्या॒वृत॑मगच्छन्। यद्दा॑रु॒मया॑णि॒ पात्रा॑णि॒ भव॑न्ति। व्या॒वृत॑मे॒व तैर्यज॑मानो गच्छति। यानि॑ दारु॒मया॑णि॒ पात्रा॑णि॒ भव॑न्ति। अ॒मुमे॒व तैर्लो॒कम॒भिज॑यति। यानि॑ मृ॒न्मया॑नि। इ॒ममे॒व तैर्लो॒कम॒भिज॑यति। ब्र॒ह्म॒वा॒दिनो॑ वदन्ति। काश्चत॑स्रः स्था॒लीर्वा॑य॒व्या सोम॒ग्रह॑णी॒रिति॑। दे॒वा वै पृश्ञि॑मदुह्रन्॥४॥

%1.4.1.5
तस्या॑ ए॒ते स्तना॑ आसन्। इ॒यं वै पृश्ञि॑। तामा॑दि॒त्या आ॑दित्यस्था॒ल्या चतु॑ष्पदः प॒शून॑दुह्रन्। यदा॑दित्यस्था॒ली भव॑ति। चतु॑ष्पद ए॒व तया॑ प॒शून् यज॑मान इ॒मां दु॑हे। तामिन्द्र॑ उक्थ्यस्था॒ल्येन्द्रि॒यम॑दुहत्। यदु॑क्थ्यस्था॒ली भव॑ति। इ॒न्द्रि॒यमे॒व तया॒ यज॑मान इ॒मां दु॑हे। तां विश्वे॑ दे॒वा आग्रयणस्था॒ल्योर्ज॑मदुह्रन्। यदाग्रयणस्था॒ली भव॑ति॥५॥

%1.4.1.6
ऊर्ज॑मे॒व तया॒ यज॑मान इ॒मां दु॑हे। तां म॑नु॒ष्या ध्रुवस्था॒ल्याऽऽयु॑रदुह्रन्। यद्ध्रु॑वस्था॒ली भव॑ति। आयु॑रे॒व तया॒ यज॑मान इ॒मां दु॑हे। स्था॒ल्या गृ॒ह्णाति॑। वा॒य॒व्ये॑न जुहोति। तस्मा॑द॒न्येन॒ पात्रे॑ण प॒शून्दु॒हन्ति॑। अ॒न्येन॒ प्रति॑गृह्णन्ति। अथो व्या॒वृत॑मे॒व तद्यज॑मानो गच्छति॥६॥\anuvakamend[ग्र॒ह॒त्वं ग्रहां जु॒होत्य॑कुर्वतादुह्रन्नाग्रयणस्था॒ली भव॑ति॒ नव॑ च]

%1.4.2.1
यु॒व सु॒राम॑मश्विना। नमु॑चावासु॒रे सचा। वि॒पि॒पा॒ना शु॑भस्पती। इन्द्रं॒ कर्म॑ स्वावतम्। पु॒त्रमि॑व पि॒तरा॑व॒श्विनो॒भा। इन्द्राव॑तं॒ कर्म॑णा द॒सना॑भिः। यत्सु॒रामं॒ व्यपि॑ब॒ शची॑भिः। सर॑स्वती त्वा मघवन्नभीष्णात्। अहाव्यग्ने ह॒विरा॒स्ये॑ते। स्रु॒चीव॑ घृ॒तं च॒मू इ॑व॒ सोम॑॥७॥

%1.4.2.2
वा॒ज॒सनि र॒यिम॒स्मे सु॒वीरम्। प्र॒श॒स्तं धे॑हि य॒शसं॑ बृ॒हन्तम्। यस्मि॒न्नश्वा॑स ऋष॒भास॑ उ॒क्षण॑। व॒शा मे॒षा अ॑वसृ॒ष्टास॒ आहु॑ताः। की॒ला॒ल॒पे सोम॑पृष्ठाय वे॒धसे। हृ॒दा म॒तिं ज॑नय॒ चारु॑म॒ग्नये। नाना॒ हि वां दे॒वहि॑त॒ सदो॑ मि॒तम्। मा ससृ॑क्षाथां पर॒मे व्यो॑मन्। सुरा॒ त्वमसि॑ शु॒ष्मिणी॒ सोम॑ ए॒षः। मा मा॑ हिसी॒ स्वां योनि॑मावि॒शन्॥८॥

%1.4.2.3
यदत्र॑ शि॒ष्ट र॒सिन॑ सु॒तस्य॑। यदिन्द्रो॒ अपि॑ब॒च्छची॑भिः। अ॒हं तद॑स्य॒ मन॑सा शि॒वेन॑। सोम॒ राजा॑नमि॒ह भ॑क्षयामि। द्वे स्रु॒ती अ॑शृणवं पितृ॒णाम्। अ॒हं दे॒वाना॑मु॒त मर्त्या॑नाम्। ताभ्या॑मि॒दं विश्वं॒ भुव॑न॒ समे॑ति। अ॒न्त॒रा पूर्व॒मप॑रं च के॒तुम्। यस्ते॑ देव वरुण गाय॒त्रछ॑न्दा॒ पाश॑। तं त॑ ए॒तेनाव॑ यजे॥९॥

%1.4.2.4
यस्ते॑ देव वरुण त्रि॒ष्टुप्छ॑न्दा॒ पाश॑। तं त॑ ए॒तेनाव॑ यजे। यस्ते॑ देव वरुण॒ जग॑तीछन्दा॒ पाश॑। तं त॑ ए॒तेनाव॑ यजे। सोमो॒ वा ए॒तस्य॑ रा॒ज्यमाद॑त्ते। यो राजा॒ सन्रा॒ज्यो वा॒ सोमे॑न॒ यज॑ते। दे॒व॒सु॒वामे॒तानि॑ ह॒वीषि॑ भवन्ति। ए॒ताव॑न्तो॒ वै दे॒वाना स॒वाः। त ए॒वास्मै॑ स॒वान्प्रय॑च्छन्ति। त ए॑नं॒ पुन॑ सुवन्ते रा॒ज्याय॑। दे॒व॒सू राजा॑ भवति॥१०॥\anuvakamend[सोम॑ आवि॒शन् य॑जे रा॒ज्यायैकं॑ च]

%1.4.3.1
उद॑स्थाद्दे॒व्यदि॑तिर्विश्वरू॒पी। आयु॑र्य॒ज्ञप॑तावधात्। इन्द्रा॑य कृण्व॒ती भा॒गम्। मि॒त्राय॒ वरु॑णाय च। इ॒यं वा अ॑ग्निहो॒त्री। इ॒यं वा ए॒तस्य॒ निषी॑दति। यस्याग्निहो॒त्री नि॒षीद॑ति। तामुत्था॑पयेत्। उद॑स्थाद्दे॒व्यदि॑ति॒रिति॑। इ॒यं वै दे॒व्यदि॑तिः॥११॥

%1.4.3.2
इ॒मामे॒वास्मा॒ उत्था॑पयति। आयु॑र्य॒ज्ञप॑तावधा॒दित्या॑ह। आयु॑रे॒वास्मि॑न्दधाति। इन्द्रा॑य कृण्व॒ती भा॒गं मि॒त्राय॒ वरु॑णाय॒ चेत्या॑ह। य॒था॒य॒जुरे॒वैतत्। अव॑र्तिं॒ वा ए॒षैतस्य॑ पा॒प्मानं॑ प्रति॒ख्याय॒ निषी॑दति। यस्याग्निहो॒त्र्युप॑सृष्टा नि॒षीद॑ति। तां दु॒ग्ध्वा ब्राह्म॒णाय॑ दद्यात्। यस्यान्नं॒ नाद्यात्। अव॑र्तिमे॒वास्मि॑न्पा॒प्मानं॒ प्रति॑मुञ्चति॥१२॥

%1.4.3.3
दु॒ग्ध्वा द॑दाति। न ह्यदृ॑ष्टा॒ दक्षि॑णा दी॒यते। पृ॒थि॒वीं वा ए॒तस्य॒ पय॒ प्रवि॑शति। यस्याग्निहो॒त्रं दु॒ह्यमा॑न॒ स्कन्द॑ति। यद॒द्य दु॒ग्धं पृ॑थि॒वीमस॑क्त। यदोष॑धीर॒प्यस॑र॒द्यदाप॑। पयो॑ गृ॒हेषु॒ पयो॑ अघ्नि॒यासु॑। पयो॑ व॒त्सेषु॒ पयो॑ अस्तु॒ तन्मयीत्या॑ह। पय॑ ए॒वात्मन्गृ॒हेषु॑ प॒शुषु॑ धत्ते। अ॒प उप॑सृजति॥१३॥

%1.4.3.4
अ॒द्भिरे॒वैन॑दाप्नोति। यो वै य॒ज्ञस्यार्ते॒ नानार्त स सृ॒जति॑। उ॒भे वै ते तर्ह्यार्च्छ॑तः। आर्च्छ॑ति॒ खलु॒ वा ए॒तद॑ग्निहो॒त्रम्। यद्दु॒ह्यमा॑न॒ स्कन्द॑ति। यद॑भिदु॒ह्यात्। आर्ते॒ नानार्तं य॒ज्ञस्य॒ ससृ॑जेत्। तदे॒व या॒दृक्की॒दृक्च॑ होत॒व्यम्। अथा॒न्यां दु॒ग्ध्वा पुन॑र्\mbox{}होत॒व्यम्। अनार्तेनै॒वार्तं॑ य॒ज्ञस्य॒ निष्क॑रोति॥१४॥

%1.4.3.5
यद्युद्द्रु॑तस्य॒ स्कन्देत्। यत्ततोऽहु॑त्वा॒ पुन॑रे॒यात्। य॒ज्ञं विच्छि॑न्द्यात्। यत्र॒ स्कन्देत्। तन्नि॒षद्य॒ पुन॑र्गृह्णीयात्। यत्रै॒व स्कन्द॑ति। तत॑ ए॒वैन॒त्पुन॑र्गृह्णाति। तदे॒व या॒दृक्की॒दृक्च॑ होत॒व्यम्। अथा॒न्यां दु॒ग्ध्वा पुन॑र्\mbox{}होत॒व्यम्। अनार्तेनै॒वार्तं॑ य॒ज्ञस्य॒ निष्क॑रोति॥१५॥

%1.4.3.6
वि वा ए॒तस्य॑ य॒ज्ञश्छि॑द्यते। यस्याग्निहो॒त्रे॑ऽधिश्रि॑ते॒ श्वाऽन्त॒रा धाव॑ति। रु॒द्रः खलु॒ वा ए॒षः। यद॒ग्निः। यद्गाम॑न्वत्या व॒र्तयेत्। रु॒द्राय॑ प॒शूनपि॑ दध्यात्। अ॒प॒शुर्यज॑मानः स्यात्। यद॒पोऽन्वतिषि॒ञ्चेत्। अ॒ना॒द्यम॒ग्नेराप॑। अ॒ना॒द्यमाभ्या॒मपि॑ दध्यात्। गार्\mbox{}ह॑पत्या॒द्भस्मा॒दाय॑। इ॒दं विष्णु॒र्विच॑क्रम॒ इति॑ वैष्ण॒व्यर्चाऽऽह॑व॒नीयाद्ध्व॒सय॒न्नुद्र॑वेत्। य॒ज्ञो वै विष्णु॑। य॒ज्ञेनै॒व य॒ज्ञ संत॑नोति। भस्म॑ना प॒दमपि॑ वपति॒ शान्त्यै॥१६॥\anuvakamend[वै दे॒व्यदि॑तिर्मुञ्चति सृजति करोति करोत्याभ्या॒मपि॑ दध्या॒त् पञ्च॑ च]

%1.4.4.1
नि वा ए॒तस्या॑हव॒नीयो॒ गार्\mbox{}ह॑पत्यं कामयते। निगार्\mbox{}ह॑पत्य आहव॒नीयम्। यस्या॒ग्निमनु॑द्धृत॒ सूर्यो॒ऽभि नि॒म्रोच॑ति। द॒र्भेण॒ हिर॑ण्यं प्र॒बद्ध्य॑ पु॒रस्ताद्धरेत्। अथा॒ग्निम्। अथाग्निहो॒त्रम्। यद्धिर॑ण्यं पु॒रस्ता॒द्धर॑ति। ज्योति॒र्वै हिर॑ण्यम्। ज्योति॑रे॒वैनं॒ पश्य॒न्नुद्ध॑रति। यद॒ग्निं पूर्व॒ हर॒त्यथाग्निहो॒त्रम्॥१७॥

%1.4.4.2
भा॒ग॒धेये॑नै॒वैनं॒ प्रण॑यति। ब्रा॒ह्म॒ण आ॑र्\mbox{}षे॒य उद्ध॑रेत्। ब्रा॒ह्म॒णो वै सर्वा॑ दे॒वता। सर्वा॑भिरे॒वैनं॑ दे॒वता॑भि॒रुद्ध॑रति। अ॒ग्नि॒हो॒त्रमु॑प॒साद्यातमि॑तोरासीत। व्र॒तमे॒व ह॒तमनु॑ म्रियते। अन्तं॒ वा ए॒ष आ॒त्मनो॑ गच्छति। यस्ताम्य॑ति। अन्त॑मे॒ष य॒ज्ञस्य॑ गच्छति। यस्या॒ग्निमनु॑द्धृ॒त सूर्यो॒ऽभि नि॒म्रोच॑ति॥१८॥

%1.4.4.3
पुन॑ स॒मन्य॑ जुहोति। अन्ते॑नै॒वान्तं॑ य॒ज्ञस्य॒ निष्क॑रोति। वरु॑णो॒ वा ए॒तस्य॑ य॒ज्ञं गृ॑ह्णाति। यस्या॒ग्निमनु॑द्धृत॒ सूर्यो॒ऽभि नि॒म्रोच॑ति। वा॒रु॒णं च॒रुं निर्व॑पेत्। तेनै॒व य॒ज्ञं निष्क्री॑णीते। नि वा ए॒तस्या॑हव॒नीयो॒ गार्\mbox{}ह॑पत्यं कामयते। नि गार्\mbox{}ह॑पत्य आहव॒नीयम्। यस्या॒ग्निमनु॑द्धृत॒ सूर्यो॒ऽभ्यु॑देति॑। च॒तु॒र्गृ॒ही॒तमाज्यं॑ पु॒रस्ताद्धरेत्॥१९॥

%1.4.4.4
अथा॒ग्निम्। अथाग्निहो॒त्रम्। यदाज्यं॑ पु॒रस्ता॒द्धर॑ति। ए॒तद्वा अ॒ग्नेः प्रि॒यं धाम॑। यदाज्यम्। प्रि॒येणै॒वैनं॒ धाम्ना॒ सम॑र्धयति। यद॒ग्निं पूर्व॒ हर॒त्यथाग्निहो॒त्रम्। भा॒ग॒धेये॑नै॒वैनं॒ प्रण॑यति। ब्रा॒ह्म॒ण आ॑र्\mbox{}षे॒य उद्ध॑रेत्। ब्रा॒ह्म॒णो वै सर्वा॑ दे॒वता॥२०॥

%1.4.4.5
सर्वा॑भिरे॒वैनं॑ दे॒वता॑भि॒रुद्ध॑रति। परा॑ची॒ वा ए॒तस्मै व्यु॒च्छन्ती॒ व्यु॑च्छति। यस्या॒ग्निमनु॑द्धृत॒ सूर्यो॒ऽभ्यु॑देति॑। उ॒षाः के॒तुना॑ जुषताम्। य॒ज्ञं दे॒वेभि॑रिन्वि॒तम्। दे॒वेभ्यो॒ मधु॑मत्तम॒ स्वाहेति॑ प्र॒त्यङ्नि॒षद्याज्ये॑न जुहुयात्। प्र॒तीची॑मे॒वास्मै॒ विवा॑सयति। अ॒ग्नि॒हो॒त्रमु॑प॒साद्यातमि॑तोरासीत। व्र॒तमे॒व ह॒तमनु॑ म्रियते। अन्तं॒ वा ए॒ष आ॒त्मनो॑ गच्छति॥२१॥

%1.4.4.6
यस्ताम्य॑ति। अन्त॑मे॒ष य॒ज्ञस्य॑ गच्छति। यस्या॒ग्निमनु॑द्धृत॒ सूर्यो॒ऽभ्यु॑देति॑। पुन॑ स॒मन्य॑ जुहोति। अन्ते॑नै॒वान्तं॑ य॒ज्ञस्य॒ निष्क॑रोति। मि॒त्रो वा ए॒तस्य॑ य॒ज्ञं गृ॑ह्णाति। यस्या॒ग्निमनु॑द्धृत॒ सूर्यो॒ऽभ्यु॑देति॑। मै॒त्रं च॒रुं निर्व॑पेत्। तेनै॒व य॒ज्ञं निष्क्री॑णीते। यस्या॑हव॒नीयेऽ नु॑द्वाते॒ गार्\mbox{}ह॑पत्य उ॒द्वायेत् ॥२२॥

%1.4.4.7
यदा॑हव॒नीय॒मनु॑द्वाप्य॒ गार्\mbox{}ह॑पत्यं॒ मन्थेत्। विच्छि॑न्द्यात्। भ्रातृ॑व्यमस्मै जनयेत्। यद्वै य॒ज्ञस्य॑ वास्त॒व्यं॑ क्रि॒यते। तदनु॑ रु॒द्रोऽव॑चरति। यत्पूर्व॑मन्वव॒स्येत्। वा॒स्त॒व्य॑म॒ग्निमुपा॑सीत। रु॒द्रोऽस्य प॒शून्घातु॑कः स्यात्। आ॒ह॒व॒नीय॑मु॒द्वाप्य॑। गार्\mbox{}ह॑पत्यं मन्थेत्॥२३॥

%1.4.4.8
इ॒तः प्र॑थ॒मं ज॑ज्ञे अ॒ग्निः। स्वाद्योने॒रधि॑ जा॒तवे॑दाः। स गा॑यत्रि॒या त्रि॒ष्टुभा॒ जग॑त्या। दे॒वेभ्यो॑ ह॒व्यं व॑हतु प्रजा॒नन्निति॑। छन्दो॑भिरे॒वैन॒ स्वाद्योने॒ प्रज॑नयति। गार्\mbox{}ह॑पत्यं मन्थति। गार्\mbox{}ह॑पत्यं॒ वा अन्वाहि॑ताग्नेः प॒शव॒ उप॑ तिष्ठन्ते। स यदु॒द्वाय॑ति। तदनु॑ प॒शवोऽप॑ क्रामन्ति। इ॒षे र॒य्यै र॑मस्व॥२४॥

%1.4.4.9
सह॑से द्यु॒म्नाय॑। ऊ॒र्जेऽपत्या॒येत्या॑ह। प॒शवो॒ वै र॒यिः। प॒शूने॒वास्मै॑ रमयति। सा॒र॒स्व॒तौ त्वोत्सौ॒ समि॑न्धाता॒मित्या॑ह। ऋ॒ख्सा॒मे वै सा॑रस्व॒तावुत्सौ। ऋ॒ख्सा॒माभ्या॑मे॒वैन॒ समि॑न्धे। स॒म्राड॑सि वि॒राड॒सीत्या॑ह। र॒थ॒न्त॒रं वै स॒म्राट्। बृ॒हद्वि॒राट्। 2५॥

%1.4.4.10
ताभ्या॑मे॒वैन॒ समि॑न्धे। वज्रो॒ वै च॒क्रम्। वज्रो॒ वा ए॒तस्य॑ य॒ज्ञं विच्छि॑नत्ति। यस्यानो॑ वा॒ रथो॑ वाऽन्त॒राऽग्नी याति॑। आ॒ह॒व॒नीय॑मु॒द्वाप्य॑। गार्\mbox{}ह॑पत्या॒दुद्ध॑रेत्। यद॑ग्ने॒ पूर्वं॒ प्रभृ॑तं प॒द हि ते। सूर्य॑स्य र॒श्मीनन्वा॑त॒तान॑। तत्र॑ रयि॒ष्ठामनु॒ सं भ॑रै॒तम्। सं न॑ सृज सुम॒त्या वाज॑व॒त्येति॑॥२६॥

%1.4.4.11
पूर्वे॑णै॒वास्य॑ य॒ज्ञेन॑ य॒ज्ञमनु॒ संत॑नोति। त्वम॑ग्ने स॒प्रथा॑ अ॒सीत्या॑ह। अ॒ग्निः सर्वा॑ दे॒वता। दे॒वता॑भिरे॒व य॒ज्ञ संत॑नोति। अ॒ग्नये॑ पथि॒कृते॑ पुरो॒डाश॑म॒ष्टाक॑पालं॒ निर्व॑पेत्। अ॒ग्निमे॒व प॑थि॒कृत॒ स्वेन॑ भाग॒धेये॒नोप॑धावति। स ए॒वैनं॑ य॒ज्ञियं॒ पन्था॒मपि॑ नयति। अ॒न॒ड्वान्दक्षि॑णा। व॒ही ह्ये॑ष समृ॑द्ध्यै॥२७॥\anuvakamend[हर॒त्यथाग्निहो॒त्रं नि॒म्रोच॑ति हरेद्दे॒वता॑ गच्छत्यु॒द्वायेन्मन्थेद्रमस्व बृ॒हद्वि॒राडिति॒ नव॑ च (नि वै पूर्वं॒ त्रीणि॑ नि॒म्रोच॑ति द॒र्भेण॒ यद्धिर॑ण्यमग्निहो॒त्रं पुन॒र्वरु॑णो वारु॒णं नि वा ए॒तस्या॒भ्यु॑देति॑ चतुर्गृही॒तमाज्यं॒ यदाज्यं॒ पराच्यु॒षाः पुन॑र्मि॒त्रो मै॒त्रं यस्या॑हव॒नीयेऽनु॑द्वाते॒ गार्\mbox{}ह॑पत्यो॒ यद्वै म॑न्थे॒दुद्ध॑रेत् ॥ )]

%1.4.5.1
यस्य॑ प्रातः सव॒ने सोमो॑ऽति॒रिच्य॑ते। माध्य॑न्दिन॒ सव॑नं का॒मय॑मानो॒ऽभ्यति॑रिच्यते। गौर्ध॑यति म॒रुता॒मिति॒ धय॑द्वतीषु कुर्वन्ति। हि॒नस्ति॒ वै स॒न्ध्यधी॑तम्। स॒न्धीव॒ खलु॒ वा ए॒तत्। यत्सव॑नस्याति॒रिच्य॑ते। यद्धय॑द्वतीषु कु॒र्वन्ति॑। स॒न्धेः शान्त्यै। गा॒य॒त्र साम॑ भवति पञ्चद॒शः स्तोम॑। तेनै॒व प्रा॑तः सव॒नान्नय॑न्ति॥२८॥

%1.4.5.2
म॒रुत्व॑तीषु कुर्वन्ति। तेनै॒व माध्य॑न्दिना॒त्सव॑ना॒न्नय॑न्ति। होतु॑श्चम॒समनून्न॑यन्ते। होताऽनु॑ शसति। म॒ध्य॒त ए॒व य॒ज्ञ स॒माद॑धाति। यस्य॒ माध्य॑न्दिने॒ सव॑ने॒ सोमो॑ऽति॒रिच्य॑ते। आ॒दि॒त्यं तृ॑तीयसव॒नं का॒मय॑मानो॒ऽभ्यति॑रिच्यते। गौ॒रि॒वी॒त साम॑ भवति। अति॑रिक्तं॒ वै गौ॑रिवी॒तम्। अति॑रिक्तं॒ यत्सव॑नस्याति॒रिच्य॑ते॥२९॥

%1.4.5.3
अति॑रिक्तस्य॒ शान्त्यै। बण्म॒हा अ॑सि सू॒र्येति॑ कुर्वन्ति। यस्यै॒वादि॒त्यस्य॒ सव॑नस्य॒ कामे॑नाति॒रिच्य॑ते। तेनै॒वैनं॒ कामे॑न॒ सम॑र्धयन्ति। गौ॒रि॒वी॒त साम॑ भवति। तेनै॒व माध्य॑न्दिना॒त्सव॑ना॒न्नय॑न्ति। स॒प्त॒द॒शः स्तोम॑। तेनै॒व तृ॑तीयसव॒नान्नय॑न्ति। होतु॑श्चम॒समनून्न॑यन्ते। होताऽनु॑ शसति॥३०॥

%1.4.5.4
म॒ध्य॒त ए॒व य॒ज्ञ स॒माद॑धाति। यस्य॑ तृतीयसव॒ने सोमो॑ऽति॒रिच्ये॑त। उ॒क्थ्यं॑ कुर्वीत। यस्यो॒क्थ्ये॑ऽति॒रिच्ये॑त। अ॒ति॒रा॒त्रं कु॑र्वीत। यस्या॑तिरा॒त्रे॑ऽति॒रिच्य॑ते। तत्त्वै दु॑ष्प्रज्ञा॒नम्। यज॑मानं॒ वा ए॒तत्प॒शव॑ आ॒साह्य॑यन्ति। बृ॒हत्साम॑ भवति। बृ॒हद्वा इ॒माल्लोँ॒कान्दा॑धार। बार्\mbox{}ह॑ताः प॒शव॑। बृ॒ह॒तैवास्मै॑ प॒शून्दा॑धार। शि॒पि॒वि॒ष्टव॑तीषु कुर्वन्ति। शि॒पि॒वि॒ष्टो वै दे॒वानां पु॒ष्टम्। पुष्ट्यै॒वैन॒ सम॑र्धयन्ति। होतु॑श्चम॒समनून्न॑यन्ते। होताऽनु॑शसति। म॒ध्य॒त ए॒व य॒ज्ञ स॒माद॑धाति॥३१॥\anuvakamend[य॒न्ति॒ सव॑नस्याति॒रिच्य॑ते शसति दाधारा॒ष्टौ च॑]

%1.4.6.1
एकै॑को॒ वै ज॒नता॑या॒मिन्द्र॑। एकं॒ वा ए॒ताविन्द्र॑म॒भि ससु॑नुतः। यौ द्वौ स सुनु॒तः। प्र॒जाप॑ति॒र्वा ए॒ष विता॑यते। यद्य॒ज्ञः। तस्य॒ ग्रावा॑णो॒ दन्ता। अ॒न्य॒त॒रं वा ए॒ते ससुन्व॒तोर्निर्ब॑प्सति। पूर्वे॑णोप॒सृत्या॑ दे॒वता॒ इत्या॑हुः। पू॒र्वो॒प॒सृ॒तस्य॒ वै श्रेयान्भवति। एति॑व॒न्त्याज्या॑नि भवन्त्य॒भिजि॑त्यै॥३२॥

%1.4.6.2
म॒रुत्व॑तीः प्रति॒पद॑। म॒रुतो॒ वै दे॒वाना॒मप॑राजितमा॒यत॑नम्। दे॒वाना॑मे॒वाप॑राजित आ॒यत॑ने यतते। उ॒भे बृ॑हद्रथन्त॒रे भ॑वतः। इ॒यं वाव र॑थन्त॒रम्। अ॒सौ बृ॒हत्। आ॒भ्यामे॒वैन॑म॒न्तरे॑ति। वा॒चश्च॒ मन॑सश्च। प्रा॒णाच्चा॑पा॒नाच्च॑। दि॒वश्च॑ पृथि॒व्याश्च॑॥३३॥

%1.4.6.3
सर्व॑स्माद्वि॒त्ताद्वेद्यात्। अ॒भि॒व॒र्तो ब्र॑ह्मसा॒मं भ॑वति। सु॒व॒र्गस्य॑ लो॒कस्या॒भिवृ॑त्त्यै। अ॒भि॒जिद्भ॑वति। सु॒व॒र्गस्य॑ लो॒कस्या॒भिजि॑त्यै। वि॒श्व॒जिद्भ॑वति। विश्व॑स्य॒ जित्यै। यस्य॒ भूयासो यज्ञक्र॒तव॒ इत्या॑हुः। स दे॒वता॑ वृङ्क्त॒ इति॑। यद्य॑ग्निष्टो॒मः सोम॑ प॒रस्ता॒त्स्यात्॥३४॥

%1.4.6.4
उ॒क्थ्यं॑ कुर्वीत। यद्यु॒क्थ॑ स्यात्। अ॒ति॒रा॒त्रं कु॑र्वीत। य॒ज्ञ॒क्र॒तुभि॑रे॒वास्य॑ दे॒वता॑ वृङ्क्ते। यो वै छन्दो॑भिरभि॒भव॑ति। स ससुन्व॒तोर॒भिभ॑वति। सं॒वे॒शाय॑ त्वोपवे॒शाय॒ त्वेत्या॑ह। छन्दासि॒ वै सं॑वे॒श उ॑पवे॒शः। छन्दो॑भिरे॒वास्य॒ छन्दास्य॒भिभ॑वति। इ॒ष्टर्गो॒ वा ऋ॒त्विजा॑मध्व॒र्युः॥३५॥

%1.4.6.5
इ॒ष्टर्ग॒ खलु॒ वै पूर्वो॒ऽर्ष्टुः क्षी॑यते। प्राणा॑पानौ मृ॒त्योर्मा॑ पात॒मित्या॑ह। प्रा॒णा॒पा॒नयो॑रे॒व श्र॑यते। प्राणा॑पानौ॒ मा मा॑हासिष्ट॒मित्या॑ह। नैनं॑ पु॒राऽऽयु॑षः प्राणापा॒नौ ज॑हितः। आर्तिं॒ वा ए॒ते निय॑न्ति। येषां दीक्षि॒तानां प्र॒मीय॑ते। तं यद॑व॒वर्जे॑युः। क्रू॒र॒कृता॑मिवैषां लो॒कः स्यात्। आह॑र द॒हेति॑ ब्रूयात्॥३६॥

%1.4.6.6
तं द॑क्षिण॒तो वेद्यै॑ नि॒धाय॑। स॒र्प॒रा॒ज्ञिया॑ ऋ॒ग्भिः स्तु॑युः। इ॒यं वै सर्प॑तो॒ राज्ञी। अ॒स्या ए॒वैनं॒ परि॑ददति। व्यृ॑द्धं॒ तदित्या॑हुः। यत्स्तु॒तमन॑नुशस्त॒मिति॑। होता प्रथ॒मः प्रा॑चीनावी॒ती मार्जा॒लीयं॒ परी॑यात्। या॒मीर॑नुब्रु॒वन्। स॒र्प॒रा॒ज्ञीनां कीर्तयेत्। उ॒भयो॑रे॒वैनं॑ लो॒कयो॒ परि॑ददति॥३७॥

%1.4.6.7
अथो॑ धु॒वन्त्ये॒वैनम्। अथो॒ न्ये॑वास्मै ह्नुवते। त्रिः परि॑यन्ति। त्रय॑ इ॒मे लो॒काः। ए॒भ्य ए॒वैनं॑ लो॒केभ्यो॑ धुवते। त्रिः पुन॒ परि॑यन्ति। षट्त्संप॑द्यन्ते। षड्वा ऋ॒तव॑। ऋ॒तुभि॑रे॒वैनं॑ धुवते। अग्न॒ आयूषि पवस॒ इति॑ प्रति॒पदं॑ कुर्वीरन्। र॒थ॒न्त॒रसा॑मैषा॒ सोम॑ स्यात्। आयु॑रे॒वात्मन्द॑धते। अथो॑ पा॒प्मान॑मे॒व वि॒जह॑तो यन्ति॥३८॥\anuvakamend[अ॒भिजि॑त्यै पृथि॒व्याश्च॒ स्याद॑ध्व॒र्युर्ब्रू॑याल्लो॒कयो॒ परि॑ददति कुर्वीर॒स्त्रीणि॑ च]

%1.4.7.1
अ॒सु॒र्यं॑ वा ए॒तस्मा॒द्वर्णं॑ कृ॒त्वा। प॒शवो॑ वी॒र्य॑मप॑ क्रामन्ति। यस्य॒ यूपो॑ वि॒रोह॑ति। त्वा॒ष्ट्रं ब॑हुरू॒पमाल॑भेत। त्वष्टा॒ वै रू॒पाणा॑मीशे। य ए॒व रू॒पाणा॒मीशे। सोऽस्मिन्प॒शून् वी॒र्यं॑ यच्छति। नास्मात्प॒शवो॑ वी॒र्य॑मप॑ क्रामन्ति। आर्तिं॒ वा ए॒ते निय॑न्ति। येषां दीक्षि॒ताना॑म॒ग्निरु॒द्वाय॑ति॥३९॥

%1.4.7.2
यदा॑हव॒नीय॑ उ॒द्वायेत्। यत्तं मन्थेत्। विच्छि॑न्द्यात्। भ्रातृ॑व्यमस्मै जनयेत्। यदा॑हव॒नीय॑ उ॒द्वायेत्। आग्नीद्ध्रा॒दुद्ध॑रेत्। यदाग्नीद्ध्र उ॒द्वायेत्। गार्\mbox{}ह॑पत्या॒दुद्ध॑रेत्। यद्गार्\mbox{}ह॑पत्य उ॒द्वायेत्। अत॑ ए॒व पुन॑र्मन्थेत्॥४०॥

%1.4.7.3
अत्र॒ वाव स निल॑यते। यत्र॒ खलु॒ वै निली॑नमुत्त॒मं पश्य॑न्ति। तदे॑नमिच्छन्ति। यस्मा॒द्दारो॑रु॒द्वायेत्। तस्या॒रणी॑ कुर्यात्। क्रु॒मु॒कमपि॑ कुर्यात्। ए॒षा वा अ॒ग्नेः प्रि॒या त॒नूः। यत्क्रु॑मु॒कः। प्रि॒ययै॒वैन॑न्त॒नुवा॒ सम॑र्धयति। गार्\mbox{}ह॑पत्यं मन्थति॥४१॥

%1.4.7.4
गार्\mbox{}ह॑पत्यो॒ वा अ॒ग्नेर्योनि॑। स्वादे॒वैनं॒ योनेर्जनयति। नास्मै॒ भ्रातृ॑व्यञ्जनयति। यस्य॒ सोम॑ उप॒दस्येत्। सु॒वर्ण॒ हिर॑ण्यन्द्वे॒धा वि॒च्छिद्य॑। ऋ॒जी॒षेऽन्यदा॑धूनु॒यात्। जु॒हु॒याद॒न्यत्। सोम॑मे॒वाभि॑षु॒णोति॑। सोमं॑ जुहोति। सोम॑स्य॒ वा अ॑भिषू॒यमा॑णस्य प्रि॒या त॒नूरुद॑क्रामत्॥४२॥

%1.4.7.5
तत्सु॒वर्ण॒ हिर॑ण्यमभवत्। यत्सु॒वर्ण॒ हिर॑ण्यं कु॒र्वन्ति॑। प्रि॒ययै॒वैन॑न्त॒नुवा॒ सम॑र्धयन्ति। यस्याक्री॑त॒ सोम॑मप॒हरे॑युः। क्री॒णी॒यादे॒व। सैव तत॒ प्राय॑श्चित्तिः। यस्य॑ क्री॒तम॑प॒हरे॑युः। आ॒दा॒राश्च॑ फाल्गु॒नानि॑ चा॒भिषु॑णुयात्। गा॒य॒त्री य सोम॒माह॑रत्। तस्य॒ योऽशुः प॒राऽप॑तत्॥४३॥

%1.4.7.6
त आ॑दा॒रा अ॑भवन्। इन्द्रो॑ वृ॒त्रम॑हन्। तस्य॑ व॒ल्कः परा॑ऽपतत्। तानि॑ फाल्गु॒नान्य॑भवन्। प॒शवो॒ वै फाल्गु॒नानि॑। प॒शव॒ सोमो॒ राजा। यदा॑दा॒राश्च॑ फाल्गु॒नानि॑ चाभिषु॒णोति॑। सोम॑मे॒व राजा॑नम॒भिषु॑णोति। शृ॒तेन॑ प्रातः सव॒ने श्री॑णीयात्। द॒ध्ना म॒ध्यन्दि॑ने॥४४॥

%1.4.7.7
नी॒त॒मि॒श्रेण॑ तृतीयसव॒ने। अ॒ग्नि॒ष्टो॒मः सोम॑ स्याद्रथन्त॒रसा॑मा। य ए॒वर्त्विजो॑ वृ॒ताः स्युः। त ए॑नं याजयेयुः। एका॒ङ्गान्दक्षि॑णान्दद्या॒त्तेभ्य॑ ए॒व। पुन॒ सोमं॑ क्रीणीयात्। य॒ज्ञेनै॒व तद्य॒ज्ञमि॑च्छति। सैव तत॒ प्राय॑श्चित्तिः। सर्वाभ्यो॒ वा ए॒ष दे॒वताभ्य॒ सर्वेभ्यः पृ॒ष्ठेभ्य॑ आ॒त्मान॒मागु॑रते। यः स॒त्राया॑गु॒रते। ए॒तावा॒न्खलु॒ वै पुरु॑षः। याव॑दस्य वि॒त्तम्। स॒र्व॒वे॒द॒सेन॑ यजेत। सर्व॑पृष्ठोऽस्य॒ सोम॑ स्यात्। सर्वाभ्य ए॒व दे॒वताभ्य॒ सर्वेभ्यः पृ॒ष्ठेभ्य॑ आ॒त्मानं॒ निष्क्री॑णीते॥४५॥\anuvakamend[उ॒द्वाय॑ति मन्थेन्मन्थत्यक्रामत्प॒राऽप॑तन्म॒ध्यन्दि॑न आगु॒रते॒ पञ्च॑ च]

%1.4.8.1
पव॑मान॒ सुव॒र्जन॑। प॒वित्रे॑ण॒ विच॑र्\mbox{}षणिः। यः पोता॒ स पु॑नातु मा। पु॒नन्तु॑ मा देवज॒नाः। पु॒नन्तु॒ मन॑वो धि॒या। पु॒नन्तु॒ विश्व॑ आ॒यव॑। जात॑वेदः प॒वित्र॑वत्। प॒वित्रे॑ण पुनाहि मा। शू॒क्रेण॑ देव॒ दीद्य॑त्। अग्ने॒ क्रत्वा॒ क्रतू॒ रनु॑॥४६॥

%1.4.8.2
यत्ते॑ प॒वित्र॑म॒र्चिषि॑। अग्ने॒ वित॑तमन्त॒रा। ब्रह्म॒ तेन॑ पुनीमहे। उ॒भाभ्यान्देव सवितः। प॒वित्रे॑ण स॒वेन॑ च। इ॒दं ब्रह्म॑ पुनीमहे। वै॒श्व॒दे॒वी पु॑न॒ती दे॒व्यागात्। यस्यै॑ ब॒ह्वीस्त॒नुवो॑ वी॒तपृ॑ष्ठाः। तया॒ मद॑न्तः सध॒माद्ये॑षु। व॒य स्या॑म॒ पत॑यो रयी॒णाम्॥४७॥

%1.4.8.3
वै॒श्वा॒न॒रो र॒श्मिभि॑र्मा पुनातु। वात॑ प्रा॒णेने॑षि॒रो म॑यो॒भूः। द्यावा॑पृथि॒वी पय॑सा॒ पयो॑भिः। ऋ॒ताव॑री य॒ज्ञिये॑ मा पुनीताम्। बृ॒हद्भि॑ सवित॒स्तृभि॑। वर्\mbox{}षि॑ष्ठैर्देव॒ मन्म॑भिः। अग्ने॒ दक्षै पुनाहि मा। येन॑ दे॒वा अपु॑नत। येनापो॑ दि॒व्यङ्कश॑। तेन॑ दि॒व्येन॒ ब्रह्म॑णा॥४८॥

%1.4.8.4
इ॒दं ब्रह्म॑ पुनीमहे। यः पा॑वमा॒नीर॒ध्येति॑। ऋषि॑भि॒ सम्भृ॑त॒ रसम्। सर्व॒ स पू॒तम॑श्ञाति। स्व॒दि॒तं मा॑त॒रिश्व॑ना। पा॒व॒मा॒नीर्यो अ॒ध्येति॑। ऋषि॑भि॒ सम्भृ॑त॒ रसम्। तस्मै॒ सर॑स्वती दुहे। क्षी॒र स॒र्पिर्मधू॑द॒कम्। पा॒व॒मा॒नीः स्व॒स्त्यय॑नीः॥४९॥

%1.4.8.5
सु॒दुघा॒ हि पय॑स्वतीः। ऋषि॑भि॒ सम्भृ॑तो॒ रस॑। ब्रा॒ह्म॒णेष्व॒मृत हि॒तम्। पा॒व॒मानीर्दि॑शन्तु नः। इ॒मं लो॒कमथो॑ अ॒मुम्। कामा॒न्त्सम॑र्धयन्तु नः। दे॒वीर्दे॒वैः स॒माभृ॑ताः। पा॒व॒मा॒नीः स्व॒स्त्यय॑नीः। सु॒दुघा॒ हि घृ॑त॒श्चुत॑। ऋषि॑भि॒ सम्भृ॑तो॒ रस॑॥५०॥

%1.4.8.6
ब्रा॒ह्म॒णेष्व॒मृत हि॒तम्। येन॑ दे॒वाः प॒वित्रे॑ण। आ॒त्मानं॑ पु॒नते॒ सदा। तेन॑ स॒हस्र॑धारेण। पा॒व॒मा॒न्यः पु॑नन्तु मा। प्रा॒जा॒प॒त्यं प॒वित्रम्। श॒तोद्या॑म हिर॒ण्मयम्। तेन॑ ब्रह्म॒विदो॑ व॒यम्। पू॒तं ब्रह्म॑ पुनीमहे। इन्द्र॑ सुनी॒ती स॒ह मा॑ पुनातु। सोम॑ स्व॒स्त्या वरु॑णः स॒मीच्या। य॒मो राजा प्रमृ॒णाभि॑ पुनातु मा। जा॒तवे॑दा मो॒र्जय॑न्त्या पुनातु॥५१॥\anuvakamend[अनु॑ रयी॒णां ब्रह्म॑णा स्व॒स्त्यय॑नीः सु॒दुघा॒ हि घृ॑त॒श्चुत॒ ऋषि॑भि॒ सम्भृ॑तो॒ रस॑ पुनातु॒ त्रीणि॑ च]

%1.4.9.1
प्र॒जा वै स॒त्रमा॑सत॒ तप॒स्तप्य॑माना॒ अजु॑ह्वतीः। दे॒वा अ॑पश्यञ्चम॒सङ्घृ॒तस्य॑ पू॒र्ण स्व॒धाम्। तमुपोद॑तिष्ठ॒न्तम॑जुहवुः। तेनार्धमा॒स ऊर्ज॒मवा॑रुन्धत। तस्मा॑दर्धमा॒से दे॒वा इ॑ज्यन्ते। पि॒तरो॑ऽपश्यञ्चम॒सङ्घृ॒तस्य॑ पू॒र्ण स्व॒धाम्। तमुपोद॑तिष्ठ॒न्तम॑जुहवुः। तेन॑ मा॒स्यूर्ज॒मवा॑रुन्धत। तस्मान्मा॒सि पि॒तृभ्य॑ क्रियते। म॒नु॒ष्या॑ अपश्यञ्चम॒सङ्घृ॒तस्य॑ पू॒र्ण स्व॒धाम्॥५२॥

%1.4.9.2
तमुपोद॑तिष्ठ॒न्तम॑जुहवुः। तेन॑ द्व॒यीमूर्ज॒मवा॑रुन्धत। तस्मा॒द्द्विरह्नो॑ मनु॒ष्येभ्य॒ उप॑ह्रियते। प्रा॒तश्च॑ सा॒यं च॑। प॒शवो॑ऽपश्यञ्चम॒सङ्घृ॒तस्य॑ पू॒र्ण स्व॒धाम्। तमुपोद॑तिष्ठ॒न्त\-म॑जुहवुः। तेन॑ त्र॒यीमूर्ज॒मवा॑रुन्धत। तस्मा॒त्रिरह्न॑ प॒शव॒ प्रेर॑ते। प्रा॒तः स॑ङ्ग॒वे सा॒यम्। असु॑रा अपश्यञ्चम॒सङ्घृ॒तस्य॑ पू॒र्ण स्व॒धाम्॥५३॥

%1.4.9.3
तमुपोद॑तिष्ठ॒न्तम॑जुहवुः। तेन॑ संवत्स॒र ऊर्ज॒मवा॑रुन्धत। ते दे॒वा अ॑मन्यन्त। अ॒मी वा इ॒दम॑भूवन्। यद्व॒य स्म इति॑। त ए॒तानि॑ चातुर्मा॒स्यान्य॑पश्यन्। तानि॒ निर॑वपन्। तैरे॒वैषा॒न्तामूर्ज॑मवृञ्जत। ततो॑ दे॒वा अभ॑वन्। पराऽसु॑राः॥५४॥

%1.4.9.4
यद्यज॑ते। यामे॒व दे॒वा ऊर्ज॑म॒वारु॑न्धत। तान्तेनाव॑रुन्धे। यत्पि॒तृभ्य॑ क॒रोति॑। यामे॒व पि॒तर॒ ऊर्ज॑म॒वारु॑न्धत। तान्तेनाव॑रुन्धे। यदा॑वस॒थेऽन्न॒ हर॑न्ति। यामे॒व म॑नु॒ष्या॑ ऊर्ज॑म॒वारु॑न्धत। तान्तेनाव॑रुन्धे। यद्दक्षि॑णा॒न्ददा॑ति॥५५॥

%1.4.9.5
यामे॒व प॒शव॒ ऊर्ज॑म॒वारु॑न्धत। तान्तेनाव॑रुन्धे। यच्चा॑तुर्मा॒स्यैर्यज॑ते। यामे॒वासु॑रा॒ ऊर्ज॑म॒वारु॑न्धत। तान्तेनाव॑रुन्धे। भव॑त्या॒त्मना। परास्य॒ भ्रातृ॑व्यो भवति। वि॒राजो॒ वा ए॒षा विक्रान्तिः। यच्चा॑तुर्मा॒स्यानि॑। वै॒श्व॒दे॒वेना॒स्मिल्लोँ॒के प्रत्य॑तिष्ठत्। व॒रु॒ण॒प्र॒घा॒सैर॒न्तरि॑क्षे। सा॒क॒मे॒धैर॒मुष्मि॑ल्लोँ॒के। ए॒ष ह॒ त्वावैतत्सर्वं॑ भवति। य ए॒वं वि॒द्वाश्चा॑तुर्मा॒स्यैर्यज॑ते॥५६॥\anuvakamend[म॒नु॒ष्या॑ अपश्यञ्चम॒सङ्घृ॒तस्य॑ पू॒र्ण स्व॒धामसु॑रा अपश्यञ्चम॒सङ्घृ॒तस्य॑ पू॒र्ण स्व॒धामसु॑रा॒ ददात्यतिष्ठच्च॒त्वारि॑ च]

%1.4.10.1
अ॒ग्निर्वाव सं॑वत्स॒रः। आ॒दि॒त्यः प॑रिवत्स॒रः। च॒न्द्रमा॑ इदावत्स॒रः। वा॒युर॑नुवत्स॒रः। यद्वैश्वदे॒वेन॒ यज॑ते। अ॒ग्निमे॒व तत्सं॑वत्स॒रमाप्नोति। तस्माद्वैश्वदे॒वेन॒ यज॑मानः। सं॒व॒त्स॒रीणा स्व॒स्तिमाशास्त॒ इत्याशा॑सीत। यद्व॑रुण\-प्रघा॒सैर्यज॑ते। आ॒दि॒त्यमे॒व तत्प॑रिवत्स॒रमाप्नोति॥५७॥

%1.4.10.2
तस्माद्वरुणप्रघा॒सैर्यज॑मानः। प॒रि॒व॒त्स॒रीणा स्व॒स्तिमाशास्त॒ इत्याशा॑सीत। यत्सा॑कमे॒धैर्यज॑ते। च॒न्द्रम॑समे॒व तदि॑दावत्स॒र\-माप्नोति। तस्मात्साकमे॒धैर्यज॑मानः। इ॒दा॒व॒त्स॒रीणा स्व॒स्तिमाशास्त॒ इत्याशा॑सीत। यत्पि॑तृय॒ज्ञेन॒ यज॑ते। दे॒वाने॒व तद॒न्वव॑स्यति। अथ॒वा अ॑स्य वा॒युश्चा॑नुवत्स॒रश्चाप्री॑ता॒\-वुच्छि॑ष्येते। यच्छु॑नासी॒रीये॑ण॒ यज॑ते॥५८॥

%1.4.10.3
वा॒युमे॒व तद॑नुवत्स॒रमाप्नोति। तस्माच्छुनासी॒रीये॑ण॒ यज॑मानः। अ॒नु॒व॒त्स॒रीणा स्व॒स्तिमाशास्त॒ इत्याशा॑सीत। सं॒व॒त्स॒रं वा ए॒ष ईप्स॒तीत्या॑हुः। यश्चा॑तुर्मा॒स्यैर्यज॑त॒ इति॑। ए॒ष ह॒ त्वै सं॑वत्स॒रमाप्नोति। य ए॒वं वि॒द्वाश्चा॑तुर्मा॒स्यैर्यज॑ते। विश्वे॑ दे॒वाः सम॑यजन्त। तेऽग्निमे॒वाय॑जन्त। त ए॒तं लो॒कम॑जयन्॥५९॥

%1.4.10.4
यस्मि॑न्न॒ग्निः। यद्वैश्वदे॒वेन॒ यज॑ते। ए॒तमे॒व लो॒कं ज॑यति। यस्मि॑न्न॒ग्निः। अ॒ग्नेरे॒व सायु॑ज्य॒मुपै॑ति। य॒दा वैश्वदे॒वेन॒ यज॑ते। अथ॑ संवत्स॒रस्य॑ गृ॒हप॑तिमाप्नोति। य॒दा सं॑वत्स॒रस्य॑ गृ॒हप॑तिमा॒प्नोति॑। अथ॑ सहस्रया॒जिन॑माप्नोति। य॒दा स॑हस्रया॒जिन॑मा॒प्नोति॑॥६०॥

%1.4.10.5
अथ॑ गृहमे॒धिन॑माप्नोति। य॒दा गृ॑हमे॒धिन॑मा॒प्नोति॑। अथा॒ग्निर्भ॑वति। य॒दाग्निर्भव॑ति। अथ॒ गौर्भ॑वति। ए॒षा वै वैश्वदे॒वस्य॒ मात्रा। ए॒तद्वा ए॒तेषा॑मव॒मम्। अतो॑तो॒ वा उत्त॑राणि॒ श्रेयासि भवन्ति। यद्विश्वे॑ दे॒वाः स॒मय॑जन्त। तद्वैश्वदे॒वस्य॑ वैश्वदेव॒त्वम्॥६१॥

%1.4.10.6
अथा॑दि॒त्यो वरु॑ण॒ रा॑जानं वरुणप्रघा॒सैर॑यजत। स ए॒तं लो॒कम॑जयत्। यस्मि॑न्नादि॒त्यः। यद्व॑रुणप्रघा॒सैर्यज॑ते। ए॒तमे॒व लो॒कं ज॑यति। यस्मि॑न्नादि॒त्यः। आ॒दि॒त्यस्यै॒व सायु॑ज्य॒मुपै॑ति। यदा॑दि॒त्यो वरु॑ण॒ राजा॑नं वरुणप्रघा॒सै\-रय॑जत। तद्व॑रुणप्रघा॒सानां वरुणप्रघास॒त्वम्। अथ॒ सोमो॒ राजा॒ छन्दासि साकमे॒धैर॑यजत॥६२॥

%1.4.10.7
स ए॒तं लो॒कम॑जयत्। यस्मिश्च॒न्द्रमा॑ वि॒भाति॑। यत्सा॑कमे॒धैर्यज॑ते। ए॒तमे॒व लो॒कं ज॑यति। यस्मिश्च॒न्द्रमा॑ वि॒भाति॑। च॒न्द्रम॑स ए॒व सायु॑ज्य॒मुपै॑ति। सोमो॒ वै च॒न्द्रमा। ए॒ष ह॒ त्वै सा॒क्षात्सोमं॑ भक्षयति। य ए॒वं वि॒द्वान्त्सा॑कमे॒धैर्यज॑ते। यत्सोम॑श्च॒ राजा॒ छन्दा॑सि च स॒मैध॑न्त॥६३॥

%1.4.10.8
तत्सा॑कमे॒धाना साकमेध॒त्वम्। अथ॒र्तव॑ पि॒तर॑ प्र॒जाप॑तिं पि॒तरं॑ पितृय॒ज्ञेना॑यजन्त। त ए॒तं लो॒कम॑जयन्। यस्मि॑न्नृ॒तव॑। यत्पि॑तृय॒ज्ञेन॒ यज॑ते। ए॒तमे॒व लो॒कं ज॑यति। यस्मि॑न्नृ॒तव॑। ऋ॒तू॒नामे॒व सायु॑ज्य॒मुपै॑ति। यदृ॒तव॑ पि॒तर॑ प्र॒जाप॑तिं पि॒तरं॑ पितृय॒ज्ञेनाय॑जन्त। तत्पि॑तृय॒ज्ञस्य॑ पितृयज्ञ॒त्वम्॥६४॥

%1.4.10.9
अथौष॑धय इ॒मं दे॒वन्त्र्य॑म्बकैरयजन्त॒ प्रथे॑म॒हीति॑। ततो॒ वै ता अ॑प्रथन्त। य ए॒वं वि॒द्वास्त्र्य॑म्बकै॒र्यज॑ते। प्रथ॑ते प्र॒जया॑ प॒शुभि॑। अथ॑ वा॒युः प॑रमे॒ष्ठिन शुनासी॒रीये॑णायजत। स ए॒तं लो॒कम॑जयत्। यस्मि॑न्वा॒युः। यच्छु॑नासी॒रीये॑ण॒ यज॑ते। ए॒तमे॒व लो॒कं ज॑यति। यस्मि॑न्वा॒युः॥६५॥

%1.4.10.10
वा॒योरे॒व सायु॑ज्य॒मुपै॑ति। ब्र॒ह्म॒वा॒दिनो॑ वदन्ति। प्र चा॑तुर्मास्यया॒जी मी॑य॒ता (३) न प्रमी॑य॒ता (३) इति॑। जीव॒न्वा ए॒ष ऋ॒तूनप्ये॑ति। यदि॑ व॒सन्ता प्र॒मीय॑ते। व॒स॒न्तो भ॑वति। यदि॑ ग्री॒ष्मे ग्री॒ष्मः। यदि॑ व॒र्॒षासु॑ व॒र्॒षाः। यदि॑ श॒रदि॑ श॒रत्। यदि॒ हेम॑न् हेम॒न्तः। ऋ॒तुर्भू॒त्वा सं॑वत्स॒रमप्ये॑ति। सं॒व॒त्स॒रः प्र॒जाप॑तिः। प्र॒जाप॑ति॒र्वावैषः॥६६॥\anuvakamend[प॒रि॒व॒त्स॒रमाप्नोति शुनासी॒रीये॑ण॒ यज॑तेऽजयन्त्सहस्रया॒जिन॑मा॒प्नोति॑ वैश्वदेव॒त्व सा॑कमे॒धैर॑यजत स॒मैध॑न्त पितृयज्ञ॒त्वं ज॑यति॒ यस्मि॑न्वा॒युर्\mbox{}हे॑म॒न्तस्त्रीणि॑ च]






\prashnaend{उ॒भये॑ यु॒व सु॒राम॒मुद॑स्था॒न्नि वै यस्य॑ प्रातः सव॒न एकै॑कोऽसु॒र्यं॑ पव॑मानः प्र॒जा वै स॒त्रमा॑सता॒ग्निर्वाव सं॑वत्स॒रो दश॑॥१०॥}{उ॒भये॒ वा उद॑स्था॒त्सर्वा॑भिर्मध्य॒तोऽत्र॒ वाव ब्राह्म॒णेष्वथ॑ गृहमे॒धिन॒ षट्थ्ष॑ष्टिः॥६६॥}{उ॒भये॒ वा वैषः॥}{हरि॑ ओम्॥}{इति श्रीकृष्णयजुर्वेदीयतैत्तिरीयब्राह्मणे प्रथमाष्टके चतुर्थः प्रपाठकः समाप्तः॥}
\clearpage
