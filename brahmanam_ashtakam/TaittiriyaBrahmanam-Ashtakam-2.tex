\chapt{अष्टकम् २}
\sect{प्रथमः प्रश्नः}
\setcounter{anuvakam}{0}
\dnsub{तैत्तिरीयब्राह्मणे द्वितीयाष्टके प्रथमः प्रपाठकः}

%2.1.1.1
अङ्गि॑रसो॒ वै स॒त्रमा॑सत। तेषां॒ पृश़्ञि॑र्घर्म॒धुगा॑सीत्। सर्जी॒षेणा॑जीवत्। तेऽब्रुवन्। कस्मै॒ नु स॒त्रमास्महे। येऽस्या ओष॑धी॒र्न ज॒नया॑म॒ इति॑। ते दि॒वो वृष्टि॑मसृजन्त। याव॑न्तः स्तो॒का अ॒वाप॑द्यन्त। ताव॑ती॒रोष॑धयोऽजायन्त। ता जा॒ताः पि॒तरो॑ वि॒षेणा॑लिम्पन्॥१॥

%2.1.1.2
तासाञ्ज॒ग्ध्वा रुप्य॒न्त्यैत्। तेऽब्रुवन्। क इ॒दमि॒त्थम॑क॒रिति॑। व॒यं भा॑ग॒धेय॑मि॒च्छमा॑ना॒ इति॑ पि॒तरोऽब्रुवन्। किव्वोँ॑ भाग॒धेय॒मिति॑। अ॒ग्नि॒हो॒त्र ए॒व नोऽप्य॒स्त्वित्य॑ब्रुवन्। तेभ्य॑ ए॒तद्भा॑ग॒धेयं॒ प्राय॑च्छन्। यद्धु॒त्वा नि॒मार्ष्टि॑। ततो॒ वै त ओष॑धीरस्वदयन्। य ए॒वं वेद॑॥२॥

%2.1.1.3
स्वद॑न्तेऽस्मा॒ ओष॑धयः। ते व॒त्समु॒पावा॑सृजन्। इ॒दन्नो॑ ह॒व्यं प्रदा॑प॒येति॑। सोऽब्रवी॒द्वरं॑ वृणै। दश॑ मा॒ रात्रीर्जा॒तन्न दो॑हन्। आ॒स॒ङ्ग॒वं मा॒त्रा स॒ह च॑रा॒णीति॑। तस्माद्व॒त्सञ्जा॒तन्दश॒ रात्री॒र्न दु॑हन्ति। आ॒स॒ङ्ग॒वं मा॒त्रा स॒ह च॑रति। वारे॑वृत॒ ह्य॑स्य। तस्माद्व॒त्स ससृष्टध॒य रु॒द्रो घातु॑कः। अति॒ हि स॒न्धान्धय॑ति॥३॥\anuvakamend[अ॒लि॒म्प॒न्वेद॒ घातु॑क॒ एकं च]

%2.1.2.1
प्र॒जाप॑तिर॒ग्निम॑सृजत। तं प्र॒जा अन्व॑सृज्यन्त। तम॑भा॒ग उपास्त। सोऽस्य प्र॒जाभि॒रपाक्रामत्। तम॑व॒रुरु॑त्समा॒नोऽन्वैत्। तम॑व॒रुध॒न्नाश॑क्नोत्। स तपो॑ऽतप्यत। सोऽग्निरुपा॑रम॒ताता॑पि॒ वै स्य प्र॒जाप॑ति॒रिति॑। स र॒राटा॒दुद॑मृष्ट॥४॥

%2.1.2.2
तद्घृ॒तम॑भवत्। तस्मा॒द्यस्य॑ दक्षिण॒तः केशा॒ उन्मृ॑ष्टाः। ताञ्ज्येष्ठल॒क्ष्मी प्रा॑जाप॒त्येत्या॑हुः। यद्र॒राटा॑दु॒दमृ॑ष्ट। तस्माद्र॒राटे॒ केशा॒ न स॑न्ति। तद॒ग्नौ प्रागृ॑ह्णात्। तद्व्य॑चिकित्सत्। जु॒हवा॒नी ३ मा हौ॒षा ३ मिति॑। तद्वि॑चिकि॒त्सायै॒ जन्म॑। य ए॒वं वि॒द्वान् वि॑चि॒कित्स॑ति॥५॥

%2.1.2.3
वसी॑य ए॒व चे॑तयते। तं वाग॒भ्य॑वदज्जु॒हुधीति॑। सोऽब्रवीत्। कस्त्वम॒सीति॑। स्वैव ते॒ वागित्य॑ब्रवीत्। सो॑ऽजुहो॒त्स्वाहेति॑। तत्स्वा॑हाका॒रस्य॒ जन्म॑। य ए॒वस्वा॑हाका॒रस्य॒ जन्म॒ वेद॑। क॒रोति॑ स्वाहाका॒रेण॑ वी॒र्यम्। यस्यै॒वं वि॒दुष॑ स्वाहाका॒रेण॒ जुह्व॑ति॥६॥

%2.1.2.4
भोगा॑यै॒वास्य॑ हु॒तं भ॑वति। तस्या॒ आहु॑त्यै॒ पुरु॑षमसृजत। द्वि॒तीय॑मजुहोत्। सोऽश्व॑मसृजत। तृ॒तीय॑मजुहोत्। स गाम॑सृजत। च॒तु॒र्थम॑जुहोत्। सोऽवि॑मसृजत। प॒ञ्च॒मम॑जुहोत्। सो॑ऽजाम॑सृजत॥७॥

%2.1.2.5
सोऽग्निर॑बिभेत्। आहु॑तीभि॒र्वै माऽऽप्नो॒तीति॑। स प्र॒जाप॑तिं॒ पुन॒ प्रावि॑शत्। तं प्र॒जाप॑तिरब्रवीत्। जाय॒स्वेति॑। सोऽब्रवीत्। किं भा॑ग॒धेय॑म॒भि ज॑निष्य॒ इति॑। तुभ्य॑मे॒वेद हू॑याता॒ इत्य॑ब्रवीत्। स ए॒तद्भा॑ग॒धेय॑म॒भ्य॑जायत। यद॑ग्निहो॒त्रम्॥८॥

%2.1.2.6
तस्मा॑दग्निहो॒त्रमु॑च्यते। तद्धू॒यमा॑नमादि॒त्योऽब्रवीत्। मा हौ॑षीः। उ॒भयो॒र्वै ना॑वे॒तदिति॑। सोऽग्निर॑ब्रवीत्। क॒थन्नौ॑ होष्य॒न्तीति॑। सा॒यमे॒व तुभ्यं॑ जु॒हुव\sn{}। प्रा॒तर्मह्य॒मित्य॑ब्रवीत्। तस्मा॑द॒ग्नये॑ सा॒य हू॑यते। सूर्या॑य प्रा॒तः॥९॥

%2.1.2.7
आ॒ग्ने॒यी वै रात्रि॑। ऐ॒न्द्रमह॑। यदनु॑दिते॒ सूर्ये प्रा॒तर्जु॑हु॒यात्। उ॒भय॑मे॒वाग्ने॒य स्यात्। उदि॑ते॒ सूर्ये प्रा॒तर्जु॑होति। तथा॒ग्नये॑ सा॒य हू॑यते। सूर्या॑य प्रा॒तः। रात्रिं॒ वा अनु॑ प्र॒जाः प्र जा॑यन्ते। अह्ना॒ प्रति॑ तिष्ठन्ति। यत्सा॒यं जु॒होति॑॥१०॥

%2.1.2.8
प्रैव तेन॑ जायते। उदि॑ते॒ सूर्ये प्रा॒तर्जु॑होति। प्रत्ये॒व तेन॑ तिष्ठति। प्र॒जाप॑तिरकामयत॒ प्रजा॑ये॒येति॑। स ए॒तद॑ग्निहो॒त्रं मि॑थु॒नम॑पश्यत्। तदुदि॑ते॒ सूर्ये॑ऽजुहोत्। यजु॑षा॒ऽन्यत्। तू॒ष्णीम॒न्यत्। ततो॒ वै स प्राजा॑यत। यस्यै॒वं वि॒दुष॒ उदि॑ते॒ सूर्येऽग्निहो॒त्रं जुह्व॑ति॥११॥

%2.1.2.9
प्रैव जा॑यते। अथो॒ यथा॒ दिवा प्रजा॒नन्नेति॑। ता॒दृगे॒व तत्। अथो॒ खल्वा॑हुः। यस्य॒ वै द्वौ पुण्यौ॑ गृ॒हे वस॑तः। यस्तयो॑र॒न्य रा॒धय॑त्य॒न्यन्न। उ॒भौ वाव स तावृ॑च्छ॒तीति॑। अ॒ग्निं वावादि॒त्यः सा॒यं प्र वि॑शति। तस्मा॑द॒ग्निर्दू॒रान्नक्त॑न्ददृशे। उ॒भे हि तेज॑सी सं॒ पद्ये॑ते॥१२॥

%2.1.2.10
उ॒द्यन्तं॒ वावादि॒त्यम॒ग्निरनु॑ स॒मारो॑हति। तस्माद्धू॒म ए॒वाग्नेर्दिवा॑ ददृशे। यद॒ग्नये॑ सा॒यं जु॑हु॒यात्। आ सूर्या॑य वृश्च्येत। यत्सूर्या॑य प्रा॒तर्जु॑हु॒यात्। आऽग्नये॑ वृश्च्येत। दे॒वताभ्यः स॒मद॑न्दध्यात्। अ॒ग्निर्ज्योति॒र्ज्योति॒ सूर्य॒ स्वाहेत्ये॒व सा॒य हो॑त॒व्यम्। सूर्यो॒ ज्योति॒र्ज्योति॑र॒ग्निः स्वाहेति॑ प्रा॒तः। तथो॒भाभ्या सा॒य हू॑यते॥१३॥

%2.1.2.11
उ॒भाभ्यां प्रा॒तः। न दे॒वताभ्यः स॒मद॑न्दधाति। अ॒ग्निर्ज्योति॒रित्या॑ह। अ॒ग्निर्वै रे॑तो॒धाः। प्र॒जा ज्योति॒रित्या॑ह। प्र॒जा ए॒वास्मै॒ प्र ज॑नयति। सूर्यो॒ ज्योति॒रित्या॑ह। प्र॒जास्वे॒व प्रजा॑तासु॒ रेतो॑ दधाति। ज्योति॑र॒ग्निः स्वाहेत्या॑ह। प्र॒जा ए॒व प्रजा॑ता अ॒स्यां प्रति॑ष्ठापयति॥१४॥

%2.1.2.12
तू॒ष्णीमुत्त॑रा॒माहु॑तिं जुहोति। मि॒थु॒न॒त्वाय॒ प्रजात्यै। यदुदि॑ते॒ सूर्ये प्रा॒तर्जु॑हु॒यात्। यथाऽति॑थये॒ प्रद्रु॑ताय शू॒न्याया॑वस॒थाया॑हा॒र्य हर॑न्ति। ता॒दृगे॒व तत्। क्वाह॒ तत॒स्तद्भव॒तीत्या॑हुः। यत्स न वेद॑। यस्मै॒ तद्धर॒न्तीति॑। तस्मा॒द्यदौ॑ष॒सं जु॒होति॑। तदे॒व सं॑प्र॒ति। अथो॒ यथा॒ प्रार्थ॑मौष॒सं प॑रि॒वेवेष्टि। ता॒दृगे॒व तत्॥१५॥\anuvakamend[अ॒मृ॒ष्ट॒ वि॒चि॒कित्स॑ति॒ जुह्व॑त्य॒जाम॑सृजताग्निहो॒त्र सूर्या॑य प्रा॒तर्जु॒होति॒ जुह्व॑ति सं॒पद्ये॑ते हूयते स्थापयति संप्र॒ति द्वे च॑]

%2.1.3.1
रु॒द्रो वा ए॒षः। यद॒ग्निः। पत्नी स्था॒ली। यन्मध्ये॒ऽग्नेर॑धि॒श्रयेत्। रु॒द्राय॒ पत्नी॒मपि॑ दध्यात्। प्र॒मायु॑का स्यात्। उदी॒चोऽङ्गा॑रान्नि॒रूह्याधि॑ श्रयति। पत्नि॑यै गोपी॒थाय॑। व्य॑न्तान्करोति। तथा॒ पत्न्यप्र॑मायुका भवति॥१६॥

%2.1.3.2
घ॒र्मो वा ए॒षोऽशान्तः। अह॑रह॒ प्र वृ॑ज्यते। यद॑ग्निहो॒त्रम्। प्रति॑षिञ्चेत्प॒शुका॑मस्य। शा॒न्तमि॑व॒ हि प॑श॒व्यम्। न प्रति॑षिञ्चेद्ब्रह्मवर्च॒सका॑मस्य। समि॑द्धमिव॒ हि ब्र॑ह्मवर्च॒सम्। अथो॒ खलु॑। प्र॒ति॒षिच्य॑मे॒व। यत्प्र॑तिषि॒ञ्चति॑॥१७॥

%2.1.3.3
तत्प॑श॒व्यम्। यज्जु॒होति॑। तद्ब्र॑ह्मवर्च॒सि। उ॒भय॑मे॒वाक॑। प्रच्यु॑तं॒ वा ए॒तद॒स्माल्लो॒कात्। अग॑तन्देवलो॒कम्। यच्छृ॒त ह॒विरन॑भिघारितम्। अ॒भि द्यो॑तयति। अ॒भ्ये॑वैन॑द्घारयति। अथो॑ देव॒त्रैवैन॑द्गमयति॥१८॥

%2.1.3.4
पर्य॑ग्नि करोति। रक्ष॑सा॒मप॑हत्यै। त्रिः पर्य॑ग्नि करोति। त्र्या॑वृ॒द्धि य॒ज्ञः। अथो॑ मेध्य॒त्वाय॑। यत्प्रा॒चीन॑मुद्वा॒सयेत्। यज॑मान शु॒चाऽर्प॑येत्। यद्द॑क्षि॒णा। पि॒तृ॒दे॒व॒त्य स्यात्। यत्प्र॒त्यक्॥१९॥

%2.1.3.5
पत्नी शु॒चाऽर्प॑येत्। उ॒दी॒चीन॒मुद्वा॑सयति। ए॒षा वै दे॑वमनु॒ष्याणा शा॒न्ता दिक्। तामे॒वैन॒दनूद्वा॑सयति॒ शान्त्यै। वर्त्म॑ करोति। य॒ज्ञस्य॒ सन्त॑त्यै। निष्ट॑पति। उपै॒व तत्स्तृ॑णाति। च॒तुरुन्न॑यति। चतु॑ष्पादः प॒शव॑॥२०॥

%2.1.3.6
प॒शूने॒वाव॑रुन्धे। सर्वान्पू॒र्णानुन्न॑यति। सर्वे॒ हि पुण्या॑ रा॒द्धाः। अ॒नूच॒ उन्न॑यति। प्र॒जाया॑ अनूचीन॒त्वाय॑। अ॒नूच्ये॒वास्य॑ प्र॒जाऽर्धु॑का भवति। संमृ॑शति॒ व्यावृ॑त्त्यै। नाहोष्य॒न्नुप॑ सादयेत्। यदहोष्यन्नुपसा॒दयेत्। यथा॒ऽन्यस्मा॑ उपनि॒धाय॑॥२१॥

%2.1.3.7
अ॒न्यस्मै प्र॒यच्छ॑ति। ता॒दृगे॒व तत्। आऽस्मै॑ वृश्च्येत। यदे॒व गार्\mbox{}ह॑पत्येऽधि॒ श्रय॑ति। तेन॒ गार्\mbox{}ह॑पत्यं प्रीणाति। अ॒ग्निर॑बिभेत्। आहु॑तयो॒ माऽत्येष्य॒न्तीति॑। स ए॒ता स॒मिध॑मपश्यत्। तामाऽध॑त्त। ततो॒ वा अ॒ग्नावाहु॑तयोऽध्रियन्त॥२२॥

%2.1.3.8
यदे॑न स॒मय॑च्छत्। तत्स॒मिध॑ समि॒त्त्वम्। स॒मिध॒मा द॑धाति। समे॒वैनं॑ यच्छति। आहु॑तीना॒न्धृत्यै। अथो॑ अग्निहो॒त्रमे॒वेध्मव॑त्करोति। आहु॑तीनां॒ प्रति॑ष्ठित्यै। ब्र॒ह्म॒वा॒दिनो॑ वदन्ति। यदेका स॒मिध॑मा॒धाय॒ द्वे आहु॑ती जु॒होति॑। अथ॒ कस्या स॒मिधि॑ द्वि॒तीया॒माहु॑तिं जुहो॒तीति॑॥२३॥

%2.1.3.9
यद्द्वे स॒मिधा॑वा द॒ध्यात्। भ्रातृ॑व्यमस्मै जनयेत्। एका स॒मिध॑मा॒धाय॑। यजु॑षा॒ऽन्यामाहु॑तिं जुहोति। उ॒भे ए॒व स॒मिद्व॑ती॒ आहु॑ती जुहोति। नास्मै॒ भ्रातृ॑व्यञ्जनयति। आदीप्तायां जुहोति। समि॑द्धमिव॒ हि ब्र॑ह्मवर्च॒सम्। अथो॒ यथाऽति॑थिं॒ ज्योति॑ष्कृ॒त्वा प॑रि॒ वेवेष्टि। ता॒दृगे॒व तत्। च॒तुरुन्न॑यति। द्विर्जु॑होति। तस्माद्द्वि॒पाच्चतु॑ष्पादमत्ति। अथो द्वि॒पद्ये॒व चतु॑ष्पद॒ प्रति॑ ष्ठापयति॥२४॥\anuvakamend[भ॒व॒ति॒ प्र॒ति॒षि॒ञ्चति॑ गमयति प्र॒त्यक्प॒शव॑ उपनि॒धायाध्रिय॒न्तेति॒ तच्च॒त्वारि॑ च]

%2.1.4.1
उ॒त्त॒राव॑तीँ॒व्वै दे॒वा आहु॑ति॒मजु॑हवुः। अवा॑ची॒मसु॑राः। ततो॑ दे॒वा अभ॑वन्। पराऽसु॑राः। यङ्का॒मये॑त॒ वसी॑यान्त्स्या॒दिति॑। कनी॑य॒स्तस्य॒ पूर्व हु॒त्वा। उत्त॑रं॒ भूयो॑ जुहुयात्। ए॒षा वा उ॑त्त॒राव॒त्याहु॑तिः। तान्दे॒वा अ॑जुहवुः। तत॒स्ते॑ऽभवन्॥२५॥

%2.1.4.2
यस्यै॒वं जुह्व॑ति। भव॑त्ये॒व। यङ्का॒मये॑त॒ पापी॑यान्त्स्या॒दिति॑। भूय॒स्तस्य॒ पूर्व हु॒त्वा। उत्त॑र॒ङ्कनी॑यो जुहुयात्। ए॒षा वा अवा॒च्याहु॑तिः। तामसु॑रा अजुहवुः। तत॒स्ते परा॑ऽभवन्। यस्यै॒वं जुह्व॑ति। परै॒व भ॑वति॥२६॥

%2.1.4.3
हु॒त्वोप॑ सादय॒त्यजा॑मित्वाय। अथो॒ व्यावृ॑त्त्यै। गार्\mbox{}ह॑पत्यं॒ प्रतीक्षते। अन॑नुध्यायिनमे॒वैनं॑ करोति। अ॒ग्नि॒हो॒त्रस्य॒ वै स्था॒णुर॑स्ति। तं य ऋ॒च्छेत्। य॒ज्ञ॒स्था॒णुमृ॑च्छेत्। ए॒ष वा अ॑ग्निहो॒त्रस्य॑ स्था॒णुः। यत्पूर्वाऽऽहु॑तिः। तां यदुत्त॑रया॒ऽभि जु॑हु॒यात्॥२७॥

%2.1.4.4
य॒ज्ञ॒स्था॒णुमृ॑च्छेत्। अ॒ति॒हाय॒ पूर्वा॒माहु॑तिं जुहोति। य॒ज्ञ॒स्था॒णुमे॒व परि॑ वृणक्ति। अथो॒ भ्रातृ॑व्यमे॒वाप्त्वाऽति॑ क्रामति। अ॒वा॒चीन सा॒यमुप॑मार्ष्टि। रेत॑ ए॒व तद्द॑धाति। ऊ॒र्ध्वं प्रा॒तः। प्र ज॑नयत्ये॒व तत्। ब्र॒ह्म॒वा॒दिनो॑ वदन्ति। च॒तुरुन्न॑यति॥२८॥

%2.1.4.5
द्विर्जु॑होति। अथ॒ क्व॑ द्वे आहु॑ती भवत॒ इति॑। अ॒ग्नौ वैश्वान॒र इति॑ ब्रूयात्। ए॒ष वा अ॒ग्निर्वैश्वान॒रः। यद्ब्राह्म॒णः। हु॒त्वा द्विः प्राश्ञा॑ति। अ॒ग्नावे॒व वैश्वान॒रे द्वे आहु॑ती जुहोति। द्विर्जु॒होति॑। द्विर्निमार्ष्टि। द्विः प्राश्ञा॑ति॥२९॥

%2.1.4.6
षट्त्संप॑द्यन्ते। षड्वा ऋ॒तव॑। ऋ॒तूने॒व प्री॑णाति। ब्र॒ह्म॒वा॒दिनो॑ वदन्ति। किं॒ दे॒व॒त्य॑मग्निहो॒त्रमिति॑। वै॒श्व॒दे॒वमिति॑ ब्रूयात्। यद्यजु॑षा जु॒होति॑। तदैन्द्रा॒ग्नम्। यत्तू॒ष्णीम्। तत्प्रा॑जाप॒त्यम्॥३०॥

%2.1.4.7
यन्नि॒मार्ष्टि॑। तदोष॑धीनाम्। यद्द्वि॒तीयम्। तत्पि॑तृ॒णाम्। यत्प्राश्ञा॑ति। तद्गर्भा॑णाम्। तस्मा॒द्गर्भा॒ अन॑श्ञन्तो वर्धन्ते। यदा॒चाम॑ति। तन्म॑नु॒ष्या॑णाम्। उद॑ङ्पर्या॒वृत्याचा॑मति॥३१॥

%2.1.4.8
आ॒त्मनो॑ गोपी॒थाय॑। निर्णे॑नेक्ति॒ शुद्ध्यै। निष्ट॑पति स्व॒गाकृ॑त्यै। उद्दि॑शति। स॒प्त॒र्॒षीने॒व प्री॑णाति। द॒क्षि॒णा प॒र्याव॑र्तते। स्वमे॒व वी॒र्य॑मनु॑ प॒र्याव॑र्तते। तस्मा॒द्दक्षि॒णोऽर्ध॑ आ॒त्मनो॑ वी॒र्या॑वत्तरः। अथो॑ आदि॒त्यस्यै॒वावृत॒मनु॑ प॒र्याव॑र्तते। हु॒त्वोप॒ समि॑न्धे॥३२॥

%2.1.4.9
ब्र॒ह्म॒व॒र्च॒सस्य॒ समि॑द्ध्यै। न ब॒र्॒हिरनु॒ प्र ह॑रेत्। असस्थितो॒ वा ए॒ष य॒ज्ञः। यद॑ग्निहो॒त्रम्। यद॑नु प्र॒हरेत्। य॒ज्ञं विच्छि॑न्द्यात्। तस्मा॒न्नानु॑ प्र॒हृत्यम्। य॒ज्ञस्य॒ सन्त॑त्यै। अ॒पो नि न॑यति। अ॒व॒भृ॒थस्यै॒व रू॒पम॑कः॥३३॥\anuvakamend[अ॒भ॒व॒न्भ॒व॒ति॒ जु॒हु॒यान्न॑यति मार्ष्टि॒ द्विः प्राश्ञा॑ति प्राजाप॒त्यमाचा॑मतीन्धेऽकः]

%2.1.5.1
ब्र॒ह्म॒वा॒दिनो॑ वदन्ति। अ॒ग्नि॒हो॒त्रप्रा॑यणा य॒ज्ञाः। किंप्रा॑यणमग्निहो॒त्रमिति॑। व॒त्सो वा अ॑ग्निहो॒त्रस्य॒ प्राय॑णम्। अ॒ग्नि॒हो॒त्रं य॒ज्ञानाम्। तस्य॑ पृथि॒वी सद॑। अ॒न्तरि॑क्ष॒माग्नीद्ध्रम्। द्यौर्\mbox{}ह॑वि॒र्धानम्। दि॒व्या आप॒ प्रोक्ष॑णयः। ओष॑धयो ब॒र्॒हिः॥३४॥

%2.1.5.2
वन॒स्पत॑य इ॒ध्मः। दिश॑ परि॒धय॑। आ॒दि॒त्यो यूप॑। यज॑मानः प॒शुः। स॒मु॒द्रो॑ऽवभृ॒थः। सं॒व॒त्स॒रः स्व॑गाका॒रः। तस्मा॒दाहि॑ताग्ने॒ सर्व॑मे॒व ब॑र्\mbox{}हि॒ष्य॑न्द॒त्तं भ॑वति। यत्सा॒यं जु॒होति॑। रात्रि॑मे॒व तेन॑ दक्षि॒ण्यां कुरुते। यत्प्रा॒तः॥३५॥

%2.1.5.3
अह॑रे॒व तेन॑ दक्षि॒ण्यं॑ कुरुते। यत्ततो॒ ददा॑ति। सा दक्षि॑णा। याव॑न्तो॒ वै दे॒वा अहु॑त॒माद\sn{}। ते परा॑ऽभवन्। त ए॒तद॑ग्निहो॒त्र सर्व॑स्यै॒व स॑मव॒दाया॑जुहवुः। तस्मा॑दाहुः। अ॒ग्नि॒हो॒त्रं वै दे॒वा गृ॒हाणां॒ निष्कृ॑तिमपश्य॒न्निति॑। यत्सा॒यं जु॒होति॑। रात्रि॑या ए॒व तद्धु॒ताद्या॑य॥३६॥

%2.1.5.4
यज॑मान॒स्याप॑राभावाय। यत्प्रा॒तः। अह्न॑ ए॒व तद्धु॒ताद्या॑य। यज॑मान॒स्याप॑राभावाय। यत्ततो॒ऽश़्ञाति॑। हु॒तमे॒व तत्। द्वयो॒ पय॑सा जुहुयात्प॒शुका॑मस्य। ए॒तद्वा अ॑ग्निहो॒त्रं मि॑थु॒नम्। य ए॒वं वेद॑। प्र प्र॒जया॑ प॒शुभि॑र्मिथु॒नैर्जा॑यते॥३७॥

%2.1.5.5
इ॒मामे॒व पूर्व॑या दु॒हे। अ॒मूमुत्त॑रया। अ॒धि॒श्रित्योत्त॑र॒मा न॑यति। योना॑वे॒व तद्रेत॑ सिञ्चति प्र॒जन॑ने। आज्ये॑न जुहुया॒त्तेज॑स्कामस्य। तेजो॒ वा आज्यम्। ते॒ज॒स्व्ये॑व भ॑वति। पय॑सा प॒शुका॑मस्य। ए॒तद्वै प॑शू॒ना रू॒पम्। रू॒पेणै॒वास्मै॑ प॒शूनव॑रुन्धे॥३८॥

%2.1.5.6
प॒शु॒माने॒व भ॑वति। द॒ध्नेन्द्रि॒यका॑मस्य। इ॒न्द्रि॒यं वै दधि॑। इ॒न्द्रि॒या॒व्ये॑व भ॑वति। य॒वा॒ग्वा ग्राम॑कामस्यौष॒धा वै म॑नु॒ष्या। भा॒ग॒धेये॑नै॒वास्मै॑ सजा॒तानव॑ रुन्धे। ग्रा॒म्ये॑व भ॑वति। अय॑ज्ञो॒ वा ए॒षः। यो॑ऽसा॒मा॥३९॥

%2.1.5.7
च॒तुरुन्न॑यति। चतु॑रक्षर रथन्त॒रम्। र॒थ॒न्त॒रस्यै॒ष वर्ण॑। उ॒परी॑व हरति। अ॒न्तरि॑क्षं वामदे॒व्यम्। वा॒म॒दे॒व्यस्यै॒ष वर्ण॑। द्विर्जु॑होति। द्व्य॑क्षरं बृ॒हत्। बृ॒ह॒त ए॒ष वर्ण॑। अ॒ग्नि॒हो॒त्रमे॒व तत्साम॑न्वत्करोति॥४०॥

%2.1.5.8
यो वा अ॑ग्निहो॒त्रस्यो॑प॒सदो॒ वेद॑। उपै॑नमुप॒सदो॑ नमन्ति। वि॒न्दत॑ उपस॒त्तारम्। उ॒न्नीयोप॑ सादयति। पृ॒थि॒वीमे॒व प्री॑णाति। हो॒ष्यन्नुप॑सादयति। अ॒न्तरि॑क्षमे॒व प्री॑णाति। हु॒त्वोप॑ सादयति। दिव॑मे॒व प्री॑णाति। ए॒ता वा अ॑ग्निहो॒त्रस्यो॑प॒सद॑॥४१॥

%2.1.5.9
य ए॒वं वेद॑। उपै॑नमुप॒सदो॑ नमन्ति। वि॒न्दत॑ उपस॒त्तारम्। यो वा अ॑ग्निहो॒त्रस्याश्रा॑वितं प्र॒त्याश्रा॑वित॒ होता॑रं ब्र॒ह्माणं॑ वषट्का॒रं वेद॑। तस्य॒ त्वे॑व हु॒तम्। प्रा॒णो वा अ॑ग्निहो॒त्रस्याश्रा॑वितम्। अ॒पा॒नः प्र॒त्याश्रा॑वितम्। मनो॒ होता। चक्षु॑र्ब्र॒ह्मा। नि॒मे॒षो व॑षट्का॒रः॥४२॥

%2.1.5.10
य ए॒वं वेद॑। तस्य॒ त्वे॑व हु॒तम्। सा॒यं॒ यावा॑नश्च॒ वै दे॒वाः प्रा॑त॒र्यावा॑णश्चाग्निहो॒त्रिणो॑ गृ॒हमाग॑च्छन्ति। तान् यन्न त॒र्पयेत्। प्र॒जयाऽस्य प॒शुभि॒र्वि ति॑ष्ठेरन्। यत्त॒र्पयेत्। तृ॒प्ता ए॑नं प्र॒जया॑ प॒शुभि॑स्तर्पयेयुः। स॒जूर्दे॒वैः सा॒यं याव॑भि॒रिति॑ सा॒य संमृ॑शति। स॒जूर्दे॒वैः प्रा॒तर्याव॑भि॒रिति॑ प्रा॒तः। ये चै॒व दे॒वाः सा॑यं॒ यावा॑नो॒ ये च॑ प्रात॒र्यावा॑णः॥४३॥

%2.1.5.11
ताने॒वोभयास्तर्पयति। त ए॑नन्तृ॒प्ताः प्र॒जया॑ प॒शुभि॑स्तर्पयन्ति। अ॒रु॒णो ह॑ स्मा॒हौप॑वेशिः। अ॒ग्नि॒हो॒त्र ए॒वाह सा॒यंप्रा॑त॒र्वज्रं॒ भ्रातृ॑व्येभ्य॒ प्र ह॑रामि। तस्मा॒न्मत्पापी॑यासो॒ भ्रातृ॑व्या॒ इति॑। च॒तुरुन्न॑यति। द्विर्जु॑होति। स॒मित्स॑प्त॒मी। स॒प्तप॑दा॒ शक्व॑री। शा॒क्व॒रो वज्र॑। अ॒ग्नि॒हो॒त्र ए॒व तत्सा॒यंप्रा॑त॒र्वज्रं॒ यज॑मानो॒ भ्रातृ॑व्याय॒ प्र ह॑रति। भव॑त्या॒त्मना। पराऽस्य॒ भ्रातृ॑व्यो भवति॥४४॥\anuvakamend[ब॒र्॒हिः प्रा॒तर्\mbox{}हु॒ताद्या॑य जायते रुन्धेऽसा॒मा क॑रोत्ये॒ता वा अ॑ग्निहो॒त्रस्यो॑प॒सदो॑ वषट्का॒रश्च॑ प्रात॒र्यावा॑णो॒ वज्र॒स्त्रीणि॑ च]

%2.1.6.1
प्र॒जाप॑तिरकामयतात्म॒न्वन्मे॑ जाये॒तेति॑। सो॑ऽजुहोत्। तस्मात्म॒न्वद॑जायत। अ॒ग्निर्वा॒युरा॑दि॒त्यः। तेऽब्रुवन्। प्र॒जाप॑तिरहौषीदात्म॒न्वन्मे॑ जाये॒तेति॑। तस्य॑ व॒यम॑जनिष्महि। जाय॑तान्न आत्म॒न्वदिति॒ ते॑ऽजुहवुः। प्रा॒णाना॑म॒ग्निः। त॒नुवै॑ वा॒युः॥४५॥

%2.1.6.2
चक्षु॑ष आदि॒त्यः। तेषा हु॒ताद॑जायत॒ गौरे॒व। तस्यै॒ पय॑सि॒ व्याय॑च्छन्त। मम॑ हु॒ताद॑जनि॒ ममेति॑। ते प्र॒जाप॑तिं प्र॒श्ञमा॑यन्। स आ॑दि॒त्योऽग्निम॑ब्रवीत्। य॒त॒रो नौ॒ जयात्। तन्नौ॑ स॒हास॒दिति॑। कस्यै कोऽहौ॑षी॒दिति॑ प्र॒जाप॑तिरब्रवी॒त्कस्यै क॒ इति॑। प्रा॒णाना॑म॒हमित्य॒ग्निः॥४६॥

%2.1.6.3
त॒नुवा॑ अ॒हमिति॑ वा॒युः। चक्षु॑षो॒ऽहमित्या॑दि॒त्यः। य ए॒व प्रा॒णाना॒महौ॑षीत्। तस्य॑ हु॒ताद॑ज॒नीति॑। अ॒ग्नेर्\mbox{}हु॒ताद॑ज॒नीति॑। तद॑ग्निहो॒त्रस्याग्निहोत्र॒त्वम्। गौर्वा अ॑ग्निहो॒त्रम्। य ए॒वं वेद॒ गौर॑ग्निहो॒त्रमिति॑। प्रा॒णा॒पा॒नाभ्या॑मे॒वाग्नि सम॑र्धयति। अव्य॑र्धुकः प्राणापा॒नाभ्यां भवति॥४७॥

%2.1.6.4
य ए॒वं वेद॑। तौ वा॒युर॑ब्रवीत्। अनु॒ मा भ॑जत॒मिति॑। यदे॒व गार्\mbox{}ह॑पत्येऽधि॒श्रित्या॑हव॒नीय॑म॒भ्यु॑द्द्रवान्॑। तेन॒ त्वां प्री॑णा॒नित्य॑ब्रूताम्। तस्मा॒द्यद्गार्\mbox{}ह॑पत्येऽधि॒श्रित्या॑हव॒नीय॑म॒भ्यु॑द्द्रव॑ति। वा॒युमे॒व तेन॑ प्रीणाति। प्र॒जाप॑तिर्दे॒वता सृ॒जमा॑नः। अ॒ग्निमे॒व दे॒वता॑नां प्रथ॒मम॑सृजत। सोऽन्यदा॑ल॒म्भ्य॑मवि॑त्वा॥४८॥

%2.1.6.5
प्र॒जाप॑तिम॒भि प॒र्याव॑र्तत। स मृ॒त्योर॑बिभेत्। सो॑ऽमुमा॑दि॒त्यमा॒त्मनो॒ निर॑मिमीत। त हु॒त्वा पराङ्प॒र्याव॑र्तत। ततो॒ वै स मृ॒त्युमपा॑जयत्। अप॑ मृ॒त्युं ज॑यति। य ए॒वं वेद॑। तस्मा॒द्यस्यै॒वं वि॒दुष॑। उ॒तैका॒हमु॒त द्व्य॒हन्न जुह्व॑ति। हु॒तमे॒वास्य॑ भवति। अ॒सौ ह्या॑दि॒त्योऽग्निहो॒त्रम्॥४९॥\anuvakamend[त॒नुवै॑ वा॒युर॒ग्निर्भ॑व॒त्यवि॑त्वा भव॒त्येकं च]

%2.1.7.1
रौ॒द्रङ्गवि॑। वा॒य॒व्य॑मुप॑सृष्टम्। आ॒श्वि॒नन्दु॒ह्यमा॑नम्। सौ॒म्यन्दु॒ग्धम्। वा॒रु॒णमधि॑ श्रितम्। वै॒श्व॒दे॒वा भि॒न्दव॑। पौ॒ष्णमुद॑न्तम्। सा॒र॒स्व॒तं वि॒ष्यन्द॑मानम्। मै॒त्र शर॑। धा॒तुरुद्वा॑सितम्। बृह॒स्पते॒रुन्नी॑तम्। स॒वि॒तुः प्र क्रान्तम्। द्या॒वा॒पृ॒थि॒व्य ह्रि॒यमा॑णम्। ऐ॒न्द्रा॒ग्नमुप॑सन्नम्। अ॒ग्नेः पूर्वाऽऽहु॑तिः। प्र॒जाप॑ते॒रुत्त॑रा। ऐ॒न्द्र हु॒तम्॥५०॥\anuvakamend[उद्वा॑सित स॒प्त च॑]

%2.1.8.1
द॒क्षि॒ण॒त उप॑ सृजति। पि॒तृ॒लो॒कमे॒व तेन॑ जयति। प्राची॒मा व॑र्तयति। दे॒व॒लो॒कमे॒व तेन॑ जयति। उदी॑चीमा॒वृत्य॑ दोग्धि। म॒नु॒ष्य॒लो॒कमे॒व तेन॑ जयति। पूर्वौ॑ दुह्याज्ज्ये॒ष्ठस्य॑ ज्यैष्ठिने॒यस्य॑। यो वा॑ ग॒तश्री॒ स्यात्। अप॑रौ दुह्यात्कनि॒ष्ठस्य॑ कानिष्ठिने॒यस्य॑। यो वा॒ बुभू॑षेत्॥५१॥

%2.1.8.2
न सं मृ॑शति। पा॒प॒व॒स्य॒सस्य॒ व्यावृ॑त्त्यै। वा॒य॒व्यं॑ वा ए॒तदुप॑सृष्टम्। आ॒श्वि॒नन्दु॒ह्यमा॑नम्। मै॒त्रन्दु॒ग्धम्। अ॒र्य॒म्ण उ॑द्वा॒स्यमा॑नम्। त्वा॒ष्ट्रमु॑न्नी॒यमा॑नम्। बृह॒स्पते॒रुन्नी॑तम्। स॒वि॒तुः प्रक्रान्तम्। द्या॒वा॒पृ॒थि॒व्य ह्रि॒यमा॑णम्॥५२॥

%2.1.8.3
ऐ॒न्द्रा॒ग्नमुप॑ सादितम्। सर्वाभ्यो॒ वा ए॒ष दे॒वताभ्यो जुहोति। योऽग्निहो॒त्रं जु॒होति॑। यथा॒ खलु॒ वै धे॒नुन्ती॒र्थे त॒र्पय॑ति। ए॒वम॑ग्निहो॒त्री यज॑मानन्तर्पयति। तृप्य॑ति प्र॒जया॑ प॒शुभि॑। प्र सु॑व॒र्गं लो॒कं जा॑नाति। पश्य॑ति पु॒त्रम्। पश्य॑ति॒ पौत्रम्। प्र प्र॒जया॑ प॒शुभि॑र्मिथु॒नैर्जा॑यते। यस्यै॒वं वि॒दुषोऽग्निहो॒त्रं जुह्व॑ति। य उ॑ चैनदे॒वं वेद॑॥५३॥\anuvakamend[बुभू॑षेद्ध्रि॒यमा॑णञ्जायते॒ द्वे च॑]

%2.1.9.1
त्रयो॒ वै प्रै॑यमे॒धा आ॑सन्। तेषा॒न्त्रिरेकोऽग्निहो॒त्रम॑जुहोत्। द्विरेक॑। स॒कृदेक॑। तेषां॒ यस्त्रिरजु॑होत्। स ऋ॒चाऽजु॑होत्। यो द्विः। स यजु॑षा। यः स॒कृत्। स तू॒ष्णीम्॥५४॥

%2.1.9.2
यश्च॒ यजु॒षाऽजु॑हो॒द्यश्च॑ तू॒ष्णीम्। तावु॒भावार्ध्नुताम्। तस्मा॒द्यजु॒षाऽऽहु॑ति॒ पूर्वा॑ होत॒व्या। तू॒ष्णीमुत्त॑रा। उ॒भे ए॒वर्धी अव॑रुन्धे। अ॒ग्निर्ज्योति॒र्ज्योति॑र॒ग्निः स्वाहेति॑ सा॒यं जु॑होति। रेत॑ ए॒व तद्द॑धाति। सूर्यो॒ ज्योति॒र्ज्योति॒ः! सूर्य॒ स्वाहेति॑ प्रा॒तः। रेत॑ ए॒व हि॒तं प्र ज॑नयति। रेतो॒ वा ए॒तस्य॑ हि॒तन्न प्र जा॑यते॥५५॥

%2.1.9.3
यस्याग्निहो॒त्रमहु॑त॒ सूर्यो॒ऽभ्यु॑देति॑। यद्यन्ते॒ स्यात्। उ॒न्नीय॒ प्राङु॒दाद्र॑वेत्। स उ॑प॒साद्यातमि॑तोरासीत। स य॒दा ताम्येत्। अथ॒ भूः स्वाहेति॑ जुहुयात्। प्र॒जाप॑ति॒र्वै भू॒तः। तमे॒वोपा॑सरत्। स ए॒वैन॒न्तत॒ उन्न॑यति। नार्ति॒मार्च्छ॑ति॒ यज॑मानः॥५६॥\anuvakamend[तू॒ष्णीञ्जा॑यते॒ यज॑मानः]

%2.1.10.1
यद॒ग्निमु॒द्धर॑ति। वस॑व॒स्तर्ह्य॒ग्निः। तस्मि॒न्॒ यस्य॒ तथा॑विधे॒ जुह्व॑ति। वसु॑ष्वे॒वास्याग्निहो॒त्र हु॒तं भ॑वति। निहि॑तो धूपा॒यञ्छे॑ते। रु॒द्रास्तर्ह्य॒ग्निः। तस्मि॒न्॒ यस्य॒ तथा॑विधे॒ जुह्व॑ति। रु॒द्रे॒ष्वे॒वास्याग्निहो॒त्र हु॒तं भ॑वति। प्र॒थ॒ममि॒ध्मम॒र्चिरा ल॑भते। आ॒दि॒त्यास्तर्ह्य॒ग्निः॥५७॥

%2.1.10.2
तस्मि॒न्॒ यस्य॒ तथा॑विधे॒ जुह्व॑ति। आ॒दि॒त्येष्वे॒वास्याग्निहो॒त्र हु॒तं भ॑वति। सर्व॑ ए॒व स॑र्व॒श इ॒ध्म आदीप्तो भवति। विश्वे॑ दे॒वास्तर्ह्य॒ग्निः। तस्मि॒न्॒ यस्य॒ तथा॑विधे॒ जुह्व॑ति। विश्वेष्वे॒वास्य॑ दे॒वेष्व॑ग्निहो॒त्र हु॒तं भ॑वति। नि॒त॒राम॒र्चिरु॒पावै॑ति लोहि॒नीके॑व भवति। इन्द्र॒स्तर्ह्य॒ग्निः। तस्मि॒न्॒ यस्य॒ तथा॑विधे॒ जुह्व॑ति। इन्द्र॑ ए॒वास्याग्निहो॒त्र हु॒तं भ॑वति॥५८॥

%2.1.10.3
अङ्गा॑रा भवन्ति। तेभ्योऽङ्गा॑रेभ्यो॒ऽर्चिरुदे॑ति। प्र॒जाप॑ति॒स्तर्ह्य॒ग्निः। तस्मि॒न्॒ यस्य॒ तथा॑विधे॒ जुह्व॑ति। प्र॒जाप॑तावे॒वास्याग्निहो॒त्र हु॒तं भ॑वति। शरोऽङ्गा॑रा॒ अध्यू॑हन्ते। ब्रह्म॒ तर्ह्य॒ग्निः। तस्मि॒न्॒ यस्य॒ तथा॑विधे॒ जुह्व॑ति। ब्रह्म॑न्ने॒वास्याग्निहो॒त्र हु॒तं भ॑वति। वसु॑षु रु॒द्रेष्वा॑दि॒त्येषु॒ विश्वे॑षु दे॒वेषु॑। इन्द्रे प्र॒जाप॑तौ॒ ब्रह्म\sn{}। अप॑रिवर्गमे॒वास्यै॒तासु॑ दे॒वता॑सु हु॒तं भ॑वति। यस्यै॒वं वि॒दुषोऽग्निहो॒त्रं जुह्व॑ति। य उ॑ चैनदे॒वं वेद॑॥५९॥\anuvakamend[आ॒दि॒त्यास्तर्ह्य॒ग्निरिन्द्र॑ ए॒वास्याग्निहो॒त्र हु॒तं भ॑वति दे॒वेषु॑ च॒त्वारि॑ च (यद॒ग्निन्निहि॑तः प्रथ॒म सर्व॑ ए॒व नि॑त॒रामङ्गा॑रा॒ शरोऽङ्गा॑रा॒ ब्रह्म॒ वसु॑ष्व॒ष्टौ ॥ )]

%2.1.11.1
ऋ॒तन्त्वा॑ स॒त्येन॒ परि॑षिञ्चा॒मीति॑ सा॒यं परि॑षिञ्चति। स॒त्यन्त्व॒र्तेन॒ परि॑षिञ्चा॒मीति॑ प्रा॒तः। अ॒ग्निर्वा ऋ॒तम्। अ॒सावा॑दि॒त्यः स॒त्यम्। अ॒ग्निमे॒व तदा॑दि॒त्येन॑ सा॒यं परि॑षिञ्चति। अ॒ग्निना॑ऽऽदि॒त्यं प्रा॒तः सः। याव॑दहोरा॒त्रे भव॑तः। ताव॑दस्य लो॒कस्य॑। नार्ति॒र्न रिष्टि॑। नान्तो॒ न प॑र्य॒न्तोऽस्ति। यस्यै॒वं वि॒दुषोऽग्निहो॒त्रं जुह्व॑ति। य उ॑चैनदे॒वं वेद॑॥६०॥\anuvakamend[अ॒स्ति॒ द्वे च॑]




\prashnaend{अङ्गि॑रसः प्र॒जाप॑तिर॒ग्नि रु॒द्र उ॑त्त॒राव॑तीं ब्रह्मवा॒दिनोऽग्निहो॒त्रप्रा॑यणा य॒ज्ञाः प्र॒जाप॑तिरकामयतात्म॒न्वद्रौ॒द्रङ्गवि॑ दक्षिण॒तस्त्रयो॒ वै यद॒ग्निमृ॒तन्त्वा॑ स॒त्येनैका॑दश॥११॥}{अङ्गि॑रस॒ प्रैव तेन॑ प॒शूने॒व यन्नि॒मार्ष्टि॒ यो वा अ॑ग्निहो॒त्रस्यो॑प॒सदो॑ दक्षिण॒तष्ष॒ष्टिः॥६०॥}{अङ्गि॑रसो॒ य उ॑चैनदे॒वं वेद॑॥}{हरि॑ ओम्॥}{इति श्रीकृष्णयजुर्वेदीयतैत्तिरीयब्राह्मणे द्वितीयाष्टके प्रथमः प्रपाठकः समाप्तः॥}
\clearpage
\sect{द्वितीयः प्रश्नः}
\setcounter{anuvakam}{0}
\dnsub{तैत्तिरीयब्राह्मणे द्वितीयाष्टके द्वितीयः प्रपाठकः}

%2.2.1.1
प्र॒जाप॑तिरकामयत प्र॒जाः सृ॑जे॒येति॑। स ए॒तन्दश॑होतारमपश्यत्। तं मन॑साऽनु॒द्रुत्य॑ दर्भस्त॒म्बे॑ऽजुहोत्। ततो॒ वै स प्र॒जा अ॑सृजत। ता अ॑स्मात्सृ॒ष्टा अपाक्रामन्। ता ग्रहे॑णागृह्णात्। तद्ग्रह॑स्य ग्रह॒त्वम्। यः का॒मये॑त॒ प्रजा॑ये॒येति॑। स दश॑होतारं॒ मन॑साऽनु॒द्रुत्य॑ दर्भस्त॒म्बे जु॑हुयात्। प्र॒जाप॑ति॒र्वै दश॑होता॥१॥

%2.2.1.2
प्र॒जाप॑तिरे॒व भू॒त्वा प्रजा॑यते। मन॑सा जुहोति। मन॑ इव॒ हि प्र॒जाप॑तिः। प्र॒जाप॑ते॒राप्त्यै। पू॒र्णया॑ जुहोति। पू॒र्ण इ॑व॒ हि प्र॒जाप॑तिः। प्र॒जाप॑ते॒राप्त्यै। न्यू॑नया जुहोति। न्यू॑ना॒द्धि प्र॒जाप॑तिः प्र॒जा असृ॑जत। प्र॒जाना॒ सृष्ट्यै॥२॥

%2.2.1.3
द॒र्भ॒स्त॒म्बे जु॑होति। ए॒तस्मा॒द्वै योने प्र॒जाप॑तिः प्र॒जा अ॑सृजत। यस्मा॑दे॒व योने प्र॒जाप॑तिः प्र॒जा असृ॑जत। तस्मा॑दे॒व योने॒ प्रजा॑यते। ब्रा॒ह्म॒णो द॑क्षिण॒त उपास्ते। ब्रा॒ह्म॒णो वै प्र॒जाना॑मुपद्र॒ष्टा। उ॒प॒द्र॒ष्टु॒मत्ये॒व प्रजा॑यते। ग्रहो॑ भवति। प्र॒जाना सृ॒ष्टाना॒न्धृत्यै। यं ब्राह्म॒णं वि॒द्यां वि॒द्वासं॒ यशो॒ नर्च्छेत्॥३॥

%2.2.1.4
सोऽर॑ण्यं प॒रेत्य॑। द॒र्भ॒स्त॒म्बमु॒द्ग्रथ्य॑। ब्रा॒ह्म॒णन्द॑क्षिण॒तो नि॒षाद्य॑। चतु॑र्होतॄ॒न्व्याच॑क्षीत। ए॒तद्वै दे॒वानां पर॒मङ्गुह्यं॒ ब्रह्म॑। यच्चतु॑र्‌होतारः। तदे॒व प्र॑का॒शं ग॑मयति। तदे॑नं प्रका॒शङ्ग॒तम्। प्र॒का॒शं प्र॒जानाङ्गमयति। द॒र्भ॒स्त॒म्बमु॒द्ग्रथ्य॒ व्याच॑ष्टे॥४॥

%2.2.1.5
अ॒ग्नि॒वान् वै द॑र्भस्त॒म्बः। अ॒ग्नि॒वत्ये॒व व्याच॑ष्टे। ब्रा॒ह्म॒णो द॑क्षिण॒त उपास्ते। ब्रा॒ह्म॒णो वै प्र॒जाना॑मुपद्र॒ष्टा। उ॒प॒द्र॒ष्टु॒मत्ये॒वैनं॒ यश॑ ऋच्छति। ई॒श्व॒रन्तं यशोर्तो॒रित्या॑हुः। यस्यान्ते व्या॒चष्ट॒ इति॑। वर॒स्तस्मै॒ देय॑। यदे॒वैनं॒ तत्रो॑प॒नम॑ति। तदे॒वाव॑ रुन्धे॥५॥

%2.2.1.6
अ॒ग्निमा॒दधा॑नो॒ दश॑होत्रा॒ऽरणि॒मव॑ दध्यात्। प्रजा॑तमे॒वैन॒मा ध॑त्ते। तेनै॒वोद्द्रुत्याग्निहो॒त्रं जु॑हुयात्। प्रजा॑तमे॒वैन॑ज्जुहोति। ह॒विर्नि॑र्व॒प्स्यन्दश॑होतारं॒ व्याच॑क्षीत। प्रजा॑तमे॒वैनं॒ निर्व॑पति। सा॒मि॒धे॒नीर॑नुव॒क्ष्यन्दश॑होतारं॒ व्याच॑क्षीत। सा॒मि॒धे॒नीरे॒व सृ॒ष्ट्वाऽऽरभ्य॒ प्रत॑नुते। अथो॑ य॒ज्ञो वै दश॑होता। य॒ज्ञमे॒व त॑नुते॥६॥

%2.2.1.7
अ॒भि॒चर॒न्दश॑होतारं जुहुयात्। नव॒ वै पुरु॑षे प्रा॒णाः। नाभि॑र्दश॒मी। सप्रा॑णमे॒वैन॑म॒भि च॑रति। ए॒ताव॒द्वै पुरु॑षस्य॒ स्वम्। याव॑त्प्रा॒णाः। याव॑दे॒वास्यास्ति॑। तद॒भि च॑रति। स्वकृ॑त॒ इरि॑णे जुहोति प्रद॒रे वा। ए॒तद्वा अ॒स्यै निर्‌ऋ॑तिगृहीतम्। निर्‌ऋ॑तिगृहीत ए॒वैनं॒ निर्‌ऋ॑त्या ग्राहयति। यद्वा॒चः क्रू॒रम्। तेन॒ वष॑ट्करोति। वा॒च ए॒वैनं॑ क्रू॒रेण॒ प्र वृ॑श्चति। ता॒जगार्ति॒मार्च्छ॑ति॥७॥\anuvakamend[दश॑होता॒ सृष्ट्या॑ ऋ॒च्छेद्व्याच॑प्टे रुन्ध ए॒व त॑नुते॒ निर्‌ऋ॑तिगृहीतं॒ पञ्च॑ च]

%2.2.2.1
प्र॒जाप॑तिरकामयत दर्‌शपूर्णमा॒सौ सृ॑जे॒येति॑। स ए॒तञ्चतु॑र्‌होतारमपश्यत्। तं मन॑साऽनु॒द्रुत्या॑हव॒नीये॑ऽजुहोत्। ततो॒ वै स द॑र्‌शपूर्णमा॒साव॑सृजत। ताव॑स्मात्सृ॒ष्टावपाक्रामताम्। तौ ग्रहे॑णागृह्णात्। तद्ग्रह॑स्य ग्रह॒त्वम्। द॒र्श॒पू॒र्ण॒मा॒सावा॒लभ॑मानः। चतु॑र्‌होतारं॒ मन॑साऽनु॒द्रुत्या॑हव॒नीये॑ जुहुयात्। द॒र्श॒पू॒र्ण॒मा॒सावे॒व सृ॒ष्ट्वाऽऽरभ्य॒ प्रत॑नुते॥८॥

%2.2.2.2
ग्रहो॑ भवति। द॒र्‌श॒पू॒र्ण॒मा॒सयो सृ॒ष्टयो॒र्धृत्यै। सो॑ऽकामयत चातुर्मा॒स्यानि॑ सृजे॒येति॑। स ए॒तं पञ्च॑होतारमपश्यत्। तं मन॑साऽनु॒द्रुत्या॑हव॒नीये॑ऽजुहोत्। ततो॒ वै स चा॑तुर्मा॒स्यान्य॑सृजत। तान्य॑स्मात्सृ॒ष्टान्यपाक्रामन्। तानि॒ ग्रहे॑णागृह्णात्। तद्ग्रह॑स्य ग्रह॒त्वम्। चा॒तु॒र्मा॒स्यान्या॒लभ॑मानः॥९॥

%2.2.2.3
पञ्च॑होतारं॒ मन॑साऽनु॒द्रुत्या॑हव॒नीये॑ जुहुयात्। चा॒तु॒र्मा॒स्यान्ये॒व सृ॒ष्ट्वाऽऽरभ्य॒ प्रत॑नुते। ग्रहो॑ भवति। चा॒तु॒र्मा॒स्याना सृ॒ष्टाना॒न्धृत्यै। सो॑ऽकामयत पशुब॒न्ध सृ॑जे॒येति॑। स ए॒त षड्ढो॑तारमपश्यत्। तं मन॑साऽनु॒द्रुत्या॑हव॒नीये॑ऽजुहोत्। ततो॒ वै स प॑शुब॒न्धम॑सृजत। सोस्मात्सृ॒ष्टोऽपाक्रामत्। तङ्ग्रहे॑णागृह्णात्॥१०॥

%2.2.2.4
तद्ग्रह॑स्य ग्रह॒त्वम्। प॒शु॒ब॒न्धेन॑ य॒क्ष्यमा॑णः। षड्ढो॑तारं॒ मन॑साऽनु॒द्रुत्या॑हव॒नीये॑ जुहुयात्। प॒शु॒ब॒न्धमे॒व सृ॒ष्ट्वाऽऽरभ्य॒ प्र त॑नुते। ग्रहो॑ भवति। प॒शु॒ब॒न्धस्य॑ सृ॒ष्टस्य॒ धृत्यै। सो॑ऽकामयत सौ॒म्यम॑ध्व॒र सृ॑जे॒येति॑। स ए॒त स॒प्तहो॑तारमपश्यत्। तं मन॑साऽनु॒द्रुत्या॑हव॒नीये॑ऽजुहोत्। ततो॒ वै स सौ॒म्यम॑ध्व॒रम॑सृजत॥११॥

%2.2.2.5
सोऽस्मात्सृ॒ष्टोऽपाक्रामत्। तङ्ग्रहे॑णागृह्णात्। तद्ग्रह॑स्य ग्रह॒त्वम्। दी॒क्षि॒ष्यमा॑णः। स॒प्तहो॑तारं॒ मन॑साऽनु॒द्रुत्या॑हव॒नीये॑ जुहुयात्। सौ॒म्यमे॒वाध्व॒र सृ॒ष्ट्वाऽऽरभ्य॒ प्र त॑नुते। ग्रहो॑ भवति। सौ॒म्यस्याध्व॒रस्य॑ सृ॒ष्टस्य॒ धृत्यै। दे॒वेभ्यो॒ वै य॒ज्ञो न प्राभ॑वत्। तमे॑ताव॒च्छः सम॑भरन्॥१२॥

%2.2.2.6
यत्सं॑भा॒राः। ततो॒ वै तेभ्यो॑ य॒ज्ञः प्राभ॑वत्। यत्सं॑भा॒रा भव॑न्ति। य॒ज्ञस्य॒ प्रभूत्यै। आ॒ति॒थ्यमा॒साद्य॒ व्याच॑ष्टे। य॒ज्ञ॒मु॒खं वा आ॑ति॒थ्यम्। मु॒ख॒त ए॒व य॒ज्ञ स॒म्भृत्य॒ प्र त॑नुते। अय॑ज्ञो॒ वा ए॒षः। यो॑ऽप॒त्नीक॑। न प्र॒जाः प्रजा॑येरन्। पत्नी॒र्व्याच॑ष्टे। य॒ज्ञमे॒वाक॑। प्र॒जानां प्र॒जन॑नाय। उ॒प॒सत्सु॒ व्याच॑ष्टे। ए॒तद्वै पत्नी॑नामा॒यत॑नम्। स्व ए॒वैना॑ आ॒यत॒नेऽव॑कल्पयति॥१३॥\anuvakamend[त॒नु॒त॒ आ॒लभ॑मानोऽगृह्णादसृजताभरञ्जायेर॒न्थ्षट्च॑]

%2.2.3.1
प्र॒जाप॑तिरकामयत॒ प्रजा॑ये॒येति॑। स तपो॑ऽतप्यत। स त्रि॒वृत॒ स्तोम॑मसृजत। तं प॑ञ्चद॒शः स्तोमो॑ मध्य॒त उद॑तृणत्। तौ पूर्वप॒क्षश्चा॑परप॒क्षश्चा॑भवताम्। पू॒र्व॒प॒क्षन्दे॒वा अन्वसृ॑ज्यन्त। अ॒प॒र॒प॒क्षमन्वसु॑राः। ततो॑ दे॒वा अभ॑वन्। पराऽसु॑राः। यङ्का॒मये॑त॒ वसी॑यान्त्स्या॒दिति॑॥१४॥

%2.2.3.2
तं पूर्वप॒क्षे या॑जयेत्। वसी॑याने॒व भ॑वति। यङ्का॒मये॑त॒ पापी॑यान्त्स्या॒दिति॑। तम॑परप॒क्षे या॑जयेत्। पापी॑याने॒व भ॑वति। तस्मात्पूर्वप॒क्षो॑ऽपरप॒क्षात्क॑रु॒ण्य॑तरः। प्र॒जाप॑ति॒र्वै दश॑होता। चतु॑र्‌होता॒ पञ्च॑होता। षड्ढो॑ता स॒प्तहो॑ता। ऋ॒तव॑ संवत्स॒रः॥१५॥

%2.2.3.3
प्र॒जाः प॒शव॑ इ॒मे लो॒काः। य ए॒वं प्र॒जाप॑तिं ब॒होर्भूयासं॒ वेद॑। ब॒होरे॒व भूयान्भवति। प्र॒जाप॑तिर्देवासु॒रान॑सृजत। स इन्द्र॒मपि॒ नासृ॑जत। तन्दे॒वा अ॑ब्रुवन्। इन्द्र॑न्नो जन॒येति॑। सोऽब्रवीत्। यथा॒ऽहय्युँ॒ष्मास्तप॒साऽसृ॑क्षि। ए॒वमिन्द्रं॑ जनयध्व॒मिति॑॥१६॥

%2.2.3.4
ते तपो॑ऽतप्यन्त। त आ॒त्मन्निन्द्र॑मपश्यन्। तम॑ब्रुवन्। जाय॒स्वेति॑। सोऽब्रवीत्। किं भा॑ग॒धेय॑म॒भि ज॑निष्य॒ इति॑। ऋ॒तून्त्सं॑वत्स॒रम्। प्र॒जाः प॒शून्। इ॒माल्लोँ॒कानित्य॑ब्रुवन्। तं वै माऽऽहु॑त्या॒ प्र ज॑नय॒तेत्य॑ब्रवीत्॥१७॥

%2.2.3.5
तञ्चतु॑र्‌होत्रा॒ प्राज॑नयन्। यः का॒मये॑त वी॒रो म॒ आजा॑ये॒तेति॑। स चतु॑र्‌होतारं जुहुयात्। प्र॒जाप॑ति॒र्वै चतु॑र्होता। प्र॒जाप॑तिरे॒व भू॒त्वा प्रजा॑यते। ज॒जन॒दिन्द्र॑मिन्द्रि॒याय॒ स्वाहेति॒ ग्रहे॑ण जुहोति। आऽस्य॑ वी॒रो जा॑यते। वी॒र हि दे॒वा ए॒तयाऽऽहु॑त्या॒ प्राज॑नयन्। आ॒दि॒त्याश्चाङ्गि॑रसश्च सुव॒र्गे लो॒केऽस्पर्धन्त। व॒यं पूर्वे॑ सुव॒र्गं लो॒कमि॑याम व॒यं पूर्व॒ इति॑॥१८॥

%2.2.3.6
त आ॑दि॒त्या ए॒तं पञ्च॑होतारमपश्यन्। तं पु॒रा प्रा॑तरनुवा॒कादाग्नीध्रेऽजुहवुः। ततो॒ वै ते पूर्वे॑ सुव॒र्गं लो॒कमा॑यन्। यः सु॑व॒र्गका॑म॒ स्यात्। स पञ्च॑होतारं पु॒रा प्रा॑तरनुवा॒कादाग्नीध्रे जुहुयात्। सं॒व॒त्स॒रो वै पञ्च॑होता। सं॒व॒त्स॒रः सु॑व॒र्गो लो॒कः। सं॒व॒त्स॒र ए॒वर्तुषु॑ प्रति॒ष्ठाय॑। सु॒व॒र्गं लो॒कमे॑ति। तेऽब्रुव॒न्नङ्गि॑रस आदि॒त्यान्॥१९॥

%2.2.3.7
क्व॑ स्थ। क्व॑ वः स॒द्भ्यो ह॒व्यं व॑क्ष्याम॒ इति॑। छन्द॒ स्वित्य॑ब्रुवन्। गा॒य॒त्रि॒यान्त्रि॒ष्टुभि॒ जग॑त्या॒मिति॑। तस्मा॒च्छन्द॑ सु स॒द्भ्य आ॑दि॒त्येभ्य॑। आ॒ङ्गी॒र॒सीः प्र॒जा ह॒व्यं व॑हन्ति। वह॑न्त्यस्मै प्र॒जा ब॒लिम्। ऐन॒मप्र॑तिख्यातं गच्छति। य ए॒वं वेद॑। द्वाद॑श॒ मासा॒ पञ्च॒र्तव॑। त्रय॑ इ॒मे लो॒काः। अ॒सावा॑दि॒त्य ए॑कवि॒शः। ए॒तस्मि॒न्वा ए॒ष श्रि॒तः। ए॒तस्मि॒न्प्रति॑ष्ठितः। य ए॒वमे॒त श्रि॒तं प्रति॑ष्ठितं॒ वेद॑। प्रत्ये॒व ति॑ष्ठति॥२०॥\anuvakamend[स्या॒दिति॑ संवत्स॒रो ज॑नयध्व॒मितीत्य॑ब्रवी॒त्पूर्व॒ इत्या॑दि॒त्यानृ॒तव॒ष्षट्च॑]

%2.2.4.1
प्र॒जाप॑तिरकामयत॒ प्र जा॑ये॒येति॑। स ए॒तन्दश॑होतारमपश्यत्। तेन॑ दश॒धाऽऽत्मानं॑ वि॒धाय॑। दश॑होत्राऽतप्यत। तस्य॒ चित्ति॒ स्रुगासीत्। चि॒त्तमाज्यम्। तस्यै॒ताव॑त्ये॒व वागासीत्। ए॒तावान्॑ यज्ञक्र॒तुः। स चतु॑र्‌होतारमसृजत। सो॑ऽनन्दत्॥२१॥

%2.2.4.2
असृ॑क्षि॒ वा इ॒ममिति॑। तस्य॒ सामो॑ ह॒विरासीत्। स चतु॑र्‌होत्राऽतप्यत। सो॑ऽताम्यत्। स भूरिति॒ व्याह॑रत्। स भूमि॑मसृजत। अ॒ग्नि॒हो॒त्रन्द॑र्‌शपूर्णमा॒सौ यजूषि। स द्वि॒तीय॑मतप्यत। सो॑ऽताम्यत्। स भुव॒ इति॒ व्याह॑रत्॥२२॥

%2.2.4.3
सोऽन्तरि॑क्षमसृजत। चा॒तु॒र्मा॒स्यानि॒ सामा॑नि। स तृ॒तीय॑मतप्यत। सो॑ऽताम्यत्। स सुव॒रिति॒ व्याह॑रत्। स दिव॑मसृजत। अ॒ग्नि॒ष्टो॒ममु॒क्थ्य॑मतिरा॒त्रमृच॑। ए॒ता वै व्याहृ॑तय इ॒मे लो॒काः। इ॒मान्खलु॒ वै लो॒काननु॑ प्र॒जाः प॒शव॒श्छन्दासि॒ प्राजा॑यन्त। य ए॒वमे॒ताः प्र॒जाप॑तेः प्रथ॒मा व्याहृ॑ती॒ प्रजा॑ता॒ वेद॑॥२३॥

%2.2.4.4
प्र प्र॒जया॑ प॒शुभि॑र्मिथु॒नैर्जा॑यते। स पञ्च॑होतारमसृजत। स ह॒विर्नावि॑न्दत। तस्मै॒ सोम॑स्त॒नुवं॒ प्राय॑च्छत्। ए॒तत्ते॑ ह॒विरिति॑। स पञ्च॑होत्राऽतप्यत। सो॑ऽताम्यत्। स प्र॒त्यङ्ङ॑बाधत। सोऽसु॑रानसृजत। तद॒स्याप्रि॑यमासीत्॥२४॥

%2.2.4.5
तद्दु॒र्वर्ण॒ हिर॑ण्यमभवत्। तद्दु॒र्वर्ण॑स्य॒ हिर॑ण्यस्य॒ जन्म॑। स द्वि॒तीय॑मतप्यत। सो॑ऽताम्यत्। स प्राङ॑बाधत। स दे॒वान॑सृजत। तद॑स्य प्रि॒यमा॑सीत्। तत्सु॒वर्ण॒ हिर॑ण्यमभवत्। तत्सु॒वर्ण॑स्य॒ हिर॑ण्यस्य॒ जन्म॑। य ए॒व सु॒वर्ण॑स्य॒ हिर॑ण्यस्य॒ जन्म॒ वेद॑॥२५॥

%2.2.4.6
सु॒वर्ण॑ आ॒त्मना॑ भवति। दु॒र्वर्णोऽस्य॒ भ्रातृ॑व्यः। तस्मात्सु॒वर्ण॒ हिर॑ण्यं भा॒र्यम्। सु॒वर्ण॑ ए॒व भ॑वति। ऐनं॑ प्रि॒यङ्ग॑च्छति॒ नाप्रि॑यम्। स स॒प्तहो॑तारमसृजत। स स॒प्तहोत्रै॒व सु॑व॒र्गं लो॒कमैत्। त्रि॒ण॒वेन॒ स्तोमे॑नै॒भ्यो लो॒केभ्योऽसु॑रा॒न्प्राणु॑दत। त्र॒य॒स्त्रि॒शेन॒ प्रत्य॑तिष्ठत्। ए॒क॒वि॒शेन॒ रुच॑मधत्त॥२६॥

%2.2.4.7
स॒प्त॒द॒शेन॒ प्राजा॑यत। य ए॒वं वि॒द्वान्त्सोमे॑न॒ यज॑ते। स॒प्तहोत्रै॒व सु॑व॒र्गं लो॒कमे॑ति। त्रि॒ण॒वेन॒ स्तोमे॑नै॒भ्यो लो॒केभ्यो॒ भ्रातृ॑व्या॒न्प्रणु॑दते। त्र॒य॒स्त्रि॒शेन॒ प्रति॑तिष्ठति। ए॒क॒वि॒शेन॒ रुच॑न्धत्ते। स॒प्त॒द॒शेन॒ प्र जा॑यते। तस्मात्सप्तद॒शः स्तोमो॒ न नि॒र्॒हृत्य॑। प्र॒जाप॑ति॒र्वै स॑प्तद॒शः। प्र॒जाप॑तिमे॒व म॑ध्य॒तो ध॑त्ते॒ प्रजात्यै॥२७॥\anuvakamend[अ॒न॒न्द॒द्भुव॒ इति॒ व्याह॑र॒द्वेदा॑सी॒द्वेदा॑धत्त॒ प्रजात्यै]

%2.2.5.1
दे॒वा वै वरु॑णमयाजयन्। स यस्यै॑यस्यै दे॒वता॑यै॒ दक्षि॑णा॒मन॑यत्। ताम॑व्लीनात्। तेऽब्रुवन्। व्या॒वृत्य॒ प्रति॑ गृह्णाम। तथा॑ नो॒ दक्षि॑णा॒ न व्लेष्य॒तीति॑। ते व्या॒वृत्य॒ प्रत्य॑गृह्णन्। ततो॒ वै तान्दक्षि॑णा॒ नाव्ली॑नात्। य ए॒वं वि॒द्वान्व्या॒वृत्य॒ दक्षि॑णां प्रतिगृ॒ह्णाति॑। नैन॒न्दक्षि॑णा व्लीनाति॥२८॥

%2.2.5.2
राजा त्वा॒ वरु॑णो नयतु देवि दक्षिणे॒ऽग्नये॒ हिर॑ण्य॒मित्या॑ह। आ॒ग्ने॒यं वै हिर॑ण्यम्। स्वयै॒वैन॑द्दे॒वत॑या॒ प्रति॑ गृह्णाति। सोमा॑य॒ वास॒ इत्या॑ह। सौ॒म्यं वै वास॑। स्वयै॒वैन॑द्दे॒वत॑या॒ प्रति॑ गृह्णाति। रु॒द्राय॒ गामित्या॑ह। रौ॒द्री वै गौः। स्वयै॒वैनान्दे॒वत॑या॒ प्रति॑गृह्णाति। वरु॑णा॒याश्व॒मित्या॑ह॥२९॥

%2.2.5.3
वा॒रु॒णो वा अश्व॑। स्वयै॒वैनं॑ दे॒वत॑या॒ प्रति॑गृह्णाति। प्र॒जाप॑तये॒ पुरु॑ष॒मित्या॑ह। प्रा॒जा॒प॒त्यो वै पुरु॑षः। स्वयै॒वैनं॑ दे॒वत॑या॒ प्रति॑ गृह्णाति। मन॑वे॒ तल्प॒मित्या॑ह। मा॒न॒वो वै तल्प॑। स्वयै॒वैनं॑ दे॒वत॑या॒ प्रति॑ गृह्णाति। उ॒त्ता॒नायाङ्गीर॒सायान॒ इत्या॑ह। इ॒यं वा उ॑त्ता॒न आङ्गीर॒सः॥३०॥

%2.2.5.4
अ॒नयै॒वैन॒त्प्रति॑ गृह्णाति। वै॒श्वा॒न॒र्यर्चा रथं॒ प्रति॑ गृह्णाति। वै॒श्वा॒न॒रो वै दे॒वत॑या॒ रथ॑। स्वयै॒वैनं॑ दे॒वत॑या॒ प्रति॑ गृह्णाति। तेना॑मृत॒त्वम॑श्या॒मित्या॑ह। अ॒मृत॑मे॒वात्मन्ध॑त्ते। वयो॑ दा॒त्र इत्या॑ह। वय॑ ए॒वैनं॑ कृ॒त्वा। सु॒व॒र्गं लो॒कं ग॑मयति। मयो॒ मह्य॑मस्तु प्रतिग्रही॒त्र इत्या॑ह॥३१॥

%2.2.5.5
यद्वै शि॒वम्। तन्मय॑। आ॒त्मन॑ ए॒वैषा परीत्तिः। क इ॒दङ्कस्मा॑ अदा॒दित्या॑ह। प्र॒जाप॑ति॒र्वै कः। स प्र॒जाप॑तये ददाति। काम॒ कामा॒येत्या॑ह। कामे॑न॒ हि ददा॑ति। कामे॑न प्रतिगृ॒ह्णाति॑। कामो॑ दा॒ता काम॑ प्रतिग्रही॒तेत्या॑ह॥३२॥

%2.2.5.6
कामो॒ हि दा॒ता। काम॑ प्रतिग्रही॒ता। काम समु॒द्रमावि॒शेत्या॑ह। स॒मु॒द्र इ॑व॒ हि काम॑। नेव॒ हि काम॒स्यान्तोऽस्ति॑। न स॑मु॒द्रस्य॑। कामे॑न त्वा॒ प्रति॑गृह्णा॒मीत्या॑ह। येन॒ कामे॑न प्रतिगृ॒ह्णाति॑। स ए॒वैन॑म॒मुष्मि॑ल्लोँ॒के काम॒ आग॑च्छति। कामै॒तत्त॑ ए॒षा ते॑ काम॒ दक्षि॒णेत्या॑ह। काम॑ ए॒व तद्यज॑मानो॒ऽमुष्मि॑ल्लोँ॒के दक्षि॑णामिच्छति। न प्र॑तिग्रही॒तरि॑। य ए॒वं वि॒द्वान्दक्षि॑णां प्रतिगृ॒ह्णाति॑। अ॒नृ॒णामे॒वैनां॒ प्रति॑ गृह्णाति॥३३॥\anuvakamend[व्ली॒ना॒त्यश्व॒मित्या॑हाङ्गीर॒सः प्र॑तिग्रही॒त्र इत्या॑ह प्रतिग्रही॒तेत्या॑ह॒ दक्षि॒णेत्या॑ह च॒त्वारि॑ च]

%2.2.6.1
अन्तो॒ वा ए॒ष य॒ज्ञस्य॑। यद्द॑श॒ममह॑। द॒श॒मेऽहन्त्सर्परा॒ज्ञिया॑ ऋ॒ग्भिः स्तु॑वन्ति। य॒ज्ञस्यै॒वान्त॑ङ्ग॒त्वा। अ॒न्नाद्य॒मव॑ रुन्धते। ति॒सृभि॑ स्तुवन्ति। त्रय॑ इ॒मे लो॒काः। ए॒भ्य ए॒व लो॒केभ्यो॒ऽन्नाद्य॒मव॑ रुन्धते। पृश्ञि॑वतीर्भवन्ति। अन्नं॒ वै पृश्ञि॑॥३४॥

%2.2.6.2
अन्न॑मे॒वाव॑ रुन्धते। मन॑सा॒ प्रस्तौ॑ति। मन॒सोद्गा॑यति। मन॑सा॒ प्रति॑ हरति। मन॑ इव॒ हि प्र॒जाप॑तिः। प्र॒जाप॑ते॒राप्त्यै। दे॒वा वै स॒र्पाः। तेषा॑मि॒य राज्ञी। यत्स॑र्परा॒ज्ञिया॑ ऋ॒ग्भिः स्तु॒वन्ति॑। अ॒स्यामे॒व प्रति॑ तिष्ठन्ति॥३५॥

%2.2.6.3
चतु॑र्‌होतॄ॒न्॒ होता॒ व्याच॑ष्टे। स्तु॒तमनु॑शसति॒ शान्त्यै। अन्तो॒ वा ए॒ष य॒ज्ञस्य॑। यद्द॑श॒ममह॑। ए॒तत्खलु॒ वै दे॒वानां पर॒मङ्गुह्यं॒ ब्रह्म॑। यच्चतु॑र्होतारः। द॒श॒मेऽह॒ श्चतु॑र्‌होतॄ॒न्व्याच॑ष्टे। य॒ज्ञस्यै॒वान्त॑ङ्ग॒त्वा। प॒र॒मन्दे॒वाना॒ङ्गुह्यं॒ ब्रह्माव॑ रुन्धे। तदे॒व प्र॑का॒शं ग॑मयति॥३६॥

%2.2.6.4
तदे॑नं प्रका॒शङ्ग॒तम्। प्र॒का॒शं प्र॒जानाङ्गमयति। वाचं॑ यच्छति। य॒ज्ञस्य॒ धृत्यै। य॒ज॒मा॒न॒दे॒व॒त्यं॑ वा अह॑। भ्रा॒तृ॒व्य॒दे॒व॒त्या॑ रात्रि॑। अह्ना॒ रात्रि॑न्ध्यायेत्। भ्रातृ॑व्यस्यै॒व तल्लो॒कं वृ॑ङ्क्ते। यद्दिवा॒ वाचं॑ विसृ॒जेत्। अह॒र्भ्रातृ॑व्या॒योच्छिषेत्। यन्नक्तं॑ विसृ॒जेत्। रात्रिं॒ भ्रातृ॑व्या॒योच्छिषेत्। अ॒धि॒वृ॒क्ष॒सू॒र्ये वाचं॒ विसृ॑जति। ए॒ताव॑न्तमे॒वास्मै॑ लो॒कमुच्छिषति। याव॑दादि॒त्योऽस्त॒मेति॑॥३७॥\anuvakamend[पृश्ञि॑ तिष्ठन्ति गमयति शिषे॒त्पञ्च॑ च]

%2.2.7.1
प्र॒जाप॑तिः प्र॒जा अ॑सृजत। ताः सृ॒ष्टाः सम॑श्लिष्यन्। ता रू॒पेणानु॒प्रावि॑शत्। तस्मा॑दाहुः। रू॒पं वै प्र॒जाप॑ति॒रिति॑। ता नाम्नाऽनु॒ प्रावि॑शत्। तस्मा॑दाहुः। नाम॒ वै प्र॒जाप॑ति॒रिति॑। तस्मा॒दप्या॑मि॒त्रौ स॒ङ्गत्य॑। नाम्ना॒ चेद्ध्वये॑ते॥३८॥

%2.2.7.2
मि॒त्रमे॒व भ॑वतः। प्र॒जाप॑तिर्देवासु॒रान॑सृजत। स इन्द्र॒मपि॒ नासृ॑जत। तन्दे॒वा अ॑ब्रुवन्। इन्द्र॑न्नो जन॒येति॑। स आ॒त्मन्निन्द्र॑मपश्यत्। तम॑सृजत। तन्त्रि॒ष्टुग्वी॒र्यं॑ भू॒त्वाऽनु॒ प्रावि॑शत्। तस्य॒ वज्र॑ पञ्चद॒शो हस्त॒ आप॑द्यत। तेनो॒दय्यासु॑रान॒भ्य॑भवत्॥३९॥

%2.2.7.3
य ए॒वं वेद॑। अ॒भि भ्रातृ॑व्यान्भवति। ते दे॒वा असु॑रैर्वि॒जित्य॑। सु॒व॒र्गं लो॒कमा॑यन्। ते॑ऽमुष्मि॑ल्लोँ॒के व्य॑क्षुध्यन्। तेऽब्रुवन्। अ॒मुत॑ प्रदानं॒ वा उप॑जिजीवि॒मेति॑। ते स॒प्तहो॑तारं य॒ज्ञं वि॒धाया॒यास्यम्। आ॒ङ्गी॒र॒सं प्राहि॑ण्वन्। ए॒तेना॒मुत्र॑ कल्प॒येति॑॥४०॥

%2.2.7.4
तस्य॒ वा इ॒यङ्कॢप्ति॑। यदि॒दङ्किं च॑। य ए॒वं वेद॑। कल्प॑तेऽस्मै। स वा अ॒यं म॑नु॒ष्ये॑षु य॒ज्ञः स॒प्तहो॑ता। अ॒मुत्र॑ स॒द्भ्यो दे॒वेभ्यो॑ ह॒व्यं व॑हति। य ए॒वं वेद॑। उपै॑नं य॒ज्ञो न॑मति। सो॑ऽमन्यत। अ॒भि वा इ॒मेऽस्माल्लो॒काद॒मुं लो॒कङ्क॑मिष्यन्त॒ इति॑। स वाच॑स्पते॒ हृदिति॒ व्याह॑रत्। तस्मात्पु॒त्रो हृद॑यम्। तस्मा॑द॒स्माल्लो॒काद॒मुं लो॒कन्नाभि का॑मयन्ते। पु॒त्रो हि हृद॑यम्॥४१॥\anuvakamend[ह्वये॑ते अभवत्कल्प॒येतीति॑ च॒त्वारि॑ च]

%2.2.8.1
दे॒वा वै चतु॑र्‌होतृभिर्य॒ज्ञम॑तन्वत। ते वि पा॒प्मना॒ भ्रातृ॑व्ये॒णाज॑यन्त। अ॒भि सु॑व॒र्गं लो॒कम॑जयन्। य ए॒वं वि॒द्वाश्चतु॑र्होतृभिर्य॒ज्ञन्त॑नु॒ते। वि पा॒प्मना॒ भ्रातृ॑व्येण जयते। अ॒भि सु॑व॒र्गं लो॒कं ज॑यति। षड्ढोत्रा प्राय॒णीय॒मा सा॑दयति। अ॒मुष्मै॒ वै लो॒काय॒ षड्ढो॑ता। घ्नन्ति॒ खलु॒ वा ए॒तत्सोमम्। यद॑भिषु॒ण्वन्ति॑॥४२॥

%2.2.8.2
ऋ॒जु॒धैवैन॑म॒मुं लो॒कं ग॑मयति। चतु॑र्‌होत्राऽऽति॒थ्यम्। यशो॒ वै चतु॑र्‌होता। यश॑ ए॒वात्मन्ध॑त्ते। पञ्च॑होत्रा प॒शुमुप॑सादयति। सु॒व॒र्ग्यो॑ वै पञ्च॑होता। यज॑मानः प॒शुः। यज॑मानमे॒व सु॑व॒र्गं लो॒कं ग॑मयति। ग्रहान्गृही॒त्वा स॒प्तहो॑तारं जुहोति। इ॒न्द्रि॒यं वै स॒प्तहो॑ता॥४३॥

%2.2.8.3
इ॒न्द्रि॒यमे॒वात्मन्ध॑त्ते। यो वै चतु॑र्‌होतॄननुसव॒नन्त॒र्पय॑ति। तृप्य॑ति प्र॒जया॑ प॒शुभि॑। उपै॑न सोमपी॒थो न॑मति। ब॒हि॒ष्प॒व॒मा॒ने दश॑होतारं॒ व्याच॑क्षीत। माध्य॑न्दिने॒ पव॑माने॒ चतु॑र्‌होतारम्। आर्भ॑वे॒ पव॑माने॒ पञ्च॑होतारम्। पि॒तृ॒य॒ज्ञे षड्ढो॑तारम्। य॒ज्ञा॒य॒ज्ञिय॑स्य स्तो॒त्रे स॒प्तहो॑तारम्। अ॒नु॒स॒व॒नमे॒वैनास्तर्पयति॥४४॥

%2.2.8.4
तृप्य॑ति प्र॒जया॑ प॒शुभि॑। उपै॑न सोमपी॒थो न॑मति। दे॒वा वै चतु॑र्‌होतृभिः स॒त्रमा॑सत। ऋद्धि॑परिमितं॒ यश॑स्कामाः। तेऽब्रुवन्। यन्न॑ प्रथ॒मं यश॑ ऋ॒च्छात्। सर्वे॑षान्न॒स्तत्स॒हास॒दिति॑। सोम॒श्चतु॑र्‌होत्रा। अ॒ग्निः पञ्च॑होत्रा। धा॒ता षड्ढोत्रा॥४५॥

%2.2.8.5
इन्द्र॑ स॒प्तहोत्रा। प्र॒जाप॑ति॒र्दश॑होत्रा। तेषा॒ सोम॒ राजा॑नं॒ यश॑ आर्च्छत्। तन्न्य॑कामयत। तेनापाक्रामत्। तेन॑ प्र॒लाय॑मचरत्। तन्दे॒वाः प्रै॒षैः प्रैष॑मैच्छन्। तत्प्रै॒षाणां प्रैष॒त्वम्। नि॒विद्भि॒र्न्य॑वेदयन्। तन्नि॒विदान्निवि॒त्त्वम्॥४६॥

%2.2.8.6
आ॒प्रीभि॑राप्नुवन्। तदा॒प्रीणा॑माप्रि॒त्वम्। तम॑घ्नन्। तस्य॒ यशो॒ व्य॑गृह्णत। ते ग्रहा॑ अभवन्। तद्ग्रहा॑णाङ्ग्रह॒त्वम्। यस्यै॒वं वि॒दुषो॒ ग्रहा॑ गृ॒ह्यन्ते। तस्य॒ त्वे॑व गृ॑ही॒ताः। तेऽब्रुवन्। यो वै न॒ श्रेष्ठोऽभूत्॥४७॥

%2.2.8.7
तम॑वधिष्म। पुन॑रि॒म सु॑वामहा॒ इति॑। तञ्छन्दो॑भिरसुवन्त। तच्छन्द॑साञ्छन्द॒स्त्वम्। साम्ना॒ समान॑यन्। तत्साम्न॑ साम॒त्वम्। उ॒क्थैरुद॑स्थापयन्। तदु॒क्थाना॑मुक्थ॒त्वम्। य ए॒वं वेद॑। प्रत्ये॒व ति॑ष्ठति॥४८॥

%2.2.8.8
सर्व॒मायु॑रेति। सोमो॒ वै यश॑। य ए॒वं वि॒द्वान्त्सोम॑मा॒गच्छ॑ति। यश॑ ए॒वैन॑मृच्छति। तस्मा॑दाहुः। यश्चै॒वं वेद॒ यश्च॒ न। तावु॒भौ सोम॒माग॑च्छतः। सोमो॒ हि यश॑। तन्त्वाऽव यश॑ ऋच्छ॒तीत्या॑हुः। यः सोमे॒ सोमं॒ प्राहेति॑। तस्मा॒त्सोमे॒ सोम॒ प्रोच्य॑। यश॑ ए॒वैन॑मृच्छति॥४९॥\anuvakamend[अ॒भि॒षु॒ण्वन्ति॑ स॒प्तहो॑ता तर्पयति॒ षड्ढोत्रा निवि॒त्त्वमभूत्तिष्ठति॒ प्राहेति॒ द्वे च॑]

%2.2.9.1
इ॒दं वा अग्रे॒ नैव किं च॒ नासीत्। न द्यौरा॑सीत्। न पृ॑थि॒वी। नान्तरि॑क्षम्। तदस॑दे॒व सन्मनो॑ऽकुरुत॒ स्यामिति॑। तद॑तप्यत। तस्मात्तेपा॒नाद्धू॒मो॑ऽजायत। तद्भूयो॑ऽतप्यत। तस्मात्तेपा॒नाद॒ग्निर॑जायत। तद्भूयो॑ऽतप्यत॥५०॥

%2.2.9.2
तस्मात्तेपा॒नाज्ज्योति॑रजायत। तद्भूयो॑ऽतप्यत। तन्मात्तेपा॒नाद॒र्चिर॑जायत। तद्भूयो॑ऽतप्यत। तस्मात्तेपा॒नान्मरी॑चयोऽजायन्त। तद्भूयो॑ऽतप्यत। तस्मात्तेपा॒नादु॑दा॒रा अ॑जायन्त। तद्भूयो॑ऽतप्यत। तद॒भ्रमि॑व॒ सम॑हन्यत। तद्व॒स्तिम॑भिनत्॥५१॥

%2.2.9.3
स स॑मु॒द्रो॑ऽभवत्। तस्मात्समु॒द्रस्य॒ न पि॑बन्ति। प्र॒जन॑नमिव॒ हि मन्य॑न्ते। तस्मात्प॒शोर्जाय॑माना॒दाप॑ पु॒रस्ताद्यन्ति। तद्दश॑हो॒ताऽन्व॑सृज्यत। प्र॒जाप॑ति॒र्वै दश॑होता। य ए॒वन्तप॑सो वी॒र्य॑ं वि॒द्वा स्तप्य॑ते। भव॑त्ये॒व। तद्वा इ॒दमाप॑ सलि॒लमा॑सीत्। सो॑ऽरोदीत्प्र॒जाप॑तिः॥५२॥

%2.2.9.4
स कस्मा॑ अज्ञि। यद्य॒स्या अप्र॑तिष्ठाया॒ इति॑। यद॒प्स्व॑वाप॑द्यत। सा पृ॑थि॒व्य॑भवत्। यद्व्यमृ॑ष्ट। तद॒न्तरि॑क्षमभवत्। यदू॒र्ध्वमु॒दमृ॑ष्ट। सा द्यौर॑भवत्। यदरो॑दीत्। तद॒नयो॑ रोद॒स्त्वम्॥५३॥

%2.2.9.5
य ए॒वं वेद॑। नास्य॑ गृ॒हे रु॑दन्ति। ए॒तद्वा ए॒षां लो॒कानां॒ जन्म॑। य ए॒वमे॒षां लो॒कानां॒ जन्म॒ वेद॑। नैषु लो॒केष्वार्ति॒मार्च्छ॑ति। स इ॒मां प्र॑ति॒ष्ठाम॑विन्दत। स इ॒मां प्र॑ति॒ष्ठां वि॒त्वाऽका॑मयत॒ प्रजा॑ये॒येति॑। स तपो॑ऽतप्यत। सोऽन्तर्वा॑नभवत्। स ज॒घना॒दसु॑रानसृजत॥५४॥

%2.2.9.6
तेभ्यो॑ मृ॒न्मये॒ पात्रेऽन्न॑मदुहत्। याऽस्य॒ सा त॒नूरासीत्। तामपा॑हत। सा तमि॑स्राऽभवत्। सो॑कामयत॒ प्रजा॑ये॒येति॑। स तपो॑ऽतप्यत। सोन्तर्वा॑नभवत्। स प्र॒जन॑नादे॒व प्र॒जा अ॑सृजत। तस्मा॑दि॒मा भूयि॑ष्ठाः। प्र॒जन॑ना॒द्ध्ये॑ना॒ असृ॑जत॥५५॥

%2.2.9.7
ताभ्यो॑ दारु॒मये॒ पात्रे॒ पयो॑ऽदुहत्। याऽस्य॒ सा त॒नूरासीत्। तामपा॑हत। सा जोत्स्ना॑ऽभवत्। सो॑ऽकामयत॒ प्रजा॑ये॒येति॑। स तपो॑ऽतप्यत। सोऽन्तर्वा॑नभवत्। स उ॑पप॒क्षाभ्या॑मे॒वर्तून॑सृजत। तेभ्यो॑ रज॒ते पात्रे॑ घृ॒तम॑दुहत्। याऽस्य॒ सा त॒नूरासीत्॥५६॥

%2.2.9.8
तामपा॑हत। सो॑ऽहोरा॒त्रयो स॒न्धिर॑भवत्। सो॑ऽकामयत॒ प्रजा॑ये॒येति॑। स तपो॑ऽतप्यत। सोऽन्तर्वा॑नभवत्। स मुखाद्दे॒वान॑सृजत। तेभ्यो॒ हरि॑ते॒ पात्रे॒ सोम॑मदुहत्। याऽस्य॒ सा त॒नूरासीत्। तामपा॑हत। तदह॑रभवत्॥५७॥

%2.2.9.9
ए॒ते वै प्र॒जाप॑ते॒र्दोहा। य ए॒वं वेद॑। दु॒ह ए॒व प्र॒जाः। दिवा॒ वै नो॑ऽभू॒दिति॑। तद्दे॒वानान्देव॒त्वम्। य ए॒वन्दे॒वानान्देव॒त्वं वेद॑। दे॒ववा॑ने॒व भ॑वति। ए॒तद्वा अ॑होरा॒त्राणां॒ जन्म॑। य ए॒वम॑होरा॒त्राणां॒ जन्म॒ वेद॑। नाहो॑रा॒त्रेष्वार्ति॒मार्च्छ॑ति॥५८॥

%2.2.9.10
अस॒तोऽधि॒ मनो॑ऽसृज्यत। मन॑ प्र॒जाप॑तिमसृजत। प्र॒जाप॑तिः प्र॒जा अ॑सृजत। तद्वा इ॒दं मन॑स्ये॒व प॑र॒मं प्रति॑ष्ठितम्। यदि॒दङ्किं च॑। तदे॒तच्छ्वो॑वस्य॒सन्नाम॒ ब्रह्म॑। व्यु॒च्छन्तीव्युच्छन्त्यस्मै॒ वस्य॑सीवस्यसी॒ व्यु॑च्छति। प्रजा॑यते प्र॒जया॑ प॒शुभि॑। प्र प॑रमे॒ष्ठिनो॒ मात्रा॑माप्नोति। य ए॒वं वेद॑॥५९॥\anuvakamend[अ॒ग्निर॑जायत॒ तद्भूयो॑ऽतप्यताभिनदरोदीत्प्र॒जाप॑तीरोद॒स्त्वम॑सृज॒तासृ॑जत घृ॒तम॑दुह॒द्याऽस्य॒ सा त॒नूरासी॒दह॑रभवदृच्छति॒ वेद॑ (इ॒दं धू॒मोऽग्निर्ज्योति॑र॒र्चिर्मरी॑चय उदा॒रास्तद॒ब्भ्र स ज॒घना॒त्सा तमि॑स्रा॒ स प्र॒जन॑ना॒त्सा जोत्स्ना॒ स उ॑पप॒क्षाभ्या॒ सो॑ऽहोरा॒त्रयो स॒न्धिः स मुखा॒त्तदह॑र्दे॒ववान्मृ॒न्मये॑ दारु॒मये॑ रज॒ते हरि॑ते॒ तेभ्य॒स्ताभ्यो॒ द्वे तेऽन्नं॒ पयो॑ घृ॒त सोमम् ॥ )]

%2.2.10.1
प्र॒जाप॑ति॒रिन्द्र॑मसृजतानुजाव॒रन्दे॒वानाम्। तं प्राहि॑णोत्। परे॑हि। ए॒तेषान्दे॒वाना॒मधि॑पतिरे॒धीति॑। तन्दे॒वा अ॑ब्रुवन्। कस्त्वमसि॑। व॒यं वै त्वच्छ्रेयासः स्म॒ इति॑। सोऽब्रवीत्। कस्त्वमसि॑ व॒यं वै त्वच्छ्रेयासः स्म॒ इति॑ मा दे॒वा अ॑वोचं॒ निति॑। अथ॒ वा इ॒दन्तर्‌हि॑ प्र॒जाप॑तौ॒ हर॑ आसीत्॥६०॥

%2.2.10.2
यद॒स्मिन्ना॑दि॒त्ये। तदे॑नमब्रवीत्। ए॒तन्मे॒ प्रय॑च्छ। अथा॒हमे॒तेषान्दे॒वाना॒मधि॑पतिर्भविष्या॒मीति॑। को॑ऽह स्या॒मित्य॑ब्रवीत्। ए॒तत्प्र॒दायेति॑। ए॒तत्स्या॒ इत्य॑ब्रवीत्। यदे॒तद्ब्रवी॒षीति॑। को ह॒ वै नाम॑ प्र॒जाप॑तिः। य ए॒वं वेद॑॥६१॥

%2.2.10.3
वि॒दुरे॑न॒न्नाम्ना। तद॑स्मै रु॒क्मं कृ॒त्वा प्रत्य॑मुञ्चत्। ततो॒ वा इन्द्रो॑ दे॒वाना॒मधि॑पतिरभवत्। य ए॒वं वेद॑। अधि॑पतिरे॒व स॑मा॒नानां भवति। सो॑ऽमन्यत। किङ्किं॒ वा अ॑कर॒मिति॑। स च॒न्द्रं म॒ आह॒रेति॒ प्राल॑पत्। तच्च॒न्द्रम॑सश्चन्द्रम॒स्त्वम्। य ए॒वं वेद॑॥६२॥

%2.2.10.4
च॒न्द्रवा॑ने॒व भ॑वति। तन्दे॒वा अ॑ब्रुवन्। सु॒वीर्यो॑ मर्या॒ यथा॑ गोपा॒यत॒ इति॑। तत्सूर्य॑स्य सूर्य॒त्वम्। य ए॒वं वेद॑। नैन॑न्दभ्नोति। कश्च॒ नास्मि॒न्वा इ॒दमि॑न्द्रि॒यं प्रत्य॑स्था॒दिति॑। तदिन्द्र॑स्येन्द्र॒त्वम्। य ए॒वं वेद॑। इ॒न्द्रि॒या॒व्ये॑व भ॑वति॥६३॥

%2.2.10.5
अ॒यं वा इ॒दं प॑र॒मो॑ऽभू॒दिति॑। तत्प॑रमे॒ष्ठिन॑ परमेष्ठि॒त्वम्। य ए॒वं वेद॑। प॒र॒मामे॒व काष्ठां गच्छति। तन्दे॒वाः स॑म॒न्तं पर्य॑विशन्। वस॑वः पु॒रस्तात्। रु॒द्रा द॑क्षिण॒तः। आ॒दि॒त्याः प॒श्चात्। विश्वे॑ दे॒वा उ॑त्तर॒तः। अङ्गि॑रसः प्र॒त्यञ्चम्॥६४॥

%2.2.10.6
सा॒ध्याः पराञ्चम्। य ए॒वं वेद॑। उपै॑न समा॒नाः संवि॑शन्ति। स प्र॒जाप॑तिरे॒व भू॒त्वा प्र॒जा आव॑यत्। ता अ॑स्मै॒ नाति॑ष्ठन्ता॒न्नाद्या॑य। ता मुखं॑ पु॒रस्ता॒त्पश्य॑न्तीः। द॒क्षि॒ण॒तः पर्या॑यन्। स द॑क्षिण॒तः पर्य॑वर्तयत। ता मुखं॑ पु॒रस्ता॒त्पश्य॑न्तीः। मुख॑न्दक्षिण॒तः॥६५॥

%2.2.10.7
प॒श्चात्पर्या॑यन्। स प॒श्चात्पर्य॑वर्तयत। ता मुखं॑ पु॒रस्ता॒त्पश्य॑न्तीः। मुख॑न्दक्षिण॒तः। मुखं॑ प॒श्चात्। उ॒त्त॒र॒तः पर्या॑यन्। स उ॑त्तर॒तः पर्य॑वर्तयत। ता मुखं॑ पु॒रस्ता॒त्पश्य॑न्तीः। मुख॑न्दष्खिण॒तः। मुखं॑ प॒श्चात्॥६६॥

%2.2.10.8
मुख॑मुत्तर॒तः। ऊ॒र्ध्वा उदा॑यन्। स उ॒परि॑ष्टा॒न्न्य॑वर्तयत। ताः स॒र्वतो॑मुखो भू॒त्वाऽऽव॑यत्। ततो॒ वै तस्मै प्र॒जा अति॑ष्ठन्ता॒न्नाद्या॑य। य ए॒वं वि॒द्वान्परि॑ च व॒र्तय॑ते॒ नि च॑। प्र॒जाप॑तिरे॒व भू॒त्वा प्र॒जा अ॑त्ति। तिष्ठ॑न्तेऽस्मै प्र॒जा अ॒न्नाद्या॑य। अ॒न्ना॒द ए॒व भव॑ति॥६७॥\anuvakamend[आ॒सी॒द्वेद॑ चन्द्रम॒स्त्वं य ए॒वं वेदेन्द्रिया॒व्ये॑व भ॑वति प्र॒त्यञ्चं॒ मुख॑न्दक्षिण॒तो मुखं॑ प॒श्चान्नव॑ च]

%2.2.11.1
प्र॒जाप॑तिरकामयत ब॒होर्भूयान्त्स्या॒मिति॑। स ए॒तन्दश॑होतारमपश्यत्। तं प्रायु॑ङ्क्त। तस्य॒ प्रयु॑क्ति ब॒होर्भूया॑नभवत्। यः का॒मये॑त ब॒होर्भूयान्त्स्या॒मिति॑। स दश॑होतारं॒ प्रयु॑ञ्जीत। ब॒होरे॒व भूयान्भवति। सो॑ऽकामयत वी॒रो म॒ आजा॑ये॒तेति॑। स दश॑होतु॒श्चतु॑र्‌होतारं॒ निर॑मिमीत। तं प्रायु॑ङ्क्त॥६८॥

%2.2.11.2
तस्य॒ प्रयु॒क्तीन्द्रो॑ऽजायत। यः का॒मये॑त वी॒रो म॒ आजा॑ये॒तेति॑। स चतु॑र्‌होतारं॒ प्रयु॑ञ्जीत। आऽस्य॑ वी॒रो जा॑यते। सो॑ऽकामयत पशु॒मान्त्स्या॒मिति॑। स चतु॑र्‌होतु॒ पञ्च॑होतारं॒ निर॑मिमीत। तं प्रायु॑ङ्क्त। तस्य॒ प्रयु॑क्ति पशु॒मान॑भवत्। यः का॒मये॑त पशु॒मान्त्स्या॒मिति॑। स पञ्च॑होतारं॒ प्रयु॑ञ्जीत॥६९॥

%2.2.11.3
प॒शु॒माने॒व भ॑वति। सो॑ऽकामयत॒र्तवो॑ मे कल्पेर॒न्निति॑। स पञ्च॑होतु॒ षड्ढो॑तारं॒ निर॑मिमीत। तं प्रायु॑ङ्क्त। तस्य॒ प्रयु॑क्त्यृ॒तवोऽस्मा अकल्पन्त। यः का॒मये॑त॒र्तवो॑ मे कल्पेर॒न्निति॑। स षड्ढो॑तारं॒ प्रयु॑ञ्जीत। कल्प॑न्तेऽस्मा ऋ॒तव॑। सो॑ऽकामयत सोम॒पः सो॑मया॒जी स्याम्। आ मे॑ सोम॒पः सो॑मया॒जी जा॑ये॒तेति॑॥७०॥

%2.2.11.4
स षड्ढो॑तुः स॒प्तहो॑तारं॒ निर॑मिमीत। तं प्रायु॑ङ्क्त। तस्य॒ प्रयु॑क्ति सोम॒पः सो॑मया॒ज्य॑भवत्। आऽस्य॑ सोम॒पः सो॑मया॒ज्य॑जायत। यः का॒मये॑त सोम॒पः सो॑मया॒जी स्याम्। आ मे॑ सोम॒पः सो॑मया॒जी जा॑ये॒तेति॑। स स॒प्तहो॑तारं॒ प्रयु॑ञ्जीत। सो॒म॒प ए॒व सो॑मया॒जी भ॑वति। आऽस्य॑ सोम॒पः सो॑मया॒जी जा॑यते। स वा ए॒ष प॒शुः प॑ञ्च॒धा प्रति॑तिष्ठति॥७१॥

%2.2.11.5
प॒द्भिर्मुखे॑न। ते दे॒वाः प॒शून् वि॒त्वा। सु॒व॒र्गं लो॒कमा॑यन्। ते॑ऽमुष्मि॑ल्लोँ॒के व्य॑क्षुध्यन्। तेऽब्रुवन्। अ॒मुत॑ प्रदानं॒ वा उप॑जिजीवि॒मेति॑। ते स॒प्तहो॑तारं य॒ज्ञं वि॒धाया॒यास्यम्। आ॒ङ्गी॒र॒सं प्राहि॑ण्वन्। ए॒तेना॒मुत्र॑ कल्प॒येति॑। तस्य॒ वा इ॒यङ्कॢप्ति॑॥७२॥

%2.2.11.6
यदि॒दङ्किं च॑। य ए॒वं वेद॑। कल्प॑तेऽस्मै। स वा अ॒यं म॑नु॒ष्ये॑षु य॒ज्ञः स॒प्तहो॑ता। अ॒मुत्र॑ स॒द्भ्यो दे॒वेभ्यो॑ ह॒व्यं व॑हति। य ए॒वं वेद॑। उपै॑नं य॒ज्ञो न॑मति। यो वै चतु॑र्‌होतृणान्नि॒दानं॒ वेद॑। नि॒दान॑वान्भवति। अ॒ग्नि॒हो॒त्रं वै दश॑होतुर्नि॒दानम्। द॒र्‌श॒पू॒र्ण॒मासौ चतु॑र्‌होतुः। चा॒तु॒र्मा॒स्यानि॒ पञ्च॑होतुः। प॒शु॒ब॒न्धष्षड्ढो॑तुः। सौ॒म्योऽध्व॒रः स॒प्तहो॑तुः। ए॒तद्वै चतु॑र्‌होतृणान्नि॒दानम्। य ए॒वं वेद॑। नि॒दान॑वान्भवति॥७३॥\anuvakamend[अ॒मि॒मी॒त॒ तं प्रायु॑ङ्क्त॒ पञ्च॑होतारं॒ प्र यु॑ञ्जीत जाये॒तेति॑ तिष्ठति॒ कॢप्ति॒र्दश॑होतुर्नि॒दान स॒प्त च॑]




\prashnaend{प्र॒जाप॑तिरकामयत प्र॒जाः सृ॑जे॒येति॑ प्र॒जाप॑तिरकामयत दर्‌शपूर्णमा॒सौ सृ॑जे॒येति॑ प्र॒जाप॑तिरकामयत॒ प्रजा॑ये॒येति॒ स तप॒ स त्रि॒वृतं॑ प्र॒जाप॑तिरकामयत॒ दश॑होतारं॒ तेन॑ दश॒धाऽऽत्मानं॑ दे॒वा वै वरु॑ण॒मन्तो॒ वै प्र॒जाप॑ति॒स्ताः सृ॒ष्टाः सम॑श्लिष्यन्दे॒वा वै चतु॑र्‌होतृभिरि॒दं वा अग्रे प्र॒जाप॑ति॒रिन्द्रं॑ प्र॒जाप॑तिरकामयत ब॒होर्भूया॒नेका॑दश॥११॥}{प्र॒जाप॑ति॒स्तद्ग्रह॑स्य प्र॒जाप॑तिरकामयता॒नयै॒वैन॒त्तस्य॒ वा इ॒यं कॢप्ति॒स्तस्मात्तेपा॒नाज्ज्योति॒र्यद॒स्मिन्ना॑दि॒त्ये स षड्ढो॑तुः स॒प्तहो॑तार॒न्त्रिस॑प्ततिः॥७३॥}{प्र॒जाप॑तिरकामयत नि॒दान॑वान्भवति॥}{हरि॑ ओम्॥}{इति श्रीकृष्णयजुर्वेदीयतैत्तिरीयब्राह्मणे द्वितीयाष्टके द्वितीयः प्रपाठकः समाप्तः॥}
\clearpage
\sect{तृतीयः प्रश्नः}
\setcounter{anuvakam}{0}
\dnsub{तैत्तिरीयब्राह्मणे द्वितीयाष्टके तृतीयः प्रपाठकः}

%2.3.1.1
ब्र॒ह्म॒वा॒दिनो॑ वदन्ति। किञ्चतु॑र्‌होतृणाञ्चतुर्‌होतृ॒त्वमिति॑। यदे॒वैषु च॑तु॒र्धा होता॑रः। तेन॒ चतु॑र्‌होतारः। तस्मा॒च्चतु॑र्‌होतार उच्यन्ते। तच्चतुर्॑होतृणाञ्चतुर्‌होतृ॒त्वम्। सोमो॒ वै चतु॑र्‌होता। अ॒ग्निः पञ्च॑होता। धा॒ता षड्ढो॑ता। इन्द्र॑ स॒प्तहो॑ता॥१॥

%2.3.1.2
प्र॒जाप॑ति॒र्दश॑होता। य ए॒वञ्चतु॑र्‌होतृणा॒मृद्धिं॒ वेद॑। ऋ॒ध्नोत्ये॒व। य ए॑षामे॒वं ब॒न्धुतां॒ वेद॑। बन्धु॑मान्भवति। य ए॑षामे॒वं कॢप्तिं॒ वेद॑। कल्प॑तेऽस्मै। य ए॑षामे॒वमा॒यत॑नं॒ वेद॑। आ॒यत॑नवान्भवति। य ए॑षामे॒वं प्र॑ति॒ष्ठां वेद॑॥२॥

%2.3.1.3
प्रत्ये॒व ति॑ष्ठति। ब्र॒ह्म॒वा॒दिनो॑ वदन्ति। दश॑होता॒ चतु॑र्‌होता। पञ्च॑होता॒ षड्ढो॑ता स॒प्तहो॑ता। अथ॒ कस्मा॒च्चतु॑र्‌होतार उच्यन्त॒ इति॑। इन्द्रो॒ वै चतु॑र्‌होता। इन्द्र॒ खलु॒ वै श्रेष्ठो॑ दे॒वता॑नामुप॒देश॑नात्। य ए॒वमिन्द्र॒ श्रेष्ठं॑ दे॒वता॑नामुप॒देश॑ना॒द्वेद॑। वसि॑ष्ठः समा॒नानां भवति। तस्मा॒च्छ्रेष्ठ॑मा॒यन्तं॑ प्रथ॒मेनै॒वानु॑ बुध्यन्ते। अ॒यमाग\sn{}। अ॒यमवा॑सा॒दिति॑। की॒र्तिर॑स्य॒ पूर्वाऽऽग॑च्छति ज॒नता॑माय॒तः। अथो॑ एनं प्रथ॒मेनै॒वानु॑ बुध्यन्ते। अ॒यमाग\sn{}। अ॒यमवा॑सा॒दिति॑॥३॥\anuvakamend[स॒प्तहो॑ता प्रति॒ष्ठां वेद॑ बुध्यन्ते॒ षट्च॑]

%2.3.2.1
दक्षि॑णां प्रतिग्रही॒ष्यन्त्स॒प्तद॑श॒कृत्वोऽपान्यात्। आ॒त्मान॑मे॒व समि॑न्धे। तेज॑से वी॒र्या॑य। अथो प्र॒जाप॑तिरे॒वैनां भू॒त्वा प्रति॑ गृह्णाति। आ॒त्मनोऽनार्त्यै। यद्ये॑न॒मार्त्वि॑ज्याद्वृ॒त सन्त॑न्नि॒र्‌हरे॑रन्। आग्नीध्रे जुहुया॒द्दश॑होतारम्। च॒तु॒र्गृ॒ही॒तेनाज्ये॑न। पु॒रस्तात्प्र॒त्यङ्तिष्ठ\sn{}। प्र॒ति॒लो॒मं वि॒ग्राहम्॥४॥

%2.3.2.2
प्रा॒णाने॒वास्योप॑ दासयति। यद्ये॑नं॒ पुन॑रुप॒ शिक्षे॑युः। आग्नीध्र ए॒व जु॑हुया॒द्दश॑होतारम्। च॒तु॒र्गृ॒ही॒तेनाज्ये॑न। प॒श्चात्प्राङासी॑नः। अ॒नु॒लो॒ममवि॑ग्राहम्। प्रा॒णाने॒वास्मै॑ कल्पयति। प्राय॑श्चित्ती॒ वाग्घोतेत्यृ॑तुमु॒खऋ॑तुमुखे जुहोति। ऋ॒तूने॒वास्मै॑ कल्पयति। कल्प॑न्तेऽस्मा ऋ॒तव॑॥५॥

%2.3.2.3
कॢ॒प्ता अ॑स्मा ऋ॒तव॒ आय॑न्ति। षड्ढो॑ता॒ वै भू॒त्वा प्र॒जाप॑तिरि॒द सर्व॑मसृजत। स मनो॑ऽसृजत। मन॒सोऽधि॑ गाय॒त्रीम॑सृजत। तद्गा॑य॒त्रीं यश॑ आर्च्छत्। तामाऽल॑भत। गा॒य॒त्रि॒या अधि॒ छन्दास्यसृजत। छन्दो॒भ्योऽधि॒ साम॑। तत्साम॒ यश॑ आर्च्छत्। तदाऽल॑भत॥६॥

%2.3.2.4
साम्नोऽधि॒ यजूष्यसृजत। यजु॒र्भ्योऽधि॒ विष्णुम्। तद्विष्णुं॒ यश॑ आर्च्छत्। तमाऽल॑भत। विष्णो॒रध्योष॑धीरसृजत। ओष॑धी॒भ्योऽधि॒ सोमम्। तत्सोमं॒ यश॑ आर्च्छत्। तमाऽल॑भत। सोमा॒दधि॑ प॒शून॑सृजत। प॒शुभ्योऽधीन्द्रम्॥७॥

%2.3.2.5
तदिन्द्रं॒ यश॑ आर्च्छत्। तदे॑न॒न्नाति॒ प्राच्य॑वत। इन्द्र॑ इव यश॒स्वी भ॑वति। य ए॒वं वेद॑। नैनं॒ यशोऽति॒ प्रच्य॑वते। यद्वा इ॒दङ्किं च॑। तत्सर्व॑मुत्ता॒न ए॒वाङ्गी॑र॒सः प्रत्य॑गृह्णात्। तदे॑नं॒ प्रति॑गृहीत॒न्नाहि॑नत्। यत्किं च॑ प्रतिगृह्णी॒यात्। तत्सर्व॑मुत्ता॒नस्त्वाङ्गीर॒सः प्रति॑गृह्णा॒त्वित्ये॒व प्रति॑गृह्णीयात्। इ॒यं वा उ॑त्ता॒न आङ्गीर॒सः। अ॒नयै॒वैन॒त्प्रति॑ गृह्णाति। नैन हिनस्ति। ब॒र्॒हिषा॒ प्रती॑या॒द्गां वाऽश्वं॑ वा। ए॒तद्वै प॑शू॒नां प्रि॒यं धाम॑। प्रि॒येणै॒वैनं॒ धाम्ना॒ प्रत्ये॑ति॥८॥\anuvakamend[वि॒ग्राह॑मृ॒तव॒स्तदाऽल॑भ॒तेन्द्र॑ङ्गृह्णीया॒थ्षट्च॑]

%2.3.3.1
यो वा अवि॑द्वान्निव॒र्तय॑ते। विशी॑र्‌षा॒ सपाप्मा॒ऽमुष्मि॑ल्लोँ॒के भ॑वति। अथ॒ यो वि॒द्वान्नि॑व॒र्तय॑ते। सशी॑र्‌षा॒ विपाप्मा॒ऽमुष्मि॑ल्लोँ॒के भ॑वति। दे॒वता॒ वै स॒प्त पुष्टि॑कामा॒ न्य॑वर्तयन्त। अ॒ग्निश्च॑ पृथि॒वी च॑। वा॒युश्चा॒न्तरि॑क्षं च। आ॒दि॒त्यश्च॒ द्यौश्च॑ च॒न्द्रमा। अ॒ग्निर्न्य॑वर्तयत। स सा॑ह॒स्रम॑पुष्यत्॥९॥

%2.3.3.2
पृ॒थि॒वी न्य॑वर्तयत। सौष॑धीभि॒र्वन॒स्पति॑भिरपुष्यत्। वा॒युर्न्य॑वर्तयत। स मरी॑चीभिरपुष्यत्। अ॒न्तरि॑क्ष॒न्न्य॑वर्तयत। तद्वयो॑भिरपुष्यत्। आ॒दि॒त्यो न्य॑वर्तयत। स र॒श्मिभि॑रपुष्यत्। द्यौर्न्य॑वर्तयत। सा नक्ष॑त्रैरपुष्यत्। च॒न्द्रमा॒ न्य॑वर्तयत। सौ॑ऽहोरा॒त्रैर॑र्धमा॒सैर्मासैर्॑ऋ॒तुभि॑ संवत्स॒रेणा॑पुष्यत्। तान्पोषान्पुष्यति। यास्तेऽपु॑ष्यन्। य ए॒वं वि॒द्वान्नि च॑ व॒र्तय॑ते॒ परि॑ च॥१०॥\anuvakamend[अ॒पु॒ष्य॒न्नक्ष॑त्रैरपुष्य॒त्पञ्च॑ च]

%2.3.4.1
तस्य॒ वा अ॒ग्नेर्‌हिर॑ण्यं प्रतिजग्र॒हुष॑। अ॒र्धमि॑न्द्रि॒यस्यापाक्रामत्। तदे॒तेनै॒व प्रत्य॑गृह्णात्। तेन॒ वै सोऽर्धमि॑न्द्रि॒यस्या॒त्मन्नु॒पाध॑त्त। अ॒र्धमि॑न्द्रि॒यस्या॒त्मन्नु॒पाध॑त्ते। य ए॒वं वि॒द्वान् हिर॑ण्यं प्रतिगृ॒ह्णाति॑। अथ॒ योऽवि॑द्वान्प्रति गृ॒ह्णाति॑। अ॒र्धम॑स्येन्द्रि॒यस्याप॑ क्रामति। तस्य॒ वै सोम॑स्य॒ वास॑ प्रतिजग्र॒हुष॑। तृती॑यमिन्द्रि॒यस्यापाक्रामत्॥११॥

%2.3.4.2
तदे॒तेनै॒व प्रत्य॑गृह्णात्। तेन॒ वै स तृती॑यमिन्द्रि॒यस्या॒त्मन्नु॒पाध॑त्त। तृती॑यमिन्द्रि॒यस्या॒त्मन्नु॒पाध॑त्ते। य ए॒वं वि॒द्वान् वास॑ प्रतिगृ॒ह्णाति॑। अथ॒ योऽवि॑द्वान्प्रति गृ॒ह्णाति॑। तृती॑यमस्येन्द्रि॒यस्याप॑ क्रामति। तस्य॒ वै रु॒द्रस्य॒ गां प्र॑तिजग्र॒हुष॑। च॒तु॒र्थमि॑न्द्रि॒यस्यापाक्रामत्। तामे॒तेनै॒व प्रत्य॑गृह्णात्। तेन॒ वै स च॑तु॒र्थमि॑न्द्रि॒यस्या॒त्मन्नु॒पाध॑त्त॥१२॥

%2.3.4.3
च॒तु॒र्थमि॑न्द्रि॒यस्या॒त्मन्नु॒पाध॑त्ते। य ए॒वं वि॒द्वान्गां प्र॑तिगृ॒ह्णाति॑। अथ॒ योऽवि॑द्वान्प्रतिगृ॒ह्णाति॑। च॒तु॒र्थम॑स्येन्द्रि॒यस्याप॑ क्रामति। तस्य॒ वै वरु॑ण॒स्याश्वं॑ प्रतिजग्र॒हुष॑। प॒ञ्च॒ममि॑न्द्रि॒यस्यापाक्रामत्। तमे॒तेनै॒व प्रत्य॑गृह्णात्। तेन॒ वै स प॑ञ्च॒ममि॑न्द्रि॒यस्या॒त्मन्नु॒पाध॑त्त। प॒ञ्च॒ममि॑न्द्रि॒यस्या॒त्मन्नु॒पाध॑त्ते। य ए॒वं वि॒द्वानश्वं॑ प्रतिगृ॒ह्णाति॑॥१३॥

%2.3.4.4
अथ॒ योऽवि॑द्वान्प्रतिगृ॒ह्णाति॑। प॒ञ्च॒मम॑स्येन्द्रि॒यस्याप॑ क्रामति। तस्य॒ वै प्र॒जाप॑ते॒ पुरु॑षं प्रतिजग्र॒हुष॑। ष॒ष्ठमि॑न्द्रि॒यस्यापाक्रामत्। तमे॒तेनै॒व प्रत्य॑गृह्णात्। तेन॒ वै स ष॒ष्ठमि॑न्द्रि॒यस्या॒त्मन्नु॒पाध॑त्त। ष॒ष्ठमि॑न्द्रि॒यस्या॒त्मन्नु॒पाध॑त्ते। य ए॒वं वि॒द्वान्पुरु॑षं प्रतिगृ॒ह्णाति॑। अथ॒ योऽवि॑द्वान्प्रतिगृ॒ह्णाति॑। ष॒ष्ठम॑स्येन्द्रि॒यस्याप॑ क्रामति॥१४॥

%2.3.4.5
तस्य॒ वै मनो॒स्तल्पं॑ प्रतिजग्र॒हुष॑। स॒प्त॒ममि॑न्द्रि॒यस्यापाक्रामत्। तमे॒तेनै॒व प्रत्य॑गृह्णात्। तेन॒ वै स स॑प्त॒ममि॑न्द्रि॒यस्या॒त्मन्नु॒पाध॑त्त। स॒प्त॒ममि॑न्द्रि॒यस्या॒त्मन्नु॒पाध॑त्ते। य ए॒वं वि॒द्वास्तल्पं॑ प्रति गृ॒ह्णाति॑। अथ॒ योऽवि॑द्वान्प्रति गृ॒ह्णाति॑। स॒प्त॒मम॑स्येन्द्रि॒यस्याप॑ क्रामति। तस्य॒ वा उ॑त्ता॒नस्याङ्गीर॒सस्याप्रा॑णत्प्रतिजग्र॒हुष॑। अ॒ष्ट॒ममि॑न्द्रि॒यस्यापाक्रामत्॥१५॥

%2.3.4.6
तदे॒तेनै॒व प्रत्य॑गृह्णात्। तेन॒ वै सोऽष्ट॒ममि॑न्द्रि॒यस्या॒त्मन्नु॒पाध॑त्त। अ॒ष्ट॒ममि॑न्द्रि॒यस्या॒त्मन्नु॒पाध॑त्ते। य ए॒वं वि॒द्वानप्रा॑णत्प्रतिगृ॒ह्णाति॑। अथ॒ योऽवि॑द्वान्प्रतिगृ॒ह्णाति॑। अ॒ष्ट॒मम॑स्येन्द्रि॒यस्याप॑ क्रामति। यद्वा इ॒दङ्किं च॑। तत्सर्व॑मुत्ता॒न ए॒वाङ्गी॑र॒सः प्रत्य॑गृह्णात्। तदे॑नं॒ प्रति॑गृहीत॒न्नाहि॑नत्। यत्किं च॑ प्रतिगृह्णी॒यात्। तत्सर्व॑मुत्ता॒नस्त्वाङ्गीर॒सः प्रति॑गृह्णा॒त्वित्ये॒व प्रति॑गृह्णीयात्। इ॒यं वा उ॑त्ता॒न आङ्गीर॒सः। अ॒नयै॒वैन॒त्प्रति॑ गृह्णाति। नैन हिनस्ति॥१६॥\anuvakamend[तृती॑यमिन्द्रि॒यस्यापाक्रामच्चतु॒र्थमि॑न्द्रि॒यस्या॒त्मन्नु॒पाध॒त्ताश्वं॑ प्रतिगृ॒ह्णाति॑ ष॒ष्ठम॑स्येन्द्रि॒यस्याप॑क्रामत्यष्ट॒ममि॑न्द्रि॒यस्यापाक्रामत्प्रतिगृह्णी॒याच्च॒त्वारि॑ च (तस्य॒ वा अ॒ग्नेर्‌हिर॑ण्य॒ सोम॑स्य॒ वास॒स्तदे॒तेन॑ रु॒द्रस्य॒ गान्तामे॒तेन॒ वरु॑ण॒स्याश्वं॑ प्र॒जाप॑ते॒ पुरु॑षं॒ मनो॒स्तल्प॒न्तमे॒तेनोत्ता॒नस्य॒ तदे॒तेनाप्रा॑ण॒द्यद्वै। अ॒र्धं तृती॑यमष्ट॒मं तच्च॑तु॒र्थं तां प॑ञ्च॒म ष॒ष्ठ स॑प्त॒मन्तम्। तदे॒तेन॒ द्वे तामे॒तेनैकं॒ तमे॒तेन॒ त्रीणि॒ तदे॒तेनैकम्॥)]

%2.3.5.1
ब्र॒ह्म॒वा॒दिनो॑ वदन्ति। यद्दश॑होतारः स॒त्रमास॑त। केन॒ ते गृ॒हप॑तिनाऽऽर्ध्नुवन्। केन॑ प्र॒जा अ॑सृज॒न्तेति॑। प्र॒जाप॑तिना॒ वै ते गृ॒हप॑तिनाऽऽर्ध्नुवन्। तेन॑ प्र॒जा अ॑सृजन्त। यच्चतु॑र्‌होतारः स॒त्रमास॑त। केन॒ ते गृ॒हप॑तिनाऽऽर्ध्नुवन्। केनौष॑धीरसृज॒न्तेति॑। सोमे॑न॒ वै ते गृ॒हप॑तिनाऽऽर्ध्नुवन्॥१७॥

%2.3.5.2
तेनौष॑धीरसृजन्त। यत्पञ्च॑होतारः स॒त्रमास॑त। केन॒ ते गृ॒हप॑तिनाऽऽर्ध्नुवन्। केनै॒भ्यो लो॒केभ्योऽसु॑रा॒न्प्राणु॑दन्त। केनै॑षां प॒शून॑वृञ्ज॒तेति॑। अ॒ग्निना॒ वै ते गृ॒हप॑तिनाऽऽर्ध्नुवन्। तेनै॒भ्यो लो॒केभ्योऽसु॑रा॒न्प्राणु॑दन्त। तेनै॑षां प॒शून॑वृञ्जत। यथ्षड्ढो॑तारः स॒त्रमास॑त। केन॒ ते गृ॒हप॑तिनाऽऽर्ध्नुवन्॥१८॥

%2.3.5.3
केन॒र्तून॑कल्पय॒न्तेति॑। धा॒त्रा वै ते गृ॒हप॑तिनाऽऽर्ध्नुवन्। तेन॒र्तून॑कल्पयन्त। यत्स॒प्तहो॑तारः स॒त्रमास॑त। केन॒ ते गृ॒हप॑तिनाऽऽर्ध्नुवन्। केन॒ सुव॑रायन्। केने॒माल्लोँ॒कान्त्सम॑तन्व॒न्निति॑। अ॒र्य॒म्णा वै ते गृ॒हप॑तिनाऽऽर्ध्नुवन्। तेन॒ सुव॑रायन्। तेने॒माल्लोँ॒कान्त्सम॑तन्व॒न्निति॑॥१९॥

%2.3.5.4
ए॒ते वै दे॒वा गृ॒हप॑तयः। तान् य ए॒वं वि॒द्वान्। अप्य॒न्यस्य॑ गार्‌हप॒ते दीक्ष॑ते। अ॒वा॒न्त॒रमे॒व स॒त्रिणा॑मृध्नोति। यो वा अ॑र्य॒मणं॒ वेद॑। दान॑कामा अस्मै प्र॒जा भ॑वन्ति। य॒ज्ञो वा अ॑र्य॒मा। आर्या॑वस॒तिरिति॒ वै तमा॑हु॒र्यं प्र॒शस॑न्ति। आर्या॑वस॒तिर्भ॑वति। य ए॒वं वेद॑॥२०॥

%2.3.5.5
यद्वा इ॒दङ्किं च॑। तत्सर्वं॒ चतु॑र्\mbox{}होतारः। चतु॑र्‌होतृ॒भ्योऽधि॑ य॒ज्ञो निर्मि॑तः। स य ए॒वं वि॒द्वान्‌ वि॒वदे॑त। अ॒हमे॒व भूयो॑ वेद। यश्चतु॑र्‌होतॄ॒न् वेदेति॑। स ह्ये॑व भूयो॒ वेद॑। यश्चतु॑र्‌होतॄ॒न् वेद॑। यो वै चतु॑र्‌होतृणा॒ होतॄ॒न् वेद॑। सर्वा॑सु प्र॒जास्वन्न॑मत्ति॥२१॥

%2.3.5.6
सर्वा॒ दिशो॒ऽभि ज॑यति। प्र॒जाप॑ति॒र्वै दश॑होतृणा॒ होता। सोम॒श्चतु॑र्‌होतृणा॒ होता। अ॒ग्निः पञ्च॑होतृणा॒ होता। धा॒ता षड्ढो॑तृणा॒ होता। अ॒र्य॒मा स॒प्तहो॑तृणा॒ होता। ए॒ते वै चतु॑र्\mbox{}होतृणा॒ होता॑रः। तान् य ए॒वं वेद॑। सर्वा॑सु प्र॒जास्वन्न॑मत्ति। सर्वा॒ दिशो॒ऽभि ज॑यति॥२२॥\anuvakamend[आ॒र्ध्नु॒व॒न्ना॒र्ध्नु॒व॒न्नित्ये॒वं वेदात्ति सर्वा॒ दिशो॒ऽभि ज॑यति (वै तेन॑ स॒त्रङ्केन॑ ॥ )]

%2.3.6.1
प्र॒जाप॑तिः प्र॒जाः सृ॒ष्ट्वा व्य॑स्रसत। स हृद॑यं भू॒तो॑ऽशयत्। आत्म॒न्॒ हा ३ इत्यह्व॑यत्। आप॒ प्रत्य॑शृण्वन्। ता अ॑ग्निहो॒त्रेणै॒व य॑ज्ञक्र॒तुनोप॑ प॒र्याव॑र्तन्त। ताः कुसि॑न्ध॒मुपौ॑हन्। तस्मा॑दग्निहो॒त्रस्य॑ यज्ञक्र॒तोः। एक॑ ऋ॒त्विक्। च॒तु॒ष्कृत्वोऽह्व॑यत्। अ॒ग्निर्वा॒युरा॑दि॒त्यश्च॒न्द्रमा॥२३॥

%2.3.6.2
ते प्रत्य॑शृण्वन्। ते द॑र्‌शपूर्णमा॒साभ्या॑मे॒व य॑ज्ञक्र॒तुनोप॑ प॒र्याव॑र्तन्त। त उपौ॑हश्च॒त्वार्यङ्गा॑नि। तस्माद्दर्‌शपूर्णमा॒सयोर्यज्ञक्र॒तोः। च॒त्वार॑ ऋ॒त्विज॑। प॒ञ्च॒कृत्वोऽह्व॑यत्। प॒शव॒ प्रत्य॑शृण्वन्। ते चा॑तुर्मा॒स्यैरे॒व य॑ज्ञक्र॒तुनोप॑ प॒र्याव॑र्तन्त। त उपौ॑हं॒ लोम॑ छ॒वीं मा॒समस्थि॑ म॒ज्जानम्। तस्माच्चातुर्मा॒स्याना॑ यज्ञक्र॒तोः॥२४॥

%2.3.6.3
पञ्च॒र्त्विज॑। ष॒ट्कृत्वोऽह्व॑यत्। ऋ॒तव॒ प्रत्य॑शृण्वन्। ते प॑शुब॒न्धेनै॒व य॑ज्ञक्र॒तुनोप॑प॒र्याव॑र्तन्त। त उपौ॑ह॒न्त्स्तना॑वा॒ण्डौ शि॒श्ञमवाञ्चं प्रा॒णम्। तस्मात्पशुब॒न्धस्य॑ यज्ञक्र॒तोः। षडृ॒त्विज॑। स॒प्त॒कृत्वोऽह्व॑यत्। होत्रा॒ प्रत्य॑शृण्वन्। ताः सौ॒म्येनै॒वाध्व॒रेण॑ यज्ञक्र॒तुनोप॑प॒र्याव॑र्तन्त॥२५॥

%2.3.6.4
ता उपौ॑हन्त्स॒प्त शी॑र्‌ष॒ण्यान्प्रा॒णान्। तस्मात्सौ॒म्यस्याध्व॒रस्य॑ यज्ञक्र॒तोः। स॒प्त होत्रा॒ प्राची॒र्वष॑ट्कुर्वन्ति। द॒श॒कृत्वोऽह्व॑यत्। तप॒ प्रत्य॑शृणोत्। तत्कर्म॑णै॒व सं॑वत्स॒रेण॒ सर्वैर्यज्ञक्र॒तुभि॒रुप॑ प॒र्याव॑र्तत। तत्सर्व॑मा॒त्मान॒मप॑रिवर्ग॒मुपौ॑हत्। तस्मात्संवत्स॒रे सर्वे॑ यज्ञक्र॒तवोऽव॑रुध्यन्ते। तस्मा॒द्दश॑होता॒ चतु॑र्‌होता। पञ्च॑होता॒ षड्ढो॑ता स॒प्तहो॑ता। एक॑होत्रे ब॒लि ह॑रन्ति। हर॑न्त्यस्मै प्र॒जा ब॒लिम्। ऐन॒मप्र॑तिख्यातं गच्छति। य ए॒वं वेद॑॥२६॥\anuvakamend[च॒न्द्रमाश्चातुर्मा॒स्यानां यज्ञक्र॒तोर॑ध्व॒रेण॑ यज्ञक्र॒तुनोप॑ प॒र्याव॑र्तन्त स॒प्तहो॑ता च॒त्वारि॑ च]

%2.3.7.1
प्र॒जाप॑ति॒ पुरु॑षमसृजत। सोऽग्निर॑ब्रवीत्। ममा॒यमन्न॑म॒स्त्विति॑। सो॑ऽबिभेत्। सर्वं॒ वै मा॒ऽयं प्र ध॑क्ष्य॒तीति॑। स ए॒ताश्चतु॑र्\mbox{}होतॄनात्म॒स्पर॑णानपश्यत्। तान॑जुहोत्। तैर्वै स आ॒त्मान॑मस्पृणोत्। यद॑ग्निहो॒त्रं जु॒होति॑। एक॑होतारमे॒व तद्य॑ज्ञक्र॒तुमाप्नोत्यग्निहो॒त्रम्॥२७॥

%2.3.7.2
कुसि॑न्धञ्चा॒त्मन॑ स्पृ॒णोति॑। आ॒दि॒त्यस्य॑ च॒ सा॑युज्यं गच्छति। च॒तुरुन्न॑यति। चतु॑र्‌होतारमे॒व तद्य॑ज्ञक्र॒तुमाप्नोति दर्‌शपूर्णमा॒सौ। च॒त्वारि॑ चा॒त्मनोऽङ्गा॑नि स्पृ॒णोति॑। आ॒दि॒त्यस्य॑ च॒ सायु॑ज्यं गच्छति। च॒तुरुन्न॑यति। स॒मित्प॑ञ्च॒मी। पञ्च॑होतारमे॒व तद्य॑ज्ञक्र॒तुमाप्नोति चातुर्मा॒स्यानि॑। लोम॑ छ॒वीं मा॒समस्थि॑ म॒ज्जानम्॥२८॥

%2.3.7.3
तानि॑ चा॒त्मन॑ स्पृ॒णोति॑। आ॒दि॒त्यस्य॑ च॒ सायु॑ज्यं गच्छति। च॒तुरुन्न॑यति। द्विर्जु॑होति। षड्ढो॑तारमे॒व तद्य॑ज्ञक्र॒तुमाप्नोति पशुब॒न्धम्। स्तना॑वा॒ण्डौ शि॒श्ञमवाञ्चं प्रा॒णम्। तानि॑ चा॒त्मन॑ स्पृ॒णोति॑। आ॒दि॒त्यस्य॑ च॒ सायु॑ज्यं गच्छति। च॒तुरुन्न॑यति। द्विर्जु॑होति॥२९॥

%2.3.7.4
स॒मित्स॑प्त॒मी। स॒प्तहो॑तारमे॒व तद्य॑ज्ञक्र॒तुमाप्नोति सौ॒म्यम॑ध्व॒रम्। स॒प्त चा॒त्मन॑ शीर्\mbox{}ष॒ण्यान्प्रा॒णान्त्स्पृ॒णोति॑। आ॒दि॒त्यस्य॑ च॒ सायु॑ज्यं गच्छति। च॒तुरुन्न॑यति। द्विर्जु॒होति॑। द्विर्निमार्ष्टि। द्विः प्राश्ञा॑ति। दश॑होतारमे॒व तद्य॑ज्ञक्र॒तुमाप्नोति संवत्स॒रम्। सर्वं॑ चा॒त्मान॒मप॑रिवर्ग स्पृ॒णोति॑। आ॒दि॒त्यस्य॑ च॒ सायु॑ज्यं गच्छति॥३०॥\anuvakamend[अ॒ग्नि॒हो॒त्रं म॒ज्जान॒न्द्विर्जु॑हो॒त्यप॑रिवर्ग स्पृ॒णोत्येकं च]

%2.3.8.1
प्र॒जाप॑तिरकामयत॒ प्र जा॑ये॒येति॑। स तपो॑ऽतप्यत। सोऽन्तर्वा॑नभवत्। स हरि॑तः श्या॒वो॑ऽभवत्। तस्मा॒त्स्त्र्य॑न्तर्व॑त्नी। हरि॑णी स॒ती श्या॒वा भ॑वति। स वि॒जाय॑मानो॒ गर्भे॑णाताम्यत्। स ता॒न्तः कृ॒ष्णः श्या॒वो॑ऽभवत्। तस्मात्ता॒न्तः कृ॒ष्णः श्या॒वो भ॑वति। तस्यासु॑रे॒वाजी॑वत्॥३१॥

%2.3.8.2
तेनासु॒नाऽसु॑रानसृजत। तदसु॑राणामसुर॒त्वम्। य ए॒वमसु॑राणामसुर॒त्वं वेद॑। असु॑माने॒व भ॑वति। नैन॒मसु॑र्जहाति। सोऽसु॑रान्त्सृ॒ष्ट्वा पि॒तेवा॑मन्यत। तदनु॑ पि॒तॄन॑सृजत। तत्पि॑तृ॒णां पि॑तृ॒त्वम्। य ए॒वं पि॑तृ॒णां पि॑तृ॒त्वं वेद॑। पि॒तेवै॒व स्वानां भवति॥३२॥

%2.3.8.3
यन्त्य॑स्य पि॒तरो॒ हवम्। स पि॒तॄन्त्सृ॒ष्ट्वाऽऽम॑नस्यत्। तदनु॑ मनु॒ष्या॑नसृजत। तन्म॑नु॒ष्या॑णां मनुष्य॒त्वम्। य ए॒वं म॑नु॒ष्या॑णां मनुष्य॒त्वं वेद॑। म॒न॒स्व्ये॑व भ॑वति। नैनं॒ मनु॑र्जहाति। तस्मै॑ मनु॒ष्यान्त्ससृजा॒नाय॑। दिवा॑ देव॒त्राऽभ॑वत्। तदनु॑ दे॒वान॑सृजत। तद्दे॒वानान्देव॒त्वम्। य ए॒वन्दे॒वानान्देव॒त्वं वेद॑। दिवा॑ है॒वास्य॑ देव॒त्रा भ॑वति। तानि॒ वा ए॒तानि॑ च॒त्वार्यम्भासि। दे॒वा म॑नु॒ष्या पि॒तरोऽसु॑राः। तेषु॒ सर्वे॒ष्वम्भो॒ नभ॑ इव भवति। य ए॒वं वेद॑॥३३॥\anuvakamend[अ॒जी॒व॒त्स्वानां भवति दे॒वान॑सृजत स॒प्त च॑]

%2.3.9.1
ब्र॒ह्म॒वा॒दिनो॑ वदन्ति। यो वा इ॒मं वि॒द्यात्। यतो॒ऽयं पव॑ते। यद॑भि॒ पव॑ते। यद॑भि सं॒पव॑ते। सर्व॒मायु॑रियात्। न पु॒राऽऽयु॑ष॒ प्र मी॑येत। प॒शु॒मान्त्स्यात्। वि॒न्देत॑ प्र॒जाम्। यो वा इ॒मं वेद॑॥३४॥

%2.3.9.2
यतो॒ऽयं पव॑ते। यद॑भि॒ पव॑ते। यद॑भि सं॒पव॑ते। सर्व॒मायु॑रेति। न पु॒राऽऽयु॑ष॒ प्र मी॑यते। प॒शु॒मान्भ॑वति। वि॒न्दते प्र॒जाम्। अ॒द्भ्यः प॑वते। अ॒पो॑ऽभि प॑वते। अ॒पो॑ऽभि संप॑वते॥३५॥

%2.3.9.3
अ॒स्याः प॑वते। इ॒माम॒भि प॑वते। इ॒माम॒भि संप॑वते। अ॒ग्नेः प॑वते। अ॒ग्निम॒भि प॑वते। अ॒ग्निम॒भि सं प॑वते। अ॒न्तरि॑क्षात्पवते। अ॒न्तरि॑क्षम॒भि प॑वते। अ॒न्तरि॑क्षम॒भि सं प॑वते। आ॒दि॒त्यात्प॑वते॥३६॥

%2.3.9.4
आ॒दि॒त्यम॒भि प॑वते। आ॒दि॒त्यम॒भि सं प॑वते। द्योः प॑वते। दिव॑म॒भि प॑वते। दिव॑म॒भि सं प॑वते। दि॒ग्भ्यः प॑वते। दिशो॒ऽभि प॑वते। दिशो॒ऽभि संप॑वते। स यत्पु॒रस्ता॒द्वाति॑। प्रा॒ण ए॒व भू॒त्वा पु॒रस्ताद्वाति॥३७॥

%2.3.9.5
तस्मात्पु॒रस्ता॒द्वान्तम्। सर्वा प्र॒जाः प्रति॑ नन्दन्ति। प्रा॒णो हि प्रि॒यः प्र॒जानाम्। प्रा॒ण इ॑व प्रि॒यः प्र॒जानां भवति। य ए॒वं वेद॑। स वा ए॒ष प्रा॒ण ए॒व। अथ॒ यद्द॑क्षिण॒तो वाति॑। मा॒त॒रिश्वै॒व भू॒त्वा द॑क्षिण॒तो वा॑ति। तस्माद्दक्षिण॒तो वान्तं॑ वि॒द्यात्। सर्वा॒ दिश॒ आ वा॑ति॥३८॥

%2.3.9.6
सर्वा॒ दिशोऽनु॒ वि वा॑ति। सर्वा॒ दिशोऽनु॒ सं वा॒तीति॑। स वा ए॒ष मा॑त॒रिश्वै॒व। अथ॒ यत्प॒श्चाद्वाति॑। पव॑मान ए॒व भू॒त्वा प॒श्चाद्वा॑ति। पू॒तम॑स्मा॒ आह॑रन्ति। पू॒तमुप॑हरन्ति। पू॒तम॑श्ञाति। य ए॒वं वेद॑। स वा ए॒ष पव॑मान ए॒व॥३९॥

%2.3.9.7
अथ॒ यदु॑त्तर॒तो वाति॑। स॒वि॒तैव भू॒त्त्वोत्त॑र॒तो वा॑ति। स॒वि॒तेव॒ स्वानां भवति। य ए॒वं वेद॑। स वा ए॒ष स॑वि॒तैव। ते य ए॑नं पु॒रस्ता॑दा॒यन्त॑मुप॒वद॑न्ति। य ए॒वास्य॑ पु॒रस्तात्पा॒प्मान॑। तास्तेऽप॑ घ्नन्ति। पु॒रस्ता॒दित॑रान्पा॒प्मन॑ सचन्ते। अथ॒ य ए॑नन्दक्षिण॒त आ॒यन्त॑मुप॒वद॑न्ति॥४०॥

%2.3.9.8
य ए॒वास्य॑ दक्षिण॒तः पा॒प्मान॑। तास्तेऽप॑ घ्नन्ति। द॒क्षि॒ण॒त इत॑रान्पा॒प्मन॑ सचन्ते। अथ॒ य ए॑नं प॒श्चादा॒यन्त॑मुप॒ वद॑न्ति। य ए॒वास्य॑ प॒श्चात्पा॒प्मान॑। तास्तेऽप॑ घ्नन्ति। प॒श्चादित॑रान्पा॒प्मन॑ सचन्ते। अथ॒ य ए॑नमुत्तर॒त आ॒यन्त॑मुप॒ वद॑न्ति। य ए॒वास्योत्तर॒तः पा॒प्मान॑। तास्तेऽप॑ घ्नन्ति॥४१॥

%2.3.9.9
उ॒त्त॒र॒त इत॑रान्पा॒प्मन॑ सचन्ते। तस्मा॑दे॒वं वि॒द्वान्। वीव॑ नृत्येत्। प्रेव॑ चलेत्। व्यस्ये॑वा॒क्ष्यौ भा॑षेत। म॒ण्टये॑दिव। क्रा॒थये॑दिव। शृ॒ङ्गा॒येते॑व। उ॒त मोप॑ वदेयुः। उ॒त मे॑ पा॒प्मान॒मप॑ हन्यु॒रिति॑। स यान्दिश स॒निमे॒ष्यन्त्स्यात्। य॒दा तान्दिशं॒ वातो॑ वा॒यात्। अथ॒ प्रवे॒यात्। प्र वा॑ धावयेत्। सा॒तमे॒व र॑दि॒तं व्यू॑ढं ग॒न्धम॒भि प्रच्य॑वते। आऽस्य॒ तं ज॑नप॒दं पूर्वा॑ की॒र्तिर्ग॑च्छति। दान॑कामा अस्मै प्र॒जा भ॑वन्ति। य ए॒वं वेद॑॥४२॥\anuvakamend[वेद॒ सं प॑वत आदि॒त्यात्प॑वते वा॒त्या वात्ये॒ष पव॑मान ए॒व द॑क्षिण॒त आ॒यन्त॑मुप॒ वद॑न्त्युत्तर॒तः पा॒प्मान॒स्ता स्तेप॑ घ्न॒न्तीत्य॒ष्टौ च॑]

%2.3.10.1
प्र॒जाप॑ति॒ सोम॒ राजा॑नमसृजत। तन्त्रयो॒ वेदा॒ अन्व॑सृज्यन्त। तान् हस्ते॑ऽकुरुत। अथ॒ ह सीता॑ सावि॒त्री। सोम॒ राजा॑नञ्चकमे। श्र॒द्धामु॒ स च॑कमे। साऽऽह॑ पि॒तरं॑ प्र॒जाप॑ति॒मुप॑ससार। त हो॑वाच। नम॑स्ते अस्तु भगवः। उप॑ त्वाऽयानि॥४३॥

%2.3.10.2
प्र त्वा॑ पद्ये। सोमं॒ वै राजा॑नङ्कामये। श्र॒द्धामु॒ स का॑मयत॒ इति॑। तस्या॑ उ॒ ह स्था॑ग॒रम॑लङ्का॒रङ्क॑ल्पयि॒त्वा। दश॑होतारं पु॒रस्ताद्व्या॒ख्याय॑। चतु॑र्\mbox{}होतारन्दक्षिण॒तः। पञ्च॑होतारं प॒श्चात्। षड्ढो॑तारमुत्तर॒तः। स॒प्तहो॑तारमु॒परि॑ष्टात्। स॒म्भा॒रैश्च॒ पत्नि॑भिश्च॒ मुखे॑ऽल॒ङ्कृत्य॑॥४४॥

%2.3.10.3
आऽस्यार्धं व॑व्राज। ता हो॒दीक्ष्यो॑वाच। उप॒ मा व॑र्त॒स्वेति॑। त हो॑वाच। भोग॒न्तु म॒ आच॑क्ष्व। ए॒तन्म॒ आच॑क्ष्व। यत्ते॑ पा॒णाविति॑। तस्या॑ उ॒ ह त्रीन् वेदा॒न्प्रद॑दौ। तस्मा॒दुह॒ स्त्रियो॒ भोग॒मैव हा॑रयन्ते। स यः का॒मये॑त प्रि॒यः स्या॒मिति॑॥४५॥

%2.3.10.4
यं वा का॒मये॑त प्रि॒यः स्या॒दिति॑। तस्मा॑ ए॒त स्था॑ग॒रम॑लङ्का॒रङ्क॑ल्पयि॒त्वा। दश॑होतारं पु॒रस्ताद्व्या॒ख्याय॑। चतु॑र्\mbox{}होतारन्दक्षिण॒तः। पञ्च॑होतारं प॒श्चात्। षड्ढो॑तारमुत्तर॒तः। स॒प्तहो॑तारमु॒परि॑ष्टात्। स॒म्भा॒रैश्च॒ पत्नि॑भिश्च॒ मुखे॑ऽल॒ङ्कृत्य॑। आस्यार्धं व्र॑जेत्। प्रि॒यो है॒व भ॑वति॥४६॥\anuvakamend[अ॒या॒न्य॒ल॒ङ्कृत्य॑ स्या॒मिति॑ भवति]

%2.3.11.1
ब्रह्मात्म॒न्वद॑सृजत। तद॑कामयत। समा॒त्मना॑ पद्ये॒येति॑। आत्म॒न्नात्म॒न्नित्याम॑न्त्रयत। तस्मै॑ दश॒म हू॒तः प्रत्य॑शृणोत्। स दश॑हूतोऽभवत्। दश॑हूतो ह॒ वै नामै॒षः। तं वा ए॒तन्दश॑हूत॒ सन्तम्। दश॑हो॒तेत्याच॑क्षते प॒रोक्षे॑ण। प॒रोक्ष॑प्रिया इव॒ हि दे॒वाः॥४७॥

%2.3.11.2
आत्म॒न्नात्म॒न्नित्याम॑न्त्रयत। तस्मै॑ सप्त॒म हू॒तः प्रत्य॑शृणोत्। स स॒प्तहू॑तोऽभवत्। स॒प्तहू॑तो ह॒ वै नामै॒षः। तं वा ए॒त स॒प्तहू॑त॒ सन्तम्। स॒प्तहो॒तेत्याच॑क्षते प॒रोक्षे॑ण। प॒रोक्ष॑प्रिया इव॒ हि दे॒वाः। आत्म॒न्नात्म॒न्नित्याम॑न्त्रयत। तस्मै॑ ष॒ष्ठ हू॒तः प्रत्य॑शृणोत्। स षड्ढू॑तोऽभवत्॥४८॥

%2.3.11.3
षड्ढू॑तो ह॒ वै नामै॒षः। तं वा ए॒त षड्ढू॑त॒ सन्तम्। षड्ढो॒तेत्याच॑क्षते प॒रोक्षे॑ण। प॒रोक्ष॑प्रिया इव॒ हि दे॒वाः। आत्म॒न्नात्म॒न्नित्याम॑न्त्रयत। तस्मै॑ पञ्च॒म हू॒तः प्रत्य॑शृणोत्। स पञ्च॑हूतोऽभवत्। पञ्च॑हूतो ह॒ वै नामै॒षः। तं वा ए॒तं पञ्च॑हूत॒ सन्तम्। पञ्च॑हो॒तेत्याच॑क्षते प॒रोक्षे॑ण॥४९॥

%2.3.11.4
प॒रोक्ष॑प्रिया इव॒ हि दे॒वाः। आत्म॒न्नात्म॒न्नित्याम॑न्त्रयत। तस्मै॑ चतु॒र्थ हू॒तः प्रत्य॑शृणोत्। स चतु॑र्\mbox{}हूतोऽभवत्। चतु॑र्‌हूतो ह॒ वै नामै॒षः। तं वा ए॒तञ्चतु॑र्‌हूत॒ सन्तम्। चतु॑र्हो॒तेत्याच॑क्षते प॒रोक्षे॑ण। प॒रोक्ष॑प्रिया इव॒ हि दे॒वाः। तम॑ब्रवीत्। त्वं वै मे॒ नेदि॑ष्ठ हू॒तः प्रत्य॑श्रौषीः। त्वयै॑नानाख्या॒तार॒ इति॑। तस्मा॒न्नु है॑ना॒श्चतु॑र्‌होतार॒ इत्याच॑क्षते। तस्माच्छुश्रू॒षुः पु॒त्राणा॒ हृद्य॑तमः। नेदि॑ष्ठो॒ हृद्य॑तमः। नेदि॑ष्ठो॒ ब्रह्म॑णो भवति। य ए॒वं वेद॑॥५०॥\anuvakamend[दे॒वाष्षड्ढू॑तोऽभव॒त्पञ्च॑हो॒तेत्याच॑क्षते प॒रोक्षे॑णाश्रौषी॒ष्षट्च॑]




\prashnaend{ब्र॒ह्म॒वा॒दिन॒ किं दक्षि॑णां॒ यो वा अवि॑द्वा॒न्तस्य॒ वै ब्र॑ह्मवा॒दिनो॒ यद्दश॑होतारः प्र॒जाप॑ति॒र्व्य॑स्रं प्र॒जाप॑ति॒ पुरु॑षं प्र॒जाप॑तिरकामयत॒ स तप॒ सोऽन्तर्वान्ब्रह्मवा॒दिनो॒ यो वा इ॒मं वि॒द्यात्प्र॒जाप॑ति॒ सोम॒ राजा॑नं॒ ब्रह्मात्म॒न्वदेका॑दश॥११॥}{ब्र॒ह्म॒वा॒दिन॒स्तस्य॒ वा अ॒ग्नेर्यद्वा इ॒दङ्किं च॑ प्र॒जाप॑तिरकामयत॒ य ए॒वास्य॑ दक्षिण॒तः प॑ञ्चा॒शत्॥५०॥}{ब्र॒ह्म॒वा॒दिनो॒ य ए॒वं वेद॑॥}{हरि॑ ओम्॥}{इति श्रीकृष्णयजुर्वेदीयतैत्तिरीयब्राह्मणे द्वितीयाष्टके तृतीयः प्रपाठकः समाप्तः॥}
\clearpage
\sect{चतुर्थः प्रश्नः}
\setcounter{anuvakam}{0}
\dnsub{तैत्तिरीयब्राह्मणे द्वितीयाष्टके चतुर्थः प्रपाठकः}

%2.4.1.1
जुष्टो॒ दमू॑ना॒ अति॑थिर्दुरो॒णे। इ॒मन्नो॑ य॒ज्ञमुप॑ याहि वि॒द्वान्। विश्वा॑ अग्नेऽभि॒युजो॑ वि॒हत्य॑। श॒त्रू॒य॒तामा भ॑रा॒ भोज॑नानि। अग्ने॒ शर्ध॑ मह॒ते सौभ॑गाय। तव॑ द्यु॒म्नान्यु॑त्त॒मानि॑ सन्तु। सञ्जास्प॒त्य सु॒यम॒मा कृ॑णुष्व। श॒त्रू॒य॒ताम॒भि ति॑ष्ठा॒ महा सि। अग्ने॒ यो नो॒ऽभितो॒ जन॑। वृको॒ वारो॒ जिघासति॥१॥

%2.4.1.2
तास्त्वं वृ॑त्रहं जहि। वस्व॒स्मभ्य॒मा भ॑र। अग्ने॒ यो नो॑ऽभि॒दास॑ति। स॒मा॒नो यश्च॒ निष्ट्य॑। इ॒ध्मस्ये॑व प्र॒क्षाय॑तः। मा तस्योच्छे॑षि॒ किञ्च॒न। त्वमि॑न्द्राभि॒भूर॑सि। दे॒वो विज्ञा॑तवीर्यः। वृ॒त्र॒हा पु॑रु॒चेत॑नः। अप॒ प्राच॑ इन्द्र॒ विश्वा अ॒मित्रान्॑॥२॥

%2.4.1.3
अपापा॑चो अभिभूते नुदस्व। अपोदी॑चो॒ अप॑शूराध॒रा च॑ ऊ॒रौ। यथा॒ तव॒ शर्म॒न्मदे॑म। तमिन्द्रं॑ वाजयामसि। म॒हे वृ॒त्राय॒ हन्त॑वे। स वृषा॑ वृष॒भो भु॑वत्। यु॒जे रथ॑ङ्ग॒वेष॑ण॒ हरि॑भ्याम्। उप॒ ब्रह्मा॑णि जुजुषा॒णम॑स्थुः। विबा॑धिष्टा॒स्य रोद॑सी महि॒त्वा। इन्द्रो॑ वृ॒त्राण्य॑प्र॒तीज॑घ॒न्वान्॥३॥

%2.4.1.4
ह॒व्य॒वाह॑मभिमाति॒षाहम्। र॒क्षो॒हणं॒ पृत॑नासु जि॒ष्णुम्। ज्योति॑ष्मन्त॒न्दीद्य॑तं॒ पुर॑न्धिम्। अ॒ग्नि स्वि॑ष्ट॒कृत॒मा हु॑वेम। स्वि॑ष्टमग्ने अ॒भि तत्पृ॑णाहि। विश्वा॑ देव॒ पृत॑ना अ॒भि ष्य। उ॒रुन्न॒ पन्थां प्रदि॒शन्विभा॑हि। ज्योति॑ष्मद्धेह्य॒जर॑न्न॒ आयु॑। त्वाम॑ग्ने ह॒विष्म॑न्तः। दे॒वं मर्ता॑स ईडते॥४॥

%2.4.1.5
मन्ये त्वा जा॒तवे॑दसम्। स ह॒व्या व॑क्ष्यानु॒षक्। विश्वा॑नि नो दु॒र्गहा॑ जातवेदः। सिन्धु॒न्न ना॒वा दु॑रि॒ताऽति॑ पर्‌षि। अग्ने॑ अत्रि॒वन्मन॑सा गृणा॒नः। अ॒स्माकं॑ बोध्यवि॒ता त॒नूनाम्। पू॒षा गा अन्वे॑तु नः। पू॒षा र॑क्ष॒त्वर्व॑तः। पू॒षा वाज सनोतु नः। पू॒षेमा आशा॒ अनु॑वेद॒ सर्वा॥५॥

%2.4.1.6
सो अ॒स्मा अभ॑यतमेन नेषत्। स्व॒स्ति॒दा अघृ॑णि॒ सर्व॑वीरः। अप्र॑युच्छन्पु॒र ए॑तु॒ प्रजा॒नन्। त्वम॑ग्ने स॒प्रथा॑ असि। जुष्टो॒ होता॒ वरेण्यः। त्वया॑ य॒ज्ञं वित॑न्वते। अ॒ग्नी रक्षासि सेधति। शु॒क्रशो॑चि॒रम॑र्त्यः। शुचि॑ पाव॒क ईड्य॑। अग्ने॒ रक्षा॑ णो॒ अह॑सः॥६॥

%2.4.1.7
प्रति॑ ष्म देव॒ रीष॑तः। तपि॑ष्ठैर॒जरो॑ दह। अग्ने॒ हसि॒ न्य॑त्रिणम्। दीद्य॒न्मर्त्ये॒ष्वा। स्वे क्षये॑ शुचिव्रत। आ वा॑त वाहि भेष॒जम्। वि वा॑त वाहि॒ यद्रप॑। त्व हि वि॒श्वभे॑षजः। दे॒वानान्दू॒त ईय॑से। द्वावि॒मौ वातौ॑ वातः॥७॥

%2.4.1.8
आ सिन्धो॒रा प॑रा॒वत॑। दक्षं॑ मे अ॒न्य आ॒वातु॑। परा॒न्यो वा॑तु॒ यद्रप॑। यद॒दो वा॑त ते गृ॒हे। अ॒मृत॑स्य नि॒धिर्\mbox{}हि॒तः। ततो॑ नो देहि जी॒वसे। ततो॑ नो धेहि भेष॒जम्। ततो॑ नो॒ मह॒ आव॑ह। वात॒ आवा॑तु भेष॒जम्। श॒म्भूर्म॑यो॒भूर्नो॑ हृ॒दे॥८॥

%2.4.1.9
प्र ण॒ आयूषि तारिषत्। त्वम॑ग्ने अ॒यासि॑। अ॒या सन्मन॑सा हि॒तः। अ॒या सन् ह॒व्यमू॑हिषे। अ॒या नो॑ धेहि भेष॒जम्। इ॒ष्टो अ॒ग्निराहु॑तः। स्वाहा॑कृतः पिपर्तु नः। स्व॒गा दे॒वेभ्य॑ इ॒दन्नम॑। कामो॑ भू॒तस्य॒ भव्य॑स्य। स॒म्राडेको॒ विरा॑जति॥९॥

%2.4.1.10
स इ॒दं प्रति॑ पप्रथे। ऋ॒तूनुत्सृ॑जते व॒शी। काम॒स्तदग्रे॒ सम॑वर्त॒ताधि॑। मन॑सो॒ रेत॑ प्रथ॒मं यदासीत्। स॒तो बन्धु॒मस॑ति॒ निर॑विन्दन्। हृ॒दि प्र॒तीष्या॑ क॒वयो॑ मनी॒षा। त्वया॑ मन्यो स॒रथ॑मारु॒जन्त॑। हर्‌ष॑माणासो धृष॒ता म॑रुत्वः। ति॒ग्मेष॑व॒ आयु॑धा स॒शिशा॑नाः। उप॒ प्रय॑न्ति॒ नरो॑ अ॒ग्निरू॑पाः॥१०॥

%2.4.1.11
म॒न्युर्भगो॑ म॒न्युरे॒वास॑ दे॒वः। म॒न्युर्‌होता॒ वरु॑णो वि॒श्ववे॑दाः। म॒न्युं विश॑ ईडते देव॒यन्ती। पा॒हि नो॑ मन्यो॒ तप॑सा॒ श्रमे॑ण। त्वम॑ग्ने व्रत॒भृच्छुचि॑। दे॒वा आसा॑दया इ॒ह। अग्ने॑ ह॒व्याय॒ वोढ॑वे। व्र॒तानुबिभ्र॑द्व्रत॒पा अदाभ्यः। यजा॑ नो दे॒वा अ॒जर॑ सु॒वीर॑। दध॒द्रत्ना॑नि सुविदा॒नो अ॑ग्ने। गो॒पा॒य नो॑ जी॒वसे॑ जातवेदः॥११॥\anuvakamend[जिघासत्य॒मित्राञ्जघ॒न्वानी॑डते॒ सर्वा॒ अह॑सो वातो हृ॒दे रा॑जत्य॒ग्निरू॑पाः सुविदा॒नो अ॑ग्न॒ एकं च]

%2.4.2.1
चक्षु॑षो हेते॒ मन॑सो हेते। वाचो॑ हेते॒ ब्रह्म॑णो हेते। यो मा॑ऽघा॒युर॑भि॒दास॑ति। तम॑ग्ने मे॒न्या मे॒निं कृ॑णु। यो मा॒ चक्षु॑षा॒ यो मन॑सा। यो वा॒चा ब्रह्म॑णाऽघा॒युर॑भि॒दास॑ति। तयाऽग्ने॒ त्वं मे॒न्या। अ॒मुम॑मे॒निं कृ॑णु। यत्किञ्चा॒सौ मन॑सा॒ यच्च॑ वा॒चा। य॒ज्ञैर्जु॒होति॒ यजु॑षा ह॒विर्भि॑॥१२॥

%2.4.2.2
तन्मृ॒त्युर्निर्\mbox{}ऋ॑त्या संविदा॒नः। पु॒रादि॒ष्टादाहु॑तीरस्य हन्तु। या॒तु॒धाना॒ निर्\mbox{}ऋ॑ति॒रादु॒रक्ष॑। ते अ॑स्य घ्न॒न्त्वनृ॑तेन स॒त्यम्। इन्द्रे॑षिता॒ आज्य॑मस्य मथ्नन्तु। मा तत्समृ॑द्धि॒ यद॒सौ क॒रोति॑। हन्मि॑ ते॒ऽहं कृ॒त ह॒विः। यो मे॑ घो॒रमची॑कृतः। अपाञ्चौ त उ॒भौ बा॒हू। अप॑नह्याम्या॒स्यम्॥१३॥

%2.4.2.3
अप॑ नह्यामि ते बा॒हू। अप॑ नह्याम्या॒स्यम्। अ॒ग्नेर्दे॒वस्य॒ ब्रह्म॑णा। सर्व॑न्तेऽवधिषं कृ॒तम्। पु॒राऽमुष्य॑ वषट्का॒रात्। य॒ज्ञन्दे॒वेषु॑ नस्कृधि। स्वि॑ष्टम॒स्माकं॑ भूयात्। माऽस्मान्प्राप॒न्नरा॑तयः। अन्ति॑ दू॒रे स॒तो अ॑ग्ने। भ्रातृ॑व्यस्याभि॒दास॑तः॥१४॥

%2.4.2.4
व॒ष॒ट्का॒रेण॒ वज्रे॑ण। कृ॒त्या ह॑न्मि कृ॒ताम॒हम्। यो मा॒ नक्त॒न्दिवा॑ सा॒यम्। प्रा॒तश्चाह्नो॑ नि॒पीय॑ति। अ॒द्या तमि॑न्द्र॒ वज्रे॑ण। भातृ॑व्यं पादयामसि। इन्द्र॑स्य गृ॒हो॑ऽसि॒ तन्त्वा। प्रप॑द्ये॒ सगु॒ साश्व॑। स॒ह यन्मे॒ अस्ति॒ तेन॑। ईडे॑ अ॒ग्निं वि॑प॒श्चितम्॥१५॥

%2.4.2.5
गि॒रा य॒ज्ञस्य॒ साध॑नम्। श्रु॒ष्टी॒वान॑न्धि॒तावा॑नम्। अग्ने॑ श॒केम॑ ते व॒यम्। यमं॑ दे॒वस्य॑ वा॒जिन॑। अति॒ द्वेषासि तरेम। अव॑तं मा॒ सम॑नसौ॒ समो॑कसौ। सचे॑तसौ॒ सरे॑तसौ। उ॒भौ माम॑वतञ्जातवेदसौ। शि॒वौ भ॑वतम॒द्य न॑। स्व॒यं कृ॑ण्वा॒नः सु॒गमप्र॑यावम्॥१६॥

%2.4.2.6
ति॒ग्मशृ॑ङ्गो वृष॒भः शोशु॑चानः। प्र॒त्न स॒धस्थ॒मनु॒ पश्य॑मानः। आ तन्तु॑म॒ग्निर्दि॒व्यन्त॑तान। त्वन्न॒स्तन्तु॑रु॒त सेतु॑रग्ने। त्वं पन्था॑ भवसि देव॒यान॑। त्वयाऽग्ने पृ॒ष्ठं व॒यमारु॑हेम। अथा॑ दे॒वैः स॑ध॒मादं॑ मदेम। उदु॑त्त॒मं मु॑मुग्धि नः। वि पाशं॑ मध्य॒मञ्चृ॑त। अवा॑ध॒मानि॑ जी॒वसे॥१७॥

%2.4.2.7
व॒य सो॑म व्र॒ते तव॑। मन॑स्त॒नूषु॒ बिभ्र॑तः। प्र॒जाव॑न्तो अशीमहि। इ॒न्द्रा॒णी दे॒वी सु॒भगा॑ सु॒पत्नी। उदशे॑न पति॒विद्ये॑ जिगाय। त्रि॒शद॑स्या ज॒घन॒य्योँज॑नानि। उ॒पस्थ॒ इन्द्र॒ स्थवि॑रं बिभर्ति। सेना॑ ह॒ नाम॑ पृथि॒वी ध॑नञ्ज॒या। वि॒श्वव्य॑चा॒ अदि॑ति॒ सूर्य॑त्वक्। इ॒न्द्रा॒णी दे॒वी प्रा॒सहा॒ ददा॑ना॥१८॥

%2.4.2.8
सा नो॑ दे॒वी सु॒हवा॒ शर्म॑ यच्छतु। आत्वा॑ऽहार्‌षम॒न्तर॑भूः। ध्रु॒वस्ति॒ष्ठावि॑चाचलिः। विश॑स्त्वा॒ सर्वा॑ वाञ्छन्तु। मा त्वद्रा॒ष्ट्रमधि॑ भ्रशत्। ध्रु॒वा द्यौर्ध्रु॒वा पृ॑थि॒वी। ध्रु॒वं विश्व॑मि॒दञ्जग॑त्। ध्रु॒वा ह॒ पर्व॑ता इ॒मे। ध्रु॒वो राजा॑ वि॒शाम॒यम्। इ॒हैवैधि॒ मा व्य॑थिष्ठाः॥१९॥

%2.4.2.9
पर्व॑त इ॒वावि॑चाचलिः। इन्द्र॑ इवे॒ह ध्रु॒वस्ति॑ष्ठ। इ॒ह रा॒ष्ट्रमु॑ धारय। अ॒भिति॑ष्ठ पृतन्य॒तः। अध॑रे सन्तु॒ शत्र॑वः। इन्द्र॑ इव वृत्र॒हा ति॑ष्ठ। अ॒पः क्षेत्रा॑णि स॒ञ्जय\sn{}। इन्द्र॑ एणमदीधरत्। ध्रु॒वन्ध्रु॒वेण॑ ह॒विषा। तस्मै॑ दे॒वा अधि॑ब्रवन्। अ॒यं च॒ ब्रह्म॑ण॒स्पति॑॥२०॥\anuvakamend[ह॒विर्भि॑रा॒स्य॑मभि॒ दास॑तो विप॒श्चित॒मप्र॑यावञ्जी॒वसे॒ ददा॑ना व्यथिष्ठा ब्रव॒न्नेकं च]

%2.4.3.1
जुष्टी॑ नरो॒ ब्रह्म॑णा वः पितृ॒णाम्। अक्ष॑मव्यय॒न्न किला॑रिषाथ। यच्छक्व॑रीषु बृह॒ता रवे॑ण। इन्द्रे॒ शुष्म॒मद॑धाथा वसिष्ठाः। पा॒व॒का न॒ सर॑स्वती। वाजे॑भिर्वा॒जिनी॑वती। य॒ज्ञं व॑ष्टु धि॒या व॑सुः। सर॑स्वत्य॒भिनो॑ नेषि॒ वस्य॑। मा प॑स्फरी॒ पय॑सा॒ मा न॒ आध॑क्। जु॒षस्व॑ नः स॒ख्या॑ वे॒श्या॑ च॥२१॥

%2.4.3.2
मा त्वक्षेत्रा॒ण्यर॑णानि गन्म। वृ॒ञ्जे ह॒विर्नम॑सा ब॒र्॒हिर॒ग्नौ। अया॑मि॒ स्रुग्घृ॒तव॑ती सुवृ॒क्तिः। अम्य॑क्षि॒ सद्म॒ सद॑ने पृथि॒व्याः। अश्रा॑यि य॒ज्ञः सूर्ये॒ न चक्षु॑। इ॒हार्वाञ्च॒मति॑ ह्वये। इन्द्रं॒ जैत्रा॑य॒ जेत॑वे। अ॒स्माक॑मस्तु॒ केव॑लः। अ॒र्वाञ्च॒मिन्द्र॑म॒मुतो॑ हवामहे। यो गो॒जिद्ध॑न॒जिद॑श्व॒जिद्यः॥२२॥

%2.4.3.3
इ॒मन्नो॑ य॒ज्ञं वि॑ह॒वे जु॑षस्व। अ॒स्य कु॑र्मो हरिवो मे॒दिन॑न्त्वा। असं॑मृष्टो जायसे मातृ॒वोः शुचि॑। म॒न्द्रः क॒विरुद॑तिष्ठो॒ विव॑स्वतः। घृ॒तेन॑ त्वा वर्धयन्नग्न आहुत। धू॒मस्ते॑ के॒तुर॑भवद्दि॒वि श्रि॒तः। अ॒ग्निरग्रे प्रथ॒मो दे॒वता॑नाम्। संया॑तानामुत्त॒मो विष्णु॑रासीत्। यज॑मानाय परि॒गृह्य॑ दे॒वान्। दी॒क्षये॒द ह॒विरा ग॑च्छतन्नः॥२३॥

%2.4.3.4
अ॒ग्निश्च॑ विष्णो॒ तप॑ उत्त॒मं म॒हः। दी॒क्षा॒पा॒लेभ्यो॒ऽवन॑त॒ हि श॑क्रा। विश्वैर्दे॒वैर्य॒ज्ञियै संविदा॒नौ। दी॒क्षाम॒स्मै यज॑मानाय धत्तम्। प्र तद्विष्णु॑ स्तवते वी॒र्या॑य। मृ॒गो न भी॒मः कु॑च॒रो गि॑रि॒ष्ठाः। यस्यो॒रुषु॑ त्रि॒षु वि॒क्रम॑णेषु। अधि॑ क्षि॒यन्ति॒ भुव॑नानि॒ विश्वा। नूमर्तो॑ दयते सनि॒ष्यन् यः। विष्ण॑व उरुगा॒याय॒ दाश॑त्॥२४॥

%2.4.3.5
प्र यः स॒त्राचा॒ मन॑सा॒ यजा॑तै। ए॒ताव॑न्त॒न्नर्य॑मा॒ विवा॑सात्। विच॑क्रमे पृथि॒वीमे॒ष ए॒ताम्। क्षेत्रा॑य॒ विष्णु॒र्मनु॑षे दश॒स्यन्। ध्रु॒वासो॑ अस्य की॒रयो॒ जना॑सः। उ॒रु॒क्षि॒ति सु॒जनि॑मा चकार। त्रिर्दे॒वः पृ॑थि॒वीमे॒ष ए॒ताम्। विच॑क्रमे श॒तर्च॑सं महि॒त्वा। प्र विष्णु॑रस्तु त॒वस॒स्तवी॑यान्। त्वे॒ष ह्य॑स्य॒ स्थवि॑रस्य॒ नाम॑॥२५॥

%2.4.3.6
होता॑रञ्चि॒त्रर॑थमध्व॒रस्य॑। य॒ज्ञस्य॑यज्ञस्य के॒तु रुश॑न्तम्। प्रत्य॑र्धिन्दे॒वस्य॑देवस्य म॒ह्ना। श्रि॒या त्व॑ग्निमति॑थिं॒ जना॑नाम्। आ नो॒ विश्वा॑भिरू॒तिभि॑ स॒जोषा। ब्रह्म॑ जुषा॒णो ह॑र्यश्व याहि। वरी॑वृज॒त्स्थवि॑रेभिः सुशिप्र। अ॒स्मे दध॒द्वृष॑ण॒ शुष्म॑मिन्द्र। इन्द्र॑ सुव॒र्॒षा ज॒नय॒न्नहा॑नि। जि॒गायो॒शिग्भि॒ पृत॑ना अभि॒ श्रीः॥२६॥

%2.4.3.7
प्रारो॑चय॒न्मन॑वे के॒तुमह्नाम्। अवि॑न्द॒ज्ज्योति॑र्बृह॒ते रणा॑य। अश्वि॑ना॒वव॑से॒ निह्व॑ये वाम्। आ नू॒नं या॑त सुकृ॒ताय॑ विप्रा। प्रा॒त॒र्यु॒क्तेन॑ सु॒वृता॒ रथे॑न। उ॒पाग॑च्छत॒मव॒साग॑तन्नः। अ॒वि॒ष्टन्धी॒ष्वश्वि॑ना न आ॒सु। प्र॒जाव॒द्रेतो॒ अह्र॑यन्नो अस्तु। आवान्तो॒के तन॑ये॒ तूतु॑जानाः। सु॒रत्ना॑सो दे॒ववी॑तिङ्गमेम॥२७॥

%2.4.3.8
त्व सो॑म॒ क्रतु॑भिः सु॒क्रतु॑र्भूः। त्वदन्दक्षै सु॒दक्षो॑ वि॒श्ववे॑दाः। त्वं वृषा॑ वृष॒त्वेभि॑र्महि॒त्वा। द्यु॒म्नेभि॑र्द्यु॒म्न्य॑भवो नृ॒चक्षा। अषा॑ढय्युँ॒त्सु पृत॑नासु॒ पप्रिम्। सु॒व॒र्॒षाम॒प्स्वां वृ॒जन॑स्य गो॒पाम्। भ॒रे॒षु॒जा सु॑क्षि॒ति सु॒श्रव॑सम्। जय॑न्त॒न्त्वामनु॑ मदेम सोम। भवा॑ मि॒त्रो न शेव्यो॑ घृ॒तासु॑तिः। विभू॑तद्युम्न एव॒ या उ॑ स॒प्रथा॥२८॥

%2.4.3.9
अधा॑ ते विष्णो वि॒दुषा॑ चि॒दृध्य॑। स्तोमो॑ य॒ज्ञस्य॒ राध्यो॑ ह॒विष्म॑तः। यः पू॒र्व्याय॑ वे॒धसे॒ नवी॑यसे। सु॒मज्जा॑नये॒ विष्ण॑वे॒ ददा॑शति। यो जा॒तम॒स्य म॑ह॒तो म॒हि ब्रवात्। सेदु॒ श्रवो॑भिर्यु॒ज्यं॑         चिद॒भ्य॑सत्। तमु॑ स्तोतारः पू॒र्व्यं यथा॑ वि॒द ऋ॒तस्य॑। गर्भ ह॒विषा॑ पिपर्तन। आऽस्य॑ जा॒नन्तो॒ नाम॑ चिद्विवक्तन। बृ॒हत्ते॑ विष्णो सुम॒तिं भ॑जामहे॥२९॥

%2.4.3.10
इ॒मा धा॒ना घृ॑त॒स्नुव॑। हरी॑ इ॒होप॑वक्षतः। इन्द्र सु॒खत॑मे॒ रथे। ए॒ष ब्र॒ह्मा प्रते॑म॒हे। वि॒दथे॑ शसिष॒ हरी। य ऋ॒त्विय॒ प्रते॑ वन्वे। व॒नुषो॑ हर्य॒तं मदम्। इन्द्रो॒ नाम॑ घृ॒तन्नयः। हरि॑भि॒श्चारु॒ सेच॑ते। श्रु॒तो ग॒ण आ त्वा॑ विशन्तु॥३०॥

%2.4.3.11
हरि॑वर्पस॒ङ्गिर॑। आच॑र्‌षणि॒प्रा वृ॑ष॒भो जना॑नाम्। राजा॑ कृष्टी॒नां पु॑रुहू॒त इन्द्र॑। स्तु॒तश्र॑व॒स्यन्नव॒सोप॑म॒द्रिक्। यु॒क्त्वा हरी॒ वृष॒णायाह्य॒र्वाङ्। प्र यत्सिन्ध॑वः प्रस॒वं यदाय\sn{}। आप॑ समु॒द्र र॒थ्ये॑व जग्मुः। अत॑श्चि॒दिन्द्र॒ सद॑सो॒ वरी॑यान्। यदी॒ सोम॑ पृ॒णति॑ दु॒ग्धो अ॒शुः। ह्वया॑मसि॒ त्वेन्द्र॑ या॒ह्य॑र्वाङ्॥३१॥

%2.4.3.12
अर॑न्ते॒ सोम॑स्त॒नुवे॑ भवाति। शत॑क्रतो मा॒दय॑स्वा सु॒तेषु॑। प्रास्मा अ॑व॒ पृत॑नासु॒ प्रयु॒त्सु। इन्द्रा॑य॒ सोमा प्र॒दिवो॒ विदा॑नाः। ऋ॒भुर्येभि॒र्वृष॑पर्वा॒ विहा॑याः। प्र॒य॒म्यमा॑णा॒न्प्रति॒ षू गृ॑भाय। इन्द्र॒ पिब॒ वृष॑धूतस्य॒ वृष्ण॑। अहे॑डमान॒ उप॑याहि य॒ज्ञम्। तुभ्यं॑ पवन्त॒ इन्द॑वः सु॒तास॑। गावो॒ न व॑ज्रिन्त्स्व॒मोको॒ अच्छ॑॥३२॥

%2.4.3.13
इन्द्रा ग॑हि प्रथ॒मो य॒ज्ञिया॑नाम्। या ते॑ का॒कुत्सुकृ॑ता॒ या वरि॑ष्ठा। यया॒ शश्व॒त्पिब॑सि॒ मध्व॑ ऊ॒र्मिम्। तया॑ पाहि॒ प्र ते॑ अध्व॒र्युर॑स्थात्। सन्ते॒ वज्रो॑ वर्ततामिन्द्र ग॒व्युः। प्रा॒त॒र्युजा॒ वि बो॑धय। अश्वि॑ना॒वेह ग॑च्छतम्। अ॒स्य सोम॑स्य पी॒तये। प्रा॒त॒र्यावा॑णा प्रथ॒मा य॑जध्वम्। पु॒रा गृध्रा॒दर॑रुषः पिबाथः। प्रा॒तर्\mbox{}हि य॒ज्ञम॒श्विना॒ दधा॑ते। प्रशसन्ति क॒वय॑ पूर्व॒भाज॑। प्रा॒तर्य॑जध्वम॒श्विना॑ हिनोत। न सा॒यम॑स्ति देव॒या अजु॑ष्टम्। उ॒तान्यो अ॒स्मद्य॑जते॒ विचा॑यः। पूर्व॑ पूर्वो॒ यज॑मानो॒ वनी॑यान्॥३३॥\anuvakamend[चा॒श्व॒जिद्यो ग॑च्छतन्नो॒ दाश॒न्नामा॑भि॒श्रीर्ग॑मेम स॒प्रथा॑ भजामहे विशन्तु या॒ह्य॑र्वाङच्छ॑ पिबाथ॒ष्षट्च॑]

%2.4.4.1
न॒क्तं॒ जा॒ताऽस्यो॑षधे। रामे॒ कृष्णे॒ असि॑क्नि च। इ॒द र॑जनि रजय। कि॒लासं॑ पलि॒तं च॒ यत्। कि॒लास॑ञ्च पलि॒तं च॑। निरि॒तो ना॑शया॒ पृष॑त्। आ न॒ स्वो अ॑श्ञुतां॒ वर्ण॑। परा श्वे॒तानि॑ पातय। असि॑तन्ते नि॒लय॑नम्। आ॒स्थान॒मसि॑त॒न्तव॑॥३४॥

%2.4.4.2
असि॑क्नियस्योषधे। निरि॒तो ना॑शया॒ पृष॑त्। अ॒स्थि॒जस्य॑ कि॒लास॑स्य। त॒नू॒जस्य॑ च॒ यत्त्व॒चि। कृ॒त्यया॑ कृ॒तस्य॒ ब्रह्म॑णा। लक्ष्म॑ श्वे॒तम॑नीनशम्। सरू॑पा॒ नाम॑ ते मा॒ता। सरू॑पो॒ नाम॑ ते पि॒ता। सरू॑पाऽस्योषधे॒ सा। सरू॑पमि॒दं कृ॑धि॥३५॥

%2.4.4.3
शु॒न हु॑वेम म॒घवा॑न॒मिन्द्रम्। अ॒स्मिन्भरे॒ नृत॑मं॒ वाज॑सातौ। शृ॒ण्वन्त॑मु॒ग्रमू॒तये॑ स॒मत्सु॑। घ्नन्तं॑ वृ॒त्राणि॑ स॒ञ्जितं॒ धना॑नाम्। धू॒नु॒थ द्यां पर्व॑तान्दा॒शुषे॒ वसु॑। नि वो॒ वना॑ जिहते॒ याम॑ नो भि॒या। को॒पय॑थ पृथि॒वीं पृ॑श्ञिमातरः। यु॒धे यदु॑ग्रा॒ पृष॑ती॒रयु॑ग्ध्वम्। प्रवे॑पयन्ति॒ पर्व॑तान्। विवि॑ञ्चन्ति॒ वन॒स्पतीन्॑॥३६॥

%2.4.4.4
प्रोवा॑रत मरुतो दु॒र्मदा॑ इव। देवा॑स॒ सर्व॑या वि॒शा। पु॒रु॒त्रा हि स॒दृङ्ङसि॑। विशो॒ विश्वा॒ अनु॑ प्र॒भु। स॒मत्सु॑ त्वा हवामहे। स॒मत्स्व॒ग्निमव॑से। वा॒ज॒यन्तो॑ हवामहे। वाजे॑षु चि॒त्ररा॑धसम्। सङ्ग॑च्छध्व॒ संव॑दध्वम्। सव्वोँ॒ मनासि जानताम्॥३७॥

%2.4.4.5
दे॒वा भा॒गं यथा॒ पूर्वे। स॒ञ्जा॒ना॒ना उ॒पास॑त। स॒मा॒नो मन्त्र॒ समि॑तिः समा॒नी। स॒मा॒नं मन॑ स॒ह चि॒त्तमे॑षाम्। स॒मा॒नङ्केतो॑ अ॒भि स र॑भध्वम्। सं॒ज्ञाने॑न वो ह॒विषा॑ यजामः। स॒मा॒नी व॒ आकू॑तिः। स॒मा॒ना हृद॑यानि वः। स॒मा॒नम॑स्तु वो॒ मन॑। यथा॑ व॒ सुस॒हास॑ति॥३८॥

%2.4.4.6
सं॒ज्ञान॑न्न॒ स्वैः। सं॒ज्ञान॒मर॑णैः। सं॒ज्ञान॑मश्विना यु॒वम्। इ॒हास्मासु॒ निय॑च्छतम्। सं॒ज्ञानं॑ मे॒ बृह॒स्पति॑। सं॒ज्ञान सवि॒ता क॑रत्। सं॒ज्ञान॑मश्विना यु॒वम्। इ॒ह मह्यं॒ नि य॑च्छतम्। उप॑ च्छा॒यामि॑व॒ घृणे। अग॑न्म॒ शर्म॑ ते व॒यम्॥३९॥

%2.4.4.7
अग्ने॒ हिर॑ण्यसन्दृशः। अद॑ब्धेभिः सवितः पा॒युभि॒ष्ट्वम्। शि॒वेभि॑र॒द्य परि॑पाहि नो॒ गयम्। हिर॑ण्यजिह्वः सुवि॒ताय॒ नव्य॑से। रक्षा॒ माकि॑र्नो अ॒घशस ईशत। मदे॑मदे॒ हि नो॑ द॒दुः। यू॒था गवा॑मृजु॒क्रतु॑। सङ्गृ॑भाय पु॒रूश॒ता। उ॒भ॒या ह॒स्त्या वसु॑। शि॒शी॒हि रा॒य आ भ॑र॥४०॥

%2.4.4.8
शिप्रि॑न्वाजानां पते। शची॑व॒स्तव॑ द॒सना। आ तू न॑ इन्द्र भाजय। गोष्वश्वे॑षु शु॒भ्रुषु॑। स॒हस्रे॑षु तुवीमघ। यद्दे॑वा देव॒हेड॑नम्। देवा॑सश्चकृ॒मा व॒यम्। आदि॑त्या॒स्तस्मान्मा यू॒यम्। ऋ॒तस्य॒र्तेन॑ मुञ्चत। ऋ॒तस्य॒र्तेना॑दित्याः॥४१॥

%2.4.4.9
यज॑त्रा मु॒ञ्चते॒ह मा। य॒ज्ञैर्वो॑ यज्ञवाहसः। आ॒शिक्ष॑न्तो॒ न शे॑किम। मेद॑स्वता॒ यज॑मानाः। स्रु॒चाऽऽज्ये॑न॒ जुह्व॑तः। अ॒का॒मा वो॑ विश्वेदेवाः। शिक्ष॑न्तो॒ नोप॑ शेकिम। यदि॒ दिवा॒ यदि॒ नक्तम्। एन॑ एन॒स्योक॑रत्। भू॒तं मा॒ तस्मा॒द्भव्यं॑ च॥४२॥

%2.4.4.10
द्रु॒प॒दादि॑व मुञ्चतु। द्रु॒प॒दादि॒वेन्मु॑मुचा॒नः। स्वि॒न्नः स्ना॒त्वी मला॑दिव। पू॒तं प॒वित्रे॑णे॒वाज्यम्। विश्वे॑ मुञ्चन्तु॒ मैन॑सः। उद्व॒यन्तम॑स॒स्परि॑। पश्य॑न्तो॒ ज्योति॒रुत्त॑रम्। दे॒वन्दे॑व॒त्रा सूर्यम्। अग॑न्म॒ ज्योति॑रुत्त॒मम्॥४३॥\anuvakamend[तव॑ कृधि॒ वन॒स्पतीञ्जानता॒मस॑ति व॒यं भ॑रादित्याश्च॒ नव॑ च]

%2.4.5.1
वृषा॒सो अ॒शुः प॑वते ह॒विष्मा॒न्त्सोम॑। इन्द्र॑स्य भा॒ग ऋ॑त॒युः श॒तायु॑। स मा॒ वृषा॑णं वृष॒भं कृ॑णोतु। प्रि॒यं वि॒शा सर्व॑वीर सु॒वीरम्। कस्य॒ वृषा॑ सु॒ते सचा। नि॒युत्वान्वृष॒भो र॑णत्। वृ॒त्र॒हा सोम॑पीतये। यस्ते॑ शृङ्ग वृषोनपात्। प्रण॑पात्कुण्ड॒पाय्य॑। न्य॑स्मिन्दध्र॒ आ मन॑॥४४॥

%2.4.5.2
त स॒ध्रीची॑रू॒तयो॒ वृष्णि॑यानि। पौस्या॑नि नि॒युत॑ सश्चु॒रिन्द्रम्। स॒मु॒द्रन्न सिन्ध॑व उ॒क्थशु॑ष्माः। उ॒रु॒व्यच॑स॒ङ्गिर॒ आ वि॑शन्ति। इन्द्रा॑य॒ गिरो॒ अनि॑शितसर्गाः। अ॒पः प्रैर॑य॒न्त्सग॑रस्य बु॒ध्नात्। यो अक्षे॑णेव च॒क्रिया॒ शची॑भिः। विष्व॑क्त॒स्तम्भ॑ पृथि॒वीमु॒त द्याम्। अक्षो॑दय॒च्छव॑सा॒ क्षाम॑बु॒ध्नम्। वार्णवा॑त॒स्तवि॑षीभि॒रिन्द्र॑॥४५॥

%2.4.5.3
दृ॒ढान्यौघ्नादु॒शमा॑न॒ ओज॑। अवा॑भिनत्क॒कुभ॒ पर्व॑तानाम्। आ नो॑ अग्ने सुके॒तुना। र॒यिं वि॒श्वायु॑पोषसम्। मा॒र्डी॒कन्धे॑हि जी॒वसे। त्व सो॑म म॒हे भगम्। त्वं यून॑ ऋताय॒ते। दक्ष॑न्दधासि जी॒वसे। रथ॑य्युँञ्जते म॒रुत॑ शु॒भे सु॒गम्। सूरो॒ न मि॑त्रावरुणा॒ गवि॑ष्टिषु॥४६॥

%2.4.5.4
रजासि चि॒त्रा विच॑रन्ति त॒न्यव॑। दि॒वः स॑म्राजा॒ पय॑सा न उक्षतम्। वाच॒ सुमि॑त्रावरुणा॒विरा॑वतीम्। प॒र्जन्य॑श्चि॒त्रां व॑दति॒ त्विषी॑मतीम्। अ॒भ्रा व॑सत मरुतः सुमा॒यया। द्यां व॑र्‌षयतमरु॒णाम॑रे॒पसम्। अयु॑क्त स॒प्त शु॒न्ध्युव॑। सूरो॒ रथ॑स्य न॒प्त्रिय॑। ताभि॑र्याति॒ स्वयु॑क्तिभिः। वहि॑ष्ठेभिर्\mbox{}वि॒हर॑न् यासि॒ तन्तुम्॥४७॥

%2.4.5.5
अ॒व॒व्यय॒न्नसि॑तन्देव॒ वस्व॑। दवि॑ध्वतो र॒श्मय॒ सूर्य॑स्य। चर्मे॒वावा॑धु॒स्तमो॑ अ॒प्स्व॑न्तः। प॒र्जन्या॑य॒ प्र गा॑यत। दि॒वस्पु॒त्राय॑ मी॒ढुषे। स नो॑ य॒वस॑मिच्छतु। अच्छा॑ वद त॒वस॑ङ्गी॒र्भिरा॒भिः। स्तु॒हि प॒र्जन्य॒न्नम॒साऽऽवि॑वास। कनि॑क्रदद्वृष॒भो जी॒रदा॑नुः। रेतो॑ दधा॒त्वोष॑धीषु॒ गर्भम्॥४८॥

%2.4.5.6
यो गर्भ॒मोष॑धीनाम्। गवां कृ॒णोत्यर्व॑ताम्। प॒र्जन्य॑ पुरु॒षीणाम्। तस्मा॒ इदा॒स्ये॑ ह॒विः। जू॒होता॒ मधु॑मत्तमम्। इडान्नः सं॒यत॑ङ्करत्। ति॒स्रो यद॑ग्ने श॒रद॒स्त्वामित्। शुचि॑ङ्घृ॒तेन॒ शुच॑यः सप॒र्यन्। नामा॑नि चिद्दधिरे य॒ज्ञिया॑नि। असू॑दयन्त त॒नुव॒ सुजा॑ताः॥४९॥

%2.4.5.7
इन्द्र॑श्च नः शुनासीरौ। इ॒मं य॒ज्ञं मि॑मिक्षतम्। गर्भ॑न्धत्त स्व॒स्तये। ययो॑रि॒दं विश्वं॒ भुव॑नमा वि॒वेश॑। ययो॑रान॒न्दो निहि॑तो॒ मह॑श्च। शुना॑सीरावृ॒तुभि॑ संविदा॒नौ। इन्द्र॑वन्तौ ह॒विरि॒दं जु॑षेथाम्। आघा॒ये अ॒ग्निमि॑न्ध॒ते। स्तृ॒णन्ति॑ ब॒र्॒हिरा॑नु॒षक्। येषा॒मिन्द्रो॒ युवा॒ सखा। अग्न॒ इन्द्र॑श्च मे॒दिना। ह॒थो वृ॒त्राण्य॑प्र॒ति। यु॒व हि वृ॑त्र॒हन्त॑मा। याभ्या॒ सुव॒रज॑य॒न्नग्र॑ ए॒व। यावा॑तस्थ॒तुर्भुव॑नस्य॒ मध्ये। प्रच॑र्‌ष॒णी वृ॑षणा॒ वज्र॑बाहू। अ॒ग्नी इन्द्रा॑वृत्र॒हणा॑ हुवे वाम्॥५०॥\anuvakamend[मन॒ इन्द्रो॒ गवि॑ष्टिषु॒ तन्तु॒ङ्गर्भ॒ सुजा॑ता॒ सखा॑ स॒प्त च॑]

%2.4.6.1
उ॒त न॑ प्रि॒या प्रि॒यासु॑। स॒प्तस्वसा॒ सुजु॑ष्टा। सर॑स्वती॒ स्तोम्या॑ऽभूत्। इ॒मा जुह्वा॑नायु॒ष्मदा नमो॑भिः। प्रति॒ स्तोम सरस्वति जुषस्व। तव॒ शर्म॑न्प्रि॒यत॑मे॒ दधा॑नाः। उप॑स्थेयाम शर॒णन्न वृ॒क्षम्। त्रिणि॑ प॒दा विच॑क्रमे। विष्णु॑र्गो॒पा अदाभ्यः। ततो॒ धर्मा॑णि धा॒रय\sn{}॥५१॥

%2.4.6.2
तद॑स्य प्रि॒यम॒भि पाथो॑ अश्याम्। नरो॒ यत्र॑ देव॒यवो॒ मद॑न्ति। उ॒रु॒क्र॒मस्य॒ स हि बन्धु॑रि॒त्था। विष्णो प॒दे प॑र॒मे मध्व॒ उत्स॑। क्र॒त्वा॒दा अ॑स्थु॒ श्रेष्ठ॑। अ॒द्य त्वा॑ व॒न्वन्त्सु॒रेक्णा। मर्त॑ आनाश सुवृ॒क्तिम्। इ॒मा ब्र॑ह्म ब्रह्मवाह। प्रि॒या त॒ आ ब॒र्॒हिः सी॑द। वी॒हि सू॑र पुरो॒डाशम्॥५२॥

%2.4.6.3
उप॑ नः सू॒नवो॒ गिर॑। शृ॒ण्वन्त्व॒मृत॑स्य॒ ये। सु॒मृ॒डी॒का भ॑वन्तु नः। अ॒द्या नो॑ देव सवितः। प्र॒जाव॑त्सावी॒ सौभ॑गम्। परा॑ दु॒ष्वप्नि॑य सुव। विश्वा॑नि देव सवितः। दु॒रि॒तानि॒ परा॑ सुव। यद्भ॒द्रन्तन्म॒ आ सु॑व। शुचि॑म॒र्कैर्बृह॒स्पतिम्॥५३॥

%2.4.6.4
अ॒ध्व॒रेषु॑ नमस्यत। अ॒ना॒म्योज॒ आ च॑के। या धा॒रय॑न्त दे॒वा सु॒दक्षा॒ दक्ष॑पितारा। अ॒सु॒र्या॑य॒ प्रम॑हसा। स इत् क्षेति॒ सुधि॑त॒ ओक॑सि॒ स्वे। तस्मा॒ इडा॑ पिन्वते विश्व॒दानी। तस्मै॒ विश॑ स्व॒यमे॒वान॑मन्ति। यस्मि॑न्ब्र॒ह्मा राज॑नि॒ पूर्व॒ एति॑। सकू॑तिमिन्द्र॒ सच्यु॑तिम्। सच्यु॑तिञ्ज॒घन॑च्युतिम्॥५४॥

%2.4.6.5
क॒नात्का॒भान्न॒ आ भ॑र। प्र॒य॒प्स्यन्नि॑व स॒क्थ्यौ। वि न॑ इन्द्र॒ मृधो॑ जहि। कनी॑खुनदिव सा॒पय\sn{}। अ॒भि न॒ सुष्टु॑तिन्नय। प्र॒जाप॑तिः स्त्रि॒यां यश॑। मु॒ष्कयो॑रदधा॒त्सपम्। काम॑स्य॒ तृप्ति॑मान॒न्दम्। तस्याग्ने भाजये॒ह मा। मोद॑ प्रमो॒द आ॑न॒न्दः॥५५॥

%2.4.6.6
मु॒ष्कयो॒र्निहि॑त॒ सप॑। सृ॒त्वेव॒ काम॑स्य तृप्याणि। दक्षि॑णानां प्रतिग्र॒हे। मन॑सश्चि॒त्तमाकू॑तिम्। वा॒चः स॒त्यम॑शीमहि। प॒शू॒ना रू॒पमन्न॑स्य। यश॒ श्रीः श्र॑यतां॒ मयि॑। यथा॒ऽहम॒स्या अतृ॑प स्त्रि॒यै पुमान्॑। यथा॒ स्त्री तृप्य॑ति पु॒सि प्रि॒ये प्रि॒या। ए॒वं भग॑स्य तृप्याणि॥५६॥

%2.4.6.7
य॒ज्ञस्य॒ काम्य॑ प्रि॒यः। ददा॒मीत्य॒ग्निर्व॑दति। तथेति॑ वा॒युरा॑ह॒ तत्। हन्तेति॑ स॒त्यञ्च॒न्द्रमा। आ॒दि॒त्यः स॒त्यमोमिति॑। आप॒स्तत्स॒त्यमा भ॑रन्। यशो॑ य॒ज्ञस्य॒ दक्षि॑णाम्। अ॒सौ मे॒ काम॒ समृ॑द्ध्यताम्। न हि स्पश॒मवि॑दन्न॒न्यम॒स्मात्। वै॒श्वा॒न॒रात्पु॑रए॒तार॑म॒ग्नेः॥५७॥

%2.4.6.8
अथे॑ममन्थन्न॒मृत॒ममू॑राः। वै॒श्वा॒न॒रङ्क्षेत्र॒जित्या॑य दे॒वाः। येषा॑मि॒मे पूर्वे॒ अर्मा॑स॒ आस\sn{}। अ॒यू॒पाः सद्म॒ विभृ॑ता पु॒रूणि॑। वैश्वा॑नर॒ त्वया॒ ते नु॒त्ताः। पृ॒थि॒वीम॒न्याम॒भित॑स्थु॒र्जना॑सः। पृ॒थि॒वीं मा॒तरं॑ म॒हीम्। अ॒न्तरि॑क्ष॒मुप॑ ब्रुवे। बृ॒ह॒तीमू॒तये॒ दिवम्। विश्वं॑ बिभर्ति पृथि॒वी॥५८॥

%2.4.6.9
अ॒न्तरि॑क्षं॒ वि प॑प्रथे। दु॒हे द्यौर्बृ॑ह॒ती पय॑। न ता न॑शन्ति॒ न द॑भाति॒ तस्क॑रः। नैना॑ अमि॒त्रो व्यथि॒राद॑धर्‌षति। दे॒वाश्च॒ याभि॒र्यज॑ते॒ ददा॑ति च। ज्योगित्ताभि॑ सचते॒ गोप॑तिः स॒ह। न ता अर्वा॑ रे॒णुक॑काटो अश्ञुते। न सस्कृत॒त्रमुप॑ यन्ति॒ ता अ॒भि। उ॒रु॒गा॒यमभ॑य॒न्तस्य॒ ता अनु॑। गावो॒ मर्त्य॑स्य॒ वि च॑रन्ति॒ यज्व॑नः॥५९॥

%2.4.6.10
रात्री॒ व्य॑ख्यदाय॒ती। पु॒रु॒त्रा दे॒व्य॑क्षभि॑। विश्वा॒ अधि॒ श्रियो॑ऽधित। उप॑ ते॒ गा इ॒वाक॑रम्। वृ॒णी॒ष्व दु॑हितर्दिवः। रात्री॒ स्तोम॒न्न जि॒ग्युषी। दे॒वीं वाच॑मजनयन्त दे॒वाः। तां वि॒श्वरू॑पाः प॒शवो॑ वदन्ति। सा नो॑ म॒न्द्रेष॒मूर्ज॒न्दुहा॑ना। धे॒नुर्वाग॒स्मानुप॒ सुष्टु॒तैतु॑॥६०॥

%2.4.6.11
यद्वाग्वद॑न्त्यविचेत॒नानि॑। राष्ट्री॑ दे॒वानान्निष॒साद॑ म॒न्द्रा। चत॑स्र॒ ऊर्ज॑न्दुदुहे॒ पयासि। क्व॑ स्विदस्याः पर॒मं ज॑गाम। गौ॒री मि॑माय सलि॒लानि॒ तक्ष॑ती। एक॑पदी द्वि॒पदी॒ सा चतु॑ष्पदी। अ॒ष्टाप॑दी॒ नव॑पदी बभू॒वुषी। स॒हस्राक्षरा पर॒मे व्यो॑मन्। तस्या समु॒द्रा अधि॒ विक्ष॑रन्ति। तेन॑ जीवन्ति प्र॒दिश॒श्चत॑स्रः॥६१॥

%2.4.6.12
तत॑ क्षरत्य॒क्षरम्। तद्विश्व॒मुप॑ जीवति। इन्द्रा॒सूरा॑ ज॒नय॑न्वि॒श्वक॑र्मा। म॒रुत्वा अस्तु ग॒णवान्त्सजा॒तवान्॑। अ॒स्य स्नु॒षा श्वशु॑रस्य॒ प्रशि॑ष्टिम्। स॒पत्ना॒ वाचं॒ मन॑सा॒ उपा॑सताम्। इन्द्र॒ सूरो॑ अतर॒द्रजासि। स्नु॒षा स॒पत्ना॒ श्वशु॑रो॒ऽयम॑स्तु। अ॒य शत्रूञ्जयतु॒ जर्‌हृ॑षाणः। अ॒यं वां॑ जयतु॒ वाज॑सातौ। अ॒ग्निः क्ष॑त्र॒भृदनि॑भृष्ट॒मोज॑। स॒ह॒स्रियो॑ दीप्यता॒मप्र॑युच्छन्। वि॒भ्राज॑मानः समिधा॒न उ॒ग्रः। आऽन्तरि॑क्षमरुह॒दग॒न्द्याम्॥६२॥\anuvakamend[धा॒रय॑न्पुरो॒डाशं॒ बृह॒स्पतिं॑ ज॒घन॑च्युतिमान॒न्दो भग॑स्य तृप्याण्य॒ग्नेः पृ॑थि॒वी यज्व॑न एतु प्र॒दिश॒श्चत॑स्रो॒ वाज॑सातौ च॒त्वारि॑ च]

%2.4.7.1
वृषाऽस्य॒शुर्वृ॑ष॒भाय॑ गृह्यसे। वृषा॒ऽयमु॒ग्रो नृ॒चक्ष॑से। दि॒व्यः क॑र्म॒ण्यो॑ हि॒तो बृ॒हन्नाम॑। वृ॒ष॒भस्य॒ या क॒कुत्। वि॒षू॒वान् वि॑ष्णो भवतु। अ॒यं यो मा॑म॒को वृषा। अथो॒ इन्द्र॑ इव दे॒वेभ्य॑। वि ब्र॑वीतु॒ जनेभ्यः। आयु॑ष्मन्तं॒ वर्च॑स्वन्तम्। अथो॒ अधि॑पतिं वि॒शाम्॥६३॥

%2.4.7.2
अ॒स्याः पृ॑थि॒व्या अध्य॑क्षम्। इ॒ममि॑न्द्र वृष॒भं कृ॑णु। यः सु॒शृङ्ग॑ सुवृष॒भः। क॒ल्याणो॒ द्रोण॒ आहि॑तः। कार्‌षी॑वल प्रगाणेन। वृ॒ष॒भेण॑ यजामहे। वृ॒ष॒भेण॒ यज॑मानाः। अक्रू॑रेणेव स॒र्पिषा। मृध॑श्च॒ सर्वा॒ इन्द्रे॑ण। पृत॑नाश्च जयामसि॥६४॥

%2.4.7.3
यस्या॒यमृ॑ष॒भो ह॒विः। इन्द्रा॑य परिणी॒यते। जया॑ति॒ शत्रु॑मा॒यन्तम्। अथो॑ हन्ति पृतन्य॒तः। नृ॒णामह॑ प्र॒णीरस॑त्। अग्र॑ उद्भिन्द॒ताम॑सत्। इन्द्र॒ शुष्म॑न्त॒नुवा॒ मेर॑यस्व। नी॒चा विश्वा॑ अ॒भिति॑ष्ठा॒भिमा॑तीः। नि शृ॑णीह्याबा॒धं यो॒ नो॒ अस्ति॑। उ॒रुन्नो॑ लो॒कं कृ॑णुहि जीरदानो॥६५॥

%2.4.7.4
प्रेह्य॒भि प्रेहि॒ प्र भ॑रा॒ सह॑स्व। मा विवे॑नो॒ वि शृ॑णुष्वा॒ जने॑षु। उदी॑डि॒तो वृ॑षभ॒ तिष्ठ॒ शुष्मै। इन्द्र॒ शत्रून्पु॒रो अ॒स्माक॑ युध्य। अग्ने॒ जेता॒ त्वं ज॑य। शत्रून्त्सहस॒ ओज॑सा। वि शत्रू॒न्॒ विमृधो॑ नुद। ए॒तन्ते॒ स्तोम॑न्तुविजात॒ विप्र॑। रथ॒न्न धीर॒ स्वपा॑ अतक्षम्। यदीद॑ग्ने॒ प्रति॒त्वन्दे॑व॒ हर्या॥६६॥

%2.4.7.5
सुव॑र्वतीर॒प ए॑ना जयेम। यो घृ॒तेना॒भिमा॑नितः। इन्द्र॒ जैत्रा॑य जज्ञिषे। स न॒ सङ्का॑सु पारय। पृ॒त॒ना॒साह्ये॑षु च। इन्द्रो॑ जिगाय पृथि॒वीम्। अ॒न्तरि॑क्ष॒ सुव॑र्म॒हत्। वृ॒त्र॒हा पु॑रु॒चेत॑नः। इन्द्रो॑ जिगाय॒ सह॑सा॒ सहासि। इन्द्रो॑ जिगाय॒ पृत॑नानि॒ विश्वा॥६७॥

%2.4.7.6
इन्द्रो॑ जा॒तो वि पुरो॑ रुरोज। स न॑ पर॒स्पा वरि॑वः कृणोतु। अ॒यं कृ॒त्नुरगृ॑भीतः। वि॒श्व॒जिदु॒द्भिदित्सोम॑। ऋषि॒र्विप्र॒ काव्ये॑न। वा॒युर॑ग्रे॒गा य॑ज्ञ॒प्रीः। सा॒कङ्ग॒न्मन॑सा य॒ज्ञम्। शि॒वो नि॒युद्भि॑ शि॒वाभि॑। वायो॑ शु॒क्रो अ॑यामि ते। मध्वो॒ अग्र॒न्दिवि॑ष्टिषु॥६८॥

%2.4.7.7
आ या॑हि॒ सोम॑ पीतये। स्वा॒रु॒हो दे॑व नि॒युत्व॑ता। इ॒ममि॑न्द्र वर्धय क्ष॒त्रिया॑णाम्। अ॒यं वि॒शां वि॒श्पति॑रस्तु॒ राजा। अ॒स्मा इ॑न्द्र॒ महि॒ वर्चासि धेहि। अ॒व॒र्चस॑ङ्कणुहि॒ शत्रु॑मस्य। इ॒ममा भ॑ज॒ ग्रामे॒ अश्वे॑षु॒ गोषु॑। निर॒मुं भ॑ज॒ यो॑ऽमित्रो॑ अस्य। वर्ष्म॑न् क्ष॒त्रस्य॑ क॒कुभि॑ श्रयस्व। ततो॑ न उ॒ग्रो वि भ॑जा॒ वसू॑नि॥६९॥

%2.4.7.8
अ॒स्मे द्या॑वापृथिवी॒ भूरि॑ वा॒मम्। सन्दु॑हाथाङ्घर्म॒दुघे॑व धे॒नुः। अ॒य राजा प्रि॒य इन्द्र॑स्य भूयात्। प्रि॒यो गवा॒मोष॑धीनामु॒तापाम्। यु॒नज्मि॑ त उत्त॒राव॑न्त॒मिन्द्रम्। येन॒ जया॑सि॒ न परा॒ जया॑सै। स त्वा॑ऽकरेकवृष॒भ स्वानाम्। अथो॑ राजन्नुत्त॒मं मा॑न॒वानाम्। उत्त॑र॒स्त्वमध॑रे ते स॒पत्ना। एक॑वृषा॒ इन्द्र॑सखा जिगी॒वान्॥७०॥

%2.4.7.9
विश्वा॒ आशा॒ पृत॑नाः स॒ञ्जयं॒ जय\sn{}। अ॒भि ति॑ष्ठ शत्रूय॒तः स॑हस्व। तुभ्यं॑ भरन्ति क्षि॒तयो॑ यविष्ठ। ब॒लिम॑ग्ने॒ अन्ति॑त॒ ओत दू॒रात्। आ भन्दि॑ष्ठस्य सुम॒तिञ्चि॑किद्धि। बृ॒हत्ते॑ अग्ने॒ महि॒ शर्म॑ भ॒द्रम्। यो दे॒ह्यो अन॑मयद्वध॒स्नैः। यो अर्य॑पत्नीरु॒षस॑श्च॒कार॑। स नि॒रुध्या॒ नहु॑षो य॒ह्वो अ॒ग्निः। विश॑श्चक्रे बलि॒हृत॒ सहो॑भिः॥७१॥

%2.4.7.10
प्र स॒द्यो अ॑ग्ने॒ अत्येष्य॒न्यान्। आ॒विर्यस्मै॒ चारु॑तरो ब॒भूथ॑। ई॒डेन्यो॑ वपु॒ष्यो॑ वि॒भावा। प्रि॒यो वि॒शामति॑थि॒र्मानु॑षीणाम्। ब्रह्म॑ज्येष्ठा वी॒र्या॑ सम्भृ॑तानि। ब्रह्माग्रे॒ ज्येष्ठ॒न्दिव॒मा त॑तान। ऋ॒तस्य॒ ब्रह्म॑ प्रथ॒मोत ज॑ज्ञे। तेना॑र्‌हति॒ ब्रह्म॑णा॒ स्पर्धि॑तु॒ङ्कः। ब्रह्म॒ स्रुचो॑ घृ॒तव॑तीः। ब्रह्म॑णा॒ स्वर॑वो मि॒ताः॥७२॥

%2.4.7.11
ब्रह्म॑ य॒ज्ञस्य॒ तन्त॑वः। ऋ॒त्विजो॒ ये ह॑वि॒ष्कृत॑। शृङ्गा॑णी॒वेच्छृ॒ङ्गिणा॒ सन्द॑दृश्रिरे। च॒षाल॑वन्त॒ स्वर॑वः पृथि॒व्याम्। ते दे॒वास॒ स्वर॑वस्तस्थि॒वास॑। नम॒ सखि॑भ्यः स॒न्नान्माऽव॑गात। अ॒भि॒भूर॒ग्निर॑तर॒द्रजासि। स्पृधो॑ वि॒हत्य॒ पृत॑ना अभि॒श्रीः। जु॒षा॒णो म॒ आहु॑तिं मामहिष्ट। ह॒त्वा स॒पत्ना॒न्॒ वरि॑वस्करन्नः। ईशा॑नन्त्वा॒ भुव॑नानामभि॒श्रियम्। स्तौम्य॑ग्न उरु॒कृत सु॒वीरम्। ह॒विर्जु॑षा॒णः स॒पत्ना अभि॒भूर॑सि। ज॒हि शत्रू॒ रप॒ मृधो॑ नुदस्व॥७३॥\anuvakamend[वि॒शां ज॑यामसि जीरदानो॒ हर्या॒ विश्वा॒ दिवि॑ष्टिषु॒ वसू॑नि जिगी॒वान्त्सहो॑भिर्मि॒ता न॑श्च॒त्वारि॑ च]

%2.4.8.1
स प्र॑त्न॒वन्नवी॑यसा। अग्ने द्यु॒म्नेन॑ सं॒यता। बृ॒हत्त॑तन्थ भा॒नुना। नव॒न्नु स्तोम॑म॒ग्नये। दि॒वः श्ये॒नाय॑ जीजनम्। वसो कु॒विद्व॒नाति॑ नः। स्वा॒रु॒हा यस्य॒ श्रियो॑ दृ॒शे। र॒यिर्वी॒रव॑तो यथा। अग्रे॑ य॒ज्ञस्य॒ चेत॑तः। अदाभ्यः पुरए॒ता॥७४॥

%2.4.8.2
अ॒ग्निर्वि॒शां मानु॑षीणाम्। तूर्णी॒ रथ॒ सदा॒ नव॑। नव॒ सोमा॑य वा॒जिने। आज्यं॒ पय॑सोऽजनि। जुष्ट॒ शुचि॑तमं॒ वसु॑। नव सोम जुषस्व नः। पी॒यूष॑स्ये॒ह तृ॑प्णुहि। यस्ते॑ भा॒ग ऋ॒ता व॒यम्। नव॑स्य सोम ते व॒यम्। आ सु॑म॒तिं वृ॑णीमहे॥७५॥

%2.4.8.3
स नो॑ रास्व सह॒स्रिण॑। नव ह॒विर्जु॑षस्व नः। ऋ॒तुभि॑ सोम॒ भूत॑मम्। तद॒ङ्ग प्रति॑हर्य नः। राजन्त्सोम स्व॒स्तये। नव॒स्तोम॒न्नव ह॒विः। इ॒न्द्रा॒ग्निभ्यां॒ नि वे॑दय। तज्जु॑षेता॒ सचे॑तसा। शुचि॒न्नु स्तोम॒न्नव॑जातम॒द्य। इन्द्राग्नी वृत्रहणा जु॒षेथाम्॥७६॥

%2.4.8.4
उ॒भा हि वा सु॒हवा॒ जोह॑वीमि। ता वाज स॒द्य उ॑श॒ते धेष्ठा। अ॒ग्निरिन्द्रो॒ नव॑स्य नः। अ॒स्य ह॒व्यस्य॑ तृप्यताम्। इ॒ह दे॒वौ स॑ह॒स्रिणौ। य॒ज्ञन्न॒ आ हि गच्छ॑ताम्। वसु॑मन्त सुव॒र्विदम्। अ॒स्य ह॒व्यस्य॑ तृप्यताम्। अ॒ग्निरिन्द्रो॒ नव॑स्य नः। विश्वान्दे॒वास्त॑र्पयत॥७७॥

%2.4.8.5
ह॒विषो॒ऽस्य नव॑स्य नः। सु॒व॒र्विदो॒ हि ज॑ज्ञि॒रे। एदं ब॒र्॒हिः सु॒ष्टरी॑मा॒ नवे॑न। अ॒यं य॒ज्ञो यज॑मानस्य भा॒गः। अ॒यं ब॑भूव॒ भुव॑नस्य॒ गर्भ॑। विश्वे॑ दे॒वा इ॒दम॒द्याग॑मिष्ठाः। इ॒मे नु द्यावा॑पृथि॒वी स॒मीची। त॒न्वा॒ने य॒ज्ञं पु॑रु॒पेश॑सन्धि॒या। आऽस्मै॑ पृणीतां॒ भुव॑नानि॒ विश्वा। प्र॒जां पुष्टि॑म॒मृत॒न्नवे॑न॥७८॥

%2.4.8.6
इ॒मे धे॒नू अ॒मृतं॒ ये दु॒हाते। पय॑स्वत्युत्त॒रामे॑तु॒ पुष्टि॑। इ॒मं य॒ज्ञं जु॒षमा॑णे॒ नवे॑न। स॒मीची॒ द्यावा॑पृथि॒वी घृ॒ताची। यवि॑ष्ठो हव्य॒वाह॑नः। चि॒त्रभा॑नुर्घु॒तासु॑तिः। नव॑जातो॒ वि रो॑चसे। अग्ने॒ तत्ते॑ महित्व॒नम्। त्वम॑ग्ने दे॒वताभ्यः। भा॒गे दे॑व॒ न मी॑यसे॥७९॥

%2.4.8.7
स ए॑ना वि॒द्वान् य॑क्ष्यसि। नव॒ स्तोमं॑ जुषस्व नः। अ॒ग्निः प्र॑थ॒मः प्राश्ञा॑तु। स हि वेद॒ यथा॑ ह॒विः। शि॒वा अ॒स्मभ्य॒मोष॑धीः। कृ॒णोतु॑ वि॒श्वच॑र्‌षणिः। भ॒द्रान्न॒ श्रेय॒ सम॑नैष्ट देवाः। त्वया॑ऽव॒सेन॒ सम॑शीमहि त्वा। स नो॑ मयो॒भूः पि॑तो॒ आ वि॑शस्व। शन्तो॒काय॑ त॒नुवे स्यो॒नः। ए॒तमु॒ त्यं मधु॑ना॒ संयु॑तं॒ यवम्। सर॑स्वत्या॒ अधि॑म॒नाव॑चर्कृषुः। इन्द्र॑ आसी॒त्सीर॑पतिः श॒तक्र॑तुः। की॒नाशा॑ आसन्म॒रुत॑ सु॒दान॑वः॥८०॥\anuvakamend[पु॒र॒ए॒ता वृ॑णीमहे जु॒षेथान्तर्पयता॒मृत॒न्नवे॑न मीयसे स्यो॒नश्च॒त्वारि॑ च]




\prashnaend{जुष्ट॒श्चक्षु॑षो॒ जुष्टी॑नरो नक्तञ्जा॒ता वृषा॒स उ॒त नो॒ वृषाऽस्य॒शुः सप्र॑त्न॒वद॒ष्टौ॥८॥}{जुष्टो॑ म॒न्युर्भगो॒ जुष्टी॑ नरो॒ हरि॑वर्पस॒ङ्गिर॒ शिप्रि॑न्वाजानामु॒त न॑ प्रि॒या यद्वाग्वद॑न्ती॒ विश्वा॒ आशा॒ अशी॑तिः॥८०॥}{जुष्ट॑ सु॒दान॑वः॥}{हरि॑ ओम्॥}{इति श्रीकृष्णयजुर्वेदीयतैत्तिरीयब्राह्मणे द्वितीयाष्टके चतुर्थः प्रपाठकः समाप्तः॥}
\clearpage
\sect{पञ्चमः प्रश्नः}
\setcounter{anuvakam}{0}
\dnsub{तैत्तिरीयब्राह्मणे द्वितीयाष्टके पञ्चमः प्रपाठकः}

%2.5.1.1
प्रा॒णो र॑क्षति॒ विश्व॒मेज॑त्। इर्यो॑ भू॒त्वा ब॑हु॒धा ब॒हूनि॑। स इत्सर्वं॒ व्या॑नशे। यो दे॒वो दे॒वेषु॑ वि॒भूर॒न्तः। आवृ॑दू॒दात् क्षेत्रिय॑ध्व॒गद्वृषा। तमित्प्रा॒णं मन॒सोप॑ शिक्षत। अग्रं॑ दे॒वाना॑मि॒दम॑त्तु नो ह॒विः। मन॑स॒श्चित्ते॒दम्। भू॒तं भव्यं॑ च गुप्यते। तद्धि दे॒वेष्व॑ग्रि॒यम्॥१॥

%2.5.1.2
आ न॑ एतु पुरश्च॒रम्। स॒ह दे॒वैरि॒म हवम्। मन॒ श्रेय॑सिश्रेयसि। कर्म॑न् य॒ज्ञप॑ति॒न्दध॑त्। जु॒षतां मे॒ वागि॒द ह॒विः। वि॒राड्दे॒वी पु॒रोहि॑ता। ह॒व्य॒वा़डन॑पायिनी। यया॑ रू॒पाणि॑ बहु॒धा वद॑न्ति। पेशासि दे॒वाः प॑र॒मे ज॒नित्रे। सा नो॑ वि॒राडन॑पस्फुरन्ती॥२॥

%2.5.1.3
वाग्दे॒वी जु॑षतामि॒द ह॒विः। चक्षु॑र्दे॒वानां॒ ज्योति॑र॒मृते॒ न्य॑क्तम्। अ॒स्य वि॒ज्ञाना॑य बहु॒धा निधी॑यते। तस्य॑ सु॒म्नम॑शीमहि। मा नो॑ हासीद्विचक्ष॒णम्। आयु॒रिन्न॒ प्रतीर्यताम्। अन॑न्धा॒श्चक्षु॑षा व॒यम्। जी॒वा ज्योति॑रशीमहि। सुव॒र्ज्योति॑रु॒तामृतम्। श्रोत्रे॑ण भ॒द्रमु॒त शृ॑ण्वन्ति स॒त्यम्। श्रोत्रे॑ण॒ वाचं॑ बहु॒धोद्यमा॑नाम्। श्रोत्रे॑ण॒ मोद॑श्च॒ मह॑श्च श्रूयते। श्रोत्रे॑ण॒ सर्वा॒ दिश॒ आ शृ॑णोमि। येन॒ प्राच्या॑ उ॒त द॑क्षि॒णा। प्र॒तीच्यै॑ दि॒शः शृ॒ण्वन्त्यु॑त्त॒रात्। तदिच्छ्रोत्रं॑ बहु॒धोद्यमा॑नम्। अ॒रान्न ने॒मिः परि॒ सर्वं॑ बभूव॥३॥\anuvakamend[अ॒ग्रि॒यमन॑पस्फुरन्ती स॒त्य स॒प्त च॑]

%2.5.2.1
उ॒देहि॑ वाजि॒न्यो अ॑स्य॒प्स्व॑न्तः। इ॒द रा॒ष्ट्रमा वि॑श सू॒नृता॑वत्। यो रोहि॑तो॒ विश्व॑मि॒दञ्ज॒जान॑। स नो॑ रा॒ष्ट्रेषु॒ सुधि॑तान्दधातु। रोहरोह॒ रोहि॑त॒ आरु॑रोह। प्र॒जाभि॒र्वृद्धिं॑ ज॒नुषा॑मु॒पस्थम्। ताभि॒ सर॑ब्धो अविद॒थ्षडु॒र्वीः। गा॒तुं प्र॒पश्य॑न्नि॒ह रा॒ष्ट्रमाऽहा। आऽहा॑र्\mbox{}षीद्रा॒ष्ट्रमि॒ह रोहि॑तः। मृधो॒ व्यास्थ॒दभ॑यन्नो अस्तु ॥४॥

%2.5.2.2
अ॒स्मभ्य॑न्द्यावापृथिवी॒ शक्व॑रीभिः। रा॒ष्ट्रन्दु॑हाथामि॒ह रे॒वती॑भिः। विम॑मर्\mbox{}श॒ रोहि॑तो वि॒श्वरू॑पः। स॒मा॒च॒क्रा॒णः प्र॒रुहो॒ रुह॑श्च। दिव॑ङ्ग॒त्वाय॑ मह॒ता म॑हि॒म्ना। वि नो॑ रा॒ष्ट्रमु॑नत्तु॒ पय॑सा॒ स्वेन॑। यास्ते॒ विश॒स्तप॑सा सं बभू॒वुः। गा॒य॒त्रं व॒त्समनु॒ तास्त॒ आऽगु॑। तास्त्वा वि॑शन्तु॒ मह॑सा॒ स्वेन॑। सं मा॑ता पु॒त्रो अ॒भ्ये॑तु॒ रोहि॑तः॥५॥

%2.5.2.3
यू॒यमु॑ग्रा मरुतः पृश्ञिमातरः। इन्द्रे॑ण स॒युजा॒ प्रमृ॑णीथ॒ शत्रून्॑। आ वो॒ रोहि॑तो अशृणोदभिद्यवः। त्रिस॑प्तासो मरुतः स्वादुसम्मुदः। रोहि॑तो॒ द्यावा॑पृथि॒वी ज॑जान। तस्मि॒स्तन्तुं॑ परमे॒ष्ठी त॑तान। तस्मि॑ञ्छिश्रिये अ॒ज एक॑पात्। अदृह॒द्द्यावा॑पृथि॒वी बले॑न। रोहि॑तो॒ द्यावा॑पृथि॒वी अ॑दृहत्। तेन॒ सुव॑ स्तभि॒तन्तेन॒ नाक॑॥६॥

%2.5.2.4
सो अ॒न्तरि॑क्षे॒ रज॑सो वि॒मान॑। तेन॑ दे॒वाः सुव॒रन्व॑विन्दन्। सु॒शेव॑न्त्वा भा॒नवो॑ दीदि॒वासम्। सम॑ग्रासो जु॒ह्वो॑ जातवेदः। उ॒क्षन्ति॑ त्वा वा॒जिन॒मा घृ॒तेन॑। सस॑मग्ने युवसे॒ भोज॑नानि। अग्ने॒ शर्ध॑ मह॒ते सौभ॑गाय। तव॑ द्यु॒म्नान्यु॑त्त॒मानि॑ सन्तु। सञ्जास्प॒त्य सु॒यम॒मा कृ॑णुष्व। श॒त्रू॒य॒ताम॒भि ति॑ष्ठा॒ महासि॥७॥\anuvakamend[अ॒स्त्वे॒तु॒ रोहि॑तो॒ नाको॒ महासि]

%2.5.3.1
पुन॑र्न॒ इन्द्रो॑ म॒घवा॑ ददातु। धना॑नि श॒क्रो धन्य॑ सु॒राधा। अ॒र्वा॒चीन॑ङ्कृणुतां याचि॒तो मन॑। श्रु॒ष्टी नो॑ अ॒स्य ह॒विषो॑ जुषा॒णः। यानि॑ नोऽजि॒नन्धना॑नि। ज॒हर्थ॑ शूर म॒न्युना। इन्द्रानु॑विन्द न॒स्तानि॑। अ॒नेन॑ ह॒विषा॒ पुन॑। इन्द्र॒ आशाभ्य॒ परि॑। सर्वा॒भ्योऽभ॑यङ्करत्॥८॥

%2.5.3.2
जेता॒ शत्रू॒न्॒ विच॑र्\mbox{}षणिः। आकूत्यै त्वा॒ कामा॑य त्वा स॒मृधे त्वा। पु॒रो द॑धे अमृत॒त्वाय॑ जी॒वसे। आकू॑तिम॒स्याव॑से। काम॑मस्य॒ समृ॑द्ध्यै। इन्द्र॑स्य युञ्जते॒ धिय॑। आकू॑तिन्दे॒वीं मन॑सः पु॒रो द॑धे। य॒ज्ञस्य॑ मा॒ता सु॒हवा॑ मे अस्तु। यदि॒च्छामि॒ मन॑सा॒ सका॑मः। वि॒देय॑मेन॒द्धृद॑ये॒ निवि॑ष्टम्॥९॥

%2.5.3.3
सेद॒ग्निर॒ग्नीरत्येत्य॒न्यान्। यत्र॑ वा॒जी तन॑यो वी॒डुपा॑णिः। स॒हस्र॑पाथा अ॒क्षरा॑ स॒मेति॑। आशा॑नान्त्वाऽऽशापा॒लेभ्य॑। च॒तुर्भ्यो॑ अ॒मृतेभ्यः। इ॒दं भू॒तस्याध्य॑क्षेभ्यः। वि॒धेम॑ ह॒विषा॑ व॒यम्। विश्वा॒ आशा॒ मधु॑ना॒ स सृ॑जामि। अ॒न॒मी॒वा आप॒ ओष॑धयो भवन्तु। अ॒यं यज॑मानो॒ मृधो॒ व्य॑स्यताम्॥१०॥

%2.5.3.4
अगृ॑भीताः प॒शव॑ सन्तु॒ सर्वे। अ॒ग्निः सोमो॒ वरु॑णो मि॒त्र इन्द्र॑। बृह॒स्पति॑ सवि॒ता यः स॑ह॒स्री। पू॒षा नो॒ गोभि॒रव॑सा॒ सर॑स्वती। त्वष्टा॑ रू॒पाणि॒ सम॑नक्तु य॒ज्ञैः। त्वष्टा॑ रू॒पाणि॒ दध॑ती॒ सर॑स्वती। पू॒षा भग सवि॒ता नो॑ ददातु। बृह॒स्पति॒र्दद॒दिन्द्र॑ स॒हस्रम्। मि॒त्रो दा॒ता वरु॑ण॒ सोमो॑ अ॒ग्निः॥११॥\anuvakamend[क॒र॒न्निवि॑ष्टमस्यता॒न्नव॑ च]

%2.5.4.1
आ नो॑ भर॒ भग॑मिन्द्र द्यु॒मन्तम्। नि ते॑ दे॒ष्णस्य॑ धीमहि प्ररे॒के। उ॒र्व इ॑व पप्रथे॒ कामो॑ अ॒स्मे। तमापृ॑णा वसुपते॒ वसू॑नाम्। इ॒मङ्कामं॑ मन्दया॒ गोभि॒रश्वै। च॒न्द्रव॑ता॒ राध॑सा प॒प्रथ॑श्च। सु॒व॒र्यवो॑ म॒तिभि॒स्तुभ्यं॒ विप्रा। इन्द्रा॑य॒ वाह॑ कुशि॒कासो॑ अक्रन्। इन्द्र॑स्य॒ नु वी॒र्या॑णि॒ प्रवो॑चम्। यानि॑ च॒कार॑ प्रथ॒मानि॑ व॒ज्री॥१२॥

%2.5.4.2
अह॒न्नहि॒मन्व॒पस्त॑तर्द। प्रव॒क्षणा॑ अभिन॒त्पर्व॑तानाम्। अह॒न्नहिं॒ पर्व॑ते शिश्रिया॒णम्। त्वष्टाऽस्मै॒ वज्र स्व॒र्य॑न्ततक्ष। वा॒श्रा इ॑व धे॒नव॒ स्यन्द॑मानाः। अञ्ज॑ समु॒द्रमव॑ जग्मु॒राप॑। वृ॒षा॒यमा॑णोऽवृणीत॒ सोमम्। त्रिक॑द्रुकेष्वपिबत्सु॒तस्य॑। आ साय॑कं म॒घवा॑ दत्त॒ वज्रम्। अह॑न्नेनं प्रथम॒जा मही॑नाम्॥१३॥

%2.5.4.3
यदिन्द्राह॑न्प्रथम॒जा मही॑नाम्। आन्मा॒यिना॒ममि॑ना॒ प्रोत मा॒याः। आत्सूर्यं॑ ज॒नय॒न्द्यामु॒षासम्। ता॒दीक्ना॒ शत्रू॒न्न किला॑विवित्से। अह॑न्वृ॒त्रं वृ॑त्र॒तरं॒ व्यसम्। इन्द्रो॒ वज्रे॑ण मह॒ता व॒धेन॑। स्कन्धासीव॒ कुलि॑शेना॒विवृ॑क्णा। अहि॑ शयत उप॒पृक्पृ॑थि॒व्याम्। अ॒यो॒ध्येव दु॒र्मद॒ आ हि जु॒ह्वे। म॒हा॒वी॒रन्तु॑विबा॒धमृ॑जी॒षम्॥१४॥

%2.5.4.4
नाता॑रीरस्य॒ समृ॑तिं व॒धानाम्। स रु॒जाना पिपिष॒ इन्द्र॑शत्रुः। विश्वो॒ विहा॑या अर॒तिः। वसु॑र्दधे॒ हस्ते॒ दक्षि॑णे। त॒रणि॒र्न शि॑श्रथत्। श्र॒व॒स्य॑या॒ न शि॑श्रथत्। विश्व॑स्मा॒ इदि॑षुध्य॒से। दे॒व॒त्रा ह॒व्यमूहि॑षे। विश्व॑स्मा॒ इत्सु॒कृते॒ वार॑मृण्वति। अ॒ग्निर्द्वारा॒ व्यृ॑ण्वति॥१५॥

%2.5.4.5
उदु॒ज्जिहा॑नो अ॒भि काम॑मी॒रय\sn{}। प्र॒पृ॒ञ्चन्विश्वा॒ भुव॑नानि पू॒र्वथा। आ के॒तुना॒ सुष॑मिद्धो॒ यजि॑ष्ठः। काम॑न्नो अग्ने अ॒भिह॑र्य दि॒ग्भ्यः। जु॒षा॒णो ह॒व्यम॒मृते॑षु दू॒ढ्य॑। आ नो॑ र॒यिं ब॑हु॒लाङ्गोम॑ती॒मिषम्। नि धे॑हि॒ यक्ष॑द॒मृते॑षु॒ भूष\sn{}। अश्वि॑ना य॒ज्ञमाग॑तम्। दा॒शुष॒ पुरु॑दससा। पू॒षा र॑क्षतु नो र॒यिम्॥१६॥

%2.5.4.6
इ॒मं य॒ज्ञम॒श्विना॑ व॒र्धय॑न्ता। इ॒मौ र॒यिं यज॑मानाय धत्तम्। इ॒मौ प॒शून्र॑क्षतां वि॒श्वतो॑ नः। पू॒षा न॑ पातु॒ सद॒मप्र॑यच्छन्। प्रते॑ म॒हे स॑रस्वति। सुभ॑गे॒ वाजि॑नीवति। स॒त्य॒वाचे॑ भरे म॒तिम्। इ॒दन्नो॑ ह॒व्यङ्घृ॒तव॑त्सरस्वति। स॒त्य॒वाचे॒ प्रभ॑रेमा ह॒वीषि॑। इ॒मानि॑ ते दुरि॒ता सौभ॑गानि। तेभि॑र्व॒य सु॒भगा॑सः स्याम॥१७॥\anuvakamend[व॒ज्र्यही॑नामृजी॒षं व्यृ॑ण्वति रक्षतु नो र॒यि सौभ॑गा॒न्येकं च]

%2.5.5.1
य॒ज्ञो रा॒यो य॒ज्ञ ई॑शे॒ वसू॑नाम्। य॒ज्ञः स॒स्याना॑मु॒त सु॑क्षिती॒नाम्। य॒ज्ञ इ॒ष्टः पू॒र्वचि॑त्तिन्दधातु। य॒ज्ञो ब्र॑ह्म॒ण्वा अप्ये॑तु दे॒वान्। अ॒यं य॒ज्ञो व॑र्धता॒ङ्गोभि॒रश्वै। इ॒यं वेदि॑ स्वप॒त्या सु॒वीरा। इ॒दं ब॒र्॒हिरति॑ ब॒र्॒हीष्य॒न्या। इ॒मं य॒ज्ञं विश्वे॑ अवन्तु दे॒वाः। भग॑ ए॒व भग॑वा अस्तु देवाः। तेन॑ व॒यं भग॑वन्तः स्याम॥१८॥

%2.5.5.2
तन्त्वा॑ भग॒ सर्व॒ इज्जो॑हवीमि। स नो॑ भग पुरए॒ता भ॑वे॒ह। भग॒ प्रणे॑त॒र्भग॒ सत्य॑राधः। भगे॒मान्धिय॒मुद॑व॒ दद॑न्नः। भग॒ प्र णो॑ जनय॒ गोभि॒रश्वै। भग॒ प्र नृभि॑र्नृ॒वन्त॑ स्याम। शश्व॑ती॒ समा॒ उप॑यन्ति लो॒काः। शश्व॑ती॒ समा॒ उप॑य॒न्त्याप॑। इ॒ष्टं पू॒र्त शश्व॑तीना॒ समा॑ना शाश्व॒तेन॑। ह॒विषे॒ष्ट्वाऽन॒न्तं लो॒कं पर॒मा रु॑रोह ॥१९॥

%2.5.5.3
इ॒यमे॒व सा या प्र॑थ॒मा व्यौच्छ॑त्। सा रू॒पाणि॑ कुरुते॒ पञ्च॑ दे॒वी। द्वे स्वसा॑रौ वयत॒स्तन्त्र॑मे॒तत्। स॒ना॒तनं॒ वित॑त॒ षण्म॑यूखम्। अवा॒न्यास्तन्तून्कि॒रतो॑ ध॒त्तो अ॒न्यान्। नाव॑पृ॒ज्याते॒ न ग॑माते॒ अन्तम्। आ वो॑ यन्तूदवा॒हासो॑ अ॒द्य। वृष्टिं॒ ये विश्वे॑ म॒रुतो॑ जु॒नन्ति॑। अ॒यय्योँ अ॒ग्निर्म॑रुत॒ समि॑द्धः। ए॒तं जु॑षध्वङ्कवयो युवानः॥२०॥

%2.5.5.4
धा॒रा॒व॒रा म॒रुतो॑ धृ॒ष्णुवो॑जसः। मृ॒गा न भी॒मास्त॑वि॒षेभि॑रू॒र्मिभि॑। अ॒ग्नयो॒ न शु॑शुचा॒ना ऋ॑जी॒षिण॑। भ्रुमि॒न्धम॑न्त॒ उप॒ गा अ॑वृण्वत। वि च॑क्रमे॒ त्रिर्दे॒वः। आ वे॒धस॒न्नील॑पृष्ठं बृ॒हन्तम्। बृह॒स्पति॒ सद॑ने सादयध्वम्। सा॒दद्यो॑नि॒न्दम॒ आ दी॑दि॒वासम्। हिर॑ण्यवर्णमरु॒ष स॑पेम। स हि शुचि॑ श॒तप॑त्र॒ स शु॒न्ध्यूः ॥२१॥

%2.5.5.5
हिर॑ण्यवाशीरिषि॒रः सु॑व॒र्॒षाः। बृह॒स्पति॒ स स्वा॑वे॒श ऋ॒ष्वाः। पू॒रू सखि॑भ्य आसु॒तिङ्क॑रिष्ठः। पूष॒ स्तव॑ व्र॒ते व॒यम्। नरि॑ष्येम क॒दाच॒न। स्तो॒तार॑स्त इ॒ह स्म॑सि। यास्ते॑ पूष॒न्ना वो॑ अ॒न्तः स॑मु॒द्रे। हि॒र॒ण्ययी॑र॒न्तरि॑क्षे॒ चर॑न्ति। याभि॑र्यासि दू॒त्या सूर्य॑स्य। कामे॑न कृ॒तश्रव॑ इ॒च्छमा॑नः॥२२॥

%2.5.5.6
अर॑ण्या॒न्यर॑ण्यान्य॒सौ। या प्रेव॒ नश्य॑सि। क॒था ग्राम॒न्न पृ॑च्छसि। न त्वा॒भीरि॑व विन्दती ३। वृ॒षा॒र॒वाय॒ वद॑ते। यदु॒पाव॑ति चिच्चि॒कः। आ॒घा॒टीभि॑रिव धा॒वय\sn{}। अ॒र॒ण्या॒निर्म॑हीयते। उ॒त गाव॑ इवादन्। उ॒तो वेश्मे॑व दृश्यते॥२३॥

%2.5.5.7
उ॒तो अ॑रण्या॒निः सा॒यम्। श॒क॒टीरि॑व सर्जति। गाम॒ङ्गैष॒ आ ह्व॑यति। दार्व॒ङ्गैष॒ उपा॑वधीत्। वस॑न्नरण्या॒न्या सा॒यम्। अक्रु॑क्ष॒दिति॑ मन्यते। न वा अ॑रण्या॒निर्\mbox{}ह॑न्ति। अ॒न्यश्चेन्नाभि॒गच्छ॑ति। स्वा॒दोः फल॑स्य ज॒ग्ध्वा। यत्र॒ कामं॒ नि प॑द्यते। आञ्ज॑नगन्धी सुर॒भीम्। ब॒ह्व॒न्नामकृ॑षीवलाम्। प्राहं मृ॒गाणां मा॒तरम्। अ॒र॒ण्या॒नीम॑शसिषम्॥२४॥\anuvakamend[स्या॒म॒ रु॒रो॒ह॒ यु॒वा॒न॒ शु॒न्ध्यूरि॒च्छमा॑नो दृश्यते॒ निप॑द्यते च॒त्वारि॑ च]

%2.5.6.1
वार्त्र॑हत्याय॒ शव॑से। पृ॒त॒ना॒साह्या॑य च। इन्द्र॒ त्वा व॑र्तयामसि। सु॒ब्रह्मा॑णं वी॒रव॑न्तं बृ॒हन्तम्। उ॒रुं ग॑भी॒रं पृ॒थुबु॑ध्नमिन्द्र। श्रु॒तर्\mbox{}षि॑मु॒ग्रम॑भिमाति॒षाहम्। अ॒स्मभ्यं॑ चि॒त्रं वृष॑ण र॒यिं दा। क्षे॒त्रि॒यै त्वा॒ निर्\mbox{}ऋ॑त्यै त्वा। द्रु॒हो मु॑ञ्चामि॒ वरु॑णस्य॒ पाशात्। अ॒ना॒गसं॒ ब्रह्म॑णे त्वा करोमि॥२५॥

%2.5.6.2
शि॒वे ते॒ द्यावा॑पृथि॒वी उ॒भे इ॒मे। शं ते॑ अ॒ग्निः स॒हाद्भिर॑स्तु। शं द्यावा॑पृथि॒वी स॒हौष॑धीभिः। शम॒न्तरि॑क्ष स॒ह वाते॑न ते। शं ते॒ चत॑स्रः प्र॒दिशो॑ भवन्तु। या दैवी॒श्चत॑स्रः प्र॒दिश॑। वात॑पत्नीर॒भि सूर्यो॑ विच॒ष्टे। तासान्त्वा ज॒रस॒ आ द॑धामि। प्र यक्ष्म॑ एतु॒ निर्\mbox{}ऋ॑तिं परा॒चैः। अमो॑चि॒ यक्ष्माद्दुरि॒तादव॑र्त्यै॥२६॥

%2.5.6.3
द्रु॒हः पाशा॒न्निर्\mbox{}ऋ॑त्यै॒ चोद॑मोचि। अहा॒ अव॑र्ति॒मवि॑दत्स्यो॒नम्। अप्य॑भूद्भ॒द्रे सु॑कृ॒तस्य॑ लो॒के। सूर्य॑मृ॒तं तम॑सो॒ ग्राह्या॒ यत्। दे॒वा अमु॑ञ्च॒न्नसृ॑ज॒न्व्ये॑नसः। ए॒वम॒हमि॒मं क्षेत्रि॒याज्जा॑मिश॒सात्। द्रु॒हो मु॑ञ्चामि॒ वरु॑णस्य॒ पाशात्। बृह॑स्पते यु॒वमिन्द्र॑श्च॒ वस्व॑। दि॒व्यस्ये॑शाथे उ॒त पार्थि॑वस्य। ध॒त्त र॒यि स्तु॑व॒ते की॒रये॑चित्॥२७॥

%2.5.6.4
यू॒यं पा॑त स्व॒स्तिभि॒ सदा॑ नः। दे॒वा॒युध॒मिन्द्र॒मा जोहु॑वानाः। वि॒श्वा॒वृध॑म॒भि ये रक्ष॑माणाः। येन॑ ह॒ता दी॒र्घमध्वा॑न॒माय\sn{}। अ॒न॒न्तमर्थ॒मनि॑वर्त्स्यमानाः। यत्ते॑ सुजाते हि॒मव॑त्सु भेष॒जम्। म॒यो॒भूः शन्त॑मा॒ यद्धृ॒दोसि॑। ततो॑ नो देहि सीबले। अ॒दो गि॒रिभ्यो॒ अधि॒ यत्प्र॒धाव॑सि। स॒शोभ॑माना क॒न्ये॑व शुभ्रे॥२८॥

%2.5.6.5
तां त्वा॒ मुद्ग॑ला ह॒विषा॑ वर्धयन्ति। सा न॑ सीबले र॒यिमा भा॑जये॒ह। पूर्वं॑ देवा॒ अप॑रेणानु॒पश्यं॒ जन्म॑भिः। जन्मा॒न्यव॑रै॒ परा॑णि। वेदा॑नि देवा अ॒यम॒स्मीति॒ माम्। अ॒ह हि॒त्वा शरी॑रं ज॒रस॑ प॒रस्तात्। प्रा॒णा॒पा॒नौ चक्षु॒ श्रोत्रम्। वाचं॒ मन॑सि॒ सम्भृ॑ताम्। हि॒त्वा शरी॑रं ज॒रस॑ प॒रस्तात्। आ भूति॒म्भूतिं॑ व॒यम॑श्ञवामहै। इ॒मा ए॒व ता उ॒षसो॒ याः प्र॑थ॒मा व्यौच्छ\sn{}। ता दे॒व्य॑ कुर्वते॒ पञ्च॑रू॒पा। शश्व॑ती॒र्नाव॑पृज्यन्ति। न ग॑म॒न्त्यन्तम्॥२९॥\anuvakamend[क॒रो॒म्यव॑र्त्यै चिच्छुभ्रेऽश्ञवामहै च॒त्वारि॑ च]

%2.5.7.1
वसू॑नां॒ त्वाऽधी॑तेन। रु॒द्राणा॑मू॒र्म्या। आ॒दि॒त्यानां॒ तेज॑सा। विश्वे॑षां दे॒वानां॒ क्रतु॑ना। म॒रुता॒मेम्ना॑ जुहोमि॒ स्वाहा। अ॒भिभू॑तिर॒हमाग॑मम्। इन्द्र॑सखा स्वा॒युध॑। आस्वाशा॑सु दु॒ष्षह॑। इ॒दं वर्चो॑ अ॒ग्निना॑ द॒त्तमागात्। यशो॒ भर्ग॒ सह॒ ओजो॒ बलं॑ च॥३०॥

%2.5.7.2
दी॒र्घा॒यु॒त्वाय॑ श॒तशा॑रदाय। प्रति॑गृभ्णामि मह॒ते वी॒र्या॑य। आयु॑रसि वि॒श्वायु॑रसि। स॒र्वायु॑रसि॒ सर्व॒मायु॑रसि। सर्व॑म्म॒ आयु॑र्भूयात्। सर्व॒मायु॑र्गेषम्। भूर्भुव॒ सुव॑। अ॒ग्निर्धर्मे॑णान्ना॒दः। मृ॒त्युर्धर्मे॒णान्न॑पतिः। ब्रह्म॑ क्ष॒त्र स्वाहा॥३१॥

%2.5.7.3
प्र॒जाप॑तिः प्रणे॒ता। बृह॒स्पति॑ पुरए॒ता। य॒मः पन्था। च॒न्द्रमा पुनर॒सुः स्वाहा। अ॒ग्निर॑न्ना॒दोऽन्न॑पतिः। अ॒न्नाद्य॑म॒स्मिन् य॒ज्ञे यज॑मानाय ददातु॒ स्वाहा। सोमो॒ राजा॒ राज॑पतिः। रा॒ज्यम॒स्मिन् य॒ज्ञे यज॑मानाय ददातु॒ स्वाहा। वरु॑णः स॒म्राट्त्स॒म्राट्प॑तिः। साम्राज्यम॒स्मिन् य॒ज्ञे यज॑मानाय ददातु॒ स्वाहा॥३२॥

%2.5.7.4
मि॒त्रः क्ष॒त्रं क्ष॒त्रप॑तिः। क्ष॒त्रम॒स्मिन् य॒ज्ञे यज॑मानाय ददातु॒ स्वाहा। इन्द्रो॒ बलं॒ बल॑पतिः। बल॑म॒स्मिन् य॒ज्ञे यज॑मानाय ददातु॒ स्वाहा। बृह॒स्पति॒र्ब्रह्म॒ ब्रह्म॑पतिः। ब्रह्मा॒स्मिन् य॒ज्ञे यज॑मानाय ददातु॒ स्वाहा। स॒वि॒ता रा॒ष्ट्र रा॒ष्ट्रप॑तिः। रा॒ष्ट्रम॒स्मिन् य॒ज्ञे यज॑मानाय ददातु॒ स्वाहा। पू॒षा वि॒शां विट्प॑तिः। विश॑म॒स्मिन् य॒ज्ञे यज॑मानाय ददातु॒ स्वाहा। सर॑स्वती॒ पुष्टि॒ पुष्टि॑पत्नी। पुष्टि॑म॒स्मिन् य॒ज्ञे यज॑मानाय ददातु॒ स्वाहा। त्वष्टा॑ पशू॒नां मि॑थु॒नाना रूप॒कृद्रू॒पप॑तिः। रु॒पेणा॒स्मिन् य॒ज्ञे यज॑मानाय प॒शून्द॑दातु॒ स्वाहा॥३३॥\anuvakamend[च॒ स्वाहा॒ साम्राज्यम॒स्मिन् य॒ज्ञे यज॑मानाय ददातु॒ स्वाहा॒ विश॑म॒स्मिन् य॒ज्ञे यज॑मानाय ददातु॒ स्वाहा॑ च॒त्वारि॑ च (अ॒ग्निः सोमो॒ वरु॑णो मि॒त्र इन्द्रो॒ बृह॒स्पति॑ सवि॒ता पू॒षा सर॑स्वती॒ त्वष्टा॒ दश॑ ॥ )]

%2.5.8.1
स ईं पाहि॒ य ऋ॑जी॒षी तरु॑त्रः। यः शिप्र॑वान्वृष॒भो यो म॑ती॒नाम्। यो गोत्र॒भिद्व॑ज्र॒भृद्यो ह॑रि॒ष्ठाः। स इ॑न्द्र चि॒त्रा अ॒भि तृ॑न्धि॒ वाजान्॑। आ ते॒ शुष्मो॑ वृष॒भ ए॑तु प॒श्चात्। ओत्त॒राद॑ध॒रागा पु॒रस्तात्। आ वि॒श्वतो॑ अ॒भिसमेत्व॒र्वाङ्। इन्द्र॑ द्यु॒म्न सुव॑र्वद्धेह्य॒स्मे। प्रोष्व॑स्मै पुरोर॒थम्। इन्द्रा॑य शू॒षम॑र्चत॥३४॥

%2.5.8.2
अ॒भीके॑ चिदु लोक॒कृत्। स॒ङ्गे स॒मत्सु॑ वृत्र॒हा। अ॒स्माकं॑ बोधि चोदि॒ता। नभ॑न्तामन्य॒केषाम्। ज्या॒का अधि॒ धन्व॑सु। इन्द्रं॑ व॒य शु॑ना॒सीरम्। अ॒स्मिन् य॒ज्ञे ह॑वामहे। आ वाजै॒रुप॑ नो गमत्। इन्द्रा॑य॒ शुना॒सीरा॑य। स्रु॒चा जु॑हुत नो ह॒विः॥३५॥

%2.5.8.3
जु॒षतां॒ प्रति॒ मेधि॑रः। प्र ह॒व्यानि॑ घृ॒तव॑न्त्यस्मै। हर्य॑श्वाय भरता स॒जोषा। इन्द्र॒र्तुभि॒र्ब्रह्म॑णा वावृधा॒नः। शु॒ना॒सी॒री ह॒विरि॒दं जु॑षस्व। वय॑ सुप॒र्णा उप॑सेदु॒रिन्द्रम्। प्रि॒यमे॑धा॒ ऋष॑यो॒ नाध॑मानाः। अप॑ ध्वा॒न्तमूर्णु॒हि पू॒र्धि चक्षु॑। मु॒मु॒ग्ध्य॑स्मान्नि॒धये॑ऽव ब॒द्धान्। बृ॒हदिन्द्रा॑य गायत॥३६॥

%2.5.8.4
मरु॑तो वृत्र॒हन्त॑मम्। येन॒ ज्योति॒रज॑नयन्नृता॒वृध॑। दे॒वं दे॒वाय॒ जागृ॑वि। कामि॒हैका॒ क इ॒मे प॑त॒ङ्गाः। मा॒न्था॒लाः कुलि॒परि॑मापतन्ति। अना॑वृतैना॒न्प्रध॑मन्तु दे॒वाः। सौप॑र्णं॒ चक्षु॑स्त॒नुवा॑ विदेय। ए॒वा व॑न्दस्व॒ वरु॑णं बृ॒हन्तम्। न॒म॒स्याधीर॑म॒मृत॑स्य गो॒पाम्। स न॒ शर्म॑ त्रि॒वरू॑थं॒ वियसत्॥३७॥

%2.5.8.5
यू॒यं पा॑त स्व॒स्तिभि॒ सदा॑ नः। नाके॑ सुप॒र्णमुप॒ यत्पत॑न्तम्। हृ॒दा वेन॑न्तो अ॒भ्यच॑क्षत त्वा। हिर॑ण्यपक्षं॒ वरु॑णस्य दू॒तम्। य॒मस्य॒ योनौ॑ शकु॒नं भु॑र॒ण्युम्। शं नो॑ दे॒वीर॒भिष्ट॑ये। आपो॑ भवन्तु पी॒तये। शय्योँर॒भि स्र॑वन्तु नः। ईशा॑ना॒ वार्या॑णाम्। क्षय॑न्तीश्चर्\mbox{}षणी॒नाम्॥३८॥

%2.5.8.6
अ॒पो या॑चामि भेष॒जम्। अ॒प्सु मे॒ सोमो॑ अब्रवीत्। अ॒न्तर्विश्वा॑नि भेष॒जा। अ॒ग्निं च॑ वि॒श्वश॑म्भुवम्। आप॑श्च वि॒श्वभे॑षजीः। यद॒प्सु ते॑ सरस्वति। गोष्वश्वे॑षु॒ यन्मधु॑। तेन॑ मे वाजिनीवति। मुख॑मङ्ग्धि सरस्वति। या सर॑स्वती वैशम्भ॒ल्या॥३९॥

%2.5.8.7
तस्यां मे रास्व। तस्यास्ते भक्षीय। तस्यास्ते भूयिष्ठ॒भाजो॑ भूयास्म। अ॒हं त्वद॑स्मि॒ मद॑सि॒ त्वमे॒तत्। ममा॑सि॒ योनि॒स्तव॒ योनि॑रस्मि। ममै॒व सन्वह॑ ह॒व्यान्य॑ग्ने। पु॒त्रः पि॒त्रे लो॑क॒कृज्जा॑तवेदः। इ॒हैव सन्तत्र॒ सन्तं॑ त्वाऽग्ने। प्रा॒णेन॑ वा॒चा मन॑सा बिभर्मि। ति॒रो मा॒ सन्त॒मायु॒र्मा प्रहा॑सीत्॥४०॥

%2.5.8.8
ज्योति॑षा त्वा वैश्वान॒रेणोप॑तिष्ठे। अ॒यं ते॒ योनि॑र्\mbox{}ऋ॒त्विय॑। यतो॑ जा॒तो अरो॑चथाः। तं जा॒नन्न॑ग्न॒ आरो॑ह। अथा॑ नो वर्धया र॒यिम्। या ते॑ अग्ने य॒ज्ञिया॑ त॒नूस्तयेह्यारो॑हा॒त्माऽऽत्मानम्। अच्छा॒ वसू॑नि कृ॒ण्वन्न॒स्मे नर्या॑ पु॒रूणि॑। य॒ज्ञो भू॒त्वा य॒ज्ञमा सी॑द॒ स्वां योनिम्। जात॑वेदो॒ भुव॒ आ जाय॑मान॒ सक्ष॑य॒ एहि॑। उ॒पाव॑रोह जातवेद॒ पुन॒स्त्वम्॥४१॥

%2.5.8.9
दे॒वेभ्यो॑ ह॒व्यं व॑ह नः प्रजा॒नन्। आयु॑ प्र॒जा र॒यिम॒स्मासु॑ धेहि। अज॑स्रो दीदिहि नो दुरो॒णे। तमिन्द्रं॑ जोहवीमि म॒घवा॑नमु॒ग्रम्। स॒त्रा दधा॑न॒मप्र॑तिष्कुत॒ शवासि। महि॑ष्ठो गी॒र्भिरा च॑ य॒ज्ञियो॑ऽव॒वर्त॑त्। रा॒ये नो॒ विश्वा॑ सु॒पथा॑ कृणोतु व॒ज्री। त्रिक॑द्रुकेषु महि॒षो यवा॑शिरं तुवि॒शुष्म॑स्तृ॒पत्। सोम॑मपिब॒द्विष्णु॑ना सु॒तं यथाऽव॑शत्। स ईं ममाद॒ महि॒ कर्म॒ कर्त॑वे म॒हामु॒रुम्॥४२॥

%2.5.8.10
सैन सश्चद्दे॒वं दे॒वः स॒त्यमिन्दु स॒त्य इन्द्र॑। वि॒दद्यती॑ स॒रमा॑ रु॒ग्णमद्रे। महि॒ पाथ॑ पू॒र्व्य स॒द्ध्रिय॑क्कः। अग्रं॑ नयत्सु॒पद्यक्ष॑राणाम्। अच्छा॒ रवं॑ प्रथ॒मा जा॑न॒तीगात्। वि॒दद्गव्य स॒रमा॑ दृ॒ढमू॒र्वम्। येना॒नुकं॒ मानु॑षी॒ भोज॑ते॒ विट्। आ ये विश्वा स्वप॒त्यानि॑ च॒क्रुः। कृ॒ण्वा॒नासो॑ अमृत॒त्वाय॑ गा॒तुम्। त्वं नृभि॑र्नृपते दे॒वहू॑तौ ॥४३॥

%2.5.8.11
भूरी॑णि वृ॒त्वा ह॑र्यश्व हसि। त्वन्निद॑स्यु॒ञ्चुमु॑रिम्। धुनिं॒ चास्वा॑पयो द॒भीत॑ये सु॒हन्तु॑। ए॒वा पा॑हि प्र॒त्नथा॒ मन्द॑तु त्वा। श्रु॒धि ब्रह्म॑ वावृधस्वो॒त गी॒र्भिः। आ॒विः सुर्यं॑ कृणु॒हि पी॒पिही॒षः। ज॒हि शत्रू र॒भि गा इ॑न्द्र तृन्धि। अग्ने॒ बाध॑स्व॒ वि मृधो॑ नुदस्व। अपामी॑वा॒ अप॒ रक्षासि सेध। अ॒स्मात्स॑मु॒द्राद्बृ॑ह॒तो दि॒वो न॑॥४४॥

%2.5.8.12
अ॒पां भू॒मान॒मुप॑ नः सृजे॒ह। यज्ञ॒ प्रति॑तिष्ठ सुम॒तौ सु॒शेवा॒ आ त्वा। वसू॑नि पुरु॒धा वि॑शन्तु। दी॒र्घमायु॒र्यज॑मानाय कृ॒ण्वन्। अधा॒मृते॑न जरि॒तार॑मङ्ग्धि। इन्द्र॑ शु॒नाव॒द्वित॑नोति॒ सीरम्। सं॒व॒त्स॒रस्य॑ प्रति॒माण॑मे॒तत्। अ॒र्कस्य॒ ज्योति॒स्तदिदा॑स॒ ज्येष्ठम्। सं॒व॒त्स॒र शु॒नव॒त्सीर॑मे॒तत्। इन्द्र॑स्य॒ राध॒ प्रय॑तं पु॒रु त्मना। तद॑र्करू॒पं वि॒मिमा॑नमेति। द्वाद॑शारे॒ प्रति॑तिष्ठ॒तीद्वृषा। अ॒श्वा॒यन्तो॑ ग॒व्यन्तो॑ वा॒जय॑न्तः। हवा॑महे॒ त्वोप॑गन्त॒वा उ॑। आ॒भूष॑न्तस्त्वा सुम॒तौ नवा॑याम्। व॒यमि॑न्द्र त्वा शु॒न हु॑वेम॥४५॥\anuvakamend[अ॒र्च॒त॒ ह॒विर्गा॑यत यसच्चर्\mbox{}षणी॒नां वै॑शम्भ॒ल्या हा॑सी॒त्त्वमु॒रुं दे॒वहू॑तौ न॒स्त्मना॒ षट्च॑]




\prashnaend{प्रा॒ण उ॒देहि॒ पुन॒रा नो॑ भर य॒ज्ञो रा॒यो वार्त्र॑हत्याय॒ वसू॑ना॒ स ईं पाह्य॒ष्टौ॥८॥}{प्रा॒णो र॑क्ष॒त्यगृ॑भीता धाराव॒रा म॒रुतो॑ दीर्घायु॒त्वाय॒ ज्योति॑षा त्वा॒ पञ्च॑चत्वारिशत्॥४५॥}{प्रा॒णः शु॒न हु॑वेम॥}{हरि॑ ओम्॥}{इति श्रीकृष्णयजुर्वेदीयतैत्तिरीयब्राह्मणे द्वितीयाष्टके पञ्चमः प्रपाठकः समाप्तः॥}
\clearpage
\sect{षष्ठमः प्रश्नः}
\setcounter{anuvakam}{0}
\dnsub{तैत्तिरीयब्राह्मणे द्वितीयाष्टके षष्ठः प्रपाठकः}

%2.6.1.1
स्वा॒द्वीं त्वा स्वा॒दुना। ती॒व्रां ती॒व्रेण॑। अ॒मृता॑म॒मृते॑न। मधु॑मतीं॒ मधु॑मता। सृ॒जामि॒ स सोमे॑न। सोमोऽस्य॒श्विभ्यां पच्यस्व। सर॑स्वत्यै पच्यस्व। इन्द्रा॑य सु॒त्राम्णे॑ पच्यस्व। परी॒तो षि॑ञ्चता सु॒तम्। सोमो॒ य उ॑त्त॒म ह॒विः॥१॥

%2.6.1.2
द॒ध॒न्वा यो नर्यो॑ अ॒प्स्व॑न्तरा। सु॒षाव॒ सोम॒मद्रि॑भिः। पु॒नातु॑ ते परि॒स्रुतम्। सोम॒ सूर्य॑स्य दुहि॒ता। वारे॑ण॒ शश्व॑ता॒ तना। वा॒युः पू॒तः प॒वित्रे॑ण। प्राङ्ख्सोमो॒ अति॑द्रुतः। इन्द्र॑स्य॒ युज्य॒ सखा। वा॒युः पू॒तः प॒वित्रे॑ण। प्र॒त्यङ्ख्सोमो॒ अति॑द्रुतः॥२॥

%2.6.1.3
इन्द्र॑स्य॒ युज्य॒ सखा। ब्रह्म॑ क्ष॒त्रं प॑वते॒ तेज॑ इन्द्रि॒यम्। सुर॑या॒ सोम॑ सु॒त आसु॑तो॒ मदा॑य। शु॒क्रेण॑ देव दे॒वता पिपृग्धि। रसे॒नान्नं॒ यज॑मानाय धेहि। कु॒विद॒ङ्ग यव॑मन्तो॒ यवं॑चित्। यथा॒ दान्त्य॑नुपू॒र्वं वि॒यूय॑। इ॒हेहै॑षां कृणुत॒ भोज॑नानि। ये ब॒र्‌हिषो॒ नमो॑वृक्तिं॒ न ज॒ग्मुः। उ॒प॒या॒मगृ॑हीतोऽस्य॒श्विभ्यां त्वा॒ जुष्टं॑ गृह्णामि॥३॥

%2.6.1.4
सर॑स्वत्या॒ इन्द्रा॑य सु॒त्राम्णे। ए॒ष ते॒ योनि॒स्तेज॑से त्वा। वी॒र्या॑य त्वा॒ बला॑य त्वा। तेजो॑ऽसि॒ तेजो॒ मयि॑ धेहि। वी॒र्य॑मसि वी॒र्यं॑ मयि॑ धेहि। बल॑मसि॒ बलं॒ मयि॑ धेहि। नाना॒ हि वां दे॒वहि॑त॒ सद॑ कृ॒तम्। मा ससृ॑क्षाथां पर॒मे व्यो॑मन्। सुरा॒ त्वमसि॑ शु॒ष्मिणी॒ सोम॑ ए॒षः। मा मा॑ हिसी॒ स्वां योनि॑मावि॒शन्॥४॥

%2.6.1.5
उ॒प॒या॒मगृ॑हीतोऽस्याश्वि॒नं तेज॑। सा॒र॒स्व॒तं वी॒र्यम्। ऐ॒न्द्रं बलम्। ए॒ष ते॒ योनि॒र्मोदा॑य त्वा। आ॒न॒न्दाय॑ त्वा॒ मह॑से त्वा। ओजो॒ऽस्योजो॒ मयि॑ धेहि। म॒न्युर॑सि म॒न्युं मयि॑ धेहि। महो॑ऽसि॒ महो॒ मयि॑ धेहि। सहो॑ऽसि॒ सहो॒ मयि॑ धेहि। या व्या॒घ्रं विषू॑चिका। उ॒भौ वृकं॑ च॒ रक्ष॑ति। श्ये॒नं प॑त॒त्रिण सि॒हम्। सेमं पा॒त्वह॑सः। सं॒पृच॑ स्थ॒ सं मा॑ भ॒द्रेण॑ पृङ्क्त। वि॒पृच॑ स्थ॒ वि मा॑ पा॒प्मना॑ पृङ्क्त॥५॥\anuvakamend[ह॒विः प्र॒त्यङ्ख्सोमो॒ अति॑द्रुतो गृह्णाम्यावि॒शन्विषू॑चिका॒ पञ्च॑ च]

%2.6.2.1
सोमो॒ राजा॒ऽमृत सु॒तः। ऋ॒जी॒षेणा॑जहान्मृ॒त्युम्। ऋ॒तेन॑ स॒त्यमि॑न्द्रि॒यम्। विपान शु॒क्रमन्ध॑सः। इन्द्र॑स्येन्द्रि॒यम्। इ॒दं पयो॒ऽमृतं॒ मधु॑। सोम॑म॒द्भ्यो व्य॑पिबत्। छन्द॑सा ह॒सः शु॑चि॒षत्। ऋ॒तेन॑ स॒त्यमि॑न्द्रि॒यम्। अ॒द्भ्यः क्षी॒रव्व्यँ॑पिबत्॥६॥

%2.6.2.2
क्रुङ्ङाङ्गिर॒सो धि॒या। ऋ॒तेन॑ स॒त्यमि॑न्द्रि॒यम्। अन्नात्परि॒स्रुतो॒ रसम्। ब्रह्म॑णा॒ व्य॑पिबत् क्ष॒त्रम्। ऋ॒तेन॑ स॒त्यमि॑न्द्रि॒यम्। रेतो॒ मूत्रं॒ विज॑हाति। योनिं॑ प्रवि॒शदि॑न्द्रि॒यम्। गर्भो॑ ज॒रायु॒णाऽऽवृ॑तः। उल्बं॑ जहाति॒ जन्म॑ना। ऋ॒तेन॑ स॒त्यमि॑न्द्रि॒यम्॥७॥

%2.6.2.3
वेदे॑न रू॒पे व्य॑करोत्। स॒ता॒स॒ती प्र॒जाप॑तिः। ऋ॒तेन॑ स॒त्यमि॑न्द्रि॒यम्। सोमे॑न॒ सोमौ॒ व्य॑पिबत्। सु॒ता॒सु॒तौ प्र॒जाप॑तिः। ऋ॒तेन॑ स॒त्यमि॑न्द्रि॒यम्। दृ॒ष्ट्वा रू॒पे व्याक॑रोत्। स॒त्या॒नृ॒ते प्र॒जाप॑तिः। अश्र॑द्धा॒मनृ॒तेऽद॑धात्। श्र॒द्धा स॒त्ये प्र॒जाप॑तिः। ऋ॒तेन॑ स॒त्यमि॑न्द्रि॒यम्। दृ॒ष्ट्वा प॑रि॒स्रुतो॒ रसम्। शु॒क्रेण॑ शु॒क्रव्व्यँ॑पिबत्। पय॒ सोमं॑ प्र॒जाप॑तिः। ऋ॒तेन॑ स॒त्यमि॑न्द्रि॒यम्। विपान शु॒क्रमन्ध॑सः। इन्द्र॑स्येन्द्रि॒यम्। इ॒दं पयो॒ऽमृतं॒ मधु॑॥८॥\anuvakamend[अ॒द्भ्यः क्षी॒रव्व्यँ॑पिब॒ज्जन्म॑न॒र्तेन॑ स॒त्यमि॑न्द्रि॒य श्र॒द्धा स॒त्ये प्र॒जाप॑तिर॒ष्टौ च॑]

%2.6.3.1
सुरा॑वन्तं बर्‌हि॒षद सु॒वीरम्। य॒ज्ञ हि॑न्वन्ति महि॒षा नमो॑भिः। दधा॑ना॒ सोम॑न्दि॒वि दे॒वता॑सु। मदे॒मेन्द्रं॒ यज॑मानाः स्व॒र्काः। यस्ते॒ रस॒ सम्भृ॑त॒ ओष॑धीषु। सोम॑स्य॒ शुष्म॒ सुर॑या सु॒तस्य॑। तेन॑ जिन्व॒ यज॑मानं॒ मदे॑न। सर॑स्वतीम॒श्विना॒विन्द्र॑म॒ग्निम्। यम॒श्विना॒ नमु॑चेरासु॒रादधि॑। सर॑स्व॒त्यस॑नोदिन्द्रि॒याय॑॥९॥

%2.6.3.2
इ॒मन्त शु॒क्रं मधु॑मन्त॒मिन्दुम्। सोम॒ राजा॑नमि॒ह भ॑क्षयामि। यदत्र॑ रि॒प्त र॒सिन॑ सु॒तस्य॑। यदिन्द्रो॒ अपि॑ब॒च्छची॑भिः। अ॒हन्तद॑स्य॒ मन॑सा शि॒वेन॑। सोम॒ राजा॑नमि॒ह भ॑क्षयामि। पि॒तृभ्य॑ स्वधा॒विभ्य॑ स्व॒धा नम॑। पि॒ता॒म॒हेभ्य॑ स्वधा॒विभ्य॑ स्व॒धा नम॑। प्रपि॑तामहेभ्यः स्वधा॒विभ्य॑ स्व॒धा नम॑। अक्ष॑न्पि॒तर॑॥१०॥

%2.6.3.3
अमी॑मदन्त पि॒तर॑। अती॑तृपन्त पि॒तर॑। अमी॑मृजन्त पि॒तर॑। पित॑र॒ शुन्ध॑ध्वम्। पु॒नन्तु॑ मा पि॒तर॑ सो॒म्यास॑। पु॒नन्तु॑ मा पिताम॒हाः। पु॒नन्तु॒ प्रपि॑तामहाः। प॒वित्रे॑ण श॒तायु॑षा। पु॒नन्तु॑ मा पिताम॒हाः। पु॒नन्तु॒ प्रपि॑तामहाः॥११॥

%2.6.3.4
प॒वित्रे॑ण श॒तायु॑षा। विश्व॒मायु॒र्व्य॑श्ञवै। अग्न॒ आयूषि पव॒सेऽग्ने॒ पव॑स्व। पव॑मान॒ सुव॒र्जन॑ पु॒नन्तु॑ मा देवज॒नाः। जात॑वेदः प॒वित्र॑व॒द्यत्ते॑ प॒वित्र॑म॒र्चिषि॑। उ॒भाभ्यान्देव सवितर्वैश्वदे॒वी पु॑न॒ती। ये स॑मा॒नाः सम॑नसः। पि॒तरो॑ यम॒राज्ये। तेषां लो॒कः स्व॒धा नम॑। य॒ज्ञो दे॒वेषु॑ कल्पताम्॥१२॥

%2.6.3.5
ये स॑जा॒ताः सम॑नसः। जी॒वा जी॒वेषु॑ माम॒काः। तेषा॒ श्रीर्मयि॑ कल्पताम्। अ॒स्मिल्लोँ॒के श॒त समा। द्वे स्रु॒ती अ॑शृणवं पितृ॒णाम्। अ॒हन्दे॒वाना॑मु॒त मर्त्या॑नाम्। याभ्या॑मि॒दं विश्व॒मेज॒त्समे॑ति। यद॑न्त॒रा पि॒तरं॑ मा॒तरं॑ च। इ॒द ह॒विः प्र॒जन॑नं मे अस्तु। दश॑वीर स॒र्वग॑ण स्व॒स्तये। आ॒त्म॒सनि॑ प्रजा॒सनि॑। प॒शु॒सन्य॑भय॒सनि॑ लोक॒सनि॑। अ॒ग्निः प्र॒जां ब॑हु॒लां मे॑ करोतु। अन्नं॒ पयो॒ रेतो॑ अ॒स्मासु॑ धत्त। रा॒यस्पोष॒मिष॒मूर्ज॑म॒स्मासु॑ दीधर॒त्स्वाहा॥१३॥\anuvakamend[इ॒न्द्रि॒याय॑ पि॒तर॑ श॒तायु॑षा पु॒नन्तु॑ मा पिताम॒हाः पु॒नन्तु॒ प्रपि॑तामहाः कल्पता स्व॒स्तये॒ पञ्च॑ च]

%2.6.4.1
सीसे॑न॒ तन्त्रं॒ मन॑सा मनी॒षिण॑। ऊ॒र्णा॒सू॒त्रेण॑ क॒वयो॑ वयन्ति। अ॒श्विना॑ य॒ज्ञ स॑वि॒ता सर॑स्वती। इन्द्र॑स्य रू॒पं वरु॑णो भिष॒ज्यन्। तद॑स्य रू॒पम॒मृत॒ शची॑भिः। ति॒स्रोऽद॑धुर्दे॒वता सररा॒णाः। लोमा॑नि॒ शष्पैर्बहु॒धा न तोक्म॑भिः। त्वग॑स्य मा॒सम॑भव॒न्न ला॒जाः। तद॒श्विना॑ भि॒षजा॑ रु॒द्रव॑र्तनी। सर॑स्वती वयति॒ पेशो॒ अन्त॑रः॥१४॥

%2.6.4.2
अस्थि॑ म॒ज्जानं॒ मास॑रैः। का॒रो॒त॒रेण॒ दध॑तो॒ गवान्त्व॒चि। सर॑स्वती॒ मन॑सा पेश॒लं वसु॑। नास॑त्याभ्यां वयति दर्‌श॒तं वपु॑। रसं॑ परि॒स्रुता॒ न रोहि॑तम्। न॒ग्नहु॒र्धीर॒स्तस॑र॒न्न वेम॑। पय॑सा शु॒क्रम॒मृतं॑ ज॒नित्रम्। सुर॑या॒ मूत्राज्जनयन्ति॒ रेत॑। अपाम॑तिन्दुर्म॒तिं बाध॑मानाः। ऊव॑ध्यं॒ वात स॒बुव॒न्तदा॒रात्॥१५॥

%2.6.4.3
इन्द्र॑ सु॒त्रामा॒ हृद॑येन स॒त्यम्। पु॒रो॒डाशे॑न सवि॒ता ज॑जान। यकृ॑त्क्लो॒मानं॒ वरु॑णो भिष॒ज्यन्। मत॑स्ने वाय॒व्यैर्न मि॑नाति पि॒त्तम्। आ॒न्त्राणि॑ स्था॒ली मधु॒ पिन्व॑माना। गुदा॒ पात्रा॑णि सु॒दुघा॒ न धे॒नुः। श्ये॒नस्य॒ पत्र॒न्न प्ली॒हा शची॑भिः। आ॒स॒न्दी नाभि॑रु॒दर॒न्न मा॒ता। कु॒म्भो व॑नि॒ष्ठुर्ज॑नि॒ता शची॑भिः। यस्मि॒न्नग्रे॒ योन्या॒ङ्गर्भो॑ अ॒न्तः ॥१६॥

%2.6.4.4
प्ला॒शीर्व्य॑क्तः श॒तधा॑र॒ उत्स॑। दु॒हे न कु॒म्भी स्व॒धां पि॒तृभ्य॑। मुख॒ सद॑स्य॒ शिर॒ इत्सदे॑न। जि॒ह्वा प॒वित्र॑म॒श्विना॒ स सर॑स्वती। चप्प॒न्न पा॒युर्भि॒षग॑स्य॒ वाल॑। व॒स्तिर्न शेपो॒ हर॑सा तर॒स्वी। अ॒श्विभ्यां॒ चक्षु॑र॒मृतं॒ ग्रहाभ्याम्। छागे॑न॒ तेजो॑ ह॒विषा॑ शृ॒तेन॑। पक्ष्मा॑णि गो॒धूमै॒ क्व॑लैरु॒तानि॑। पेशो॒ न शु॒क्लमसि॑तं वसाते॥१७॥

%2.6.4.5
अवि॒र्न मे॒षो न॒सि वी॒र्या॑य। प्रा॒णस्य॒ पन्था॑ अ॒मृतो॒ ग्रहाभ्याम्। सर॑स्व॒त्युप॒वाकैर्व्या॒नम्। नस्या॑नि ब॒र्॒हिर्बद॑रैर्जजान। इन्द्र॑स्य रू॒पमृ॑ष॒भो बला॑य। कर्णाभ्या॒ श्रोत्र॑म॒मृत॒ङ्ग्रहाभ्याम्। यवा॒ न ब॒र्॒हिर्भ्रु॒वि केस॑राणि। क॒र्कन्धु॑ जज्ञे॒ मधु॑ सार॒घं मुखात्। आ॒त्मन्नु॒पस्थे॒ न वृक॑स्य॒ लोम॑। मुखे॒ श्मश्रू॑णि॒ न व्याघ्रलो॒मम्॥१८॥

%2.6.4.6
केशा॒ न शी॒र्॒षन्‌ यश॑से श्रि॒यै शिखा। सि॒हस्य॒ लोम॒ त्विषि॑रिन्द्रि॒याणि॑। अङ्गान्या॒त्मन्भि॒षजा॒ तद॒श्विना। आ॒त्मान॒मङ्गै॒ सम॑धा॒त्सर॑स्वती। इन्द्र॑स्य रू॒प श॒तमा॑न॒मायु॑। च॒न्द्रेण॒ ज्योति॑र॒मृत॒न्दधा॑ना। सर॑स्वती॒ योन्या॒ङ्गर्भ॑म॒न्तः। अ॒श्विभ्यां॒ पत्नी॒ सुकृ॑तं बिभर्ति। अ॒पा रसे॑न॒ वरु॑णो॒ न साम्ना। इन्द्र श्रि॒यै ज॒नय॑न्न॒प्सु राजा। तेज॑ पशू॒ना ह॒विरि॑न्द्रि॒याव॑त्। प॒रि॒स्रुता॒ पय॑सा सार॒घं मधु॑। अ॒श्विभ्यान्दु॒ग्धं भि॒षजा॒ सर॑स्वत्या सुतासु॒ताभ्याम्। अ॒मृत॒ सोम॒ इन्दु॑॥१९॥\anuvakamend[अन्त॑र आ॒राद॒न्तर्व॑साते व्याघ्रलो॒म राजा॑ च॒त्वारि॑ च]

%2.6.5.1
मि॒त्रो॑ऽसि॒ वरु॑णोऽसि। सम॒हं विश्वैर्दे॒वैः। क्ष॒त्रस्य॒ नाभि॑रसि। क्ष॒त्रस्य॒ योनि॑रसि। स्यो॒नामा सी॑द। सु॒षदा॒मा सी॑द। मा त्वा॑ हिसीत्। मा मा॑ हिसीत्। निष॑साद धृ॒तव्र॑तो॒ वरु॑णः। प॒स्त्यास्वा॥२०॥

%2.6.5.2
साम्राज्याय सु॒क्रतु॑। दे॒वस्य॑ त्वा सवि॒तुः प्र॑स॒वे। अ॒श्विनोर्बा॒हुभ्याम्। पू॒ष्णो हस्ताभ्याम्। अ॒श्विनो॒र्भैष॑ज्येन। तेज॑से ब्रह्मवर्च॒साया॒भिषि॑ञ्चामि। दे॒वस्य॑ त्वा सवि॒तुः प्र॑स॒वे। अ॒श्विनोर्बा॒हुभ्याम्। पू॒ष्णो हस्ताभ्याम्। सर॑स्वत्यै॒ भैष॑ज्येन॥२१॥

%2.6.5.3
वी॒र्या॑या॒न्नाद्या॑या॒भिषि॑ञ्चामि। दे॒वस्य॑ त्वा सवि॒तुः प्र॑स॒वे। अ॒श्विनोर्बा॒हुभ्याम्। पू॒ष्णो हस्ताभ्याम्। इन्द्र॑स्येन्द्रि॒येण॑। श्रि॒यै यश॑से॒ बला॑या॒भिषि॑ञ्चामि। को॑ऽसि कत॒मो॑ऽसि। कस्मै त्वा॒ काय॑ त्वा। सुश्लो॒काँ (४) सुम॑ङ्ग॒लाँ (४) सत्य॑रा॒जा (३) न्। शिरो॑ मे॒ श्रीः॥२२॥

%2.6.5.4
यशो॒ मुखम्। त्विषि॒ केशाश्च॒ श्मश्रू॑णि। राजा॑ मे प्रा॒णो॑ऽमृतम्। स॒म्राट्चक्षु॑। वि॒राट्छ्रोत्रम्। जि॒ह्वा मे॑ भ॒द्रम्। वाङ्मह॑। मनो॑ म॒न्युः। स्व॒राड्भाम॑। मोदा प्रमो॒दा अ॒ङ्गुली॒रङ्गा॑नि॥२३॥

%2.6.5.5
चि॒त्तं मे॒ सह॑। बा॒हू मे॒ बल॑मिन्द्रि॒यम्। हस्तौ॑ मे॒ कर्म॑ वी॒र्यम्। आ॒त्मा क्ष॒त्रमुरो॒ मम॑। पृ॒ष्टीर्मे॑ रा॒ष्ट्रमु॒दर॒मसौ। ग्री॒वाश्च॒ श्रोण्यौ। ऊ॒रू अ॑र॒त्नी जानु॑नी। विशो॒ मेऽङ्गा॑नि स॒र्वत॑। नाभि॑र्मे चि॒त्तं वि॒ज्ञानम्। पा॒युर्मेऽप॑चितिर्भ॒सत्॥२४॥

%2.6.5.6
आ॒न॒न्द॒न॒न्दावा॒ण्डौ मे। भग॒ सौभाग्यं॒ पस॑। जङ्घाभ्यां प॒द्भ्यां धर्मोऽस्मि। वि॒शि राजा॒ प्रति॑ष्ठितः। प्रति॑ क्ष॒त्रे प्रति॑तिष्ठामि रा॒ष्ट्रे। प्रत्यश्वे॑षु॒ प्रति॑तिष्ठामि॒ गोषु॑। प्रत्यङ्गे॑षु॒ प्रति॑तिष्ठाम्या॒त्मन्। प्रति॑ प्रा॒णेषु॒ प्रति॑तिष्ठामि पु॒ष्टे। प्रति॒ द्यावा॑पृथि॒व्योः। प्रति॑तिष्ठामि य॒ज्ञे॥२५॥

%2.6.5.7
त्र॒या दे॒वा एका॑दश। त्र॒य॒स्त्रि॒शाः सु॒राध॑सः। बृह॒स्पति॑पुरोहिताः। दे॒वस्य॑ सवि॒तुः स॒वे। दे॒वा दे॒वैर॑वन्तु मा। प्र॒थ॒मा द्वि॒तीयै। द्वि॒तीयास्तृ॒तीयै। तृ॒तीया स॒त्येन॑। स॒त्यं य॒ज्ञेन॑। य॒ज्ञो यजु॑र्भिः॥२६॥

%2.6.5.8
यजूषि॒ साम॑भिः। सामान्यृ॒ग्भिः। ऋचो॑ या॒ज्या॑भिः। या॒ज्या॑ वषट्का॒रैः। व॒ष॒ट्का॒रा आहु॑तिभिः। आहु॑तयो मे॒ कामा॒न्त्सम॑र्धयन्तु। भूः स्वाहा। लोमा॑नि॒ प्रय॑ति॒र्मम॑। त्वङ्म॒ आन॑ति॒राग॑तिः। मा॒सं म॒ उप॑नतिः। वस्वस्थि॑। म॒ज्जा म॒ आन॑तिः॥२७॥\anuvakamend[प॒स्त्यास्वा सर॑स्वत्यै॒ भैष॑ज्येन॒ श्रीरङ्गा॑नि भ॒सद्य॒ज्ञे य॒ज्ञो यजु॑र्भि॒रुप॑नति॒र्द्वे च॑]

%2.6.6.1
यद्दे॑वा देव॒हेड॑नम्। देवा॑सश्चकृ॒मा व॒यम्। अ॒ग्निर्मा॒ तस्मा॒देन॑सः। विश्वान्मुञ्च॒त्वह॑सः। यदि॒ दिवा॒ यदि॒ नक्तम्। एनासि चकृ॒मा व॒यम्। वा॒युर्मा॒ तस्मा॒देन॑सः। विश्वान्मुञ्च॒त्वह॑सः। यदि॒ जाग्र॒द्यदि॒ स्वप्ने। एनासि चकृ॒मा व॒यम्॥२८॥

%2.6.6.2
सूर्यो॑ मा॒ तस्मा॒देन॑सः। विश्वान्मुञ्च॒त्वह॑सः। यद्ग्रामे॒ यदर॑ण्ये। यत्स॒भायां॒ यदि॑न्द्रि॒ये। यच्छू॒द्रे यद॒र्ये। एन॑श्चकृ॒मा व॒यम्। यदेक॒स्याधि॒ धर्म॑णि। तस्या॑व॒यज॑नमसि। यदापो॒ अघ्नि॑या॒ वरु॒णेति॒ शपा॑महे। ततो॑ वरुण नो मुञ्च॥२९॥

%2.6.6.3
अव॑भृथ निचङ्कुण निचे॒रुर॑सि निचङ्कुण। अव॑ दे॒वैर्दे॒वकृ॑त॒मेनो॑ऽयाट्। अव॒ मर्त्यै॒र्मर्त्य॑कृतम्। उ॒रोरा नो॑ देव रि॒षस्पा॑हि। सु॒मि॒त्रा न॒ आप॒ ओष॑धयः सन्तु। दु॒र्मि॒त्रास्तस्मै॑ भूयासुः। योऽस्मान्द्वेष्टि॑। यं च॑ व॒यं द्वि॒ष्मः। द्रु॒प॒दादि॒वेन्मु॑मुचा॒नः। स्वि॒न्नः स्ना॒त्वी मला॑दिव॥३०॥

%2.6.6.4
पू॒तं प॒वित्रे॑णे॒वाज्यम्। आप॑ शुन्धन्तु॒ मैन॑सः। उद्व॒यन्तम॑स॒स्परि॑। पश्य॑न्तो॒ ज्योति॒रुत्त॑रम्। दे॒वन्दे॑व॒त्रा सूर्यम्। अग॑न्म॒ ज्योति॑रुत्त॒मम्। प्रति॑युतो॒ वरु॑णस्य॒ पाश॑। प्रत्य॑स्तो॒ वरु॑णस्य॒ पाश॑। एधोऽस्येधिषी॒महि॑। स॒मिद॑सि ॥३१॥

%2.6.6.5
तेजो॑ऽसि॒ तेजो॒ मयि॑ धेहि। अ॒पो अन्व॑चारिषम्। रसे॑न॒ सम॑सृक्ष्महि। पय॑स्वा अग्न॒ आग॑मम्। तं मा॒ ससृ॑ज॒ वर्च॑सा। प्र॒जया॑ च॒ धने॑न च। स॒माव॑वर्ति पृथि॒वी। समु॒षाः। समु॒ सूर्य॑। समु॒ विश्व॑मि॒दञ्जग॑त्। वै॒श्वा॒न॒रज्यो॑तिर्भूयासम्। वि॒भुङ्काम॒व्व्यँ॑श्ञवै। भूः स्वाहा॥३२॥\anuvakamend[स्वप्न॒ एनासि चकृ॒मा व॒यं मु॑ञ्च॒ मला॑दिव स॒मिद॑सि॒ जग॒त्रीणि॑ च]

%2.6.7.1
होता॑ यक्षत्स॒मिधेन्द्र॑मि॒डस्प॒दे। नाभा॑ पृथि॒व्या अधि॑। दि॒वो वर्ष्म॒न्त्समि॑ध्यते। ओजि॑ष्ठश्चर्‌षणी॒ सहान्॑। वेत्वाज्य॑स्य॒ होत॒र्यज॑। होता॑ यक्ष॒त्तनू॒नपा॑तम्। ऊ॒तिभि॒र्जेता॑र॒मप॑राजितम्। इन्द्रं॑ दे॒व सु॑व॒र्विदम्। प॒थिभि॒र्मधु॑मत्तमैः। नरा॒शसे॑न॒ तेज॑सा॥३३॥

%2.6.7.2
वेत्वाज्य॑स्य॒ होत॒र्यज॑। होता॑ यक्ष॒दिडा॑भि॒रिन्द्र॑मीडि॒तम्। आ॒जुह्वा॑न॒मम॑र्त्यम्। दे॒वो दे॒वैः सवीर्यः। वज्र॑हस्तः पुरन्द॒रः। वेत्वाज्य॑स्य॒ होत॒र्यज॑। होता॑यक्षद्ब॒र्॒हिषीन्द्र॑न्निषद्व॒रम्। वृ॒ष॒भन्नर्या॑पसम्। वसु॑भीरु॒द्रैरा॑दि॒त्यैः। स॒युग्भि॑र्ब॒र्॒हिरास॑दत्॥३४॥

%2.6.7.3
वेत्वाज्य॑स्य॒ होत॒र्यज॑। होता॑ यक्ष॒दोजो॒ न वी॒र्यम्। सहो॒ द्वार॒ इन्द्र॑मवर्धयन्। सु॒प्रा॒य॒णा विश्र॑यन्तामृता॒वृध॑। द्वार॒ इन्द्रा॑य मी॒ढुषे। वि॒यन्त्वाज्य॑स्य॒ होत॒र्यज॑। होता॑ यक्षदु॒षे इन्द्र॑स्य धे॒नू। सु॒दुघे॑ मा॒तरौ॑ म॒ही। सवा॒तरौ॒ न तेज॑सी। व॒त्समिन्द्र॑मवर्धताम्॥३५॥

%2.6.7.4
वी॒तामाज्य॑स्य॒ होत॒र्यज॑। होता॑ यक्ष॒द्दैव्या॒ होता॑रा। भि॒षजा॒ सखा॑या। ह॒विषेन्द्रं॑ भिषज्यतः। क॒वी दे॒वौ प्रचे॑तसौ। इन्द्रा॑य धत्त इन्द्रि॒यम्। वी॒तामाज्य॑स्य॒ होत॒र्यज॑। होता॑ यक्षत्ति॒स्रो दे॒वीः। त्रय॑स्त्रि॒धात॑वो॒पस॑। इडा॒ सर॑स्वती॒ भार॑ती॥३६॥

%2.6.7.5
म॒हीन्द्र॑पत्नीर्‌ह॒विष्म॑तीः। वि॒यन्त्वाज्य॑स्य॒ होत॒र्यज॑। होता॑ यक्ष॒त्त्वष्टा॑र॒मिन्द्रं॑ दे॒वम्। भि॒षज सु॒यज॑ङ्घृत॒श्रियम्। पु॒रु॒रूप सु॒रेत॑सं म॒घोनिम्। इन्द्रा॑य॒ त्वष्टा॒ दध॑दिन्द्रि॒याणि॑। वेत्वाज्य॑स्य॒ होत॒र्यज॑। होता॑ यक्ष॒द्वन॒स्पतिम्। श॒मि॒तार श॒तक्र॑तुम्। धि॒यो जो॒ष्टार॑मिन्द्रि॒यम्॥३७॥

%2.6.7.6
मध्वा॑ सम॒ञ्जन्प॒थिभि॑ सु॒गेभि॑। स्वदा॑ति ह॒व्यं मधु॑ना घृ॒तेन॑। वेत्वाज्य॑स्य॒ होत॒र्यज॑। होता॑ यक्ष॒दिन्द्र॒ स्वाहाऽऽज्य॑स्य। स्वाहा॒ मेद॑सः। स्वाहा स्तो॒कानाम्। स्वाहा॒ स्वाहा॑कृतीनाम्। स्वाहा॑ ह॒व्यसूक्तीनाम्। स्वाहा॑ दे॒वा आज्य॒पान्। स्वाहेन्द्र हो॒त्राज्जु॑षा॒णाः। इन्द्र॒ आज्य॑स्य वियन्तु। होत॒र्यज॑॥३८॥\anuvakamend[तेज॑साऽऽसददवर्धतां॒ भार॑तीन्द्रि॒यं जु॑षा॒णा द्वे च॑ (स॒मिधेन्द्र॒न्तनू॒नपा॑त॒मिडा॑भिर्ब॒र्॒हिष्योज॑ उ॒षे दैव्या॑ ति॒स्रस्त्वष्टा॑रं॒ वन॒स्पति॒मिन्द्रम् ॥ स॒मिधेन्द्रं॑ च॒तुर्वेत्वेको॑ वि॒यन्तु॒ द्विर्वी॒तामेको॑ वि॒यन्तु॒ द्विर्वेत्वेको॑ वि॒यन्तु॒ होत॒र्यज॑ ॥ )]

%2.6.8.1
समि॑द्ध॒ इन्द्र॑ उ॒षसा॒मनी॑के। पु॒रो॒रुचा॑ पूर्व॒कृद्वा॑वृधा॒नः। त्रि॒भिर्दे॒वैस्त्रि॒शता॒ वज्र॑बाहुः। ज॒घान॑ वृ॒त्रं वि दुरो॑ ववार। नरा॒शस॒ प्रति॒शूरो॒ मिमा॑नः। तनू॒नपा॒त्प्रति॑ य॒ज्ञस्य॒ धाम॑। गोभि॑र्व॒पावा॒न्मधु॑ना सम॒ञ्जन्। हिर॑ण्यैश्च॒न्द्री य॑जति॒ प्रचे॑ताः। ई॒डि॒तो दे॒वैर्‌हरि॑वा अभि॒ष्टिः। आ॒जुह्वा॑नो ह॒विषा॒ शर्ध॑मानः॥३९॥

%2.6.8.2
पु॒र॒न्द॒रो म॒घवा॒न् वज्र॑बाहुः। आया॑तु य॒ज्ञमुप॑नो जुषा॒णः। जु॒षा॒णो ब॒र्‌हिर्हरि॑वान्न॒ इन्द्र॑। प्रा॒चीन सीदत्प्र॒दिशा॑ पृथि॒व्याः। उ॒रु॒व्यचा॒ प्रथ॑मान स्यो॒नम्। आ॒दि॒त्यैर॒क्तं वसु॑भिः स॒जोषा। इन्द्र॒न्दुर॑ कव॒ष्यो॑ धाव॑मानाः। वृषा॑णं यन्तु॒ जन॑यः सु॒पत्नी। द्वारो॑ दे॒वीर॒भितो॒ विश्र॑यन्ताम्। सु॒वीरा॑ वी॒रं प्रथ॑माना॒ महो॑भिः॥४०॥

%2.6.8.3
उ॒षासा॒नक्ता॑ बृह॒ती बृ॒हन्तम्। पय॑स्वती सु॒दुघे॒ शूर॒मिन्द्रम्। पेश॑स्वती॒ तन्तु॑ना स॒व्व्यँय॑न्ती। दै॒वानां दे॒वं य॑जतः सुरु॒क्मे। दैव्या॒ मिमा॑ना॒ मन॑सा पुरु॒त्रा। होता॑रा॒विन्द्रं॑ प्रथ॒मा सु॒वाचा। मू॒र्धन् य॒ज्ञस्य॒ मधु॑ना॒ दधा॑ना। प्रा॒चीनं॒ ज्योति॑र्\mbox{}ह॒विषा॑ वृधातः। ति॒स्रो दे॒वीर्‌ ह॒विषा॒ वर्ध॑मानाः। इन्द्रं॑ जुषा॒णा वृष॑ण॒न्न पत्नी॥४१॥

%2.6.8.4
अच्छि॑न्न॒न्तन्तुं॒ पय॑सा॒ सर॑स्वती। इडा॑ दे॒वी भार॑ती वि॒श्वतूर्तिः। त्वष्टा॒ दध॒दिन्द्रा॑य॒ शुष्मम्। अपा॒कोचि॑ष्टुर्य॒शसे॑ पु॒रूणि॑। वृषा॒ यज॒न्वृष॑णं॒ भूरि॑रेताः। मू॒र्धन् य॒ज्ञस्य॒ सम॑नक्तु दे॒वान्। वन॒स्पति॒रव॑सृष्टो॒ न पाशै। त्मन्या॑ सम॒ञ्जञ्छ॑मि॒ता न दे॒वः। इन्द्र॑स्य ह॒व्यैर्ज॒ठरं॑ पृणा॒नः। स्वदा॑ति ह॒व्यं मधु॑ना घृ॒तेन॑। स्तो॒काना॒मिन्दुं॒ प्रति॒ शूर॒ इन्द्र॑। वृ॒षा॒यमा॑णो वृष॒भस्तु॑रा॒षाट्। घृ॒त॒प्रुषा॒ मधु॑ना ह॒व्यमु॒न्दन्। मू॒र्धन् य॒ज्ञस्य॑ जुषता॒ स्वाहा॥४२॥\anuvakamend[शर्ध॑मानो॒ महो॑भि॒ पत्नीर्घृ॒तेन॑ च॒त्वारि॑ च]

%2.6.9.1
आच॑र्‌षणि॒प्रा वि॒वेष॒ यन्मा। त स॒ध्रीची। स॒त्यमित्तन्न त्वावा अ॒न्यो अस्ति॑। इन्द्र॑ दे॒वो न मर्त्यो॒ ज्यायान्॑। अह॒न्नहिं॑ परि॒शया॑न॒मर्ण॑। अवा॑सृजो॒ऽपो अच्छा॑ समु॒द्रम्। प्रस॑साहिषे पुरुहूत॒ शत्रून्॑। ज्येष्ठ॑स्ते॒ शुष्म॑ इ॒ह रा॒तिर॑स्तु। इन्द्रा भ॑र॒ दक्षि॑णेना॒ वसू॑नि। पति॒ सिन्धू॑नामसि रे॒वती॑नाम्। स शेवृ॑ध॒मधि॑ धाद्द्यु॒म्नम॒स्मे। महि॑ क्ष॒त्रं ज॑ना॒षाडि॑न्द्र॒ तव्यम्। रक्षा॑ च नो म॒घोन॑ पा॒हि सू॒रीन्। रा॒ये च॑ नः स्वप॒त्या इ॒षे धा॥४३॥\anuvakamend[रे॒वती॑नाञ्च॒त्वारि॑ च]

%2.6.10.1
दे॒वं ब॒र्॒हिरिन्द्र सुदे॒वन्दे॒वैः। वी॒रव॑त्स्ती॒र्णं वेद्या॑मवर्धयत्। वस्तोर्वृ॒तं प्राक्तोर्भृ॒तम्। रा॒या ब॒र्॒हिष्म॒तोऽत्य॑गात्। व॒सु॒वने॑ वसु॒धेय॑स्य वेतु॒ यज॑। दे॒वीर्द्वार॒ इन्द्र सङ्घा॒ते। वि॒ड्वीर्याम॑न्नवर्धयन्। आ व॒त्सेन॒ तरु॑णेन कुमा॒रेण॑ चमीवि॒ता अपार्वा॑णम्। रे॒णुक॑काटन्नुदन्ताम्। व॒सु॒वने॑ वसु॒धेय॑स्य वियन्तु॒ यज॑॥४४॥

%2.6.10.2
दे॒वी उ॒षासा॒नक्ता। इन्द्रं॑ य॒ज्ञे प्र॑य॒त्य॑ह्वेताम्। दैवी॒र्विश॒ प्राया॑सिष्टाम्। सुप्री॑ते॒ सुधि॑ते अभूताम्। व॒सु॒वने॑ वसु॒धेय॑स्य वीतां॒ यज॑। दे॒वी जोष्ट्री॒ वसु॑धिती। दे॒वमिन्द्र॑मवर्धताम्। अयाव्य॒न्याघा द्वेषासि। आन्यावाक्षी॒द्वसु॒ वार्या॑णि। यज॑मानाय शिक्षि॒ते॥४५॥

%2.6.10.3
व॒सु॒वने॑ वसु॒धेय॑स्य वीतां॒ यज॑। दे॒वी ऊ॒र्जाहु॑ती॒ दुघे॑ सु॒दुघे। पय॒सेन्द्र॑मवर्धताम्। इष॒मूर्ज॑म॒न्याऽवाक्षीत्। सग्धि॒ सपी॑तिम॒न्या। नवे॑न॒ पूर्व॒न्दय॑माने। पु॒रा॒णेन॒ नवम्। अधा॑ता॒मूर्ज॑मू॒र्जाहु॑ती॒ वसु॒ वार्या॑णि। यज॑मानाय शिक्षि॒ते। व॒सु॒वने॑ वसु॒धेय॑स्य वीतां॒ यज॑॥४६॥

%2.6.10.4
दे॒वा दैव्या॒ होता॑रा। दे॒वमिन्द्र॑मवर्धताम्। ह॒ताघ॑शसा॒वाभार्ष्टां॒ वसु॒वार्या॑णि। यज॑मानाय शिक्षि॒तौ। व॒सु॒वने॑ वसु॒धेय॑स्य वीतां॒ यज॑। दे॒वीस्ति॒स्रस्ति॒स्रो दे॒वीः। पति॒मिन्द्र॑मवर्धयन्। अस्पृ॑क्ष॒द्भार॑ती॒ दिवम्। रु॒द्रैर्य॒ज्ञ सर॑स्वती। इडा॒ वसु॑मती गृ॒हान्॥४७॥

%2.6.10.5
व॒सु॒वने॑ वसु॒धेय॑स्य वियन्तु॒ यज॑। दे॒व इन्द्रो॒ नरा॒शस॑। त्रि॒व॒रू॒थस्त्रि॑वन्धु॒रः। दे॒वमिन्द्र॑मवर्धयत्। श॒तेन॑ शितिपृ॒ष्ठाना॒माहि॑तः। स॒हस्रे॑ण॒ प्रव॑र्तते। मि॒त्रावरु॒णेद॑स्य हो॒त्रमर्‌ह॑तः। बृह॒स्पति॑ स्तो॒त्रम्। अ॒श्विनाऽऽध्व॑र्यवम्। व॒सु॒वने॑ वसु॒धेय॑स्य वेतु॒ यज॑॥४८॥

%2.6.10.6
दे॒व इन्द्रो॒ वन॒स्पति॑। हिर॑ण्यवर्णो॒ मधु॑शाखः सुपिप्प॒लः। दे॒वमिन्द्र॑मवर्धयत्। दिव॒मग्रे॑णाप्रात्। आऽन्तरि॑क्षं पृथि॒वीम॑दृहीत्। व॒सु॒वने॑ वसु॒धेय॑स्य वेतु॒ यज॑। दे॒वं ब॒र्॒हिर्वारि॑तीनाम्। दे॒वमिन्द्र॑मवर्धयत्। स्वा॒स॒स्थमिन्द्रे॒णास॑न्नम्। अ॒न्या ब॒र्॒हीष्य॒भ्य॑भूत्। व॒सु॒वने॑ वसु॒धेयस्य॑ वेतु॒ यज॑। दे॒वो अ॒ग्निः स्वि॑ष्ट॒कृत्। दे॒वमिन्द्र॑मवर्धयत्। स्वि॑ष्टं कु॒र्वन्त्स्वि॑ष्ट॒कृत्। स्वि॑ष्टम॒द्य क॑रोतु नः। व॒सु॒वने॑ वसु॒धेय॑स्य वेतु॒ यज॑॥४९॥\anuvakamend[वि॒य॒न्तु॒ यज॑ शिक्षि॒ते शि॑क्षि॒ते व॑सु॒वने॑ वसु॒धेय॑स्य वीताँ॒य्यज॑ गृ॒हान् वे॑तु॒ यजा॑भू॒थ्षट्च॑ (दे॒वं ब॒र्॒हिर्दे॒वीर्द्वारो॑ दे॒वी उ॒षासा॒नक्ता॑ दे॒वी जोष्ट्री॑ दे॒वी ऊ॒र्जाहु॑ती दे॒वा दैव्या॒ होता॑रा शिक्षि॒तौ दे॒वीस्ति॒स्रस्ति॒स्रो दे॒वीर्दे॒व इन्द्रो॒ नरा॒शसो॑ दे॒व इन्द्रो॒ वन॒स्पति॑र्दे॒वं ब॒र्॒हिर्वारि॑तीनान्दे॒वो अ॒ग्निः स्वि॑ष्ट॒कृद्दे॒वम्। वे॒तु॒ वि॒य॒न्तु॒ च॒तुर्वी॑ता॒मेको॑ वियन्तु च॒तुर्वेत्ववर्धयदवर्धय॒न्त्रिर॑वर्धता॒मेको॑ऽ वर्धयश्च॒तुर॑वर्धयत्। वस्तो॒रा व॒त्सेन॒ दैवी॒रया॒वीषह॒ताऽस्पृ॑क्षच्छ॒तेन॒ दिव स्वास॒स्थ स्वि॑ष्ट शिक्षि॒ते शि॑क्षि॒ते शि॑क्षि॒तौ ॥ )]

%2.6.11.1
होता॑ यक्षत्स॒मिधा॒ऽग्निमि॒डस्प॒दे। अ॒श्विनेन्द्र॒ सर॑स्वतीम्। अ॒जो धू॒म्रो न गो॒धूमै॒ क्व॑लैर्भेष॒जम्। मधु॒ शष्पै॒र्न तेज॑ इन्द्रि॒यम्। पय॒ सोम॑ परि॒स्रुता॑ घृ॒तं मधु॑। वि॒यन्त्वाज्य॑स्य॒ होत॒र्यज॑। होता॑ यक्ष॒त्तनू॒नपा॒त्सर॑स्वती। अवि॑र्मे॒षो न भे॑ष॒जम्। प॒था मधु॑म॒ताभ॑रन्। अ॒श्विनेन्द्रा॑य वी॒र्यम्॥५०॥

%2.6.11.2
बद॑रैरुप॒वाका॑भिर्भेष॒जन्तोक्म॑भिः। पय॒ सोम॑ परि॒स्रुता॑ घृ॒तं मधु॑। वि॒यन्त्वाज्य॑स्य॒ होत॒र्यज॑। होता॑ यक्ष॒न्नरा॒शसं॒ न न॒ग्नहुम्। पति॒ सुरा॑यै भेष॒जम्। मे॒षः सर॑स्वती भि॒षक्। रथो॒ न च॒न्द्र्य॑श्विनोर्व॒पा इन्द्र॑स्य वी॒र्यम्। बद॑रैरुप॒वाका॑भिर्भेष॒जन्तोक्म॑भिः। पय॒ सोम॑ परि॒स्रुता॑ घृ॒तं मधु॑। वि॒यन्त्वाज्य॑स्य॒ होत॒र्यज॑॥५१॥

%2.6.11.3
होता॑ यक्षदि॒डेडि॒त आ॒जुह्वा॑न॒ सर॑स्वतीम्। इन्द्रं॒ बले॑न व॒र्धय\sn{}। ऋ॒ष॒भेण॒ गवेन्द्रि॒यम्। अ॒श्विनेन्द्रा॑य वी॒र्यम्। यवै क॒र्कन्धु॑भिः। मधु॑ ला॒जैर्न मास॑रम्। पय॒ सोम॑ परि॒स्रुता॑ घृ॒तं मधु॑। वि॒यन्त्वाज्य॑स्य॒ होत॒र्यज॑। होता॑ यक्षद्ब॒र्॒हिः सु॒ष्टरी॒मोर्ण॑म्रदाः। भि॒षङ्नास॑त्या॥५२॥

%2.6.11.4
भि॒षजा॒ऽश्विनाऽश्वा॒ शिशु॑मती। भि॒षग्धे॒नुः सर॑स्वती। भि॒षग्दु॒ह इन्द्रा॑य भेष॒जम्। पय॒ सोम॑ परि॒स्रुता॑ घृ॒तं मधु॑। वि॒यन्त्वाज्य॑स्य॒ होत॒र्यज॑। होता॑ यक्ष॒द्दुरो॒ दिश॑। क॒व॒ष्यो॑ न व्यच॑स्वतीः। अ॒श्विभ्या॒न्न दुरो॒ दिश॑। इन्द्रो॒ न रोद॑सी॒ दुघे। दु॒हे कामा॒न्त्सर॑स्वती॥५३॥

%2.6.11.5
अ॒श्विनेन्द्रा॑य भेष॒जम्। शु॒क्रन्न ज्योति॑रिन्द्रि॒यम्। पय॒ सोम॑ परि॒स्रुता॑ घृ॒तं मधु॑। वि॒यन्त्वाज्य॑स्य॒ होत॒र्यज॑। होता॑ यक्षत्सु॒पेश॑सो॒षे नक्त॒न्दिवा। अ॒श्विना॑ सञ्जाना॒ने। समं॑ जाते॒ सर॑स्वत्या। त्विषि॒मिन्द्रे॒ न भे॑ष॒जम्। श्ये॒नो न रज॑सा हृ॒दा। पय॒ सोम॑ परि॒स्रुता॑ घृ॒तं मधु॑॥५४॥

%2.6.11.6
वि॒यन्त्वाज्य॑स्य॒ होत॒र्यज॑। होता॑ यक्ष॒द्दैव्या॒ होता॑रा भि॒षजा॒ऽश्विना। इन्द्र॒न्न जागृ॑वी॒ दिवा॒ नक्त॒न्न भे॑ष॒जैः। शूष॒ सर॑स्वती भि॒षक्। सीसे॑न दुह इन्द्रि॒यम्। पय॒ सोम॑ परि॒स्रुता॑ घृ॒तं मधु॑। वि॒यन्त्वाज्य॑स्य॒ होत॒र्यज॑। होता॑ यक्षत्ति॒स्रो दे॒वीर्न भे॑ष॒जम्। त्रय॑स्त्रि॒धात॑वो॒ऽपस॑। रू॒पमिन्द्रे॑ हिर॒ण्ययम्॥५५॥

%2.6.11.7
अ॒श्विनेडा॒ न भार॑ती। वा॒चा सर॑स्वती। मह॒ इन्द्रा॑य दधुरिन्द्रि॒यम्। पय॒ सोम॑ परि॒स्रुता॑ घृ॒तं मधु॑। वि॒यन्त्वाज्य॑स्य॒ होत॒र्यज॑। होता॑ यक्ष॒त्त्वष्टा॑र॒मिन्द्र॑म॒श्विना। भि॒षज॒न्न सर॑स्वतीम्। ओजो॒ न जू॒तिरि॑न्द्रि॒यम्। वृको॒ न र॑भ॒सो भि॒षक्। यश॒ सुर॑या भेष॒जम्॥५६॥

%2.6.11.8
श्रि॒या न मास॑रम्। पय॒ सोम॑ परि॒स्रुता॑ घृ॒तं मधु॑। वि॒यन्त्वाज्य॑स्य॒ होत॒र्यज॑। होता॑ यक्ष॒द्वन॒स्पतिम्। श॒मि॒तार श॒तक्र॑तुम्। भी॒मन्न म॒न्यु राजा॑नव्व्याँ॒घ्रन्नम॑सा॒ऽश्विना॒ भामम्। सर॑स्वती भि॒षक्। इन्दा॑य दुह इन्द्रि॒यम्। पय॒ सोम॑ परि॒स्रुता॑ घृ॒तं मधु॑। वि॒यन्त्वाज्य॑स्य॒ होत॒र्यज॑॥५७॥

%2.6.11.9
होता॑ यक्षद॒ग्नि स्वाहाऽऽज्य॑स्य स्तो॒कानाम्। स्वाहा॒ मेद॑सां॒ पृथ॑क्। स्वाहा॒ छाग॑म॒श्विभ्याम्। स्वाहा॑ मे॒ष सर॑स्वत्यै। स्वाह॑र्‌ष॒भमिन्द्रा॑य सि॒हाय॒ सह॑सेन्द्रि॒यम्। स्वाहा॒ऽग्निन्न भे॑ष॒जम्। स्वाहा॒ सोम॑मिन्द्रि॒यम्। स्वाहेन्द्र सु॒त्रामा॑ण सवि॒तारं॒ वरु॑णं भि॒षजां॒ पतिम्। स्वाहा॒ वन॒स्पतिं॑ प्रि॒यं पाथो॒ न भे॑ष॒जम्। स्वाहा॑ दे॒वा आज्य॒पान्॥५८॥

%2.6.11.10
स्वाहा॒ऽग्नि हो॒त्राज्जु॑षा॒णो अ॒ग्निर्भे॑ष॒जम्। पय॒ सोम॑ परि॒स्रुता॑ घृ॒तं मधु॑। वि॒यन्त्वाज्य॑स्य॒ होत॒र्यज॑। होता॑ यक्षद॒श्विना॒ सर॑स्वती॒मिन्द्र सु॒त्रामा॑णम्। इ॒मे सोमा सु॒रामा॑णः। छागै॒र्न मे॒षैर्\mbox{}ऋ॑ष॒भैः सु॒ताः। शष्पै॒र्न तोक्म॑भिः। ला॒जैर्मह॑स्वन्तः। मदा॒ मास॑रेण॒ परि॑ष्कृताः। शु॒क्राः पय॑स्वन्तो॒ऽमृता। प्रस्थि॑ता वो मधु॒श्चुत॑। तान॒श्विना॒ सर॑स्व॒तीन्द्र॑ सु॒त्रामा॑ वृत्र॒हा। जु॒षन्ता सौ॒म्यं मधु॑। पिब॑न्तु॒ मद॑न्तु वि॒यन्तु॒ सोमम्। होत॒र्यज॑॥५९॥\anuvakamend[वी॒र्यं॑ वि॒यन्त्वाज्य॑स्य॒ होत॒र्यज॒ नास॑त्या॒ सर॑स्वती॒ मधु॑ हिर॒ण्ययं॑ भेष॒जं वि॒यन्त्वाज्य॑स्य॒ होत॒र्यजाज्य॒पान॒मृता॒ पञ्च॑ च (स॒मिधा॒ऽग्नि षट्। तनू॒नपात्स॒प्त। नरा॒शस॒मृषि॑। इ॒डेडि॒तो यवै॑र॒ष्टौ। ब॒र्‌हि॒ स॒प्त। दुरो॒ऽश्विना॒ नव॑। सु॒पेश॒सर्‌षि॑। दैव्या॒ होता॑रा॒ सीसे॑न॒ रस॑। ति॒स्रस्त्वष्टा॑रम॒ष्टाव॑ष्टौ। वन॒स्पति॒मृषि॑। अ॒ग्निन्त्रयो॑दश। अ॒श्विना॒ द्वाद॑श त्रयोदश। स॒मिधा॒ऽग्निं बद॑रै॒र्बद॑रै॒र्यवै॑र॒श्विना॒ त्विषि॑म॒श्विना॒ न भे॑ष॒ज रू॒पम॒श्विना॑ भी॒मं भामम् ॥ )]

%2.6.12.1
समि॑द्धो अ॒ग्निर॑श्विना। त॒प्तो घ॒र्मो वि॒राट्त्सु॒तः। दु॒हे धे॒नुः सर॑स्वती। सोम शु॒क्रमि॒हेन्द्रि॒यम्। त॒नू॒पा भि॒षजा॑ सु॒ते। अ॒श्विनो॒भा सर॑स्वती। मध्वा॒ रजासीन्द्रि॒यम्। इन्द्रा॑य प॒थिभि॑र्वहान्। इन्द्रा॒येन्दु॒ सर॑स्वती। नरा॒शसे॑न न॒ग्नहु॑॥६०॥

%2.6.12.2
अधा॑ताम॒श्विना॒ मधु॑। भे॒ष॒जं भि॒षजा॑ सु॒ते। आ॒जुह्वा॑ना॒ सर॑स्वती। इन्द्रा॑येन्द्रि॒याणि॑ वी॒र्यम्। इडा॑भिर॒श्विना॒विषम्। समूर्ज॒ स र॒यिन्द॑धुः। अश्वि॑ना॒ नमु॑चेः सु॒तम्। सोम शु॒क्रं प॑रि॒स्रुता। सर॑स्वती॒ तमाभ॑रत्। ब॒र्॒हिषेन्द्रा॑य॒ पात॑वे॥६१॥

%2.6.12.3
क॒व॒ष्यो॑ न व्यच॑स्वतीः। अ॒श्विभ्यां॒ न दु॒रो दिश॑। इन्द्रो॒ न रोद॑सी॒ दुघे। दु॒हे कामा॒न्त्सर॑स्वती। उ॒षासा॒ नक्त॑मश्विना। दिवेन्द्र सा॒यमि॑न्द्रि॒यैः। स॒ञ्जा॒ना॒ने सु॒पेश॑सा। समं॑ जाते॒ सर॑स्वत्या। पा॒तन्नो॑ अश्विना॒ दिवा। पा॒हि नक्त सरस्वति॥६२॥

%2.6.12.4
दैव्या॑ होतारा भिषजा। पा॒तमिन्द्र॒ सचा॑ सु॒ते। ति॒स्रस्त्रे॒धा सर॑स्वती। अ॒श्विना॒ भार॒तीडा। ती॒व्रं प॑रि॒स्रुता॒ सोमम्। इन्द्रा॑य सुषवु॒र्मदम्। अश्वि॑ना भेष॒जं मधु॑। भे॒ष॒जन्न॒ सर॑स्वती। इन्द्रे॒ त्वष्टा॒ यश॒ श्रियम्। रू॒प रू॑पमधुः सु॒ते। ऋ॒तु॒थेन्द्रो॒ वन॒स्पति॑। श॒श॒मा॒नः प॑रि॒स्रुता। की॒लाल॑म॒श्विभ्यां॒ मधु॑। दु॒हे धे॒नुः सर॑स्वती। गोभि॒र्न सोम॑मश्विना। मास॑रेण परि॒ष्कृता। सम॑धाता॒ सर॑स्वत्या। स्वाहेन्द्रे॑ सु॒तं मधु॑॥६३॥\anuvakamend[न॒ग्नहु॒ पात॑वे सरस्वत्यधुः सु॒तेऽष्टौ च॑]

%2.6.13.1
अ॒श्विना॑ ह॒विरि॑न्द्रि॒यम्। नमु॑चेर्धि॒या सर॑स्वती। आ शु॒क्रमा॑सु॒राद्व॒सु। म॒घमिन्द्रा॑य जभ्रिरे। यम॒श्विना॒ सर॑स्वती। ह॒विषेन्द्र॒मव॑र्धयन्। स बि॑भेद व॒लं म॒घम्। नमु॑चावासु॒रे सचा। तमिन्द्रं॑ प॒शव॒ सचा। अ॒श्विनो॒भा सर॑स्वती॥६४॥

%2.6.13.2
दधा॑ना अ॒भ्य॑नूषत। ह॒विषा॑ य॒ज्ञमि॑न्द्रि॒यम्। य इन्द्र॑ इन्द्रि॒यन्द॒धुः। स॒वि॒ता वरु॑णो॒ भग॑। स सु॒त्रामा॑ ह॒विष्प॑तिः। यज॑मानाय सश्चत। स॒वि॒ता वरु॑णो॒ऽदध॑त्। यज॑मानाय दा॒शुषे। आद॑त्त॒ नमु॑चे॒र्वसु॑। सु॒त्रामा॒ बल॑मिन्द्रि॒यम्॥६५॥

%2.6.13.3
वरु॑णः क्ष॒त्रमि॑न्द्रि॒यम्। भगे॑न सवि॒ता श्रियम्। सु॒त्रामा॒ यश॑सा॒ बलम्। दधा॑ना य॒ज्ञमा॑शत। अश्वि॑ना॒ गोभि॑रिन्द्रि॒यम्। अश्वे॑भिर्वी॒र्यं॑ बलम्। ह॒विषेन्द्र॒ सर॑स्वती। यज॑मानमवर्धयन्। ता नास॑त्या सु॒पेश॑सा। हिर॑ण्यवर्तनी॒ नरा। सर॑स्वती ह॒विष्म॑ती। इन्द्र॒ कर्म॑सु नोऽवत। ता भि॒षजा॑ सु॒कर्म॑णा। सा सु॒दुघा॒ सर॑स्वती। स वृ॑त्र॒हा श॒तक्र॑तुः। इन्द्रा॑य दधुरिन्द्रि॒यम्॥६६॥\anuvakamend[उ॒भा सर॑स्वती॒ बल॑मिन्द्रि॒यन्नरा॒ षट्च॑]

%2.6.14.1
दे॒वं ब॒र्॒हिः स॑रस्वती। सु॒दे॒वमिन्द्रे॑ अ॒श्विना। तेजो॒ न चक्षु॑र॒क्ष्योः। ब॒र्॒हिषा॑ दधुरिन्द्रि॒यम्। व॒सु॒वने॑ वसु॒धेय॑स्य वियन्तु॒ यज॑। दे॒वीर्द्वारो॑ अ॒श्विना। भि॒षजेन्द्रे॒ सर॑स्वती। प्रा॒णन्न वी॒र्य॑न्न॒सि। द्वारो॑ दधुरिन्द्रि॒यम्। व॒सु॒वने॑ वसु॒धेय॑स्य वियन्तु॒ यज॑॥६७॥

%2.6.14.2
दे॒वी उ॒षासा॑व॒श्विना। भि॒षजेन्द्रे॒ सर॑स्वती। बल॒न्न वाच॑मा॒स्ये। उ॒षाभ्यान्दधुरिन्द्रि॒यम्। व॒सु॒वने॑ वसु॒धेय॑स्य वियन्तु॒ यज॑। दे॒वी जोष्ट्री॑ अ॒श्विना। सु॒त्रामेन्द्रे॒ सर॑स्वती। श्रोत्र॒न्न कर्ण॑यो॒र्यश॑। जोष्ट्रीभ्यान्दधुरिन्द्रि॒यम्। व॒सु॒वने॑ वसु॒धेय॑स्य वियन्तु॒ यज॑॥६८॥

%2.6.14.3
दे॒वी ऊ॒र्जाहु॑ती॒ दुघे॑ सु॒दुघे। पय॒सेन्द्र॒ सर॑स्वत्य॒श्विना॑ भि॒षजा॑वत। शु॒क्रन्न ज्योति॒ स्तन॑यो॒राहु॑ती धत्त इन्द्रि॒यम्। व॒सु॒वने॑ वसु॒धेय॑स्य वियन्तु॒ यज॑। दे॒वा दे॒वानां भि॒षजा। होता॑रा॒विन्द्र॑म॒श्विना। व॒ष॒ट्का॒रैः सर॑स्वती। त्विषि॒न्न हृद॑ये म॒तिम्। होतृ॑भ्यान्दधुरिन्द्रि॒यम्। व॒सु॒वने॑ वसु॒धेय॑स्य वियन्तु॒ यज॑॥६९॥

%2.6.14.4
दे॒वीस्ति॒स्रस्ति॒स्रो दे॒वीः। सर॑स्वत्य॒श्विना॒ भार॒तीडा। शूष॒न्न मध्ये॒ नाभ्याम्। इन्द्रा॑य दधुरिन्द्रि॒यम्। व॒सु॒वने॑ वसु॒धेय॑स्य वियन्तु॒ यज॑। दे॒व इन्द्रो॒ नरा॒शस॑। त्रि॒व॒रू॒थः सर॑स्वत्या॒ऽश्विभ्या॑मीयते॒ रथ॑। रेतो॒ न रू॒पम॒मृतं॑ ज॒नित्रम्। इन्द्रा॑य॒ त्वष्टा॒ दध॑दिन्द्रि॒याणि॑। व॒सु॒वने॑ वसु॒धेय॑स्य वियन्तु॒ यज॑॥७०॥

%2.6.14.5
दे॒व इन्द्रो॒ वन॒स्पति॑। हिर॑ण्यपर्णो अ॒श्विभ्याम्। सर॑स्वत्याः सुपिप्प॒लः। इन्द्रा॑य पच्यते॒ मधु॑। ओजो॒ न जू॒तिमृ॑ष॒भो न भामम्। वन॒स्पति॑र्नो॒ दध॑दिन्द्रि॒याणि॑। व॒सु॒वने॑ वसु॒धेय॑स्य वियन्तु॒ यज॑। दे॒वं ब॒र्॒हिर्वारि॑तीनाम्। अ॒ध्व॒रे स्ती॒र्णम॒श्विभ्याम्। ऊर्ण॑म्रदा॒ सर॑स्वत्याः॥७१॥

%2.6.14.6
स्यो॒नमि॑न्द्र ते॒ सद॑। ई॒शायै॑ म॒न्यु राजा॑नं ब॒र्॒हिषा॑ दधुरिन्द्रि॒यम्। व॒सु॒वने॑ वसु॒धेय॑स्य वियन्तु॒ यज॑। दे॒वो अ॒ग्निः स्वि॑ष्ट॒कृत्। दे॒वान् य॑क्षद्यथाय॒थम्। होता॑रा॒विन्द्र॑म॒श्विना। वा॒चा वाच॒ सर॑स्वतीम्। अ॒ग्नि सोम स्विष्ट॒कृत्। स्वि॑ष्ट॒ इन्द्र॑ सु॒त्रामा॑ सवि॒ता वरु॑णो भि॒षक्। इ॒ष्टो दे॒वो वन॒स्पति॑। स्वि॑ष्टा दे॒वा आज्य॒पाः। इ॒ष्टो अ॒ग्निर॒ग्निना। होता॑ हो॒त्रे स्वि॑ष्ट॒कृत्। यशो॒ न दध॑दिन्द्रि॒यम्। ऊर्ज॒मप॑चिति स्व॒धाम्। व॒सु॒वने॑ वसु॒धेय॑स्य वियन्तु॒ यज॑॥७२॥\anuvakamend[द्वारो॑ दधुरिन्द्रि॒यं व॑सु॒वने॑ वसु॒धेय॑स्य वियन्तु॒ यज॒ जोष्ट्रीभ्यान्दधुरिन्द्रि॒यं व॑सु॒वने॑ वसु॒धेय॑स्य वियन्तु॒ यज॒ होतृ॑भ्यान्दधुरिन्द्रि॒यं व॑सु॒वने॑ वसु॒धेय॑स्य वियन्तु॒ यजेन्द्रि॒याणि॑ वसु॒वने॑ वसु॒धेय॑स्य वियन्तु॒ यज॒ सर॑स्वत्या॒ वन॒स्पति॒ष्षट्च॑ (दे॒वं ब॒र्॒हिर्दे॒वीर्द्वारो॑ दे॒वी उ॒षासा॑व॒श्विना॑ दे॒वी जोष्ट्री॑ दे॒वी ऊ॒र्जाहु॑ती दे॒वा दे॒वानां भि॒षजा॑ वषट्का॒रैर्दे॒वीस्ति॒स्रस्ति॒स्रो दे॒वीर्दे॒व इन्द्रो॒ नरा॒शसो॑ दे॒व इन्द्रो॒ वन॒स्पति॑र्दे॒वं ब॒र्॒हिर्वारि॑तीनान्दे॒वो अ॒ग्निः स्वि॑ष्ट॒कृद्दे॒वान्। स॒मिधा॒ऽग्निन्दे॒वं ब॒र्॒हिः सर॑स्वत्य॒श्विना॒ सर्व॑ वियन्तु। द्वार॑स्ति॒स्रः सर्व॑वियन्तु। अ॒ज इन्द्र॒मोजो॒ऽग्निं पर॒ सर॑स्वतीम्। नक्तं॒ पूर्व॒ सर॑स्वति। अ॒न्यत्र॒ सर॑स्वती। भि॒षक्पूर्व॑न्दुह इन्द्रि॒यम्। अ॒न्यत्र॑ दधुरिन्द्रि॒यम्। सौ॒त्रा॒म॒ण्या सु॑तासु॒ती। अ॒ञ्जन्त्य॒यं यज॑मानः ॥ )]

%2.6.15.1
अ॒ग्निम॒द्य होता॑रमवृणीत। अ॒य सु॑तासु॒ती यज॑मानः। पच॑न्प॒क्तीः। पच॑न्पुरो॒डाशान्॑। गृ॒ह्णन्ग्रहान्॑। ब॒ध्नन्न॒श्विभ्या॒ञ्छाग॒ सर॑स्वत्या॒ इन्द्रा॑य। ब॒ध्नन्त्सर॑स्वत्यै मे॒षमिन्द्रा॑या॒श्विभ्याम्। ब॒ध्नन्निन्द्रा॑यर्‌ष॒भम॒श्विभ्या॒ सर॑स्वत्यै। सू॒प॒स्था अ॒द्य दे॒वो वन॒स्पति॑रभवत्। अ॒श्विभ्या॒ञ्छागे॑न॒ सर॑स्वत्या॒ इन्द्रा॑य॥७३॥

%2.6.15.2
सर॑स्वत्यै मे॒षेणेन्द्रा॑या॒श्विभ्याम्। इन्द्रा॑यर्‌ष॒भेणा॒श्विभ्या॒ सर॑स्वत्यै। अक्ष॒ स्तान्मे॑द॒स्तः प्रति॑पच॒ताग्र॑भीषुः। अवी॑वृधन्त॒ ग्रहै। अपा॑ताम॒श्विना॒ सर॑स्व॒तीन्द्र॑ सु॒त्रामा॑ वृत्र॒हा। सोमान्त्सु॒राम्ण॑। उपो॑ उक्थाम॒दाः श्रौ॒द्विमदा॑ अदन्। अवी॑वृधन्ताङ्गू॒षैः। त्वाम॒द्यर्‌ष॑ आर्‌षेयर्‌षीणान्नपादवृणीत। अ॒य सु॑तासु॒ती यज॑मानः। ब॒हुभ्य॒ आ सङ्ग॑तेभ्यः। ए॒ष मे॑ दे॒वेषु॒ वसु॒ वार्या य॑क्ष्यत॒ इति॑। ता या दे॒वा दे॑व॒दाना॒न्यदु॑। तान्य॑स्मा॒ आ च॒ शास्व॑। आ च॑ गुरस्व। इ॒षि॒तश्च॑ होत॒रसि॑ भद्र॒वाच्या॑य॒ प्रेषि॑तो॒ मानु॑षः। सू॒क्त॒वा॒काय॑ सू॒क्ता ब्रू॑हि॥७४॥\anuvakamend[इन्द्रा॑य॒ यज॑मानः स॒प्त च॑]

%2.6.16.1
उ॒शन्त॑स्त्वा हवामह॒ आ नो॑ अग्ने सुके॒तुना। त्व सो॑म म॒हे भग॒न्त्व सो॑म॒ प्रचि॑कितो मनी॒षा। त्वया॒ हि न॑ पि॒तर॑ सोम॒ पूर्वे॒ त्व सो॑म पि॒तृभि॑ संविदा॒नः। बर्‌हि॑षदः पितर॒ आऽहं पि॒तॄन्। उप॑हूताः पि॒तरोऽग्नि॑ष्वात्ताः पितरः। अ॒ग्निष्वा॒त्तानृ॑तु॒मतो॑ हवामहे। नरा॒शसे॑ सोमपी॒थं य आ॒शुः। ते नो॒ अर्व॑न्तः सु॒हवा॑ भवन्तु। शन्नो॑ भवन्तु द्वि॒पदे॒ शञ्चतु॑ष्पदे। ये अ॑ग्निष्वा॒त्ता येऽन॑ग्निष्वात्ताः॥७५॥

%2.6.16.2
अ॒हो॒मुच॑ पि॒तर॑ सो॒म्यास॑। परेऽव॑रे॒ऽमृता॑सो॒ भव॑न्तः। अधि॑ ब्रुवन्तु॒ ते अ॑वन्त्व॒स्मान्। वा॒न्या॑यै दु॒ग्धे जु॒षमा॑णाः कर॒म्भम्। उ॒दीरा॑णा॒ अव॑रे॒ परे॑ च। अ॒ग्नि॒ष्वा॒त्ता ऋ॒तुभि॑ संविदा॒नाः। इन्द्र॑वन्तो ह॒विरि॒दं जु॑षन्ताम्। यद॑ग्ने कव्यवाहन॒ त्वम॑ग्न ईडि॒तो जा॑तवेदः। मात॑ली क॒व्यैः। ये ता॑तृ॒पुर्दे॑व॒त्रा जेह॑मानाः। हो॒त्रा॒वृध॒ स्तोम॑तष्टासो अ॒र्कैः। आऽग्ने॑ याहि सुवि॒दत्रे॑भिर॒र्वाङ्। स॒त्यैः क॒व्यैः पि॒तृभि॑र्घर्म॒सद्भि॑। ह॒व्य॒वाह॑म॒जरं॑ पुरुप्रि॒यम्। अ॒ग्निङ्घृ॒तेन॑ ह॒विषा॑ सप॒र्यन्। उपा॑सदङ्कव्य॒वाहं॑ पितृ॒णाम्। स न॑ प्र॒जां वी॒रव॑ती॒ समृ॑ण्वतु॥७६॥\anuvakamend[अन॑ग्निष्वात्ता॒ जेह॑मानाः स॒प्त च॑]

%2.6.17.1
होता॑ यक्षदि॒डस्प॒दे। स॒मि॒धा॒नं म॒हद्यश॑। सुष॑मिद्धं॒ वरेण्यम्। अ॒ग्निमिन्द्रं॑ वयो॒धसम्। गा॒य॒त्रीञ्छन्द॑ इन्द्रि॒यम्। त्र्यवि॒ङ्गां वयो॒ दध॑त्। वेत्वाज्य॑स्य॒ होत॒र्यज॑। होता॑ यक्ष॒च्छुचि॑व्रतम्। तनू॒नपा॑तमु॒द्भिदम्। यङ्गर्भ॒मदि॑तिर्द॒धे॥७७॥

%2.6.17.2
शुचि॒मिन्द्रं॑ वयो॒धसम्। उ॒ष्णिह॒ञ्छन्द॑ इन्द्रि॒यम्। दि॒त्य॒वाह॒ङ्गां वयो॒ दध॑त्। वेत्वाज्य॑स्य॒ होत॒र्यज॑। होता॑ यक्षदी॒डेन्यम्। ई॒डि॒तं वृ॑त्र॒हन्त॑मम्। इडा॑भि॒रीड्य॒ सह॑। सोम॒मिन्द्रं॑ वयो॒धसम्। अ॒नु॒ष्टुभ॒ञ्छन्द॑ इन्द्रि॒यम्। त्रि॒व॒त्सङ्गां वयो॒ दध॑त्॥७८॥

%2.6.17.3
वेत्वाज्य॑स्य॒ होत॒र्यज॑। होता॑ यक्षत्सुबर्\mbox{}हि॒षदम्। पू॒ष॒ण्वन्त॒मम॑र्त्यम्। सीद॑न्तं ब॒र्॒हिषि॑ प्रि॒ये। अ॒मृतेन्द्रं॑ वयो॒धसम्। बृ॒ह॒तीञ्छन्द॑ इन्द्रि॒यम्। पञ्चा॑वि॒ङ्गां वयो॒ दध॑त्। वेत्वाज्य॑स्य॒ होत॒र्यज॑। होता॑यक्ष॒द्व्यच॑स्वतीः। सु॒प्रा॒य॒णा ऋ॑ता॒वृध॑॥७९॥

%2.6.17.4
द्वारो॑ दे॒वीर्‌हि॑र॒ण्ययी। ब्र॒ह्माण॒ इन्द्रं॑ वयो॒धसम्। प॒ङ्क्तिञ्छन्द॑ इ॒हेन्द्रि॒यम्। तु॒र्य॒वाह॒ङ्गां वयो॒ दध॑त्। वेत्वाज्य॑स्य॒ होत॒र्यज॑। होता॑ यक्षत्सु॒पेश॑से। सु॒शि॒ल्पे बृ॑ह॒ती उ॒भे। नक्तो॒षासा॒ न द॑र्‌श॒ते। विश्व॒मिन्द्रं॑ वयो॒धसम्। त्रि॒ष्टुभ॒ञ्छन्द॑ इन्द्रि॒यम्॥८०॥

%2.6.17.5
प॒ष्ठ॒वाह॒ङ्गां वयो॒ दध॑त्। वेत्वाज्य॑स्य॒ होत॒र्यज॑। होता॑ यक्ष॒त्प्रचे॑तसा। दे॒वाना॑मुत्त॒मं यश॑। होता॑रा॒ दैव्या॑ क॒वी। स॒युजेन्द्रं॑ वयो॒धसम्। जग॑ती॒ञ्छन्द॑ इ॒हेन्द्रि॒यम्। अ॒न॒ड्वाह॒ङ्गां वयो॒ दध॑त्। वेत्वाज्य॑स्य॒ होत॒र्यज॑। होता॑ यक्ष॒त्पेश॑स्वतीः॥८१॥

%2.6.17.6
ति॒स्रो दे॒वीर्‌हि॑र॒ण्ययी। भार॑तीर्बृह॒तीर्म॒हीः। पति॒मिन्द्रं॑ वयो॒धसम्। वि॒राज॒ञ्छन्द॑ इ॒हेन्द्रि॒यम्। धे॒नुङ्गान्न वयो॒ दध॑त्। वेत्वाज्य॑स्य॒ होत॒र्यज॑। होता॑ यक्षत्सु॒रेत॑सम्। त्वष्टा॑रं पुष्टि॒वर्ध॑नम्। रू॒पाणि॒ बिभ्र॑तं॒ पृथ॑क्। पुष्टि॒मिन्द्रं॑ वयो॒धसम्॥८२॥

%2.6.17.7
द्वि॒पद॒ञ्छन्द॑ इ॒हेन्द्रि॒यम्। उ॒क्षाण॒ङ्गान्न वयो॒ दध॑त्। वेत्वाज्य॑स्य॒ होत॒र्यज॑। होता॑ यक्षच्छ॒तक्र॑तुम्। हिर॑ण्यपर्णमु॒क्थिनम्। र॒श॒नां बिभ्र॑तं व॒शिम्। भग॒मिन्द्रं॑ वयो॒धसम्। क॒कुभ॒ञ्छन्द॑ इ॒हेन्द्रि॒यम्। व॒शां वे॒हत॒ङ्गान्न वयो॒ दध॑त्। वेत्वाज्य॑स्य॒ होत॒र्यज॑। होता॑ यक्ष॒त्स्वाहा॑कृतीः। अ॒ग्निङ्गृ॒हप॑तिं॒ पृथ॑क्। वरु॑णं भेष॒जङ्क॒विम्। क्ष॒त्रमिन्द्रं॑ वयो॒धसम्। अति॑च्छन्दस॒ञ्छन्द॑ इन्द्रि॒यम्। बृ॒हदृ॑ष॒भङ्गां वयो॒ दध॑त्। वेत्वाज्य॑स्य॒ होत॒र्यज॑॥८३॥\anuvakamend[द॒धे दध॑दृता॒वृध॑ इन्द्रि॒यं पेश॑स्वतीर्वयो॒धसं॒ वेत्वाज्य॑स्य॒ होत॒र्यज॑ स॒प्त च॑ (इ॒डस्प॒देऽग्निङ्गा॑य॒त्रीन्त्र्यविम्। शुचि॑व्रत॒ शुचि॑मु॒ष्णिह॑न्दित्य॒वाहम्। ई॒डेन्य॒ सोम॑मनु॒ष्टुभ॑न्त्रिव॒त्सम्। सु॒ब॒र्॒हि॒षद॑म॒मृतेन्द्रं॑ बृह॒तीं पञ्चा॑विम्। व्यच॑स्वतीः सुप्राय॒णा द्वारो ब्र॒ह्माण॑ प॒ङ्क्तिमि॒ह तु॑र्य॒वाहम्। सु॒पेश॑से॒ विश्व॒मिन्द्र॑न्त्रि॒ष्टुभं॑ पष्ठ॒वाहम्। प्रचे॑तसा स॒युजेन्द्रं॒ जग॑तीमि॒हान॒ड्वाहम्। पेश॑स्वतीस्ति॒स्रः पतिँ॑व्वि॒राज॑मि॒ह धे॒नुन्न। सु॒रेत॑स॒न्त्वष्टा॑रं॒ पुष्टि॒मिन्द्रं॑ द्वि॒पद॑मि॒होक्षाण॒न्न। श॒तक्र॑तुं॒ भग॒मिन्द्र॑ङ्क॒कुभ॑मि॒ह व॒शान्न। स्वाहा॑कृतीः क्ष॒त्रमति॑च्छन्दसं बृ॒हदृ॑ष॒भङ्गां वयो॒ दध॑दिन्द्रि॒यमृषि॑ वसु॒ नव॑ द॒शेहेन्द्रि॒यमष्ट॑ नव दश॒ गान्न वयो॒ दध॑दि॒डस्प॒दे सर्व॑ वेतु ॥ )]

%2.6.18.1
समि॑द्धो अ॒ग्निः स॒मिधा। सुष॑मिद्धो॒ वरेण्यः। गा॒य॒त्री छन्द॑ इन्द्रि॒यम्। त्र्यवि॒र्गौर्वयो॑ दधुः। तनू॒नपा॒च्छुचि॑व्रतः। त॒नू॒पाच्च॒ सर॑स्वती। उ॒ष्णिक्छन्द॑ इन्द्रि॒यम्। दि॒त्य॒वाड्गौर्वयो॑ दधुः। इडा॑भिर॒ग्निरीड्य॑। सोमो॑ दे॒वो अम॑र्त्यः॥८४॥

%2.6.18.2
अ॒नु॒ष्टुप्छन्द॑ इन्द्रि॒यम्। त्रि॒व॒त्सो गौर्वयो॑ दधुः। सु॒ब॒र्॒हिर॒ग्निः पू॑ष॒ण्वान्। स्ती॒र्णब॑र्‌हि॒रम॑र्त्यः। बृ॒ह॒ती छन्द॑ इन्द्रि॒यम्। पञ्चा॑वि॒र्गौर्वयो॑ दधुः। दुरो॑ दे॒वीर्दिशो॑ म॒हीः। ब्र॒ह्मा दे॒वो बृह॒स्पति॑। प॒ङ्क्तिश्छन्द॑ इ॒हेन्द्रि॒यम्। तु॒र्य॒वाड्गौर्वयो॑ दधुः॥८५॥

%2.6.18.3
उ॒षे य॒ह्वी सु॒पेश॑सा। विश्वे॑ दे॒वा अम॑र्त्याः। त्रि॒ष्टुप्छन्द॑ इन्द्रि॒यम्। प॒ष्ठ॒वाद्गौर्वयो॑ दधुः। दैव्या॑ होतारा भिषजा। इन्द्रे॑ण स॒युजा॑ यु॒जा। जग॑ती॒ छन्द॑ इ॒हेन्द्रि॒यम्। अ॒न॒ड्वान्गौर्वयो॑ दधुः। ति॒स्र इडा॒ सर॑स्वती। भार॑ती म॒रुतो॒ विश॑॥८६॥

%2.6.18.4
वि॒राट्छन्द॑ इ॒हेन्द्रि॒यम्। धे॒नुर्गौर्न वयो॑ दधुः। त्वष्टा॑ तु॒रीपो॒ अद्भु॑तः। इ॒न्द्रा॒ग्नी पु॑ष्टि॒वर्ध॑ना। द्वि॒पाच्छन्द॑ इ॒हेन्द्रि॒यम्। उ॒क्षा गौर्न वयो॑ दधुः। श॒मि॒ता नो॒ वन॒स्पति॑। स॒वि॒ता प्र॑सु॒वन्भगम्। क॒कुच्छन्द॑ इ॒हेन्द्रि॒यम्। व॒शा वे॒हद्गौर्न वयो॑ दधुः। स्वाहा॑ य॒ज्ञं वरु॑णः। सु॒क्ष॒त्रो भे॑ष॒जङ्क॑रत्। अति॑च्छन्दा॒श्छन्द॑ इन्द्रि॒यम्। बृ॒हदृ॑ष॒भो गौर्वयो॑ दधुः॥८७॥\anuvakamend[अम॑र्त्यस्तुर्य॒वाड्गौर्वयो॑ दधु॒र्विशो॑ व॒शा वे॒हद्गौर्न वयो॑ दधुश्च॒त्वारि॑ च]

%2.6.19.1
व॒स॒न्तेन॒र्तुना॑ दे॒वाः। वस॑वस्त्रि॒वृता स्तु॒तम्। र॒थ॒न्त॒रेण॒ तेज॑सा। ह॒विरिन्द्रे॒ वयो॑ दधुः। ग्री॒ष्मेण॑ दे॒वा ऋ॒तुना। रु॒द्राः प॑ञ्चद॒शे स्तु॒तम्। बृ॒ह॒ता यश॑सा॒ बलम्। ह॒विरिन्द्रे॒ वयो॑ दधुः। व॒र्॒षाभि॑र्\mbox{}ऋ॒तुना॑ऽऽदि॒त्याः। स्तोमे॑ सप्तद॒शे स्तु॒तम्॥८८॥

%2.6.19.2
वै॒रू॒पेण॑ वि॒शौज॑सा। ह॒विरिन्द्रे॒ वयो॑ दधुः। शा॒र॒देन॒र्तुना॑ दे॒वाः। ए॒क॒वि॒श ऋ॒भव॑ स्तु॒तम्। वै॒रा॒जेन॑ श्रि॒या श्रियम्। ह॒विरिन्द्रे॒ वयो॑ दधुः। हे॒म॒न्तेन॒र्तुना॑ दे॒वाः। म॒रुत॑स्त्रिण॒वे स्तु॒तम्। बले॑न॒ शक्व॑री॒ सह॑। ह॒विरिन्द्रे॒ वयो॑ दधुः। शै॒शि॒रेण॒र्तुना॑ दे॒वाः। त्र॒य॒स्त्रि॒शे॑ऽमृत स्तु॒तम्। स॒त्येन॑ रे॒वती क्ष॒त्रम्। ह॒विरिन्द्रे॒ वयो॑ दधुः॥८९॥\anuvakamend[स्तोमे॑ सप्तद॒शे स्तु॒त सहो॑ ह॒विरिन्द्रे॒ वयो॑ दधुश्च॒त्वारि॑ च (व॒स॒न्तेन॑ ग्री॒ष्मेण॑ व॒र्‌षाभि॑ शार॒देन॑ हेम॒न्तेन॑ शैशि॒रेण॒ षट् ॥ )]

%2.6.20.1
दे॒वं ब॒र्॒हिरिन्द्रं॑ वयो॒धसम्। दे॒वन्दे॒वम॑वर्धयत्। गा॒य॒त्रि॒या छन्द॑सेन्द्रि॒यम्। तेज॒ इन्द्रे॒ वयो॒ दध॑त्। व॒सु॒वने॑ वसु॒धेय॑स्य वेतु॒ यज॑। दे॒वीर्द्वारो॑ दे॒वमिन्द्रं॑ वयो॒धसम्। दे॒वीर्दे॒वम॑वर्धयन्। उ॒ष्णिहा॒ छन्द॑सेन्द्रि॒यम्। प्रा॒णमिन्द्रे॒ वयो॒ दध॑त्। व॒सु॒वने॑ वसु॒धेय॑स्य वियन्तु॒ यज॑॥९०॥

%2.6.20.2
दे॒वी दे॒वं व॑यो॒धसम्। उ॒षे इन्द्र॑मवर्धताम्। अ॒नु॒ष्टुभा॒ छन्द॑सेन्द्रि॒यम्। वाच॒मिन्द्रे॒ वयो॒ दध॑त्। व॒सु॒वने॑ वसु॒धेय॑स्य वीतां॒ यज॑। दे॒वी जोष्ट्री॑ दे॒वमिन्द्रं॑ वयो॒धसम्। दे॒वी दे॒वम॑वर्धताम्। बृ॒ह॒त्या छन्द॑सेन्द्रि॒यम्। श्रोत्र॒मिन्द्रे॒ वयो॒ दध॑त्। व॒सु॒वने॑ वसु॒धेय॑स्य वीतां॒ यज॑॥९१॥

%2.6.20.3
दे॒वी ऊ॒र्जाहु॑ती दे॒वमिन्द्रं॑ वयो॒धसम्। दे॒वी दे॒वम॑वर्धताम्। प॒ङ्क्त्या छन्द॑सेन्द्रि॒यम्। शु॒क्रमिन्द्रे॒ वयो॒ दध॑त्। व॒सु॒वने॑ वसु॒धेय॑स्य वीतां॒ यज॑। दे॒वा दैव्या॒ होता॑रा दे॒वमिन्द्रं॑ वयो॒धसम्। दे॒वा दे॒वम॑वर्धताम्। त्रि॒ष्टुभा॒ छन्द॑सेन्द्रि॒यम्। त्विषि॒मिन्द्रे॒ वयो॒ दध॑त्। व॒सु॒वने॑ वसु॒धेय॑स्य वीतां॒ यज॑॥९२॥

%2.6.20.4
दे॒वीस्ति॒स्रस्ति॒स्रो दे॒वीर्व॑यो॒धसम्। पति॒मिन्द्र॑मवर्धयन्। जग॑त्या॒ छन्द॑सेन्द्रि॒यम्। बल॒मिन्द्रे॒ वयो॒ दध॑त्। व॒सु॒व॒ने॑ वसु॒धेय॑स्य वियन्तु॒ यज॑। दे॒वो नरा॒शसो॑ दे॒वमिन्द्रं॑ वयो॒धसम्। दे॒वो दे॒वम॑वर्धयत्। वि॒राजा॒ छन्द॑सेन्द्रि॒यम्। रेत॒ इन्द्रे॒ वयो॒ दध॑त्। व॒सु॒वने॑ वसु॒धेय॑स्य वेतु॒ यज॑॥९३॥

%2.6.20.5
दे॒वो वन॒स्पति॑र्दे॒वमिन्द्रं॑ वयो॒धसम्। दे॒वो दे॒वम॑वर्धयत्। द्वि॒पदा॒ छन्द॑सेन्द्रि॒यम्। भग॒मिन्द्रे॒ वयो॒ दध॑त्। व॒सु॒वने॑ वसु॒धेय॑स्य वेतु॒ यज॑। दे॒वं ब॒र्॒हिर्वारि॑तीनान्दे॒वमिन्द्रं॑ वयो॒धसम्। दे॒वन्दे॒वम॑वर्धयत्। क॒कुभा॒ छन्द॑सेन्द्रि॒यम्। यश॒ इन्द्रे॒ वयो॒ दध॑त्। व॒सु॒वने॑ वसु॒धेय॑स्य वेतु॒ यज॑। दे॒वो अ॒ग्निः स्वि॑ष्ट॒कृद्दे॒वमिन्द्रं॑ वयो॒धसम्। दे॒वो दे॒वम॑वर्धयत्। अति॑च्छन्दसा॒ छन्द॑सेन्द्रि॒यम्। क्ष॒त्रमिन्द्रे॒ वयो॒ दध॑त्। व॒सु॒वने॑ वसु॒धेय॑स्य वेतु॒ यज॑॥९४॥\anuvakamend[वि॒य॒न्तु॒ यज॑ वीतां॒ यज॑ वीतां॒ यज॑ वेतु॒ यज॑ वेतु॒ यज॒ पञ्च॑ च (दे॒वं ब॒र्॒हिर्गा॑यत्रि॒या तेज॑। दे॒वीर्द्वार॑ उ॒ष्णिहा प्रा॒णम्। दे॒वी दे॒वमु॒षे अ॑नु॒ष्टुभा॒ वाचम्। दे॒वी जोष्ट्री॑ बृह॒त्या श्रोत्रम्। दे॒वी ऊ॒र्जाहु॑ती प॒ङ्क्त्या शु॒क्रम्। दे॒वा दैव्या॒ होता॑रा त्रि॒ष्टुभा॒ त्विषिम्। दे॒वीस्ति॒स्रस्ति॒स्रो दे॒वीः पतिं॒ जग॑त्या॒ बलम्। दे॒वो नरा॒शसो॑ वि॒राजा॒ रेत॑। दे॒वो वन॒स्पति॑र्द्वि॒पदा॒ भगम्। दे॒वं ब॒र्॒हिर्वारि॑तीनाङ्क॒कुभा॒ यश॑। दे॒वो अ॒ग्निः स्वि॑ष्ट॒कृदति॑च्छन्दसा क्ष॒त्रम्। वे॒तु॒ वि॒य॒न्तु॒ च॒तुर्वी॑ता॒मेको॑ वियन्तु च॒तुर्वेत्ववर्धयदवर्धयश्च॒तुर॑वर्धता॒मेको॑ऽवर्धय श्च॒तुर॑वर्धयत् ॥ )]




\prashnaend{स्वा॒द्वीन्त्वा॒ सोम॒ सुरा॑वन्त सीसे॑न मि॒त्रो॑ऽसि॒ यद्दे॑वा॒ होता॑ यक्षत्स॒मिधेन्द्र॒ समि॑द्ध॒ इन्द्र॒ आच॑र्‌षणि॒प्रा दे॒वं ब॒र्॒हिर्‌होता॑ यक्षत्स॒मिधा॒ऽग्नि समि॑द्धो अ॒ग्निर॑श्विना॒ऽश्विना॑ ह॒विरि॑न्द्रि॒यन्दे॒वं ब॒र्॒हिः सर॑स्वत्य॒ग्निम॒द्योशन्तो॒ होता॑ यक्षदि॒डस्प॒दे समि॑द्धो अ॒ग्निः स॒मिधा॑ वस॒न्तेन॒र्तुना॑ दे॒वं ब॒र्॒हिरिन्द्रं॑ वयो॒धसं॑ विश॒तिः॥२०॥}{स्वा॒द्वीन्त्वाऽमी॑मदन्त पि॒तर॒ साम्राज्याय पू॒तं प॒वित्रे॑णो॒षासा॒नक्ता॒ बद॑रै॒रधा॑तान्दे॒व इन्द्रो॒ वन॒स्पति॑ पष्ठ॒वाह॒ङ्गान्दे॒वी दे॒वं व॑यो॒धसं॒ चतु॑र्नवतिः॥९४॥}{स्वा॒द्वीन्त्वा॑ वेतु॒ यज॑॥}{हरि॑ ओम्॥}{इति श्रीकृष्णयजुर्वेदीयतैत्तिरीयब्राह्मणे द्वितीयाष्टके षष्ठः प्रपाठकः समाप्तः॥}
\clearpage
\sect{सप्तमः प्रश्नः}
\setcounter{anuvakam}{0}
\dnsub{तैत्तिरीयब्राह्मणे द्वितीयाष्टके सप्तमः प्रपाठकः}

%2.7.1.1
त्रि॒वृत्स्तोमो॑ भवति। ब्र॒ह्म॒व॒र्च॒सं वै त्रि॒वृत्। ब्र॒ह्म॒व॒र्च॒समे॒वाव॑ रुन्धे। अ॒ग्नि॒ष्टो॒मः सोमो॑ भवति। ब्र॒ह्म॒व॒र्च॒सं वा अ॑ग्निष्टो॒मः। ब्र॒ह्म॒व॒र्च॒समे॒वाव॑ रुन्धे। र॒थ॒न्त॒र साम॑ भवति। ब्र॒ह्म॒व॒र्च॒सं वै र॑थन्त॒रम्। ब्र॒ह्म॒व॒र्च॒समे॒वाव॑ रुन्धे। प॒रि॒स्र॒जी होता॑ भवति॥१॥

%2.7.1.2
अ॒रु॒णो मि॑र्मि॒रस्त्रिशु॑क्रः। ए॒तद्वै ब्र॑ह्मवर्च॒सस्य॑ रू॒पम्। रू॒पेणै॒व ब्र॑ह्मवर्च॒समव॑ रुन्धे। बृह॒स्पति॑रकामयत दे॒वानां पुरो॒धाङ्ग॑च्छेय॒मिति॑। स ए॒तं बृ॑हस्पतिस॒वम॑पश्यत्। तमाऽह॑रत्। तेना॑यजत। ततो॒ वै स दे॒वानां पुरो॒धाम॑गच्छत्। यः पु॑रो॒धाका॑म॒ स्यात्। स बृ॑हस्पतिस॒वेन॑ यजेत॥२॥

%2.7.1.3
पु॒रो॒धामे॒व ग॑च्छति। तस्य॑ प्रातः सव॒ने स॒न्नेषु॑ नाराश॒सेषु॑। एका॑दश॒ दक्षि॑णा नीयन्ते। एका॑दश॒ माध्य॑न्दिने॒ सव॑ने स॒न्नेषु॑ नाराश॒सेषु॑। एका॑दश तृतीयसव॒ने स॒न्नेषु॑ नाराश॒सेषु॑। त्रय॑स्त्रिश॒त्संप॑द्यन्ते। त्रय॑स्त्रिश॒द्वै दे॒वता। दे॒वता॑ ए॒वाव॑रुन्धे। अश्व॑श्चतुस्त्रि॒शः। प्रा॒जा॒प॒त्यो वा अश्व॑॥३॥

%2.7.1.4
प्र॒जाप॑तिश्चतुस्त्रि॒शो दे॒वता॑नाम्। याव॑तीरे॒व दे॒वता। ता ए॒वाव॑रुन्धे। कृ॒ष्णा॒जि॒ने॑ऽभिषि॑ञ्चति। ब्रह्म॑णो॒ वा ए॒तद्रू॒पम्। यत्कृ॑ष्णाजि॒नम्। ब्र॒ह्म॒व॒र्च॒सेनै॒वैन॒ सम॑र्धयति। आज्ये॑ना॒भिषि॑ञ्चति। तेजो॒ वा आज्यम्। तेज॑ ए॒वास्मि॑न्दधाति॥४॥\anuvakamend[होता॑ भवति यजेत॒ वा अश्वो॑ दधाति]

%2.7.2.1
यदाग्ने॒यो भव॑ति। अ॒ग्निमु॑खा॒ ह्यृद्धि॑। अथ॒ यत्पौ॒ष्णः। पुष्टि॒र्वै पू॒षा। पुष्टि॒र्वैश्य॑स्य। पुष्टि॑मे॒वाव॑ रुन्धे। प्र॒स॒वाय॑ सावि॒त्रः। अथ॒ यत्त्वा॒ष्ट्रः। त्वष्टा॒ हि रू॒पाणि॑ विक॒रोति॑। नि॒र्व॒रु॒ण॒त्वाय॑ वारु॒णः॥५॥

%2.7.2.2
अथो॒ य ए॒व कश्च॒ सन्त्सू॒यते। स हि वा॑रु॒णः। अथ॒ यद्वैश्वदे॒वः। वै॒श्व॒दे॒वो हि वैश्य॑। अथ॒ यन्मा॑रु॒तः। मा॒रु॒तो हि वैश्य॑। स॒प्तैतानि॑ ह॒वीषि॑ भवन्ति। स॒प्तग॑णा॒ वै म॒रुत॑। पृश्ञि॑ पष्ठौ॒ही मा॑रु॒त्या ल॑भ्यते। विड्वै म॒रुत॑। विश॑ ए॒वैतन्म॑ध्य॒तो॑ऽभिषि॑च्यते। तस्मा॒द्वा ए॒ष वि॒शः प्रि॒यः। वि॒शो हि म॑ध्य॒तो॑ऽभिषि॒च्यते। ऋ॒ष॒भ॒च॒र्मेऽध्य॒भिषि॑ञ्चति। स हि प्र॑जनयि॒ता। द॒ध्नाऽभिषि॑ञ्चति। ऊर्ग्वा अ॒न्नाद्य॒न्दधि॑। ऊ॒र्जैवैन॑म॒न्नाद्ये॑न॒ सम॑र्धयति॥६॥\anuvakamend[वा॒रु॒णो विड्वै म॒रुतो॒ऽष्टौ च॑]

%2.7.3.1
यदाग्ने॒यो भव॑ति। आ॒ग्ने॒यो वै ब्राह्म॒णः। अथ॒ यत्सौ॒म्यः। सौ॒म्यो हि ब्राह्म॒णः। प्र॒स॒वायै॒व सा॑वि॒त्रः। अथ॒ यद्बा॑र्\mbox{}हस्प॒त्यः। ए॒तद्वै ब्राह्म॒णस्य॑ वाक्प॒तीयम्। अथ॒ यद॑ग्नीषो॒मीय॑। आ॒ग्ने॒यो वै ब्राह्म॒णः। तौ य॒दा स॒ङ्गच्छे॑ते ॥७॥

%2.7.3.2
अथ॑ वी॒र्या॑वत्तरो भवति। अथ॒ यत्सा॑रस्व॒तः। ए॒तद्धि प्र॒त्यक्षं॑ ब्राह्म॒णस्य॑ वाक्प॒तीयम्। नि॒र्व॒रु॒ण॒त्वायै॒व वा॑रु॒णः। अथो॒ य ए॒व कश्च॒ सन्त्सू॒यते। स हि वा॑रु॒णः। अथ॒ यद्द्या॑वापृथि॒व्य॑। इन्द्रो॑ वृ॒त्राय॒ वज्र॒मुद॑यच्छत्। तन्द्यावा॑पृथि॒वी नान्व॑मन्येताम्। तमे॒तेनै॒व भा॑ग॒धेये॒नान्व॑मन्येताम्॥८॥

%2.7.3.3
वज्र॑स्य॒ वा ए॒षो॑ऽनुमा॒नाय॑। अनु॑मतवज्रः सूयाता॒ इति॑। अ॒ष्टावे॒तानि॑ ह॒वीषि॑ भवन्ति। अ॒ष्टाक्ष॑रा गाय॒त्री। गा॒य॒त्री ब्र॑ह्मवर्च॒सम्। गा॒य॒त्रि॒यैव ब॑ह्मवर्च॒समव॑ रुन्धे। हिर॑ण्येन घृ॒तमुत्पु॑नाति। तेज॑स ए॒व रु॒चे। कृ॒ष्णा॒जि॒ने॑ऽभिषि॑ञ्चति। ब्रह्म॑णो॒ वा ए॒तदृ॑ख्सा॒मयो॑ रू॒पम्। यत्कृ॑ष्णाजि॒नम्। ब्रह्म॑न्ने॒वैन॑मृख्सा॒मयो॒रध्य॒भिषि॑ञ्चति। घृ॒तेना॒भिषि॑ञ्चति। तथा॑ वी॒र्या॑वत्तरो भवति॥९॥\anuvakamend[स॒ङ्गच्छे॑ते भाग॒धेये॒नान्व॑मन्येता रू॒पञ्च॒त्वारि॑ च]

%2.7.4.1
न वै सोमे॑न॒ सोम॑स्य स॒वोऽस्ति। ह॒तो ह्ये॑षः। अ॒भिषु॑तो॒ ह्ये॑षः। न हि ह॒तः सू॒यते। सौ॒मी सू॒तव॑शा॒मा ल॑भते। सोमो॒ वै रे॑तो॒धाः। रेत॑ ए॒व तद्द॑धाति। सौ॒म्यर्चाऽभिषि॑ञ्चति। रे॒तो॒धा ह्ये॑षा। रेत॒ सोम॑। रेत॑ ए॒वास्मि॑न्दधाति। यत्किं च॑ राज॒सूय॑मृ॒ते सोमम्। तत्सर्वं॑ भवति। अषा॑ढय्युँ॒त्सु पृत॑नासु॒ पप्रिम्। सु॒व॒र्॒षाम॒प्स्वां वृ॒जन॑स्य गो॒पाम्। भ॒रे॒षु॒जा सु॑क्षि॒ति सु॒श्रव॑सम्। जय॑न्त॒न्त्वामनु॑ मदेम सोम॥१०॥\anuvakamend[रेत॒ सोम॑ स॒प्त च॑]

%2.7.5.1
यो वै सोमे॑न सू॒यते। स दे॑वस॒वः। यः प॒शुना॑ सू॒यते। स दे॑वस॒वः। य इष्ट्या॑ सू॒यते। स म॑नुष्यस॒वः। ए॒तं वै पृथ॑ये दे॒वाः प्राय॑च्छन्। ततो॒ वै सोऽप्या॑र॒ण्यानां पशू॒नाम॑सूयत। याव॑ती॒ किय॑तीश्च प्र॒जा वाचं॒ वद॑न्ति। तासा॒ सर्वा॑सा सूयते॥११॥

%2.7.5.2
य ए॒तेन॒ यज॑ते। य उ॑ चैनमे॒वं वेद॑। ना॒रा॒श॒स्यर्चाऽभिषि॑ञ्चति। म॒नु॒ष्या॑ वै नरा॒शस॑। नि॒ह्नुत्य॒ वावैतत्। अथा॒भिषि॑ञ्चति। यत्किं च॑ राज॒सूय॑मनुत्तरवे॒दीकम्। तत्सर्वं॑ भवति। ये मे॑ पञ्चा॒शत॑न्द॒दुः। अश्वा॑ना स॒धस्तु॑तिः। द्यु॒मद॑ग्ने॒ महि॒ श्रव॑। बृ॒हत्कृ॑धि म॒घोनाम्। नृ॒वद॑मृत नृ॒णाम्॥१२॥\anuvakamend[सू॒य॒ते॒ स॒धस्तु॑ति॒स्त्रीणि॑ च]

%2.7.6.1
ए॒ष गो॑स॒वः। ष॒ट्त्रि॒श उ॒क्थ्यो॑ बृ॒हत्सा॑मा। पव॑माने कण्वरथन्त॒रं भ॑वति। यो वै वा॑ज॒पेय॑। स स॑म्राट्त्स॒वः। यो रा॑ज॒सूय॑। स व॑रुणस॒वः। प्र॒जाप॑ति॒ स्वाराज्यं परमे॒ष्ठी। स्वाराज्य॒ङ्गौरे॒व। गौरि॑व भवति॥१३॥

%2.7.6.2
य ए॒तेन॒ यज॑ते। य उ॑ चैनमे॒वं वेद॑। उ॒भे बृ॑हद्रथन्त॒रे भ॑वतः। तद्धि स्वाराज्यम्। अ॒युत॒न्दक्षि॑णाः। तद्धि स्वाराज्यम्। प्र॒ति॒धुषा॒ऽभिषि॑ञ्चति। तद्धि स्वाराज्यम्। अनु॑द्धते॒ वेद्यै॑ दक्षिण॒त आ॑हव॒नीय॑स्य बृह॒तः स्तो॒त्रं प्रत्य॒भिषि॑ञ्चति। इ॒यं वाव र॑थन्त॒रम्॥१४॥

%2.7.6.3
अ॒सौ बृ॒हत्। अ॒नयो॑रे॒वैन॒मन॑न्तर्\mbox{}हितम॒भिषि॑ञ्चति। प॒शु॒स्तो॒मो वा ए॒षः। तेन॑ गोस॒वः। ष॒ट्त्रि॒शः सर्व॑। रे॒वज्जा॒तः सह॑सा वृ॒द्धः। क्ष॒त्राणां क्षत्र॒भृत्त॑मो वयो॒धाः। म॒हान्म॑हि॒त्वे त॑स्तभा॒नः। क्ष॒त्रे रा॒ष्ट्रे च॑ जागृहि। प्र॒जाप॑तेस्त्वा परमे॒ष्ठिन॒ स्वाराज्येना॒भिषि॑ञ्चा॒मीत्या॑ह। स्वाराज्यमे॒वैन॑ङ्गमयति॥१५॥\anuvakamend[इ॒व॒ भ॒व॒ति॒ र॒थ॒न्त॒रमा॒हैकं च]

%2.7.7.1
सि॒हे व्या॒घ्र उ॒त या पृदा॑कौ। त्विषि॑र॒ग्नौ ब्राह्म॒णे सूर्ये॒ या। इन्द्रं॒ या दे॒वी सु॒भगा॑ ज॒जान॑। सा न॒ आग॒न्वर्च॑सा संविदा॒ना। या रा॑ज॒न्ये॑ दुन्दु॒भावाय॑तायाम्। अश्व॑स्य॒ क्रन्द्ये॒ पुरु॑षस्य मा॒यौ। इन्द्रं॒ या दे॒वी सु॒भगा॑ ज॒जान॑। सा न॒ आग॒न्वर्च॑सा संविदा॒ना। या ह॒स्तिनि॑ द्वी॒पिनि॒ या हिर॑ण्ये। त्विषि॒रश्वे॑षु॒ पुरु॑षेषु॒ गोषु॑॥१६॥

%2.7.7.2
इन्द्रं॒ या दे॒वी सु॒भगा॑ ज॒जान॑। सा न॒ आग॒न्वर्च॑सा संविदा॒ना। रथे॑ अ॒क्षेषु॑ वृष॒भस्य॒ वाजे। वाते॑ प॒र्जन्ये॒ वरु॑णस्य॒ शुष्मे। इन्द्रं॒ या दे॒वी सु॒भगा॑ ज॒जान॑। सा न॒ आग॒न्वर्च॑सा संविदा॒ना। राड॑सि वि॒राड॑सि। स॒म्राड॑सि स्व॒राड॑सि। इन्द्रा॑य त्वा॒ तेज॑स्वते॒ तेज॑स्वन्त श्रीणामि। इन्द्रा॑य॒ त्वौज॑स्वत॒ ओज॑स्वन्त श्रीणामि॥१७॥

%2.7.7.3
इन्द्रा॑य त्वा॒ पय॑स्वते॒ पय॑स्वन्त श्रीणामि। इन्द्रा॑य॒ त्वाऽऽयु॑ष्मत॒ आयु॑ष्मन्त श्रीणामि। तेजो॑ऽसि। तत्ते॒ प्र य॑च्छामि। तेज॑स्वदस्तु मे॒ मुखम्। तेज॑स्व॒च्छिरो॑ अस्तु मे। तेज॑स्वान् वि॒श्वत॑ प्र॒त्यङ्ङ्। तेज॑सा॒ संपि॑पृग्धि मा। ओजो॑ऽसि। तत्ते॒ प्र य॑च्छामि॥१८॥

%2.7.7.4
ओज॑स्वदस्तु मे॒ मुखम्। ओज॑स्व॒च्छिरो॑ अस्तु मे। ओज॑स्वान् वि॒श्वत॑ प्र॒त्यङ्ङ्। ओज॑सा॒ सं पि॑पृग्धि मा। पयो॑ऽसि। तत्ते॒ प्र य॑च्छामि। पय॑स्वदस्तु मे॒ मुखम्। पय॑स्व॒च्छिरो॑ अस्तु मे। पय॑स्वान् वि॒श्वत॑ प्र॒त्यङ्ङ्। पय॑सा॒ सं पि॑पृग्धि मा॥१९॥

%2.7.7.5
आयु॑रसि। तत्ते॒ प्र य॑च्छामि। आयु॑ष्मदस्तु मे॒ मुखम्। आयु॑ष्म॒च्छिरो॑ अस्तु मे। आयु॑ष्मान् वि॒श्वत॑ प्र॒त्यङ्ङ्। आयु॑षा॒ सं पि॑पृग्धि मा। इ॒मम॑ग्न॒ आयु॑षे॒ वर्च॑से कृधि। प्रि॒य रेतो॑ वरुण सोम राजन्। मा॒तेवास्मा अदिते॒ शर्म॑ यच्छ। विश्वे॑ देवा॒ जर॑दष्टि॒र्यथाऽस॑त्॥२०॥

%2.7.7.6
आयु॑रसि वि॒श्वायु॑रसि। स॒र्वायु॑रसि॒ सर्व॒मायु॑रसि। यतो॒ वातो॒ मनो॑जवाः। यत॒ क्षर॑न्ति॒ सिन्ध॑वः। तासान्त्वा॒ सर्वा॑सा रु॒चा। अ॒भिषि॑ञ्चामि॒ वर्च॑सा। स॒मु॒द्र इ॑वासि ग॒ह्मना। सोम॑ इवा॒स्यदाभ्यः। अ॒ग्निरि॑व वि॒श्वत॑ प्र॒त्यङ्ङ्। सूर्य॑ इव॒ ज्योति॑षा वि॒भूः॥२१॥

%2.7.7.7
अ॒पाय्योँ द्रव॑णे॒ रस॑। तम॒हम॒स्मा आ॑मुष्याय॒णाय॑। तेज॑से ब्रह्मवर्च॒साय॑ गृह्णामि। अ॒पां य ऊ॒र्मौ रस॑। तम॒हम॒स्मा आ॑मुष्याय॒णाय॑। ओज॑से वी॒र्या॑य गृह्णामि। अ॒पाय्योँ म॑ध्य॒तो रस॑। तम॒हम॒स्मा आ॑मुष्याय॒णाय॑। पुष्ट्यै प्र॒जन॑नाय गृह्णामि। अ॒पाय्योँ य॒ज्ञियो॒ रस॑। तम॒हम॒स्मा आ॑मुष्याय॒णाय॑। आयु॑षे दीर्घायु॒त्वाय॑ गृह्णामि॥२२॥\anuvakamend[गोष्वोज॑स्वन्त श्रीणा॒म्योजो॑ऽसि॒ तत्ते॒ प्रय॑च्छामि॒ पय॑सा॒ संपि॑पृग्धि॒ माऽस॑द्वि॒भूर्य॒ज्ञियो॒ रसो॒ द्वे च॑]

%2.7.8.1
अ॒भिप्रेहि॑ वी॒रय॑स्व। उ॒ग्रश्चेत्ता॑ सपत्न॒हा। आति॑ष्ठ मित्र॒वर्ध॑नः। तुभ्यं॑ दे॒वा अधि॑ब्रवन्। अ॒ङ्कौ न्य॒ङ्काव॒भित॒ आति॑ष्ठ वृत्रह॒न्रथम्। आ॒तिष्ठ॑न्तं॒ परि॒ विश्वे॑ अभूषन्। श्रियं॒ वसा॑नश्चरति॒ स्वरो॑चाः। म॒हत्तद॒स्यासु॑रस्य॒ नाम॑। आ वि॒श्वरू॑पो अ॒मृता॑नि तस्थौ। अनु॒ त्वेन्द्रो॑ मद॒त्वनु॒ बृह॒स्पति॑॥२३॥

%2.7.8.2
अनु॒ सोमो॒ अन्व॒ग्निरा॑वीत्। अनु॑ त्वा॒ विश्वे॑ दे॒वा अ॑वन्तु। अनु॑ स॒प्त राजा॑नो॒ य उ॒ताभिषि॑क्ताः। अनु॑ त्वा मि॒त्रावरु॑णावि॒हाव॑तम्। अनु॒ द्यावा॑पृथि॒वी वि॒श्वश॑म्भू। सूर्यो॒ अहो॑भि॒रनु॑ त्वाऽवतु। च॒न्द्रमा॒ नक्ष॑त्रै॒रनु॑ त्वाऽवतु। द्यौश्च॑ त्वा पृथि॒वी च॒ प्रचे॑तसा। शु॒क्रो बृ॒हद्दक्षि॑णा त्वा पिपर्तु। अनु॑ स्व॒धा चि॑किता॒ सोमो॑ अ॒ग्निः। आऽयं पृ॑णक्तु॒ रज॑सी उ॒पस्थम्॥२४॥\anuvakamend[बृह॒स्पति॒ सोमो॑ अ॒ग्निरेकं च]

%2.7.9.1
प्र॒जाप॑तिः प्र॒जा अ॑सृजत। ता अ॑स्मात्सृ॒ष्टाः परा॑चीरायन्। स ए॒तं प्र॒जाप॑तिरोद॒नम॑पश्यत्। सोऽन्नं॑ भू॒तो॑ऽतिष्ठत्। ता अ॒न्यत्रा॒न्नाद्य॒मवि॑त्वा। प्र॒जाप॑तिं प्र॒जा उ॒पाव॑र्तन्त। अन्न॑मे॒वैनं॑ भू॒तं पश्य॑न्तीः प्र॒जा उ॒पाव॑र्तन्ते। य ए॒तेन॒ यज॑ते। य उ॑ चैनमे॒वं वेद॑। सर्वा॒ण्यन्ना॑नि भवन्ति॥२५॥

%2.7.9.2
सर्वे॒ पुरु॑षाः। सर्वाण्ये॒वान्ना॒न्यव॑ रुन्धे। सर्वा॒न्पुरु॑षान्। राड॑सि वि॒राड॒सीत्या॑ह। स्वाराज्यमे॒वैन॑ङ्गमयति। यद्धिर॑ण्य॒न्ददा॑ति। तेज॒स्तेनाव॑रुन्धे। यत्ति॑सृध॒न्वम्। वी॒र्य॑न्तेन॑। यदष्ट्राम्॥२६॥

%2.7.9.3
पुष्टि॒न्तेन॑। यत्क॑म॒ण्डलुम्। आयु॒ष्टेन॑। यद्धिर॑ण्यमा ब॒ध्नाति॑। ज्योति॒र्वै हिर॑ण्यम्। ज्योति॑रे॒वास्मि॑न्दधाति। अथो॒ तेजो॒ वै हिर॑ण्यम्। तेज॑ ए॒वात्मन्ध॑त्ते। यदो॑द॒नं प्रा॒श्ञाति॑। ए॒तदे॒व सर्व॑मव॒रुध्य॑॥२७॥

%2.7.9.4
तद॑स्मिन्नेक॒धाऽधात्। रो॒हि॒ण्याङ्का॒र्य॑। यद्ब्राह्म॒ण ए॒व रो॑हि॒णी। तस्मा॑दे॒व। अथो॒ वर्ष्मै॒वैन समा॒नानां करोति। उ॒द्य॒ता सूर्ये॑ण का॒र्य॑। उ॒द्यन्तं॒ वा ए॒त सर्वा प्र॒जाः प्रति॑नन्दन्ति। दि॒दृ॒क्षेण्यो॑ दर्\mbox{}श॒नीयो॑ भवति। य ए॒वं वेद॑। ब्र॒ह्म॒वा॒दिनो॑ वदन्ति॥२८॥

%2.7.9.5
अ॒वेत्यो॑ऽवभृ॒था ३ ना ३ इति॑। यद्द॑र्भपुञ्जी॒लैः प॒वय॑ति। तत्स्वि॑दे॒वावै॑ति। तन्नावै॑ति। त्रि॒भिः प॑वयति। त्रय॑ इ॒मे लो॒काः। ए॒भिरे॒वैनं॑ लो॒कैः प॑वयति। अथो॑ अ॒पां वा ए॒तत्तेजो॒ वर्च॑। यद्द॒र्भाः। यद्द॑र्भपुञ्जी॒लैः प॒वय॑ति। अ॒पामे॒वैन॒न्तेज॑सा॒ वर्च॑सा॒ऽभिषि॑ञ्चति॥२९॥\anuvakamend[भ॒व॒न्त्यष्ट्रा॑मव॒रुध्य॑ वदन्ति द॒र्भा यद्द॑र्भपुञ्जी॒लैः प॒वय॒त्येकं च]

%2.7.10.1
प्र॒जाप॑तिरकामयत ब॒होर्भूयान्त्स्या॒मिति॑। स ए॒तं प॑ञ्चशार॒दीय॑मपश्यत्। तमाऽह॑रत्। तेना॑यजत। ततो॒ वै स ब॒होर्भूया॑नभवत्। यः का॒मये॑त ब॒होर्भूयान्त्स्या॒मिति॑। स प॑ञ्चशार॒दीये॑न यजेत। ब॒होरे॒व भूयान्भवति। म॒रु॒त्स्तो॒मो वा ए॒षः। म॒रुतो॒ हि दे॒वानां॒ भूयि॑ष्ठाः॥३०॥

%2.7.10.2
ब॒हुर्भ॑वति। य ए॒तेन॒ यज॑ते। य उ॑चैनमे॒वं वेद॑। प॒ञ्च॒शा॒र॒दीयो॑ भवति। पञ्च॒ वा ऋ॒तव॑ संवत्स॒रः। ऋ॒तुष्वे॒व सं॑वत्स॒रे प्रति॑तिष्ठति। अथो॒ पञ्चाक्षरा प॒ङ्क्तिः। पाङ्क्तो॑ य॒ज्ञः। य॒ज्ञमे॒वाव॑ रुन्धे। स॒प्त॒द॒श स्तोमा॒ नाति॑ यन्ति। स॒प्त॒द॒शः प्र॒जाप॑तिः। प्र॒जाप॑ते॒राप्त्यै॥३१॥\anuvakamend[भूयि॑ष्ठा यन्ति॒ द्वे च॑]

%2.7.11.1
अ॒गस्त्यो॑ म॒रुद्भ्य॑ उ॒क्ष्णः प्रौक्ष॑त्। तानिन्द्र॒ आद॑त्त। त ए॑नं॒ वज्र॑मु॒द्यत्या॒भ्या॑यन्त। तान॒गस्त्य॑श्चै॒वेन्द्र॑श्च कयाशु॒भीये॑नाशमयताम्। ताञ्छा॒न्तानुपाह्वयत। यत्क॑याशु॒भीयं॒ भव॑ति॒ शान्त्यै। तस्मा॑दे॒त ऐन्द्रामारु॒ता उ॒क्षाण॑ सव॒नीया॑ भवन्ति। त्रय॑ प्रथ॒मेऽह॒न्ना ल॑भ्यन्ते। ए॒वं द्वि॒तीये। ए॒वन्तृ॒तीये॥३२॥

%2.7.11.2
ए॒वञ्च॑तु॒र्थे। पञ्चोत्त॒मेऽह॒न्ना ल॑भ्यन्ते। वर्\mbox{}षि॑ष्ठमिव॒ ह्ये॑तदह॑। वर्\mbox{}षि॑ष्ठः समा॒नानां भवति। य ए॒तेन॒ यज॑ते। य उ॑चैनमे॒वं वेद॑। स्वाराज्यं॒ वा ए॒ष य॒ज्ञः। ए॒तेन॒ वा एक॒या वा॑ कान्द॒मः स्वाराज्यमगच्छत्। स्वराज्यं गच्छति। य ए॒तेन॒ यज॑ते॥३३॥

%2.7.11.3
य उ॑ चैनमे॒वं वेद॑। मा॒रु॒तो वा ए॒ष स्तोम॑। ए॒तेन॒ वै म॒रुतो॑ दे॒वानां॒ भूयि॑ष्ठा अभवन्। भूयि॑ष्ठः समा॒नानां भवति। य ए॒तेन॒ यज॑ते। य उ॑ चैनमे॒वं वेद॑। प॒ञ्च॒शा॒र॒दीयो॒ वा ए॒ष य॒ज्ञः। आ प॑ञ्च॒मात्पुरु॑षा॒दन्न॑मत्ति। य ए॒तेन॒ यज॑ते। य उ॑ चैनमे॒वं वेद॑। स॒प्त॒द॒श स्तोमा॒ नाति॑ यन्ति। स॒प्त॒द॒शः प्र॒जाप॑तिः। प्र॒जाप॑तेरे॒व नैति॑॥३४॥\anuvakamend[तृ॒तीये॑ गच्छति॒ य ए॒तेन॒ यज॑तेऽत्ति॒ य ए॒तेन॒ यज॑ते॒ य उ॑ चैनमे॒वं वेद॒ त्रीणि॑ च (अ॒गस्त्य॒ स्वाराज्यं मारु॒तः प॑ञ्चशार॒दीयो॒ वा ए॒ष य॒ज्ञः स॑प्तद॒शं प्र॒जाप॑तेरे॒व नैति॑ ॥ )]

%2.7.12.1
अ॒स्या जरा॑सो द॒मा म॒रित्रा। अ॒र्चद्धू॑मासो अ॒ग्नय॑ पाव॒काः। श्वि॒ची॒चय॑ श्वा॒त्रासो॑ भुर॒ण्यव॑। व॒न॒र्॒षदो॑ वा॒यवो॒ न सोमा। यजा॑ नो मि॒त्रावरु॑णा। यजा॑ दे॒वा ऋ॒तं बृ॒हत्। अग्ने॒ यक्षि॒ स्वन्दमम्। अश्वि॑ना॒ पिब॑त सु॒तम्। दीद्य॑ग्नी शुचिव्रता। ऋ॒तुना॑ यज्ञवाहसा॥३५॥

%2.7.12.2
द्वे विरू॑पे चरत॒ स्वर्थे। अ॒न्याऽन्या॑ व॒त्समुप॑ धापयेते। हरि॑र॒न्यस्यां॒ भव॑ति स्व॒धावान्॑। शु॒क्रो अ॒न्यस्यान्ददृशे सु॒वर्चा। पू॒र्वा॒प॒रञ्च॑रतो मा॒ययै॒तौ। शिशू॒ क्रीड॑न्तौ॒ परि॑ यातो अध्व॒रम्। विश्वान्य॒न्यो भुव॑नाऽभि॒ चष्टे। ऋ॒तून॒न्यो वि॒दध॑ज्जायते॒ पुन॑। त्रीणि॑ श॒ता त्रीष॒हस्राण्य॒ग्निम्। त्रि॒शच्च॑ दे॒वा नव॑ चाऽसपर्यन्॥३६॥

%2.7.12.3
औक्ष॑न्घृ॒तैरास्तृ॑णन्ब॒र्॒हिर॑स्मै। आदिद्धोता॑र॒न्न्य॑षादयन्त। अ॒ग्निना॒ऽग्निः समि॑ध्यते। क॒विर्गृ॒हप॑ति॒र्युवा। ह॒व्य॒वाड्जु॒ह्वास्यः। अ॒ग्निर्दे॒वानां ज॒ठरम्। पू॒तद॑क्षः क॒विक्र॑तुः। दे॒वो दे॒वेभि॒रा ग॑मत्। अ॒ग्नि॒श्रियो॑ म॒रुतो॑ वि॒श्वकृ॑ष्टयः। आ त्वे॒षमु॒ग्रमव॑ ईमहे व॒यम्॥३७॥

%2.7.12.4
ते स्वा॒निनो॑ रु॒द्रिया॑ व॒र्॒षनि॑र्णिजः। सि॒हा न हे॒षक्र॑तवः सु॒दान॑वः। यदु॑त्त॒मे म॑रुतो मध्य॒मे वा। यद्वा॑ऽव॒मे सु॑भगासो दि॒वि ष्ठ। ततो॑ नो रुद्रा उ॒त वा॒ऽन्वस्य॑। अग्ने॑ वि॒त्ताद्ध॒विषो॒ यद्यजा॑मः। ईडे॑ अ॒ग्नि स्वव॑स॒न्नमो॑भिः। इ॒ह प्र॑स॒प्तो वि च॑ यत्कृ॒तन्न॑। रथै॑रिव॒ प्रभ॑रे वाज॒यद्भि॑। प्र॒द॒क्षि॒णिन्म॒रुता॒ स्तोम॑मृद्ध्याम्॥३८॥

%2.7.12.5
श्रु॒धि श्रु॑त्कर्ण॒ वह्नि॑भिः। दे॒वैर॑ग्ने स॒याव॑भिः। आसी॑दन्तु ब॒र्॒हिषि॑। मि॒त्रो वरु॑णो अर्य॒मा। प्रा॒त॒र्यावा॑णो अध्व॒रम्। विश्वे॑षा॒मदि॑तिर्य॒ज्ञिया॑नाम्। विश्वे॑षा॒मति॑थि॒र्मानु॑षाणाम्। अ॒ग्निर्दे॒वाना॒मव॑ आवृणा॒नः। सु॒मृ॒डी॒को भवतु वि॒श्ववे॑दाः। त्वे अ॑ग्ने सुम॒तिं भिक्ष॑माणाः॥३९॥

%2.7.12.6
दि॒वि श्रवो॑ दधिरे य॒ज्ञिया॑सः। नक्ता॑ च च॒क्रुरु॒षसा॒ विरू॑पे। कृ॒ष्णं च॒ वर्ण॑मरु॒णं च॒ सन्धु॑। त्वाम॑ग्न आदि॒त्यास॑ आ॒स्यम्। त्वाञ्जि॒ह्वा शुच॑यश्चक्रिरे कवे। त्वा रा॑ति॒षाचो॑ अध्व॒रेषु॑ सश्चिरे। त्वे दे॒वा ह॒विर॑द॒न्त्याहु॑तम्। नि त्वा॑ य॒ज्ञस्य॒ साध॑नम्। अग्ने॒ होता॑रमृ॒त्विजम्। व॒नु॒ष्वद्दे॑व धीमहि॒ प्रचे॑तसम्। जी॒रन्दू॒तमम॑र्त्यम्॥४०॥\anuvakamend[य॒ज्ञ॒वा॒ह॒सा॒स॒प॒र्य॒न्व॒यमृ॑द्ध्यां॒ भिक्ष॑माणा॒ प्रचे॑तस॒मेकं च]

%2.7.13.1
तिष्ठा॒ हरी॒ रथ॒ आ यु॒ज्यमा॑ना या॒हि। वा॒युर्न नि॒युतो॑ नो॒ अच्छ॑। पिबा॒स्यन्धो॑ अ॒भिसृ॑ष्टो अ॒स्मे। इन्द्र॒ स्वाहा॑ ररि॒मा ते॒ मदा॑य। कस्य॒ वृषा॑ सु॒ते सचा। नि॒युत्वान्वृष॒भो र॑णत्। वृ॒त्र॒हा सोम॑पीतये। इन्द्रं॑ व॒यम्म॑हाध॒ने। इन्द्र॒मर्भे॑ हवामहे। युजं॑ वृ॒त्रेषु॑ व॒ज्रिणम्॥४१॥

%2.7.13.2
द्वि॒ता यो वृ॑त्र॒हन्त॑मः। वि॒द इन्द्र॑ श॒तक्र॑तुः। उप॑ नो॒ हरि॑भिः सु॒तम्। स सूर॒ आज॒नयं॒ ज्योति॒रिन्द्रम्। अ॒या धि॒या त॒रणि॒रद्रि॑बर्\mbox{}हाः। ऋ॒तेन॑ शु॒ष्मी नव॑मानो अ॒र्कैः। व्यु॑स्रिधो॑ अ॒स्रो अद्रि॑र्बिभेद। उ॒तत्यदा॒श्वश्वि॑यम्। यदि॑न्द्र॒ नाहु॑षी॒ष्वा। अग्रे॑ वि॒क्षु प्रतीद॑यत्॥४२॥

%2.7.13.3
भरे॒ष्विन्द्र सु॒हव हवामहे। अ॒हो॒मुच सु॒कृत॒न्दैव्यं॒ जनम्। अ॒ग्निम्मि॒त्रं वरु॑ण सा॒तये॒ भगम्। द्यावा॑पृथि॒वी म॒रुत॑ स्व॒स्तये। म॒हि क्षेत्रं॑ पु॒रुश्च॒न्द्रं वि वि॒द्वान्। आदित्सखि॑भ्यश्च॒रथ॒ समै॑रत्। इन्द्रो॒ नृभि॑रजन॒द्दीद्या॑नः सा॒कम्। सूर्य॑मु॒षस॑ङ्गा॒तुम॒ग्निम्। उ॒रुन्नो॑ लो॒कमनु॑ नेषि वि॒द्वान्। सुव॑र्व॒ज्ज्योति॒रभ॑य स्व॒स्ति॥४३॥

%2.7.13.4
ऋ॒ष्वा त॑ इन्द्र॒ स्थवि॑रस्य बा॒हू। उप॑स्थेयाम शर॒णा बृ॒हन्ता। आ नो॒ विश्वा॑भिरू॒तिभि॑ स॒जोषा। ब्रह्म॑ जुषा॒णो ह॑र्यश्व याहि। वरी॑वृज॒त्स्थवि॑रेभिः सुशिप्र। अ॒स्मे दध॒द्वृष॑ण॒ शुष्म॑मिन्द्र। इन्द्रा॑य॒ गाव॑ आ॒शिरम्। दु॒दु॒ह्रे व॒ज्रिणे॒ मधु॑। यत्सी॑मुपह्व॒रे वि॒दत्। तास्ते॑ वज्रिन्धे॒नवो॑ जोजयुर्नः॥४४॥

%2.7.13.5
गभ॑स्तयो नि॒युतो॑ वि॒श्ववा॑राः। अह॑रह॒र्भूय॒ इज्जोगु॑वानाः। पू॒र्णा इ॑न्द्र क्षु॒मतो॒ भोज॑नस्य। इ॒मान्ते॒ धियं॒ प्र भ॑रे म॒हो म॒हीम्। अ॒स्य स्तो॒त्रे धि॒षणा॒ यत्त॑ आन॒जे। तमु॑त्स॒वे च॑ प्रस॒वे च॑ सास॒हिम्। इन्द्रं॑ दे॒वास॒ शव॑सा मद॒न्ननु॑॥४५॥\anuvakamend[व॒ज्रिण॑मयत्स्व॒स्ति जो॑जयुर्नः स॒प्त च॑]

%2.7.14.1
प्र॒जाप॑तिः प॒शून॑सृजत। तेऽस्मात्सृ॒ष्टाः परां च आयन्। तान॑ग्निष्टो॒मेन॒ नाप्नोत्। तानु॒क्थ्ये॑न॒ नाप्नोत्। तान्थ्षो॑ड॒शिना॒ नाप्नोत्। तान्रात्रि॑या॒ नाप्नोत्। तान्त्स॒न्धिना॒ नाप्नोत्। सोऽग्निम॑ब्रवीत्। इ॒मान्म॑ ई॒प्सेति॑। तान॒ग्निस्त्रि॒वृता॒ स्तोमे॑न॒ नाप्नोत्॥४६॥

%2.7.14.2
स इन्द्र॑मब्रवीत्। इ॒मान्म॑ ई॒प्सेति॑। तानिन्द्र॑ पञ्चद॒शेन॒ स्तोमे॑न॒ नाप्नोत्। स विश्वान्दे॒वान॑ब्रवीत्। इ॒मान्म॑ ईप्स॒तेति॑। तान् विश्वे॑दे॒वाः स॑प्तद॒शेन॒ स्तोमे॑न॒ नाप्नु॑वन्। स विष्णु॑मब्रवीत्। इ॒मान्म॑ ई॒प्सेति॑। ताऩ् विष्णु॑रेकवि॒शेन॒ स्तोमे॑नाप्नोत्। वा॒र॒व॒न्तीये॑नावारयत॥४७॥

%2.7.14.3
इ॒दं विष्णु॒र्वि च॑क्रम॒ इति॒ व्य॑क्रमत। यस्मात्प॒शव॒ प्रप्रेव॒ भ्रशे॑रन्। स ए॒तेन॑ यजेत। यदाप्नोत्। तद॒प्तोर्याम॑स्याप्तोर्याम॒त्वम्। ए॒तेन॒ वै दे॒वा जैत्वा॑नि जि॒त्वा। यङ्काम॒मका॑मयन्त॒ तमाप्नुवन्। यङ्काम॑ङ्का॒मय॑ते। तमे॒तेनाप्नोति॥४८॥\anuvakamend[स्तोमे॑न॒ नाप्नो॑दवारयत॒ नव॑ च]

%2.7.15.1
व्या॒घ्रो॑ऽयम॒ग्नौ च॑रति॒ प्रवि॑ष्टः। ऋषी॑णां पु॒त्रो अ॑भिशस्ति॒पा अ॒यम्। न॒म॒स्का॒रेण॒ नम॑सा ते जुहोमि। मा दे॒वानां मिथु॒याक॑र्म भा॒गम्। सावी॒र्॒हि दे॑व प्रस॒वाय॑ पित्रे। व॒र्ष्माण॑मस्मै वरि॒माण॑मस्मै। अथा॒स्मभ्य सवितः स॒र्वता॑ता। दि॒वेदि॑व॒ आ सु॑वा॒ भूरि॑ प॒श्वः। भू॒तो भू॒तेषु॑ चरति॒ प्रवि॑ष्टः। स भू॒ताना॒मधि॑पतिर्बभूव॥४९॥

%2.7.15.2
तस्य॑ मृ॒त्यौ च॑रति राज॒सूयम्। स राजा॑ रा॒ज्यमनु॑ मन्यतामि॒दम्। येभि॒ शिल्पै पप्रथा॒नामदृहत्। येभि॒र्द्याम॒भ्यपिशत्प्र॒जाप॑तिः। येभि॒र्वाचं॑ वि॒श्वरू॑पा स॒मव्य॑यत्। तेने॒मम॑ग्न इ॒ह वर्च॑सा॒ सम॑ङ्ग्धि। येभि॑रादि॒त्यस्तप॑ति॒ प्र के॒तुभि॑। येभि॒ सूर्यो॑ ददृ॒शे चि॒त्रभा॑नुः। येभि॒र्वाचं॑ पुष्क॒लेभि॒रव्य॑यत्। तेने॒मम॑ग्न इ॒ह वर्च॑सा॒ सम॑ङ्ग्धि॥५०॥

%2.7.15.3
आऽयं भा॑तु॒ शव॑सा॒ पञ्च॑ कृ॒ष्टीः। इन्द्र॑ इव ज्ये॒ष्ठो भ॑वतु प्र॒जावान्॑। अ॒स्मा अ॑स्तु पुष्क॒लञ्चि॒त्रभा॑नु। आऽयं पृ॑णक्तु॒ रज॑सी उ॒पस्थम्। यत्ते॒ शिल्प॑ङ्कश्यप रोच॒नाव॑त्। इ॒न्द्रि॒याव॑त्पुष्क॒लञ्चि॒त्रभा॑नु। यस्मि॒न्त्सूर्या॒ अर्पि॑ताः स॒प्त सा॒कम्। तस्मि॒न्राजा॑न॒मधि॒ विश्र॑ये॒मम्। द्यौर॑सि पृथि॒व्य॑सि। व्या॒घ्रो वैया॒घ्रेऽधि॑ ॥५१॥

%2.7.15.4
विश्र॑यस्व॒ दिशो॑ म॒हीः। विश॑स्त्वा॒ सर्वा॑ वाञ्छन्तु। मा त्वद्रा॒ष्ट्रमधि॑ भ्रशत्। या दि॒व्या आप॒ पय॑सा सम्बभू॒वुः। या अ॒न्तरि॑क्ष उ॒त पार्थि॑वी॒र्याः। तासान्त्वा॒ सर्वा॑सा रु॒चा। अ॒भिषि॑ञ्चामि॒ वर्च॑सा। अ॒भि त्वा॒ वर्च॑साऽसिचन्दि॒व्येन॑। पय॑सा स॒ह। यथासा॑ राष्ट्र॒वर्ध॑नः॥५२॥

%2.7.15.5
तथा त्वा सवि॒ता क॑रत्। इन्द्रं॒ विश्वा॑ अवीवृधन्। स॒मु॒द्रव्य॑चस॒ङ्गिर॑। र॒थीत॑म रथी॒नाम्। वाजा॑ना॒ सत्प॑तिं॒ पतिम्। वस॑वस्त्वा पु॒रस्ता॑द॒भिषि॑ञ्चन्तु गाय॒त्रेण॒ छन्द॑सा। रु॒द्रास्त्वा॑ दक्षिण॒तो॑ऽभिषि॑ञ्चन्तु॒ त्रैष्टु॑भेन॒ छन्द॑सा। आ॒दि॒त्यास्त्वा॑ प॒श्चाद॒भिषि॑ञ्चन्तु॒ जाग॑तेन॒ छन्द॑सा। विश्वे त्वा दे॒वा उ॑त्तर॒तो॑ऽभिषि॑ञ्च॒न्त्वानु॑ष्टुभेन॒ छन्द॑सा। बृह॒स्पति॑स्त्वो॒परि॑ष्टाद॒भिषि॑ञ्चतु॒ पाङ्क्ते॑न॒ छन्द॑सा॥५३॥

%2.7.15.6
अ॒रु॒णन्त्वा॒ वृक॑मु॒ग्रङ्ख॑जङ्क॒रम्। रोच॑मानं म॒रुता॒मग्रे॑ अ॒र्चिष॑। सूर्य॑वन्तं म॒घवा॑नं विषास॒हिम्। इन्द्र॑मु॒क्थेषु॑ नाम॒हूत॑म हुवेम। प्र बा॒हवा॑ सिसृतञ्जी॒वसे॑ नः। आ नो॒ गव्यू॑तिमुक्षतङ्घृ॒तेन॑। आ नो॒ जने श्रवयतय्युँवाना। श्रु॒तं मे॑ मित्रावरुणा॒ हवे॒मा। इन्द्र॑स्य ते वीर्य॒कृत॑। बा॒हू उ॒पाव॑ हरामि॥५४॥\anuvakamend[ब॒भू॒वाव्य॑य॒त्तेने॒मम॑ग्न इ॒ह वर्च॑सा॒ सम॑ङ्ग्धि॒ वैया॒घ्रेऽधि॑ राष्ट्र॒वर्ध॑न॒ पाङ्क्ते॑न॒ छन्द॑सो॒पाव॑हरामि]

%2.7.16.1
अ॒भि प्रेहि॑ वी॒रय॑स्व। उ॒ग्रश्चेत्ता॑ सपत्न॒हा। आति॑ष्ठ वृत्र॒हन्त॑मः। तुभ्यं॑ दे॒वा अधि॑ब्रवन्। अ॒ङ्कौ न्य॒ङ्काव॒भितो॒ रथ॒य्यौँ। ध्वा॒न्तं वा॑ता॒ग्रमनु॑ स॒ञ्चर॑न्तौ। दू॒रेहे॑तिरिन्द्रि॒यावान्पत॒त्री। ते नो॒ऽग्नय॒ पप्र॑यः पारयन्तु। नम॑स्त ऋषे गद। अव्य॑थायै त्वा स्व॒धायै त्वा॥५५॥

%2.7.16.2
मा न॑ इन्द्रा॒भित॒स्त्वदृ॒ष्वारि॑ष्टासः। ए॒वा ब्र॑ह्म॒न्तवेद॑स्तु। तिष्ठा॒ रथे॒ अधि॒ यद्वज्र॑हस्तः। आ र॒श्मीन्दे॑व युवसे॒ स्वश्व॑। आ ति॑ष्ठ वृत्रहन्ना॒तिष्ठ॑न्तं॒ परि॑। अनु॒ त्वेन्द्रो॑ मद॒त्वनु॑ त्वा मि॒त्रावरु॑णौ। द्यौश्च॑ त्वा पृथि॒वी च॒ प्रचे॑तसा। शु॒क्रो बृ॒द्दक्षि॑णा त्वा पिपर्तु। अनु॑ स्व॒धा चि॑किता॒ सोमो॑ अ॒ग्निः। अनु॑ त्वाऽवतु सवि॒ता स॒वेन॑॥५६॥

%2.7.16.3
इन्द्रं॒ विश्वा॑ अवीवृधन्। स॒मु॒द्रव्य॑चस॒ङ्गिर॑। र॒थीत॑म रथी॒नाम्। वाजा॑ना॒ सत्प॑तिं॒ पतिम्। परि॑मा से॒न्या घोषा। ज्यानां वृञ्जन्तु गृ॒ध्नव॑। मे॒थि॒ष्ठाः पिन्व॑माना इ॒ह। माङ्गोप॑तिम॒भि संवि॑शन्तु। तन्मेऽनु॑मति॒रनु॑ मन्यताम्। तन्मा॒ता पृ॑थि॒वी तत्पि॒ता द्यौः॥५७॥

%2.7.16.4
तद्ग्रावा॑णः सोम॒सुतो॑ मयो॒भुव॑। तद॑श्विना शृणुत सौभगा यु॒वम्। अव॑ ते॒ हेड॒ उदु॑त्त॒मम्। ए॒ना व्या॒घ्रं प॑रिषस्वजा॒नाः। सि॒ह हि॑न्वन्ति मह॒ते सौभ॑गाय। स॒मु॒द्रन्न सु॒हव॑न्तस्थि॒वासम्। म॒र्मृ॒ज्यन्ते द्वी॒पिन॑म॒प्स्व॑न्तः। उद॒सावे॑तु॒ सूर्य॑। उदि॒दं मा॑म॒कं वच॑। उदि॑हि देव सूर्य। स॒ह व॒ग्नुना॒ मम॑। अ॒हं वा॒चो वि॒वाच॑नम्। मयि॒ वाग॑स्तु धर्ण॒सिः। यन्तु॑ न॒दयो॒ वर्\mbox{}ष॑न्तु प॒र्जन्या। सु॒पि॒प्प॒ला ओष॑धयो भवन्तु। अन्न॑वतामोद॒नव॑तामा॒मिक्ष॑वताम्। ए॒षा राजा॑ भूयसाम्॥५८॥\anuvakamend[स्व॒धायै त्वा स॒वेन॒ द्यौः सूर्य स॒प्त च॑]

%2.7.17.1
ये के॒शिन॑ प्रथ॒माः स॒त्रमास॑त। येभि॒राभृ॑तं॒ यदि॒दं वि॒रोच॑ते। तेभ्यो॑ जुहोमि बहु॒धा घृ॒तेन॑। रा॒यस्पोषे॑णे॒मं वर्च॑सा॒ स सृ॑जाथ। नर्ते ब्रह्म॑ण॒स्तप॑सो विमो॒कः। द्वि॒नाम्नी॑ दी॒क्षा व॒शिनी॒ ह्यु॑ग्रा। प्र केशा सु॒वते॑ का॒ण्डिनो॑ भवन्ति। तेषां ब्र॒ह्मेदीशे॒ वप॑नस्य॒ नान्यः। आ रो॑ह॒ प्रोष्ठं॒ विष॑हस्व॒ शत्रून्॑। अवास्राग्दी॒क्षा व॒शिनी॒ ह्यु॑ग्रा॥५९॥

%2.7.17.2
दे॒हि दक्षि॑णां॒ प्रति॑र॒स्वायु॑। अथा॑मुच्यस्व॒ वरु॑णस्य॒ पाशात्। येनाव॑पत्सवि॒ता क्षु॒रेण॑। सोम॑स्य॒ राज्ञो॒ वरु॑णस्य वि॒द्वान्। तेन॑ ब्रह्माणो वपते॒दम॒स्योर्जेमम्। र॒य्या वर्च॑सा॒ स सृ॑जाथ। मा ते॒ केशा॒ननु॑ गा॒द्वर्च॑ ए॒तत्। तथा॑ धा॒ता क॑रोतु ते। तुभ्य॒मिन्द्रो॒ बृह॒स्पति॑। स॒वि॒ता वर्च॒ आद॑धात्॥६०॥

%2.7.17.3
तेभ्यो॑ नि॒धानं॑ बहु॒धा व्यैच्छ\sn{}। अ॒न्त॒रा द्यावा॑पृथि॒वी अ॒पः सुव॑। द॒र्भ॒स्त॒म्बे वी॒र्य॑कृते नि॒धाय॑। पौस्ये॑ने॒मं वर्च॑सा॒ स सृ॑जाथ। बल॑न्ते बाहु॒वोः स॑वि॒ता द॑धातु। सोम॑स्त्वाऽनक्तु॒ पय॑सा घृ॒तेन॑। स्त्री॒षु रू॒पम॑श्विनै॒तन्नि ध॑त्तम्। पौस्ये॑ने॒मं वर्च॑सा॒ ससृ॑जाथ। यत्सी॒मन्त॒ङ्कङ्क॑तस्ते लि॒लेख॑। यद्वा क्षु॒रः प॑रिव॒वर्ज॒ वपस्ते। स्त्री॒षु रू॒पम॑श्विनै॒तन्नि ध॑त्तम्। पौस्ये॑ने॒म स सृ॑जाथो वी॒र्ये॑ण॥६१॥\anuvakamend[अवास्राग्दी॒क्षा व॒शिनी॒ ह्यु॑ग्राऽद॑धाद्व॒वर्ज॒ वप स्ते॒ द्वे च॑]

%2.7.18.1
इन्द्रं॒ वै स्वाविशो॑ म॒रुतो॒ नापा॑चायन्। सोऽन॑पचाय्यमान ए॒तं वि॑घ॒नम॑पश्यत्। तमाऽह॑रत्। तेना॑यजत। तेनै॒वासा॒न्त स स्त॒म्भव्व्यँ॑हन्। यद्व्यह\sn{}। तद्वि॑घ॒नस्य॑ विघन॒त्वम्। वि पा॒प्मानं॒ भ्रातृ॑व्य हते। य ए॒तेन॒ यज॑ते। य उ॑ चैनमे॒वं वेद॑॥६२॥

%2.7.18.2
य राजा॑नं॒ विशो॒ नाप॒चाये॑युः। यो वा ब्राह्म॒णस्तम॑सा पा॒प्मना॒ प्रावृ॑त॒ स्यात्। स ए॒तेन॑ यजेत। वि॒घ॒नेनै॒वैन॑द्वि॒हत्य॑। वि॒शामाधि॑पत्यं गच्छति। तस्य॒ द्वे द्वा॑द॒शे स्तो॒त्रे भव॑तः। द्वे च॑तुर्वि॒शे। औद्भि॑द्यमे॒व तत्। ए॒तद्वै क्ष॒त्रस्यौद्भि॑द्यम्। यद॑स्मै॒ स्वाविशो॑ ब॒लि हर॑न्ति॥६३॥

%2.7.18.3
हर॑न्त्यस्मै॒ विशो॑ ब॒लिम्। ऐन॒मप्र॑तिख्यातं गच्छति। य ए॒वं वेद॑। प्र॒बाहु॒ग्वा अग्रे क्ष॒त्राण्याते॑पुः। तेषा॒मिन्द्र॑ क्ष॒त्राण्याद॑त्त। न वा इ॒मानि॑ क्ष॒त्राण्य॑भूव॒न्निति॑। तन्नक्ष॑त्राणां नक्षत्र॒त्वम्। आ श्रेय॑सो॒ भ्रातृ॑व्यस्य॒ तेज॑ इन्द्रि॒यन्द॑त्ते। य ए॒तेन॒ यज॑ते। य उ॑ चैनमे॒वं वेद॑॥६४॥

%2.7.18.4
तद्यथा॑ ह॒ वै स॑च॒क्रिणौ॒ कप्ल॑कावु॒पाव॑हितौ॒ स्याताम्। ए॒वमे॒तौ यु॒ग्मन्तौ॒ स्तोमौ। अ॒युक्षु॒ स्तोमे॑षु क्रियेते। पा॒प्मनोऽप॑हत्यै। अप॑ पा॒प्मानं॒ भ्रातृ॑व्य हते। य ए॒तेन॒ यज॑ते। य उ॑ चैनमे॒वं वेद॑। तद्यथा॑ ह॒ वै सू॑तग्राम॒ण्य॑। ए॒वञ्छन्दासि। तेष्व॒सावा॑दि॒त्यो बृ॑ह॒तीर॒भ्यू॑ढः॥६५॥

%2.7.18.5
स॒तोबृ॑हतीषु स्तुवते स॒तो बृ॑हन्। प्र॒जया॑ प॒शुभि॑रसा॒नीत्ये॒व। व्यति॑षक्ताभिः स्तुवते। व्यति॑षक्तं॒ वै क्ष॒त्रं वि॒शा। वि॒शैवैनं॑ क्ष॒त्रेण॒ व्यति॑षजति। व्यति॑षक्ताभिः स्तुवते। व्यति॑षक्तो॒ वै ग्रा॑म॒णीः स॑जा॒तैः। स॒जा॒तैरे॒वैन॒व्व्यँति॑षजति। व्यति॑षक्ताभिः स्तुवते। व्यति॑षक्तो॒ वै पुरु॑षः पा॒प्मभि॑। व्यति॑षक्ताभिरे॒वास्य॑ पा॒प्मनो॑ नुदते॥६६॥\anuvakamend[वेद॒ हर॑न्त्येनमे॒वं वेदा॒भ्यू॑ढः पा॒प्मभि॒रेकं च]




\prashnaend{त्रि॒वृद्यदाग्ने॒योऽग्निमु॑खा॒ ह्यृद्धि॒र्यदाग्ने॒य आग्ने॒यो न वै सोमे॑न॒ यो वै सोमे॑नै॒ष गो॑स॒वः सि॒हे॑ऽभि प्रेहि॑ मित्र॒वर्ध॑नः प्र॒जाप॑ति॒स्ता ओ॑द॒नं प्र॒जाप॑तिरकामयत ब॒होर्भूया॑न॒गस्त्यो॒स्या जरा॑स॒स्तिष्ठा॒ हरी प्र॒जाप॑तिः प॒शून्व्या॒घ्रो॑ऽयम॒भिप्रेहि॑ वृत्र॒हन्त॑मो॒ ये के॒शिन॒ इन्द्रं॒ वा अ॒ष्टाद॑श॥१८॥}{त्रि॒वृद्यो वै सोमे॒नायु॑रसि ब॒हुर्भ॑वति॒ तिष्ठा॒ हरी॒रथ॒ आयं भा॑तु॒ तेभ्यो॑ नि॒धान॒ षट्थ्ष॑ष्टिः॥६६॥}{त्रि॒वृत्पा॒प्मनो॑ नुदते॥}{हरि॑ ओम्॥}{इति श्रीकृष्णयजुर्वेदीयतैत्तिरीयब्राह्मणे द्वितीयाष्टके सप्तमः प्रपाठकः समाप्तः॥}
\clearpage
\sect{अष्टमः प्रश्नः}
\setcounter{anuvakam}{0}
\dnsub{तैत्तिरीयब्राह्मणे द्वितीयाष्टके अष्टमः प्रपाठकः}

%2.8.1.1
पीवोन्ना रयि॒वृध॑ सुमे॒धाः। श्वे॒तः सि॑षक्ति नि॒युता॑मभि॒श्रीः। ते वा॒यवे॒ सम॑नसो॒ वित॑स्थुः। विश्वेन्नर॑ स्वप॒त्यानि॑ चक्रुः। रा॒येऽनु यञ्ज॒ज्ञतू॒ रोद॑सी उ॒भे। रा॒ये दे॒वी धि॒षणा॑ धाति दे॒वम्। अधा॑ वा॒युन्नि॒युत॑ सश्चत॒ स्वाः। उ॒त श्वे॒तं वसु॑धितिन्निरे॒के। आ वा॑यो॒ प्र याभि॑। प्र वा॒युमच्छा॑ बृह॒ती म॑नी॒षा॥१॥

%2.8.1.2
बृ॒हद्र॑यिं वि॒श्ववा॑रा रथ॒प्राम्। द्यु॒तद्या॑मा नि॒युत॒ पत्य॑मानः। क॒विः क॒विमि॑यक्षसि प्रयज्यो। आ नो॑ नि॒युद्भि॑ श॒तिनी॑भिरध्व॒रम्। स॒ह॒स्रिणी॑भि॒रुप॑ याहि य॒ज्ञम्। वायो॑ अ॒स्मिन् ह॒विषि॑ मादयस्व। यू॒यं पा॑त स्व॒स्तिभि॒ सदा॑ नः। प्रजा॑पते॒ न त्वदे॒तान्य॒न्यः। विश्वा॑ जा॒तानि॒ परि॒ ता ब॑भूव। यत्का॑मास्ते जुहु॒मस्तन्नो॑ अस्तु॥२॥

%2.8.1.3
व॒य स्या॑म॒ पत॑यो रयी॒णाम्। र॒यी॒णां पतिं॑ यज॒तं बृ॒हन्तम्। अ॒स्मिन्भरे॒ नृत॑मं॒ वाज॑सातौ। प्र॒जाप॑तिं प्रथम॒जामृ॒तस्य॑। यजा॑म दे॒वमधि॑ नो ब्रवीतु। प्रजा॑पते॒ त्वन्नि॑धि॒पाः पु॑रा॒णः। दे॒वानां पि॒ता ज॑नि॒ता प्र॒जानाम्। पति॒र्विश्व॑स्य॒ जग॑तः पर॒स्पाः। ह॒विर्नो॑ देव विह॒वे जु॑षस्व। तवे॒मे लो॒काः प्र॒दिशो॒ दिश॑श्च॥३॥

%2.8.1.4
प॒रा॒वतो॑ नि॒वत॑ उ॒द्वत॑श्च। प्रजा॑पते विश्व॒सृज्जी॒वध॑न्य इ॒दन्नो॑ देव। प्रति॑हर्य ह॒व्यम्। प्र॒जाप॑तिं प्रथ॒मं य॒ज्ञिया॑नाम्। दे॒वाना॒मग्रे॑ यज॒तं य॑जध्वम्। स नो॑ ददातु॒ द्रवि॑ण सु॒वीर्यम्। रा॒यस्पोषं॒ वि ष्य॑तु॒ नाभि॑म॒स्मे। यो रा॒य ईशे॑ शतदा॒य उ॒क्थ्य॑। यः प॑शू॒ना र॑क्षि॒ता विष्ठि॑तानाम्। प्र॒जाप॑तिः प्रथम॒जा ऋ॒तस्य॑॥४॥

%2.8.1.5
स॒हस्र॑धामा जुषता ह॒विर्न॑। सोमा॑पूषणे॒मौ दे॒वौ। सोमा॑पूषणा॒ रज॑सो वि॒मानम्। स॒प्तच॑क्र॒ रथ॒मवि॑श्वमिन्वम्। वि॒षू॒वृतं॒ मन॑सा यु॒ज्यमा॑नम्। तञ्जि॑न्वथो वृषणा॒ पञ्च॑रश्मिम्। दि॒व्य॑न्यः सद॑नञ्च॒क्र उ॒च्चा। पृ॒थि॒व्याम॒न्यो अध्य॒न्तरि॑क्षे। ताव॒स्मभ्यं॑ पुरु॒वारं॑ पुरु॒क्षुम्। रा॒यस्पोषं॒ विष्य॑ता॒न्नाभि॑म॒स्मे॥५॥

%2.8.1.6
धियं॑ पू॒षा जि॑न्वतु विश्वमि॒न्वः। र॒यि सोमो॑ रयि॒पति॑र्दधातु। अव॑तु दे॒व्यदि॑तिरन॒र्वा। बृ॒हद्व॑देम वि॒दथे॑ सु॒वीरा। विश्वान्य॒न्यो भुव॑ना ज॒जान॑। विश्व॑म॒न्यो अ॑भि॒चक्षा॑ण एति। सोमा॑पूषणा॒वव॑त॒न्धियं॑ मे। यु॒वभ्यां॒ विश्वा॒ पृत॑ना जयेम। उदु॑त्त॒मं व॑रु॒णास्त॑भ्ना॒द्द्याम्। यत्किञ्चे॒दङ्कि॑त॒वास॑। अव॑ ते॒ हेड॒स्तत्त्वा॑ यामि। आ॒दि॒त्याना॒मव॑सा॒ न द॑क्षि॒णा। धा॒रय॑न्त आदि॒त्यास॑स्ति॒स्रो भूमीर्धारयन्। य॒ज्ञो दे॒वाना॒ शुचि॑र॒पः॥६॥\anuvakamend[म॒नी॒षाऽस्तु॑ च॒र्तस्या॒स्मे कि॑त॒वास॑श्च॒त्वारि॑ च]

%2.8.2.1
ते शु॒क्रास॒ शुच॑यो रश्मि॒वन्त॑। सीद॑न्नादि॒त्या अधि॑ ब॒र्॒हिषि॑ प्रि॒ये। कामे॑न दे॒वाः स॒रथ॑न्दि॒वो न॑। आ यान्तु य॒ज्ञमुप॑ नो जुषा॒णाः। ते सू॒नवो॒ अदि॑तेः पीव॒सामिषम्। घृ॒तं पिन्व॒त्प्रति॑हर्यन्नृते॒जाः। प्र य॒ज्ञिया॒ यज॑मानाय येमुरे। आ॒दि॒त्याः कामं॑ पितु॒मन्त॑म॒स्मे। आ न॑ पु॒त्रा अदि॑तेर्यान्तु य॒ज्ञम्। आ॒दि॒त्यास॑ प॒थिभि॑र्देव॒यानै ॥७॥

%2.8.2.2
अ॒स्मे काम॑न्दा॒शुषे॑ स॒न्नम॑न्तः। पुरो॒डाश॑ङ्घृ॒तव॑न्तं जुषन्ताम्। स्क॒भा॒यत॒ निर्\mbox{}ऋ॑ति॒ सेध॒ताम॑तिम्। प्र र॒श्मिभि॒र्यत॑माना अमृध्राः। आदि॑त्या॒ काम॒ प्रय॑तां॒ वष॑ट्कृतिम्। जु॒षध्व॑न्नो ह॒व्यदा॑तिं यजत्राः। आ॒दि॒त्यान्काम॒मव॑से हुवेम। ये भू॒तानि॑ ज॒नय॑न्तो विचि॒ख्युः। सीद॑न्तु पु॒त्रा अदि॑तेरु॒पस्थम्। स्ती॒र्णं ब॒र्॒हिर्\mbox{}ह॑वि॒रद्या॑य दे॒वाः॥८॥

%2.8.2.3
स्ती॒र्णं ब॒र्॒हिः सी॑दता य॒ज्ञे अ॒स्मिन्। ध्रा॒जाः सेध॑न्तो॒ अम॑तिन्दु॒रेवाम्। अ॒स्मभ्यं॑ पुत्रा अदिते॒ प्र यसत। आदि॑त्या॒ काम॑ ह॒विषो॑ जुषा॒णाः। अग्ने॒ नय॑ सु॒पथा॑ रा॒ये अ॒स्मान्। विश्वा॑नि देव व॒युना॑नि वि॒द्वान्। यु॒यो॒ध्य॑स्मज्जु॑हुरा॒णमेन॑। भूयि॑ष्ठान्ते॒ नम॑ उक्तिं विधेम। प्र व॑ शु॒क्राय॑ भा॒नवे॑ भरध्वम्। ह॒व्यं म॒तिञ्चा॒ग्नये॒ सुपू॑तम्॥९॥

%2.8.2.4
यो दैव्या॑नि॒ मानु॑षा ज॒नूषि॑। अ॒न्तर्विश्वा॑नि वि॒द्मना॒ जिगा॑ति। अच्छा॒ गिरो॑ म॒तयो॑ देव॒यन्ती। अ॒ग्निं य॑न्ति॒ द्रवि॑णं॒ भिक्ष॑माणाः। सु॒स॒न्दृश सु॒प्रती॑क॒ स्वञ्चम्। ह॒व्य॒वाह॑मर॒तिं मानु॑षाणाम्। अग्ने॒ त्वम॒स्मद्यु॑यो॒ध्यमी॑वाः। अन॑ग्नित्रा अ॒भ्य॑मन्त कृ॒ष्टीः। पुन॑र॒स्मभ्य सुवि॒ताय॑ देव। क्षां विश्वे॑भिर॒जरे॑भिर्यजत्र॥१०॥

%2.8.2.5
अग्ने॒ त्वं पा॑रया॒ नव्यो॑ अ॒स्मान्। स्व॒स्तिभि॒रति॑ दु॒र्गाणि॒ विश्वा। पूश्च॑ पृ॒थ्वी ब॑हु॒ला न॑ उ॒र्वी। भवा॑ तो॒काय॒ तन॑याय॒ शय्योँः। प्रका॑रवो मन॒ना व॒च्यमा॑नाः। दे॒व॒द्रीचीन्नयथ देव॒यन्त॑। द॒क्षि॒णा॒वाड्वा॒जिनी॒ प्राच्ये॑ति। ह॒विर्भर॑न्त्य॒ग्नये॑ घृ॒ताची। इन्द्र॒न्नरो॑ यु॒जे रथम्। ज॒गृ॒भ्णाते॒ दक्षि॑णमिन्द्र॒ हस्तम्॥११॥

%2.8.2.6
व॒सू॒यवो॑ वसुपते॒ वसू॑नाम्। वि॒द्मा हि त्वा॒ गोप॑ति शूर॒ गोनाम्। अ॒स्मभ्यं॑ चित्रं वृष॑ण र॒यिन्दा। तवे॒दं विश्व॑म॒भित॑ पश॒व्यम्। यत्पश्य॑सि॒ चक्ष॑सा॒ सूर्य॑स्य। गवा॑मसि॒ गोप॑ति॒रेक॑ इन्द्र। भ॒क्षी॒महि॑ ते॒ प्रय॑तस्य॒ वस्व॑। समि॑न्द्र णो॒ मन॑सा नेषि॒ गोभि॑। स सू॒रिभि॑र्मघव॒न्त्स स्व॒स्त्या। सं ब्रह्म॑णा दे॒वकृ॑तं॒ यदस्ति॑॥१२॥

%2.8.2.7
सन्दे॒वाना सुम॒त्या य॒ज्ञिया॑नाम्। आ॒राच्छत्रु॒मप॑ बाधस्व दू॒रम्। उ॒ग्रो यः शम्ब॑ पुरुहूत॒ तेन॑। अ॒स्मे धे॑हि॒ यव॑म॒द्गोम॑दिन्द्र। कृ॒धीधियं॑ जरि॒त्रे वाज॑रत्नाम्। आ वे॒धस॒ स हि शुचि॑। बृह॒स्पति॑ प्रथ॒मञ्जाय॑मानः। म॒हो ज्योति॑षः पर॒मे व्यो॑मन्। स॒प्तास्य॑स्तुविजा॒तो रवे॑ण। वि स॒प्तर॑श्मिरधम॒त्तमासि॥१३॥

%2.8.2.8
बृह॒स्पति॒ सम॑जय॒द्वसू॑नि। म॒हो व्र॒जान्गोम॑तो दे॒व ए॒षः। अ॒पः सिषा॑स॒न्त्सुव॒रप्र॑तीत्तः। बृह॒स्पति॒र्॒हन्त्य॒मित्र॑म॒र्कैः। बृह॑स्पते॒ पर्ये॒वा पि॒त्रे। आ नो॑ दि॒वः पावी॑रवी। इ॒मा जुह्वा॑ना॒ यस्ते॒ स्तन॑। सर॑स्वत्य॒भि नो॑ नेषि। इ॒य शुष्मे॑भिर्बिस॒खा इ॑वारुजत्। सानु॑ गिरी॒णान्त॑वि॒षेभि॑रू॒र्मिभि॑। पा॒रा॒व॒द॒घ्नीमव॑से सुवृ॒क्तिभि॑। सर॑स्वती॒मा वि॑वासेम धी॒तिभि॑॥१४॥\anuvakamend[दे॒व॒यानैर्दे॒वाः सुपू॑तं यजत्र॒ हस्त॒मस्ति॒ तमास्यू॒र्मिभि॒र्द्वे च॑]

%2.8.3.1
सोमो॑ धे॒नु सोमो॒ अर्व॑न्तमा॒शुम्। सोमो॑ वी॒रङ्क॑र्म॒ण्य॑न्ददातु। सा॒द॒न्यं॑ विद॒थ्य स॒भेयम्। पि॒तु॒ श्रव॑ण॒य्योँ ददा॑शदस्मै। अषा॑ढय्युँ॒त्सु त्व सो॑म॒ क्रतु॑भिः। या ते॒ धामा॑नि ह॒विषा॒ यज॑न्ति। त्वमि॒मा ओष॑धीः सोम॒ विश्वा। त्वम॒पो अ॑जनय॒स्त्वङ्गाः। त्वमात॑तन्थो॒र्व॑न्तरि॑क्षम्। त्वञ्ज्योति॑षा॒ वि तमो॑ ववर्थ॥१५॥

%2.8.3.2
या ते॒ धामा॑नि दि॒वि या पृ॑थि॒व्याम्। या पर्व॑ते॒ष्वोष॑धीष्व॒प्सु। तेभि॑र्नो॒ विश्वै सु॒मना॒ अहे॑डन्। राजन्त्सोम॒ प्रति॑ ह॒व्या गृ॑भाय। विष्णो॒र्नुक॒न्तद॑स्य प्रि॒यम्। प्र तद्विष्णु॑। प॒रो मात्र॑या त॒नुवा॑ वृधान। न ते॑ महि॒त्वमन्व॑श्ञुवन्ति। उ॒भे ते॑ विद्म॒ रज॑सी पृथि॒व्या विष्णो॑ देव॒ त्वम्। प॒र॒मस्य॑ वित्से॥१६॥

%2.8.3.3
विच॑क्रमे॒ त्रिर्दे॒वः। आ ते॑ म॒हो यो जा॒त ए॒व। अ॒भि गो॒त्राणि॑। आभि॒ स्पृधो॑ मिथ॒तीररि॑षण्यन्। अ॒मित्र॑स्य व्यथया म॒न्युमि॑न्द्र। आभि॒र्विश्वा॑ अभि॒युजो॒ विषू॑चीः। आर्या॑य॒ विशोव॑तारी॒र्दासी। अ॒य शृ॑ण्वे॒ अध॒ जय॑न्नु॒त घ्नन्। अ॒यमु॒त प्र कृ॑णुते यु॒धा गाः। य॒दा स॒त्यं कृ॑णु॒ते म॒न्युमिन्द्र॑॥१७॥

%2.8.3.4
विश्व॑न्दृ॒ढं भ॑यत॒ एज॑दस्मात्। अनु॑ स्व॒धाम॑क्षर॒न्नापो॑ अस्य। अव॑र्धत॒ मध्य॒ आ ना॒व्या॑नाम्। स॒ध्री॒चीने॑न॒ मन॑सा॒ तमि॑न्द्र॒ ओजि॑ष्ठेन। हन्म॑नाहन्न॒भिद्यून्। म॒रुत्व॑न्तं वृष॒भं वा॑वृधा॒नम्। अक॑वारिन्दि॒व्य शा॒समिन्द्रम्। वि॒श्वा॒साह॒मव॑से॒ नूत॑नाय। उ॒ग्र स॑हो॒दामि॒ह त हु॑वेम। जनि॑ष्ठा उ॒ग्रः सह॑से तु॒राय॑॥१८॥

%2.8.3.5
म॒न्द्र ओजि॑ष्ठो बहु॒लाभि॑मानः। अव॑र्ध॒न्निन्द्रं॑ म॒रुत॑श्चि॒दत्र॑। मा॒ता यद्वी॒रन्द॒धन॒द्धनि॑ष्ठा। क्व॑स्यावो॑ मरुतः स्व॒धाऽऽसीत्। यन्मामेक स॒मध॑त्ताहि॒हत्ये। अ॒ह ह्यु॑ग्रस्त॑वि॒षस्तुवि॑ष्मान्। विश्व॑स्य॒ शत्रो॒रन॑मं वध॒स्नैः। वृ॒त्रस्य॑ त्वा श्व॒सथा॒ दीष॑माणाः। विश्वे॑ दे॒वा अ॑जहु॒र्ये सखा॑यः। म॒रुद्भि॑रिन्द्र स॒ख्यन्ते॑ अस्तु॥१९॥

%2.8.3.6
अथे॒मा विश्वा॒ पृत॑ना जयासि। वधीं वृ॒त्रं म॑रुत इन्द्रि॒येण॑। स्वेन॒ भामे॑न तवि॒षो ब॑भू॒वान्। अ॒हमे॒ता मन॑वे वि॒श्वश्च॑न्द्राः। सु॒गा अ॒पश्च॑कर॒ वज्र॑बाहुः। स यो वृषा॒ वृष्णि॑येभि॒ समो॑काः। म॒हो दि॒वः पृ॑थि॒व्याश्च॑ स॒म्राट्। स॒ती॒नस॑त्वा॒ हव्यो॒ भरे॑षु। म॒रुत्वान्नो भव॒त्विन्द्र॑ ऊ॒ती। इन्द्रो॑ वृ॒त्रम॑तरद्वृत्र॒तूर्ये॥२०॥

%2.8.3.7
अ॒ना॒धृ॒ष्यो म॒घवा॒ शूर॒ इन्द्र॑। अन्वे॑नं॒ विशो॑ अमदन्त पू॒र्वीः। अ॒य राजा॒ जग॑तश्चर्\mbox{}षणी॒नाम्। स ए॒व वी॒रः स उ॑ वी॒र्या॑वान्। स ए॑करा॒जो जग॑तः पर॒स्पाः। य॒दा वृ॒त्रमत॑र॒च्छूर॒ इन्द्र॑। अथा॑भवद्दमि॒ताभिक्र॑तूनाम्। इन्द्रो॑ य॒ज्ञं व॒र्धय॑न्वि॒श्ववे॑दाः। पु॒रो॒डाश॑स्य जुषता ह॒विर्न॑। वृ॒त्रन्ती॒र्त्वा दा॑न॒वं वज्र॑बाहुः॥२१॥

%2.8.3.8
दिशो॑ऽदृहद्दृहि॒ता दृह॑णेन। इ॒मं य॒ज्ञं व॒र्धय॑न्वि॒श्ववे॑दाः। पु॒रो॒डाशं॒ प्रति॑ गृभ्णा॒त्विन्द्र॑। य॒दा वृ॒त्रमत॑र॒च्छूर॒ इन्द्र॑। अथै॑करा॒जो अ॑भव॒ज्जना॑नाम्। इन्द्रो॑ दे॒वाञ्छ॑म्बर॒हत्य॑ आवत्। इन्द्रो॑ दे॒वाना॑मभवत्पुरो॒गाः। इन्द्रो॑ य॒ज्ञे ह॒विषा॑ वावृधा॒नः। वृ॒त्र॒तूर्नो॒ अभ॑य॒ शर्म॑ यसत्। यः स॒प्त सिन्धू॒ रद॑धात्पृथि॒व्याम्। यः स॒प्त लो॒कानकृ॑णो॒द्दिश॑श्च। इन्द्रो॑ ह॒विष्मा॒न्त्सग॑णो म॒रुद्भि॑। वृ॒त्र॒तूर्नो॑ य॒ज्ञमि॒होप॑ यासत्॥२२॥\anuvakamend[व॒व॒र्थ॒ वि॒त्स॒ इन्द्र॑स्तु॒रायास्तु वृत्र॒तूर्ये॒ वज्र॑बाहुः पृथि॒व्यान्त्रीणि॑ च]

%2.8.4.1
इन्द्र॒स्तर॑स्वानभिमाति॒होग्रः। हिर॑ण्यवाशीरिषि॒रः सु॑व॒र्॒षाः। तस्य॑ व॒य सु॑म॒तौ य॒ज्ञिय॑स्य। अपि॑ भ॒द्रे सौ॑मन॒से स्या॑म। हिर॑ण्यवर्णो॒ अभ॑यङ्कृणोतु। अ॒भि॒मा॒ति॒हेन्द्र॒ पृत॑नासु जि॒ष्णुः। स न॒ शर्म॑ त्रि॒वरू॑थं॒ वि यसत्। यू॒यं पा॑त स्व॒स्तिभि॒ सदा॑ नः। इन्द्र स्तुहि व॒ज्रिण॒ स्तोम॑पृष्ठम्। पु॒रो॒डाश॑स्य जुषता ह॒विर्न॑॥२३॥

%2.8.4.2
ह॒त्वाभिमा॑ती॒ पृत॑ना॒ सह॑स्वान्। अथाभ॑यङ्कृणुहि वि॒श्वतो॑ नः। स्तु॒हि शूरं॑ व॒ज्रिण॒मप्र॑तीत्तम्। अ॒भि॒मा॒ति॒हनं॑ पुरुहू॒तमिन्द्रम्। य एक॒ इच्छ॒तप॑ति॒र्जने॑षु। तस्मा॒ इन्द्रा॑य ह॒विरा जु॑होत। इन्द्रो॑ दे॒वाना॑मधि॒पाः पु॒रोहि॑तः। दि॒शां पति॑रभवद्वा॒जिनी॑वान्। अ॒भि॒मा॒ति॒हा त॑वि॒षस्तुवि॑ष्मान्। अ॒स्मभ्यं॑ चित्रं वृष॑ण र॒यिन्दात्॥२४॥

%2.8.4.3
य इ॒मे द्यावा॑पृथि॒वी म॑हि॒त्वा। बले॒नादृहदभिमाति॒हेन्द्र॑। स नो॑ ह॒विः प्रति॑ गृभ्णातु रा॒तये। दे॒वानान्दे॒वो नि॑धि॒पा नो॑ अव्यात्। अन॑वस्ते॒ रथं॒ वृष्णे॒ यत्ते। इन्द्र॑स्य॒ नु वी॒र्याण्यह॒न्नहिम्। इन्द्रो॑ या॒तोऽव॑सितस्य॒ राजा। शम॑स्य च शृ॒ङ्गिणो॒ वज्र॑बाहुः। सेदु॒ राजा क्षेति चर्\mbox{}षणी॒नाम्। अ॒रान्न ने॒मिः परि॒ ता ब॑भूव॥२५॥

%2.8.4.4
अ॒भि सि॒ध्मो अ॑जिगादस्य॒ शत्रून्॑। विति॒ग्मेन॑ वृष॒भेणा॒ पुरो॑भेत्। सं वज्रे॑णासृजद्वृ॒त्रमिन्द्र॑। प्र स्वां म॒तिम॑तिर॒च्छाश॑दानः। विष्णुं॑ दे॒वं वरु॑णमू॒तये॒ भगम्। मेद॑सा दे॒वा व॒पया॑ यजध्वम्। ता नो॑ य॒ज्ञमाग॑तं वि॒श्वधे॑ना। प्र॒जाव॑द॒स्मे द्रवि॑णे॒ह ध॑त्तम्। मेद॑सा दे॒वा व॒पया॑ यजध्वम्। विष्णुं॑ च दे॒वं वरु॑णं च रा॒तिम्॥२६॥

%2.8.4.5
ता नो॒ अमी॑वा अप॒ बाध॑मानौ। इ॒मं य॒ज्ञं जु॒षमा॑णा॒वुपेतम्। विष्णू॑वरुणा यु॒वम॑ध्व॒राय॑ नः। वि॒शे जना॑य॒ महि॒ शर्म॑ यच्छतम्। दी॒र्घप्र॑यज्ज्यू ह॒विषा॑ वृधा॒ना। ज्योति॒षाऽरा॑तीर्दहत॒न्तमासि। ययो॒रोज॑सा स्कभि॒ता रजासि। वी॒र्ये॑भिर्वी॒रत॑मा॒ शवि॑ष्ठा। याऽपत्ये॑ ते॒ अप्र॑तीत्ता॒ सहो॑भिः। विष्णू॑ अग॒न्वरु॑णा पू॒र्वहू॑तौ॥२७॥

%2.8.4.6
विष्णू॑वरुणावभिशस्ति॒पावाम्। दे॒वा य॑जन्त ह॒विषा॑ घृ॒तेन॑। अपामी॑वा सेधत र॒क्षस॑श्च। अथा॑धत्तं॒ यज॑मानाय॒ शय्योँः। अ॒हो॒मुचा॑ वृष॒भा सु॒प्रतूर्ती। दे॒वानान्दे॒वत॑मा॒ शचि॑ष्ठा। विष्णू॑वरुणा॒ प्रति॑हर्यतन्नः। इ॒दन्नरा॒ प्रय॑तमू॒तये॑ ह॒विः। म॒ही नु द्यावा॑पृथि॒वी इ॒ह ज्येष्ठे। रु॒चा भ॑वता शु॒चय॑द्भिर॒र्कैः॥२८॥

%2.8.4.7
यत्सीं॒ वरि॑ष्ठे बृह॒ती वि॑मि॒न्वन्। नृ॒वद्भ्यो॒क्षा प॑प्रथा॒नेभि॒रेवै। प्रपूर्व॒जे पि॒तरा॒ नव्य॑सीभिः। गी॒र्भिः कृ॑णुध्व॒ सद॑ने ऋ॒तस्य॑। आ नो द्यावापृथिवी॒ दैव्ये॑न। जने॑न यातं॒ महि॑ वां॒ वरू॑थम्। स इत्स्वपा॒ भुव॑नेष्वास। य इ॒मे द्यावा॑पृथि॒वी ज॒जान॑। उ॒र्वी ग॑भी॒रे रज॑सी सु॒मेके। अ॒व॒शे धीर॒ शच्या॒ समै॑रत्॥२९॥

%2.8.4.8
भूरि॒न्द्वे अच॑रन्ती॒ चर॑न्तम्। प॒द्वन्त॒ङ्गर्भ॑म॒पदी॑दधाते। नित्य॒न्न सू॒नुं पि॒त्रोरु॒पस्थे। तं पि॑पृत रोदसी सत्य॒वाचम्। इ॒दन्द्या॑वापृथिवी स॒त्यम॑स्तु। पित॒र्मात॒र्यदि॒होप॑ ब्रु॒वे वाम्। भू॒तन्दे॒वाना॑मव॒मे अवो॑भिः। विद्यामे॒षं वृ॒जनं॑ जी॒रदा॑नुम्। उ॒र्वी पृ॒थ्वी ब॑हु॒ले दू॒रे अ॑न्ते। उप॑ ब्रुवे॒ नम॑सा य॒ज्ञे अ॒स्मिन्। दधा॑ते॒ ये सु॒भगे॑ सु॒प्रतूर्ती। द्यावा॒ रक्ष॑तं पृथि॒वी नो॒ अभ्वात्। या जा॒ता ओष॑ध॒योऽति॒ विश्वा परि॒ष्ठाः। या ओष॑धय॒ सोम॑राज्ञीरश्वाव॒ती सो॑मव॒तीम्। ओष॑धी॒रिति॑ मातरो॒ऽन्या वो॑ अ॒न्याम॑वतु॥३०॥\anuvakamend[ह॒विर्नो॑ दाद्भभूव रा॒तिं पू॒र्वहू॑ताव॒र्कैरै॑रद॒स्मिन्पञ्च॑ च]

%2.8.5.1
शुचि॒न्नु स्तोम॒ श्ञथ॑द्वृ॒त्रम्। उ॒भा वा॑मिन्द्राग्नी॒ प्र च॑र्\mbox{}ष॒णिभ्य॑। आ वृ॑त्रहणा गी॒र्भिर्विप्र॑। ब्रह्म॑णस्पते॒ त्वम॒स्य य॒न्ता। सू॒क्तस्य॑ बोधि॒ तन॑यं च जिन्व। विश्व॒न्तद्भ॒द्रं यद॒वन्ति॑ दे॒वाः। बृ॒हद्व॑देम वि॒दथे॑ सु॒वीरा। स ई स॒त्येभि॒ सखि॑भिः शु॒चद्भि॑। गोधा॑यसं॒ विध॑न॒सैर॑तर्दत्। ब्रह्म॑ण॒स्पति॒र्वृष॑भिर्व॒राहै॥३१॥

%2.8.5.2
घ॒र्मस्वे॑देभि॒र्द्रवि॑ण॒व्व्याँ॑नट्। ब्रह्म॑ण॒स्पते॑रभवद्यथाव॒शम्। स॒त्यो म॒न्युर्महि॒ कर्मा॑ करिष्य॒तः। यो गा उ॒दाज॒त्स दि॒वे वि चा॑भजत्। म॒हीव॑ री॒तिः शव॑सा सर॒त्पृथ॑क्। इन्धा॑नो अ॒ग्निं व॑नवद्वनुष्य॒तः। कृ॒तब्र॑ह्मा शूशुवद्रा॒तह॑व्य॒ इत्। जा॒तेन॑ जा॒तमति॒सृत्प्र सृसते। यं य॒य्युँज॑ङ्कृणु॒ते ब्रह्म॑ण॒स्पति॑। ब्रह्म॑णस्पते सु॒यम॑स्य वि॒श्वहा॥३२॥

%2.8.5.3
रा॒यः स्या॑म र॒थ्यो॑ विव॑स्वतः। वी॒रेषु॑ वी॒रा उप॑पृङ्ग्धि न॒स्त्वम्। यदीशा॑नो॒ ब्रह्म॑णा॒ वेषि॑ मे॒ हवम्। स इज्जने॑न॒ स वि॒शा स जन्म॑ना। स पु॒त्रैर्वाजं॑ भरते॒ धना॒ नृभि॑। दे॒वानां॒ यः पि॒तर॑मा॒ विवा॑सति। श्र॒द्धाम॑ना ह॒विषा॒ ब्रह्म॑ण॒स्पतिम्। यास्ते॑ पूष॒न्नावो॑ अ॒न्तः। शु॒क्रन्ते॑ अ॒न्यत्पू॒षेमा आशा। प्रप॑थे प॒थाम॑जनिष्ट पू॒षा ॥३३॥

%2.8.5.4
प्रप॑थे दि॒वः प्रप॑थे पृथि॒व्याः। उ॒भे अ॒भि प्रि॒यत॑मे स॒धस्थे। आ च॒ परा॑ च चरति प्रजा॒नन्। पू॒षा सु॒बन्धु॑र्दि॒व आ पृ॑थि॒व्याः। इ॒डस्पति॑र्म॒घवा॑ द॒स्मव॑र्चाः। तन्दे॒वासो॒ अद॑दुः सू॒र्यायै। कामे॑न कृ॒तन्त॒वस॒ स्वञ्चम्। अ॒जाऽश्व॑ पशु॒पा वाज॑बस्त्यः। धि॒यं॒ जि॒न्वो विश्वे॒ भुव॑ने॒ अर्पि॑तः। अष्ट्रां पू॒षा शि॑थि॒रामु॒द्वरी॑वृजत्॥३४॥

%2.8.5.5
स॒ञ्चक्षा॑णो॒ भुव॑ना दे॒व ई॑यते। शुची॑ वो ह॒व्या म॑रुत॒ शुची॑नाम्। शुचि हिनोम्यध्व॒र शुचि॑भ्यः। ऋ॒तेन॑ स॒त्यमृत॒साप॑ आयन्। शुचि॑जन्मान॒ शुच॑यः पाव॒काः। प्र चि॒त्रम॒र्कं गृ॑ण॒ते तु॒राय॑। मारु॑ताय॒ स्वत॑वसे भरध्वम्। ये सहासि॒ सह॑सा॒ सह॑न्ते। रेज॑ते अग्ने पृथि॒वी म॒खेभ्य॑। असे॒ष्वा म॑रुतः खा॒दयो॑ वः॥३५॥

%2.8.5.6
वक्ष॑ सुरु॒क्मा उप॑ शिश्रिया॒णाः। वि वि॒द्युतो॒ न वृ॒ष्टिभी॑ रुचा॒नाः। अनु॑ स्व॒धामायु॑धै॒र्यच्छ॑मानाः। या व॒ शर्म॑ शशमा॒नाय॒ सन्ति॑। त्रि॒धातू॑नि दा॒शुषे॑ यच्छ॒ताधि॑। अ॒स्मभ्य॒न्तानि॑ मरुतो॒ विय॑न्त। र॒यिन्नो॑ धत्त वृषणः सु॒वीरम्। इ॒मे तु॒रं म॒रुतो॑ रामयन्ति। इ॒मे सह॒ सह॑स॒ आ न॑मन्ति। इ॒मे शसं॑वनुष्य॒तो नि पान्ति॥३६॥

%2.8.5.7
गु॒रुद्वेषो॒ अर॑रुषे दधन्ति। अ॒रा इ॒वेदच॑रमा॒ अहे॑व। प्रप्र॑ जायन्ते॒ अक॑वा॒ महो॑भिः। पृश्ञे प्रु॒त्रा उ॑प॒मासो॒ रभि॑ष्ठाः। स्वया॑ म॒त्या म॒रुत॒ सं मि॑मिक्षुः। अनु॑ ते दायि म॒ह इ॑न्द्रि॒याय॑। स॒त्रा ते॒ विश्व॒मनु॑ वृत्र॒हत्ये। अनु॑ क्ष॒त्रमनु॒ सहो॑ यजत्र। इन्द्र॑ दे॒वेभि॒रनु॑ ते नृ॒षह्ये। य इन्द्र॒ शुष्मो॑ मघवन्ते॒ अस्ति॑॥३७॥

%2.8.5.8
शिक्षा॒ सखि॑भ्यः पुरुहूत॒ नृभ्य॑। त्व हि दृ॒ढा म॑घव॒न्विचे॑ताः। अपा॑वृधि॒ परि॑वृति॒न्न राध॑। इन्द्रो॒ राजा॒ जग॑तश्चर्‌षणी॒नाम्। अ॒धि॒क्षमि॒ विषु॑रूपं॒ यदस्ति॑। ततो॑ ददातु दा॒शुषे॒ वसू॑नि। चोद॒द्राध॒ उप॑स्तुतश्चिद॒र्वाक्। तमु॑ष्टुहि॒ यो अ॒भिभूत्योजाः। व॒न्वन्नवा॑तः पुरुहू॒त इन्द्र॑। अषा॑ढमु॒ग्र सह॑मानमा॒भिः॥३८॥

%2.8.5.9
गी॒र्भिर्व॑र्ध वृष॒भञ्च॑र्\mbox{}षणी॒नाम्। स्थू॒रस्य॑ रा॒यो बृ॑ह॒तो य ईशे। तमु॑ ष्टवाम वि॒दथे॒ष्विन्द्रम्। यो वा॒युना॒ जय॑ति॒ गोम॑तीषु। प्र धृ॑ष्णु॒या न॑यति॒ वस्यो॒ अच्छ॑। आ ते॒ शुष्मो॑ वृष॒भ ए॑तु प॒श्चात्। ओत्त॒राद॑ध॒रागा पु॒रस्तात्। आ वि॒श्वतो॑ अ॒भिसमेत्व॒र्वाङ्। इन्द्र॑ द्यु॒म्न सुव॑र्वद्धेह्य॒स्मे॥३९॥\anuvakamend[व॒राहैर्वि॒श्वहा॑ऽजनिष्ट पू॒षोद्वरी॑वृजत्खा॒दयो॑ वः पा॒न्त्यस्त्या॒भिर्नव॑ च]

%2.8.6.1
आ दे॒वो या॑तु सवि॒ता सु॒रत्न॑। अ॒न्त॒रि॒क्ष॒प्रा वह॑मानो॒ अश्वै। हस्ते॒ दधा॑नो॒ नर्या॑ पु॒रूणि॑। नि॒वे॒शयं॑ च प्रसु॒वं च॒ भूम॑। अ॒भीवृ॑त॒ङ्कृश॑नैर्वि॒श्वरू॑पम्। हिर॑ण्यशम्यं यज॒तो बृ॒हन्तम्। आस्था॒द्रथ सवि॒ता चि॒त्रभा॑नुः। कृ॒ष्णा रजा सि॒ तवि॑षी॒न्दधा॑नः। सघा॑ नो दे॒वः स॑वि॒ता स॒वाय॑। आ सा॑विष॒द्वसु॑पति॒र्वसू॑नि॥४०॥

%2.8.6.2
वि॒श्रय॑माणो॒ अम॑तिमुरू॒चीम्। म॒र्त॒भोज॑न॒मध॑रासतेन। विजनाञ्छ्या॒वाः शि॑ति॒पादो॑ अख्यन्। रथ॒ हिर॑ण्यप्रउगं॒ वह॑न्तः। शश्व॒द्दिश॑ सवि॒तुर्दैव्य॑स्य। उ॒पस्थे॒ विश्वा॒ भुव॑नानि तस्थुः। वि सु॑प॒र्णो अ॒न्तरि॑क्षाण्यख्यत्। ग॒भी॒रवे॑पा॒ असु॑रः सुनी॒थः। क्वे॑दानी॒ सूर्य॒ कश्चि॑केत। क॒त॒मान्द्या र॒श्मिर॒स्या त॑तान॥४१॥

%2.8.6.3
भग॒न्धियं॑ वा॒जय॑न्त॒ पुर॑न्धिम्। नरा॒शसो॒ ग्नास्पति॑र्नो अव्यात्। आ ये वा॒मस्य॑ सङ्ग॒थे र॑यी॒णाम्। प्रि॒या दे॒वस्य॑ सवि॒तुः स्या॑म। आ नो॒ विश्वे॒ अस्क्रा॑गमन्तु दे॒वाः। मि॒त्रो अ॑र्य॒मा वरु॑णः स॒जोषा। भुव॒न्॒ यथा॑ नो॒ विश्वे॑ वृ॒धास॑। करन्त्सु॒षाहा॑ विथु॒रन्न शव॑। शन्नो॑ दे॒वा वि॒श्वदे॑वा भवन्तु। श सर॑स्वती स॒ह धी॒भिर॑स्तु॥४२॥

%2.8.6.4
शम॑भि॒षाच॒ शमु॑ राति॒षाच॑। शन्नो॑ दि॒व्याः पार्थि॑वा॒ शन्नो॒ अप्या। ये स॑वि॒तुः स॒त्यस॑वस्य॒ विश्वे। मि॒त्रस्य॑ व्र॒ते वरु॑णस्य दे॒वाः। ते सौभ॑गं वी॒रव॒द्गोम॒दप्न॑। दधा॑तन॒ द्रवि॑णञ्चि॒त्रम॒स्मे। अग्ने॑ या॒हि दू॒त्यं॑ वारि॑षेण्यः। दे॒वा अच्छा ब्रह्म॒कृता॑ ग॒णेन॑। सर॑स्वतीं म॒रुतो॑ अ॒श्विना॒पः। य॒क्षि॒ दे॒वान्र॑त्न॒धेया॑य॒ विश्वान्॑॥४३॥

%2.8.6.5
द्यौः पि॑त॒ पृथि॑वि॒ मात॒रध्रु॑क्। अग्ने भ्रातर्वसवो मृ॒डता॑ नः। विश्व॑ आदित्या अदिते स॒जोषा। अ॒स्मभ्य॒ शर्म॑ बहु॒लं वि य॑न्त। विश्वे॑ देवाः शृणु॒तेम हवं॑ मे। ये अ॒न्तरि॑क्षे॒ य उप॒ द्यवि॒ ष्ठ। ये अ॑ग्निजि॒ह्वा उ॒त वा॒ यज॑त्राः। आ॒सद्या॒स्मिन्ब॒र्॒हिषि॑ मादयध्वम्। आ वां मित्रावरुणा ह॒व्यजु॑ष्टिम्। नम॑सा देवा॒वव॑साववृत्याम्॥४४॥

%2.8.6.6
अ॒स्माकं॒ ब्रह्म॒ पृत॑नासु सह्या अ॒स्माकम्। वृ॒ष्टिर्दि॒व्या सु॑पा॒रा। यु॒वं वस्त्रा॑णि पीव॒सा व॑साथे। यु॒वोरच्छि॑द्रा॒ मन्त॑वो ह॒ सर्गा। अवा॑तिरत॒मनृ॑तानि॒ विश्वा। ऋ॒तेन॑ मित्रावरुणा सचेथे। तत्सु वां मित्रावरुणा महि॒त्वम्। ई॒र्मा त॒स्थुषी॒रह॑भिर्दुदुह्रे। विश्वा पिन्वथ॒ स्वस॑रस्य॒ धेना। अनु॑ वा॒मेक॑ प॒विरा व॑वर्ति॥४५॥

%2.8.6.7
यद्बहि॑ष्ठ॒न्नाति॒ विदे॑ सुदानू। अच्छि॑द्र॒ शर्म॒ भुव॑नस्य गोपा। ततो॑ नो मित्रावरुणाववीष्टम्। सिषा॑सन्तो जी(जि?)गि॒वास॑ स्याम। आ नो मित्रावरुणा ह॒व्यदा॑तिम्। घृ॒तैर्गव्यू॑तिमुक्षत॒मिडा॑भिः। प्रति॑ वा॒मत्र॒ वर॒मा जना॑य। पृ॒णी॒तमु॒द्नो दि॒व्यस्य॒ चारो। प्र बा॒हवा॑ सिसृतञ्जी॒वसे॑ नः। आ नो॒ गव्यू॑तिमुक्षतङ्घृ॒तेन॑॥४६॥

%2.8.6.8
आ नो॒ जने श्रवयतय्युँवाना। श्रु॒तं मे॑ मित्रावरुणा॒ हवे॒मा। इ॒मा रु॒द्राय॑ स्थि॒रध॑न्वने॒ गिर॑। क्षि॒प्रेष॑वे दे॒वाय॑ स्व॒धाम्ने। अषा॑ढाय॒ सह॑मानाय मी॒ढुषे। ति॒ग्मायु॑धाय भरता शृ॒णोत॑न। त्वाद॑त्तेभी रुद्र॒ शन्त॑मेभिः। श॒त हिमा॑ अशीय भेष॒जेभि॑। व्य॑स्मद्द्वेषो॑ वित॒रव्व्यँह॑। व्यमी॑वाश्चातयस्वा॒ विषू॑चीः॥४७॥

%2.8.6.9
अर्\mbox{}ह॑न्बिभर्\mbox{}षि॒ मा न॑स्तो॒के। आ ते॑ पितर्मरुता सु॒म्नमे॑तु। मा न॒ सूर्य॑स्य स॒न्दृशो॑ युयोथाः। अ॒भि नो॑ वी॒रो अर्व॑ति क्षमेत। प्र जा॑येमहि रुद्र प्र॒जाभि॑। ए॒वा ब॑भ्रो वृषभ चेकितान। यथा॑ देव॒ न हृ॑णी॒षे न हसि॑। हा॒व॒न॒श्रूर्नो॑ रुद्रे॒ह बो॑धि। बृ॒हद्व॑देम वि॒दथे॑ सु॒वीरा। परि॑ णो रु॒द्रस्य॑ हे॒तिः स्तु॒हि श्रु॒तम्। मीढु॑ष्ट॒मार्\mbox{}ह॑न्बिभर्\mbox{}षि। त्वम॑ग्ने रु॒द्र आ वो॒ राजा॑नम्॥४८॥\anuvakamend[वसू॑नि ततानास्तु॒ विश्वान्॑ ववृत्यां ववर्ति घृ॒तेन॒ विषू॑चीः श्रु॒तन्द्वे च॑]

%2.8.7.1
सूर्यो॑ दे॒वीमु॒षस॒ रोच॑माना॒मर्य॑। न योषा॑म॒भ्ये॑ति प॒श्चात्। यत्रा॒ नरो॑ देव॒यन्तो॑ यु॒गानि॑। वि॒त॒न्वते॒ प्रति॑ भ॒द्राय॑ भ॒द्रम्। भ॒द्रा अश्वा॑ ह॒रित॒ सूर्य॑स्य। चि॒त्रा एद॑ग्वा अनु॒माद्या॑सः। न॒म॒स्यन्तो॑ दि॒व आ पृ॒ष्ठम॑स्थुः। परि॒ द्यावा॑पृथि॒वी य॑न्ति स॒द्यः। तत्सूर्य॑स्य देव॒त्वन्तन्म॑हि॒त्वम्। म॒ध्या कर्तो॒र्वित॑त॒ सञ्ज॑भार॥४९॥

%2.8.7.2
य॒देदयु॑क्त ह॒रित॑ स॒धस्थात्। आद्रात्री॒ वास॑स्तनुते सि॒मस्मै। तन्मि॒त्रस्य॒ वरु॑णस्याभि॒चक्षे। सूर्यो॑ रू॒पं कृ॑णुते॒ द्योरु॒पस्थे। अ॒न॒न्तम॒न्यद्रुश॑दस्य॒ पाज॑। कृ॒ष्णम॒न्यद्ध॒रित॒ सं भ॑रन्ति। अ॒द्या दे॑वा॒ उदि॑ता॒ सूर्य॑स्य। निरह॑सः पिपृ॒तान्निर॑व॒द्यात्। तन्नो॑ मि॒त्रो वरु॑णो मामहन्ताम्। अदि॑ति॒ सिन्धु॑ पृथि॒वी उ॒त द्यौः॥५०॥

%2.8.7.3
दि॒वो रु॒क्म उ॑रु॒चक्षा॒ उदे॑ति। दू॒रे अ॑र्थस्त॒रणि॒र्भ्राज॑मानः। नू॒नञ्जना॒ सूर्ये॑ण॒ प्रसू॑ताः। आयन्नर्था॑नि कृ॒णव॒न्नपासि। शन्नो॑ भव॒ चक्ष॑सा॒ शन्नो॒ अह्ना। शं भा॒नुना॒ श हि॒मा शङ्घृ॒णेन॑। यथा॒ शम॒स्मै शमस॑द्दुरो॒णे। तत्सूर्य॒ द्रवि॑णन्धे॒हि चि॒त्रम्। चि॒त्रन्दे॒वाना॒मुद॑गा॒दनी॑कम्। चक्षु॑र्मि॒त्रस्य॒ वरु॑णस्या॒ग्नेः॥५१॥

%2.8.7.4
आप्रा॒ द्यावा॑पृथि॒वी अ॒न्तरि॑क्षम्। सूर्य॑ आ॒त्मा जग॑तस्त॒स्थुष॑श्च। त्वष्टा॒ दध॒त्तन्न॑स्तु॒रीपम्। त्वष्टा॑ वी॒रं पि॒शङ्ग॑रूपः। दशे॒मन्त्वष्टु॑र्जनयन्त॒ गर्भम्। अत॑न्द्रासो युव॒तयो॒ बिभ॑र्त्रम्। ति॒ग्मानी॑क॒ स्वय॑शस॒ञ्जने॑षु। वि॒रोच॑मानं॒ परि॑षीन्नयन्ति। आविष्ट्यो॑ वर्धते॒ चारु॑रासु। जि॒ह्माना॑मू॒र्ध्वस्वय॑शा उ॒पस्थे॥५२॥

%2.8.7.5
उ॒भे त्वष्टु॑र्बिभ्यतु॒र्जाय॑मानात्। प्र॒तीची॑ सि॒हं प्रति॑जोषयेते। मि॒त्रो जना॒न्प्र स मि॑त्र। अ॒यं मि॒त्रो न॑म॒स्य॑ सु॒शेव॑। राजा॑ सुक्ष॒त्रो अ॑जनिष्ट वे॒धाः। तस्य॑ व॒य सु॑म॒तौ य॒ज्ञिय॑स्य। अपि॑ भ॒द्रे सौ॑मन॒से स्या॑म। अ॒न॒मी॒वास॒ इड॑या॒ मद॑न्तः। मि॒तज्म॑वो॒ वरि॑म॒न्ना पृ॑थि॒व्याः। आ॒दि॒त्यस्य॑ व्र॒तमु॑प॒क्ष्यन्त॑॥५३॥

%2.8.7.6
व॒यं मि॒त्रस्य॑ सुम॒तौ स्या॑म। मि॒त्रन्न ई शिम्या॒ गोषु॑ ग॒व्यव॑त्। स्वा॒धियो॑ वि॒दथे॑ अ॒प्स्वजी॑जनन्। अरे॑जयता॒ रोद॑सी॒ पाज॑सा गि॒रा। प्रति॑ प्रि॒यं य॑ज॒तञ्ज॒नुषा॒मव॑। म॒हा आ॑दि॒त्यो नम॑सोप॒सद्य॑। या॒त॒यज्ज॑नो गृण॒ते सु॒शेव॑। तस्मा॑ ए॒तत्पन्य॑तमाय॒ जुष्टम्। अ॒ग्नौ मि॒त्राय॑ ह॒विरा जु॑होत। आ वा॒ रथो॒ रोद॑सी बद्बधा॒नः॥५४॥

%2.8.7.7
हि॒र॒ण्ययो॒ वृष॑भिर्या॒त्वश्वै। घृ॒तव॑र्तनिः प॒विभी॑रुचा॒नः। इ॒षाव्वोँ॒ढा नृ॒पति॑र्वा॒जिनी॑वान्। स प॑प्रथा॒नो अ॒भि पञ्च॒ भूम॑। त्रि॒व॒न्धु॒रो मन॒साया॑तु यु॒क्तः। विशो॒ येन॒ गच्छ॑थो देव॒यन्ती। कुत्रा॑ चि॒द्याम॑मश्विना॒ दधा॑ना। स्वश्वा॑ य॒शसाऽऽया॑तम॒र्वाक्। दस्रा॑ नि॒धिं मधु॑मन्तं पिबाथः। वि वा॒ रथो॑ व॒ध्वा॑ याद॑मानः॥५५॥

%2.8.7.8
अन्तान्दि॒वो बा॑धते वर्त॒निभ्याम्। यु॒वोः श्रियं॒ परि॒ योषा॑वृणीत। सूरो॑ दुहि॒ता परि॑तक्मियायाम्। यद्दे॑व॒यन्त॒मव॑थ॒ शची॑भिः। परि॑घ्र॒ सवां॒ मना॑वां॒ वयो॑गाम्। यो ह॒स्यवा रथिरा॒वस्त॑ उ॒स्राः। रथो॑ युजा॒नः प॑रि॒याति॑ व॒र्तिः। तेन॑ न॒ शँय्योरु॒षसो॒ व्यु॑ष्टौ। न्य॑श्विना वहतं य॒ज्ञे अ॒स्मिन्। यु॒वं भु॒ज्युमव॑विद्ध समु॒द्रे॥५६॥

%2.8.7.9
उदू॑हथु॒रर्ण॑सो॒ अस्रि॑धानैः। प॒त॒त्रिभि॑रश्र॒मैर॑व्य॒थिभि॑। द॒सना॑भिरश्विना पा॒रय॑न्ता। अग्नी॑षोमा॒ यो अ॒द्य वाम्। इ॒दं वच॑ सप॒र्यति॑। तस्मै॑ धत्त सु॒वीर्यम्। गवां॒ पोष॒ स्वश्वि॑यम्। यो अ॒ग्नीषोमा॑ ह॒विषा॑ सप॒र्यात्। दे॒व॒द्रीचा॒ मन॑सा॒ यो घृ॒तेन॑। तस्य॑ व्र॒त र॑क्षतं पा॒तमह॑सः॥५७॥

%2.8.7.10
वि॒शे जना॑य॒ महि॒ शर्म॑ यच्छतम्। अग्नी॑षोमा॒ य आहु॑तिम्। यो वा॒न्दाशाद्ध॒विष्कृ॑तिम्। स प्र॒जया॑ सु॒वीर्यम्। विश्व॒मायु॒र्व्य॑श्ञवत्। अग्नी॑षोमा॒ चेति॒ तद्वी॒र्यं॑ वाम्। यदमु॑ष्णीतमव॒सं प॒णिङ्गोः। अवा॑तिरतं॒ प्रथ॑यस्य॒ शेष॑। अवि॑न्दतं॒ ज्योति॒रेकं॑ ब॒हुभ्य॑। अग्नी॑षोमावि॒म सु मेऽग्नी॑षोमा ह॒विष॒ प्रस्थि॑तस्य॥५८॥\anuvakamend[ज॒भा॒र॒ द्यौर॒ग्नेरु॒पस्थ॑ उप॒क्ष्यन्तो॑ बद्बधा॒नो व॒ध्वा॑ याद॑मानः समु॒द्रेऽह॑स॒ प्रस्थि॑तस्य]

%2.8.8.1
अ॒हम॑स्मि प्रथम॒जा ऋ॒तस्य॑। पूर्वं॑ दे॒वेभ्यो॑ अ॒मृत॑स्य॒ नाभि॑। यो मा॒ ददा॑ति॒ स इदे॒वमावा। अ॒हमन्न॒मन्न॑न॒दन्त॑मद्मि। पूर्व॑म॒ग्नेरपि॑ दह॒त्यन्नम्। य॒त्तौ हा॑साते अहमुत्त॒रेषु॑। व्यात्त॑मस्य प॒शव॑ सु॒जम्भम्। पश्य॑न्ति॒ धीरा॒ प्रच॑रन्ति॒ पाका। जहाम्य॒न्यन्न ज॑हाम्य॒न्यम्। अ॒हमन्नं॒ वश॒मिच्च॑रामि॥५९॥

%2.8.8.2
स॒मा॒नमर्थं॒ पर्ये॑मि भु॒ञ्जत्। को मामन्नं॑ मनु॒ष्यो॑ दयेत। परा॑के॒ अन्न॒न्निहि॑तं लो॒क ए॒तत्। विश्वैर्दे॒वैः पि॒तृभि॑र्गु॒प्तमन्नम्। यद॒द्यते॑ लु॒प्यते॒ यत्प॑रो॒प्यते। श॒त॒त॒मी सा त॒नूर्मे॑ बभूव। म॒हान्तौ॑ च॒रू स॑कृद्दु॒ग्धेन॑ पप्रौ। दिवं॑ च॒ पृश्ञि॑ पृथि॒वीं च॑ सा॒कम्। तत्सं॒पिब॑न्तो॒ न मि॑नन्ति वे॒धस॑। नैतद्भूयो॒ भव॑ति॒ नो कनी॑यः॥६०॥

%2.8.8.3
अन्नं॑ प्रा॒णमन्न॑मपा॒नमा॑हुः। अन्नं॑ मृ॒त्युन्तमु॑ जी॒वातु॑माहुः। अन्नं॑ ब्र॒ह्माणो॑ ज॒रसं॑  वदन्ति। अन्न॑माहुः प्र॒जन॑नं प्र॒जानाम्। मोघ॒मन्नं॑ विन्दते॒ अप्र॑चेताः। स॒त्यं ब्र॑वीमि व॒ध इत्स तस्य॑। नार्य॒मणं॒ पुष्य॑ति॒ नो सखा॑यम्। केव॑लाघो भवति केवला॒दी। अ॒हं मे॒घः स्त॒नय॒न्वऱ़्ष॑न्नस्मि। माम॑दन्त्य॒हम॑द्म्य॒न्यान्॥६१॥

%2.8.8.4
अ॒ह सद॒मृतो॑ भवामि। मदा॑दि॒त्या अधि॒ सर्वे॑ तपन्ति। दे॒वीं वाच॑मजनयन्त॒ यद्वाग्वद॑न्ती। अ॒न॒न्तामन्ता॒दधि॒ निर्मि॑तां म॒हीम्। यस्यान्दे॒वा अ॑दधु॒र्भोज॑नानि। एकाक्षरां द्वि॒पदा॒ षट्प॑दां च। वाचं॑ दे॒वा उप॑ जीवन्ति॒ विश्वे। वाचं॑ दे॒वा उप॑ जीवन्ति॒ विश्वे। वाच॑ङ्गन्ध॒र्वाः प॒शवो॑ मनु॒ष्या। वा॒चीमा विश्वा॒ भुव॑ना॒न्यर्पि॑ता॥६२॥

%2.8.8.5
सा नो॒ हवं॑ जुषता॒मिन्द्र॑पत्नी। वाग॒क्षरं॑ प्रथम॒जा ऋ॒तस्य॑। वेदा॑नां मा॒ताऽमृत॑स्य॒ नाभि॑। सा नो॑ जुषा॒णोप॑ य॒ज्ञमागात्। अव॑न्ती दे॒वी सु॒हवा॑ मे अस्तु। यामृष॑यो मन्त्र॒कृतो॑ मनी॒षिण॑। अ॒न्वैच्छं॑ दे॒वास्तप॑सा॒ श्रमे॑ण। तान्दे॒वीं वाच ह॒विषा॑ यजामहे। सा नो॑ दधातु सुकृ॒तस्य॑ लो॒के। च॒त्वारि॒ वाक्परि॑मिता प॒दानि॑॥६३॥

%2.8.8.6
तानि॑ विदुर्ब्राह्म॒णा ये म॑नी॒षिण॑। गुहा॒ त्रीणि॒ निहि॑ता॒ नेङ्ग॑यन्ति। तु॒रीयं॑ वा॒चो म॑नु॒ष्या॑ वदन्ति। श्र॒द्धया॒ऽग्निः समि॑ध्यते। श्र॒द्धया॑ विन्दते ह॒विः। श्र॒द्धां भग॑स्य मू॒र्धनि॑। वच॒सा वे॑दयामसि। प्रि॒य श्र॑द्धे॒ दद॑तः। प्रि॒य श्र॑द्धे॒ दिदा॑सतः। प्रि॒यं भो॒जेषु॒ यज्व॑सु॥६४॥

%2.8.8.7
इ॒दं म॑ उदि॒तं कृ॑धि। यथा॑ दे॒वा असु॑रेषु। श्र॒द्धामु॒ग्रेषु॑ चक्रि॒रे। ए॒वं भो॒जेषु॒ यज्व॑सु। अ॒स्माक॑मुदि॒तं कृ॑धि। श्र॒द्धान्दे॑वा॒ यज॑मानाः। वा॒युगो॑पा॒ उपा॑सते। श्र॒द्धा हृ॑द॒य्य॑याऽऽकूत्या। श्र॒द्धया॑ हूयते ह॒विः। श्र॒द्धां प्रा॒तर्\mbox{}ह॑वामहे॥६५॥

%2.8.8.8
श्र॒द्धां म॒ध्यन्दि॑नं॒ परि॑। श्र॒द्धा सूर्य॑स्य नि॒म्रुचि॑। श्रद्धे॒ श्रद्धा॑पये॒ह मा। श्र॒द्धा दे॒वानधि॑ वस्ते। श्र॒द्धा विश्व॑मि॒दञ्जग॑त्। श्र॒द्धाङ्काम॑स्य मा॒तरम्। ह॒विषा॑ वर्धयामसि। ब्रह्म॑ जज्ञा॒नं प्र॑थ॒मं पु॒रस्तात्। वि सी॑म॒तः सु॒रुचो॑ वे॒न आ॑वः। स बु॒ध्निया॑ उप॒ मा अ॑स्य वि॒ष्ठाः॥६६॥

%2.8.8.9
स॒तश्च॒ योनि॒मस॑तश्च॒ विव॑। पि॒ता वि॒राजा॑मृष॒भो र॑यी॒णाम्। अ॒न्तरि॑क्षं वि॒श्वरू॑प॒ आवि॑वेश। तम॒र्कैर॒भ्य॑र्चन्ति व॒त्सम्। ब्रह्म॒ सन्तं॒ ब्रह्म॑णा व॒र्धय॑न्तः। ब्रह्म॑ दे॒वान॑जनयत्। ब्रह्म॒ विश्व॑मि॒दञ्जग॑त्। ब्रह्म॑णः क्ष॒त्रन्निर्मि॑तम्। ब्रह्म॑ ब्राह्म॒ण आ॒त्मना। अ॒न्तर॑स्मिन्नि॒मे लो॒काः॥६७॥

%2.8.8.10
अ॒न्तर्विश्व॑मि॒दञ्जग॑त्। ब्रह्मै॒व भू॒तानां॒ ज्येष्ठम्। तेन॒ को॑ऽर्\mbox{}हति॒ स्पर्धि॑तुम्। ब्रह्म॑न्दे॒वास्त्रय॑स्त्रिशत्। ब्रह्म॑न्निन्द्रप्रजाप॒ती। ब्रह्म॑न् ह॒ विश्वा॑ भू॒तानि॑। ना॒वीवा॒न्तः स॒माहि॑ता। चत॑स्र॒ आशा॒ प्रच॑रन्त्व॒ग्नय॑। इ॒मन्नो॑ य॒ज्ञन्न॑यतु प्रजा॒नन्। घृ॒तं पिन्व॑न्न॒जर सु॒वीरम्॥६८॥

%2.8.8.12
ब्रह्म॑ स॒मिद्भ॑व॒त्याहु॑तीनाम्। आ गावो॑ अग्मन्नु॒त भ॒द्रम॑क्रन्। सीद॑न्तु गो॒ष्ठे र॒णय॑न्त्व॒स्मे। प्र॒जाव॑तीः पुरु॒रूपा॑ इ॒ह स्युः। इन्द्रा॑य पू॒र्वीरु॒षसो॒ दुहा॑नाः। इन्द्रो॒ यज्व॑ने पृण॒ते च॑ शिक्षति। उपेद्द॑दाति॒ न स्वं मु॑षायति। भूयो॑भूयो र॒यमिद॑स्य व॒र्धय\sn{}। अभि॑न्ने खि॒ल्ले नि द॑धाति देव॒युम्। न ता न॑शन्ति॒ न ता अर्वा॥६९॥

%2.8.8.13
गावो॒ भगो॒ गाव॒ इन्द्रो॑ मे अच्छात्। गाव॒ सोम॑स्य प्रथ॒मस्य॑ भ॒क्षः। इ॒मा या गाव॒ सज॑नास॒ इन्द्र॑। इ॒च्छामीद्धृ॒दा मन॑सा चि॒दिन्द्रम्। यू॒यङ्गा॑वो मेदयथा कृ॒शञ्चि॑त्। अ॒श्ली॒लञ्चि॑त्कृणुथा सु॒प्रती॑कम्। भ॒द्रङ्गृ॒हं कृ॑णुथ भद्रवाचः। बृ॒हद्वो॒ वय॑ उच्यते स॒भासु॑। प्र॒जाव॑तीः सू॒यव॑स रि॒शन्ती। शु॒द्धा अ॒पः सु॑प्रपा॒णे पिब॑न्तीः। मा व॑ स्ते॒न ई॑शत॒ माऽघशसः। परि॑ वो हे॒ती रु॒द्रस्य॑ वृञ्ज्यात्। उपे॒दमु॑प॒पर्च॑नम्। आ॒सु गोषूप॑पृच्यताम्। उप॑र्\mbox{}ष॒भस्य॒ रेत॑सि। उपेन्द्र॒ तव॑ वी॒र्ये॥७०॥\anuvakamend[च॒रा॒मि॒ कनी॑यो॒ऽन्यानर्पि॑ता प॒दानि॒ यज्व॑सु हवामहे वि॒ष्ठा लो॒काः सु॒वीर॒मर्वा॒ पिब॑न्ती॒ष्षट्च॑]

%2.8.9.1
ता सूर्याचन्द्र॒मसा॑ विश्व॒भृत्त॑मा म॒हत्। तेजो॒ वसु॑मद्राजतो दि॒वि। सामात्माना चरतः सामचा॒रिणा। ययोर्व्र॒तन्न म॒मे जातु॑ दे॒वयो। उ॒भावन्तौ॒ परि॑ यात॒ अर्म्या। दि॒वो न र॒श्मी स्त॑नु॒तो व्य॑र्ण॒वे। उ॒भा भु॑व॒न्ती भुव॑ना क॒विक्र॑तू। सूर्या॒ न च॒न्द्रा च॑रतो ह॒ताम॑ती। पती द्यु॒मद्वि॑श्व॒विदा॑ उ॒भा दि॒वः। सूर्या॑ उ॒भा च॒न्द्रम॑सा विचक्ष॒णा॥७१॥

%2.8.9.2
वि॒श्ववा॑रा वरिवो॒भा वरेण्या। ता नो॑ऽवतं मति॒मन्ता॒ महि॑व्रता। वि॒श्व॒वप॑री प्र॒तर॑णा तर॒न्ता। सु॒व॒र्विदा॑ दृ॒शये॒ भूरि॑रश्मी। सूर्या॒ हि च॒न्द्रा वसु॑ त्वे॒षद॑र्शता। म॒न॒स्विनो॒भानु॑चर॒तोनु॒ सन्दिवम्। अ॒स्य श्रवो॑ न॒द्य॑ स॒प्त बि॑भ्रति। द्यावा॒ क्षामा॑ पृथि॒वी द॑र्\mbox{}श॒तं वपु॑। अ॒स्मे सूर्याचन्द्र॒मसा॑ऽभि॒चक्षे। श्र॒द्धेकमि॑न्द्र चरतो विचर्तु॒रम्॥७२॥

%2.8.9.3
पू॒र्वा॒प॒रञ्च॑रतो मा॒ययै॒तौ। शिशू॒ क्रीड॑न्तौ॒ परि॑ यातो अध्व॒रम्। विश्वान्य॒न्यो भुव॑नाऽभि॒ चष्टे। ऋ॒तून॒न्यो वि॒दध॑ज्जायते॒ पुन॑। हिर॑ण्यवर्णा॒ शुच॑यः पाव॒का यासा॒ राजा। यासान्दे॒वाः शि॒वेन॑ मा॒ चक्षु॑षा पश्यत। आपो॑ भ॒द्रा आदित्प॑श्यामि। नास॑दासी॒न्नो सदा॑सीत्त॒दानीम्। नासी॒द्रजो॒ नो व्यो॑मा प॒रो यत्। किमाव॑रीव॒ कुह॒ कस्य॒ शर्म\sn{}॥७३॥

%2.8.9.4
अम्भ॒ किमा॑सी॒द्गह॑नङ्गभी॒रम्। न मृ॒त्युर॒मृत॒न्तर्\mbox{}हि॒ न। रात्रि॑या॒ अह्न॑ आसीत्प्रके॒तः। आनी॑दवा॒त स्व॒धया॒ तदेकम्। तस्माद्धा॒न्यन्न प॒रः किञ्च॒नास॑। तम॑ आसी॒त्तम॑सा गू॒ढमग्रे प्रके॒तम्। स॒लि॒ल सर्व॑मा इ॒दम्। तु॒च्छेना॒भ्वपि॑हितं॒ यदासीत्। तम॑स॒स्तन्म॑हि॒ना जा॑य॒तैकम्। काम॒स्तदग्रे॒ सम॑वर्त॒ताधि॑॥७४॥

%2.8.9.5
मन॑सो॒ रेत॑ प्रथ॒मं यदासीत्। स॒तो बन्धु॒मस॑ति॒ निर॑विन्दन्। हृ॒दि प्र॒तीष्या॑ क॒वयो॑ मनी॒षा। ति॒र॒श्चीनो॒ वित॑तो र॒श्मिरे॑षाम्। अ॒धः स्वि॑दा॒सी ३ दु॒परि॑ स्विदासी ३ त्। रे॒तो॒धा आ॑सन्महि॒मान॑ आसन्। स्व॒धा अ॒वस्ता॒त्प्रय॑तिः प॒रस्तात्। को अ॒द्धा वे॑द॒ क इ॒ह प्र वो॑चत्। कुत॒ आजा॑ता॒ कुत॑ इ॒यं विसृ॑ष्टिः। अ॒र्वाग्दे॒वा अ॒स्य वि॒सर्ज॑नाय॥७५॥

%2.8.9.6
अथा॒ को वे॑द॒ यत॑ आब॒भूव॑। इ॒यं विसृ॑ष्टि॒र्यत॑ आब॒भूव॑। यदि॑ वा द॒धे यदि॑ वा॒ न। यो अ॒स्याध्य॑क्षः पर॒मे व्यो॑मन्। सो अ॒ङ्ग वे॑द॒ यदि॑ वा॒ न वेद॑। किस्वि॒द्वन॒ङ्क उ॒ स वृ॒क्ष आ॑सीत्। यतो॒ द्यावा॑पृथि॒वी नि॑ष्टत॒क्षुः। मनी॑षिणो॒ मन॑सा पृ॒च्छतेदु॒तत्। यद॒ध्यति॑ष्ठ॒द्भुव॑नानि धा॒रय\sn{}। ब्रह्म॒ वनं॒ ब्रह्म॒ स वृ॒क्ष आ॑सीत्॥७६॥

%2.8.9.7
यतो॒ द्यावा॑पृथि॒वी नि॑ष्टत॒क्षुः। मनी॑षिणो॒ मन॑सा॒ विब्र॑वीमि वः। ब्रह्मा॒ध्यति॑ष्ठ॒द्भुव॑नानि धा॒रय\sn{}। प्रा॒तर॒ग्निं प्रा॒तरिन्द्र हवामहे। प्रा॒तर्मि॒त्रावरु॑णा प्रा॒तर॒श्विना। प्रा॒तर्भगं॑ पू॒षणं॒ ब्रह्म॑ण॒स्पतिम्। प्रा॒तः सोम॑मु॒त रु॒द्र हु॑वेम। प्रा॒त॒र्जितं॒ भग॑मु॒ग्र हु॑वेम। व॒यं पु॒त्रमदि॑ते॒र्यो वि॑ध॒र्ता। आ॒ध्रश्चि॒द्यं मन्य॑मानस्तु॒रश्चि॑त्॥७७॥

%2.8.9.8
राजा॑ चि॒द्यं भगं॑ भ॒क्षीत्याह॑। भग॒ प्रणे॑त॒र्भग॒ सत्य॑राधः। भगे॒मान्धिय॒मुद॑व॒ दद॑न्नः। भग॒ प्र णो॑ जनय॒ गोभि॒रश्वै। भग॒ प्र नृभि॑र्नृ॒वन्त॑ स्याम। उ॒तेदानीं॒ भग॑वन्तः स्याम। उ॒त प्रपि॒त्व उ॒त मध्ये॒ अह्नाम्। उ॒तोदि॑ता मघव॒न्त्सूर्य॑स्य। व॒यन्दे॒वाना सुम॒तौ स्या॑म। भग॑ ए॒व भग॑वा अस्तु देवाः॥७८॥

%2.8.9.9
तेन॑ व॒यं भग॑वन्तः स्याम। तन्त्वा॑ भग॒ सर्व॒ इज्जो॑हवीमि। स नो॑ भग पुरए॒ता भ॑वे॒ह। सम॑ध्व॒रायो॒षसो॑ नमन्त। द॒धि॒क्रावे॑व॒ शुच॑ये प॒दाय॑। अ॒र्वा॒ची॒नं व॑सु॒विदं॒ भग॑न्नः। रथ॑मि॒वाश्वा॑ वा॒जिन॒ आव॑हन्तु। अश्वा॑वती॒र्गोम॑तीर्न उ॒षास॑। वी॒रव॑ती॒ सद॑मुच्छन्तु भ॒द्राः। घृ॒तन्दुहा॑ना वि॒श्वत॒ प्रपी॑नाः। यू॒यं पा॑त स्व॒स्तिभि॒ सदा॑ नः॥७९॥\anuvakamend[वि॒च॒क्ष॒णा वि॑चर्तु॒र शर्म॒न्नधि॑ वि॒सर्ज॑नाय॒ ब्रह्म॒ वनं॒ ब्रह्म॒ स वृ॒क्ष आ॑सीत्तु॒रश्चि॑द्देवा॒ प्रपी॑ना॒ एकं च]
\prashnaend{पीवोन्ना॒न्ते शु॒क्रास॒ सोमो॑ धे॒नुमिन्द्र॒स्तर॑स्वा॒ञ्छुचि॒मा दे॒वो या॑तु॒ सूर्यो॑ दे॒वीम॒हम॑स्मि॒ ता सूर्याचन्द्र॒मसा॒ नव॑॥९॥}{पीवोन्ना॒मग्ने॒ त्वं पा॑रयानाधृ॒ष्यः शुचि॒न्नु वि॒श्रय॑माणो दि॒वो रु॒क्मोऽन्नं॑ प्रा॒णमन्न॒न्ता सूर्याचन्द्र॒मसा॒ नव॑सप्ततिः॥७९॥}{पीवोन्नाय्यूँ॒यं पा॑त स्व॒स्तिभि॒ सदा॑ नः॥}{हरि॑ ओम्॥}{इति श्रीकृष्णयजुर्वेदीयतैत्तिरीयब्राह्मणे द्वितीयाष्टके अष्टमः प्रपाठकः समाप्तः॥}
\clearpage
%%% END ASHTAKAM
