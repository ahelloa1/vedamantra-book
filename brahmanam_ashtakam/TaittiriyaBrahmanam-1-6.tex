\sect{षष्ठमः प्रश्नः}
\setcounter{anuvakam}{0}
\dnsub{तैत्तिरीयब्राह्मणे प्रथमाष्टके षष्ठः प्रपाठकः}

%1.6.1.1
अनु॑मत्यै पुरो॒डाश॑म॒ष्टाक॑पालं॒ निर्व॑पति। ये प्र॒त्यञ्च॒ शम्या॑या अव॒शीय॑न्ते। तन्नैर्॑ऋ॒तमेक॑कपालम्। इ॒यं वा अनु॑मतिः। इ॒यं निर्\mbox{}ऋ॑तिः। नै॒र्॒ऋ॒तेन॒ पूर्वे॑ण॒ प्रच॑रति। पा॒प्मान॑मे॒व निर्\mbox{}ऋ॑तिं॒ पूर्वां नि॒रव॑दयते। एक॑कपालो भवति। ए॒क॒धैव निर्\mbox{}ऋ॑तिं नि॒रव॑दयते। यदहु॑त्वा॒ गार्\mbox{}ह॑पत्य ई॒युः॥१॥

%1.6.1.2
रु॒द्रो भू॒त्वाऽग्निर॑नू॒त्थाय॑। अ॒ध्व॒र्युं च॒ यज॑मानं च हन्यात्। वीहि॒ स्वाहाऽऽहु॑तिं जुषा॒ण इत्या॑ह। आहु॑त्यै॒वैन शमयति। नार्ति॒मार्च्छ॑त्यध्व॒र्युर्न यज॑मानः। ए॒को॒ल्मु॒केन॑ यन्ति। तद्धि निर्\mbox{}ऋ॑त्यै भाग॒धेयम्। इ॒मान्दिशं॑ यन्ति। ए॒षा वै निर्\mbox{}ऋ॑त्यै॒ दिक्। स्वाया॑मे॒व दि॒शि निर्\mbox{}ऋ॑तिं नि॒रव॑दयते॥२॥

%1.6.1.3
स्वकृ॑त॒ इरि॑णे जुहोति प्रद॒रे वा। ए॒तद्वै निर्\mbox{}ऋ॑त्या आ॒यत॑नम्। स्व ए॒वायत॑ने॒ निर्\mbox{}ऋ॑तिं नि॒रव॑दयते। ए॒ष ते॑ निर्‌ऋते भा॒ग इत्या॑ह। निर्दि॑शत्ये॒वैनाम्। भूते॑ ह॒विष्म॑त्य॒सीत्या॑ह। भूति॑मे॒वोपाव॑र्तते। मु॒ञ्चेममह॑स॒ इत्या॑ह। अह॑स ए॒वैनं॑ मुञ्चति। अ॒ङ्गु॒ष्ठाभ्यां जुहोति॥३॥

%1.6.1.4
अ॒न्त॒त ए॒व निर्\mbox{}ऋ॑तिं नि॒रव॑दयते। कृ॒ष्णं वास॑ कृ॒ष्णतू॑ष॒न्दक्षि॑णा। ए॒तद्वै निर्\mbox{}ऋ॑त्यै रू॒पम्। रू॒पेणै॒व निर्\mbox{}ऋ॑तिं नि॒रव॑दयते। अप्र॑तीक्ष॒माय॑न्ति। निर्\mbox{}ऋ॑त्या अ॒न्तर्\mbox{}हि॑त्यै। स्वाहा॒ नमो॒ य इ॒दञ्च॒कारेति॒ पुन॒रेत्य॒ गार्\mbox{}ह॑पत्ये जुहोति। आहु॑त्यै॒व न॑म॒स्यन्तो॒ गार्\mbox{}ह॑पत्यमु॒पाव॑र्तन्ते। आ॒नु॒म॒तेन॒ प्रच॑रति। इ॒यं वा अनु॑मतिः॥४॥

%1.6.1.5
इ॒यमे॒वास्मै॑ रा॒ज्यमनु॑ मन्यते। धे॒नुर्दक्षि॑णा। इ॒मामे॒व धे॒नुं कु॑रुते। आ॒दि॒त्यञ्च॒रुं निर्व॑पति। उ॒भयीष्वे॒व प्र॒जास्व॒भिषि॑च्यते। दैवी॑षु च॒ मानु॑षीषु च। वरो॒ दक्षि॑णा। वरो॒ हि रा॒ज्य समृ॑द्ध्यै। आ॒ग्ना॒वै॒ष्ण॒वमेका॑दशकपालं॒ निर्व॑पति। अ॒ग्निः सर्वा॑ दे॒वता॥५॥

%1.6.1.6
विष्णु॑र्य॒ज्ञः। दे॒वताश्चै॒व य॒ज्ञञ्चाव॑ रुन्धे। वा॒म॒नो व॒ही दक्षि॑णा। यद्व॒ही। तेनाग्ने॒यः। यद्वा॑म॒नः। तेन॑ वैष्ण॒वः समृ॑द्ध्यै। अ॒ग्नी॒षो॒मीय॒मेका॑दशकपालं॒ निर्व॑पति। अ॒ग्नीषोमाभ्यां॒ वा इन्द्रो॑ वृ॒त्रम॑ह॒न्निति॑। यद॑ग्नीषो॒मीय॒मेका॑दशकपालं नि॒र्वप॑ति॥६॥

%1.6.1.7
वार्त्र॑घ्नमे॒व विजि॑त्यै। हिर॑ण्य॒न्दक्षि॑णा॒ समृ॑द्ध्यै। इन्द्रो॑ वृ॒त्र ह॒त्वा। दे॒वता॑भिश्चेन्द्रि॒येण॑ च॒ व्यार्ध्यत। स ए॒तमैन्द्रा॒ग्नमेका॑दशकपालमपश्यत्। तन्निर॑वपत्। तेन॒ वै स दे॒वताश्चेन्द्रि॒यञ्चावा॑रुन्ध। यदैन्द्रा॒ग्नमेका॑दशकपालं नि॒र्वप॑ति। दे॒वताश्चै॒व तेनेन्द्रि॒यं च॒ यज॑मा॒नोऽव॑रुन्धे। ऋ॒ष॒भो व॒ही दक्षि॑णा॥७॥

%1.6.1.8
यद्व॒ही। तेनाग्ने॒यः। यदृ॑ष॒भः। तेनै॒न्द्रः समृ॑द्ध्यै। आ॒ग्ने॒यम॒ष्टाक॑पालं॒ निर्व॑पति। ऐ॒न्द्रन्दधि॑। यदाग्ने॒यो भव॑ति। अ॒ग्निर्वै य॑ज्ञमु॒खम्। य॒ज्ञ॒मु॒खमे॒वर्द्धिं॑ पु॒रस्ताद्धत्ते। यदै॒न्द्रन्दधि॑॥८॥

%1.6.1.9
इ॒न्द्रि॒यमे॒वाव॑रुन्धे। ऋ॒ष॒भो व॒ही दक्षि॑णा। यद्व॒ही। तेनाग्ने॒यः। यदृ॑ष॒भः। तेनै॒न्द्रः समृ॑द्ध्यै। याव॑ती॒र्वै प्र॒जा ओष॑धीना॒महु॑ताना॒माश्ञ\sn{}। ताः परा॑ऽभवन्। आ॒ग्र॒य॒णं भ॑वति हु॒ताद्या॑य। यज॑मान॒स्याप॑राभावाय॥९॥

%1.6.1.10
दे॒वा वा ओष॑धीष्वा॒जिम॑युः। ता इ॑न्द्रा॒ग्नी उद॑जयताम्। तावे॒तमैन्द्रा॒ग्नन्द्वाद॑शकपालं॒ निर॑वृणाताम्। यदैन्द्रा॒ग्नो भव॒त्युज्जि॑त्यै। द्वाद॑शकपालो भवति। द्वाद॑श॒ मासा संवत्स॒रः। सं॒व॒त्स॒रेणै॒वास्मा॒ अन्न॒मव॑रुन्धे। वै॒श्व॒दे॒वश्च॒रुर्भ॑वति। वै॒श्व॒दे॒वं वा अन्नम्। अन्न॑मे॒वास्मै स्वदयति॥१०॥

%1.6.1.11
प्र॒थ॒म॒जो व॒त्सो दक्षि॑णा॒ समृ॑द्ध्यै। सौ॒म्य श्या॑मा॒कं च॒रुं निर्व॑पति। सोमो॒ वा अ॑कृष्टप॒च्यस्य॒ राजा। अ॒कृ॒ष्ट॒प॒च्यमे॒वास्मै स्वदयति। वासो॒ दक्षि॑णा। सौ॒म्य हि दे॒वत॑या॒ वास॒ समृ॑द्ध्यै। सर॑स्वत्यै च॒रुं निर्व॑पति। सर॑स्वते च॒रुम्। मि॒थु॒नमे॒वाव॑ रुन्धे। मि॒थु॒नौ गावौ॒ दक्षि॑णा॒ समृ॑द्ध्यै। एति॒ वा ए॒ष य॑ज्ञमु॒खादृध्या। योऽग्नेर्दे॒वता॑या॒ एति॑। अ॒ष्टावे॒तानि॑ ह॒वीषि॑ भवन्ति। अ॒ष्टाक्ष॑रा गाय॒त्री। गा॒य॒त्रोऽग्निः। तेनै॒व य॑ज्ञमु॒खादृध्या॑ अ॒ग्नेर्दे॒वता॑यै॒ नैति॑॥११॥\anuvakamend[ई॒युर्नि॒रव॑दयतेऽङ्गु॒ष्ठाभ्यां जुहो॒त्यनु॑मतिर्दे॒वता॑ नि॒र्वप॑ति व॒ही दक्षि॑णा॒ यदै॒न्द्रन्दध्यप॑राभावाय स्वदयति॒ गावौ॒ दक्षि॑णा॒ समृ॑द्ध्यै॒ षट्च॑]

%1.6.2.1
वै॒श्व॒दे॒वेन॒ वै प्र॒जाप॑तिः प्र॒जा अ॑सृजत। ताः सृ॒ष्टा न प्राजा॑यन्त। सोऽग्निर॑कामयत। अ॒हमि॒माः प्रज॑नयेय॒मिति॑। स प्र॒जाप॑तये॒ शुच॑मदधात्। सो॑ऽशोचत्प्र॒जामि॒च्छमा॑नः। तस्मा॒द्यञ्च॑ प्र॒जा भु॒नक्ति॒ यं च॒ न। तावु॒भौ शो॑चतः प्र॒जामि॒च्छमा॑नौ। तास्व॒ग्निमप्य॑सृजत्। ता अ॒ग्निरध्यैत्॥१२॥

%1.6.2.2
सोमो॒ रेतो॑ऽदधात्। स॒वि॒ता प्राज॑नयत्। सर॑स्वती॒ वाच॑मदधात्। पू॒षाऽपो॑षयत्। ते वा ए॒ते त्रिः सं॑वत्स॒रस्य॒ प्रयु॑ज्यन्ते। ये दे॒वाः पुष्टि॑पतयः। सं॒व॒त्स॒रो वै प्र॒जाप॑तिः। सं॒व॒त्स॒रेणै॒वास्मै प्र॒जाः प्राज॑नयत्। ताः प्र॒जा जा॒ता म॒रुतोऽघ्नन्। अ॒स्मानपि॒ न प्रायु॑क्ष॒तेति॑॥१३॥

%1.6.2.3
स ए॒तं प्र॒जाप॑तिर्मारु॒त स॒प्तक॑पालमपश्यत्। तन्निर॑वपत्। ततो॒ वै प्र॒जाभ्यो॑ऽकल्पत। यन्मा॑रु॒तो नि॑रु॒प्यते। य॒ज्ञस्य॒ कॢप्त्यै। प्र॒जाना॒मघा॑ताय। स॒प्तक॑पालो भवति। स॒प्तग॑णा॒ वै म॒रुत॑। ग॒ण॒श ए॒वास्मै॒ विश॑ङ्कल्पयति। स प्र॒जाप॑तिरशोचत्॥१४॥

%1.6.2.4
याः पूर्वा प्र॒जा असृ॑क्षि। म॒रुत॒स्ता अ॑वधिषुः। क॒थमप॑राः सृजे॒येति॑। तस्य॒ शुष्म॑ आ॒ण्डं भू॒तं निर॑वर्तत। तद्व्युद॑हरत्। तद॑पोषयत्। तत्प्राजा॑यत। आ॒ण्डस्य॒ वा ए॒तद्रू॒पम्। यदा॒मिक्षा। यद्व्यु॒द्धर॑ति॥१५॥

%1.6.2.5
प्र॒जा ए॒व तद्यज॑मानः पोषयति। वै॒श्व॒दे॒व्या॑मिक्षा॑ भवति। वै॒श्व॒दे॒व्यो॑ वै प्र॒जाः। प्र॒जा ए॒वास्मै॒ प्रज॑नयति। वाजि॑न॒मान॑यति। प्र॒जास्वे॒व प्रजा॑तासु॒ रेतो॑ दधाति। द्या॒वा॒पृ॒थि॒व्य॑ एक॑कपालो भवति। प्र॒जा ए॒व प्रजा॑ता॒ द्यावा॑पृथि॒वीभ्या॑मुभ॒यत॒ परि॑ गृह्णाति। दे॒वा॒सु॒राः संय॑त्ता आसन्। सोऽग्निर॑ब्रवीत्॥१६॥

%1.6.2.6
मामग्रे॑ यजत। मया॒ मुखे॒नासु॑राञ्जेष्य॒थेति॑। मां द्वि॒तीय॒मिति॒ सोमोऽब्रवीत्। मया॒ राज्ञा॑ जेष्य॒थेति॑। मान्तृ॒तीय॒मिति॑ सवि॒ता। मया॒ प्रसू॑ता जेष्य॒थेति॑। माञ्च॑तु॒र्थीमिति॒ सर॑स्वती। इ॒न्द्रि॒यं वो॒ऽहं धास्या॒मीति॑। मां प॑ञ्च॒ममिति॑ पू॒षा। मया प्रति॒ष्ठया॑ जेष्य॒थेति॑॥१७॥

%1.6.2.7
तेऽग्निना॒ मुखे॒नासु॑रानजयन्। सोमे॑न॒ राज्ञा। स॒वि॒त्रा प्रसू॑ताः। सर॑स्वतीन्द्रि॒यम॑दधात्। पू॒षा प्र॑ति॒ष्ठाऽऽसीत्। ततो॒ वै दे॒वा व्य॑जयन्त। यदे॒तानि॑ ह॒वीषि॑ निरु॒प्यन्ते॒ विजि॑त्यै। नोत्त॑रवे॒दिमुप॑वपति। प॒शवो॒ वा उ॑त्तरवे॒दिः। अजा॑ता इव॒ ह्ये॑तर्\mbox{}हि॑ प॒शव॑॥१८॥\anuvakamend[ऐ॒दित्य॑शोचद्व्यु॒द्धर॑त्यब्रवीत्प्रति॒ष्ठया॑ जेष्य॒थेत्ये॒तर्\mbox{}हि॑ प॒शव॑]

%1.6.3.1
त्रि॒वृद्ब॒र्॒हिर्भ॑वति। मा॒ता पि॒ता पु॒त्रः। तदे॒व तन्मि॑थु॒नम्। उल्ब॒ङ्गर्भो॑ ज॒रायु॑। तदे॒व तन्मि॑थु॒नम्। त्रे॒धा ब॒र्॒हिः सन्न॑द्धं भवति। त्रय॑ इ॒मे लो॒काः। ए॒ष्वे॑व लो॒केषु॒ प्रति॑ तिष्ठति। ए॒क॒धा पुन॒ सन्न॑द्धं भवति। एक॑ इव॒ ह्य॑यं लो॒कः॥१९॥

%1.6.3.2
अ॒स्मिन्ने॒व तेन॑ लो॒के प्रति॑तिष्ठति। प्र॒सुवो॑ भवन्ति। प्र॒थ॒म॒जामे॒व पुष्टि॒मव॑रुन्धे। प्र॒थ॒म॒जो व॒त्सो दक्षि॑णा॒ समृ॑द्ध्यै। पृ॒ष॒दा॒ज्यं गृ॑ह्णाति। प॒शवो॒ वै पृ॑षदा॒ज्यम्। प॒शूने॒वाव॑ रुन्धे। प॒ञ्च॒गृ॒ही॒तं भ॑वति। पाङ्क्ता॒ हि प॒शव॑। ब॒हु॒रू॒पं भ॑वति॥२०॥

%1.6.3.3
ब॒हु॒रू॒पा हि प॒शव॒ समृ॑द्ध्यै। अ॒ग्निं म॑न्थन्ति। अ॒ग्निमु॑खा॒ वै प्र॒जाप॑तिः प्र॒जा अ॑सृजत। यद॒ग्निं मन्थ॑न्ति। अ॒ग्निमु॑खा ए॒व तत्प्र॒जा यज॑मानः सृजते। नव॑ प्रया॒जा इ॑ज्यन्ते। नवा॑नूया॒जाः। अ॒ष्टौ ह॒वीषि॑। द्वावा॑घा॒रौ। द्वावाज्य॑भागौ॥२१॥

%1.6.3.4
त्रि॒शत्सम्प॑द्यन्ते। त्रि॒शद॑क्षरा वि॒राट्। अन्नं॑ वि॒राट्। वि॒राजै॒वान्नाद्य॒मव॑रुन्धे। यज॑मानो॒ वा एक॑कपालः। तेज॒ आज्यम्। यदेक॑कपाल॒ आज्य॑मा॒नय॑ति। यज॑मानमे॒व तेज॑सा॒ सम॑र्धयति। यज॑मानो॒ वा एक॑कपालः। प॒शव॒ आज्यम्॥२२॥

%1.6.3.5
यदेक॑कपाल॒ आज्य॑मा॒नय॑ति। यज॑मानमे॒व प॒शुभि॒ सम॑र्धयति। यदल्प॑मा॒नयेत्। अल्पा॑ एनं प॒शवो॑ भु॒ञ्जन्त॒ उप॑तिष्ठेरन्। यद्ब॒ह्वा॑नयेत्। ब॒हव॑ एनं प॒शवोऽभु॑ञ्जन्त॒ उप॑तिष्ठेरन्। ब॒ह्वा॑नीया॒विः पृ॑ष्ठं कुर्यात्। ब॒हव॑ ए॒वैनं॑ प॒शवो॑ भु॒ञ्जन्त॒ उप॑तिष्ठन्ते। यज॑मानो॒ वा एक॑कपालः। यदेक॑कपालस्याव॒द्येत्॥२३॥

%1.6.3.6
यज॑मान॒स्याव॑द्येत्। उद्वा॒ माद्ये॒द्यज॑मानः। प्र वा॑ मीयेत। स॒कृदे॒व हो॑त॒व्य॑। स॒कृदि॑व॒ हि सु॑व॒र्गो लो॒कः। हु॒त्वाऽभि जु॑होति। यज॑मानमे॒व सु॑व॒र्गं लो॒कं ग॑मयि॒त्वा। तेज॑सा॒ सम॑र्धयति। यज॑मानो॒ वा एक॑कपालः। सु॒व॒र्गो लो॒क आ॑हव॒नीय॑॥२४॥

%1.6.3.7
यदेक॑कपालमाहव॒नीये॑ जु॒होति॑। यज॑मानमे॒व सु॑व॒र्गं लो॒कं ग॑मयति। यद्धस्ते॑न जुहु॒यात्। सु॒व॒र्गाल्लो॒काद्यज॑मान॒मव॑विध्येत्। स्रु॒चा जु॑होति। सु॒व॒र्गस्य॑ लो॒कस्य॒ सम॑ष्ट्यै। यत्प्राङ्पद्ये॑त। दे॒व॒लो॒कम॒भिज॑येत्। यद्द॑क्षि॒णा पि॑तृलो॒कम्। यत्प्र॒त्यक्॥२५॥

%1.6.3.8
रक्षासि य॒ज्ञ ह॑न्युः। यदुदङ्ङ्॑। म॒नु॒ष्य॒लो॒कम॒भिज॑येत्। प्रति॑ष्ठितो होत॒व्य॑। एक॑कपालं॒ वै प्र॑ति॒तिष्ठ॑न्त॒न्द्यावा॑पृथि॒वी अनु॒ प्रति॑तिष्ठतः। द्यावा॑पृथि॒वी ऋ॒तव॑। ऋ॒तून् य॒ज्ञः। य॒ज्ञं यज॑मानः। यज॑मानं प्र॒जाः। तस्मा॒त्प्रति॑ष्ठितो होत॒व्य॑॥२६॥

%1.6.3.9
वा॒जिनो॑ यजति। अ॒ग्निर्वा॒युः सूर्य॑। ते वै वा॒जिन॑। ताने॒व तद्य॑जति। अथो॒ खल्वा॑हुः। छन्दासि॒ वै वा॒जिन॒ इति॑। तान्ये॒व तद्य॑जति। ऋ॒ख्सा॒मे वा इन्द्र॑स्य॒ हरी॑ सोम॒पानौ। तयो परि॒धय॑ आ॒धानम्। वाजि॑नं भाग॒धेयम्॥२७॥

%1.6.3.10
यदप्र॑हृत्य परि॒धीं जु॑हु॒यात्। अ॒न्तरा॑धानाभ्याङ्घा॒सं प्रय॑च्छेत्। प्र॒हृत्य॑ परि॒धीं जु॑होति। निरा॑धानाभ्यामे॒व घा॒सं प्रय॑च्छति। ब॒र्॒हिषि॑ विषि॒ञ्चन्वाजि॑न॒मा न॑यति। प्र॒जा वै ब॒र्॒हिः। रेतो॒ वाजि॑नम्। प्र॒जास्वे॒व रेतो॑ दधाति। स॒मु॒प॒हूय॑ भक्षयन्ति। ए॒तत्सो॑मपीथा॒ ह्ये॑ते। अथो॑ आ॒त्मन्ने॒व रेतो॑ दधते। यज॑मान उत्त॒मो भ॑क्षयति। प॒शवो॒ वै वाजि॑नम्। यज॑मान ए॒व प॒शून्प्रति॑ष्ठापयन्ति॥२८॥\anuvakamend[लो॒को ब॑हुरू॒पं भ॑व॒त्याज्य॑भागौ प॒शव॒ आज्य॑मव॒द्येदा॑हव॒नीय॑ प्र॒त्यक्तस्मा॒त्प्रति॑ष्ठितो होत॒व्यो॑ भाग॒धेय॑मे॒ते च॒त्वारि॑ च]

%1.6.4.1
प्र॒जाप॑तिः सवि॒ता भू॒त्वा प्र॒जा अ॑सृजत। ता ए॑न॒मत्य॑मन्यन्त। ता अ॑स्मा॒दपाक्रामन्। ता वरु॑णो भू॒त्वा प्र॒जा वरु॑णेनाग्राहयत्। ताः प्र॒जा वरु॑णगृहीताः। प्र॒जाप॑तिं॒ पुन॒रुपा॑धावन्ना॒थमि॒च्छमा॑नाः। स ए॒तान्प्र॒जाप॑तिर्वरुणप्रघा॒सान॑पश्यत्। तां निर॑वपत्। तैर्वै स प्र॒जा व॑रुणपा॒शाद॑मुञ्चत्। यद्व॑रुणप्रघा॒सा नि॑रु॒प्यन्ते॥२९॥

%1.6.4.2
प्र॒जाना॒मव॑रुणग्राहाय। तासा॒न्दक्षि॑णो बा॒हुर्न्य॑क्न॒ आसीत्। स॒व्यः प्रसृ॑तः। स ए॒तां द्वि॒तीयान्दक्षिण॒तो वेदि॒मुद॑हन्। ततो॒ वै स प्र॒जाना॒न्दक्षि॑णं बा॒हुं प्रासा॑रयत्। यद्द्वि॒तीयान्दक्षिण॒तो वेदि॑मु॒द्धन्ति॑। प्र॒जाना॑मे॒व तद्यज॑मानो॒ दक्षि॑णं बा॒हुं प्रसा॑रयति। तस्माच्चातुर्मास्यया॒ज्य॑मुष्मि॑ल्लोँ॒क उ॑भ॒याबा॑हुः। य॒ज्ञाभि॑जित॒ ह्य॑स्य। पृ॒थ॒मा॒त्राद्वैदी॒ अस॑म्भिन्ने भवतः॥३०॥

%1.6.4.3
तस्मात्पृथमा॒त्रं व्यसौ। उत्त॑रस्यां॒ वेद्या॑मुत्तरवे॒दिमुप॑ वपति। प॒शवो॒ वा उ॑त्तरवे॒दिः। प॒शूने॒वाव॑रुन्धे। अथो॑ यज्ञप॒रुषोऽन॑न्तरित्यै। ए॒तद्ब्राह्मणान्ये॒व पञ्च॑ ह॒वीषि॑। अथै॒ष ऐन्द्रा॒ग्नो भ॑वति। प्रा॒णा॒पा॒नौ वा ए॒तौ दे॒वानाम्। यदि॑न्द्रा॒ग्नी। यदैन्द्रा॒ग्नो भव॑ति॥३१॥

%1.6.4.4
प्रा॒णा॒पा॒नावे॒वाव॑ रुन्धे। ओजो॒ बलं॒ वा ए॒तौ दै॒वानाम्। यदि॑न्द्रा॒ग्नी। यदैन्द्रा॒ग्नो भव॑ति। ओजो॒ बल॑मे॒वाव॑ रुन्धे। मा॒रु॒त्या॑मिक्षा॑ भवति। वा॒रु॒ण्या॑मिक्षा। मे॒षी च॑ मे॒षश्च॑ भवतः। मि॒थु॒ना ए॒व प्र॒जा व॑रुणपा॒शान्मु॑ञ्चति। लो॒म॒शौ भ॑वतो मेध्य॒त्वाय॑॥३२॥

%1.6.4.5
श॒मी॒प॒र्णान्युप॑ वपति। घा॒समे॒वाभ्या॒मपि॑ यच्छति। प्र॒जाप॑तिम॒न्नाद्य॒न्नोपा॑नमत्। स ए॒तेन॑ श॒तेध्मे॑न ह॒विषा॒ऽन्नाद्य॒मवा॑रुन्ध। यत्प॑रः श॒तानि॑ शमीप॒र्णानि॒ भव॑न्ति। अ॒न्नाद्य॒स्याव॑रुद्ध्यै। सौ॒म्यानि॒ वै क॒रीरा॑णि। सौ॒म्या खलु॒ वा आहु॑तिर्दि॒वो वृष्टि॑ञ्च्यावयति। यत्क॒रीरा॑णि॒ भव॑न्ति। सौ॒म्ययै॒वाहु॑त्या दि॒वो वृष्टि॒मव॑रुन्धे। का॒य एक॑कपालो भवति। प्र॒जानाङ्क॒न्त्वाय॑। प्र॒ति॒पू॒रु॒षङ्क॑रम्भपा॒त्राणि॑ भवन्ति। जा॒ता ए॒व प्र॒जा व॑रुणपा॒शान्मु॑ञ्चति। एक॒मति॑रिक्तम्। ज॒नि॒ष्यमा॑णा ए॒व प्र॒जा व॑रुणपा॒शान्मु॑ञ्चति॥३३॥\anuvakamend[नि॒रु॒प्यन्ते॑ भवतो॒ भव॑ति मेध्य॒त्वाय॑ रुन्धे॒ षट्च॑]

%1.6.5.1
उत्त॑रस्यां॒ वेद्या॑म॒न्यानि॑ ह॒वीषि॑ सादयति। दक्षि॑णायां मारु॒तीम्। अ॒प॒धु॒रमे॒वैना॑ युनक्ति। अथो॒ ओज॑ ए॒वासा॒मव॑ हरति। तस्मा॒द्ब्रह्म॑णश्च क्ष॒त्राच्च॒ विशोऽन्यतोऽपक्र॒मिणी। मा॒रु॒त्या पूर्व॑या॒ प्रच॑रति। अनृ॑तमे॒वाव॑ यजते। वा॒रु॒ण्योत्त॑रया। अ॒न्त॒त ए॒व वरु॑ण॒मव॑ यजते। यदे॒वाध्व॒र्युः क॒रोति॑॥३४॥

%1.6.5.2
तत्प्र॑तिप्रस्था॒ता क॑रोति। तस्मा॒द्यच्छ्रेयान्क॒रोति॑। तत्पापी॑यान्करोति। पत्नीं वाचयति। मेध्या॑मे॒वैनां करोति। अथो॒ तप॑ ए॒वैना॒मुप॑ नयति। यज्जा॒र सन्त॒न्न प्र॑ब्रू॒यात्। प्रि॒यं ज्ञा॒ति रु॑न्ध्यात्। अ॒सौ मे॑ जा॒र इति॒ निर्दि॑शेत्। नि॒र्दिश्यै॒वैनं॑ वरुणपा॒शेन॑ ग्राहयति॥३५॥

%1.6.5.3
प्र॒घा॒स्यान्॑ हवामह॒ इति॒ पत्नी॑मु॒दान॑यति। अह्व॑तै॒वैनाम्। यत्पत्नी॑ पुरोनुवा॒क्या॑मनुब्रू॒यात्। निर्वीर्यो॒ यज॑मानः स्यात्। यज॑मा॒नोऽन्वा॑ह। आ॒त्मन्ने॒व वी॒र्य॑न्धत्ते। उ॒भौ या॒ज्या सवीर्य॒त्वाय॑। यद्ग्रामे॒ यदर॑ण्य॒ इत्या॑ह। य॒थो॒दि॒तमे॒व वरु॑ण॒मव॑ यजते। य॒ज॒मा॒न॒दे॒व॒त्यो॑ वा आ॑हव॒नीय॑॥३६॥

%1.6.5.4
भ्रा॒तृ॒व्य॒दे॒व॒त्यो॑ दक्षि॑णः। यदा॑हव॒नीये॑ जुहु॒यात्। यज॑मानं वरुणपा॒शेन॑ ग्राहयेत्। दक्षि॑णे॒ऽग्नौ जु॑होति। भ्रातृ॑व्यमे॒व व॑रुणपा॒शेन॑ ग्राहयति। शूर्पे॑ण जुहोति। अन्य॑मे॒व वरु॑ण॒मव॑ यजते। शी॒र्॒षन्न॑धि नि॒धाय॑ जुहोति। शी॒र्\mbox{}ष॒त ए॒व वरु॑ण॒मव॑ यजते। प्र॒त्यङ्तिष्ठं॑ जुहोति॥३७॥

%1.6.5.5
प्र॒त्यङ्ङे॒व व॑रुणपा॒शान्निर्मु॑च्यते। अक्र॒न्कर्म॑ कर्म॒कृत॒ इत्या॑ह। दे॒वाऽनृ॒णं नि॑रव॒दाय॑। अ॒नृ॒णा गृ॒हानुप॒ प्रेतेति॒ वावैतदा॑ह। वरु॑णगृहीतं॒ वा ए॒तद्य॒ज्ञस्य॑। यद्यजु॑षा गृही॒तस्या॑ति॒रिच्य॑ते। तुषाश्च निष्का॒सश्च॑। तुषैश्च निष्का॒सेन॑ चावभृ॒थमवै॑ति। वरु॑णगृहीतेनै॒व वरु॑ण॒मव॑यजते। अ॒पो॑ऽवभृ॒थमवै॑ति॥३८॥

%1.6.5.6
अ॒प्सु वै वरु॑णः। सा॒क्षादे॒व वरु॑ण॒मव॑यजते। प्रति॑युतो॒ वरु॑णस्य॒ पाश॒ इत्या॑ह। व॒रु॒ण॒पा॒शादे॒व निर्मु॑च्यते। अप्र॑तीक्ष॒मा य॑न्ति। वरु॑णस्या॒न्तर्\mbox{}हि॑त्यै। एधोऽष्येधिषी॒महीत्या॑ह। स॒मिधै॒वाग्निन्न॑म॒स्यन्त॑ उ॒पाय॑न्ति। तेजो॑ऽसि॒ तेजो॒ मयि॑ धे॒हीत्या॑ह। तेज॑ ए॒वात्मन्ध॑त्ते॥३९॥\anuvakamend[क॒रोति॑ ग्राहयत्याहव॒नीय॒स्तिष्ठं॑ जुहोत्य॒पो॑ऽवभृ॒थमवै॑ति धत्ते]

%1.6.6.1
दे॒वा॒सु॒राः संय॑त्ता आसन्। सोऽग्निर॑ब्रवीत्। ममे॒यमनी॑कवती त॒नूः। तां प्री॑णीत। अथासु॑रान॒भि भ॑विष्य॒थेति॑। ते दे॒वा अ॒ग्नयेऽनी॑कवते पुरो॒डाश॑म॒ष्टाक॑पालं॒ निर॑वपन्। सोऽग्निरनी॑कवा॒न्त्स्वेन॑ भाग॒धेये॑न प्री॒तः। च॒तु॒र्धाऽनी॑कान्यजनयत। ततो॑ दे॒वा अभ॑वन्। पराऽसु॑राः॥४०॥

%1.6.6.2
यद॒ग्नयेऽनी॑कवते पुरो॒डाश॑म॒ष्टाक॑पालं नि॒र्वप॑ति। अ॒ग्निमे॒वानी॑कवन्त॒ स्वेन॑ भाग॒धेये॑न प्रीणाति। सोऽग्निरनी॑कवा॒न्त्स्वेन॑ भाग॒धेये॑न प्री॒तः। च॒तु॒र्धाऽनी॑कानि जनयते। अ॒सौ वा आ॑दि॒त्योऽग्निरनी॑कवान्। तस्य॑ र॒श्मयोऽनी॑कानि। सा॒क सूर्ये॑णोद्य॒ता निर्व॑पति। सा॒क्षादे॒वास्मा॒ अनी॑कानि जनयति। तेऽसु॑रा॒ परा॑जिता॒ यन्त॑। द्यावा॑पृथि॒वी उपाश्रयन्॥४१॥

%1.6.6.3
ते दे॒वा म॒रुद्भ्य॑ सान्तप॒नेभ्य॑श्च॒रुं निर॑वपन्। यन्म॒रुद्भ्य॑ सान्तप॒नेभ्य॑श्च॒रुं नि॒र्वप॑ति। द्यावा॑पृथि॒वीभ्या॑मे॒व तदु॑भ॒यतो॒ यज॑मानो॒ भ्रातृ॑व्या॒न्त्सन्त॑पति। म॒ध्यन्दि॑ने॒ निर्व॑पति। तर्\mbox{}हि॒ हि तेक्ष्णि॑ष्ठ॒न्तप॑ति। च॒रुर्भ॑वति। स॒र्वत॑ ए॒वैना॒न्त्सन्त॑पति। ते दे॒वाः श्वो॑विज॒यिन॒ सन्त॑। सर्वा॑सान्दु॒ग्धे गृ॑हमे॒धीयं॑ च॒रुं निर॑वपन्॥४२॥

%1.6.6.4
आशि॑ता ए॒वाद्योप॑वसाम। कस्य॒ वाऽहे॒दम्। कस्य॑ वा॒ श्वो भ॑वि॒तेति॑। स शृ॒तो॑ऽभवत्। तस्याहु॑तस्य॒ नाश्ञ\sn{}। न हि दे॒वा अहु॑तस्या॒श्ञन्ति॑। तेऽब्रुवन्। कस्मा॑ इ॒म होष्याम॒ इति॑। म॒रुद्भ्यो॑ गृहमे॒धिभ्य॒ इत्य॑ब्रुवन्। तं म॒रुद्भ्यो॑ गृहमे॒धिभ्यो॑ऽजुहवुः॥४३॥

%1.6.6.5
ततो॑ दे॒वा अभ॑वन्। पराऽसु॑राः। यस्यै॒वं वि॒दुषो॑ म॒रुद्भ्यो॑ गृहमे॒धिभ्यो॑ गृ॒हे जुह्व॑ति। भव॑त्या॒त्मना। पराऽस्य॒ भ्रातृ॑व्यो भवति। यद्वै य॒ज्ञस्य॑ पाक॒त्रा क्रि॒यते। प॒श॒व्य॑न्तत्। पा॒क॒त्रा वा ए॒तत्क्रि॑यते। यन्नेध्माब॒र्॒हिर्भव॑ति। न सा॑मिधे॒नीर॒न्वाह॑॥४४॥

%1.6.6.6
न प्र॑या॒जा इ॒ज्यन्ते। नानू॑या॒जाः। य ए॒वं वेद॑। प॒शु॒मान्भ॑वति। आज्य॑भागौ यजति। य॒ज्ञस्यै॒व चक्षु॑षी॒ नान्तरे॑ति। म॒रुतो॑ गृहमे॒धिनो॑ यजति। भा॒ग॒धेये॑नै॒वैना॒न्त्सम॑र्धयति। अ॒ग्निस्वि॑ष्ट॒कृतं॑ यजति॒ प्रति॑ष्ठित्यै। इडान्तो भवति। प॒शवो॒ वा इडा। प॒शुष्वे॒वोपरि॑ष्टा॒त्प्रति॑तिष्ठति॥४५॥\anuvakamend[असु॑रा अश्रयन्गृहमे॒धीयं॑ च॒रुं निर॑वपन्नजुहवुर॒न्वाहेडान्तो भवति॒ द्वे च॑]

%1.6.7.1
यत्पत्नी॑ गृहमे॒धीय॑स्याश्ञी॒यात्। गृ॒ह॒मे॒ध्ये॑व स्यात्। वि त्व॑स्य य॒ज्ञ ऋ॑ध्येत। यन्नाश्ञी॒यात्। अगृ॑हमेधी स्यात्। नास्य॑ य॒ज्ञो व्यृ॑द्ध्येत। प्रति॑वेशं पचेयुः। तस्याश्ञीयात्। गृ॒ह॒मे॒ध्ये॑व भ॑वति। नास्य॑ य॒ज्ञो व्यृ॑द्ध्यते॥४६॥

%1.6.7.2
ते दे॒वा गृ॑हमे॒धीये॑ने॒ष्ट्वा। आशि॑ता अभवन्। आञ्ज॑ता॒भ्य॑ञ्जत। अनु॑ व॒त्सान॑वासयन्। तेभ्योऽसु॑रा॒ क्षुधं॒ प्राहि॑ण्वन्। सा दे॒वेषु॑ लो॒कमवि॑त्वा। असु॑रा॒न्पुन॑रगच्छत्। गृ॒ह॒मे॒धीये॑ने॒ष्ट्वा। आशि॑ता भवन्ति। आञ्ज॑ते॒ऽभ्य॑ञ्जते॥४७॥

%1.6.7.3
अनु॑ व॒त्सान् वा॑सयन्ति। भ्रातृ॑व्यायै॒व तद्यज॑मान॒ क्षुधं॒ प्रहि॑णोति। ते दे॒वा गृ॑हमे॒धीये॑ने॒ष्ट्वा। इन्द्रा॑य निष्का॒सन्न्य॑दधुः। अ॒स्माने॒व श्व इन्द्रो॒ निहि॑तभाग उपावर्ति॒तेति॑। तानिन्द्रो॒ निहि॑तभाग उ॒पाव॑र्तत। गृ॒ह॒मे॒धीये॑ने॒ष्ट्वा। इन्द्रा॑य निष्का॒सं निद॑ध्यात्। इन्द्र॑ ए॒वैनं॒ निहि॑तभाग उ॒पाव॑र्तते। गार्\mbox{}ह॑पत्ये जुहोति॥४८॥

%1.6.7.4
भा॒ग॒धेये॑नै॒वैन॒ सम॑र्धयति। ऋ॒ष॒भमाह्व॑यति। व॒ष॒ट्का॒र ए॒वास्य॒ सः। अथो॑ इन्द्रि॒यमे॒व तद्वी॒र्यं॑ यज॑मानो॒ भ्रातृव्य॑स्य वृङ्क्ते। इन्द्रो॑ वृ॒त्र ह॒त्वा। परां परा॒वत॑मगच्छत्। अपा॑राध॒मिति॒ मन्य॑मानः। सोऽब्रवीत्। क इ॒दं वे॑दिष्य॒तीति॑। तेऽब्रुवन्म॒रुतो॒ वरं॑ वृणामहै॥४९॥

%1.6.7.5
अथ॑ व॒यं वे॑दाम। अ॒स्मभ्य॑मे॒व प्र॑थ॒म ह॒विर्निरु॑प्याता॒ इति॑। त ए॑न॒मध्य॑क्रीडन्। तत्क्री॒डिनाङ्क्रीडि॒त्वम्। यन्म॒रुद्भ्य॑ क्री॒डिभ्य॑ प्रथ॒म ह॒विर्नि॑रु॒प्यते॒ विजि॑त्यै। सा॒क सूर्ये॑णोद्य॒ता निर्व॑पति। ए॒तस्मि॒न्वै लो॒क इन्द्रो॑ वृ॒त्रम॑ह॒न्त्समृ॑द्ध्यै। ए॒तद्ब्राह्मणान्ये॒व पञ्च॑ ह॒वीषि॑। ए॒तद्ब्राह्मण ऐन्द्रा॒ग्नः। अथै॒ष ऐ॒न्द्रश्च॒रुर्भ॑वति॥५०॥

%1.6.7.6
उ॒द्धारं वा ए॒तमिन्द्र॒ उद॑हरत। वृ॒त्र ह॒त्वा। अ॒न्यासु॑ दे॒वता॒स्वधि॑। यदे॒ष ऐ॒न्द्रश्च॒रुर्भव॑ति। उ॒द्धा॒रमे॒व तं यज॑मान॒ उद्ध॑रते। अ॒न्यासु॑ प्र॒जास्वधि॑। वै॒श्व॒क॒र्म॒ण एक॑कपालो भवति। विश्वान्ये॒व तेन॒ कर्मा॑णि॒ यज॑मा॒नोऽव॑रुन्धे॥५१॥\anuvakamend[ऋ॒द्ध्य॒ते॒ऽभ्य॑ञ्जते जुहोति वृणामहै भवत्य॒ष्टौ च॑]

%1.6.8.1
वै॒श्व॒दे॒वेन॒ वै प्र॒जाप॑तिः प्र॒जा अ॑सृजत। ता व॑रुणप्रघा॒सैर्व॑रुणपा॒शाद॑मुञ्चत्। सा॒क॒मे॒धैः प्रत्य॑स्थापयत्। त्र्य॑म्बकै रु॒द्रं नि॒रवा॑दयत। पि॒तृ॒य॒ज्ञेन॑ सुव॒र्गं लो॒कम॑गमयत्। यद्वैश्वदे॒वेन॒ यज॑ते। प्र॒जा ए॒व तद्यज॑मानः सृजते। ता व॑रुणप्रघा॒सैर्व॑रुणपा॒शान्मु॑ञ्चति। सा॒क॒मे॒धैः प्रति॑ष्ठापयति। त्र्य॑म्बकै रु॒द्रं नि॒रव॑दयते॥५२॥

%1.6.8.2
पि॒तृ॒य॒ज्ञेन॑ सुव॒र्गं लो॒कं ग॑मयति। द॒क्षि॒ण॒तः प्रा॑चीनावी॒ती निर्व॑पति। द॒क्षि॒णावृ॒द्धि पि॑तृ॒णाम्। अना॑दृत्य॒ तत्। उ॒त्त॒र॒त ए॒वोप॒वीय॒ निर्व॑पेत्। उ॒भये॒ हि दे॒वाश्च॑ पि॒तर॑श्चे॒ज्यन्ते। अथो॒ यदे॒व द॑क्षिणा॒र्धे॑ऽधि॒ श्रय॑ति। तेन॑ दक्षि॒णावृ॑त्। सोमा॑य पितृ॒मते॑ पुरो॒डाश॒ षट्क॑पालं॒ निर्व॑पति। सं॒व॒त्स॒रो वै सोम॑ पितृ॒मान्॥५३॥

%1.6.8.3
सं॒व॒त्स॒रमे॒व प्री॑णाति। पि॒तृभ्यो॑ बर्\mbox{}हि॒षद्भ्यो॑ धा॒नाः। मासा॒ वै पि॒तरो॑ बर्\mbox{}हि॒षद॑। मासा॑ने॒व प्री॑णाति। यस्मि॒न्वा ऋ॒तौ पुरु॑षः प्र॒मीय॑ते। सोऽस्या॒मुष्मि॑ल्लोँ॒के भ॑वति। ब॒हु॒रू॒पा धा॒ना भ॑वन्ति। अ॒हो॒रा॒त्राणा॑म॒भिजि॑त्यै। पि॒तृभ्योऽग्निष्वा॒त्तेभ्यो॑ म॒न्थम्। अ॒र्ध॒मा॒सा वै पि॒तरोऽग्निष्वा॒त्ताः॥५४॥

%1.6.8.4
अ॒र्ध॒मा॒साने॒व प्री॑णाति। अ॒भि॒वा॒न्या॑यै दु॒ग्धे भ॑वति। सा हि पि॑तृदेव॒त्य॑न्दु॒हे। यत्पू॒र्णम्। तन्म॑नु॒ष्या॑णाम्। उ॒प॒र्य॒र्धो दे॒वानाम्। अ॒र्धः पि॑तृ॒णाम्। अ॒र्ध उप॑मन्थति। अ॒र्धो हि पि॑तृ॒णाम्। एक॒योप॑मन्थति॥५५॥

%1.6.8.5
एका॒ हि पि॑तृ॒णाम्। द॒क्षि॒णोप॑मन्थति। द॒क्षि॒णावृ॒द्धि पि॑तृ॒णाम्। अना॑र॒भ्योप॑मन्थति। तद्धि पि॒तॄन्गच्छ॑ति। इ॒मान्दिशं॒ वेदि॒मुद्ध॑न्ति। उ॒भये॒ हि दे॒वाश्च॑ पि॒तर॑श्चे॒ज्यन्ते। चतु॑ स्रक्तिर्भवति। सर्वा॒ ह्यनु॒ दिश॑ पि॒तर॑। अखा॑ता भवति॥५६॥

%1.6.8.6
खा॒ता हि दे॒वानाम्। म॒ध्य॒तोऽग्निराधी॑यते। अ॒न्त॒तो हि दे॒वाना॑माधी॒यते। वर्\mbox{}षी॑यानि॒ध्म इ॒ध्माद्भ॑वति॒ व्यावृ॑त्त्यै। परि॑श्रयति। अ॒न्तर्\mbox{}हि॑तो॒ हि पि॑तृलो॒को म॑नुष्यलो॒कात्। यत्परु॑षि दि॒नम्। तद्दे॒वानाम्। यद॑न्त॒रा। तन्म॑नु॒ष्या॑णाम्॥५७॥

%1.6.8.7
यत्समू॑लम्। तत्पि॑तृ॒णाम्। समू॑लं ब॒र्॒हिर्भ॑वति॒ व्यावृ॑त्त्यै। द॒क्षि॒णा स्तृ॑णाति। द॒क्षि॒णावृ॒द्धि पि॑तृ॒णाम्। त्रिः पर्ये॑ति। तृ॒तीये॒ वा इ॒तो लो॒के पि॒तर॑। ताने॒व प्री॑णाति। त्रिः पुन॒ पर्ये॑ति। षट्त्सं प॑द्यन्ते॥५८॥

%1.6.8.8
षड्वा ऋ॒तव॑। ऋ॒तूने॒व प्री॑णाति। यत्प्र॑स्त॒रं यजु॑षा गृह्णी॒यात्। प्र॒मायु॑को॒ यज॑मानः स्यात्। यन्न गृ॑ह्णी॒यात्। अ॒ना॒य॒त॒नः स्यात्। तू॒ष्णीमे॒व न्य॑स्येत्। न प्र॒मायु॑को॒ भव॑ति। नाना॑यत॒नः। यत्रीन्प॑रि॒धीन्प॑रिद॒ध्यात्॥५९॥

%1.6.8.9
मृ॒त्युना॒ यज॑मानं॒ परि॑गृह्णीयात्। यन्न प॑रिद॒ध्यात्। रक्षासि य॒ज्ञ ह॑न्युः। द्वौ प॑रि॒धी परि॑दधाति। रक्ष॑सा॒मप॑हत्यै। अथो॑ मृ॒त्योरे॒व यज॑मान॒मुत्सृ॑जति। यत्रीणि॑ त्रीणि ह॒वीष्यु॑दा॒हरे॑युः। त्रय॑स्त्रय एषा सा॒कं प्रमी॑येरन्। एकै॑कमनू॒चीनान्यु॒दाह॑रन्ति। एकै॑क ए॒वैषा॑म॒न्वञ्च॒ प्रमी॑यते। क॒शिपु॑ कशिप॒व्या॑य। उ॒प॒बर्\mbox{}ह॑णमुपबर्\mbox{}ह॒ण्या॑य। आञ्ज॑नमाञ्ज॒न्या॑य। अ॒भ्यञ्ज॑नमभ्यञ्ज॒न्या॑य। य॒था॒भा॒गमे॒वैनान्प्रीणाति॥६०॥\anuvakamend[नि॒रव॑दयते पितृ॒मान॑ग्निष्वा॒त्ता एक॒योप॑ मन्थ॒त्यखा॑ता भवति मनु॒ष्या॑णां पद्यन्ते परिद॒ध्यान्मी॑यते॒ पञ्च॑ च]

%1.6.9.1
अ॒ग्नये॑ दे॒वेभ्य॑ पि॒तृभ्य॑ समि॒ध्यमा॑ना॒यानु॑ ब्रू॒हीत्या॑ह। उ॒भये॒ हि दै॒वाश्च॑ पि॒तर॑श्चे॒ज्यन्ते। एका॒मन्वा॑ह। एका॒ हि पि॑तृ॒णाम्। त्रिरन्वा॑ह। त्रिर्\mbox{}हि दे॒वानाम्। आ॒घा॒रावाघा॑रयति। य॒ज्ञ॒प॒रुषो॒रन॑न्तरित्यै। नार्\mbox{}षे॒यं वृ॑णीते। न होता॑रम्॥६१॥

%1.6.9.2
यदा॑र्\mbox{}षे॒यं वृ॑णी॒त। यद्धोता॑रम्। प्र॒मायु॑को॒ यज॑मानः स्यात्। प्र॒मायु॑को॒ होता। तस्मा॒न्न वृ॑णीते। यज॑मानस्य॒ होतु॑र्गोपी॒थाय॑। अप॑ बर्\mbox{}हिषः प्रया॒जान् य॑जति। प्र॒जा वै ब॒र्॒हिः। प्र॒जा ए॒व मृ॒त्योरुत्सृ॑जति। आज्य॑भागौ यजति॥६२॥

%1.6.9.3
य॒ज्ञस्यै॒व चक्षु॑षी॒ नान्तरे॑ति। प्रा॒ची॒ना॒वी॒ती सोमं॑ यजति। पि॒तृ॒दे॒व॒त्या॑ हि। ए॒षाऽऽहु॑तिः। पञ्च॒कृत्वोऽव॑ द्यति। पञ्च॒ ह्ये॑ता दे॒वता। द्वे पु॑रोऽनुवा॒क्ये। या॒ज्या॑ दे॒वता॑ वषट्का॒रः। ता ए॒व प्री॑णाति। सन्त॑त॒मव॑ द्यति॥६३॥

%1.6.9.4
ऋ॒तू॒ना सन्त॑त्यै। प्रैवैभ्य॒ पूर्व॑या पुरोऽनुवा॒क्य॑याऽऽह। प्रण॑यति द्वि॒तीय॑या। ग॒मय॑ति या॒ज्य॑या। तृ॒तीये॒ वा इ॒तो लो॒के पि॒तर॑। अह्न॑ ए॒वैना॒न्पूर्व॑या पुरोऽनुवा॒क्य॑या॒ऽत्यान॑यति। रात्रि॑यै द्वि॒तीय॑या। ऐवैनान्॑ या॒ज्य॑या गमयति। द॒क्षि॒ण॒तो॑ऽव॒दाय॑। उद॒ङ्ङति॑ क्रामति॒ व्यावृ॑त्त्यै॥६४॥

%1.6.9.5
आ स्व॒धेत्याश्रा॑वयति। अस्तु॑ स्व॒धेति॑ प्र॒त्याश्रा॑वयति। स्व॒धा नम॒ इति॒ वष॑ट्करोति। स्व॒धा॒का॒रो हि पि॑तृ॒णाम्। सोम॒मग्रे॑ यजति। सोम॑प्रयाजा॒ हि पि॒तर॑। सोमं॑ पितृ॒मन्तं॑ यजति। सं॒व॒त्स॒रो वै सोम॑ पितृ॒मान्। सं॒व॒त्स॒रमे॒व तद्य॑जति। पि॒तॄन्ब॑र्\mbox{}हि॒षदो॑ यजति॥६५॥

%1.6.9.6
ये वै यज्वा॑नः। ते पि॒तरो॑ बर्\mbox{}हि॒षद॑। ताने॒व तद्य॑जति। पि॒तॄन॑ग्निष्वा॒त्तान् य॑जति। ये वा अय॑ज्वानो गृहमे॒धिन॑। ते पि॒तरोऽग्निष्वा॒त्ताः। ताने॒व तद्य॑जति। अ॒ग्निङ्क॑व्य॒वाह॑नं यजति। य ए॒व पि॑तृ॒णाम॒ग्निः। तमे॒व तद्य॑जति॥६६॥

%1.6.9.7
अथो॒ यथा॒ऽग्नि स्वि॑ष्ट॒कृतं॒ यज॑ति। ता॒दृगे॒व तत्। ए॒तत्ते॑ तत॒ ये च॒ त्वामन्विति॑ ति॒सृषु॑ स्र॒क्तीषु॒ निद॑धाति। तस्मा॒दा तृ॒तीया॒त्पुरु॑षा॒न्नाम॒ न गृ॑ह्णन्ति। ए॒ताव॑न्तो॒ हीज्यन्ते। अत्र॑ पितरो यथाभा॒गं म॑न्दध्व॒मित्या॑ह। ह्लीका॒ हि पि॒तर॑। उद॑ञ्चो॒ निष्क्रा॑मन्ति। ए॒षा वै म॑नु॒ष्या॑णा॒न्दिक्। स्वामे॒व तद्दिश॒मनु॒ निष्क्रा॑मन्ति॥६७॥

%1.6.9.8
आ॒ह॒व॒नीय॒मुप॑तिष्ठन्ते। न्ये॑वास्मै॒ तद्ध्नु॑वते। यत्स॒त्या॑हव॒नीये। अथा॒न्यत्र॒ चर॑न्ति। आतमि॑तो॒रुप॑तिष्ठन्ते। अ॒ग्निमे॒वोप॑द्र॒ष्टारं॑ कृ॒त्वा। पि॒तॄन्नि॒रव॑दयन्ते। अन्तं॒ वा ए॒ते प्रा॒णानां गच्छन्ति। य आतमि॑तोरुप॒ तिष्ठ॑न्ते। सु॒स॒न्दृश॑न्त्वा व॒यमित्या॑ह॥६८॥

%1.6.9.9
प्रा॒णो वै सु॑स॒न्दृक्। प्रा॒णमे॒वात्मन्द॑धते। योजा॒ न्वि॑न्द्र ते॒ हरी॒ इत्या॑ह। प्रा॒णमे॒व पुन॑रयुक्त। अक्ष॒न्नमी॑मदन्त॒ हीति॒ गार्\mbox{}ह॑पत्य॒मुप॑तिष्ठन्ते। अक्ष॒न्नमी॑मद॒न्ताथ॒ त्वोप॑तिष्ठामह॒ इति॒ वावैतदा॑ह। अमी॑मदन्त पि॒तर॑ सो॒म्या इत्य॒भि प्रप॑द्यन्ते। अमी॑मदन्त पि॒तरोऽथ॑ त्वा॒ऽभि प्रप॑द्यामह॒ इति॒ वावैतदा॑ह। अ॒पः परि॑षिञ्चति। मा॒र्जय॑त्ये॒वैनान्॑॥६९॥

%1.6.9.10
अथो॑ त॒र्पय॑त्ये॒व। तृप्य॑ति प्र॒जया॑ प॒शुभि॑। य ए॒वं वेद॑। अप॑ बर्\mbox{}हिषावनूया॒जौ य॑जति। प्र॒जा वै ब॒र्॒हिः। प्र॒जा ए॒व मृ॒त्योरुत्सृ॑जति। च॒तुर॑ प्रया॒जान् य॑जति। द्वाव॑नूया॒जौ। षट्त्सं प॑द्यन्ते। षड्वा ऋ॒तव॑। ऋ॒तूने॒व प्री॑णाति। न पत्न्यन्वास्ते। न संया॑जयन्ति। यत्पत्न्य॒न्वासी॑त। यत्सं॑या॒जये॑युः। प्र॒मायु॑का स्यात्। तस्मा॒न्नान्वास्ते। न संया॑जयन्ति। पत्नि॑यै गोपी॒थाय॑॥७०॥\anuvakamend[होता॑र॒माज्य॑भागौ यजति॒ सन्त॑त॒मव॑द्यति॒ व्यावृ॑त्त्यै बर्\mbox{}हि॒षदो॑ यजति॒ तमे॒व तद्य॑ज॒त्यनु॒ निष्क्रा॑मन्त्याहैनानृ॒तवो॒ नव॑ च]

%1.6.10.1
प्र॒ति॒पू॒रु॒षमेक॑कपालां॒ निर्व॑पति। जा॒ता ए॒व प्र॒जा रु॒द्रान्नि॒रव॑दयते। एक॒मति॑रिक्तम्। ज॒नि॒ष्यमा॑णा ए॒व प्र॒जा रु॒द्रान्नि॒रव॑दयते। एक॑कपाला भवन्ति। ए॒क॒धैव रु॒द्रन्नि॒रव॑दयते। नाभिघा॑रयति। यद॑भिघा॒रयेत्। अ॒न्त॒र॒व॒चा॒रिण रु॒द्रं कु॑र्यात्। ए॒को॒ल्मु॒केन॑ यन्ति॥७१॥

%1.6.10.2
तद्धि रु॒द्रस्य॑ भाग॒धेयम्। इ॒मान्दिशं॑ यन्ति। ए॒षा वै रु॒द्रस्य॒ दिक्। स्वाया॑मे॒व दि॒शि रु॒द्रन्नि॒रव॑दयते। रु॒द्रो वा अ॑प॒शुका॑या॒ आहु॑त्यै॒ नाति॑ष्ठत। अ॒सौ ते॑ प॒शुरिति॒ निर्दि॑शे॒द्यं द्वि॒ष्यात्। यमे॒व द्वेष्टि॑। तम॑स्मै प॒शुं निर्दि॑शति। यदि॒ न द्वि॒ष्यात्। आ॒खुस्ते॑ प॒शुरिति॑ ब्रूयात्॥७२॥

%1.6.10.3
न ग्रा॒म्यान्प॒शून् हि॒नस्ति॑। नार॒ण्यान्। च॒तु॒ष्प॒थे जु॑होति। ए॒ष वा अ॑ग्नी॒नां पड्बी॑शो॒ नाम॑। अ॒ग्नि॒वत्ये॒व जु॑होति। म॒ध्य॒मेन॑ प॒र्णेन॑ जुहोति। स्रुग्घ्ये॑षा। अथो॒ खलु॑। अ॒न्त॒मेनै॒व हो॑त॒व्यम्। अ॒न्त॒त ए॒व रु॒द्रं नि॒रव॑दयते॥७३॥

%1.6.10.4
ए॒ष ते॑ रुद्र भा॒गः स॒ह स्वस्राऽम्बि॑क॒येत्या॑ह। श॒रद्वा अ॒स्याम्बि॑का॒ स्वसा। तया॒ वा ए॒ष हि॑नस्ति। य हि॒नस्ति॑। तयै॒वैन स॒ह श॑मयति। भे॒ष॒जङ्गव॒ इत्या॑ह। याव॑न्त ए॒व ग्रा॒म्याः प॒शव॑। तेभ्यो॑ भेष॒जं क॑रोति। अवाम्ब रु॒द्रम॑दिम॒हीत्या॑ह। आ॒शिष॑मे॒वैतामा शास्ते॥७४॥

%1.6.10.5
त्र्य॑म्बकं यजामह॒ इत्या॑ह। मृ॒त्योर्मु॑क्षीय॒ माऽमृता॒दिति॒ वावैतदा॑ह। उत्कि॑रन्ति। भग॑स्य लीप्सन्ते। मूते॑कृ॒त्वाऽऽस॑जन्ति। यथा॒ जनं॑ य॒ते॑ऽव॒सं क॒रोति॑। ता॒दृगे॒व तत्। ए॒ष ते॑ रुद्र भा॒ग इत्या॑ह नि॒रव॑त्त्यै। अप्र॑तीक्ष॒मा य॑न्ति। अ॒पः परि॑षिञ्चति। रु॒द्रस्या॒न्तर्\mbox{}हि॑त्यै। प्र वा ए॒तेऽस्माल्लो॒काच्च्य॑वन्ते। ये त्र्य॑म्बकै॒श्चर॑न्ति। आ॒दि॒त्यञ्च॒रुं पुन॒रेत्य॒ निर्व॑पति। इ॒यं वा अदि॑तिः। अ॒स्यामे॒व प्रति॑ तिष्ठन्ति॥७५॥\anuvakamend[य॒न्ति॒ ब्रू॒या॒न्नि॒रव॑दयते शास्ते सिञ्चति॒ षट्च॑]




\prashnaend{अनु॑मत्यै वैश्वदे॒वेन॒ ताः सृ॒ष्टास्त्रि॒वृत्प्र॒जाप॑तिः सवि॒तोत्त॑रस्यान्देवासु॒राः सोऽग्निर्यत्पत्नी॑ वैश्वदे॒वेन॒ ता व॑रुणप्रघा॒सैर॒ग्नये॑ दे॒वेभ्य॑ प्रतिपूरु॒षन्दश॑॥१०॥}{अनु॑मत्यै प्रथम॒जो व॒त्सो ब॑हुरू॒पा हि प॒शव॒स्तस्मात्पृथमा॒त्रं यद॒ग्नयेऽनी॑कवत उद्धा॒रं वा अ॒ग्नये॑ दे॒वेभ्य॑ प्रतिपूरु॒षं पञ्च॑सप्ततिः॥७५॥}{अनु॑मत्यै॒ प्रति॑तिष्ठन्ति॥}{हरि॑ ओम्॥}{इति श्रीकृष्णयजुर्वेदीयतैत्तिरीयब्राह्मणे प्रथमाष्टके षष्ठः प्रपाठकः समाप्तः॥}
\clearpage
