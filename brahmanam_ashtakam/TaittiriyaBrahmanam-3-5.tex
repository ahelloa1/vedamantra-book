\sect{पञ्चमः प्रश्नः}
\setcounter{anuvakam}{0}
\dnsub{तैत्तिरीयब्राह्मणे तृतीयाष्टके पञ्चमः प्रपाठकः}

%3.5.1.1
स॒त्यं प्रप॑द्ये। ऋ॒तं प्रप॑द्ये। अ॒मृतं॒ प्रप॑द्ये। प्र॒जाप॑तेः प्रि॒यान्त॒नुव॒मनार्तां॒ प्रप॑द्ये। इ॒दम॒हं प॑ञ्चद॒शेन॒ वज्रे॑ण। द्वि॒षन्तं॒ भ्रातृ॑व्य॒मव॑ क्रामामि। योऽस्मान्द्वेष्टि॑। यं च॑ व॒यं द्वि॒ष्मः। भूर्भुव॒ सुव॑। हिम्॥१॥\anuvakamend[स॒त्यन्दश॑]

%3.5.2.1
प्र वो॒ वाजा॑ अ॒भिद्य॑वः। ह॒विष्म॑न्तो घृ॒ताच्या। दे॒वाञ्जि॑गाति सुम्न॒युः। अग्न॒ आया॑हि वी॒तये। गृ॒णा॒नो ह॒व्यदा॑तये। नि होता॑ सत्सि ब॒र्॒हिषि॑। तन्त्वा॑ स॒मिद्भि॑रङ्गिरः। घृ॒तेन॑ वर्धयामसि। बृ॒हच्छो॑चा यविष्ठ्य। स न॑ पृ॒थुः श्र॒वाय्यम्॥२॥

%3.5.2.2
अच्छा॑ देव विवाससि। बृ॒हद॑ग्ने सु॒वीर्यम्। ई॒डेन्यो॑ नम॒स्य॑स्ति॒रः। तमासि दर्‌श॒तः। सम॒ग्निरि॑ध्यते॒ वृषा। वृषो॑ अ॒ग्निः समि॑ध्यते। अश्वो॒ न दे॑व॒वाह॑नः। त ह॒विष्म॑न्त ईडते। वृष॑णन्त्वा व॒यं वृष\sn{}। वृषा॑ण॒ समि॑धीमहि॥३॥

%3.5.2.3
अग्ने॒ दीद्य॑तं बृ॒हत्। अ॒ग्निं दू॒तं वृ॑णीमहे। होता॑रं वि॒श्ववे॑दसम्। अ॒स्य य॒ज्ञस्य॑ सु॒क्रतुम्। स॒मि॒ध्यमा॑नो अध्व॒रे। अ॒ग्निः पा॑व॒क ईड्य॑। शो॒चिष्के॑श॒स्तमी॑महे। समि॑द्धो अग्न आहुत। दे॒वान् य॑क्षि स्वध्वर। त्व हि ह॑व्य॒वाडसि॑। आ जु॑होत दुव॒स्यत॑। अ॒ग्निं प्र॑य॒त्य॑ध्व॒रे। वृ॒णीध्व ह॑व्य॒वाह॑नम्। त्वं वरु॑ण उ॒त मि॒त्रो अ॑ग्ने। त्वां व॑र्धन्ति म॒तिभि॒र्वसि॑ष्ठाः। त्वे वसु॑ सुषण॒नानि॑ सन्तु। यू॒यं पा॑त स्व॒स्तिभि॒ सदा॑ नः ॥४॥\anuvakamend[श्र॒वाय्य॑मिधीम॒ह्यसि॑ स॒प्त च॑]

%3.5.3.1
अग्ने॑ म॒हा अ॑सि ब्राह्मण भारत। असा॒वसौ। दे॒वेद्धो॒ मन्वि॑द्धः। ऋषि॑ष्टुतो॒ विप्रा॑नुमदितः। क॒वि॒श॒स्तो ब्रह्म॑सशितो घृ॒ताह॑वनः। प्र॒णीर्य॒ज्ञानाम्। र॒थीर॑ध्व॒राणाम्। अ॒तूर्तो॒ होता। तूर्णि॑र्‌हव्य॒वाट्। आस्पात्रं॑ जु॒हूर्दे॒वानाम्॥५॥

%3.5.3.2
च॒म॒सो दे॑व॒पान॑। अ॒रा इ॑वाग्ने ने॒मिर्दे॒वास्त्वं प॑रि॒भूर॑सि। आ व॑ह दे॒वान् यज॑मानाय। अ॒ग्निम॑ग्न॒ आव॑ह। सोम॒माव॑ह। अ॒ग्निमाव॑ह। प्र॒जाप॑ति॒माव॑ह। अ॒ग्नीषोमा॒वाव॑ह। इ॒न्द्रा॒ग्नी आव॑ह। इन्द्र॒माव॑ह। म॒हे॒न्द्रमाव॑ह। दे॒वा आज्य॒पा आव॑ह। अ॒ग्नि हो॒त्रायाव॑ह। स्वं म॑हि॒मान॒मा व॑ह। आ चाग्ने दे॒वान् वह॑। सु॒यजा॑ च यज जातवेदः॥६॥\anuvakamend[दे॒वाना॒मिन्द्र॒मा व॑ह॒ षट् च॑]

%3.5.4.1
अ॒ग्निर्\mbox{}होता॒ वेत्व॒ग्निः। हो॒त्रं वेत्तु प्रावि॒त्रम्। स्मो व॒यम्। सा॒धु ते॑ यजमान दे॒वता। घृ॒तव॑तीमध्वर्यो॒ स्रुच॒मास्य॑स्व। दे॒वा॒युवं॑ वि॒श्ववा॑राम्। ईडा॑महै दे॒वा ई॒डेन्यान्॑। न॒म॒स्याम॑ नम॒स्यान्॑। यजा॑म य॒ज्ञियान्॑॥७॥\anuvakamend[अ॒ग्निर्‌होता॒ नव॑]

%3.5.5.1
स॒मिधो॑ अग्न॒ आज्य॑स्य वियन्तु। तनू॒नपा॑दग्न॒ आज्य॑स्य वेतु। इ॒डो अ॑ग्न॒ आज्य॑स्य वियन्तु। ब॒र्॒हिर॑ग्न॒ आज्य॑स्य वेतु। स्वाहा॒ऽग्निम्। स्वाहा॒ सोमम्। स्वाहा॒ऽग्निम्। स्वाहा प्र॒जाप॑तिम्। स्वाहा॒ऽग्नीषोमौ। स्वाहेन्द्रा॒ग्नी। स्वाहेन्द्रम्। स्वाहा॑ महे॒न्द्रम्। स्वाहा॑ दे॒वा आज्य॒पान्। स्वाहा॒ऽग्नि हो॒त्राज्जु॑षा॒णाः। अग्न॒ आज्य॑स्य वियन्तु॥८॥\anuvakamend[इ॒न्द्रा॒ग्नी पञ्च॑ च]

%3.5.6.1
अ॒ग्निर्वृ॒त्राणि॑ जङ्घनत्। द्र॒वि॒ण॒स्युर्वि॑प॒न्यया। समि॑द्धः शु॒क्र आहु॑तः। जु॒षा॒णो अ॒ग्निराज्य॑स्य वेतु। त्व सो॑मासि॒ सत्प॑तिः। त्व राजो॒त वृ॑त्र॒हा। त्वं भ॒द्रो अ॑सि॒ क्रतु॑। जु॒षा॒णः सोम॒ आज्य॑स्य ह॒विषो॑ वेतु। अ॒ग्निः प्र॒त्नेन॒ जन्म॑ना। शुम्भा॑नस्त॒नुव॒ स्वाम्। क॒विर्विप्रे॑ण वावृधे। जु॒षा॒णो अ॒ग्निराज्य॑स्य वेतु। सोम॑ गी॒र्भिष्ट्वा॑ व॒यम्। व॒र्धया॑मो वचो॒विद॑। सु॒मृ॒डी॒को न॒ आवि॑श। जु॒षा॒णः सोम॒ आज्य॑स्य ह॒विषो॑ वेतु॥९॥\anuvakamend[स्वा षट् च॑]

%3.5.7.1
अ॒ग्निर्मू॒र्धा दि॒वः क॒कुत्। पति॑ पृथि॒व्या अ॒यम्। अ॒पा रेतासि जिन्वति। भुवो॑ य॒ज्ञस्य॒ रज॑सश्च ने॒ता। यत्रा॑ नि॒युद्भि॒ सच॑से शि॒वाभि॑। दि॒वि मू॒र्धान॑न्दधिषे सुव॒र्॒षाम्। जि॒ह्वाम॑ग्ने चकृषे हव्य॒वाहम्। प्रजा॑पते॒ न त्वदे॒तान्य॒न्यः। विश्वा॑ जा॒तानि॒ परि॒ ता ब॑भूव। यत्का॑मास्ते जहु॒मस्तन्नो॑ अस्तु॥१०॥

%3.5.7.2
व॒य स्या॑म॒ पत॑यो रयी॒णाम्। स वे॑द पु॒त्रः पि॒तर॒ स मा॒तरम्। स सू॒नुर्भु॑व॒त्स भु॑व॒त्पुन॑र्मघः। स द्यामौर्णो॑द॒न्तरि॑क्ष॒ स सुव॑। स विश्वा॒ भुवो॑ अभव॒त्स आभ॑वत्। अग्नी॑षोमा॒ सवे॑दसा। सहू॑ती वनत॒ङ्गिर॑। सन्दे॑व॒त्रा ब॑भूवथुः। यु॒वमे॒तानि॑ दि॒वि रो॑च॒नानि॑। अ॒ग्निश्च॑ सोम॒ सक्र॑तू अधत्तम्॥११॥

%3.5.7.3
यु॒व सिन्धू र॒भिश॑स्तेरव॒द्यात्। अग्नी॑षोमा॒वमु॑ञ्चतङ्गृभी॒तान्। इन्द्राग्नी रोच॒ना दि॒वः। परि॒ वाजे॑षु भूषथः। तद्वाञ्चेति॒ प्रवी॒र्यम्। श्ञथ॑द्वृ॒त्रमु॒त स॑नोति॒ वाजम्। इन्द्रा॒ यो अ॒ग्नी सहु॑री सप॒र्यात्। इ॒र॒ज्यन्ता॑ वस॒व्य॑स्य॒ भूरे। सह॑स्तमा॒ सह॑सा वाज॒यन्ता। एन्द्र॑ सान॒सि र॒यिम्॥१२॥

%3.5.7.4
स॒जित्वा॑न सदा॒सहम्। वर्‌षि॑ष्ठमू॒तये॑ भर। प्रस॑साहिषे पुरुहूत॒ शत्रून्॑। ज्येष्ठ॑स्ते॒ शुष्म॑ इ॒ह रा॒तिर॑स्तु। इन्द्रा भ॑र॒ दक्षि॑णेना॒ वसू॑नि। पति॒ सिन्धू॑नामसि रे॒वती॑नाम्। म॒हा इन्द्रो॒ य ओज॑सा। प॒र्जन्यो॑ वृष्टि॒मा इ॑व। स्तोमैर्व॒त्सस्य॑ वावृधे। म॒हा इन्द्रो॑ नृ॒वदाच॑र्‌षणि॒प्राः॥१३॥

%3.5.7.5
उ॒त द्वि॒बर्‌हा॑ अमि॒नः सहो॑भिः। अ॒स्म॒द्रिय॑ग्वावृधे वी॒र्या॑य। उ॒रुः पृ॒थुः सुकृ॑तः क॒र्तृभि॑र्भूत्। पि॒प्री॒हि दे॒वा उ॑श॒तो य॑विष्ठ। वि॒द्वा ऋ॒तूर्\mbox{}ऋ॑तुपते यजे॒ह। ये दैव्या॑ ऋ॒त्विज॒स्तेभि॑रग्ने। त्व होतॄ॑णाम॒स्याय॑जिष्ठः। अ॒ग्नि स्वि॑ष्ट॒कृतम्। अया॑ड॒ग्निर॒ग्नेः प्रि॒या धामा॑नि। अया॒ट्त्सोम॑स्य प्रि॒या धामा॑नि॥१४॥

%3.5.7.6
अया॑ड॒ग्नेः प्रि॒या धामा॑नि। अयाट्प्र॒जाप॑तेः प्रि॒या धामा॑नि। अया॑ड॒ग्नीषोम॑योः प्रि॒या धामा॑नि। अया॑डिन्द्राग्नि॒योः प्रि॒या धामा॑नि। अया॒डिन्द्र॑स्य प्रि॒या धामा॑नि। अयाण्महे॒न्द्रस्य॑ प्रि॒या धामा॑नि। अयाड्दे॒वाना॑माज्य॒पानां प्रि॒या धामा॑नि। यक्ष॑द॒ग्नेर्‌होतु॑ प्रि॒या धामा॑नि। यक्ष॒त्स्वं म॑हि॒मानम्। आय॑जता॒मेज्या॒ इष॑। कृ॒णोतु॒ सो अ॑ध्व॒रा जा॒तवे॑दाः। जु॒षता ह॒विः। अग्ने॒ यद॒द्य वि॒शो अ॑ध्वरस्य होतः। पाव॑क शोचे॒ वेष्ट्व हि यज्वा। ऋ॒ता य॑जासि महि॒ना वियद्भूः। ह॒व्या व॑ह यविष्ठ॒ या ते॑ अ॒द्य॥१५॥\anuvakamend[अ॒स्त्व॒ध॒त्त॒ र॒यिञ्च॑र्‌षणि॒प्राः सोम॑स्य प्रि॒या धामा॒नीष॒ष्षट्च॑]

%3.5.8.1
उप॑हूत रथन्त॒र स॒ह पृ॑थि॒व्या। उप॑ मा रथन्त॒र स॒ह पृ॑थि॒व्या ह्व॑यताम्। उप॑हूतं वामदे॒व्य स॒हान्तरि॑क्षेण। उप॑ मा वामदे॒व्य स॒हान्तरि॑क्षेण ह्वयताम्। उप॑हूतं बृ॒हत्स॒ह दि॒वा। उप॑ मा बृ॒हत्स॒ह दि॒वा ह्व॑यताम्। उप॑हूताः स॒प्त होत्रा। उप॑ मा स॒प्त होत्रा ह्वयन्ताम्। उप॑हूता धे॒नुः स॒हर्‌ष॑भा। उप॑ मा धे॒नुः स॒हर्‌ष॑भा ह्वयताम्॥१६॥

%3.5.8.2
उप॑हूतो भ॒क्षः सखा। उप॑ मा भ॒क्षः सखा ह्वयताम्। उप॑हू॒ताँ(४)हो। इडोप॑हूता। उप॑हू॒तेडा। उपो॑ अ॒स्मा इडा ह्वयताम्। इडोप॑हूता। उप॑हू॒तेडा। मा॒न॒वी घृ॒तप॑दी मैत्रावरु॒णी। ब्रह्म॑ दे॒वकृ॑त॒मुप॑हूतम्॥१७॥

%3.5.8.3
दैव्या॑ अध्व॒र्यव॒ उप॑हूताः। उप॑हूता मनु॒ष्या। य इ॒मं य॒ज्ञमवान्॑। ये य॒ज्ञप॑तिं॒ वर्धान्॑। उप॑हूते॒ द्यावा॑पृथि॒वी। पू॒र्व॒जे ऋ॒ताव॑री। दे॒वी दे॒वपु॑त्रे। उप॑हूतो॒ऽयं यज॑मानः। उत्त॑रस्यान्देवय॒ज्याया॒मुप॑हूतः। भूय॑सि हवि॒ष्कर॑ण॒ उप॑हूतः। दि॒व्ये धाम॒न्नुप॑हूतः। इ॒दं मे॑ दे॒वा ह॒विर्जु॑षन्ता॒मिति॒ तस्मि॒न्नुप॑हूतः। विश्व॑मस्य प्रि॒यमुप॑हूतम्। विश्व॑स्य प्रि॒यस्योप॑हूत॒स्योप॑हूतः॥१८॥\anuvakamend[स॒हर्‌ष॑भा ह्वयता॒मुप॑हूत हवि॒ष्कर॑ण॒ उप॑हूतश्च॒त्वारि॑ च]

%3.5.9.1
दे॒वं ब॒र्‌हिः। व॒सु॒वने॑ वसु॒धेय॑स्य वेतु। दे॒वो नरा॒शस॑। व॒सु॒वने॑ वसु॒धेय॑स्य वेतु। दे॒वो अ॒ग्निः स्वि॑ष्ट॒कृत्। सु॒द्रवि॑णा म॒न्द्रः क॒विः। स॒त्यम॑न्माय॒जी होता। होतु॑र्‌होतु॒राय॑जीयान्। अग्ने॒ यान् दे॒वानयाट्। या अपि॑प्रेः। ये ते॑ हो॒त्रे अम॑त्सत। ता स॑स॒नुषी॒ होत्रान्देवङ्ग॒माम्। दि॒वि दे॒वेषु॑ य॒ज्ञमेर॑ये॒मम्। स्वि॒ष्ट॒कृच्चाग्ने॒ होताऽभू। व॒सु॒वने॑ वसु॒धेय॑स्य नमोवा॒के वीहि॑॥१९॥\anuvakamend[अपि॑प्रे॒ पञ्च॑ च]

%3.5.10.1
इ॒दन्द्या॑वापृथिवी भ॒द्रम॑भूत्। आर्ध्म॑ सूक्तवा॒कम्। उ॒त न॑मोवा॒कम्। ऋ॒ध्यास्म॑ सू॒क्तोच्य॑मग्ने। त्व सूक्त॒वाग॑सि। उप॑श्रितो दि॒वः पृ॑थि॒व्योः। ओम॑न्वती ते॒ऽस्मिन् य॒ज्ञे य॑जमान॒ द्यावा॑पृथि॒वी स्ताम्। श॒ङ्ग॒ये जी॒रदा॑नू। अत्र॑स्नू॒ अप्र॑वेदे। उ॒रुग॑व्यूती अभय॒ङ्कृतौ॥२०॥

%3.5.10.2
वृ॒ष्टिद्या॑वा री॒त्या॑पा। श॒म्भुवौ॑ मयो॒भुवौ। ऊर्ज॑स्वती च॒ पय॑स्वती च। सू॒प॒च॒र॒णा च॑ स्वधिचर॒णा च॑। तयो॑रा॒विदि॑। अ॒ग्निरि॒द ह॒विर॑जुषत। अवी॑वृधत॒ महो॒ ज्यायो॑ऽकृत। सोम॑ इ॒दह॒विर॑जुषत। अवी॑वृधत॒ महो॒ ज्यायो॑ऽकृत। अ॒ग्निरि॒द ह॒विर॑जुषत॥२१॥

%3.5.10.3
अवी॑वृधत॒ महो॒ ज्यायो॑ऽकृत। प्र॒जाप॑तिरि॒द ह॒विर॑जुषत। अवी॑वृधत॒ महो॒ ज्यायो॑ऽकृत। अ॒ग्नीषोमा॑वि॒द ह॒विर॑जुषेताम्। अवी॑वृधेतां॒ महो॒ ज्यायोऽक्राताम्। इ॒न्द्रा॒ग्नी इ॒द ह॒विर॑जुषेताम्। अवी॑वृधेतां॒ महो॒ ज्यायोऽक्राताम्। इन्द्र॑ इ॒द ह॒विर॑जुषत। अवी॑वृधत॒ महो॒ ज्यायो॑ऽकृत। म॒हे॒न्द्र इ॒द ह॒विर॑जुषत॥२२॥

%3.5.10.4
अवी॑वृधत॒ महो॒ ज्यायो॑ऽकृत। दे॒वा आज्य॒पा आज्य॑मजुषन्त। अवी॑वृधन्त॒ महो॒ ज्यायोऽक्रत। अ॒ग्निर्‌हो॒त्रेणे॒द ह॒विर॑जुषत। अवी॑वृधत॒ महो॒ ज्यायो॑ऽकृत। अ॒स्यामृध॒द्धोत्रा॑यान्देवङ्ग॒मायाम्। आशास्ते॒ऽयं यज॑मानो॒ऽसौ। आयु॒रा शास्ते। सु॒प्र॒जा॒स्त्वमा शास्ते। स॒जा॒त॒व॒न॒स्यामा शास्ते॥२३॥

%3.5.10.5
उत्त॑रान्देवय॒ज्यामा शास्ते। भूयो॑ हवि॒ष्कर॑ण॒मा शास्ते। दि॒व्यन्धामा शास्ते। विश्वं॑ प्रि॒यमा शास्ते। यद॒नेन॑ ह॒विषाऽऽशास्ते। तद॑श्या॒त्तदृ॑ध्यात्। तद॑स्मै दे॒वा रा॑सन्ताम्। तद॒ग्निर्दे॒वो दे॒वेभ्यो॒ वन॑ते। व॒यम॒ग्नेर्मानु॑षाः। इ॒ष्टं च॑ वी॒तं च॑। उ॒भे च॑ नो॒ द्यावा॑पृथि॒वी अह॑सस्पाताम्। इ॒ह गति॑र्वा॒मस्ये॒दं च॑। नमो॑ दे॒वेभ्य॑॥२४॥\anuvakamend[अ॒भ॒य॒ङ्कृता॑वकृता॒ग्निरि॒द ह॒विर॑जुषत महे॒न्द्र इ॒द ह॒विर॑जुषत सजातवन॒स्यामा शास्ते वी॒तं च॒ त्रीणि॑ च]

%3.5.11.1
तच्छ॒य्योँरा वृ॑णीमहे। गा॒तुं य॒ज्ञाय॑। गा॒तुं य॒ज्ञप॑तये। दैवी स्व॒स्तिर॑स्तु नः। स्व॒स्तिर्मानु॑षेभ्यः। ऊ॒र्ध्वञ्जि॑गातु भेष॒जम्। शन्नो॑ अस्तु द्वि॒पदे। शञ्चतु॑ष्पदे॥२५॥\anuvakamend[तच्छ॒य्योँर॒ष्टौ]

%3.5.12.1
आप्या॑यस्व॒ सन्ते। इ॒ह त्वष्टा॑रमग्रि॒यन्तन्न॑स्तु॒रीपम्। दे॒वानां॒ पत्नी॑रुश॒तीर॑वन्तु नः। प्राव॑न्तु नस्तु॒जये॒ वाज॑सातये। याः पार्थि॑वासो॒ या अ॒पामपि॑ व्र॒ते। ता नो॑ देवीः सुहवा॒ शर्म॑ यच्छत। उ॒त ग्ना वि॑यन्तु दे॒वप॑त्नीः। इ॒न्द्रा॒ण्य॑ग्नाय्य॒श्विनी॒ राट्। आ रोद॑सी वरुणा॒नी शृ॑णोतु। वि॒यन्तु॑ दे॒वीर्य ऋ॒तुर्जनी॑नाम्॥२६॥

%3.5.12.2
अ॒ग्निर्‌होता॑ गृ॒हप॑ति॒ स राजा। विश्वा॑ वेद॒ जनि॑मा जा॒तवे॑दाः। दे॒वाना॑मु॒त यो मर्त्या॑नाम्। यजि॑ष्ठ॒ स प्र य॑जतामृ॒तावा। व॒यमु॑ त्वा गृहपते॒ जना॑नाम्। अग्ने॒ अक॑र्म स॒मिधा॑ बृ॒हन्तम्। अ॒स्थू॒रि णो॒ गार्‌ह॑पत्यानि सन्तु। ति॒ग्मेन॑ न॒स्तेज॑सा॒ सशि॑शाधि॥२७॥\anuvakamend[जनी॑नाम॒ष्टौ च॑]

%3.5.13.1
उप॑हूत रथन्त॒र स॒ह पृ॑थि॒व्या। उप॑ मा रथन्त॒र स॒ह पृ॑थि॒व्या ह्व॑यताम्। उप॑हूतं वामदे॒व्य स॒हान्तरि॑क्षेण। उप॑ मा वामदे॒व्य स॒हान्तरि॑क्षेण ह्वयताम्। उप॑हूतं बृ॒हत्स॒ह दि॒वा। उप॑ मा बृ॒हत्स॒ह दि॒वा ह्व॑यताम्। उप॑हूताः स॒प्त होत्रा। उप॑ मा स॒प्त होत्रा ह्वयन्ताम्। उप॑हूता धे॒नुः स॒हर्ष॑भा। उप॑ मा धे॒नुः स॒हर्‌ष॑भा ह्वयताम्॥२८॥

%3.5.13.2
उप॑हूतो भ॒क्षः सखा। उप॑ मा भ॒क्षः सखा ह्वयताम्। उप॑हू॒ताँ(४)हो। इडोप॑हूता। उप॑हू॒तेडा। उपो॑ अ॒स्मा इडा ह्वयताम्। इडोप॑हूता। उप॑हू॒तेडा। मा॒न॒वी घृ॒तप॑दी मैत्रावरु॒णी। ब्रह्म॑ दे॒वकृ॑त॒मुप॑हूतम्॥२९॥

%3.5.13.3
दैव्या॑ अध्व॒र्यव॒ उप॑हूताः। उप॑हूता मनु॒ष्या। य इ॒मं य॒ज्ञमवान्॑। ये य॒ज्ञप॑त्नीं॒ वर्धान्॑। उप॑हूते॒ द्यावा॑पृथि॒वी। पू॒र्व॒जे ऋ॒ताव॑री। दे॒वी दे॒वपु॑त्रे। उप॑हूते॒यं यज॑माना। इ॒न्द्राणीवा॑ऽविध॒वा। अदि॑तिरिव सुपु॒त्रा। उत्त॑रस्यान्देवय॒ज्याया॒मुप॑हूता। भूय॑सि हवि॒ष्कर॑ण॒ उप॑हूता। दि॒व्ये धाम॒न्नुप॑हूता। इ॒दं मे॑ दे॒वा ह॒विर्जु॑षन्ता॒मिति॒ तस्मि॒न्नुप॑हूता। विश्व॑मस्याः प्रि॒यमुप॑हूतम्। विश्व॑स्य प्रि॒यस्योप॑हूत॒स्योप॑हूता॥३०॥\anuvakamend[स॒हर्\mbox{}ष॑भा ह्वयता॒मुप॑हूत सुपु॒त्रा षट्च॑]




\prashnaend{स॒त्यं प्रवोऽग्ने॑ म॒हान॒ग्निर्\mbox{}होता॑ स॒मिधो॒ऽग्निर्वृ॒त्राण्य॒ग्निर्मू॒र्धोप॑हूतन्दे॒वं ब॒र्॒हिरि॒दन्द्या॑वापृथिवी॒ तच्छ॒य्योँरा प्या॑य॒स्वोप॑हूत॒न्त्रयो॑दश॥१३॥}{स॒त्यं व॒य स्या॑म वृ॒ष्टिद्या॑वा त्रि॒शत्॥३०॥}{स॒त्यमुप॑हूता॥}{हरि॑ ओम्॥}{इति श्रीकृष्णयजुर्वेदीयतैत्तिरीयब्राह्मणे तृतीयाष्टके पञ्चमः प्रपाठकः समाप्तः॥}
\clearpage
