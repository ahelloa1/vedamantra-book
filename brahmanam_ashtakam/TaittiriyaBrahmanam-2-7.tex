\sect{सप्तमः प्रश्नः}
\setcounter{anuvakam}{0}
\dnsub{तैत्तिरीयब्राह्मणे द्वितीयाष्टके सप्तमः प्रपाठकः}

%2.7.1.1
त्रि॒वृत्स्तोमो॑ भवति। ब्र॒ह्म॒व॒र्च॒सं वै त्रि॒वृत्। ब्र॒ह्म॒व॒र्च॒समे॒वाव॑ रुन्धे। अ॒ग्नि॒ष्टो॒मः सोमो॑ भवति। ब्र॒ह्म॒व॒र्च॒सं वा अ॑ग्निष्टो॒मः। ब्र॒ह्म॒व॒र्च॒समे॒वाव॑ रुन्धे। र॒थ॒न्त॒र साम॑ भवति। ब्र॒ह्म॒व॒र्च॒सं वै र॑थन्त॒रम्। ब्र॒ह्म॒व॒र्च॒समे॒वाव॑ रुन्धे। प॒रि॒स्र॒जी होता॑ भवति॥१॥

%2.7.1.2
अ॒रु॒णो मि॑र्मि॒रस्त्रिशु॑क्रः। ए॒तद्वै ब्र॑ह्मवर्च॒सस्य॑ रू॒पम्। रू॒पेणै॒व ब्र॑ह्मवर्च॒समव॑ रुन्धे। बृह॒स्पति॑रकामयत दे॒वानां पुरो॒धाङ्ग॑च्छेय॒मिति॑। स ए॒तं बृ॑हस्पतिस॒वम॑पश्यत्। तमाऽह॑रत्। तेना॑यजत। ततो॒ वै स दे॒वानां पुरो॒धाम॑गच्छत्। यः पु॑रो॒धाका॑म॒ स्यात्। स बृ॑हस्पतिस॒वेन॑ यजेत॥२॥

%2.7.1.3
पु॒रो॒धामे॒व ग॑च्छति। तस्य॑ प्रातः सव॒ने स॒न्नेषु॑ नाराश॒सेषु॑। एका॑दश॒ दक्षि॑णा नीयन्ते। एका॑दश॒ माध्य॑न्दिने॒ सव॑ने स॒न्नेषु॑ नाराश॒सेषु॑। एका॑दश तृतीयसव॒ने स॒न्नेषु॑ नाराश॒सेषु॑। त्रय॑स्त्रिश॒त्संप॑द्यन्ते। त्रय॑स्त्रिश॒द्वै दे॒वता। दे॒वता॑ ए॒वाव॑रुन्धे। अश्व॑श्चतुस्त्रि॒शः। प्रा॒जा॒प॒त्यो वा अश्व॑॥३॥

%2.7.1.4
प्र॒जाप॑तिश्चतुस्त्रि॒शो दे॒वता॑नाम्। याव॑तीरे॒व दे॒वता। ता ए॒वाव॑रुन्धे। कृ॒ष्णा॒जि॒ने॑ऽभिषि॑ञ्चति। ब्रह्म॑णो॒ वा ए॒तद्रू॒पम्। यत्कृ॑ष्णाजि॒नम्। ब्र॒ह्म॒व॒र्च॒सेनै॒वैन॒ सम॑र्धयति। आज्ये॑ना॒भिषि॑ञ्चति। तेजो॒ वा आज्यम्। तेज॑ ए॒वास्मि॑न्दधाति॥४॥\anuvakamend[होता॑ भवति यजेत॒ वा अश्वो॑ दधाति]

%2.7.2.1
यदाग्ने॒यो भव॑ति। अ॒ग्निमु॑खा॒ ह्यृद्धि॑। अथ॒ यत्पौ॒ष्णः। पुष्टि॒र्वै पू॒षा। पुष्टि॒र्वैश्य॑स्य। पुष्टि॑मे॒वाव॑ रुन्धे। प्र॒स॒वाय॑ सावि॒त्रः। अथ॒ यत्त्वा॒ष्ट्रः। त्वष्टा॒ हि रू॒पाणि॑ विक॒रोति॑। नि॒र्व॒रु॒ण॒त्वाय॑ वारु॒णः॥५॥

%2.7.2.2
अथो॒ य ए॒व कश्च॒ सन्त्सू॒यते। स हि वा॑रु॒णः। अथ॒ यद्वैश्वदे॒वः। वै॒श्व॒दे॒वो हि वैश्य॑। अथ॒ यन्मा॑रु॒तः। मा॒रु॒तो हि वैश्य॑। स॒प्तैतानि॑ ह॒वीषि॑ भवन्ति। स॒प्तग॑णा॒ वै म॒रुत॑। पृश्ञि॑ पष्ठौ॒ही मा॑रु॒त्या ल॑भ्यते। विड्वै म॒रुत॑। विश॑ ए॒वैतन्म॑ध्य॒तो॑ऽभिषि॑च्यते। तस्मा॒द्वा ए॒ष वि॒शः प्रि॒यः। वि॒शो हि म॑ध्य॒तो॑ऽभिषि॒च्यते। ऋ॒ष॒भ॒च॒र्मेऽध्य॒भिषि॑ञ्चति। स हि प्र॑जनयि॒ता। द॒ध्नाऽभिषि॑ञ्चति। ऊर्ग्वा अ॒न्नाद्य॒न्दधि॑। ऊ॒र्जैवैन॑म॒न्नाद्ये॑न॒ सम॑र्धयति॥६॥\anuvakamend[वा॒रु॒णो विड्वै म॒रुतो॒ऽष्टौ च॑]

%2.7.3.1
यदाग्ने॒यो भव॑ति। आ॒ग्ने॒यो वै ब्राह्म॒णः। अथ॒ यत्सौ॒म्यः। सौ॒म्यो हि ब्राह्म॒णः। प्र॒स॒वायै॒व सा॑वि॒त्रः। अथ॒ यद्बा॑र्\mbox{}हस्प॒त्यः। ए॒तद्वै ब्राह्म॒णस्य॑ वाक्प॒तीयम्। अथ॒ यद॑ग्नीषो॒मीय॑। आ॒ग्ने॒यो वै ब्राह्म॒णः। तौ य॒दा स॒ङ्गच्छे॑ते ॥७॥

%2.7.3.2
अथ॑ वी॒र्या॑वत्तरो भवति। अथ॒ यत्सा॑रस्व॒तः। ए॒तद्धि प्र॒त्यक्षं॑ ब्राह्म॒णस्य॑ वाक्प॒तीयम्। नि॒र्व॒रु॒ण॒त्वायै॒व वा॑रु॒णः। अथो॒ य ए॒व कश्च॒ सन्त्सू॒यते। स हि वा॑रु॒णः। अथ॒ यद्द्या॑वापृथि॒व्य॑। इन्द्रो॑ वृ॒त्राय॒ वज्र॒मुद॑यच्छत्। तन्द्यावा॑पृथि॒वी नान्व॑मन्येताम्। तमे॒तेनै॒व भा॑ग॒धेये॒नान्व॑मन्येताम्॥८॥

%2.7.3.3
वज्र॑स्य॒ वा ए॒षो॑ऽनुमा॒नाय॑। अनु॑मतवज्रः सूयाता॒ इति॑। अ॒ष्टावे॒तानि॑ ह॒वीषि॑ भवन्ति। अ॒ष्टाक्ष॑रा गाय॒त्री। गा॒य॒त्री ब्र॑ह्मवर्च॒सम्। गा॒य॒त्रि॒यैव ब॑ह्मवर्च॒समव॑ रुन्धे। हिर॑ण्येन घृ॒तमुत्पु॑नाति। तेज॑स ए॒व रु॒चे। कृ॒ष्णा॒जि॒ने॑ऽभिषि॑ञ्चति। ब्रह्म॑णो॒ वा ए॒तदृ॑ख्सा॒मयो॑ रू॒पम्। यत्कृ॑ष्णाजि॒नम्। ब्रह्म॑न्ने॒वैन॑मृख्सा॒मयो॒रध्य॒भिषि॑ञ्चति। घृ॒तेना॒भिषि॑ञ्चति। तथा॑ वी॒र्या॑वत्तरो भवति॥९॥\anuvakamend[स॒ङ्गच्छे॑ते भाग॒धेये॒नान्व॑मन्येता रू॒पञ्च॒त्वारि॑ च]

%2.7.4.1
न वै सोमे॑न॒ सोम॑स्य स॒वोऽस्ति। ह॒तो ह्ये॑षः। अ॒भिषु॑तो॒ ह्ये॑षः। न हि ह॒तः सू॒यते। सौ॒मी सू॒तव॑शा॒मा ल॑भते। सोमो॒ वै रे॑तो॒धाः। रेत॑ ए॒व तद्द॑धाति। सौ॒म्यर्चाऽभिषि॑ञ्चति। रे॒तो॒धा ह्ये॑षा। रेत॒ सोम॑। रेत॑ ए॒वास्मि॑न्दधाति। यत्किं च॑ राज॒सूय॑मृ॒ते सोमम्। तत्सर्वं॑ भवति। अषा॑ढय्युँ॒त्सु पृत॑नासु॒ पप्रिम्। सु॒व॒र्॒षाम॒प्स्वां वृ॒जन॑स्य गो॒पाम्। भ॒रे॒षु॒जा सु॑क्षि॒ति सु॒श्रव॑सम्। जय॑न्त॒न्त्वामनु॑ मदेम सोम॥१०॥\anuvakamend[रेत॒ सोम॑ स॒प्त च॑]

%2.7.5.1
यो वै सोमे॑न सू॒यते। स दे॑वस॒वः। यः प॒शुना॑ सू॒यते। स दे॑वस॒वः। य इष्ट्या॑ सू॒यते। स म॑नुष्यस॒वः। ए॒तं वै पृथ॑ये दे॒वाः प्राय॑च्छन्। ततो॒ वै सोऽप्या॑र॒ण्यानां पशू॒नाम॑सूयत। याव॑ती॒ किय॑तीश्च प्र॒जा वाचं॒ वद॑न्ति। तासा॒ सर्वा॑सा सूयते॥११॥

%2.7.5.2
य ए॒तेन॒ यज॑ते। य उ॑ चैनमे॒वं वेद॑। ना॒रा॒श॒स्यर्चाऽभिषि॑ञ्चति। म॒नु॒ष्या॑ वै नरा॒शस॑। नि॒ह्नुत्य॒ वावैतत्। अथा॒भिषि॑ञ्चति। यत्किं च॑ राज॒सूय॑मनुत्तरवे॒दीकम्। तत्सर्वं॑ भवति। ये मे॑ पञ्चा॒शत॑न्द॒दुः। अश्वा॑ना स॒धस्तु॑तिः। द्यु॒मद॑ग्ने॒ महि॒ श्रव॑। बृ॒हत्कृ॑धि म॒घोनाम्। नृ॒वद॑मृत नृ॒णाम्॥१२॥\anuvakamend[सू॒य॒ते॒ स॒धस्तु॑ति॒स्त्रीणि॑ च]

%2.7.6.1
ए॒ष गो॑स॒वः। ष॒ट्त्रि॒श उ॒क्थ्यो॑ बृ॒हत्सा॑मा। पव॑माने कण्वरथन्त॒रं भ॑वति। यो वै वा॑ज॒पेय॑। स स॑म्राट्त्स॒वः। यो रा॑ज॒सूय॑। स व॑रुणस॒वः। प्र॒जाप॑ति॒ स्वाराज्यं परमे॒ष्ठी। स्वाराज्य॒ङ्गौरे॒व। गौरि॑व भवति॥१३॥

%2.7.6.2
य ए॒तेन॒ यज॑ते। य उ॑ चैनमे॒वं वेद॑। उ॒भे बृ॑हद्रथन्त॒रे भ॑वतः। तद्धि स्वाराज्यम्। अ॒युत॒न्दक्षि॑णाः। तद्धि स्वाराज्यम्। प्र॒ति॒धुषा॒ऽभिषि॑ञ्चति। तद्धि स्वाराज्यम्। अनु॑द्धते॒ वेद्यै॑ दक्षिण॒त आ॑हव॒नीय॑स्य बृह॒तः स्तो॒त्रं प्रत्य॒भिषि॑ञ्चति। इ॒यं वाव र॑थन्त॒रम्॥१४॥

%2.7.6.3
अ॒सौ बृ॒हत्। अ॒नयो॑रे॒वैन॒मन॑न्तर्\mbox{}हितम॒भिषि॑ञ्चति। प॒शु॒स्तो॒मो वा ए॒षः। तेन॑ गोस॒वः। ष॒ट्त्रि॒शः सर्व॑। रे॒वज्जा॒तः सह॑सा वृ॒द्धः। क्ष॒त्राणां क्षत्र॒भृत्त॑मो वयो॒धाः। म॒हान्म॑हि॒त्वे त॑स्तभा॒नः। क्ष॒त्रे रा॒ष्ट्रे च॑ जागृहि। प्र॒जाप॑तेस्त्वा परमे॒ष्ठिन॒ स्वाराज्येना॒भिषि॑ञ्चा॒मीत्या॑ह। स्वाराज्यमे॒वैन॑ङ्गमयति॥१५॥\anuvakamend[इ॒व॒ भ॒व॒ति॒ र॒थ॒न्त॒रमा॒हैकं च]

%2.7.7.1
सि॒हे व्या॒घ्र उ॒त या पृदा॑कौ। त्विषि॑र॒ग्नौ ब्राह्म॒णे सूर्ये॒ या। इन्द्रं॒ या दे॒वी सु॒भगा॑ ज॒जान॑। सा न॒ आग॒न्वर्च॑सा संविदा॒ना। या रा॑ज॒न्ये॑ दुन्दु॒भावाय॑तायाम्। अश्व॑स्य॒ क्रन्द्ये॒ पुरु॑षस्य मा॒यौ। इन्द्रं॒ या दे॒वी सु॒भगा॑ ज॒जान॑। सा न॒ आग॒न्वर्च॑सा संविदा॒ना। या ह॒स्तिनि॑ द्वी॒पिनि॒ या हिर॑ण्ये। त्विषि॒रश्वे॑षु॒ पुरु॑षेषु॒ गोषु॑॥१६॥

%2.7.7.2
इन्द्रं॒ या दे॒वी सु॒भगा॑ ज॒जान॑। सा न॒ आग॒न्वर्च॑सा संविदा॒ना। रथे॑ अ॒क्षेषु॑ वृष॒भस्य॒ वाजे। वाते॑ प॒र्जन्ये॒ वरु॑णस्य॒ शुष्मे। इन्द्रं॒ या दे॒वी सु॒भगा॑ ज॒जान॑। सा न॒ आग॒न्वर्च॑सा संविदा॒ना। राड॑सि वि॒राड॑सि। स॒म्राड॑सि स्व॒राड॑सि। इन्द्रा॑य त्वा॒ तेज॑स्वते॒ तेज॑स्वन्त श्रीणामि। इन्द्रा॑य॒ त्वौज॑स्वत॒ ओज॑स्वन्त श्रीणामि॥१७॥

%2.7.7.3
इन्द्रा॑य त्वा॒ पय॑स्वते॒ पय॑स्वन्त श्रीणामि। इन्द्रा॑य॒ त्वाऽऽयु॑ष्मत॒ आयु॑ष्मन्त श्रीणामि। तेजो॑ऽसि। तत्ते॒ प्र य॑च्छामि। तेज॑स्वदस्तु मे॒ मुखम्। तेज॑स्व॒च्छिरो॑ अस्तु मे। तेज॑स्वान् वि॒श्वत॑ प्र॒त्यङ्ङ्। तेज॑सा॒ संपि॑पृग्धि मा। ओजो॑ऽसि। तत्ते॒ प्र य॑च्छामि॥१८॥

%2.7.7.4
ओज॑स्वदस्तु मे॒ मुखम्। ओज॑स्व॒च्छिरो॑ अस्तु मे। ओज॑स्वान् वि॒श्वत॑ प्र॒त्यङ्ङ्। ओज॑सा॒ सं पि॑पृग्धि मा। पयो॑ऽसि। तत्ते॒ प्र य॑च्छामि। पय॑स्वदस्तु मे॒ मुखम्। पय॑स्व॒च्छिरो॑ अस्तु मे। पय॑स्वान् वि॒श्वत॑ प्र॒त्यङ्ङ्। पय॑सा॒ सं पि॑पृग्धि मा॥१९॥

%2.7.7.5
आयु॑रसि। तत्ते॒ प्र य॑च्छामि। आयु॑ष्मदस्तु मे॒ मुखम्। आयु॑ष्म॒च्छिरो॑ अस्तु मे। आयु॑ष्मान् वि॒श्वत॑ प्र॒त्यङ्ङ्। आयु॑षा॒ सं पि॑पृग्धि मा। इ॒मम॑ग्न॒ आयु॑षे॒ वर्च॑से कृधि। प्रि॒य रेतो॑ वरुण सोम राजन्। मा॒तेवास्मा अदिते॒ शर्म॑ यच्छ। विश्वे॑ देवा॒ जर॑दष्टि॒र्यथाऽस॑त्॥२०॥

%2.7.7.6
आयु॑रसि वि॒श्वायु॑रसि। स॒र्वायु॑रसि॒ सर्व॒मायु॑रसि। यतो॒ वातो॒ मनो॑जवाः। यत॒ क्षर॑न्ति॒ सिन्ध॑वः। तासान्त्वा॒ सर्वा॑सा रु॒चा। अ॒भिषि॑ञ्चामि॒ वर्च॑सा। स॒मु॒द्र इ॑वासि ग॒ह्मना। सोम॑ इवा॒स्यदाभ्यः। अ॒ग्निरि॑व वि॒श्वत॑ प्र॒त्यङ्ङ्। सूर्य॑ इव॒ ज्योति॑षा वि॒भूः॥२१॥

%2.7.7.7
अ॒पाय्योँ द्रव॑णे॒ रस॑। तम॒हम॒स्मा आ॑मुष्याय॒णाय॑। तेज॑से ब्रह्मवर्च॒साय॑ गृह्णामि। अ॒पां य ऊ॒र्मौ रस॑। तम॒हम॒स्मा आ॑मुष्याय॒णाय॑। ओज॑से वी॒र्या॑य गृह्णामि। अ॒पाय्योँ म॑ध्य॒तो रस॑। तम॒हम॒स्मा आ॑मुष्याय॒णाय॑। पुष्ट्यै प्र॒जन॑नाय गृह्णामि। अ॒पाय्योँ य॒ज्ञियो॒ रस॑। तम॒हम॒स्मा आ॑मुष्याय॒णाय॑। आयु॑षे दीर्घायु॒त्वाय॑ गृह्णामि॥२२॥\anuvakamend[गोष्वोज॑स्वन्त श्रीणा॒म्योजो॑ऽसि॒ तत्ते॒ प्रय॑च्छामि॒ पय॑सा॒ संपि॑पृग्धि॒ माऽस॑द्वि॒भूर्य॒ज्ञियो॒ रसो॒ द्वे च॑]

%2.7.8.1
अ॒भिप्रेहि॑ वी॒रय॑स्व। उ॒ग्रश्चेत्ता॑ सपत्न॒हा। आति॑ष्ठ मित्र॒वर्ध॑नः। तुभ्यं॑ दे॒वा अधि॑ब्रवन्। अ॒ङ्कौ न्य॒ङ्काव॒भित॒ आति॑ष्ठ वृत्रह॒न्रथम्। आ॒तिष्ठ॑न्तं॒ परि॒ विश्वे॑ अभूषन्। श्रियं॒ वसा॑नश्चरति॒ स्वरो॑चाः। म॒हत्तद॒स्यासु॑रस्य॒ नाम॑। आ वि॒श्वरू॑पो अ॒मृता॑नि तस्थौ। अनु॒ त्वेन्द्रो॑ मद॒त्वनु॒ बृह॒स्पति॑॥२३॥

%2.7.8.2
अनु॒ सोमो॒ अन्व॒ग्निरा॑वीत्। अनु॑ त्वा॒ विश्वे॑ दे॒वा अ॑वन्तु। अनु॑ स॒प्त राजा॑नो॒ य उ॒ताभिषि॑क्ताः। अनु॑ त्वा मि॒त्रावरु॑णावि॒हाव॑तम्। अनु॒ द्यावा॑पृथि॒वी वि॒श्वश॑म्भू। सूर्यो॒ अहो॑भि॒रनु॑ त्वाऽवतु। च॒न्द्रमा॒ नक्ष॑त्रै॒रनु॑ त्वाऽवतु। द्यौश्च॑ त्वा पृथि॒वी च॒ प्रचे॑तसा। शु॒क्रो बृ॒हद्दक्षि॑णा त्वा पिपर्तु। अनु॑ स्व॒धा चि॑किता॒ सोमो॑ अ॒ग्निः। आऽयं पृ॑णक्तु॒ रज॑सी उ॒पस्थम्॥२४॥\anuvakamend[बृह॒स्पति॒ सोमो॑ अ॒ग्निरेकं च]

%2.7.9.1
प्र॒जाप॑तिः प्र॒जा अ॑सृजत। ता अ॑स्मात्सृ॒ष्टाः परा॑चीरायन्। स ए॒तं प्र॒जाप॑तिरोद॒नम॑पश्यत्। सोऽन्नं॑ भू॒तो॑ऽतिष्ठत्। ता अ॒न्यत्रा॒न्नाद्य॒मवि॑त्वा। प्र॒जाप॑तिं प्र॒जा उ॒पाव॑र्तन्त। अन्न॑मे॒वैनं॑ भू॒तं पश्य॑न्तीः प्र॒जा उ॒पाव॑र्तन्ते। य ए॒तेन॒ यज॑ते। य उ॑ चैनमे॒वं वेद॑। सर्वा॒ण्यन्ना॑नि भवन्ति॥२५॥

%2.7.9.2
सर्वे॒ पुरु॑षाः। सर्वाण्ये॒वान्ना॒न्यव॑ रुन्धे। सर्वा॒न्पुरु॑षान्। राड॑सि वि॒राड॒सीत्या॑ह। स्वाराज्यमे॒वैन॑ङ्गमयति। यद्धिर॑ण्य॒न्ददा॑ति। तेज॒स्तेनाव॑रुन्धे। यत्ति॑सृध॒न्वम्। वी॒र्य॑न्तेन॑। यदष्ट्राम्॥२६॥

%2.7.9.3
पुष्टि॒न्तेन॑। यत्क॑म॒ण्डलुम्। आयु॒ष्टेन॑। यद्धिर॑ण्यमा ब॒ध्नाति॑। ज्योति॒र्वै हिर॑ण्यम्। ज्योति॑रे॒वास्मि॑न्दधाति। अथो॒ तेजो॒ वै हिर॑ण्यम्। तेज॑ ए॒वात्मन्ध॑त्ते। यदो॑द॒नं प्रा॒श्ञाति॑। ए॒तदे॒व सर्व॑मव॒रुध्य॑॥२७॥

%2.7.9.4
तद॑स्मिन्नेक॒धाऽधात्। रो॒हि॒ण्याङ्का॒र्य॑। यद्ब्राह्म॒ण ए॒व रो॑हि॒णी। तस्मा॑दे॒व। अथो॒ वर्ष्मै॒वैन समा॒नानां करोति। उ॒द्य॒ता सूर्ये॑ण का॒र्य॑। उ॒द्यन्तं॒ वा ए॒त सर्वा प्र॒जाः प्रति॑नन्दन्ति। दि॒दृ॒क्षेण्यो॑ दर्\mbox{}श॒नीयो॑ भवति। य ए॒वं वेद॑। ब्र॒ह्म॒वा॒दिनो॑ वदन्ति॥२८॥

%2.7.9.5
अ॒वेत्यो॑ऽवभृ॒था ३ ना ३ इति॑। यद्द॑र्भपुञ्जी॒लैः प॒वय॑ति। तत्स्वि॑दे॒वावै॑ति। तन्नावै॑ति। त्रि॒भिः प॑वयति। त्रय॑ इ॒मे लो॒काः। ए॒भिरे॒वैनं॑ लो॒कैः प॑वयति। अथो॑ अ॒पां वा ए॒तत्तेजो॒ वर्च॑। यद्द॒र्भाः। यद्द॑र्भपुञ्जी॒लैः प॒वय॑ति। अ॒पामे॒वैन॒न्तेज॑सा॒ वर्च॑सा॒ऽभिषि॑ञ्चति॥२९॥\anuvakamend[भ॒व॒न्त्यष्ट्रा॑मव॒रुध्य॑ वदन्ति द॒र्भा यद्द॑र्भपुञ्जी॒लैः प॒वय॒त्येकं च]

%2.7.10.1
प्र॒जाप॑तिरकामयत ब॒होर्भूयान्त्स्या॒मिति॑। स ए॒तं प॑ञ्चशार॒दीय॑मपश्यत्। तमाऽह॑रत्। तेना॑यजत। ततो॒ वै स ब॒होर्भूया॑नभवत्। यः का॒मये॑त ब॒होर्भूयान्त्स्या॒मिति॑। स प॑ञ्चशार॒दीये॑न यजेत। ब॒होरे॒व भूयान्भवति। म॒रु॒त्स्तो॒मो वा ए॒षः। म॒रुतो॒ हि दे॒वानां॒ भूयि॑ष्ठाः॥३०॥

%2.7.10.2
ब॒हुर्भ॑वति। य ए॒तेन॒ यज॑ते। य उ॑चैनमे॒वं वेद॑। प॒ञ्च॒शा॒र॒दीयो॑ भवति। पञ्च॒ वा ऋ॒तव॑ संवत्स॒रः। ऋ॒तुष्वे॒व सं॑वत्स॒रे प्रति॑तिष्ठति। अथो॒ पञ्चाक्षरा प॒ङ्क्तिः। पाङ्क्तो॑ य॒ज्ञः। य॒ज्ञमे॒वाव॑ रुन्धे। स॒प्त॒द॒श स्तोमा॒ नाति॑ यन्ति। स॒प्त॒द॒शः प्र॒जाप॑तिः। प्र॒जाप॑ते॒राप्त्यै॥३१॥\anuvakamend[भूयि॑ष्ठा यन्ति॒ द्वे च॑]

%2.7.11.1
अ॒गस्त्यो॑ म॒रुद्भ्य॑ उ॒क्ष्णः प्रौक्ष॑त्। तानिन्द्र॒ आद॑त्त। त ए॑नं॒ वज्र॑मु॒द्यत्या॒भ्या॑यन्त। तान॒गस्त्य॑श्चै॒वेन्द्र॑श्च कयाशु॒भीये॑नाशमयताम्। ताञ्छा॒न्तानुपाह्वयत। यत्क॑याशु॒भीयं॒ भव॑ति॒ शान्त्यै। तस्मा॑दे॒त ऐन्द्रामारु॒ता उ॒क्षाण॑ सव॒नीया॑ भवन्ति। त्रय॑ प्रथ॒मेऽह॒न्ना ल॑भ्यन्ते। ए॒वं द्वि॒तीये। ए॒वन्तृ॒तीये॥३२॥

%2.7.11.2
ए॒वञ्च॑तु॒र्थे। पञ्चोत्त॒मेऽह॒न्ना ल॑भ्यन्ते। वर्\mbox{}षि॑ष्ठमिव॒ ह्ये॑तदह॑। वर्\mbox{}षि॑ष्ठः समा॒नानां भवति। य ए॒तेन॒ यज॑ते। य उ॑चैनमे॒वं वेद॑। स्वाराज्यं॒ वा ए॒ष य॒ज्ञः। ए॒तेन॒ वा एक॒या वा॑ कान्द॒मः स्वाराज्यमगच्छत्। स्वराज्यं गच्छति। य ए॒तेन॒ यज॑ते॥३३॥

%2.7.11.3
य उ॑ चैनमे॒वं वेद॑। मा॒रु॒तो वा ए॒ष स्तोम॑। ए॒तेन॒ वै म॒रुतो॑ दे॒वानां॒ भूयि॑ष्ठा अभवन्। भूयि॑ष्ठः समा॒नानां भवति। य ए॒तेन॒ यज॑ते। य उ॑ चैनमे॒वं वेद॑। प॒ञ्च॒शा॒र॒दीयो॒ वा ए॒ष य॒ज्ञः। आ प॑ञ्च॒मात्पुरु॑षा॒दन्न॑मत्ति। य ए॒तेन॒ यज॑ते। य उ॑ चैनमे॒वं वेद॑। स॒प्त॒द॒श स्तोमा॒ नाति॑ यन्ति। स॒प्त॒द॒शः प्र॒जाप॑तिः। प्र॒जाप॑तेरे॒व नैति॑॥३४॥\anuvakamend[तृ॒तीये॑ गच्छति॒ य ए॒तेन॒ यज॑तेऽत्ति॒ य ए॒तेन॒ यज॑ते॒ य उ॑ चैनमे॒वं वेद॒ त्रीणि॑ च (अ॒गस्त्य॒ स्वाराज्यं मारु॒तः प॑ञ्चशार॒दीयो॒ वा ए॒ष य॒ज्ञः स॑प्तद॒शं प्र॒जाप॑तेरे॒व नैति॑ ॥ )]

%2.7.12.1
अ॒स्या जरा॑सो द॒मा म॒रित्रा। अ॒र्चद्धू॑मासो अ॒ग्नय॑ पाव॒काः। श्वि॒ची॒चय॑ श्वा॒त्रासो॑ भुर॒ण्यव॑। व॒न॒र्॒षदो॑ वा॒यवो॒ न सोमा। यजा॑ नो मि॒त्रावरु॑णा। यजा॑ दे॒वा ऋ॒तं बृ॒हत्। अग्ने॒ यक्षि॒ स्वन्दमम्। अश्वि॑ना॒ पिब॑त सु॒तम्। दीद्य॑ग्नी शुचिव्रता। ऋ॒तुना॑ यज्ञवाहसा॥३५॥

%2.7.12.2
द्वे विरू॑पे चरत॒ स्वर्थे। अ॒न्याऽन्या॑ व॒त्समुप॑ धापयेते। हरि॑र॒न्यस्यां॒ भव॑ति स्व॒धावान्॑। शु॒क्रो अ॒न्यस्यान्ददृशे सु॒वर्चा। पू॒र्वा॒प॒रञ्च॑रतो मा॒ययै॒तौ। शिशू॒ क्रीड॑न्तौ॒ परि॑ यातो अध्व॒रम्। विश्वान्य॒न्यो भुव॑नाऽभि॒ चष्टे। ऋ॒तून॒न्यो वि॒दध॑ज्जायते॒ पुन॑। त्रीणि॑ श॒ता त्रीष॒हस्राण्य॒ग्निम्। त्रि॒शच्च॑ दे॒वा नव॑ चाऽसपर्यन्॥३६॥

%2.7.12.3
औक्ष॑न्घृ॒तैरास्तृ॑णन्ब॒र्॒हिर॑स्मै। आदिद्धोता॑र॒न्न्य॑षादयन्त। अ॒ग्निना॒ऽग्निः समि॑ध्यते। क॒विर्गृ॒हप॑ति॒र्युवा। ह॒व्य॒वाड्जु॒ह्वास्यः। अ॒ग्निर्दे॒वानां ज॒ठरम्। पू॒तद॑क्षः क॒विक्र॑तुः। दे॒वो दे॒वेभि॒रा ग॑मत्। अ॒ग्नि॒श्रियो॑ म॒रुतो॑ वि॒श्वकृ॑ष्टयः। आ त्वे॒षमु॒ग्रमव॑ ईमहे व॒यम्॥३७॥

%2.7.12.4
ते स्वा॒निनो॑ रु॒द्रिया॑ व॒र्॒षनि॑र्णिजः। सि॒हा न हे॒षक्र॑तवः सु॒दान॑वः। यदु॑त्त॒मे म॑रुतो मध्य॒मे वा। यद्वा॑ऽव॒मे सु॑भगासो दि॒वि ष्ठ। ततो॑ नो रुद्रा उ॒त वा॒ऽन्वस्य॑। अग्ने॑ वि॒त्ताद्ध॒विषो॒ यद्यजा॑मः। ईडे॑ अ॒ग्नि स्वव॑स॒न्नमो॑भिः। इ॒ह प्र॑स॒प्तो वि च॑ यत्कृ॒तन्न॑। रथै॑रिव॒ प्रभ॑रे वाज॒यद्भि॑। प्र॒द॒क्षि॒णिन्म॒रुता॒ स्तोम॑मृद्ध्याम्॥३८॥

%2.7.12.5
श्रु॒धि श्रु॑त्कर्ण॒ वह्नि॑भिः। दे॒वैर॑ग्ने स॒याव॑भिः। आसी॑दन्तु ब॒र्॒हिषि॑। मि॒त्रो वरु॑णो अर्य॒मा। प्रा॒त॒र्यावा॑णो अध्व॒रम्। विश्वे॑षा॒मदि॑तिर्य॒ज्ञिया॑नाम्। विश्वे॑षा॒मति॑थि॒र्मानु॑षाणाम्। अ॒ग्निर्दे॒वाना॒मव॑ आवृणा॒नः। सु॒मृ॒डी॒को भवतु वि॒श्ववे॑दाः। त्वे अ॑ग्ने सुम॒तिं भिक्ष॑माणाः॥३९॥

%2.7.12.6
दि॒वि श्रवो॑ दधिरे य॒ज्ञिया॑सः। नक्ता॑ च च॒क्रुरु॒षसा॒ विरू॑पे। कृ॒ष्णं च॒ वर्ण॑मरु॒णं च॒ सन्धु॑। त्वाम॑ग्न आदि॒त्यास॑ आ॒स्यम्। त्वाञ्जि॒ह्वा शुच॑यश्चक्रिरे कवे। त्वा रा॑ति॒षाचो॑ अध्व॒रेषु॑ सश्चिरे। त्वे दे॒वा ह॒विर॑द॒न्त्याहु॑तम्। नि त्वा॑ य॒ज्ञस्य॒ साध॑नम्। अग्ने॒ होता॑रमृ॒त्विजम्। व॒नु॒ष्वद्दे॑व धीमहि॒ प्रचे॑तसम्। जी॒रन्दू॒तमम॑र्त्यम्॥४०॥\anuvakamend[य॒ज्ञ॒वा॒ह॒सा॒स॒प॒र्य॒न्व॒यमृ॑द्ध्यां॒ भिक्ष॑माणा॒ प्रचे॑तस॒मेकं च]

%2.7.13.1
तिष्ठा॒ हरी॒ रथ॒ आ यु॒ज्यमा॑ना या॒हि। वा॒युर्न नि॒युतो॑ नो॒ अच्छ॑। पिबा॒स्यन्धो॑ अ॒भिसृ॑ष्टो अ॒स्मे। इन्द्र॒ स्वाहा॑ ररि॒मा ते॒ मदा॑य। कस्य॒ वृषा॑ सु॒ते सचा। नि॒युत्वान्वृष॒भो र॑णत्। वृ॒त्र॒हा सोम॑पीतये। इन्द्रं॑ व॒यम्म॑हाध॒ने। इन्द्र॒मर्भे॑ हवामहे। युजं॑ वृ॒त्रेषु॑ व॒ज्रिणम्॥४१॥

%2.7.13.2
द्वि॒ता यो वृ॑त्र॒हन्त॑मः। वि॒द इन्द्र॑ श॒तक्र॑तुः। उप॑ नो॒ हरि॑भिः सु॒तम्। स सूर॒ आज॒नयं॒ ज्योति॒रिन्द्रम्। अ॒या धि॒या त॒रणि॒रद्रि॑बर्\mbox{}हाः। ऋ॒तेन॑ शु॒ष्मी नव॑मानो अ॒र्कैः। व्यु॑स्रिधो॑ अ॒स्रो अद्रि॑र्बिभेद। उ॒तत्यदा॒श्वश्वि॑यम्। यदि॑न्द्र॒ नाहु॑षी॒ष्वा। अग्रे॑ वि॒क्षु प्रतीद॑यत्॥४२॥

%2.7.13.3
भरे॒ष्विन्द्र सु॒हव हवामहे। अ॒हो॒मुच सु॒कृत॒न्दैव्यं॒ जनम्। अ॒ग्निम्मि॒त्रं वरु॑ण सा॒तये॒ भगम्। द्यावा॑पृथि॒वी म॒रुत॑ स्व॒स्तये। म॒हि क्षेत्रं॑ पु॒रुश्च॒न्द्रं वि वि॒द्वान्। आदित्सखि॑भ्यश्च॒रथ॒ समै॑रत्। इन्द्रो॒ नृभि॑रजन॒द्दीद्या॑नः सा॒कम्। सूर्य॑मु॒षस॑ङ्गा॒तुम॒ग्निम्। उ॒रुन्नो॑ लो॒कमनु॑ नेषि वि॒द्वान्। सुव॑र्व॒ज्ज्योति॒रभ॑य स्व॒स्ति॥४३॥

%2.7.13.4
ऋ॒ष्वा त॑ इन्द्र॒ स्थवि॑रस्य बा॒हू। उप॑स्थेयाम शर॒णा बृ॒हन्ता। आ नो॒ विश्वा॑भिरू॒तिभि॑ स॒जोषा। ब्रह्म॑ जुषा॒णो ह॑र्यश्व याहि। वरी॑वृज॒त्स्थवि॑रेभिः सुशिप्र। अ॒स्मे दध॒द्वृष॑ण॒ शुष्म॑मिन्द्र। इन्द्रा॑य॒ गाव॑ आ॒शिरम्। दु॒दु॒ह्रे व॒ज्रिणे॒ मधु॑। यत्सी॑मुपह्व॒रे वि॒दत्। तास्ते॑ वज्रिन्धे॒नवो॑ जोजयुर्नः॥४४॥

%2.7.13.5
गभ॑स्तयो नि॒युतो॑ वि॒श्ववा॑राः। अह॑रह॒र्भूय॒ इज्जोगु॑वानाः। पू॒र्णा इ॑न्द्र क्षु॒मतो॒ भोज॑नस्य। इ॒मान्ते॒ धियं॒ प्र भ॑रे म॒हो म॒हीम्। अ॒स्य स्तो॒त्रे धि॒षणा॒ यत्त॑ आन॒जे। तमु॑त्स॒वे च॑ प्रस॒वे च॑ सास॒हिम्। इन्द्रं॑ दे॒वास॒ शव॑सा मद॒न्ननु॑॥४५॥\anuvakamend[व॒ज्रिण॑मयत्स्व॒स्ति जो॑जयुर्नः स॒प्त च॑]

%2.7.14.1
प्र॒जाप॑तिः प॒शून॑सृजत। तेऽस्मात्सृ॒ष्टाः परां च आयन्। तान॑ग्निष्टो॒मेन॒ नाप्नोत्। तानु॒क्थ्ये॑न॒ नाप्नोत्। तान्थ्षो॑ड॒शिना॒ नाप्नोत्। तान्रात्रि॑या॒ नाप्नोत्। तान्त्स॒न्धिना॒ नाप्नोत्। सोऽग्निम॑ब्रवीत्। इ॒मान्म॑ ई॒प्सेति॑। तान॒ग्निस्त्रि॒वृता॒ स्तोमे॑न॒ नाप्नोत्॥४६॥

%2.7.14.2
स इन्द्र॑मब्रवीत्। इ॒मान्म॑ ई॒प्सेति॑। तानिन्द्र॑ पञ्चद॒शेन॒ स्तोमे॑न॒ नाप्नोत्। स विश्वान्दे॒वान॑ब्रवीत्। इ॒मान्म॑ ईप्स॒तेति॑। तान् विश्वे॑दे॒वाः स॑प्तद॒शेन॒ स्तोमे॑न॒ नाप्नु॑वन्। स विष्णु॑मब्रवीत्। इ॒मान्म॑ ई॒प्सेति॑। ताऩ् विष्णु॑रेकवि॒शेन॒ स्तोमे॑नाप्नोत्। वा॒र॒व॒न्तीये॑नावारयत॥४७॥

%2.7.14.3
इ॒दं विष्णु॒र्वि च॑क्रम॒ इति॒ व्य॑क्रमत। यस्मात्प॒शव॒ प्रप्रेव॒ भ्रशे॑रन्। स ए॒तेन॑ यजेत। यदाप्नोत्। तद॒प्तोर्याम॑स्याप्तोर्याम॒त्वम्। ए॒तेन॒ वै दे॒वा जैत्वा॑नि जि॒त्वा। यङ्काम॒मका॑मयन्त॒ तमाप्नुवन्। यङ्काम॑ङ्का॒मय॑ते। तमे॒तेनाप्नोति॥४८॥\anuvakamend[स्तोमे॑न॒ नाप्नो॑दवारयत॒ नव॑ च]

%2.7.15.1
व्या॒घ्रो॑ऽयम॒ग्नौ च॑रति॒ प्रवि॑ष्टः। ऋषी॑णां पु॒त्रो अ॑भिशस्ति॒पा अ॒यम्। न॒म॒स्का॒रेण॒ नम॑सा ते जुहोमि। मा दे॒वानां मिथु॒याक॑र्म भा॒गम्। सावी॒र्॒हि दे॑व प्रस॒वाय॑ पित्रे। व॒र्ष्माण॑मस्मै वरि॒माण॑मस्मै। अथा॒स्मभ्य सवितः स॒र्वता॑ता। दि॒वेदि॑व॒ आ सु॑वा॒ भूरि॑ प॒श्वः। भू॒तो भू॒तेषु॑ चरति॒ प्रवि॑ष्टः। स भू॒ताना॒मधि॑पतिर्बभूव॥४९॥

%2.7.15.2
तस्य॑ मृ॒त्यौ च॑रति राज॒सूयम्। स राजा॑ रा॒ज्यमनु॑ मन्यतामि॒दम्। येभि॒ शिल्पै पप्रथा॒नामदृहत्। येभि॒र्द्याम॒भ्यपिशत्प्र॒जाप॑तिः। येभि॒र्वाचं॑ वि॒श्वरू॑पा स॒मव्य॑यत्। तेने॒मम॑ग्न इ॒ह वर्च॑सा॒ सम॑ङ्ग्धि। येभि॑रादि॒त्यस्तप॑ति॒ प्र के॒तुभि॑। येभि॒ सूर्यो॑ ददृ॒शे चि॒त्रभा॑नुः। येभि॒र्वाचं॑ पुष्क॒लेभि॒रव्य॑यत्। तेने॒मम॑ग्न इ॒ह वर्च॑सा॒ सम॑ङ्ग्धि॥५०॥

%2.7.15.3
आऽयं भा॑तु॒ शव॑सा॒ पञ्च॑ कृ॒ष्टीः। इन्द्र॑ इव ज्ये॒ष्ठो भ॑वतु प्र॒जावान्॑। अ॒स्मा अ॑स्तु पुष्क॒लञ्चि॒त्रभा॑नु। आऽयं पृ॑णक्तु॒ रज॑सी उ॒पस्थम्। यत्ते॒ शिल्प॑ङ्कश्यप रोच॒नाव॑त्। इ॒न्द्रि॒याव॑त्पुष्क॒लञ्चि॒त्रभा॑नु। यस्मि॒न्त्सूर्या॒ अर्पि॑ताः स॒प्त सा॒कम्। तस्मि॒न्राजा॑न॒मधि॒ विश्र॑ये॒मम्। द्यौर॑सि पृथि॒व्य॑सि। व्या॒घ्रो वैया॒घ्रेऽधि॑ ॥५१॥

%2.7.15.4
विश्र॑यस्व॒ दिशो॑ म॒हीः। विश॑स्त्वा॒ सर्वा॑ वाञ्छन्तु। मा त्वद्रा॒ष्ट्रमधि॑ भ्रशत्। या दि॒व्या आप॒ पय॑सा सम्बभू॒वुः। या अ॒न्तरि॑क्ष उ॒त पार्थि॑वी॒र्याः। तासान्त्वा॒ सर्वा॑सा रु॒चा। अ॒भिषि॑ञ्चामि॒ वर्च॑सा। अ॒भि त्वा॒ वर्च॑साऽसिचन्दि॒व्येन॑। पय॑सा स॒ह। यथासा॑ राष्ट्र॒वर्ध॑नः॥५२॥

%2.7.15.5
तथा त्वा सवि॒ता क॑रत्। इन्द्रं॒ विश्वा॑ अवीवृधन्। स॒मु॒द्रव्य॑चस॒ङ्गिर॑। र॒थीत॑म रथी॒नाम्। वाजा॑ना॒ सत्प॑तिं॒ पतिम्। वस॑वस्त्वा पु॒रस्ता॑द॒भिषि॑ञ्चन्तु गाय॒त्रेण॒ छन्द॑सा। रु॒द्रास्त्वा॑ दक्षिण॒तो॑ऽभिषि॑ञ्चन्तु॒ त्रैष्टु॑भेन॒ छन्द॑सा। आ॒दि॒त्यास्त्वा॑ प॒श्चाद॒भिषि॑ञ्चन्तु॒ जाग॑तेन॒ छन्द॑सा। विश्वे त्वा दे॒वा उ॑त्तर॒तो॑ऽभिषि॑ञ्च॒न्त्वानु॑ष्टुभेन॒ छन्द॑सा। बृह॒स्पति॑स्त्वो॒परि॑ष्टाद॒भिषि॑ञ्चतु॒ पाङ्क्ते॑न॒ छन्द॑सा॥५३॥

%2.7.15.6
अ॒रु॒णन्त्वा॒ वृक॑मु॒ग्रङ्ख॑जङ्क॒रम्। रोच॑मानं म॒रुता॒मग्रे॑ अ॒र्चिष॑। सूर्य॑वन्तं म॒घवा॑नं विषास॒हिम्। इन्द्र॑मु॒क्थेषु॑ नाम॒हूत॑म हुवेम। प्र बा॒हवा॑ सिसृतञ्जी॒वसे॑ नः। आ नो॒ गव्यू॑तिमुक्षतङ्घृ॒तेन॑। आ नो॒ जने श्रवयतय्युँवाना। श्रु॒तं मे॑ मित्रावरुणा॒ हवे॒मा। इन्द्र॑स्य ते वीर्य॒कृत॑। बा॒हू उ॒पाव॑ हरामि॥५४॥\anuvakamend[ब॒भू॒वाव्य॑य॒त्तेने॒मम॑ग्न इ॒ह वर्च॑सा॒ सम॑ङ्ग्धि॒ वैया॒घ्रेऽधि॑ राष्ट्र॒वर्ध॑न॒ पाङ्क्ते॑न॒ छन्द॑सो॒पाव॑हरामि]

%2.7.16.1
अ॒भि प्रेहि॑ वी॒रय॑स्व। उ॒ग्रश्चेत्ता॑ सपत्न॒हा। आति॑ष्ठ वृत्र॒हन्त॑मः। तुभ्यं॑ दे॒वा अधि॑ब्रवन्। अ॒ङ्कौ न्य॒ङ्काव॒भितो॒ रथ॒य्यौँ। ध्वा॒न्तं वा॑ता॒ग्रमनु॑ स॒ञ्चर॑न्तौ। दू॒रेहे॑तिरिन्द्रि॒यावान्पत॒त्री। ते नो॒ऽग्नय॒ पप्र॑यः पारयन्तु। नम॑स्त ऋषे गद। अव्य॑थायै त्वा स्व॒धायै त्वा॥५५॥

%2.7.16.2
मा न॑ इन्द्रा॒भित॒स्त्वदृ॒ष्वारि॑ष्टासः। ए॒वा ब्र॑ह्म॒न्तवेद॑स्तु। तिष्ठा॒ रथे॒ अधि॒ यद्वज्र॑हस्तः। आ र॒श्मीन्दे॑व युवसे॒ स्वश्व॑। आ ति॑ष्ठ वृत्रहन्ना॒तिष्ठ॑न्तं॒ परि॑। अनु॒ त्वेन्द्रो॑ मद॒त्वनु॑ त्वा मि॒त्रावरु॑णौ। द्यौश्च॑ त्वा पृथि॒वी च॒ प्रचे॑तसा। शु॒क्रो बृ॒द्दक्षि॑णा त्वा पिपर्तु। अनु॑ स्व॒धा चि॑किता॒ सोमो॑ अ॒ग्निः। अनु॑ त्वाऽवतु सवि॒ता स॒वेन॑॥५६॥

%2.7.16.3
इन्द्रं॒ विश्वा॑ अवीवृधन्। स॒मु॒द्रव्य॑चस॒ङ्गिर॑। र॒थीत॑म रथी॒नाम्। वाजा॑ना॒ सत्प॑तिं॒ पतिम्। परि॑मा से॒न्या घोषा। ज्यानां वृञ्जन्तु गृ॒ध्नव॑। मे॒थि॒ष्ठाः पिन्व॑माना इ॒ह। माङ्गोप॑तिम॒भि संवि॑शन्तु। तन्मेऽनु॑मति॒रनु॑ मन्यताम्। तन्मा॒ता पृ॑थि॒वी तत्पि॒ता द्यौः॥५७॥

%2.7.16.4
तद्ग्रावा॑णः सोम॒सुतो॑ मयो॒भुव॑। तद॑श्विना शृणुत सौभगा यु॒वम्। अव॑ ते॒ हेड॒ उदु॑त्त॒मम्। ए॒ना व्या॒घ्रं प॑रिषस्वजा॒नाः। सि॒ह हि॑न्वन्ति मह॒ते सौभ॑गाय। स॒मु॒द्रन्न सु॒हव॑न्तस्थि॒वासम्। म॒र्मृ॒ज्यन्ते द्वी॒पिन॑म॒प्स्व॑न्तः। उद॒सावे॑तु॒ सूर्य॑। उदि॒दं मा॑म॒कं वच॑। उदि॑हि देव सूर्य। स॒ह व॒ग्नुना॒ मम॑। अ॒हं वा॒चो वि॒वाच॑नम्। मयि॒ वाग॑स्तु धर्ण॒सिः। यन्तु॑ न॒दयो॒ वर्\mbox{}ष॑न्तु प॒र्जन्या। सु॒पि॒प्प॒ला ओष॑धयो भवन्तु। अन्न॑वतामोद॒नव॑तामा॒मिक्ष॑वताम्। ए॒षा राजा॑ भूयसाम्॥५८॥\anuvakamend[स्व॒धायै त्वा स॒वेन॒ द्यौः सूर्य स॒प्त च॑]

%2.7.17.1
ये के॒शिन॑ प्रथ॒माः स॒त्रमास॑त। येभि॒राभृ॑तं॒ यदि॒दं वि॒रोच॑ते। तेभ्यो॑ जुहोमि बहु॒धा घृ॒तेन॑। रा॒यस्पोषे॑णे॒मं वर्च॑सा॒ स सृ॑जाथ। नर्ते ब्रह्म॑ण॒स्तप॑सो विमो॒कः। द्वि॒नाम्नी॑ दी॒क्षा व॒शिनी॒ ह्यु॑ग्रा। प्र केशा सु॒वते॑ का॒ण्डिनो॑ भवन्ति। तेषां ब्र॒ह्मेदीशे॒ वप॑नस्य॒ नान्यः। आ रो॑ह॒ प्रोष्ठं॒ विष॑हस्व॒ शत्रून्॑। अवास्राग्दी॒क्षा व॒शिनी॒ ह्यु॑ग्रा॥५९॥

%2.7.17.2
दे॒हि दक्षि॑णां॒ प्रति॑र॒स्वायु॑। अथा॑मुच्यस्व॒ वरु॑णस्य॒ पाशात्। येनाव॑पत्सवि॒ता क्षु॒रेण॑। सोम॑स्य॒ राज्ञो॒ वरु॑णस्य वि॒द्वान्। तेन॑ ब्रह्माणो वपते॒दम॒स्योर्जेमम्। र॒य्या वर्च॑सा॒ स सृ॑जाथ। मा ते॒ केशा॒ननु॑ गा॒द्वर्च॑ ए॒तत्। तथा॑ धा॒ता क॑रोतु ते। तुभ्य॒मिन्द्रो॒ बृह॒स्पति॑। स॒वि॒ता वर्च॒ आद॑धात्॥६०॥

%2.7.17.3
तेभ्यो॑ नि॒धानं॑ बहु॒धा व्यैच्छ\sn{}। अ॒न्त॒रा द्यावा॑पृथि॒वी अ॒पः सुव॑। द॒र्भ॒स्त॒म्बे वी॒र्य॑कृते नि॒धाय॑। पौस्ये॑ने॒मं वर्च॑सा॒ स सृ॑जाथ। बल॑न्ते बाहु॒वोः स॑वि॒ता द॑धातु। सोम॑स्त्वाऽनक्तु॒ पय॑सा घृ॒तेन॑। स्त्री॒षु रू॒पम॑श्विनै॒तन्नि ध॑त्तम्। पौस्ये॑ने॒मं वर्च॑सा॒ ससृ॑जाथ। यत्सी॒मन्त॒ङ्कङ्क॑तस्ते लि॒लेख॑। यद्वा क्षु॒रः प॑रिव॒वर्ज॒ वपस्ते। स्त्री॒षु रू॒पम॑श्विनै॒तन्नि ध॑त्तम्। पौस्ये॑ने॒म स सृ॑जाथो वी॒र्ये॑ण॥६१॥\anuvakamend[अवास्राग्दी॒क्षा व॒शिनी॒ ह्यु॑ग्राऽद॑धाद्व॒वर्ज॒ वप स्ते॒ द्वे च॑]

%2.7.18.1
इन्द्रं॒ वै स्वाविशो॑ म॒रुतो॒ नापा॑चायन्। सोऽन॑पचाय्यमान ए॒तं वि॑घ॒नम॑पश्यत्। तमाऽह॑रत्। तेना॑यजत। तेनै॒वासा॒न्त स स्त॒म्भव्व्यँ॑हन्। यद्व्यह\sn{}। तद्वि॑घ॒नस्य॑ विघन॒त्वम्। वि पा॒प्मानं॒ भ्रातृ॑व्य हते। य ए॒तेन॒ यज॑ते। य उ॑ चैनमे॒वं वेद॑॥६२॥

%2.7.18.2
य राजा॑नं॒ विशो॒ नाप॒चाये॑युः। यो वा ब्राह्म॒णस्तम॑सा पा॒प्मना॒ प्रावृ॑त॒ स्यात्। स ए॒तेन॑ यजेत। वि॒घ॒नेनै॒वैन॑द्वि॒हत्य॑। वि॒शामाधि॑पत्यं गच्छति। तस्य॒ द्वे द्वा॑द॒शे स्तो॒त्रे भव॑तः। द्वे च॑तुर्वि॒शे। औद्भि॑द्यमे॒व तत्। ए॒तद्वै क्ष॒त्रस्यौद्भि॑द्यम्। यद॑स्मै॒ स्वाविशो॑ ब॒लि हर॑न्ति॥६३॥

%2.7.18.3
हर॑न्त्यस्मै॒ विशो॑ ब॒लिम्। ऐन॒मप्र॑तिख्यातं गच्छति। य ए॒वं वेद॑। प्र॒बाहु॒ग्वा अग्रे क्ष॒त्राण्याते॑पुः। तेषा॒मिन्द्र॑ क्ष॒त्राण्याद॑त्त। न वा इ॒मानि॑ क्ष॒त्राण्य॑भूव॒न्निति॑। तन्नक्ष॑त्राणां नक्षत्र॒त्वम्। आ श्रेय॑सो॒ भ्रातृ॑व्यस्य॒ तेज॑ इन्द्रि॒यन्द॑त्ते। य ए॒तेन॒ यज॑ते। य उ॑ चैनमे॒वं वेद॑॥६४॥

%2.7.18.4
तद्यथा॑ ह॒ वै स॑च॒क्रिणौ॒ कप्ल॑कावु॒पाव॑हितौ॒ स्याताम्। ए॒वमे॒तौ यु॒ग्मन्तौ॒ स्तोमौ। अ॒युक्षु॒ स्तोमे॑षु क्रियेते। पा॒प्मनोऽप॑हत्यै। अप॑ पा॒प्मानं॒ भ्रातृ॑व्य हते। य ए॒तेन॒ यज॑ते। य उ॑ चैनमे॒वं वेद॑। तद्यथा॑ ह॒ वै सू॑तग्राम॒ण्य॑। ए॒वञ्छन्दासि। तेष्व॒सावा॑दि॒त्यो बृ॑ह॒तीर॒भ्यू॑ढः॥६५॥

%2.7.18.5
स॒तोबृ॑हतीषु स्तुवते स॒तो बृ॑हन्। प्र॒जया॑ प॒शुभि॑रसा॒नीत्ये॒व। व्यति॑षक्ताभिः स्तुवते। व्यति॑षक्तं॒ वै क्ष॒त्रं वि॒शा। वि॒शैवैनं॑ क्ष॒त्रेण॒ व्यति॑षजति। व्यति॑षक्ताभिः स्तुवते। व्यति॑षक्तो॒ वै ग्रा॑म॒णीः स॑जा॒तैः। स॒जा॒तैरे॒वैन॒व्व्यँति॑षजति। व्यति॑षक्ताभिः स्तुवते। व्यति॑षक्तो॒ वै पुरु॑षः पा॒प्मभि॑। व्यति॑षक्ताभिरे॒वास्य॑ पा॒प्मनो॑ नुदते॥६६॥\anuvakamend[वेद॒ हर॑न्त्येनमे॒वं वेदा॒भ्यू॑ढः पा॒प्मभि॒रेकं च]




\prashnaend{त्रि॒वृद्यदाग्ने॒योऽग्निमु॑खा॒ ह्यृद्धि॒र्यदाग्ने॒य आग्ने॒यो न वै सोमे॑न॒ यो वै सोमे॑नै॒ष गो॑स॒वः सि॒हे॑ऽभि प्रेहि॑ मित्र॒वर्ध॑नः प्र॒जाप॑ति॒स्ता ओ॑द॒नं प्र॒जाप॑तिरकामयत ब॒होर्भूया॑न॒गस्त्यो॒स्या जरा॑स॒स्तिष्ठा॒ हरी प्र॒जाप॑तिः प॒शून्व्या॒घ्रो॑ऽयम॒भिप्रेहि॑ वृत्र॒हन्त॑मो॒ ये के॒शिन॒ इन्द्रं॒ वा अ॒ष्टाद॑श॥१८॥}{त्रि॒वृद्यो वै सोमे॒नायु॑रसि ब॒हुर्भ॑वति॒ तिष्ठा॒ हरी॒रथ॒ आयं भा॑तु॒ तेभ्यो॑ नि॒धान॒ षट्थ्ष॑ष्टिः॥६६॥}{त्रि॒वृत्पा॒प्मनो॑ नुदते॥}{हरि॑ ओम्॥}{इति श्रीकृष्णयजुर्वेदीयतैत्तिरीयब्राह्मणे द्वितीयाष्टके सप्तमः प्रपाठकः समाप्तः॥}
\clearpage
