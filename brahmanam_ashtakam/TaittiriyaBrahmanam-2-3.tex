\sect{तृतीयः प्रश्नः}
\setcounter{anuvakam}{0}
\dnsub{तैत्तिरीयब्राह्मणे द्वितीयाष्टके तृतीयः प्रपाठकः}

%2.3.1.1
ब्र॒ह्म॒वा॒दिनो॑ वदन्ति। किञ्चतु॑र्‌होतृणाञ्चतुर्‌होतृ॒त्वमिति॑। यदे॒वैषु च॑तु॒र्धा होता॑रः। तेन॒ चतु॑र्‌होतारः। तस्मा॒च्चतु॑र्‌होतार उच्यन्ते। तच्चतुर्॑होतृणाञ्चतुर्‌होतृ॒त्वम्। सोमो॒ वै चतु॑र्‌होता। अ॒ग्निः पञ्च॑होता। धा॒ता षड्ढो॑ता। इन्द्र॑ स॒प्तहो॑ता॥१॥

%2.3.1.2
प्र॒जाप॑ति॒र्दश॑होता। य ए॒वञ्चतु॑र्‌होतृणा॒मृद्धिं॒ वेद॑। ऋ॒ध्नोत्ये॒व। य ए॑षामे॒वं ब॒न्धुतां॒ वेद॑। बन्धु॑मान्भवति। य ए॑षामे॒वं कॢप्तिं॒ वेद॑। कल्प॑तेऽस्मै। य ए॑षामे॒वमा॒यत॑नं॒ वेद॑। आ॒यत॑नवान्भवति। य ए॑षामे॒वं प्र॑ति॒ष्ठां वेद॑॥२॥

%2.3.1.3
प्रत्ये॒व ति॑ष्ठति। ब्र॒ह्म॒वा॒दिनो॑ वदन्ति। दश॑होता॒ चतु॑र्‌होता। पञ्च॑होता॒ षड्ढो॑ता स॒प्तहो॑ता। अथ॒ कस्मा॒च्चतु॑र्‌होतार उच्यन्त॒ इति॑। इन्द्रो॒ वै चतु॑र्‌होता। इन्द्र॒ खलु॒ वै श्रेष्ठो॑ दे॒वता॑नामुप॒देश॑नात्। य ए॒वमिन्द्र॒ श्रेष्ठं॑ दे॒वता॑नामुप॒देश॑ना॒द्वेद॑। वसि॑ष्ठः समा॒नानां भवति। तस्मा॒च्छ्रेष्ठ॑मा॒यन्तं॑ प्रथ॒मेनै॒वानु॑ बुध्यन्ते। अ॒यमाग\sn{}। अ॒यमवा॑सा॒दिति॑। की॒र्तिर॑स्य॒ पूर्वाऽऽग॑च्छति ज॒नता॑माय॒तः। अथो॑ एनं प्रथ॒मेनै॒वानु॑ बुध्यन्ते। अ॒यमाग\sn{}। अ॒यमवा॑सा॒दिति॑॥३॥\anuvakamend[स॒प्तहो॑ता प्रति॒ष्ठां वेद॑ बुध्यन्ते॒ षट्च॑]

%2.3.2.1
दक्षि॑णां प्रतिग्रही॒ष्यन्त्स॒प्तद॑श॒कृत्वोऽपान्यात्। आ॒त्मान॑मे॒व समि॑न्धे। तेज॑से वी॒र्या॑य। अथो प्र॒जाप॑तिरे॒वैनां भू॒त्वा प्रति॑ गृह्णाति। आ॒त्मनोऽनार्त्यै। यद्ये॑न॒मार्त्वि॑ज्याद्वृ॒त सन्त॑न्नि॒र्‌हरे॑रन्। आग्नीध्रे जुहुया॒द्दश॑होतारम्। च॒तु॒र्गृ॒ही॒तेनाज्ये॑न। पु॒रस्तात्प्र॒त्यङ्तिष्ठ\sn{}। प्र॒ति॒लो॒मं वि॒ग्राहम्॥४॥

%2.3.2.2
प्रा॒णाने॒वास्योप॑ दासयति। यद्ये॑नं॒ पुन॑रुप॒ शिक्षे॑युः। आग्नीध्र ए॒व जु॑हुया॒द्दश॑होतारम्। च॒तु॒र्गृ॒ही॒तेनाज्ये॑न। प॒श्चात्प्राङासी॑नः। अ॒नु॒लो॒ममवि॑ग्राहम्। प्रा॒णाने॒वास्मै॑ कल्पयति। प्राय॑श्चित्ती॒ वाग्घोतेत्यृ॑तुमु॒खऋ॑तुमुखे जुहोति। ऋ॒तूने॒वास्मै॑ कल्पयति। कल्प॑न्तेऽस्मा ऋ॒तव॑॥५॥

%2.3.2.3
कॢ॒प्ता अ॑स्मा ऋ॒तव॒ आय॑न्ति। षड्ढो॑ता॒ वै भू॒त्वा प्र॒जाप॑तिरि॒द सर्व॑मसृजत। स मनो॑ऽसृजत। मन॒सोऽधि॑ गाय॒त्रीम॑सृजत। तद्गा॑य॒त्रीं यश॑ आर्च्छत्। तामाऽल॑भत। गा॒य॒त्रि॒या अधि॒ छन्दास्यसृजत। छन्दो॒भ्योऽधि॒ साम॑। तत्साम॒ यश॑ आर्च्छत्। तदाऽल॑भत॥६॥

%2.3.2.4
साम्नोऽधि॒ यजूष्यसृजत। यजु॒र्भ्योऽधि॒ विष्णुम्। तद्विष्णुं॒ यश॑ आर्च्छत्। तमाऽल॑भत। विष्णो॒रध्योष॑धीरसृजत। ओष॑धी॒भ्योऽधि॒ सोमम्। तत्सोमं॒ यश॑ आर्च्छत्। तमाऽल॑भत। सोमा॒दधि॑ प॒शून॑सृजत। प॒शुभ्योऽधीन्द्रम्॥७॥

%2.3.2.5
तदिन्द्रं॒ यश॑ आर्च्छत्। तदे॑न॒न्नाति॒ प्राच्य॑वत। इन्द्र॑ इव यश॒स्वी भ॑वति। य ए॒वं वेद॑। नैनं॒ यशोऽति॒ प्रच्य॑वते। यद्वा इ॒दङ्किं च॑। तत्सर्व॑मुत्ता॒न ए॒वाङ्गी॑र॒सः प्रत्य॑गृह्णात्। तदे॑नं॒ प्रति॑गृहीत॒न्नाहि॑नत्। यत्किं च॑ प्रतिगृह्णी॒यात्। तत्सर्व॑मुत्ता॒नस्त्वाङ्गीर॒सः प्रति॑गृह्णा॒त्वित्ये॒व प्रति॑गृह्णीयात्। इ॒यं वा उ॑त्ता॒न आङ्गीर॒सः। अ॒नयै॒वैन॒त्प्रति॑ गृह्णाति। नैन हिनस्ति। ब॒र्॒हिषा॒ प्रती॑या॒द्गां वाऽश्वं॑ वा। ए॒तद्वै प॑शू॒नां प्रि॒यं धाम॑। प्रि॒येणै॒वैनं॒ धाम्ना॒ प्रत्ये॑ति॥८॥\anuvakamend[वि॒ग्राह॑मृ॒तव॒स्तदाऽल॑भ॒तेन्द्र॑ङ्गृह्णीया॒थ्षट्च॑]

%2.3.3.1
यो वा अवि॑द्वान्निव॒र्तय॑ते। विशी॑र्‌षा॒ सपाप्मा॒ऽमुष्मि॑ल्लोँ॒के भ॑वति। अथ॒ यो वि॒द्वान्नि॑व॒र्तय॑ते। सशी॑र्‌षा॒ विपाप्मा॒ऽमुष्मि॑ल्लोँ॒के भ॑वति। दे॒वता॒ वै स॒प्त पुष्टि॑कामा॒ न्य॑वर्तयन्त। अ॒ग्निश्च॑ पृथि॒वी च॑। वा॒युश्चा॒न्तरि॑क्षं च। आ॒दि॒त्यश्च॒ द्यौश्च॑ च॒न्द्रमा। अ॒ग्निर्न्य॑वर्तयत। स सा॑ह॒स्रम॑पुष्यत्॥९॥

%2.3.3.2
पृ॒थि॒वी न्य॑वर्तयत। सौष॑धीभि॒र्वन॒स्पति॑भिरपुष्यत्। वा॒युर्न्य॑वर्तयत। स मरी॑चीभिरपुष्यत्। अ॒न्तरि॑क्ष॒न्न्य॑वर्तयत। तद्वयो॑भिरपुष्यत्। आ॒दि॒त्यो न्य॑वर्तयत। स र॒श्मिभि॑रपुष्यत्। द्यौर्न्य॑वर्तयत। सा नक्ष॑त्रैरपुष्यत्। च॒न्द्रमा॒ न्य॑वर्तयत। सौ॑ऽहोरा॒त्रैर॑र्धमा॒सैर्मासैर्॑ऋ॒तुभि॑ संवत्स॒रेणा॑पुष्यत्। तान्पोषान्पुष्यति। यास्तेऽपु॑ष्यन्। य ए॒वं वि॒द्वान्नि च॑ व॒र्तय॑ते॒ परि॑ च॥१०॥\anuvakamend[अ॒पु॒ष्य॒न्नक्ष॑त्रैरपुष्य॒त्पञ्च॑ च]

%2.3.4.1
तस्य॒ वा अ॒ग्नेर्‌हिर॑ण्यं प्रतिजग्र॒हुष॑। अ॒र्धमि॑न्द्रि॒यस्यापाक्रामत्। तदे॒तेनै॒व प्रत्य॑गृह्णात्। तेन॒ वै सोऽर्धमि॑न्द्रि॒यस्या॒त्मन्नु॒पाध॑त्त। अ॒र्धमि॑न्द्रि॒यस्या॒त्मन्नु॒पाध॑त्ते। य ए॒वं वि॒द्वान् हिर॑ण्यं प्रतिगृ॒ह्णाति॑। अथ॒ योऽवि॑द्वान्प्रति गृ॒ह्णाति॑। अ॒र्धम॑स्येन्द्रि॒यस्याप॑ क्रामति। तस्य॒ वै सोम॑स्य॒ वास॑ प्रतिजग्र॒हुष॑। तृती॑यमिन्द्रि॒यस्यापाक्रामत्॥११॥

%2.3.4.2
तदे॒तेनै॒व प्रत्य॑गृह्णात्। तेन॒ वै स तृती॑यमिन्द्रि॒यस्या॒त्मन्नु॒पाध॑त्त। तृती॑यमिन्द्रि॒यस्या॒त्मन्नु॒पाध॑त्ते। य ए॒वं वि॒द्वान् वास॑ प्रतिगृ॒ह्णाति॑। अथ॒ योऽवि॑द्वान्प्रति गृ॒ह्णाति॑। तृती॑यमस्येन्द्रि॒यस्याप॑ क्रामति। तस्य॒ वै रु॒द्रस्य॒ गां प्र॑तिजग्र॒हुष॑। च॒तु॒र्थमि॑न्द्रि॒यस्यापाक्रामत्। तामे॒तेनै॒व प्रत्य॑गृह्णात्। तेन॒ वै स च॑तु॒र्थमि॑न्द्रि॒यस्या॒त्मन्नु॒पाध॑त्त॥१२॥

%2.3.4.3
च॒तु॒र्थमि॑न्द्रि॒यस्या॒त्मन्नु॒पाध॑त्ते। य ए॒वं वि॒द्वान्गां प्र॑तिगृ॒ह्णाति॑। अथ॒ योऽवि॑द्वान्प्रतिगृ॒ह्णाति॑। च॒तु॒र्थम॑स्येन्द्रि॒यस्याप॑ क्रामति। तस्य॒ वै वरु॑ण॒स्याश्वं॑ प्रतिजग्र॒हुष॑। प॒ञ्च॒ममि॑न्द्रि॒यस्यापाक्रामत्। तमे॒तेनै॒व प्रत्य॑गृह्णात्। तेन॒ वै स प॑ञ्च॒ममि॑न्द्रि॒यस्या॒त्मन्नु॒पाध॑त्त। प॒ञ्च॒ममि॑न्द्रि॒यस्या॒त्मन्नु॒पाध॑त्ते। य ए॒वं वि॒द्वानश्वं॑ प्रतिगृ॒ह्णाति॑॥१३॥

%2.3.4.4
अथ॒ योऽवि॑द्वान्प्रतिगृ॒ह्णाति॑। प॒ञ्च॒मम॑स्येन्द्रि॒यस्याप॑ क्रामति। तस्य॒ वै प्र॒जाप॑ते॒ पुरु॑षं प्रतिजग्र॒हुष॑। ष॒ष्ठमि॑न्द्रि॒यस्यापाक्रामत्। तमे॒तेनै॒व प्रत्य॑गृह्णात्। तेन॒ वै स ष॒ष्ठमि॑न्द्रि॒यस्या॒त्मन्नु॒पाध॑त्त। ष॒ष्ठमि॑न्द्रि॒यस्या॒त्मन्नु॒पाध॑त्ते। य ए॒वं वि॒द्वान्पुरु॑षं प्रतिगृ॒ह्णाति॑। अथ॒ योऽवि॑द्वान्प्रतिगृ॒ह्णाति॑। ष॒ष्ठम॑स्येन्द्रि॒यस्याप॑ क्रामति॥१४॥

%2.3.4.5
तस्य॒ वै मनो॒स्तल्पं॑ प्रतिजग्र॒हुष॑। स॒प्त॒ममि॑न्द्रि॒यस्यापाक्रामत्। तमे॒तेनै॒व प्रत्य॑गृह्णात्। तेन॒ वै स स॑प्त॒ममि॑न्द्रि॒यस्या॒त्मन्नु॒पाध॑त्त। स॒प्त॒ममि॑न्द्रि॒यस्या॒त्मन्नु॒पाध॑त्ते। य ए॒वं वि॒द्वास्तल्पं॑ प्रति गृ॒ह्णाति॑। अथ॒ योऽवि॑द्वान्प्रति गृ॒ह्णाति॑। स॒प्त॒मम॑स्येन्द्रि॒यस्याप॑ क्रामति। तस्य॒ वा उ॑त्ता॒नस्याङ्गीर॒सस्याप्रा॑णत्प्रतिजग्र॒हुष॑। अ॒ष्ट॒ममि॑न्द्रि॒यस्यापाक्रामत्॥१५॥

%2.3.4.6
तदे॒तेनै॒व प्रत्य॑गृह्णात्। तेन॒ वै सोऽष्ट॒ममि॑न्द्रि॒यस्या॒त्मन्नु॒पाध॑त्त। अ॒ष्ट॒ममि॑न्द्रि॒यस्या॒त्मन्नु॒पाध॑त्ते। य ए॒वं वि॒द्वानप्रा॑णत्प्रतिगृ॒ह्णाति॑। अथ॒ योऽवि॑द्वान्प्रतिगृ॒ह्णाति॑। अ॒ष्ट॒मम॑स्येन्द्रि॒यस्याप॑ क्रामति। यद्वा इ॒दङ्किं च॑। तत्सर्व॑मुत्ता॒न ए॒वाङ्गी॑र॒सः प्रत्य॑गृह्णात्। तदे॑नं॒ प्रति॑गृहीत॒न्नाहि॑नत्। यत्किं च॑ प्रतिगृह्णी॒यात्। तत्सर्व॑मुत्ता॒नस्त्वाङ्गीर॒सः प्रति॑गृह्णा॒त्वित्ये॒व प्रति॑गृह्णीयात्। इ॒यं वा उ॑त्ता॒न आङ्गीर॒सः। अ॒नयै॒वैन॒त्प्रति॑ गृह्णाति। नैन हिनस्ति॥१६॥\anuvakamend[तृती॑यमिन्द्रि॒यस्यापाक्रामच्चतु॒र्थमि॑न्द्रि॒यस्या॒त्मन्नु॒पाध॒त्ताश्वं॑ प्रतिगृ॒ह्णाति॑ ष॒ष्ठम॑स्येन्द्रि॒यस्याप॑क्रामत्यष्ट॒ममि॑न्द्रि॒यस्यापाक्रामत्प्रतिगृह्णी॒याच्च॒त्वारि॑ च (तस्य॒ वा अ॒ग्नेर्‌हिर॑ण्य॒ सोम॑स्य॒ वास॒स्तदे॒तेन॑ रु॒द्रस्य॒ गान्तामे॒तेन॒ वरु॑ण॒स्याश्वं॑ प्र॒जाप॑ते॒ पुरु॑षं॒ मनो॒स्तल्प॒न्तमे॒तेनोत्ता॒नस्य॒ तदे॒तेनाप्रा॑ण॒द्यद्वै। अ॒र्धं तृती॑यमष्ट॒मं तच्च॑तु॒र्थं तां प॑ञ्च॒म ष॒ष्ठ स॑प्त॒मन्तम्। तदे॒तेन॒ द्वे तामे॒तेनैकं॒ तमे॒तेन॒ त्रीणि॒ तदे॒तेनैकम्॥)]

%2.3.5.1
ब्र॒ह्म॒वा॒दिनो॑ वदन्ति। यद्दश॑होतारः स॒त्रमास॑त। केन॒ ते गृ॒हप॑तिनाऽऽर्ध्नुवन्। केन॑ प्र॒जा अ॑सृज॒न्तेति॑। प्र॒जाप॑तिना॒ वै ते गृ॒हप॑तिनाऽऽर्ध्नुवन्। तेन॑ प्र॒जा अ॑सृजन्त। यच्चतु॑र्‌होतारः स॒त्रमास॑त। केन॒ ते गृ॒हप॑तिनाऽऽर्ध्नुवन्। केनौष॑धीरसृज॒न्तेति॑। सोमे॑न॒ वै ते गृ॒हप॑तिनाऽऽर्ध्नुवन्॥१७॥

%2.3.5.2
तेनौष॑धीरसृजन्त। यत्पञ्च॑होतारः स॒त्रमास॑त। केन॒ ते गृ॒हप॑तिनाऽऽर्ध्नुवन्। केनै॒भ्यो लो॒केभ्योऽसु॑रा॒न्प्राणु॑दन्त। केनै॑षां प॒शून॑वृञ्ज॒तेति॑। अ॒ग्निना॒ वै ते गृ॒हप॑तिनाऽऽर्ध्नुवन्। तेनै॒भ्यो लो॒केभ्योऽसु॑रा॒न्प्राणु॑दन्त। तेनै॑षां प॒शून॑वृञ्जत। यथ्षड्ढो॑तारः स॒त्रमास॑त। केन॒ ते गृ॒हप॑तिनाऽऽर्ध्नुवन्॥१८॥

%2.3.5.3
केन॒र्तून॑कल्पय॒न्तेति॑। धा॒त्रा वै ते गृ॒हप॑तिनाऽऽर्ध्नुवन्। तेन॒र्तून॑कल्पयन्त। यत्स॒प्तहो॑तारः स॒त्रमास॑त। केन॒ ते गृ॒हप॑तिनाऽऽर्ध्नुवन्। केन॒ सुव॑रायन्। केने॒माल्लोँ॒कान्त्सम॑तन्व॒न्निति॑। अ॒र्य॒म्णा वै ते गृ॒हप॑तिनाऽऽर्ध्नुवन्। तेन॒ सुव॑रायन्। तेने॒माल्लोँ॒कान्त्सम॑तन्व॒न्निति॑॥१९॥

%2.3.5.4
ए॒ते वै दे॒वा गृ॒हप॑तयः। तान् य ए॒वं वि॒द्वान्। अप्य॒न्यस्य॑ गार्‌हप॒ते दीक्ष॑ते। अ॒वा॒न्त॒रमे॒व स॒त्रिणा॑मृध्नोति। यो वा अ॑र्य॒मणं॒ वेद॑। दान॑कामा अस्मै प्र॒जा भ॑वन्ति। य॒ज्ञो वा अ॑र्य॒मा। आर्या॑वस॒तिरिति॒ वै तमा॑हु॒र्यं प्र॒शस॑न्ति। आर्या॑वस॒तिर्भ॑वति। य ए॒वं वेद॑॥२०॥

%2.3.5.5
यद्वा इ॒दङ्किं च॑। तत्सर्वं॒ चतु॑र्\mbox{}होतारः। चतु॑र्‌होतृ॒भ्योऽधि॑ य॒ज्ञो निर्मि॑तः। स य ए॒वं वि॒द्वान्‌ वि॒वदे॑त। अ॒हमे॒व भूयो॑ वेद। यश्चतु॑र्‌होतॄ॒न् वेदेति॑। स ह्ये॑व भूयो॒ वेद॑। यश्चतु॑र्‌होतॄ॒न् वेद॑। यो वै चतु॑र्‌होतृणा॒ होतॄ॒न् वेद॑। सर्वा॑सु प्र॒जास्वन्न॑मत्ति॥२१॥

%2.3.5.6
सर्वा॒ दिशो॒ऽभि ज॑यति। प्र॒जाप॑ति॒र्वै दश॑होतृणा॒ होता। सोम॒श्चतु॑र्‌होतृणा॒ होता। अ॒ग्निः पञ्च॑होतृणा॒ होता। धा॒ता षड्ढो॑तृणा॒ होता। अ॒र्य॒मा स॒प्तहो॑तृणा॒ होता। ए॒ते वै चतु॑र्\mbox{}होतृणा॒ होता॑रः। तान् य ए॒वं वेद॑। सर्वा॑सु प्र॒जास्वन्न॑मत्ति। सर्वा॒ दिशो॒ऽभि ज॑यति॥२२॥\anuvakamend[आ॒र्ध्नु॒व॒न्ना॒र्ध्नु॒व॒न्नित्ये॒वं वेदात्ति सर्वा॒ दिशो॒ऽभि ज॑यति (वै तेन॑ स॒त्रङ्केन॑ ॥ )]

%2.3.6.1
प्र॒जाप॑तिः प्र॒जाः सृ॒ष्ट्वा व्य॑स्रसत। स हृद॑यं भू॒तो॑ऽशयत्। आत्म॒न्॒ हा ३ इत्यह्व॑यत्। आप॒ प्रत्य॑शृण्वन्। ता अ॑ग्निहो॒त्रेणै॒व य॑ज्ञक्र॒तुनोप॑ प॒र्याव॑र्तन्त। ताः कुसि॑न्ध॒मुपौ॑हन्। तस्मा॑दग्निहो॒त्रस्य॑ यज्ञक्र॒तोः। एक॑ ऋ॒त्विक्। च॒तु॒ष्कृत्वोऽह्व॑यत्। अ॒ग्निर्वा॒युरा॑दि॒त्यश्च॒न्द्रमा॥२३॥

%2.3.6.2
ते प्रत्य॑शृण्वन्। ते द॑र्‌शपूर्णमा॒साभ्या॑मे॒व य॑ज्ञक्र॒तुनोप॑ प॒र्याव॑र्तन्त। त उपौ॑हश्च॒त्वार्यङ्गा॑नि। तस्माद्दर्‌शपूर्णमा॒सयोर्यज्ञक्र॒तोः। च॒त्वार॑ ऋ॒त्विज॑। प॒ञ्च॒कृत्वोऽह्व॑यत्। प॒शव॒ प्रत्य॑शृण्वन्। ते चा॑तुर्मा॒स्यैरे॒व य॑ज्ञक्र॒तुनोप॑ प॒र्याव॑र्तन्त। त उपौ॑हं॒ लोम॑ छ॒वीं मा॒समस्थि॑ म॒ज्जानम्। तस्माच्चातुर्मा॒स्याना॑ यज्ञक्र॒तोः॥२४॥

%2.3.6.3
पञ्च॒र्त्विज॑। ष॒ट्कृत्वोऽह्व॑यत्। ऋ॒तव॒ प्रत्य॑शृण्वन्। ते प॑शुब॒न्धेनै॒व य॑ज्ञक्र॒तुनोप॑प॒र्याव॑र्तन्त। त उपौ॑ह॒न्त्स्तना॑वा॒ण्डौ शि॒श्ञमवाञ्चं प्रा॒णम्। तस्मात्पशुब॒न्धस्य॑ यज्ञक्र॒तोः। षडृ॒त्विज॑। स॒प्त॒कृत्वोऽह्व॑यत्। होत्रा॒ प्रत्य॑शृण्वन्। ताः सौ॒म्येनै॒वाध्व॒रेण॑ यज्ञक्र॒तुनोप॑प॒र्याव॑र्तन्त॥२५॥

%2.3.6.4
ता उपौ॑हन्त्स॒प्त शी॑र्‌ष॒ण्यान्प्रा॒णान्। तस्मात्सौ॒म्यस्याध्व॒रस्य॑ यज्ञक्र॒तोः। स॒प्त होत्रा॒ प्राची॒र्वष॑ट्कुर्वन्ति। द॒श॒कृत्वोऽह्व॑यत्। तप॒ प्रत्य॑शृणोत्। तत्कर्म॑णै॒व सं॑वत्स॒रेण॒ सर्वैर्यज्ञक्र॒तुभि॒रुप॑ प॒र्याव॑र्तत। तत्सर्व॑मा॒त्मान॒मप॑रिवर्ग॒मुपौ॑हत्। तस्मात्संवत्स॒रे सर्वे॑ यज्ञक्र॒तवोऽव॑रुध्यन्ते। तस्मा॒द्दश॑होता॒ चतु॑र्‌होता। पञ्च॑होता॒ षड्ढो॑ता स॒प्तहो॑ता। एक॑होत्रे ब॒लि ह॑रन्ति। हर॑न्त्यस्मै प्र॒जा ब॒लिम्। ऐन॒मप्र॑तिख्यातं गच्छति। य ए॒वं वेद॑॥२६॥\anuvakamend[च॒न्द्रमाश्चातुर्मा॒स्यानां यज्ञक्र॒तोर॑ध्व॒रेण॑ यज्ञक्र॒तुनोप॑ प॒र्याव॑र्तन्त स॒प्तहो॑ता च॒त्वारि॑ च]

%2.3.7.1
प्र॒जाप॑ति॒ पुरु॑षमसृजत। सोऽग्निर॑ब्रवीत्। ममा॒यमन्न॑म॒स्त्विति॑। सो॑ऽबिभेत्। सर्वं॒ वै मा॒ऽयं प्र ध॑क्ष्य॒तीति॑। स ए॒ताश्चतु॑र्\mbox{}होतॄनात्म॒स्पर॑णानपश्यत्। तान॑जुहोत्। तैर्वै स आ॒त्मान॑मस्पृणोत्। यद॑ग्निहो॒त्रं जु॒होति॑। एक॑होतारमे॒व तद्य॑ज्ञक्र॒तुमाप्नोत्यग्निहो॒त्रम्॥२७॥

%2.3.7.2
कुसि॑न्धञ्चा॒त्मन॑ स्पृ॒णोति॑। आ॒दि॒त्यस्य॑ च॒ सा॑युज्यं गच्छति। च॒तुरुन्न॑यति। चतु॑र्‌होतारमे॒व तद्य॑ज्ञक्र॒तुमाप्नोति दर्‌शपूर्णमा॒सौ। च॒त्वारि॑ चा॒त्मनोऽङ्गा॑नि स्पृ॒णोति॑। आ॒दि॒त्यस्य॑ च॒ सायु॑ज्यं गच्छति। च॒तुरुन्न॑यति। स॒मित्प॑ञ्च॒मी। पञ्च॑होतारमे॒व तद्य॑ज्ञक्र॒तुमाप्नोति चातुर्मा॒स्यानि॑। लोम॑ छ॒वीं मा॒समस्थि॑ म॒ज्जानम्॥२८॥

%2.3.7.3
तानि॑ चा॒त्मन॑ स्पृ॒णोति॑। आ॒दि॒त्यस्य॑ च॒ सायु॑ज्यं गच्छति। च॒तुरुन्न॑यति। द्विर्जु॑होति। षड्ढो॑तारमे॒व तद्य॑ज्ञक्र॒तुमाप्नोति पशुब॒न्धम्। स्तना॑वा॒ण्डौ शि॒श्ञमवाञ्चं प्रा॒णम्। तानि॑ चा॒त्मन॑ स्पृ॒णोति॑। आ॒दि॒त्यस्य॑ च॒ सायु॑ज्यं गच्छति। च॒तुरुन्न॑यति। द्विर्जु॑होति॥२९॥

%2.3.7.4
स॒मित्स॑प्त॒मी। स॒प्तहो॑तारमे॒व तद्य॑ज्ञक्र॒तुमाप्नोति सौ॒म्यम॑ध्व॒रम्। स॒प्त चा॒त्मन॑ शीर्\mbox{}ष॒ण्यान्प्रा॒णान्त्स्पृ॒णोति॑। आ॒दि॒त्यस्य॑ च॒ सायु॑ज्यं गच्छति। च॒तुरुन्न॑यति। द्विर्जु॒होति॑। द्विर्निमार्ष्टि। द्विः प्राश्ञा॑ति। दश॑होतारमे॒व तद्य॑ज्ञक्र॒तुमाप्नोति संवत्स॒रम्। सर्वं॑ चा॒त्मान॒मप॑रिवर्ग स्पृ॒णोति॑। आ॒दि॒त्यस्य॑ च॒ सायु॑ज्यं गच्छति॥३०॥\anuvakamend[अ॒ग्नि॒हो॒त्रं म॒ज्जान॒न्द्विर्जु॑हो॒त्यप॑रिवर्ग स्पृ॒णोत्येकं च]

%2.3.8.1
प्र॒जाप॑तिरकामयत॒ प्र जा॑ये॒येति॑। स तपो॑ऽतप्यत। सोऽन्तर्वा॑नभवत्। स हरि॑तः श्या॒वो॑ऽभवत्। तस्मा॒त्स्त्र्य॑न्तर्व॑त्नी। हरि॑णी स॒ती श्या॒वा भ॑वति। स वि॒जाय॑मानो॒ गर्भे॑णाताम्यत्। स ता॒न्तः कृ॒ष्णः श्या॒वो॑ऽभवत्। तस्मात्ता॒न्तः कृ॒ष्णः श्या॒वो भ॑वति। तस्यासु॑रे॒वाजी॑वत्॥३१॥

%2.3.8.2
तेनासु॒नाऽसु॑रानसृजत। तदसु॑राणामसुर॒त्वम्। य ए॒वमसु॑राणामसुर॒त्वं वेद॑। असु॑माने॒व भ॑वति। नैन॒मसु॑र्जहाति। सोऽसु॑रान्त्सृ॒ष्ट्वा पि॒तेवा॑मन्यत। तदनु॑ पि॒तॄन॑सृजत। तत्पि॑तृ॒णां पि॑तृ॒त्वम्। य ए॒वं पि॑तृ॒णां पि॑तृ॒त्वं वेद॑। पि॒तेवै॒व स्वानां भवति॥३२॥

%2.3.8.3
यन्त्य॑स्य पि॒तरो॒ हवम्। स पि॒तॄन्त्सृ॒ष्ट्वाऽऽम॑नस्यत्। तदनु॑ मनु॒ष्या॑नसृजत। तन्म॑नु॒ष्या॑णां मनुष्य॒त्वम्। य ए॒वं म॑नु॒ष्या॑णां मनुष्य॒त्वं वेद॑। म॒न॒स्व्ये॑व भ॑वति। नैनं॒ मनु॑र्जहाति। तस्मै॑ मनु॒ष्यान्त्ससृजा॒नाय॑। दिवा॑ देव॒त्राऽभ॑वत्। तदनु॑ दे॒वान॑सृजत। तद्दे॒वानान्देव॒त्वम्। य ए॒वन्दे॒वानान्देव॒त्वं वेद॑। दिवा॑ है॒वास्य॑ देव॒त्रा भ॑वति। तानि॒ वा ए॒तानि॑ च॒त्वार्यम्भासि। दे॒वा म॑नु॒ष्या पि॒तरोऽसु॑राः। तेषु॒ सर्वे॒ष्वम्भो॒ नभ॑ इव भवति। य ए॒वं वेद॑॥३३॥\anuvakamend[अ॒जी॒व॒त्स्वानां भवति दे॒वान॑सृजत स॒प्त च॑]

%2.3.9.1
ब्र॒ह्म॒वा॒दिनो॑ वदन्ति। यो वा इ॒मं वि॒द्यात्। यतो॒ऽयं पव॑ते। यद॑भि॒ पव॑ते। यद॑भि सं॒पव॑ते। सर्व॒मायु॑रियात्। न पु॒राऽऽयु॑ष॒ प्र मी॑येत। प॒शु॒मान्त्स्यात्। वि॒न्देत॑ प्र॒जाम्। यो वा इ॒मं वेद॑॥३४॥

%2.3.9.2
यतो॒ऽयं पव॑ते। यद॑भि॒ पव॑ते। यद॑भि सं॒पव॑ते। सर्व॒मायु॑रेति। न पु॒राऽऽयु॑ष॒ प्र मी॑यते। प॒शु॒मान्भ॑वति। वि॒न्दते प्र॒जाम्। अ॒द्भ्यः प॑वते। अ॒पो॑ऽभि प॑वते। अ॒पो॑ऽभि संप॑वते॥३५॥

%2.3.9.3
अ॒स्याः प॑वते। इ॒माम॒भि प॑वते। इ॒माम॒भि संप॑वते। अ॒ग्नेः प॑वते। अ॒ग्निम॒भि प॑वते। अ॒ग्निम॒भि सं प॑वते। अ॒न्तरि॑क्षात्पवते। अ॒न्तरि॑क्षम॒भि प॑वते। अ॒न्तरि॑क्षम॒भि सं प॑वते। आ॒दि॒त्यात्प॑वते॥३६॥

%2.3.9.4
आ॒दि॒त्यम॒भि प॑वते। आ॒दि॒त्यम॒भि सं प॑वते। द्योः प॑वते। दिव॑म॒भि प॑वते। दिव॑म॒भि सं प॑वते। दि॒ग्भ्यः प॑वते। दिशो॒ऽभि प॑वते। दिशो॒ऽभि संप॑वते। स यत्पु॒रस्ता॒द्वाति॑। प्रा॒ण ए॒व भू॒त्वा पु॒रस्ताद्वाति॥३७॥

%2.3.9.5
तस्मात्पु॒रस्ता॒द्वान्तम्। सर्वा प्र॒जाः प्रति॑ नन्दन्ति। प्रा॒णो हि प्रि॒यः प्र॒जानाम्। प्रा॒ण इ॑व प्रि॒यः प्र॒जानां भवति। य ए॒वं वेद॑। स वा ए॒ष प्रा॒ण ए॒व। अथ॒ यद्द॑क्षिण॒तो वाति॑। मा॒त॒रिश्वै॒व भू॒त्वा द॑क्षिण॒तो वा॑ति। तस्माद्दक्षिण॒तो वान्तं॑ वि॒द्यात्। सर्वा॒ दिश॒ आ वा॑ति॥३८॥

%2.3.9.6
सर्वा॒ दिशोऽनु॒ वि वा॑ति। सर्वा॒ दिशोऽनु॒ सं वा॒तीति॑। स वा ए॒ष मा॑त॒रिश्वै॒व। अथ॒ यत्प॒श्चाद्वाति॑। पव॑मान ए॒व भू॒त्वा प॒श्चाद्वा॑ति। पू॒तम॑स्मा॒ आह॑रन्ति। पू॒तमुप॑हरन्ति। पू॒तम॑श्ञाति। य ए॒वं वेद॑। स वा ए॒ष पव॑मान ए॒व॥३९॥

%2.3.9.7
अथ॒ यदु॑त्तर॒तो वाति॑। स॒वि॒तैव भू॒त्त्वोत्त॑र॒तो वा॑ति। स॒वि॒तेव॒ स्वानां भवति। य ए॒वं वेद॑। स वा ए॒ष स॑वि॒तैव। ते य ए॑नं पु॒रस्ता॑दा॒यन्त॑मुप॒वद॑न्ति। य ए॒वास्य॑ पु॒रस्तात्पा॒प्मान॑। तास्तेऽप॑ घ्नन्ति। पु॒रस्ता॒दित॑रान्पा॒प्मन॑ सचन्ते। अथ॒ य ए॑नन्दक्षिण॒त आ॒यन्त॑मुप॒वद॑न्ति॥४०॥

%2.3.9.8
य ए॒वास्य॑ दक्षिण॒तः पा॒प्मान॑। तास्तेऽप॑ घ्नन्ति। द॒क्षि॒ण॒त इत॑रान्पा॒प्मन॑ सचन्ते। अथ॒ य ए॑नं प॒श्चादा॒यन्त॑मुप॒ वद॑न्ति। य ए॒वास्य॑ प॒श्चात्पा॒प्मान॑। तास्तेऽप॑ घ्नन्ति। प॒श्चादित॑रान्पा॒प्मन॑ सचन्ते। अथ॒ य ए॑नमुत्तर॒त आ॒यन्त॑मुप॒ वद॑न्ति। य ए॒वास्योत्तर॒तः पा॒प्मान॑। तास्तेऽप॑ घ्नन्ति॥४१॥

%2.3.9.9
उ॒त्त॒र॒त इत॑रान्पा॒प्मन॑ सचन्ते। तस्मा॑दे॒वं वि॒द्वान्। वीव॑ नृत्येत्। प्रेव॑ चलेत्। व्यस्ये॑वा॒क्ष्यौ भा॑षेत। म॒ण्टये॑दिव। क्रा॒थये॑दिव। शृ॒ङ्गा॒येते॑व। उ॒त मोप॑ वदेयुः। उ॒त मे॑ पा॒प्मान॒मप॑ हन्यु॒रिति॑। स यान्दिश स॒निमे॒ष्यन्त्स्यात्। य॒दा तान्दिशं॒ वातो॑ वा॒यात्। अथ॒ प्रवे॒यात्। प्र वा॑ धावयेत्। सा॒तमे॒व र॑दि॒तं व्यू॑ढं ग॒न्धम॒भि प्रच्य॑वते। आऽस्य॒ तं ज॑नप॒दं पूर्वा॑ की॒र्तिर्ग॑च्छति। दान॑कामा अस्मै प्र॒जा भ॑वन्ति। य ए॒वं वेद॑॥४२॥\anuvakamend[वेद॒ सं प॑वत आदि॒त्यात्प॑वते वा॒त्या वात्ये॒ष पव॑मान ए॒व द॑क्षिण॒त आ॒यन्त॑मुप॒ वद॑न्त्युत्तर॒तः पा॒प्मान॒स्ता स्तेप॑ घ्न॒न्तीत्य॒ष्टौ च॑]

%2.3.10.1
प्र॒जाप॑ति॒ सोम॒ राजा॑नमसृजत। तन्त्रयो॒ वेदा॒ अन्व॑सृज्यन्त। तान् हस्ते॑ऽकुरुत। अथ॒ ह सीता॑ सावि॒त्री। सोम॒ राजा॑नञ्चकमे। श्र॒द्धामु॒ स च॑कमे। साऽऽह॑ पि॒तरं॑ प्र॒जाप॑ति॒मुप॑ससार। त हो॑वाच। नम॑स्ते अस्तु भगवः। उप॑ त्वाऽयानि॥४३॥

%2.3.10.2
प्र त्वा॑ पद्ये। सोमं॒ वै राजा॑नङ्कामये। श्र॒द्धामु॒ स का॑मयत॒ इति॑। तस्या॑ उ॒ ह स्था॑ग॒रम॑लङ्का॒रङ्क॑ल्पयि॒त्वा। दश॑होतारं पु॒रस्ताद्व्या॒ख्याय॑। चतु॑र्\mbox{}होतारन्दक्षिण॒तः। पञ्च॑होतारं प॒श्चात्। षड्ढो॑तारमुत्तर॒तः। स॒प्तहो॑तारमु॒परि॑ष्टात्। स॒म्भा॒रैश्च॒ पत्नि॑भिश्च॒ मुखे॑ऽल॒ङ्कृत्य॑॥४४॥

%2.3.10.3
आऽस्यार्धं व॑व्राज। ता हो॒दीक्ष्यो॑वाच। उप॒ मा व॑र्त॒स्वेति॑। त हो॑वाच। भोग॒न्तु म॒ आच॑क्ष्व। ए॒तन्म॒ आच॑क्ष्व। यत्ते॑ पा॒णाविति॑। तस्या॑ उ॒ ह त्रीन् वेदा॒न्प्रद॑दौ। तस्मा॒दुह॒ स्त्रियो॒ भोग॒मैव हा॑रयन्ते। स यः का॒मये॑त प्रि॒यः स्या॒मिति॑॥४५॥

%2.3.10.4
यं वा का॒मये॑त प्रि॒यः स्या॒दिति॑। तस्मा॑ ए॒त स्था॑ग॒रम॑लङ्का॒रङ्क॑ल्पयि॒त्वा। दश॑होतारं पु॒रस्ताद्व्या॒ख्याय॑। चतु॑र्\mbox{}होतारन्दक्षिण॒तः। पञ्च॑होतारं प॒श्चात्। षड्ढो॑तारमुत्तर॒तः। स॒प्तहो॑तारमु॒परि॑ष्टात्। स॒म्भा॒रैश्च॒ पत्नि॑भिश्च॒ मुखे॑ऽल॒ङ्कृत्य॑। आस्यार्धं व्र॑जेत्। प्रि॒यो है॒व भ॑वति॥४६॥\anuvakamend[अ॒या॒न्य॒ल॒ङ्कृत्य॑ स्या॒मिति॑ भवति]

%2.3.11.1
ब्रह्मात्म॒न्वद॑सृजत। तद॑कामयत। समा॒त्मना॑ पद्ये॒येति॑। आत्म॒न्नात्म॒न्नित्याम॑न्त्रयत। तस्मै॑ दश॒म हू॒तः प्रत्य॑शृणोत्। स दश॑हूतोऽभवत्। दश॑हूतो ह॒ वै नामै॒षः। तं वा ए॒तन्दश॑हूत॒ सन्तम्। दश॑हो॒तेत्याच॑क्षते प॒रोक्षे॑ण। प॒रोक्ष॑प्रिया इव॒ हि दे॒वाः॥४७॥

%2.3.11.2
आत्म॒न्नात्म॒न्नित्याम॑न्त्रयत। तस्मै॑ सप्त॒म हू॒तः प्रत्य॑शृणोत्। स स॒प्तहू॑तोऽभवत्। स॒प्तहू॑तो ह॒ वै नामै॒षः। तं वा ए॒त स॒प्तहू॑त॒ सन्तम्। स॒प्तहो॒तेत्याच॑क्षते प॒रोक्षे॑ण। प॒रोक्ष॑प्रिया इव॒ हि दे॒वाः। आत्म॒न्नात्म॒न्नित्याम॑न्त्रयत। तस्मै॑ ष॒ष्ठ हू॒तः प्रत्य॑शृणोत्। स षड्ढू॑तोऽभवत्॥४८॥

%2.3.11.3
षड्ढू॑तो ह॒ वै नामै॒षः। तं वा ए॒त षड्ढू॑त॒ सन्तम्। षड्ढो॒तेत्याच॑क्षते प॒रोक्षे॑ण। प॒रोक्ष॑प्रिया इव॒ हि दे॒वाः। आत्म॒न्नात्म॒न्नित्याम॑न्त्रयत। तस्मै॑ पञ्च॒म हू॒तः प्रत्य॑शृणोत्। स पञ्च॑हूतोऽभवत्। पञ्च॑हूतो ह॒ वै नामै॒षः। तं वा ए॒तं पञ्च॑हूत॒ सन्तम्। पञ्च॑हो॒तेत्याच॑क्षते प॒रोक्षे॑ण॥४९॥

%2.3.11.4
प॒रोक्ष॑प्रिया इव॒ हि दे॒वाः। आत्म॒न्नात्म॒न्नित्याम॑न्त्रयत। तस्मै॑ चतु॒र्थ हू॒तः प्रत्य॑शृणोत्। स चतु॑र्\mbox{}हूतोऽभवत्। चतु॑र्‌हूतो ह॒ वै नामै॒षः। तं वा ए॒तञ्चतु॑र्‌हूत॒ सन्तम्। चतु॑र्हो॒तेत्याच॑क्षते प॒रोक्षे॑ण। प॒रोक्ष॑प्रिया इव॒ हि दे॒वाः। तम॑ब्रवीत्। त्वं वै मे॒ नेदि॑ष्ठ हू॒तः प्रत्य॑श्रौषीः। त्वयै॑नानाख्या॒तार॒ इति॑। तस्मा॒न्नु है॑ना॒श्चतु॑र्‌होतार॒ इत्याच॑क्षते। तस्माच्छुश्रू॒षुः पु॒त्राणा॒ हृद्य॑तमः। नेदि॑ष्ठो॒ हृद्य॑तमः। नेदि॑ष्ठो॒ ब्रह्म॑णो भवति। य ए॒वं वेद॑॥५०॥\anuvakamend[दे॒वाष्षड्ढू॑तोऽभव॒त्पञ्च॑हो॒तेत्याच॑क्षते प॒रोक्षे॑णाश्रौषी॒ष्षट्च॑]




\prashnaend{ब्र॒ह्म॒वा॒दिन॒ किं दक्षि॑णां॒ यो वा अवि॑द्वा॒न्तस्य॒ वै ब्र॑ह्मवा॒दिनो॒ यद्दश॑होतारः प्र॒जाप॑ति॒र्व्य॑स्रं प्र॒जाप॑ति॒ पुरु॑षं प्र॒जाप॑तिरकामयत॒ स तप॒ सोऽन्तर्वान्ब्रह्मवा॒दिनो॒ यो वा इ॒मं वि॒द्यात्प्र॒जाप॑ति॒ सोम॒ राजा॑नं॒ ब्रह्मात्म॒न्वदेका॑दश॥११॥}{ब्र॒ह्म॒वा॒दिन॒स्तस्य॒ वा अ॒ग्नेर्यद्वा इ॒दङ्किं च॑ प्र॒जाप॑तिरकामयत॒ य ए॒वास्य॑ दक्षिण॒तः प॑ञ्चा॒शत्॥५०॥}{ब्र॒ह्म॒वा॒दिनो॒ य ए॒वं वेद॑॥}{हरि॑ ओम्॥}{इति श्रीकृष्णयजुर्वेदीयतैत्तिरीयब्राह्मणे द्वितीयाष्टके तृतीयः प्रपाठकः समाप्तः॥}
\clearpage
