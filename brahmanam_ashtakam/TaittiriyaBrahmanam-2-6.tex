\sect{षष्ठमः प्रश्नः}
\setcounter{anuvakam}{0}
\dnsub{तैत्तिरीयब्राह्मणे द्वितीयाष्टके षष्ठः प्रपाठकः}

%2.6.1.1
स्वा॒द्वीं त्वा स्वा॒दुना। ती॒व्रां ती॒व्रेण॑। अ॒मृता॑म॒मृते॑न। मधु॑मतीं॒ मधु॑मता। सृ॒जामि॒ स सोमे॑न। सोमोऽस्य॒श्विभ्यां पच्यस्व। सर॑स्वत्यै पच्यस्व। इन्द्रा॑य सु॒त्राम्णे॑ पच्यस्व। परी॒तो षि॑ञ्चता सु॒तम्। सोमो॒ य उ॑त्त॒म ह॒विः॥१॥

%2.6.1.2
द॒ध॒न्वा यो नर्यो॑ अ॒प्स्व॑न्तरा। सु॒षाव॒ सोम॒मद्रि॑भिः। पु॒नातु॑ ते परि॒स्रुतम्। सोम॒ सूर्य॑स्य दुहि॒ता। वारे॑ण॒ शश्व॑ता॒ तना। वा॒युः पू॒तः प॒वित्रे॑ण। प्राङ्ख्सोमो॒ अति॑द्रुतः। इन्द्र॑स्य॒ युज्य॒ सखा। वा॒युः पू॒तः प॒वित्रे॑ण। प्र॒त्यङ्ख्सोमो॒ अति॑द्रुतः॥२॥

%2.6.1.3
इन्द्र॑स्य॒ युज्य॒ सखा। ब्रह्म॑ क्ष॒त्रं प॑वते॒ तेज॑ इन्द्रि॒यम्। सुर॑या॒ सोम॑ सु॒त आसु॑तो॒ मदा॑य। शु॒क्रेण॑ देव दे॒वता पिपृग्धि। रसे॒नान्नं॒ यज॑मानाय धेहि। कु॒विद॒ङ्ग यव॑मन्तो॒ यवं॑चित्। यथा॒ दान्त्य॑नुपू॒र्वं वि॒यूय॑। इ॒हेहै॑षां कृणुत॒ भोज॑नानि। ये ब॒र्‌हिषो॒ नमो॑वृक्तिं॒ न ज॒ग्मुः। उ॒प॒या॒मगृ॑हीतोऽस्य॒श्विभ्यां त्वा॒ जुष्टं॑ गृह्णामि॥३॥

%2.6.1.4
सर॑स्वत्या॒ इन्द्रा॑य सु॒त्राम्णे। ए॒ष ते॒ योनि॒स्तेज॑से त्वा। वी॒र्या॑य त्वा॒ बला॑य त्वा। तेजो॑ऽसि॒ तेजो॒ मयि॑ धेहि। वी॒र्य॑मसि वी॒र्यं॑ मयि॑ धेहि। बल॑मसि॒ बलं॒ मयि॑ धेहि। नाना॒ हि वां दे॒वहि॑त॒ सद॑ कृ॒तम्। मा ससृ॑क्षाथां पर॒मे व्यो॑मन्। सुरा॒ त्वमसि॑ शु॒ष्मिणी॒ सोम॑ ए॒षः। मा मा॑ हिसी॒ स्वां योनि॑मावि॒शन्॥४॥

%2.6.1.5
उ॒प॒या॒मगृ॑हीतोऽस्याश्वि॒नं तेज॑। सा॒र॒स्व॒तं वी॒र्यम्। ऐ॒न्द्रं बलम्। ए॒ष ते॒ योनि॒र्मोदा॑य त्वा। आ॒न॒न्दाय॑ त्वा॒ मह॑से त्वा। ओजो॒ऽस्योजो॒ मयि॑ धेहि। म॒न्युर॑सि म॒न्युं मयि॑ धेहि। महो॑ऽसि॒ महो॒ मयि॑ धेहि। सहो॑ऽसि॒ सहो॒ मयि॑ धेहि। या व्या॒घ्रं विषू॑चिका। उ॒भौ वृकं॑ च॒ रक्ष॑ति। श्ये॒नं प॑त॒त्रिण सि॒हम्। सेमं पा॒त्वह॑सः। सं॒पृच॑ स्थ॒ सं मा॑ भ॒द्रेण॑ पृङ्क्त। वि॒पृच॑ स्थ॒ वि मा॑ पा॒प्मना॑ पृङ्क्त॥५॥\anuvakamend[ह॒विः प्र॒त्यङ्ख्सोमो॒ अति॑द्रुतो गृह्णाम्यावि॒शन्विषू॑चिका॒ पञ्च॑ च]

%2.6.2.1
सोमो॒ राजा॒ऽमृत सु॒तः। ऋ॒जी॒षेणा॑जहान्मृ॒त्युम्। ऋ॒तेन॑ स॒त्यमि॑न्द्रि॒यम्। विपान शु॒क्रमन्ध॑सः। इन्द्र॑स्येन्द्रि॒यम्। इ॒दं पयो॒ऽमृतं॒ मधु॑। सोम॑म॒द्भ्यो व्य॑पिबत्। छन्द॑सा ह॒सः शु॑चि॒षत्। ऋ॒तेन॑ स॒त्यमि॑न्द्रि॒यम्। अ॒द्भ्यः क्षी॒रव्व्यँ॑पिबत्॥६॥

%2.6.2.2
क्रुङ्ङाङ्गिर॒सो धि॒या। ऋ॒तेन॑ स॒त्यमि॑न्द्रि॒यम्। अन्नात्परि॒स्रुतो॒ रसम्। ब्रह्म॑णा॒ व्य॑पिबत् क्ष॒त्रम्। ऋ॒तेन॑ स॒त्यमि॑न्द्रि॒यम्। रेतो॒ मूत्रं॒ विज॑हाति। योनिं॑ प्रवि॒शदि॑न्द्रि॒यम्। गर्भो॑ ज॒रायु॒णाऽऽवृ॑तः। उल्बं॑ जहाति॒ जन्म॑ना। ऋ॒तेन॑ स॒त्यमि॑न्द्रि॒यम्॥७॥

%2.6.2.3
वेदे॑न रू॒पे व्य॑करोत्। स॒ता॒स॒ती प्र॒जाप॑तिः। ऋ॒तेन॑ स॒त्यमि॑न्द्रि॒यम्। सोमे॑न॒ सोमौ॒ व्य॑पिबत्। सु॒ता॒सु॒तौ प्र॒जाप॑तिः। ऋ॒तेन॑ स॒त्यमि॑न्द्रि॒यम्। दृ॒ष्ट्वा रू॒पे व्याक॑रोत्। स॒त्या॒नृ॒ते प्र॒जाप॑तिः। अश्र॑द्धा॒मनृ॒तेऽद॑धात्। श्र॒द्धा स॒त्ये प्र॒जाप॑तिः। ऋ॒तेन॑ स॒त्यमि॑न्द्रि॒यम्। दृ॒ष्ट्वा प॑रि॒स्रुतो॒ रसम्। शु॒क्रेण॑ शु॒क्रव्व्यँ॑पिबत्। पय॒ सोमं॑ प्र॒जाप॑तिः। ऋ॒तेन॑ स॒त्यमि॑न्द्रि॒यम्। विपान शु॒क्रमन्ध॑सः। इन्द्र॑स्येन्द्रि॒यम्। इ॒दं पयो॒ऽमृतं॒ मधु॑॥८॥\anuvakamend[अ॒द्भ्यः क्षी॒रव्व्यँ॑पिब॒ज्जन्म॑न॒र्तेन॑ स॒त्यमि॑न्द्रि॒य श्र॒द्धा स॒त्ये प्र॒जाप॑तिर॒ष्टौ च॑]

%2.6.3.1
सुरा॑वन्तं बर्‌हि॒षद सु॒वीरम्। य॒ज्ञ हि॑न्वन्ति महि॒षा नमो॑भिः। दधा॑ना॒ सोम॑न्दि॒वि दे॒वता॑सु। मदे॒मेन्द्रं॒ यज॑मानाः स्व॒र्काः। यस्ते॒ रस॒ सम्भृ॑त॒ ओष॑धीषु। सोम॑स्य॒ शुष्म॒ सुर॑या सु॒तस्य॑। तेन॑ जिन्व॒ यज॑मानं॒ मदे॑न। सर॑स्वतीम॒श्विना॒विन्द्र॑म॒ग्निम्। यम॒श्विना॒ नमु॑चेरासु॒रादधि॑। सर॑स्व॒त्यस॑नोदिन्द्रि॒याय॑॥९॥

%2.6.3.2
इ॒मन्त शु॒क्रं मधु॑मन्त॒मिन्दुम्। सोम॒ राजा॑नमि॒ह भ॑क्षयामि। यदत्र॑ रि॒प्त र॒सिन॑ सु॒तस्य॑। यदिन्द्रो॒ अपि॑ब॒च्छची॑भिः। अ॒हन्तद॑स्य॒ मन॑सा शि॒वेन॑। सोम॒ राजा॑नमि॒ह भ॑क्षयामि। पि॒तृभ्य॑ स्वधा॒विभ्य॑ स्व॒धा नम॑। पि॒ता॒म॒हेभ्य॑ स्वधा॒विभ्य॑ स्व॒धा नम॑। प्रपि॑तामहेभ्यः स्वधा॒विभ्य॑ स्व॒धा नम॑। अक्ष॑न्पि॒तर॑॥१०॥

%2.6.3.3
अमी॑मदन्त पि॒तर॑। अती॑तृपन्त पि॒तर॑। अमी॑मृजन्त पि॒तर॑। पित॑र॒ शुन्ध॑ध्वम्। पु॒नन्तु॑ मा पि॒तर॑ सो॒म्यास॑। पु॒नन्तु॑ मा पिताम॒हाः। पु॒नन्तु॒ प्रपि॑तामहाः। प॒वित्रे॑ण श॒तायु॑षा। पु॒नन्तु॑ मा पिताम॒हाः। पु॒नन्तु॒ प्रपि॑तामहाः॥११॥

%2.6.3.4
प॒वित्रे॑ण श॒तायु॑षा। विश्व॒मायु॒र्व्य॑श्ञवै। अग्न॒ आयूषि पव॒सेऽग्ने॒ पव॑स्व। पव॑मान॒ सुव॒र्जन॑ पु॒नन्तु॑ मा देवज॒नाः। जात॑वेदः प॒वित्र॑व॒द्यत्ते॑ प॒वित्र॑म॒र्चिषि॑। उ॒भाभ्यान्देव सवितर्वैश्वदे॒वी पु॑न॒ती। ये स॑मा॒नाः सम॑नसः। पि॒तरो॑ यम॒राज्ये। तेषां लो॒कः स्व॒धा नम॑। य॒ज्ञो दे॒वेषु॑ कल्पताम्॥१२॥

%2.6.3.5
ये स॑जा॒ताः सम॑नसः। जी॒वा जी॒वेषु॑ माम॒काः। तेषा॒ श्रीर्मयि॑ कल्पताम्। अ॒स्मिल्लोँ॒के श॒त समा। द्वे स्रु॒ती अ॑शृणवं पितृ॒णाम्। अ॒हन्दे॒वाना॑मु॒त मर्त्या॑नाम्। याभ्या॑मि॒दं विश्व॒मेज॒त्समे॑ति। यद॑न्त॒रा पि॒तरं॑ मा॒तरं॑ च। इ॒द ह॒विः प्र॒जन॑नं मे अस्तु। दश॑वीर स॒र्वग॑ण स्व॒स्तये। आ॒त्म॒सनि॑ प्रजा॒सनि॑। प॒शु॒सन्य॑भय॒सनि॑ लोक॒सनि॑। अ॒ग्निः प्र॒जां ब॑हु॒लां मे॑ करोतु। अन्नं॒ पयो॒ रेतो॑ अ॒स्मासु॑ धत्त। रा॒यस्पोष॒मिष॒मूर्ज॑म॒स्मासु॑ दीधर॒त्स्वाहा॥१३॥\anuvakamend[इ॒न्द्रि॒याय॑ पि॒तर॑ श॒तायु॑षा पु॒नन्तु॑ मा पिताम॒हाः पु॒नन्तु॒ प्रपि॑तामहाः कल्पता स्व॒स्तये॒ पञ्च॑ च]

%2.6.4.1
सीसे॑न॒ तन्त्रं॒ मन॑सा मनी॒षिण॑। ऊ॒र्णा॒सू॒त्रेण॑ क॒वयो॑ वयन्ति। अ॒श्विना॑ य॒ज्ञ स॑वि॒ता सर॑स्वती। इन्द्र॑स्य रू॒पं वरु॑णो भिष॒ज्यन्। तद॑स्य रू॒पम॒मृत॒ शची॑भिः। ति॒स्रोऽद॑धुर्दे॒वता सररा॒णाः। लोमा॑नि॒ शष्पैर्बहु॒धा न तोक्म॑भिः। त्वग॑स्य मा॒सम॑भव॒न्न ला॒जाः। तद॒श्विना॑ भि॒षजा॑ रु॒द्रव॑र्तनी। सर॑स्वती वयति॒ पेशो॒ अन्त॑रः॥१४॥

%2.6.4.2
अस्थि॑ म॒ज्जानं॒ मास॑रैः। का॒रो॒त॒रेण॒ दध॑तो॒ गवान्त्व॒चि। सर॑स्वती॒ मन॑सा पेश॒लं वसु॑। नास॑त्याभ्यां वयति दर्‌श॒तं वपु॑। रसं॑ परि॒स्रुता॒ न रोहि॑तम्। न॒ग्नहु॒र्धीर॒स्तस॑र॒न्न वेम॑। पय॑सा शु॒क्रम॒मृतं॑ ज॒नित्रम्। सुर॑या॒ मूत्राज्जनयन्ति॒ रेत॑। अपाम॑तिन्दुर्म॒तिं बाध॑मानाः। ऊव॑ध्यं॒ वात स॒बुव॒न्तदा॒रात्॥१५॥

%2.6.4.3
इन्द्र॑ सु॒त्रामा॒ हृद॑येन स॒त्यम्। पु॒रो॒डाशे॑न सवि॒ता ज॑जान। यकृ॑त्क्लो॒मानं॒ वरु॑णो भिष॒ज्यन्। मत॑स्ने वाय॒व्यैर्न मि॑नाति पि॒त्तम्। आ॒न्त्राणि॑ स्था॒ली मधु॒ पिन्व॑माना। गुदा॒ पात्रा॑णि सु॒दुघा॒ न धे॒नुः। श्ये॒नस्य॒ पत्र॒न्न प्ली॒हा शची॑भिः। आ॒स॒न्दी नाभि॑रु॒दर॒न्न मा॒ता। कु॒म्भो व॑नि॒ष्ठुर्ज॑नि॒ता शची॑भिः। यस्मि॒न्नग्रे॒ योन्या॒ङ्गर्भो॑ अ॒न्तः ॥१६॥

%2.6.4.4
प्ला॒शीर्व्य॑क्तः श॒तधा॑र॒ उत्स॑। दु॒हे न कु॒म्भी स्व॒धां पि॒तृभ्य॑। मुख॒ सद॑स्य॒ शिर॒ इत्सदे॑न। जि॒ह्वा प॒वित्र॑म॒श्विना॒ स सर॑स्वती। चप्प॒न्न पा॒युर्भि॒षग॑स्य॒ वाल॑। व॒स्तिर्न शेपो॒ हर॑सा तर॒स्वी। अ॒श्विभ्यां॒ चक्षु॑र॒मृतं॒ ग्रहाभ्याम्। छागे॑न॒ तेजो॑ ह॒विषा॑ शृ॒तेन॑। पक्ष्मा॑णि गो॒धूमै॒ क्व॑लैरु॒तानि॑। पेशो॒ न शु॒क्लमसि॑तं वसाते॥१७॥

%2.6.4.5
अवि॒र्न मे॒षो न॒सि वी॒र्या॑य। प्रा॒णस्य॒ पन्था॑ अ॒मृतो॒ ग्रहाभ्याम्। सर॑स्व॒त्युप॒वाकैर्व्या॒नम्। नस्या॑नि ब॒र्॒हिर्बद॑रैर्जजान। इन्द्र॑स्य रू॒पमृ॑ष॒भो बला॑य। कर्णाभ्या॒ श्रोत्र॑म॒मृत॒ङ्ग्रहाभ्याम्। यवा॒ न ब॒र्॒हिर्भ्रु॒वि केस॑राणि। क॒र्कन्धु॑ जज्ञे॒ मधु॑ सार॒घं मुखात्। आ॒त्मन्नु॒पस्थे॒ न वृक॑स्य॒ लोम॑। मुखे॒ श्मश्रू॑णि॒ न व्याघ्रलो॒मम्॥१८॥

%2.6.4.6
केशा॒ न शी॒र्॒षन्‌ यश॑से श्रि॒यै शिखा। सि॒हस्य॒ लोम॒ त्विषि॑रिन्द्रि॒याणि॑। अङ्गान्या॒त्मन्भि॒षजा॒ तद॒श्विना। आ॒त्मान॒मङ्गै॒ सम॑धा॒त्सर॑स्वती। इन्द्र॑स्य रू॒प श॒तमा॑न॒मायु॑। च॒न्द्रेण॒ ज्योति॑र॒मृत॒न्दधा॑ना। सर॑स्वती॒ योन्या॒ङ्गर्भ॑म॒न्तः। अ॒श्विभ्यां॒ पत्नी॒ सुकृ॑तं बिभर्ति। अ॒पा रसे॑न॒ वरु॑णो॒ न साम्ना। इन्द्र श्रि॒यै ज॒नय॑न्न॒प्सु राजा। तेज॑ पशू॒ना ह॒विरि॑न्द्रि॒याव॑त्। प॒रि॒स्रुता॒ पय॑सा सार॒घं मधु॑। अ॒श्विभ्यान्दु॒ग्धं भि॒षजा॒ सर॑स्वत्या सुतासु॒ताभ्याम्। अ॒मृत॒ सोम॒ इन्दु॑॥१९॥\anuvakamend[अन्त॑र आ॒राद॒न्तर्व॑साते व्याघ्रलो॒म राजा॑ च॒त्वारि॑ च]

%2.6.5.1
मि॒त्रो॑ऽसि॒ वरु॑णोऽसि। सम॒हं विश्वैर्दे॒वैः। क्ष॒त्रस्य॒ नाभि॑रसि। क्ष॒त्रस्य॒ योनि॑रसि। स्यो॒नामा सी॑द। सु॒षदा॒मा सी॑द। मा त्वा॑ हिसीत्। मा मा॑ हिसीत्। निष॑साद धृ॒तव्र॑तो॒ वरु॑णः। प॒स्त्यास्वा॥२०॥

%2.6.5.2
साम्राज्याय सु॒क्रतु॑। दे॒वस्य॑ त्वा सवि॒तुः प्र॑स॒वे। अ॒श्विनोर्बा॒हुभ्याम्। पू॒ष्णो हस्ताभ्याम्। अ॒श्विनो॒र्भैष॑ज्येन। तेज॑से ब्रह्मवर्च॒साया॒भिषि॑ञ्चामि। दे॒वस्य॑ त्वा सवि॒तुः प्र॑स॒वे। अ॒श्विनोर्बा॒हुभ्याम्। पू॒ष्णो हस्ताभ्याम्। सर॑स्वत्यै॒ भैष॑ज्येन॥२१॥

%2.6.5.3
वी॒र्या॑या॒न्नाद्या॑या॒भिषि॑ञ्चामि। दे॒वस्य॑ त्वा सवि॒तुः प्र॑स॒वे। अ॒श्विनोर्बा॒हुभ्याम्। पू॒ष्णो हस्ताभ्याम्। इन्द्र॑स्येन्द्रि॒येण॑। श्रि॒यै यश॑से॒ बला॑या॒भिषि॑ञ्चामि। को॑ऽसि कत॒मो॑ऽसि। कस्मै त्वा॒ काय॑ त्वा। सुश्लो॒काँ (४) सुम॑ङ्ग॒लाँ (४) सत्य॑रा॒जा (३) न्। शिरो॑ मे॒ श्रीः॥२२॥

%2.6.5.4
यशो॒ मुखम्। त्विषि॒ केशाश्च॒ श्मश्रू॑णि। राजा॑ मे प्रा॒णो॑ऽमृतम्। स॒म्राट्चक्षु॑। वि॒राट्छ्रोत्रम्। जि॒ह्वा मे॑ भ॒द्रम्। वाङ्मह॑। मनो॑ म॒न्युः। स्व॒राड्भाम॑। मोदा प्रमो॒दा अ॒ङ्गुली॒रङ्गा॑नि॥२३॥

%2.6.5.5
चि॒त्तं मे॒ सह॑। बा॒हू मे॒ बल॑मिन्द्रि॒यम्। हस्तौ॑ मे॒ कर्म॑ वी॒र्यम्। आ॒त्मा क्ष॒त्रमुरो॒ मम॑। पृ॒ष्टीर्मे॑ रा॒ष्ट्रमु॒दर॒मसौ। ग्री॒वाश्च॒ श्रोण्यौ। ऊ॒रू अ॑र॒त्नी जानु॑नी। विशो॒ मेऽङ्गा॑नि स॒र्वत॑। नाभि॑र्मे चि॒त्तं वि॒ज्ञानम्। पा॒युर्मेऽप॑चितिर्भ॒सत्॥२४॥

%2.6.5.6
आ॒न॒न्द॒न॒न्दावा॒ण्डौ मे। भग॒ सौभाग्यं॒ पस॑। जङ्घाभ्यां प॒द्भ्यां धर्मोऽस्मि। वि॒शि राजा॒ प्रति॑ष्ठितः। प्रति॑ क्ष॒त्रे प्रति॑तिष्ठामि रा॒ष्ट्रे। प्रत्यश्वे॑षु॒ प्रति॑तिष्ठामि॒ गोषु॑। प्रत्यङ्गे॑षु॒ प्रति॑तिष्ठाम्या॒त्मन्। प्रति॑ प्रा॒णेषु॒ प्रति॑तिष्ठामि पु॒ष्टे। प्रति॒ द्यावा॑पृथि॒व्योः। प्रति॑तिष्ठामि य॒ज्ञे॥२५॥

%2.6.5.7
त्र॒या दे॒वा एका॑दश। त्र॒य॒स्त्रि॒शाः सु॒राध॑सः। बृह॒स्पति॑पुरोहिताः। दे॒वस्य॑ सवि॒तुः स॒वे। दे॒वा दे॒वैर॑वन्तु मा। प्र॒थ॒मा द्वि॒तीयै। द्वि॒तीयास्तृ॒तीयै। तृ॒तीया स॒त्येन॑। स॒त्यं य॒ज्ञेन॑। य॒ज्ञो यजु॑र्भिः॥२६॥

%2.6.5.8
यजूषि॒ साम॑भिः। सामान्यृ॒ग्भिः। ऋचो॑ या॒ज्या॑भिः। या॒ज्या॑ वषट्का॒रैः। व॒ष॒ट्का॒रा आहु॑तिभिः। आहु॑तयो मे॒ कामा॒न्त्सम॑र्धयन्तु। भूः स्वाहा। लोमा॑नि॒ प्रय॑ति॒र्मम॑। त्वङ्म॒ आन॑ति॒राग॑तिः। मा॒सं म॒ उप॑नतिः। वस्वस्थि॑। म॒ज्जा म॒ आन॑तिः॥२७॥\anuvakamend[प॒स्त्यास्वा सर॑स्वत्यै॒ भैष॑ज्येन॒ श्रीरङ्गा॑नि भ॒सद्य॒ज्ञे य॒ज्ञो यजु॑र्भि॒रुप॑नति॒र्द्वे च॑]

%2.6.6.1
यद्दे॑वा देव॒हेड॑नम्। देवा॑सश्चकृ॒मा व॒यम्। अ॒ग्निर्मा॒ तस्मा॒देन॑सः। विश्वान्मुञ्च॒त्वह॑सः। यदि॒ दिवा॒ यदि॒ नक्तम्। एनासि चकृ॒मा व॒यम्। वा॒युर्मा॒ तस्मा॒देन॑सः। विश्वान्मुञ्च॒त्वह॑सः। यदि॒ जाग्र॒द्यदि॒ स्वप्ने। एनासि चकृ॒मा व॒यम्॥२८॥

%2.6.6.2
सूर्यो॑ मा॒ तस्मा॒देन॑सः। विश्वान्मुञ्च॒त्वह॑सः। यद्ग्रामे॒ यदर॑ण्ये। यत्स॒भायां॒ यदि॑न्द्रि॒ये। यच्छू॒द्रे यद॒र्ये। एन॑श्चकृ॒मा व॒यम्। यदेक॒स्याधि॒ धर्म॑णि। तस्या॑व॒यज॑नमसि। यदापो॒ अघ्नि॑या॒ वरु॒णेति॒ शपा॑महे। ततो॑ वरुण नो मुञ्च॥२९॥

%2.6.6.3
अव॑भृथ निचङ्कुण निचे॒रुर॑सि निचङ्कुण। अव॑ दे॒वैर्दे॒वकृ॑त॒मेनो॑ऽयाट्। अव॒ मर्त्यै॒र्मर्त्य॑कृतम्। उ॒रोरा नो॑ देव रि॒षस्पा॑हि। सु॒मि॒त्रा न॒ आप॒ ओष॑धयः सन्तु। दु॒र्मि॒त्रास्तस्मै॑ भूयासुः। योऽस्मान्द्वेष्टि॑। यं च॑ व॒यं द्वि॒ष्मः। द्रु॒प॒दादि॒वेन्मु॑मुचा॒नः। स्वि॒न्नः स्ना॒त्वी मला॑दिव॥३०॥

%2.6.6.4
पू॒तं प॒वित्रे॑णे॒वाज्यम्। आप॑ शुन्धन्तु॒ मैन॑सः। उद्व॒यन्तम॑स॒स्परि॑। पश्य॑न्तो॒ ज्योति॒रुत्त॑रम्। दे॒वन्दे॑व॒त्रा सूर्यम्। अग॑न्म॒ ज्योति॑रुत्त॒मम्। प्रति॑युतो॒ वरु॑णस्य॒ पाश॑। प्रत्य॑स्तो॒ वरु॑णस्य॒ पाश॑। एधोऽस्येधिषी॒महि॑। स॒मिद॑सि ॥३१॥

%2.6.6.5
तेजो॑ऽसि॒ तेजो॒ मयि॑ धेहि। अ॒पो अन्व॑चारिषम्। रसे॑न॒ सम॑सृक्ष्महि। पय॑स्वा अग्न॒ आग॑मम्। तं मा॒ ससृ॑ज॒ वर्च॑सा। प्र॒जया॑ च॒ धने॑न च। स॒माव॑वर्ति पृथि॒वी। समु॒षाः। समु॒ सूर्य॑। समु॒ विश्व॑मि॒दञ्जग॑त्। वै॒श्वा॒न॒रज्यो॑तिर्भूयासम्। वि॒भुङ्काम॒व्व्यँ॑श्ञवै। भूः स्वाहा॥३२॥\anuvakamend[स्वप्न॒ एनासि चकृ॒मा व॒यं मु॑ञ्च॒ मला॑दिव स॒मिद॑सि॒ जग॒त्रीणि॑ च]

%2.6.7.1
होता॑ यक्षत्स॒मिधेन्द्र॑मि॒डस्प॒दे। नाभा॑ पृथि॒व्या अधि॑। दि॒वो वर्ष्म॒न्त्समि॑ध्यते। ओजि॑ष्ठश्चर्‌षणी॒ सहान्॑। वेत्वाज्य॑स्य॒ होत॒र्यज॑। होता॑ यक्ष॒त्तनू॒नपा॑तम्। ऊ॒तिभि॒र्जेता॑र॒मप॑राजितम्। इन्द्रं॑ दे॒व सु॑व॒र्विदम्। प॒थिभि॒र्मधु॑मत्तमैः। नरा॒शसे॑न॒ तेज॑सा॥३३॥

%2.6.7.2
वेत्वाज्य॑स्य॒ होत॒र्यज॑। होता॑ यक्ष॒दिडा॑भि॒रिन्द्र॑मीडि॒तम्। आ॒जुह्वा॑न॒मम॑र्त्यम्। दे॒वो दे॒वैः सवीर्यः। वज्र॑हस्तः पुरन्द॒रः। वेत्वाज्य॑स्य॒ होत॒र्यज॑। होता॑यक्षद्ब॒र्॒हिषीन्द्र॑न्निषद्व॒रम्। वृ॒ष॒भन्नर्या॑पसम्। वसु॑भीरु॒द्रैरा॑दि॒त्यैः। स॒युग्भि॑र्ब॒र्॒हिरास॑दत्॥३४॥

%2.6.7.3
वेत्वाज्य॑स्य॒ होत॒र्यज॑। होता॑ यक्ष॒दोजो॒ न वी॒र्यम्। सहो॒ द्वार॒ इन्द्र॑मवर्धयन्। सु॒प्रा॒य॒णा विश्र॑यन्तामृता॒वृध॑। द्वार॒ इन्द्रा॑य मी॒ढुषे। वि॒यन्त्वाज्य॑स्य॒ होत॒र्यज॑। होता॑ यक्षदु॒षे इन्द्र॑स्य धे॒नू। सु॒दुघे॑ मा॒तरौ॑ म॒ही। सवा॒तरौ॒ न तेज॑सी। व॒त्समिन्द्र॑मवर्धताम्॥३५॥

%2.6.7.4
वी॒तामाज्य॑स्य॒ होत॒र्यज॑। होता॑ यक्ष॒द्दैव्या॒ होता॑रा। भि॒षजा॒ सखा॑या। ह॒विषेन्द्रं॑ भिषज्यतः। क॒वी दे॒वौ प्रचे॑तसौ। इन्द्रा॑य धत्त इन्द्रि॒यम्। वी॒तामाज्य॑स्य॒ होत॒र्यज॑। होता॑ यक्षत्ति॒स्रो दे॒वीः। त्रय॑स्त्रि॒धात॑वो॒पस॑। इडा॒ सर॑स्वती॒ भार॑ती॥३६॥

%2.6.7.5
म॒हीन्द्र॑पत्नीर्‌ह॒विष्म॑तीः। वि॒यन्त्वाज्य॑स्य॒ होत॒र्यज॑। होता॑ यक्ष॒त्त्वष्टा॑र॒मिन्द्रं॑ दे॒वम्। भि॒षज सु॒यज॑ङ्घृत॒श्रियम्। पु॒रु॒रूप सु॒रेत॑सं म॒घोनिम्। इन्द्रा॑य॒ त्वष्टा॒ दध॑दिन्द्रि॒याणि॑। वेत्वाज्य॑स्य॒ होत॒र्यज॑। होता॑ यक्ष॒द्वन॒स्पतिम्। श॒मि॒तार श॒तक्र॑तुम्। धि॒यो जो॒ष्टार॑मिन्द्रि॒यम्॥३७॥

%2.6.7.6
मध्वा॑ सम॒ञ्जन्प॒थिभि॑ सु॒गेभि॑। स्वदा॑ति ह॒व्यं मधु॑ना घृ॒तेन॑। वेत्वाज्य॑स्य॒ होत॒र्यज॑। होता॑ यक्ष॒दिन्द्र॒ स्वाहाऽऽज्य॑स्य। स्वाहा॒ मेद॑सः। स्वाहा स्तो॒कानाम्। स्वाहा॒ स्वाहा॑कृतीनाम्। स्वाहा॑ ह॒व्यसूक्तीनाम्। स्वाहा॑ दे॒वा आज्य॒पान्। स्वाहेन्द्र हो॒त्राज्जु॑षा॒णाः। इन्द्र॒ आज्य॑स्य वियन्तु। होत॒र्यज॑॥३८॥\anuvakamend[तेज॑साऽऽसददवर्धतां॒ भार॑तीन्द्रि॒यं जु॑षा॒णा द्वे च॑ (स॒मिधेन्द्र॒न्तनू॒नपा॑त॒मिडा॑भिर्ब॒र्॒हिष्योज॑ उ॒षे दैव्या॑ ति॒स्रस्त्वष्टा॑रं॒ वन॒स्पति॒मिन्द्रम् ॥ स॒मिधेन्द्रं॑ च॒तुर्वेत्वेको॑ वि॒यन्तु॒ द्विर्वी॒तामेको॑ वि॒यन्तु॒ द्विर्वेत्वेको॑ वि॒यन्तु॒ होत॒र्यज॑ ॥ )]

%2.6.8.1
समि॑द्ध॒ इन्द्र॑ उ॒षसा॒मनी॑के। पु॒रो॒रुचा॑ पूर्व॒कृद्वा॑वृधा॒नः। त्रि॒भिर्दे॒वैस्त्रि॒शता॒ वज्र॑बाहुः। ज॒घान॑ वृ॒त्रं वि दुरो॑ ववार। नरा॒शस॒ प्रति॒शूरो॒ मिमा॑नः। तनू॒नपा॒त्प्रति॑ य॒ज्ञस्य॒ धाम॑। गोभि॑र्व॒पावा॒न्मधु॑ना सम॒ञ्जन्। हिर॑ण्यैश्च॒न्द्री य॑जति॒ प्रचे॑ताः। ई॒डि॒तो दे॒वैर्‌हरि॑वा अभि॒ष्टिः। आ॒जुह्वा॑नो ह॒विषा॒ शर्ध॑मानः॥३९॥

%2.6.8.2
पु॒र॒न्द॒रो म॒घवा॒न् वज्र॑बाहुः। आया॑तु य॒ज्ञमुप॑नो जुषा॒णः। जु॒षा॒णो ब॒र्‌हिर्हरि॑वान्न॒ इन्द्र॑। प्रा॒चीन सीदत्प्र॒दिशा॑ पृथि॒व्याः। उ॒रु॒व्यचा॒ प्रथ॑मान स्यो॒नम्। आ॒दि॒त्यैर॒क्तं वसु॑भिः स॒जोषा। इन्द्र॒न्दुर॑ कव॒ष्यो॑ धाव॑मानाः। वृषा॑णं यन्तु॒ जन॑यः सु॒पत्नी। द्वारो॑ दे॒वीर॒भितो॒ विश्र॑यन्ताम्। सु॒वीरा॑ वी॒रं प्रथ॑माना॒ महो॑भिः॥४०॥

%2.6.8.3
उ॒षासा॒नक्ता॑ बृह॒ती बृ॒हन्तम्। पय॑स्वती सु॒दुघे॒ शूर॒मिन्द्रम्। पेश॑स्वती॒ तन्तु॑ना स॒व्व्यँय॑न्ती। दै॒वानां दे॒वं य॑जतः सुरु॒क्मे। दैव्या॒ मिमा॑ना॒ मन॑सा पुरु॒त्रा। होता॑रा॒विन्द्रं॑ प्रथ॒मा सु॒वाचा। मू॒र्धन् य॒ज्ञस्य॒ मधु॑ना॒ दधा॑ना। प्रा॒चीनं॒ ज्योति॑र्\mbox{}ह॒विषा॑ वृधातः। ति॒स्रो दे॒वीर्‌ ह॒विषा॒ वर्ध॑मानाः। इन्द्रं॑ जुषा॒णा वृष॑ण॒न्न पत्नी॥४१॥

%2.6.8.4
अच्छि॑न्न॒न्तन्तुं॒ पय॑सा॒ सर॑स्वती। इडा॑ दे॒वी भार॑ती वि॒श्वतूर्तिः। त्वष्टा॒ दध॒दिन्द्रा॑य॒ शुष्मम्। अपा॒कोचि॑ष्टुर्य॒शसे॑ पु॒रूणि॑। वृषा॒ यज॒न्वृष॑णं॒ भूरि॑रेताः। मू॒र्धन् य॒ज्ञस्य॒ सम॑नक्तु दे॒वान्। वन॒स्पति॒रव॑सृष्टो॒ न पाशै। त्मन्या॑ सम॒ञ्जञ्छ॑मि॒ता न दे॒वः। इन्द्र॑स्य ह॒व्यैर्ज॒ठरं॑ पृणा॒नः। स्वदा॑ति ह॒व्यं मधु॑ना घृ॒तेन॑। स्तो॒काना॒मिन्दुं॒ प्रति॒ शूर॒ इन्द्र॑। वृ॒षा॒यमा॑णो वृष॒भस्तु॑रा॒षाट्। घृ॒त॒प्रुषा॒ मधु॑ना ह॒व्यमु॒न्दन्। मू॒र्धन् य॒ज्ञस्य॑ जुषता॒ स्वाहा॥४२॥\anuvakamend[शर्ध॑मानो॒ महो॑भि॒ पत्नीर्घृ॒तेन॑ च॒त्वारि॑ च]

%2.6.9.1
आच॑र्‌षणि॒प्रा वि॒वेष॒ यन्मा। त स॒ध्रीची। स॒त्यमित्तन्न त्वावा अ॒न्यो अस्ति॑। इन्द्र॑ दे॒वो न मर्त्यो॒ ज्यायान्॑। अह॒न्नहिं॑ परि॒शया॑न॒मर्ण॑। अवा॑सृजो॒ऽपो अच्छा॑ समु॒द्रम्। प्रस॑साहिषे पुरुहूत॒ शत्रून्॑। ज्येष्ठ॑स्ते॒ शुष्म॑ इ॒ह रा॒तिर॑स्तु। इन्द्रा भ॑र॒ दक्षि॑णेना॒ वसू॑नि। पति॒ सिन्धू॑नामसि रे॒वती॑नाम्। स शेवृ॑ध॒मधि॑ धाद्द्यु॒म्नम॒स्मे। महि॑ क्ष॒त्रं ज॑ना॒षाडि॑न्द्र॒ तव्यम्। रक्षा॑ च नो म॒घोन॑ पा॒हि सू॒रीन्। रा॒ये च॑ नः स्वप॒त्या इ॒षे धा॥४३॥\anuvakamend[रे॒वती॑नाञ्च॒त्वारि॑ च]

%2.6.10.1
दे॒वं ब॒र्॒हिरिन्द्र सुदे॒वन्दे॒वैः। वी॒रव॑त्स्ती॒र्णं वेद्या॑मवर्धयत्। वस्तोर्वृ॒तं प्राक्तोर्भृ॒तम्। रा॒या ब॒र्॒हिष्म॒तोऽत्य॑गात्। व॒सु॒वने॑ वसु॒धेय॑स्य वेतु॒ यज॑। दे॒वीर्द्वार॒ इन्द्र सङ्घा॒ते। वि॒ड्वीर्याम॑न्नवर्धयन्। आ व॒त्सेन॒ तरु॑णेन कुमा॒रेण॑ चमीवि॒ता अपार्वा॑णम्। रे॒णुक॑काटन्नुदन्ताम्। व॒सु॒वने॑ वसु॒धेय॑स्य वियन्तु॒ यज॑॥४४॥

%2.6.10.2
दे॒वी उ॒षासा॒नक्ता। इन्द्रं॑ य॒ज्ञे प्र॑य॒त्य॑ह्वेताम्। दैवी॒र्विश॒ प्राया॑सिष्टाम्। सुप्री॑ते॒ सुधि॑ते अभूताम्। व॒सु॒वने॑ वसु॒धेय॑स्य वीतां॒ यज॑। दे॒वी जोष्ट्री॒ वसु॑धिती। दे॒वमिन्द्र॑मवर्धताम्। अयाव्य॒न्याघा द्वेषासि। आन्यावाक्षी॒द्वसु॒ वार्या॑णि। यज॑मानाय शिक्षि॒ते॥४५॥

%2.6.10.3
व॒सु॒वने॑ वसु॒धेय॑स्य वीतां॒ यज॑। दे॒वी ऊ॒र्जाहु॑ती॒ दुघे॑ सु॒दुघे। पय॒सेन्द्र॑मवर्धताम्। इष॒मूर्ज॑म॒न्याऽवाक्षीत्। सग्धि॒ सपी॑तिम॒न्या। नवे॑न॒ पूर्व॒न्दय॑माने। पु॒रा॒णेन॒ नवम्। अधा॑ता॒मूर्ज॑मू॒र्जाहु॑ती॒ वसु॒ वार्या॑णि। यज॑मानाय शिक्षि॒ते। व॒सु॒वने॑ वसु॒धेय॑स्य वीतां॒ यज॑॥४६॥

%2.6.10.4
दे॒वा दैव्या॒ होता॑रा। दे॒वमिन्द्र॑मवर्धताम्। ह॒ताघ॑शसा॒वाभार्ष्टां॒ वसु॒वार्या॑णि। यज॑मानाय शिक्षि॒तौ। व॒सु॒वने॑ वसु॒धेय॑स्य वीतां॒ यज॑। दे॒वीस्ति॒स्रस्ति॒स्रो दे॒वीः। पति॒मिन्द्र॑मवर्धयन्। अस्पृ॑क्ष॒द्भार॑ती॒ दिवम्। रु॒द्रैर्य॒ज्ञ सर॑स्वती। इडा॒ वसु॑मती गृ॒हान्॥४७॥

%2.6.10.5
व॒सु॒वने॑ वसु॒धेय॑स्य वियन्तु॒ यज॑। दे॒व इन्द्रो॒ नरा॒शस॑। त्रि॒व॒रू॒थस्त्रि॑वन्धु॒रः। दे॒वमिन्द्र॑मवर्धयत्। श॒तेन॑ शितिपृ॒ष्ठाना॒माहि॑तः। स॒हस्रे॑ण॒ प्रव॑र्तते। मि॒त्रावरु॒णेद॑स्य हो॒त्रमर्‌ह॑तः। बृह॒स्पति॑ स्तो॒त्रम्। अ॒श्विनाऽऽध्व॑र्यवम्। व॒सु॒वने॑ वसु॒धेय॑स्य वेतु॒ यज॑॥४८॥

%2.6.10.6
दे॒व इन्द्रो॒ वन॒स्पति॑। हिर॑ण्यवर्णो॒ मधु॑शाखः सुपिप्प॒लः। दे॒वमिन्द्र॑मवर्धयत्। दिव॒मग्रे॑णाप्रात्। आऽन्तरि॑क्षं पृथि॒वीम॑दृहीत्। व॒सु॒वने॑ वसु॒धेय॑स्य वेतु॒ यज॑। दे॒वं ब॒र्॒हिर्वारि॑तीनाम्। दे॒वमिन्द्र॑मवर्धयत्। स्वा॒स॒स्थमिन्द्रे॒णास॑न्नम्। अ॒न्या ब॒र्॒हीष्य॒भ्य॑भूत्। व॒सु॒वने॑ वसु॒धेयस्य॑ वेतु॒ यज॑। दे॒वो अ॒ग्निः स्वि॑ष्ट॒कृत्। दे॒वमिन्द्र॑मवर्धयत्। स्वि॑ष्टं कु॒र्वन्त्स्वि॑ष्ट॒कृत्। स्वि॑ष्टम॒द्य क॑रोतु नः। व॒सु॒वने॑ वसु॒धेय॑स्य वेतु॒ यज॑॥४९॥\anuvakamend[वि॒य॒न्तु॒ यज॑ शिक्षि॒ते शि॑क्षि॒ते व॑सु॒वने॑ वसु॒धेय॑स्य वीताँ॒य्यज॑ गृ॒हान् वे॑तु॒ यजा॑भू॒थ्षट्च॑ (दे॒वं ब॒र्॒हिर्दे॒वीर्द्वारो॑ दे॒वी उ॒षासा॒नक्ता॑ दे॒वी जोष्ट्री॑ दे॒वी ऊ॒र्जाहु॑ती दे॒वा दैव्या॒ होता॑रा शिक्षि॒तौ दे॒वीस्ति॒स्रस्ति॒स्रो दे॒वीर्दे॒व इन्द्रो॒ नरा॒शसो॑ दे॒व इन्द्रो॒ वन॒स्पति॑र्दे॒वं ब॒र्॒हिर्वारि॑तीनान्दे॒वो अ॒ग्निः स्वि॑ष्ट॒कृद्दे॒वम्। वे॒तु॒ वि॒य॒न्तु॒ च॒तुर्वी॑ता॒मेको॑ वियन्तु च॒तुर्वेत्ववर्धयदवर्धय॒न्त्रिर॑वर्धता॒मेको॑ऽ वर्धयश्च॒तुर॑वर्धयत्। वस्तो॒रा व॒त्सेन॒ दैवी॒रया॒वीषह॒ताऽस्पृ॑क्षच्छ॒तेन॒ दिव स्वास॒स्थ स्वि॑ष्ट शिक्षि॒ते शि॑क्षि॒ते शि॑क्षि॒तौ ॥ )]

%2.6.11.1
होता॑ यक्षत्स॒मिधा॒ऽग्निमि॒डस्प॒दे। अ॒श्विनेन्द्र॒ सर॑स्वतीम्। अ॒जो धू॒म्रो न गो॒धूमै॒ क्व॑लैर्भेष॒जम्। मधु॒ शष्पै॒र्न तेज॑ इन्द्रि॒यम्। पय॒ सोम॑ परि॒स्रुता॑ घृ॒तं मधु॑। वि॒यन्त्वाज्य॑स्य॒ होत॒र्यज॑। होता॑ यक्ष॒त्तनू॒नपा॒त्सर॑स्वती। अवि॑र्मे॒षो न भे॑ष॒जम्। प॒था मधु॑म॒ताभ॑रन्। अ॒श्विनेन्द्रा॑य वी॒र्यम्॥५०॥

%2.6.11.2
बद॑रैरुप॒वाका॑भिर्भेष॒जन्तोक्म॑भिः। पय॒ सोम॑ परि॒स्रुता॑ घृ॒तं मधु॑। वि॒यन्त्वाज्य॑स्य॒ होत॒र्यज॑। होता॑ यक्ष॒न्नरा॒शसं॒ न न॒ग्नहुम्। पति॒ सुरा॑यै भेष॒जम्। मे॒षः सर॑स्वती भि॒षक्। रथो॒ न च॒न्द्र्य॑श्विनोर्व॒पा इन्द्र॑स्य वी॒र्यम्। बद॑रैरुप॒वाका॑भिर्भेष॒जन्तोक्म॑भिः। पय॒ सोम॑ परि॒स्रुता॑ घृ॒तं मधु॑। वि॒यन्त्वाज्य॑स्य॒ होत॒र्यज॑॥५१॥

%2.6.11.3
होता॑ यक्षदि॒डेडि॒त आ॒जुह्वा॑न॒ सर॑स्वतीम्। इन्द्रं॒ बले॑न व॒र्धय\sn{}। ऋ॒ष॒भेण॒ गवेन्द्रि॒यम्। अ॒श्विनेन्द्रा॑य वी॒र्यम्। यवै क॒र्कन्धु॑भिः। मधु॑ ला॒जैर्न मास॑रम्। पय॒ सोम॑ परि॒स्रुता॑ घृ॒तं मधु॑। वि॒यन्त्वाज्य॑स्य॒ होत॒र्यज॑। होता॑ यक्षद्ब॒र्॒हिः सु॒ष्टरी॒मोर्ण॑म्रदाः। भि॒षङ्नास॑त्या॥५२॥

%2.6.11.4
भि॒षजा॒ऽश्विनाऽश्वा॒ शिशु॑मती। भि॒षग्धे॒नुः सर॑स्वती। भि॒षग्दु॒ह इन्द्रा॑य भेष॒जम्। पय॒ सोम॑ परि॒स्रुता॑ घृ॒तं मधु॑। वि॒यन्त्वाज्य॑स्य॒ होत॒र्यज॑। होता॑ यक्ष॒द्दुरो॒ दिश॑। क॒व॒ष्यो॑ न व्यच॑स्वतीः। अ॒श्विभ्या॒न्न दुरो॒ दिश॑। इन्द्रो॒ न रोद॑सी॒ दुघे। दु॒हे कामा॒न्त्सर॑स्वती॥५३॥

%2.6.11.5
अ॒श्विनेन्द्रा॑य भेष॒जम्। शु॒क्रन्न ज्योति॑रिन्द्रि॒यम्। पय॒ सोम॑ परि॒स्रुता॑ घृ॒तं मधु॑। वि॒यन्त्वाज्य॑स्य॒ होत॒र्यज॑। होता॑ यक्षत्सु॒पेश॑सो॒षे नक्त॒न्दिवा। अ॒श्विना॑ सञ्जाना॒ने। समं॑ जाते॒ सर॑स्वत्या। त्विषि॒मिन्द्रे॒ न भे॑ष॒जम्। श्ये॒नो न रज॑सा हृ॒दा। पय॒ सोम॑ परि॒स्रुता॑ घृ॒तं मधु॑॥५४॥

%2.6.11.6
वि॒यन्त्वाज्य॑स्य॒ होत॒र्यज॑। होता॑ यक्ष॒द्दैव्या॒ होता॑रा भि॒षजा॒ऽश्विना। इन्द्र॒न्न जागृ॑वी॒ दिवा॒ नक्त॒न्न भे॑ष॒जैः। शूष॒ सर॑स्वती भि॒षक्। सीसे॑न दुह इन्द्रि॒यम्। पय॒ सोम॑ परि॒स्रुता॑ घृ॒तं मधु॑। वि॒यन्त्वाज्य॑स्य॒ होत॒र्यज॑। होता॑ यक्षत्ति॒स्रो दे॒वीर्न भे॑ष॒जम्। त्रय॑स्त्रि॒धात॑वो॒ऽपस॑। रू॒पमिन्द्रे॑ हिर॒ण्ययम्॥५५॥

%2.6.11.7
अ॒श्विनेडा॒ न भार॑ती। वा॒चा सर॑स्वती। मह॒ इन्द्रा॑य दधुरिन्द्रि॒यम्। पय॒ सोम॑ परि॒स्रुता॑ घृ॒तं मधु॑। वि॒यन्त्वाज्य॑स्य॒ होत॒र्यज॑। होता॑ यक्ष॒त्त्वष्टा॑र॒मिन्द्र॑म॒श्विना। भि॒षज॒न्न सर॑स्वतीम्। ओजो॒ न जू॒तिरि॑न्द्रि॒यम्। वृको॒ न र॑भ॒सो भि॒षक्। यश॒ सुर॑या भेष॒जम्॥५६॥

%2.6.11.8
श्रि॒या न मास॑रम्। पय॒ सोम॑ परि॒स्रुता॑ घृ॒तं मधु॑। वि॒यन्त्वाज्य॑स्य॒ होत॒र्यज॑। होता॑ यक्ष॒द्वन॒स्पतिम्। श॒मि॒तार श॒तक्र॑तुम्। भी॒मन्न म॒न्यु राजा॑नव्व्याँ॒घ्रन्नम॑सा॒ऽश्विना॒ भामम्। सर॑स्वती भि॒षक्। इन्दा॑य दुह इन्द्रि॒यम्। पय॒ सोम॑ परि॒स्रुता॑ घृ॒तं मधु॑। वि॒यन्त्वाज्य॑स्य॒ होत॒र्यज॑॥५७॥

%2.6.11.9
होता॑ यक्षद॒ग्नि स्वाहाऽऽज्य॑स्य स्तो॒कानाम्। स्वाहा॒ मेद॑सां॒ पृथ॑क्। स्वाहा॒ छाग॑म॒श्विभ्याम्। स्वाहा॑ मे॒ष सर॑स्वत्यै। स्वाह॑र्‌ष॒भमिन्द्रा॑य सि॒हाय॒ सह॑सेन्द्रि॒यम्। स्वाहा॒ऽग्निन्न भे॑ष॒जम्। स्वाहा॒ सोम॑मिन्द्रि॒यम्। स्वाहेन्द्र सु॒त्रामा॑ण सवि॒तारं॒ वरु॑णं भि॒षजां॒ पतिम्। स्वाहा॒ वन॒स्पतिं॑ प्रि॒यं पाथो॒ न भे॑ष॒जम्। स्वाहा॑ दे॒वा आज्य॒पान्॥५८॥

%2.6.11.10
स्वाहा॒ऽग्नि हो॒त्राज्जु॑षा॒णो अ॒ग्निर्भे॑ष॒जम्। पय॒ सोम॑ परि॒स्रुता॑ घृ॒तं मधु॑। वि॒यन्त्वाज्य॑स्य॒ होत॒र्यज॑। होता॑ यक्षद॒श्विना॒ सर॑स्वती॒मिन्द्र सु॒त्रामा॑णम्। इ॒मे सोमा सु॒रामा॑णः। छागै॒र्न मे॒षैर्\mbox{}ऋ॑ष॒भैः सु॒ताः। शष्पै॒र्न तोक्म॑भिः। ला॒जैर्मह॑स्वन्तः। मदा॒ मास॑रेण॒ परि॑ष्कृताः। शु॒क्राः पय॑स्वन्तो॒ऽमृता। प्रस्थि॑ता वो मधु॒श्चुत॑। तान॒श्विना॒ सर॑स्व॒तीन्द्र॑ सु॒त्रामा॑ वृत्र॒हा। जु॒षन्ता सौ॒म्यं मधु॑। पिब॑न्तु॒ मद॑न्तु वि॒यन्तु॒ सोमम्। होत॒र्यज॑॥५९॥\anuvakamend[वी॒र्यं॑ वि॒यन्त्वाज्य॑स्य॒ होत॒र्यज॒ नास॑त्या॒ सर॑स्वती॒ मधु॑ हिर॒ण्ययं॑ भेष॒जं वि॒यन्त्वाज्य॑स्य॒ होत॒र्यजाज्य॒पान॒मृता॒ पञ्च॑ च (स॒मिधा॒ऽग्नि षट्। तनू॒नपात्स॒प्त। नरा॒शस॒मृषि॑। इ॒डेडि॒तो यवै॑र॒ष्टौ। ब॒र्‌हि॒ स॒प्त। दुरो॒ऽश्विना॒ नव॑। सु॒पेश॒सर्‌षि॑। दैव्या॒ होता॑रा॒ सीसे॑न॒ रस॑। ति॒स्रस्त्वष्टा॑रम॒ष्टाव॑ष्टौ। वन॒स्पति॒मृषि॑। अ॒ग्निन्त्रयो॑दश। अ॒श्विना॒ द्वाद॑श त्रयोदश। स॒मिधा॒ऽग्निं बद॑रै॒र्बद॑रै॒र्यवै॑र॒श्विना॒ त्विषि॑म॒श्विना॒ न भे॑ष॒ज रू॒पम॒श्विना॑ भी॒मं भामम् ॥ )]

%2.6.12.1
समि॑द्धो अ॒ग्निर॑श्विना। त॒प्तो घ॒र्मो वि॒राट्त्सु॒तः। दु॒हे धे॒नुः सर॑स्वती। सोम शु॒क्रमि॒हेन्द्रि॒यम्। त॒नू॒पा भि॒षजा॑ सु॒ते। अ॒श्विनो॒भा सर॑स्वती। मध्वा॒ रजासीन्द्रि॒यम्। इन्द्रा॑य प॒थिभि॑र्वहान्। इन्द्रा॒येन्दु॒ सर॑स्वती। नरा॒शसे॑न न॒ग्नहु॑॥६०॥

%2.6.12.2
अधा॑ताम॒श्विना॒ मधु॑। भे॒ष॒जं भि॒षजा॑ सु॒ते। आ॒जुह्वा॑ना॒ सर॑स्वती। इन्द्रा॑येन्द्रि॒याणि॑ वी॒र्यम्। इडा॑भिर॒श्विना॒विषम्। समूर्ज॒ स र॒यिन्द॑धुः। अश्वि॑ना॒ नमु॑चेः सु॒तम्। सोम शु॒क्रं प॑रि॒स्रुता। सर॑स्वती॒ तमाभ॑रत्। ब॒र्॒हिषेन्द्रा॑य॒ पात॑वे॥६१॥

%2.6.12.3
क॒व॒ष्यो॑ न व्यच॑स्वतीः। अ॒श्विभ्यां॒ न दु॒रो दिश॑। इन्द्रो॒ न रोद॑सी॒ दुघे। दु॒हे कामा॒न्त्सर॑स्वती। उ॒षासा॒ नक्त॑मश्विना। दिवेन्द्र सा॒यमि॑न्द्रि॒यैः। स॒ञ्जा॒ना॒ने सु॒पेश॑सा। समं॑ जाते॒ सर॑स्वत्या। पा॒तन्नो॑ अश्विना॒ दिवा। पा॒हि नक्त सरस्वति॥६२॥

%2.6.12.4
दैव्या॑ होतारा भिषजा। पा॒तमिन्द्र॒ सचा॑ सु॒ते। ति॒स्रस्त्रे॒धा सर॑स्वती। अ॒श्विना॒ भार॒तीडा। ती॒व्रं प॑रि॒स्रुता॒ सोमम्। इन्द्रा॑य सुषवु॒र्मदम्। अश्वि॑ना भेष॒जं मधु॑। भे॒ष॒जन्न॒ सर॑स्वती। इन्द्रे॒ त्वष्टा॒ यश॒ श्रियम्। रू॒प रू॑पमधुः सु॒ते। ऋ॒तु॒थेन्द्रो॒ वन॒स्पति॑। श॒श॒मा॒नः प॑रि॒स्रुता। की॒लाल॑म॒श्विभ्यां॒ मधु॑। दु॒हे धे॒नुः सर॑स्वती। गोभि॒र्न सोम॑मश्विना। मास॑रेण परि॒ष्कृता। सम॑धाता॒ सर॑स्वत्या। स्वाहेन्द्रे॑ सु॒तं मधु॑॥६३॥\anuvakamend[न॒ग्नहु॒ पात॑वे सरस्वत्यधुः सु॒तेऽष्टौ च॑]

%2.6.13.1
अ॒श्विना॑ ह॒विरि॑न्द्रि॒यम्। नमु॑चेर्धि॒या सर॑स्वती। आ शु॒क्रमा॑सु॒राद्व॒सु। म॒घमिन्द्रा॑य जभ्रिरे। यम॒श्विना॒ सर॑स्वती। ह॒विषेन्द्र॒मव॑र्धयन्। स बि॑भेद व॒लं म॒घम्। नमु॑चावासु॒रे सचा। तमिन्द्रं॑ प॒शव॒ सचा। अ॒श्विनो॒भा सर॑स्वती॥६४॥

%2.6.13.2
दधा॑ना अ॒भ्य॑नूषत। ह॒विषा॑ य॒ज्ञमि॑न्द्रि॒यम्। य इन्द्र॑ इन्द्रि॒यन्द॒धुः। स॒वि॒ता वरु॑णो॒ भग॑। स सु॒त्रामा॑ ह॒विष्प॑तिः। यज॑मानाय सश्चत। स॒वि॒ता वरु॑णो॒ऽदध॑त्। यज॑मानाय दा॒शुषे। आद॑त्त॒ नमु॑चे॒र्वसु॑। सु॒त्रामा॒ बल॑मिन्द्रि॒यम्॥६५॥

%2.6.13.3
वरु॑णः क्ष॒त्रमि॑न्द्रि॒यम्। भगे॑न सवि॒ता श्रियम्। सु॒त्रामा॒ यश॑सा॒ बलम्। दधा॑ना य॒ज्ञमा॑शत। अश्वि॑ना॒ गोभि॑रिन्द्रि॒यम्। अश्वे॑भिर्वी॒र्यं॑ बलम्। ह॒विषेन्द्र॒ सर॑स्वती। यज॑मानमवर्धयन्। ता नास॑त्या सु॒पेश॑सा। हिर॑ण्यवर्तनी॒ नरा। सर॑स्वती ह॒विष्म॑ती। इन्द्र॒ कर्म॑सु नोऽवत। ता भि॒षजा॑ सु॒कर्म॑णा। सा सु॒दुघा॒ सर॑स्वती। स वृ॑त्र॒हा श॒तक्र॑तुः। इन्द्रा॑य दधुरिन्द्रि॒यम्॥६६॥\anuvakamend[उ॒भा सर॑स्वती॒ बल॑मिन्द्रि॒यन्नरा॒ षट्च॑]

%2.6.14.1
दे॒वं ब॒र्॒हिः स॑रस्वती। सु॒दे॒वमिन्द्रे॑ अ॒श्विना। तेजो॒ न चक्षु॑र॒क्ष्योः। ब॒र्॒हिषा॑ दधुरिन्द्रि॒यम्। व॒सु॒वने॑ वसु॒धेय॑स्य वियन्तु॒ यज॑। दे॒वीर्द्वारो॑ अ॒श्विना। भि॒षजेन्द्रे॒ सर॑स्वती। प्रा॒णन्न वी॒र्य॑न्न॒सि। द्वारो॑ दधुरिन्द्रि॒यम्। व॒सु॒वने॑ वसु॒धेय॑स्य वियन्तु॒ यज॑॥६७॥

%2.6.14.2
दे॒वी उ॒षासा॑व॒श्विना। भि॒षजेन्द्रे॒ सर॑स्वती। बल॒न्न वाच॑मा॒स्ये। उ॒षाभ्यान्दधुरिन्द्रि॒यम्। व॒सु॒वने॑ वसु॒धेय॑स्य वियन्तु॒ यज॑। दे॒वी जोष्ट्री॑ अ॒श्विना। सु॒त्रामेन्द्रे॒ सर॑स्वती। श्रोत्र॒न्न कर्ण॑यो॒र्यश॑। जोष्ट्रीभ्यान्दधुरिन्द्रि॒यम्। व॒सु॒वने॑ वसु॒धेय॑स्य वियन्तु॒ यज॑॥६८॥

%2.6.14.3
दे॒वी ऊ॒र्जाहु॑ती॒ दुघे॑ सु॒दुघे। पय॒सेन्द्र॒ सर॑स्वत्य॒श्विना॑ भि॒षजा॑वत। शु॒क्रन्न ज्योति॒ स्तन॑यो॒राहु॑ती धत्त इन्द्रि॒यम्। व॒सु॒वने॑ वसु॒धेय॑स्य वियन्तु॒ यज॑। दे॒वा दे॒वानां भि॒षजा। होता॑रा॒विन्द्र॑म॒श्विना। व॒ष॒ट्का॒रैः सर॑स्वती। त्विषि॒न्न हृद॑ये म॒तिम्। होतृ॑भ्यान्दधुरिन्द्रि॒यम्। व॒सु॒वने॑ वसु॒धेय॑स्य वियन्तु॒ यज॑॥६९॥

%2.6.14.4
दे॒वीस्ति॒स्रस्ति॒स्रो दे॒वीः। सर॑स्वत्य॒श्विना॒ भार॒तीडा। शूष॒न्न मध्ये॒ नाभ्याम्। इन्द्रा॑य दधुरिन्द्रि॒यम्। व॒सु॒वने॑ वसु॒धेय॑स्य वियन्तु॒ यज॑। दे॒व इन्द्रो॒ नरा॒शस॑। त्रि॒व॒रू॒थः सर॑स्वत्या॒ऽश्विभ्या॑मीयते॒ रथ॑। रेतो॒ न रू॒पम॒मृतं॑ ज॒नित्रम्। इन्द्रा॑य॒ त्वष्टा॒ दध॑दिन्द्रि॒याणि॑। व॒सु॒वने॑ वसु॒धेय॑स्य वियन्तु॒ यज॑॥७०॥

%2.6.14.5
दे॒व इन्द्रो॒ वन॒स्पति॑। हिर॑ण्यपर्णो अ॒श्विभ्याम्। सर॑स्वत्याः सुपिप्प॒लः। इन्द्रा॑य पच्यते॒ मधु॑। ओजो॒ न जू॒तिमृ॑ष॒भो न भामम्। वन॒स्पति॑र्नो॒ दध॑दिन्द्रि॒याणि॑। व॒सु॒वने॑ वसु॒धेय॑स्य वियन्तु॒ यज॑। दे॒वं ब॒र्॒हिर्वारि॑तीनाम्। अ॒ध्व॒रे स्ती॒र्णम॒श्विभ्याम्। ऊर्ण॑म्रदा॒ सर॑स्वत्याः॥७१॥

%2.6.14.6
स्यो॒नमि॑न्द्र ते॒ सद॑। ई॒शायै॑ म॒न्यु राजा॑नं ब॒र्॒हिषा॑ दधुरिन्द्रि॒यम्। व॒सु॒वने॑ वसु॒धेय॑स्य वियन्तु॒ यज॑। दे॒वो अ॒ग्निः स्वि॑ष्ट॒कृत्। दे॒वान् य॑क्षद्यथाय॒थम्। होता॑रा॒विन्द्र॑म॒श्विना। वा॒चा वाच॒ सर॑स्वतीम्। अ॒ग्नि सोम स्विष्ट॒कृत्। स्वि॑ष्ट॒ इन्द्र॑ सु॒त्रामा॑ सवि॒ता वरु॑णो भि॒षक्। इ॒ष्टो दे॒वो वन॒स्पति॑। स्वि॑ष्टा दे॒वा आज्य॒पाः। इ॒ष्टो अ॒ग्निर॒ग्निना। होता॑ हो॒त्रे स्वि॑ष्ट॒कृत्। यशो॒ न दध॑दिन्द्रि॒यम्। ऊर्ज॒मप॑चिति स्व॒धाम्। व॒सु॒वने॑ वसु॒धेय॑स्य वियन्तु॒ यज॑॥७२॥\anuvakamend[द्वारो॑ दधुरिन्द्रि॒यं व॑सु॒वने॑ वसु॒धेय॑स्य वियन्तु॒ यज॒ जोष्ट्रीभ्यान्दधुरिन्द्रि॒यं व॑सु॒वने॑ वसु॒धेय॑स्य वियन्तु॒ यज॒ होतृ॑भ्यान्दधुरिन्द्रि॒यं व॑सु॒वने॑ वसु॒धेय॑स्य वियन्तु॒ यजेन्द्रि॒याणि॑ वसु॒वने॑ वसु॒धेय॑स्य वियन्तु॒ यज॒ सर॑स्वत्या॒ वन॒स्पति॒ष्षट्च॑ (दे॒वं ब॒र्॒हिर्दे॒वीर्द्वारो॑ दे॒वी उ॒षासा॑व॒श्विना॑ दे॒वी जोष्ट्री॑ दे॒वी ऊ॒र्जाहु॑ती दे॒वा दे॒वानां भि॒षजा॑ वषट्का॒रैर्दे॒वीस्ति॒स्रस्ति॒स्रो दे॒वीर्दे॒व इन्द्रो॒ नरा॒शसो॑ दे॒व इन्द्रो॒ वन॒स्पति॑र्दे॒वं ब॒र्॒हिर्वारि॑तीनान्दे॒वो अ॒ग्निः स्वि॑ष्ट॒कृद्दे॒वान्। स॒मिधा॒ऽग्निन्दे॒वं ब॒र्॒हिः सर॑स्वत्य॒श्विना॒ सर्व॑ वियन्तु। द्वार॑स्ति॒स्रः सर्व॑वियन्तु। अ॒ज इन्द्र॒मोजो॒ऽग्निं पर॒ सर॑स्वतीम्। नक्तं॒ पूर्व॒ सर॑स्वति। अ॒न्यत्र॒ सर॑स्वती। भि॒षक्पूर्व॑न्दुह इन्द्रि॒यम्। अ॒न्यत्र॑ दधुरिन्द्रि॒यम्। सौ॒त्रा॒म॒ण्या सु॑तासु॒ती। अ॒ञ्जन्त्य॒यं यज॑मानः ॥ )]

%2.6.15.1
अ॒ग्निम॒द्य होता॑रमवृणीत। अ॒य सु॑तासु॒ती यज॑मानः। पच॑न्प॒क्तीः। पच॑न्पुरो॒डाशान्॑। गृ॒ह्णन्ग्रहान्॑। ब॒ध्नन्न॒श्विभ्या॒ञ्छाग॒ सर॑स्वत्या॒ इन्द्रा॑य। ब॒ध्नन्त्सर॑स्वत्यै मे॒षमिन्द्रा॑या॒श्विभ्याम्। ब॒ध्नन्निन्द्रा॑यर्‌ष॒भम॒श्विभ्या॒ सर॑स्वत्यै। सू॒प॒स्था अ॒द्य दे॒वो वन॒स्पति॑रभवत्। अ॒श्विभ्या॒ञ्छागे॑न॒ सर॑स्वत्या॒ इन्द्रा॑य॥७३॥

%2.6.15.2
सर॑स्वत्यै मे॒षेणेन्द्रा॑या॒श्विभ्याम्। इन्द्रा॑यर्‌ष॒भेणा॒श्विभ्या॒ सर॑स्वत्यै। अक्ष॒ स्तान्मे॑द॒स्तः प्रति॑पच॒ताग्र॑भीषुः। अवी॑वृधन्त॒ ग्रहै। अपा॑ताम॒श्विना॒ सर॑स्व॒तीन्द्र॑ सु॒त्रामा॑ वृत्र॒हा। सोमान्त्सु॒राम्ण॑। उपो॑ उक्थाम॒दाः श्रौ॒द्विमदा॑ अदन्। अवी॑वृधन्ताङ्गू॒षैः। त्वाम॒द्यर्‌ष॑ आर्‌षेयर्‌षीणान्नपादवृणीत। अ॒य सु॑तासु॒ती यज॑मानः। ब॒हुभ्य॒ आ सङ्ग॑तेभ्यः। ए॒ष मे॑ दे॒वेषु॒ वसु॒ वार्या य॑क्ष्यत॒ इति॑। ता या दे॒वा दे॑व॒दाना॒न्यदु॑। तान्य॑स्मा॒ आ च॒ शास्व॑। आ च॑ गुरस्व। इ॒षि॒तश्च॑ होत॒रसि॑ भद्र॒वाच्या॑य॒ प्रेषि॑तो॒ मानु॑षः। सू॒क्त॒वा॒काय॑ सू॒क्ता ब्रू॑हि॥७४॥\anuvakamend[इन्द्रा॑य॒ यज॑मानः स॒प्त च॑]

%2.6.16.1
उ॒शन्त॑स्त्वा हवामह॒ आ नो॑ अग्ने सुके॒तुना। त्व सो॑म म॒हे भग॒न्त्व सो॑म॒ प्रचि॑कितो मनी॒षा। त्वया॒ हि न॑ पि॒तर॑ सोम॒ पूर्वे॒ त्व सो॑म पि॒तृभि॑ संविदा॒नः। बर्‌हि॑षदः पितर॒ आऽहं पि॒तॄन्। उप॑हूताः पि॒तरोऽग्नि॑ष्वात्ताः पितरः। अ॒ग्निष्वा॒त्तानृ॑तु॒मतो॑ हवामहे। नरा॒शसे॑ सोमपी॒थं य आ॒शुः। ते नो॒ अर्व॑न्तः सु॒हवा॑ भवन्तु। शन्नो॑ भवन्तु द्वि॒पदे॒ शञ्चतु॑ष्पदे। ये अ॑ग्निष्वा॒त्ता येऽन॑ग्निष्वात्ताः॥७५॥

%2.6.16.2
अ॒हो॒मुच॑ पि॒तर॑ सो॒म्यास॑। परेऽव॑रे॒ऽमृता॑सो॒ भव॑न्तः। अधि॑ ब्रुवन्तु॒ ते अ॑वन्त्व॒स्मान्। वा॒न्या॑यै दु॒ग्धे जु॒षमा॑णाः कर॒म्भम्। उ॒दीरा॑णा॒ अव॑रे॒ परे॑ च। अ॒ग्नि॒ष्वा॒त्ता ऋ॒तुभि॑ संविदा॒नाः। इन्द्र॑वन्तो ह॒विरि॒दं जु॑षन्ताम्। यद॑ग्ने कव्यवाहन॒ त्वम॑ग्न ईडि॒तो जा॑तवेदः। मात॑ली क॒व्यैः। ये ता॑तृ॒पुर्दे॑व॒त्रा जेह॑मानाः। हो॒त्रा॒वृध॒ स्तोम॑तष्टासो अ॒र्कैः। आऽग्ने॑ याहि सुवि॒दत्रे॑भिर॒र्वाङ्। स॒त्यैः क॒व्यैः पि॒तृभि॑र्घर्म॒सद्भि॑। ह॒व्य॒वाह॑म॒जरं॑ पुरुप्रि॒यम्। अ॒ग्निङ्घृ॒तेन॑ ह॒विषा॑ सप॒र्यन्। उपा॑सदङ्कव्य॒वाहं॑ पितृ॒णाम्। स न॑ प्र॒जां वी॒रव॑ती॒ समृ॑ण्वतु॥७६॥\anuvakamend[अन॑ग्निष्वात्ता॒ जेह॑मानाः स॒प्त च॑]

%2.6.17.1
होता॑ यक्षदि॒डस्प॒दे। स॒मि॒धा॒नं म॒हद्यश॑। सुष॑मिद्धं॒ वरेण्यम्। अ॒ग्निमिन्द्रं॑ वयो॒धसम्। गा॒य॒त्रीञ्छन्द॑ इन्द्रि॒यम्। त्र्यवि॒ङ्गां वयो॒ दध॑त्। वेत्वाज्य॑स्य॒ होत॒र्यज॑। होता॑ यक्ष॒च्छुचि॑व्रतम्। तनू॒नपा॑तमु॒द्भिदम्। यङ्गर्भ॒मदि॑तिर्द॒धे॥७७॥

%2.6.17.2
शुचि॒मिन्द्रं॑ वयो॒धसम्। उ॒ष्णिह॒ञ्छन्द॑ इन्द्रि॒यम्। दि॒त्य॒वाह॒ङ्गां वयो॒ दध॑त्। वेत्वाज्य॑स्य॒ होत॒र्यज॑। होता॑ यक्षदी॒डेन्यम्। ई॒डि॒तं वृ॑त्र॒हन्त॑मम्। इडा॑भि॒रीड्य॒ सह॑। सोम॒मिन्द्रं॑ वयो॒धसम्। अ॒नु॒ष्टुभ॒ञ्छन्द॑ इन्द्रि॒यम्। त्रि॒व॒त्सङ्गां वयो॒ दध॑त्॥७८॥

%2.6.17.3
वेत्वाज्य॑स्य॒ होत॒र्यज॑। होता॑ यक्षत्सुबर्\mbox{}हि॒षदम्। पू॒ष॒ण्वन्त॒मम॑र्त्यम्। सीद॑न्तं ब॒र्॒हिषि॑ प्रि॒ये। अ॒मृतेन्द्रं॑ वयो॒धसम्। बृ॒ह॒तीञ्छन्द॑ इन्द्रि॒यम्। पञ्चा॑वि॒ङ्गां वयो॒ दध॑त्। वेत्वाज्य॑स्य॒ होत॒र्यज॑। होता॑यक्ष॒द्व्यच॑स्वतीः। सु॒प्रा॒य॒णा ऋ॑ता॒वृध॑॥७९॥

%2.6.17.4
द्वारो॑ दे॒वीर्‌हि॑र॒ण्ययी। ब्र॒ह्माण॒ इन्द्रं॑ वयो॒धसम्। प॒ङ्क्तिञ्छन्द॑ इ॒हेन्द्रि॒यम्। तु॒र्य॒वाह॒ङ्गां वयो॒ दध॑त्। वेत्वाज्य॑स्य॒ होत॒र्यज॑। होता॑ यक्षत्सु॒पेश॑से। सु॒शि॒ल्पे बृ॑ह॒ती उ॒भे। नक्तो॒षासा॒ न द॑र्‌श॒ते। विश्व॒मिन्द्रं॑ वयो॒धसम्। त्रि॒ष्टुभ॒ञ्छन्द॑ इन्द्रि॒यम्॥८०॥

%2.6.17.5
प॒ष्ठ॒वाह॒ङ्गां वयो॒ दध॑त्। वेत्वाज्य॑स्य॒ होत॒र्यज॑। होता॑ यक्ष॒त्प्रचे॑तसा। दे॒वाना॑मुत्त॒मं यश॑। होता॑रा॒ दैव्या॑ क॒वी। स॒युजेन्द्रं॑ वयो॒धसम्। जग॑ती॒ञ्छन्द॑ इ॒हेन्द्रि॒यम्। अ॒न॒ड्वाह॒ङ्गां वयो॒ दध॑त्। वेत्वाज्य॑स्य॒ होत॒र्यज॑। होता॑ यक्ष॒त्पेश॑स्वतीः॥८१॥

%2.6.17.6
ति॒स्रो दे॒वीर्‌हि॑र॒ण्ययी। भार॑तीर्बृह॒तीर्म॒हीः। पति॒मिन्द्रं॑ वयो॒धसम्। वि॒राज॒ञ्छन्द॑ इ॒हेन्द्रि॒यम्। धे॒नुङ्गान्न वयो॒ दध॑त्। वेत्वाज्य॑स्य॒ होत॒र्यज॑। होता॑ यक्षत्सु॒रेत॑सम्। त्वष्टा॑रं पुष्टि॒वर्ध॑नम्। रू॒पाणि॒ बिभ्र॑तं॒ पृथ॑क्। पुष्टि॒मिन्द्रं॑ वयो॒धसम्॥८२॥

%2.6.17.7
द्वि॒पद॒ञ्छन्द॑ इ॒हेन्द्रि॒यम्। उ॒क्षाण॒ङ्गान्न वयो॒ दध॑त्। वेत्वाज्य॑स्य॒ होत॒र्यज॑। होता॑ यक्षच्छ॒तक्र॑तुम्। हिर॑ण्यपर्णमु॒क्थिनम्। र॒श॒नां बिभ्र॑तं व॒शिम्। भग॒मिन्द्रं॑ वयो॒धसम्। क॒कुभ॒ञ्छन्द॑ इ॒हेन्द्रि॒यम्। व॒शां वे॒हत॒ङ्गान्न वयो॒ दध॑त्। वेत्वाज्य॑स्य॒ होत॒र्यज॑। होता॑ यक्ष॒त्स्वाहा॑कृतीः। अ॒ग्निङ्गृ॒हप॑तिं॒ पृथ॑क्। वरु॑णं भेष॒जङ्क॒विम्। क्ष॒त्रमिन्द्रं॑ वयो॒धसम्। अति॑च्छन्दस॒ञ्छन्द॑ इन्द्रि॒यम्। बृ॒हदृ॑ष॒भङ्गां वयो॒ दध॑त्। वेत्वाज्य॑स्य॒ होत॒र्यज॑॥८३॥\anuvakamend[द॒धे दध॑दृता॒वृध॑ इन्द्रि॒यं पेश॑स्वतीर्वयो॒धसं॒ वेत्वाज्य॑स्य॒ होत॒र्यज॑ स॒प्त च॑ (इ॒डस्प॒देऽग्निङ्गा॑य॒त्रीन्त्र्यविम्। शुचि॑व्रत॒ शुचि॑मु॒ष्णिह॑न्दित्य॒वाहम्। ई॒डेन्य॒ सोम॑मनु॒ष्टुभ॑न्त्रिव॒त्सम्। सु॒ब॒र्॒हि॒षद॑म॒मृतेन्द्रं॑ बृह॒तीं पञ्चा॑विम्। व्यच॑स्वतीः सुप्राय॒णा द्वारो ब्र॒ह्माण॑ प॒ङ्क्तिमि॒ह तु॑र्य॒वाहम्। सु॒पेश॑से॒ विश्व॒मिन्द्र॑न्त्रि॒ष्टुभं॑ पष्ठ॒वाहम्। प्रचे॑तसा स॒युजेन्द्रं॒ जग॑तीमि॒हान॒ड्वाहम्। पेश॑स्वतीस्ति॒स्रः पतिँ॑व्वि॒राज॑मि॒ह धे॒नुन्न। सु॒रेत॑स॒न्त्वष्टा॑रं॒ पुष्टि॒मिन्द्रं॑ द्वि॒पद॑मि॒होक्षाण॒न्न। श॒तक्र॑तुं॒ भग॒मिन्द्र॑ङ्क॒कुभ॑मि॒ह व॒शान्न। स्वाहा॑कृतीः क्ष॒त्रमति॑च्छन्दसं बृ॒हदृ॑ष॒भङ्गां वयो॒ दध॑दिन्द्रि॒यमृषि॑ वसु॒ नव॑ द॒शेहेन्द्रि॒यमष्ट॑ नव दश॒ गान्न वयो॒ दध॑दि॒डस्प॒दे सर्व॑ वेतु ॥ )]

%2.6.18.1
समि॑द्धो अ॒ग्निः स॒मिधा। सुष॑मिद्धो॒ वरेण्यः। गा॒य॒त्री छन्द॑ इन्द्रि॒यम्। त्र्यवि॒र्गौर्वयो॑ दधुः। तनू॒नपा॒च्छुचि॑व्रतः। त॒नू॒पाच्च॒ सर॑स्वती। उ॒ष्णिक्छन्द॑ इन्द्रि॒यम्। दि॒त्य॒वाड्गौर्वयो॑ दधुः। इडा॑भिर॒ग्निरीड्य॑। सोमो॑ दे॒वो अम॑र्त्यः॥८४॥

%2.6.18.2
अ॒नु॒ष्टुप्छन्द॑ इन्द्रि॒यम्। त्रि॒व॒त्सो गौर्वयो॑ दधुः। सु॒ब॒र्॒हिर॒ग्निः पू॑ष॒ण्वान्। स्ती॒र्णब॑र्‌हि॒रम॑र्त्यः। बृ॒ह॒ती छन्द॑ इन्द्रि॒यम्। पञ्चा॑वि॒र्गौर्वयो॑ दधुः। दुरो॑ दे॒वीर्दिशो॑ म॒हीः। ब्र॒ह्मा दे॒वो बृह॒स्पति॑। प॒ङ्क्तिश्छन्द॑ इ॒हेन्द्रि॒यम्। तु॒र्य॒वाड्गौर्वयो॑ दधुः॥८५॥

%2.6.18.3
उ॒षे य॒ह्वी सु॒पेश॑सा। विश्वे॑ दे॒वा अम॑र्त्याः। त्रि॒ष्टुप्छन्द॑ इन्द्रि॒यम्। प॒ष्ठ॒वाद्गौर्वयो॑ दधुः। दैव्या॑ होतारा भिषजा। इन्द्रे॑ण स॒युजा॑ यु॒जा। जग॑ती॒ छन्द॑ इ॒हेन्द्रि॒यम्। अ॒न॒ड्वान्गौर्वयो॑ दधुः। ति॒स्र इडा॒ सर॑स्वती। भार॑ती म॒रुतो॒ विश॑॥८६॥

%2.6.18.4
वि॒राट्छन्द॑ इ॒हेन्द्रि॒यम्। धे॒नुर्गौर्न वयो॑ दधुः। त्वष्टा॑ तु॒रीपो॒ अद्भु॑तः। इ॒न्द्रा॒ग्नी पु॑ष्टि॒वर्ध॑ना। द्वि॒पाच्छन्द॑ इ॒हेन्द्रि॒यम्। उ॒क्षा गौर्न वयो॑ दधुः। श॒मि॒ता नो॒ वन॒स्पति॑। स॒वि॒ता प्र॑सु॒वन्भगम्। क॒कुच्छन्द॑ इ॒हेन्द्रि॒यम्। व॒शा वे॒हद्गौर्न वयो॑ दधुः। स्वाहा॑ य॒ज्ञं वरु॑णः। सु॒क्ष॒त्रो भे॑ष॒जङ्क॑रत्। अति॑च्छन्दा॒श्छन्द॑ इन्द्रि॒यम्। बृ॒हदृ॑ष॒भो गौर्वयो॑ दधुः॥८७॥\anuvakamend[अम॑र्त्यस्तुर्य॒वाड्गौर्वयो॑ दधु॒र्विशो॑ व॒शा वे॒हद्गौर्न वयो॑ दधुश्च॒त्वारि॑ च]

%2.6.19.1
व॒स॒न्तेन॒र्तुना॑ दे॒वाः। वस॑वस्त्रि॒वृता स्तु॒तम्। र॒थ॒न्त॒रेण॒ तेज॑सा। ह॒विरिन्द्रे॒ वयो॑ दधुः। ग्री॒ष्मेण॑ दे॒वा ऋ॒तुना। रु॒द्राः प॑ञ्चद॒शे स्तु॒तम्। बृ॒ह॒ता यश॑सा॒ बलम्। ह॒विरिन्द्रे॒ वयो॑ दधुः। व॒र्॒षाभि॑र्\mbox{}ऋ॒तुना॑ऽऽदि॒त्याः। स्तोमे॑ सप्तद॒शे स्तु॒तम्॥८८॥

%2.6.19.2
वै॒रू॒पेण॑ वि॒शौज॑सा। ह॒विरिन्द्रे॒ वयो॑ दधुः। शा॒र॒देन॒र्तुना॑ दे॒वाः। ए॒क॒वि॒श ऋ॒भव॑ स्तु॒तम्। वै॒रा॒जेन॑ श्रि॒या श्रियम्। ह॒विरिन्द्रे॒ वयो॑ दधुः। हे॒म॒न्तेन॒र्तुना॑ दे॒वाः। म॒रुत॑स्त्रिण॒वे स्तु॒तम्। बले॑न॒ शक्व॑री॒ सह॑। ह॒विरिन्द्रे॒ वयो॑ दधुः। शै॒शि॒रेण॒र्तुना॑ दे॒वाः। त्र॒य॒स्त्रि॒शे॑ऽमृत स्तु॒तम्। स॒त्येन॑ रे॒वती क्ष॒त्रम्। ह॒विरिन्द्रे॒ वयो॑ दधुः॥८९॥\anuvakamend[स्तोमे॑ सप्तद॒शे स्तु॒त सहो॑ ह॒विरिन्द्रे॒ वयो॑ दधुश्च॒त्वारि॑ च (व॒स॒न्तेन॑ ग्री॒ष्मेण॑ व॒र्‌षाभि॑ शार॒देन॑ हेम॒न्तेन॑ शैशि॒रेण॒ षट् ॥ )]

%2.6.20.1
दे॒वं ब॒र्॒हिरिन्द्रं॑ वयो॒धसम्। दे॒वन्दे॒वम॑वर्धयत्। गा॒य॒त्रि॒या छन्द॑सेन्द्रि॒यम्। तेज॒ इन्द्रे॒ वयो॒ दध॑त्। व॒सु॒वने॑ वसु॒धेय॑स्य वेतु॒ यज॑। दे॒वीर्द्वारो॑ दे॒वमिन्द्रं॑ वयो॒धसम्। दे॒वीर्दे॒वम॑वर्धयन्। उ॒ष्णिहा॒ छन्द॑सेन्द्रि॒यम्। प्रा॒णमिन्द्रे॒ वयो॒ दध॑त्। व॒सु॒वने॑ वसु॒धेय॑स्य वियन्तु॒ यज॑॥९०॥

%2.6.20.2
दे॒वी दे॒वं व॑यो॒धसम्। उ॒षे इन्द्र॑मवर्धताम्। अ॒नु॒ष्टुभा॒ छन्द॑सेन्द्रि॒यम्। वाच॒मिन्द्रे॒ वयो॒ दध॑त्। व॒सु॒वने॑ वसु॒धेय॑स्य वीतां॒ यज॑। दे॒वी जोष्ट्री॑ दे॒वमिन्द्रं॑ वयो॒धसम्। दे॒वी दे॒वम॑वर्धताम्। बृ॒ह॒त्या छन्द॑सेन्द्रि॒यम्। श्रोत्र॒मिन्द्रे॒ वयो॒ दध॑त्। व॒सु॒वने॑ वसु॒धेय॑स्य वीतां॒ यज॑॥९१॥

%2.6.20.3
दे॒वी ऊ॒र्जाहु॑ती दे॒वमिन्द्रं॑ वयो॒धसम्। दे॒वी दे॒वम॑वर्धताम्। प॒ङ्क्त्या छन्द॑सेन्द्रि॒यम्। शु॒क्रमिन्द्रे॒ वयो॒ दध॑त्। व॒सु॒वने॑ वसु॒धेय॑स्य वीतां॒ यज॑। दे॒वा दैव्या॒ होता॑रा दे॒वमिन्द्रं॑ वयो॒धसम्। दे॒वा दे॒वम॑वर्धताम्। त्रि॒ष्टुभा॒ छन्द॑सेन्द्रि॒यम्। त्विषि॒मिन्द्रे॒ वयो॒ दध॑त्। व॒सु॒वने॑ वसु॒धेय॑स्य वीतां॒ यज॑॥९२॥

%2.6.20.4
दे॒वीस्ति॒स्रस्ति॒स्रो दे॒वीर्व॑यो॒धसम्। पति॒मिन्द्र॑मवर्धयन्। जग॑त्या॒ छन्द॑सेन्द्रि॒यम्। बल॒मिन्द्रे॒ वयो॒ दध॑त्। व॒सु॒व॒ने॑ वसु॒धेय॑स्य वियन्तु॒ यज॑। दे॒वो नरा॒शसो॑ दे॒वमिन्द्रं॑ वयो॒धसम्। दे॒वो दे॒वम॑वर्धयत्। वि॒राजा॒ छन्द॑सेन्द्रि॒यम्। रेत॒ इन्द्रे॒ वयो॒ दध॑त्। व॒सु॒वने॑ वसु॒धेय॑स्य वेतु॒ यज॑॥९३॥

%2.6.20.5
दे॒वो वन॒स्पति॑र्दे॒वमिन्द्रं॑ वयो॒धसम्। दे॒वो दे॒वम॑वर्धयत्। द्वि॒पदा॒ छन्द॑सेन्द्रि॒यम्। भग॒मिन्द्रे॒ वयो॒ दध॑त्। व॒सु॒वने॑ वसु॒धेय॑स्य वेतु॒ यज॑। दे॒वं ब॒र्॒हिर्वारि॑तीनान्दे॒वमिन्द्रं॑ वयो॒धसम्। दे॒वन्दे॒वम॑वर्धयत्। क॒कुभा॒ छन्द॑सेन्द्रि॒यम्। यश॒ इन्द्रे॒ वयो॒ दध॑त्। व॒सु॒वने॑ वसु॒धेय॑स्य वेतु॒ यज॑। दे॒वो अ॒ग्निः स्वि॑ष्ट॒कृद्दे॒वमिन्द्रं॑ वयो॒धसम्। दे॒वो दे॒वम॑वर्धयत्। अति॑च्छन्दसा॒ छन्द॑सेन्द्रि॒यम्। क्ष॒त्रमिन्द्रे॒ वयो॒ दध॑त्। व॒सु॒वने॑ वसु॒धेय॑स्य वेतु॒ यज॑॥९४॥\anuvakamend[वि॒य॒न्तु॒ यज॑ वीतां॒ यज॑ वीतां॒ यज॑ वेतु॒ यज॑ वेतु॒ यज॒ पञ्च॑ च (दे॒वं ब॒र्॒हिर्गा॑यत्रि॒या तेज॑। दे॒वीर्द्वार॑ उ॒ष्णिहा प्रा॒णम्। दे॒वी दे॒वमु॒षे अ॑नु॒ष्टुभा॒ वाचम्। दे॒वी जोष्ट्री॑ बृह॒त्या श्रोत्रम्। दे॒वी ऊ॒र्जाहु॑ती प॒ङ्क्त्या शु॒क्रम्। दे॒वा दैव्या॒ होता॑रा त्रि॒ष्टुभा॒ त्विषिम्। दे॒वीस्ति॒स्रस्ति॒स्रो दे॒वीः पतिं॒ जग॑त्या॒ बलम्। दे॒वो नरा॒शसो॑ वि॒राजा॒ रेत॑। दे॒वो वन॒स्पति॑र्द्वि॒पदा॒ भगम्। दे॒वं ब॒र्॒हिर्वारि॑तीनाङ्क॒कुभा॒ यश॑। दे॒वो अ॒ग्निः स्वि॑ष्ट॒कृदति॑च्छन्दसा क्ष॒त्रम्। वे॒तु॒ वि॒य॒न्तु॒ च॒तुर्वी॑ता॒मेको॑ वियन्तु च॒तुर्वेत्ववर्धयदवर्धयश्च॒तुर॑वर्धता॒मेको॑ऽवर्धय श्च॒तुर॑वर्धयत् ॥ )]




\prashnaend{स्वा॒द्वीन्त्वा॒ सोम॒ सुरा॑वन्त सीसे॑न मि॒त्रो॑ऽसि॒ यद्दे॑वा॒ होता॑ यक्षत्स॒मिधेन्द्र॒ समि॑द्ध॒ इन्द्र॒ आच॑र्‌षणि॒प्रा दे॒वं ब॒र्॒हिर्‌होता॑ यक्षत्स॒मिधा॒ऽग्नि समि॑द्धो अ॒ग्निर॑श्विना॒ऽश्विना॑ ह॒विरि॑न्द्रि॒यन्दे॒वं ब॒र्॒हिः सर॑स्वत्य॒ग्निम॒द्योशन्तो॒ होता॑ यक्षदि॒डस्प॒दे समि॑द्धो अ॒ग्निः स॒मिधा॑ वस॒न्तेन॒र्तुना॑ दे॒वं ब॒र्॒हिरिन्द्रं॑ वयो॒धसं॑ विश॒तिः॥२०॥}{स्वा॒द्वीन्त्वाऽमी॑मदन्त पि॒तर॒ साम्राज्याय पू॒तं प॒वित्रे॑णो॒षासा॒नक्ता॒ बद॑रै॒रधा॑तान्दे॒व इन्द्रो॒ वन॒स्पति॑ पष्ठ॒वाह॒ङ्गान्दे॒वी दे॒वं व॑यो॒धसं॒ चतु॑र्नवतिः॥९४॥}{स्वा॒द्वीन्त्वा॑ वेतु॒ यज॑॥}{हरि॑ ओम्॥}{इति श्रीकृष्णयजुर्वेदीयतैत्तिरीयब्राह्मणे द्वितीयाष्टके षष्ठः प्रपाठकः समाप्तः॥}
\clearpage
