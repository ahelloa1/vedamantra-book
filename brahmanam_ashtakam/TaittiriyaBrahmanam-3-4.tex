\sect{चतुर्थः प्रश्नः}
\setcounter{anuvakam}{0}
\dnsub{तैत्तिरीयब्राह्मणे तृतीयाष्टके चतुर्थः प्रपाठकः}

%3.4.1.1
ब्रह्म॑णे ब्राह्म॒णमाल॑भते। क्ष॒त्राय॑ राज॒न्यम्। म॒रुद्भ्यो॒ वैश्यम्। तप॑से शू॒द्रम्। तम॑से॒ तस्क॑रम्। नार॑काय वीर॒हणम्। पा॒प्मने क्ली॒बम्। आ॒क्र॒याया॑यो॒गूम्। कामा॑य पुश्च॒लूम्। अति॑क्रुष्टाय माग॒धम्॥१॥

%3.4.2.1
गी॒ताय॑ सू॒तम्। नृ॒त्ताय॑ शैलू॒षम्। धर्मा॑य सभाच॒रम्। न॒र्माय॑ रे॒भम्। नरि॑ष्ठायै भीम॒लम्। हसा॑य॒ कारिम्। आ॒न॒न्दाय॑ स्त्रीष॒खम्। प्रमुदे॑ कुमारीपु॒त्रम्। मे॒धायै॑ रथका॒रम्। धैर्या॑य॒ तक्षा॑णम्॥२॥

%3.4.3.1
श्रमा॑य कौला॒लम्। मा॒यायै॑ कार्मा॒रम्। रू॒पाय॑ मणिका॒रम्। शुभे॑ व॒पम्। श॒र॒व्या॑या इषुका॒रम्। हे॒त्यै ध॑न्वका॒रम्। कर्म॑णे ज्याका॒रम्। दि॒ष्टाय॑ रज्जुस॒र्गम्। मृ॒त्य॑वे मृग॒युम्। अन्त॑काय श्व॒नितम्॥३॥

%3.4.4.1
स॒न्धये॑ जा॒रम्। गे॒हायो॑पप॒तिम्। निर्\mbox{}ऋ॑त्यै परिवि॒त्तम्। आर्त्यै॑ परिविविदा॒नम्। अराध्यै दिधिषू॒पतिम्। प॒वित्रा॑य भि॒षजम्। प्र॒ज्ञाना॑य नक्षत्रद॒र्॒शम्। निष्कृ॑त्यै पेशस्का॒रीम्। बला॑योप॒दाम्। वर्णा॑यानू॒रुधम्॥४॥

%3.4.5.1
न॒दीभ्य॑ पौञ्जि॒ष्टम्। ऋ॒क्षीकाभ्यो॒ नैषा॑दम्। पु॒रु॒ष॒व्या॒घ्राय॑ दु॒र्मदम्। प्र॒युद्भ्य॒ उन्म॑त्तम्। ग॒न्ध॒र्वा॒प्स॒राभ्यो॒ व्रात्यम्। स॒र्प॒दे॒व॒ज॒नेभ्योऽप्र॑तिपदम्। अवेभ्यः कित॒वम्। इ॒र्यता॑या॒ अकि॑तवम्। पि॒शा॒चेभ्यो॑ बिदलका॒रम्। या॒तु॒धानेभ्यः कण्टकका॒रम्॥५॥

%3.4.6.1
उ॒त्सा॒देभ्य॑ कु॒ब्जम्। प्र॒मुदे॑ वाम॒नम्। द्वा॒र्भ्यः स्रा॒मम्। स्वप्ना॑या॒न्धम्। अध॑र्माय बधि॒रम्। सं॒ज्ञाना॑य स्मरका॒रीम्। प्र॒का॒मोद्या॑योप॒सदम्। आ॒शि॒क्षायै प्र॒श्ञिनम्। उ॒प॒शि॑क्षाया॑ अभिप्र॒श्ञिनम्। म॒र्यादा॑यै प्रश्ञविवा॒कम्॥६॥

%3.4.7.1
ऋत्यै स्ते॒नहृ॑दयम्। वैर॑हत्याय॒ पिशु॑नम्। विवि॑त्त्यै क्ष॒त्तारम्। औप॑द्रष्टाय सङ्ग्रही॒तारम्। बला॑यानुच॒रम्। भू॒म्ने प॑रिष्क॒न्दम्। प्रि॒याय॑ प्रियवा॒दिनम्। अरि॑ष्ट्या अश्वसा॒दम्। मेधा॑य वासः पल्पू॒लीम्। प्र॒का॒माय॑ रजयि॒त्रीम्॥७॥

%3.4.8.1
भायै॑ दार्वाहा॒रम्। प्र॒भाया॑ आग्ने॒न्धम्। नाक॑स्य पृ॒ष्ठाया॑भिषे॒क्तारम्। ब्र॒ध्नस्य॑ वि॒ष्टपा॑य पात्रनिर्णे॒गम्। दे॒व॒लो॒काय॑ पेशि॒तारम्। म॒नु॒ष्य॒लो॒काय॑ प्रकरि॒तारम्। सर्वेभ्यो लो॒केभ्य॑ उपसे॒क्तारम्। अव॑र्त्यै व॒धायो॑पमन्थि॒तारम्। सु॒व॒र्गाय॑ लो॒काय॑ भाग॒दुघम्। वर्\mbox{}षि॑ष्ठाय॒ नाका॑य परिवे॒ष्टारम्॥८॥

%3.4.9.1
अर्मेभ्यो हस्ति॒पम्। ज॒वायाश्व॒पम्। पुष्ट्यै॑ गोपा॒लम्। तेज॑सेऽजपा॒लम्। वी॒र्या॑याविपा॒लम्। इरा॑यै की॒नाशम्। की॒लाला॑य सुराका॒रम्। भ॒द्राय॑ गृह॒पम्। श्रेय॑से वित्त॒धम्। अध्य॑क्षायानुक्ष॒त्तारम्॥९॥

%3.4.10.1
म॒न्यवे॑ऽयस्ता॒पम्। क्रोधा॑य निस॒रम्। शोका॑याभिस॒रम्। उ॒त्कू॒ल॒वि॒कू॒लाभ्यान्त्रि॒स्थिनम्। योगा॑य यो॒क्तारम्। क्षेमा॑य विमो॒क्तारम्। वपु॑षे मानस्कृ॒तम्। शीला॑याञ्जनीका॒रम्। निर्\mbox{}ऋ॑त्यै कोशका॒रीम्। य॒माया॒सूम्॥१०॥

%3.4.11.1
य॒म्यै॑ यम॒सूम्। अथ॑र्व॒भ्योऽव॑तोकाम्। सं॒व॒त्स॒राय॑ पर्या॒रिणीम्। प॒रि॒व॒त्स॒रायावि॑जाताम्। इ॒दा॒व॒त्स॒राया॑प॒स्कद्व॑रीम्। इ॒द्व॒त्स॒राया॒तीत्व॑वरीम्। व॒त्स॒राय॒ विज॑र्जराम्। स॒र्व॒न्त्स॒राय॒ पलि॑क्नीम्। वना॑य वन॒पम्। अ॒न्यतो॑रण्याय दाव॒पम्॥११॥

%3.4.12.1
सरोभ्यो धैव॒रम्। वेश॑न्ताभ्यो॒ दाशम्। उ॒प॒स्थाव॑रीभ्यो॒ बैन्दम्। न॒ड्व॒लाभ्य॑ शौष्क॒लम्। पा॒र्या॑य कैव॒र्तम्। अ॒वा॒र्या॑य मार्गा॒रम्। ती॒र्थेभ्य॑ आ॒न्दम्। विष॑मेभ्यो मैना॒लम्। स्वनेभ्य॒ पर्ण॑कम्। गुहाभ्य॒ किरा॑तम्। सानु॑भ्यो॒ जम्भ॑कम्। पर्व॑तेभ्य॒ किंपू॑रुषम्॥१२॥

%3.4.13.1
प्र॒ति॒श्रुत्का॑या ऋतु॒लम्। घोषा॑य भ॒षम्। अन्ता॑य बहुवा॒दिनम्। अ॒न॒न्ताय॒ मूकम्। मह॑से वीणावा॒दम्। क्रोशा॑य तूणव॒ध्मम्। आ॒क्र॒न्दाय॑ दुन्दुभ्याघा॒तम्। अ॒व॒र॒स्प॒राय॑ शङ्ख॒ध्मम्। ऋ॒भुभ्यो॑जिनसन्धा॒यम्। सा॒ध्येभ्य॑श्चर्म॒म्णम्।॥१३॥

%3.4.14.1
बी॒भ॒त्सायै॑ पौल्क॒सम्। भूत्यै॑ जागर॒णम्। अभूत्यै स्वप॒नम्। तु॒लायै॑ वाणि॒जम्। वर्णा॑य हिरण्यका॒रम्। विश्वेभ्यो दे॒वेभ्य॑ सिध्म॒लम्। प॒श्चा॒द्दो॒षाय॑ ग्ला॒वम्। ऋत्यै॑ जनवा॒दिनम्। व्यृ॑द्ध्या अपग॒ल्भम्। स॒श॒राय॑ प्र॒च्छिदम्॥१४॥

%3.4.15.1
हसा॑य पुश्च॒लूमा ल॑भते। वी॒णा॒वा॒दङ्गण॑कङ्गी॒ताय॑। याद॑से शाबु॒ल्याम्। न॒र्माय॑ भद्रव॒तीम्। तू॒ण॒व॒ध्मङ्ग्रा॑म॒ण्यं॑ पाणिसङ्घा॒तन्नृ॒त्ताय॑। मोदा॑यानु॒क्रोश॑कम्। आ॒न॒न्दाय॑ तल॒वम्॥१५॥

%3.4.16.1
अ॒क्ष॒रा॒जाय॑ कित॒वम्। कृ॒ताय॑ सभा॒विनम्। त्रेता॑या आदिनवद॒र्॒शम्। द्वा॒प॒राय॑ बहि॒ सदम्। कल॑ये सभास्था॒णुम्। दु॒ष्कृ॒ताय॑ च॒रका॑चार्यम्। अध्व॑ने ब्रह्मचा॒रिणम्। पि॒शा॒चेभ्य॑ सैल॒गम्। पि॒पा॒सायै॑ गोव्य॒च्छम्। निर्\mbox{}ऋ॑त्यै गोघा॒तम्। क्षु॒धे गो॑विक॒र्तम्। क्षु॒त्तृ॒ष्णाभ्या॒न्तम्। यो गां वि॒कृन्त॑न्तं मा॒सं भिक्ष॑माण उप॒तिष्ठ॑ते॥१६॥

%3.4.17.1
भूम्यै॑ पीठस॒र्पिण॒मा ल॑भते। अ॒ग्नयेऽस॒लम्। वा॒यवे॑ चाण्डा॒लम्। अ॒न्तरि॑क्षाय वशन॒र्तिनम्। दि॒वे ख॑ल॒तिम्। सूर्या॑य हर्य॒क्षम्। च॒न्द्रम॑से मिर्मि॒रम्। नक्ष॑त्रेभ्यः कि॒लासम्। अह्ने॑ शु॒क्लं पि॑ङ्ग॒लम्। रात्रि॑यै कृ॒ष्णं पि॑ङ्गा॒क्षम्॥१७॥

%3.4.18.1
वा॒चे पुरु॑ष॒मा ल॑भते। प्रा॒णम॑पा॒नव्व्याँ॒नमु॑दा॒न स॑मा॒नन्तान् वा॒यवे। सूर्या॑य॒ चक्षु॒रा ल॑भते। मन॑श्च॒न्द्रम॑से। दि॒ग्भ्यः श्रोत्रम्। प्र॒जाप॑तये॒ पुरु॑षम्॥१८॥

%3.4.19.1
अथै॒तानरू॑पेभ्य॒ आल॑भते। अति॑ह्रस्व॒मति॑दीर्घम्। अतिकृ॑श॒मत्यसलम्। अति॑शुक्ल॒मति॑कृष्णम्। अति॑श्लक्ष्ण॒मति॑लोमशम्। अति॑किरिट॒मति॑दन्तुरम्। अति॑मिर्मिर॒मति॑मेमिषम्। आ॒शायै॑ जा॒मिम्। प्र॒ती॒क्षायै॑ कुमा॒रीम्॥१९॥%\anuvakamend[नो॒ द्वि॒ष्म इति॑ परा॒वत॑मर्पयति]




\prashnaend{ब्रह्म॑णे गी॒ताय॒ श्रमा॑य स॒न्धये॑ न॒दीभ्य॑ उत्सा॒देभ्य॒ ऋत्यै॒ भाया॒ अर्मेभ्यो म॒न्यवे॑ य॒म्यै॑ दश॑दश॒ सरोभ्यो॒ द्वाद॑श प्रति॒श्रुत्का॑यै बीभ॒त्सायै॒ दश॑दश॒ हसा॑य स॒प्ताक्ष॑रा॒जाय॒ त्रयो॑दश॒ भूम्यै॒ दश॑ वा॒चे षडथ॒ नवैका॒न्नविशतिः॥१९॥}{ब्रह्म॑णे य॒म्यै॑ नव॑दश॥१९॥}{ब्रह्म॑णे कुमा॒रीम्॥}{हरि॑ ओम्॥}{इति श्रीकृष्णयजुर्वेदीयतैत्तिरीयब्राह्मणे तृतीयाष्टके चतुर्थः प्रपाठकः समाप्तः॥}
\clearpage
