\sect{पञ्चमः प्रश्नः}
\setcounter{anuvakam}{0}
\dnsub{तैत्तिरीयब्राह्मणे प्रथमाष्टके पञ्चमः प्रपाठकः}

%1.5.1.1
अ॒ग्नेः कृत्ति॑काः। शु॒क्रं प॒रस्ता॒ज्ज्योति॑र॒वस्तात्। प्र॒जाप॑ते रोहि॒णी। आप॑ प॒रस्ता॒दोष॑धयो॒ऽवस्तात्। सोम॑स्येन्व॒का वित॑तानि। प॒रस्ता॒द्वय॑न्तो॒ऽवस्तात्। रु॒द्रस्य॑ बा॒हू। मृ॒ग॒यव॑ प॒रस्ताद्विक्षा॒रो॑ऽवस्तात्। आदि॑त्यै॒ पुन॑र्वसू। वात॑ प॒रस्ता॑दा॒र्द्रम॒वस्तात्॥१॥

%1.5.1.2
बृह॒स्पतेस्ति॒ष्य॑। जुह्व॑तः प॒रस्ता॒द्यज॑माना अ॒वस्तात्। स॒र्पाणा॑माश्रे॒षाः। अ॒भ्या॒गच्छ॑न्तः प॒रस्ता॑दभ्या॒नृत्य॑न्तो॒\-ऽवस्तात्। पि॒तृ॒णां म॒घाः। रु॒दन्त॑ प॒रस्ता॑दपभ्र॒शो॑ऽवस्तात्। अ॒र्य॒म्णः पूर्वे॒ फल्गु॑नी। जा॒या प॒रस्ता॑दृष॒भो॑ऽवस्तात्। भग॒स्योत्त॑रे। व॒ह॒तव॑ प॒रस्ता॒द्वह॑माना अ॒वस्तात्॥२॥

%1.5.1.3
दे॒वस्य॑ सवि॒तुर्\mbox{}हस्त॑। प्र॒स॒वः प॒रस्तात्स॒निर॒वस्तात्। इन्द्र॑स्य चि॒त्रा। ऋ॒तं प॒रस्तात्स॒त्यम॒वस्तात्। वा॒योर्निष्ट्या व्र॒तति॑। प॒रस्ता॒दसि॑द्धिर॒वस्तात्। इ॒न्द्रा॒ग्नि॒योर्विशा॑खे। यु॒गानि॑ प॒रस्तात्कृ॒षमा॑णा अ॒वस्तात्। मि॒त्रस्या॑नूरा॒धाः। अ॒भ्या॒रोह॑त्प॒रस्ता॑द॒भ्यारू॑ढम॒वस्तात्॥३॥

%1.5.1.4
इन्द्र॑स्य रोहि॒णी। शृ॒णत्प॒रस्तात्प्रतिशृ॒णद॒वस्तात्। निर्\mbox{}ऋ॑त्यै मूल॒वर्\mbox{}ह॑णी। प्र॒ति॒भ॒ञ्जन्त॑ प॒रस्तात्प्रतिशृ॒णन्तो॒ऽवस्तात्। अ॒पां पूर्वा॑ अषा॒ढाः। वर्च॑ प॒रस्ता॒त्समि॑तिर॒वस्तात्। विश्वे॑षां दे॒वाना॒मुत्त॑राः। अ॒भि॒जय॑त्प॒रस्ता॑द॒भिजि॑तम॒वस्तात्। विष्णो श्रो॒णा पृ॒च्छमा॑नाः। प॒रस्ता॒त्पन्था॑ अ॒वस्तात्॥४॥

%1.5.1.5
वसू॑ना॒ श्रवि॑ष्ठाः। भू॒तं प॒रस्ता॒द्भूति॑र॒वस्तात्। इन्द्र॑स्य श॒तभि॑षक्। वि॒श्वव्य॑चाः प॒रस्ताद्वि॒श्वक्षि॑तिर॒वस्तात्। अ॒जस्यैक॑पद॒ पूर्वे प्रोष्ठप॒दाः। वै॒श्वा॒न॒रं प॒रस्ताद्वैश्वावस॒वम॒\-वस्तात्। अहेर्बु॒ध्निय॒स्योत्त॑रे। अ॒भि॒षि॒ञ्चन्त॑ प॒रस्ता॑दभि\-षु॒ण्वन्तो॒ऽवस्तात्। पू॒ष्णो रे॒वती। गाव॑ प॒रस्ताद्व॒त्सा अ॒वस्तात्। अ॒श्विनो॑रश्व॒युजौ। ग्राम॑ प॒रस्ता॒त्सेना॒ऽवस्तात्। य॒मस्या॑प॒भर॑णीः। अ॒प॒कर्\mbox{}ष॑न्तः प॒रस्ता॑दप॒वह॑न्तो॒ऽवस्तात्। पू॒र्णा प॒श्चाद्यत्ते॑ दे॒वा अद॑धुः॥५॥\anuvakamend[आ॒र्द्रम॒वस्ता॒द्वह॑माना अ॒वस्ता॑द॒भ्यारू॑ढम॒वस्ता॒त्पन्था॑ अ॒वस्ताद्व॒त्सा अ॒वस्ता॒त्पञ्च॑ च]

%1.5.2.1
यत्पुण्यं॒ नक्ष॑त्रम्। तद्बट्कु॑र्वीतोपव्यु॒षम्। य॒दा वै सूर्य॑ उ॒देति॑। अथ॒ नक्ष॑त्रं॒ नैति॑। याव॑ति॒ तत्र॒ सूर्यो॒ गच्छेत्। यत्र॑ जघ॒न्यं॑ पश्येत्। ताव॑ति कुर्वीत यत्का॒री स्यात्। पु॒ण्या॒ह ए॒व कु॑रुते। ए॒व ह॒ वै य॒ज्ञेषुं॑ च श॒तद्यु॑म्नं च मा॒त्स्यो नि॑रवसाय॒यां च॑कार॥६॥

%1.5.2.2
यो वै न॑क्ष॒त्रियं॑ प्र॒जाप॑तिं॒ वेद॑। उ॒भयो॑रेनं लो॒कयोर्विदुः। हस्त॑ ए॒वास्य॒ हस्त॑। चि॒त्रा शिर॑। निष्ट्या॒ हृद॑यम्। ऊ॒रू विशा॑खे। प्र॒ति॒ष्ठाऽनू॑रा॒धाः। ए॒ष वै न॑क्ष॒त्रिय॑ प्र॒जाप॑तिः। य ए॒वं वेद॑। उ॒भयो॑रेनं लो॒कयोर्विदुः॥७॥

%1.5.2.3
अ॒स्मिश्चा॒मुष्मिश्च। यां का॒मये॑त दुहि॒तरं॑ प्रि॒या स्या॒दिति॑। तां निष्ट्या॑यां दद्यात्। प्रि॒यैव भ॑वति। नेव॒ तु पुन॒राग॑च्छति। अ॒भि॒जिन्नाम॒ नक्ष॑त्रम्। उ॒परि॑ष्टादषा॒ढानाम्। अ॒वस्ताच्छ्रो॒णायै। दे॒वा॒सु॒राः संय॑त्ता आसन्। ते दे॒वास्तस्मि॒न्नक्ष॑त्रे॒ऽभ्य॑जयन्॥८॥

%1.5.2.4
यद॒भ्यज॑यन्। तद॑भि॒जितो॑ऽभिजि॒त्त्वम्। यं का॒मये॑तानपज॒य्यं ज॑ये॒दिति॑। तमे॒तस्मि॒न्नक्ष॑त्रे यातयेत्। अ॒न॒प॒ज॒य्यमे॒व ज॑यति। पा॒पप॑राजितमिव॒ तु। प्र॒जाप॑तिः प॒शून॑सृजत। ते नक्ष॑त्रं नक्षत्र॒मुपा॑तिष्ठन्त। ते स॒माव॑न्त ए॒वाभ॑वन्। ते रे॒वती॒मुपा॑तिष्ठन्त॥९॥

%1.5.2.5
ते रे॒वत्यां॒ प्राभ॑वन्। तस्माद्रे॒वत्यां पशू॒नां कु॑र्वीत। यत्किं चार्वा॒चीन॒ सोमात्। प्रैव भ॑वन्ति। स॒लि॒लं वा इ॒दम॑न्त॒रासीत्। यदत॑रन्। तत्तार॑काणां तारक॒त्वम्। यो वा इ॒ह यज॑ते। अ॒मु स लो॒कं न॑क्षते। तन्नक्ष॑त्राणां नक्षत्र॒त्वम्॥१०॥

%1.5.2.6
दे॒व॒गृ॒हा वै नक्ष॑त्राणि। य ए॒वं वेद॑। गृ॒ह्ये॑व भ॑वति। यानि॒ वा इ॒मानि॑ पृथि॒व्याश्चि॒त्राणि॑। तानि॒ नक्ष॑त्राणि। तस्मा॑दश्ली॒लना॑मश्चि॒त्रे। नाव॑स्ये॒न्न य॑जेत। यथा॑ पापा॒हे कु॑रु॒ते। ता॒दृगे॒व तत्। दे॒व॒न॒क्ष॒त्राणि॒ वा अ॒न्यानि॑॥११॥

%1.5.2.7
य॒म॒न॒क्ष॒त्राण्य॒न्यानि॑। कृत्ति॑काः प्रथ॒मम्। विशा॑खे उत्त॒मम्। तानि॑ देवनक्ष॒त्राणि॑। अ॒नू॒रा॒धाः प्र॑थ॒मम्। अ॒प॒भर॑णीरुत्त॒मम्। तानि॑ यमनक्ष॒त्राणि॑। यानि॑ देवनक्ष॒त्राणि॑। तानि॒ दक्षि॑णेन॒ परि॑यन्ति। यानि॑ यमनक्ष॒त्राणि॑॥१२॥

%1.5.2.8
तान्युत्त॑रेण। अन्वे॑षामरा॒त्स्मेति॑। तद॑नूरा॒धाः। ज्ये॒ष्ठमे॑षाम\-वधि॒ष्मेति॑। तज्ज्येष्ठ॒घ्नी। मूल॑मेषामवृक्षा॒मेति॑। तन्मू॑ल॒वर्\mbox{}ह॑णी। यन्नास॑हन्त। तद॑षा॒ढाः। यदश्लो॑णत्॥१३॥

%1.5.2.9
तच्छ्रो॒णा। यदशृ॑णोत्। तच्छ्रवि॑ष्ठाः। यच्छ॒तमभि॑षज्यन्। तच्छ॒तभि॑षक्। प्रो॒ष्ठ॒प॒देषूद॑यच्छन्त। रे॒वत्या॑मरवन्त। अ॒श्व॒युजो॑रयुञ्जत। अ॒प॒भर॑णी॒ष्वपा॑वहन्। तानि॒ वा ए॒तानि॑ यमनक्ष॒त्राणि॑। यान्ये॒व दे॑वनक्ष॒त्राणि॑। तेषु॑ कुर्वीत यत्का॒री स्यात्। पु॒ण्या॒ह ए॒व कु॑रुते॥१४॥\anuvakamend[च॒का॒रै॒वं वेदो॒भयो॑रेनं लो॒कयोर्विदुरजयन्रे॒वती॒मुपा॑तिष्ठन्त नक्षत्र॒त्वम॒न्यानि॒ यानि॑ यमनक्ष॒त्राण्यश्लो॑णद्यमनक्ष॒त्राणि॒ त्रीणि॑ च]

%1.5.3.1
दे॒वस्य॑ सवि॒तुः प्रा॒तः प्र॑स॒वः प्रा॒णः। वरु॑णस्य सा॒यमा॑स॒वो॑ऽपा॒नः। यत्प्र॑ती॒चीनं॑ प्रात॒स्तनात्। प्रा॒चीन सङ्ग॒वात्। ततो॑ दे॒वा अ॑ग्निष्टो॒मं निर॑मिमत। तत्तदात्त॑वीर्यं निर्मा॒र्गः। मि॒त्रस्य॑ सङ्ग॒वः। तत्पुण्यं॑ तेज॒स्व्यह॑। तस्मा॒त्तर्\mbox{}हि॑ प॒शव॑ स॒माय॑न्ति। यत्प्र॑ती॒चीन सङ्ग॒वात्॥१५॥

%1.5.3.2
प्रा॒चीनं॑ म॒ध्यं दि॑नात्। ततो॑ दे॒वा उ॒क्थ्यं॑ निर॑मिमत। तत्तदात्त॑वीर्यं निर्मा॒र्गः। बृह॒स्पतेर्म॒ध्यं दि॑नः। तत्पुण्यं॑ तेज॒स्व्यह॑। तस्मा॒त्तर्\mbox{}हि॒ तेक्ष्णि॑ष्ठं तपति। यत्प्र॑ती॒चीनं॑ म॒ध्यं दि॑नात्। प्रा॒चीन॑मपरा॒ह्णात्। ततो॑ दे॒वाः षो॑ड॒शिनं॒ निर॑मिमत। तत्तदात्त॑वीर्यं निर्मा॒र्गः॥१६॥

%1.5.3.3
भग॑स्यापरा॒ह्णः। तत्पुण्यं॑ तेज॒स्व्यह॑। तस्मा॑दपरा॒ह्णे कु॑मा॒र्यो॑ भग॑मि॒च्छमा॑नाश्चरन्ति। यत्प्र॑ती॒चीन॑मपरा॒ह्णात्। प्रा॒चीन सा॒यात्। ततो॑ दे॒वा अ॑तिरा॒त्रं निर॑मिमत। तत्तदात्त॑वीर्यं निर्मा॒र्गः। वरु॑णस्य सा॒यम्। तत्पुण्यं॑ तेज॒स्व्यह॑। तस्मा॒त्तर्\mbox{}हि॒ नानृ॑तं वदेत्॥१७॥

%1.5.3.4
ब्रा॒ह्म॒णो वा अ॑ष्टावि॒शो नक्ष॑त्राणाम्। स॒मा॒नस्याह्न॒ पञ्च॒ पुण्या॑नि॒ नक्ष॑त्राणि। च॒त्वार्य॑श्ली॒लानि॑। तानि॒ नव॑। यच्च॑ प॒रस्ता॒न्नक्ष॑त्राणां॒ यच्चा॒वस्तात्। तान्येका॑दश। ब्रा॒ह्म॒णो द्वा॑द॒शः। य ए॒वं वि॒द्वान्त्सं॑वत्स॒रं व्र॒तं चर॑ति। सं॒व॒त्स॒रेणै॒वास्य॑ व्र॒तं गु॒प्तं भ॑वति। स॒मा॒नस्याह्न॒ पञ्च॒ पुण्या॑नि॒ नक्ष॑त्राणि। च॒त्वार्य॑श्ली॒लानि॑। तानि॒ नव॑। आ॒ग्ने॒यी रात्रि॑। ऐ॒न्द्रमह॑। तान्येका॑दश। आ॒दि॒त्यो द्वा॑द॒शः। य ए॒वं वि॒द्वान्त्सं॑वत्स॒रं व्र॒तं चर॑ति। सं॒व॒त्स॒रेणै॒वास्य॑ व्र॒तं गु॒प्तं भ॑वति॥१८॥\anuvakamend[स॒ङ्ग॒वाथ्षो॑ड॒शिनं॒ निर॑मिमत॒ तत्तदात्त॑वीर्यं निर्मा॒र्गो व॑देद्भवति समा॒नस्याह्न॒ पञ्च॒ पुण्या॑नि॒ नक्ष॑त्राण्य॒ष्टौ च॑]

%1.5.4.1
ब्र॒ह्म॒वा॒दिनो॑ वदन्ति। कति॒ पात्रा॑णि य॒ज्ञं व॑ह॒न्तीति॑। त्रयो॑द॒शेति॑ ब्रूयात्। स यद्ब्रू॒यात्। कस्तानि॒ निर॑मिमी॒तेति॑। प्र॒जाप॑ति॒रिति॑ ब्रूयात्। स यद्ब्रू॒यात्। कुत॒स्तानि॒ निर॑मिमी॒तेति॑। आ॒त्मन॒ इति॑। प्रा॒णा॒पा॒नाभ्या॑मे॒वोपा\-श्वन्तर्या॒मौ निर॑मिमीत॥१९॥

%1.5.4.2
व्या॒नादु॑पाशु॒सव॑नम्। वा॒च ऐन्द्रवाय॒वम्। द॒क्ष॒क्र॒तुभ्यां मैत्रावरु॒णम्। श्रोत्रा॑दाश्वि॒नम्। चक्षु॑षः शु॒क्राम॒न्थिनौ। आ॒त्मन॑ आग्रय॒णम्। अङ्गेभ्य उ॒क्थ्यम्। आयु॑षो ध्रु॒वम्। प्र॒ति॒ष्ठाया॑ ऋतुपा॒त्रे। य॒ज्ञं वाव तं प्र॒जाप॑ति॒र्निर॑मिमीत। स निर्मि॑तो॒ नाद्ध्रि॑यत॒ सम॑व्लीयत। स ए॒तान्प्र॒जाप॑तिरपिवा॒पान॑पश्यत्। तां निर॑वपत्। तैर्वै स य॒ज्ञमप्य॑वपत्। यद॑पिवा॒पा भव॑न्ति। य॒ज्ञस्य॒ धृत्या॒ असंव्लयाय॥२०॥\anuvakamend[उ॒पा॒श्व॒न्त॒र्या॒मौ निर॑मिमीतामिमीत॒ षट्च॑]

%1.5.5.1
ऋ॒तमे॒व प॑रमे॒ष्ठि। ऋ॒तं नात्ये॑ति॒ किञ्च॒न। ऋ॒ते स॑मु॒द्र आहि॑तः। ऋ॒ते भूमि॑रि॒यश्रि॒ता। अ॒ग्निस्ति॒ग्मेन॑ शो॒चिषा। तप॒ आक्रान्तमु॒ष्णिहा। शिर॒स्तप॒स्याहि॑तम्। वै॒श्वा॒न॒रस्य॒ तेज॑सा। ऋ॒तेनास्य॒ नि व॑र्तये। स॒त्येन॒ परि॑ वर्तये। तप॑सा॒ऽस्यानु॑ वर्तये। शि॒वेना॒स्योप॑ वर्तये। श॒ग्मेनास्या॒भि व॑र्तये। तदृ॒तं तत्स॒त्यम्। तद्व्र॒तं तच्छ॑केयम्। तेन॑ शकेयं॒ तेन॑ राध्यासम्॥२१॥

%1.5.5.2
यद्घ॒र्मः प॒र्यव॑र्तयत्। अन्तान्पृथि॒व्या दि॒वः। अ॒ग्निरीशा॑न॒ ओज॑सा। वरु॑णो धी॒तिभि॑ स॒ह। इन्द्रो॑ म॒रुद्भि॒ सखि॑भिः स॒ह। अ॒ग्निस्ति॒ग्मेन॑ शो॒चिषा। तप॒ आक्रान्तमु॒ष्णिहा। शिर॒स्तप॒स्याहि॑तम्। वै॒श्वा॒न॒रस्य॒ तेज॑सा। ऋ॒तेनास्य॒ नि व॑र्तये। स॒त्येन॒ परि॑ वर्तये। तप॑सा॒ऽस्यानु॑ वर्तये। शि॒वेना॒स्योप॑ वर्तये। श॒ग्मेनास्या॒भि व॑र्तये। तदृ॒तं तत्स॒त्यम्। तद्व्र॒तं तच्छ॑केयम्। तेन॑ शकेयं॒ तेन॑ राध्यासम्॥२२॥

%1.5.5.3
यो अ॒स्याः पृ॑थि॒व्यास्त्व॒चि। नि॒व॒र्तय॒त्योष॑धीः। अ॒ग्निरीशा॑न॒ ओज॑सा। वरु॑णो धी॒तिभि॑ स॒ह। इन्द्रो॑ म॒रुद्भि॒ सखि॑भिः स॒ह। अ॒ग्निस्ति॒ग्मेन॑ शो॒चिषा। तप॒ आक्रान्तमु॒ष्णिहा। शिर॒स्तप॒स्याहि॑तम्। वै॒श्वा॒न॒रस्य॒ तेज॑सा। ऋ॒तेनास्य॒ नि व॑र्तये। स॒त्येन॒ परि॑ वर्तये। तप॑सा॒ऽस्यानु॑ वर्तये। शि॒वेना॒स्योप॑ वर्तये। श॒ग्मेनास्या॒भि व॑र्तये। तदृ॒तं तत्स॒त्यम्। तद्व्र॒तं तच्छ॑केयम्। तेन॑ शकेयं॒ तेन॑ राध्यासम्॥२३॥

%1.5.5.4
एकं॒ मास॒मुद॑सृजत्। प॒र॒मे॒ष्ठी प्र॒जाभ्य॑। तेनाभ्यो॒ मह॒ आव॑हत्। अ॒मृतं॒ मर्त्याभ्यः। प्र॒जामनु॒ प्र जा॑यसे। तदु॑ ते मर्त्या॒मृतम्। येन॒ मासा॑ अर्धमा॒साः। ऋ॒तव॑ परिवत्स॒राः। येन॒ ते ते प्रजापते। ई॒जा॒नस्य॒ न्यव॑र्तयन्। तेना॒हम॒स्य ब्रह्म॑णा। निव॑र्तयामि जी॒वसे। अ॒ग्निस्ति॒ग्मेन॑ शो॒चिषा। तप॒ आक्रान्तमु॒ष्णिहा। शिर॒स्तप॒स्याहि॑तम्। वै॒श्वा॒न॒रस्य॒ तेज॑सा। ऋ॒तेनास्य॒ नि व॑र्तये। स॒त्येन॒ परि॑ वर्तये। तप॑सा॒ऽस्यानु॑ वर्तये। शि॒वेना॒स्योप॑ वर्तये। श॒ग्मेनास्या॒भि व॑र्तये। तदृ॒तं तत्स॒त्यम्। तद्व्र॒तं तच्छ॑केयम्। तेन॑ शकेयं॒ तेन॑ राध्यासम्॥२४॥\anuvakamend[(परि॑वर्तये स॒हाभिव॑र्तय उ॒ष्णिहा॑ राध्यास॒न्न्यव॑र्तय॒न्नुप॑वर्तये च॒त्वारि॑ च)। ऋ॒तमे॒व षोड॑श। यद्घ॒र्मो यो अ॒स्याः सप्तद॑शसप्तदश। एकं॒ मासं॒ चतु॑र्विशतिः।]

%1.5.6.1
दे॒वा वै यद्य॒ज्ञेऽकु॑र्वत। तदसु॑रा अकुर्वत। तेऽसु॑रा ऊ॒र्ध्वं पृ॒ष्ठेभ्यो॒ नाप॑श्यन्। ते केशा॒नग्रे॑ऽवपन्त। अथ॒ श्मश्रू॑णि। अथो॑पप॒क्षौ। तत॒स्तेऽवाञ्च आयन्। परा॑ऽभवन्। यस्यै॒वं वप॑न्ति। अवा॑ङेति॥२५॥

%1.5.6.2
अथो॒ परै॒व भ॑वति। अथ॑ दे॒वा ऊ॒र्ध्वं पृ॒ष्ठेभ्यो॑ऽपश्यन्। त उ॑पप॒क्षावग्रे॑ऽवपन्त। अथ॒ श्मश्रू॑णि। अथ॒ केशान्॑। तत॒स्ते॑ऽभवन्। सु॒व॒र्गं लो॒कमा॑यन्। यस्यै॒वं वप॑न्ति। भव॑त्या॒त्मना। अथो॑ सुव॒र्गं लो॒कमे॑ति॥२६॥

%1.5.6.3
अथै॒तन्मनु॑र्व॒प्त्रे मि॑थु॒नम॑पश्यत्। स श्मश्रू॒ण्यग्रे॑ऽवपत। अथो॑पप॒क्षौ। अथ॒ केशान्॑। ततो॒ वै स प्राजा॑यत प्र॒जया॑ प॒शुभि॑। यस्यै॒वं वप॑न्ति। प्र प्र॒जया॑ प॒शुभि॑र्मिथु॒नैर्जा॑यते। दे॒वा॒सु॒राः संय॑त्ता आसन्। ते सं॑वत्स॒रे व्याय॑च्छन्त। तान्दे॒वाश्चा॑तुर्मा॒स्यैरे॒वाभि प्रायु॑ञ्जत॥२७॥

%1.5.6.4
वै॒श्व॒दे॒वेन॑ च॒तुरो॑ मा॒सो॑ऽवृञ्ज॒तेन्द्र॑राजानः। ताञ्छी॒र्॒षं नि चाव॑र्तयन्त॒ परि॑ च। व॒रु॒ण॒प्र॒घा॒सैश्च॒तुरो॑ मा॒सो॑ऽवृञ्जत॒ वरु॑णराजानः। ताञ्छी॒र्॒षं नि चाव॑र्तयन्त॒ परि॑ च। सा॒क॒मे॒धैश्च॒तुरो॑ मा॒सो॑ऽवृञ्जत॒ सोम॑राजानः। ताञ्छी॒र्॒षं नि चाव॑र्तयन्त॒ परि॑ च। या सं॑वत्स॒र उ॑पजी॒वाऽऽसीत्। तामे॑षामवृञ्जत। ततो॑ दे॒वा अभ॑वन्। पराऽसु॑राः॥२८॥

%1.5.6.5
य ए॒वं वि॒द्वाश्चा॑तुर्मा॒स्यैर्यज॑ते। भ्रातृ॑व्यस्यै॒व मा॒सो वृ॒क्त्वा। शी॒र्॒षं नि च॑ व॒र्तय॑ते॒ परि॑ च। यैषा सं॑वत्स॒र उ॑पजी॒वा। वृ॒ङ्क्ते तां भ्रातृ॑व्यस्य। क्षु॒धाऽस्य॒ भ्रातृ॑व्य॒ परा॑ भवति। लो॒हि॒ता॒य॒सेन॒ नि व॑र्तयते। यद्वा इ॒माम॒ग्निर्\mbox{}ऋ॒तावाग॑ते निव॒र्तय॑ति। ए॒तदे॒वैना रू॒पं कृ॒त्वा निव॑र्तयति। सा तत॒ श्वश्वो॒ भूय॑सी॒ भव॑न्त्येति॥२९॥

%1.5.6.6
प्र जा॑यते। य ए॒वं वि॒द्वाल्लोँ॑हिताय॒सेन॑ निव॒र्तय॑ते। ए॒तदे॒व रू॒पं कृ॒त्वा नि व॑र्तयते। स तत॒ श्वश्वो॒ भूया॒न्भव॑न्नेति। प्रैव जा॑यते। त्रे॒ण्या श॑ल॒ल्या नि व॑र्तयेत। त्रीणि॑त्रीणि॒ वै दे॒वाना॑मृ॒द्धानि॑। त्रीणि॒ छन्दासि। त्रीणि॒ सव॑नानि। त्रय॑ इ॒मे लो॒काः॥३०॥

%1.5.6.7
ऋ॒ध्यामे॒व तद्वी॒र्य॑ ए॒षु लो॒केषु॒ प्रति॑ तिष्ठति। यच्चा॑तुर्मास्यया॒ज्यात्मनो॒ नाव॒द्येत्। दे॒वेभ्य॒ आवृ॑श्च्येत। च॒तृ॒षुच॑तृषु॒ मासे॑षु॒ नि व॑र्तयेत। प॒रोक्ष॑मे॒व तद्दे॒वेभ्य॑ आ॒त्मनोऽव॑द्य॒त्यनाव्रस्काय। दे॒वानां॒ वा ए॒ष आनी॑तः। यश्चा॑तुर्मास्यया॒जी। य ए॒वं वि॒द्वान्नि च॑ व॒र्तय॑ते॒ परि॑ च। दे॒वता॑ ए॒वाप्ये॑ति। नास्य॑ रु॒द्रः प्र॒जां प॒शून॒भि म॑न्यते॥३१॥\anuvakamend[ए॒त्ये॒त्य॒यु॒ञ्ज॒तासु॑रा एति लो॒का म॑न्यते]

%1.5.7.1
आयु॑षः प्रा॒ण सन्त॑नु। प्रा॒णाद॑पा॒न सन्त॑नु। अ॒पा॒नाद्व्या॒न सन्त॑नु। व्या॒नाच्चक्षु॒ सन्त॑नु। चक्षु॑ष॒ श्रोत्र॒ सन्त॑नु। श्रोत्रा॒न्मन॒ सन्त॑नु। मन॑सो॒ वाच॒ सन्त॑नु। वा॒च आ॒त्मान॒ सन्त॑नु। आ॒त्मन॑ पृथि॒वी सन्त॑नु। पृ॒थि॒व्या अ॒न्तरि॑क्ष॒ सन्त॑नु। अ॒न्तरि॑क्षा॒द्दिव॒ सन्त॑नु। दिव॒ सुव॒ सन्त॑नु॥३२॥\anuvakamend[अ॒न्तरि॑क्ष॒ सन्त॑नु॒ द्वे च॑]

%1.5.8.1
इन्द्रो॑ दधी॒चो अ॒स्थभि॑। वृ॒त्राण्यप्र॑तिष्कुतः। ज॒घान॑ नव॒तीर्नव॑। इ॒च्छन्नश्व॑स्य॒ यच्छिर॑। पर्व॑ते॒ष्वप॑श्रितम्। तद्वि॑दच्छर्य॒णाव॑ति। अत्राह॒ गोरम॑न्वत। नाम॒ त्वष्टु॑रपी॒च्यम्। इ॒त्था च॒न्द्रम॑सो गृ॒हे। इन्द्र॒मिद्गा॒थिनो॑ बृ॒हत्॥३३॥

%1.5.8.2
इन्द्र॑म॒र्केभि॑र॒र्किण॑। इन्द्रं॒ वाणी॑रनूषत। इन्द्र॒ इद्धर्यो॒ सचा। सम्मि॑श्ल॒ आव॑चो॒ युजा। इन्द्रो॑ व॒ज्री हि॑र॒ण्यय॑। इन्द्रो॑ दी॒र्घाय॒ चक्ष॑से। आ सूर्य रोहयद्दि॒वि। वि गोभि॒रद्रि॑मैरयत्। इन्द्र॒ वाजे॑षु नो अव। स॒हस्र॑प्रधनेषु च॥३४॥

%1.5.8.3
उ॒ग्र उ॒ग्राभि॑रू॒तिभि॑। तमिन्द्रं॑ वाजयामसि। म॒हे वृ॒त्राय॒ हन्त॑वे। स वृषा॑ वृष॒भो भु॑वत्। इन्द्र॒ स दाम॑ने कृ॒तः। ओजि॑ष्ठ॒ स बले॑ हि॒तः। द्यु॒म्नी श्लो॒की स सौ॒म्य॑। गि॒रा वज्रो॒ न सम्भृ॑तः। सब॑लो॒ अन॑पच्युतः। व॒व॒क्षुरु॒ग्रो अस्तृ॑तः॥३५॥\anuvakamend[बृ॒हच्चास्तृ॑तः]

%1.5.9.1
दे॒वा॒सु॒राः संय॑त्ता आसन्। स प्र॒जाप॑ति॒रिन्द्रं॑ ज्ये॒ष्ठं पु॒त्रमप॒ न्य॑धत्त। नेदे॑न॒मसु॑रा॒ बली॑यासोऽहन॒न्निति॑। प्र॒ह्रादो॑ ह॒ वै का॑याध॒वः। वि॒रोच॑न॒ स्वं पु॒त्रमप॒ न्य॑धत्त। नेदे॑नं दे॒वा अ॑हन॒न्निति॑। ते दे॒वाः प्र॒जाप॑तिमुपस॒मेत्यो॑चुः। नारा॒जक॑स्य यु॒द्धम॑स्ति। इन्द्र॒मन्वि॑च्छा॒मेति॑। तं॒ य॑ज्ञक्र॒तुभि॒रन्वैच्छन्॥३६॥

%1.5.9.2
तं॒ य॑ज्ञक्र॒तुभि॒र्नान्व॑विन्दन्। तमिष्टि॑भि॒रन्वैच्छन्। तमिष्टि॑भि॒रन्व॑विन्दन्। तदिष्टी॑नामिष्टि॒त्वम्। एष्ट॑यो ह॒ वै नाम॑। ता इष्ट॑य॒ इत्याच॑क्षते प॒रोक्षे॑ण। प॒रोक्ष॑प्रिया इव॒ हि दे॒वाः। तस्मा॑ ए॒तमाग्नावैष्ण॒वमेका॑दशकपालं दीक्ष॒णीयं॒ निर॑वपन्। तद॑प॒द्रुत्या॑तन्वत। तान्प॑त्नीसंया॒जान्त॒ उपा॑नयन्॥३७॥

%1.5.9.3
ते तद॑न्तमे॒व कृ॒त्वोद॑द्रवन्। ते प्रा॑य॒णीय॑म॒भि स॒मारो॑हन्। तद॑प॒द्रुत्या॑तन्वत। ताञ्छ॒य्य्वँ॑न्त॒ उपा॑नयन्। ते तद॑न्तमे॒व कृ॒त्वोद॑द्रवन्। त आ॑ति॒थ्यम॒भि स॒मारो॑हन्। तद॑प॒द्रुत्या॑तन्वत। तानिडान्त॒ उपा॑नयन्। ते तद॑न्तमे॒व कृ॒त्वोद॑द्रवन्। तस्मा॑दे॒ता ए॒तद॑न्ता॒ इष्ट॑य॒ सन्ति॑ष्ठन्ते॥३८॥

%1.5.9.4
ए॒व हि दे॒वा अकु॑र्वत। इति॑ दे॒वा अ॑कुर्वत। इत्यु॒ वै म॑नु॒ष्या कुर्वते। ते दे॒वा ऊ॑चुः। यद्वा इ॒दमु॒च्चैर्य॒ज्ञेन॒ चरा॑म। तन्नोऽसु॑राः पा॒प्माऽनु॑विन्दन्ति। उ॒पा॒शू॑प॒सदा॑ चराम। तथा॒ नोऽसु॑राः पा॒प्मा नानु॑वेत्स्य॒न्तीति॑। त उ॑पा॒शू॑प॒सद॑मतन्वत। ति॒स्र ए॒व सा॑मिधे॒नीर॒नूच्य॑॥३९॥

%1.5.9.5
स्रु॒वेणा॑घा॒रमा॒घार्य॑। ति॒स्रः परा॑ची॒राहु॑तीर्\mbox{}हु॒त्वा। स्रु॒वेणो॑प॒सदं॑ जुह॒वां च॑क्रुः। उ॒ग्रं वचो॒ अपा॑वधीन्त्वे॒षं वचो॒ अपा॑वधी॒स्वाहेति॑। अ॒श॒न॒या॒पि॒पा॒से ह॒ वा उ॒ग्रं वच॑। एन॑श्च॒ वैर॑हत्यं च त्वे॒षं वच॑। ए॒त ह॒ वाव तच्च॑तुर्धाविहि॒तं पा॒प्मानं॑ दे॒वा अप॑जघ्निरे। तथो॑ ए॒वैतदे॑वं॒विद्यज॑मानः। ति॒स्र ए॒व सा॑मिधे॒नीर॒नूच्य॑। स्रु॒वेणा॑घा॒रमा॒घार्य॑॥४०॥

%1.5.9.6
ति॒स्रः परा॑ची॒राहु॑तीर्\mbox{}हु॒त्वा। स्रु॒वेणो॑प॒सदं॑ जुहोति। उ॒ग्रं वचो॒ अपा॑वधीन्त्वे॒षं वचो॒ अपा॑वधी॒ स्वाहेति॑। अ॒श॒न॒या॒पि॒पा॒से ह॒ वा उ॒ग्रं वच॑। एन॑श्च॒ वैर॑हत्यं च त्वे॒षं वच॑। ए॒तमे॒व तच्च॑तुर्धाविहि॒तं पा॒प्मानं॒ यज॑मा॒नोऽप॑ हते। ते॑ऽभि॒नीयै॒वाह॑ प॒शुमाऽल॑भन्त। अह्न॑ ए॒व तद्दे॒वा अव॑र्तिं पा॒प्मानं॑ मृ॒त्युमप॑जघ्निरे। तेना॑भि॒नीये॑व॒ रात्रे॒ प्राच॑रन्। रात्रि॑या ए॒व तद्दे॒वा अव॑र्तिं पा॒प्मानं॑ मृ॒त्युमप॑जघ्निरे॥४१॥

%1.5.9.7
तस्मा॑दभि॒नीयै॒वाह॑ प॒शुमा ल॑भेत। अह्न॑ ए॒व तद्यज॑मा॒नोऽव॑र्तिं पा॒प्मानं॒ भ्रातृ॑व्या॒नप॑ नुदते। तेना॑भि॒नीये॑व॒ रात्रे॒ प्रच॑रेत्। रात्रि॑या ए॒व तद्यज॑मा॒नोऽव॑र्तिं पा॒प्मानं॒ भ्रातृ॑व्या॒नप॑ नुदते। स ए॒ष उ॑पवस॒थीयेऽह॑न्द्विदेव॒त्य॑ प॒शुरा ल॑भ्यते। द्व॒यं वा अ॒स्मिल्लोँ॒के यज॑मानः। अस्थि॑ च मा॒सं च॑। अस्थि॑ चै॒व तेन॑ मा॒सं च॒ यज॑मान॒ सस्कु॑रुते। ता वा ए॒ताः पञ्च॑ दे॒वता। अ॒ग्नीषोमा॑व॒ग्निर्मि॒त्रावरु॑णौ॥४२॥

%1.5.9.8
प॒ञ्च॒प॒ञ्ची वै यज॑मानः। त्वङ्मा॒स स्नावाऽस्थि॑ म॒ज्जा। ए॒तमे॒व तत्प॑ञ्चधाविहि॒तमा॒त्मानं॑ वरुणपा॒शान्मु॑ञ्चति। भे॒ष॒जता॑यै निर्वरुण॒त्वाय॑। त स॒प्तभि॒श्छन्दो॑भिः प्रा॒तर॑ह्वयन्। तस्मात्स॒प्त च॑तुरुत्त॒राणि॒ छन्दासि प्रातरनुवा॒केऽनूच्यन्ते। तमे॒तयो॑पस॒मेत्योपा॑सीदन्। उपास्मै गायता नर॒ इति॑। तस्मा॑दे॒तया॑ बहिष्पवमा॒न उ॑प॒सद्य॑॥४३॥\anuvakamend[ऐ॒च्छ॒न्न॒न॒य॒स्ति॒ष्ठ॒न्ते॒ऽनूच्या॒नूच्य॑ स्रु॒वेणा॑घा॒रमा॒घार्य॒ रात्रि॑या ए॒व तद्दे॒वा अव॑र्तिं पा॒प्मानं॑ मृ॒त्युमप॑जिघ्निरे मि॒त्रावरु॑णौ॒ नव॑ च (दे॒वा यज॑मानो दे॒वा दे॒वा यज॑मानो॒ यज॑मानोऽलभन्त॒ प्राच॑रल्लँभेत॒ प्रच॑रे॒दाल॑भ॒न्ताल॑भेत मृ॒त्युमप॑जघ्निरे॒ भ्रातृ॑व्यान्॥)]

%1.5.10.1
स स॑मु॒द्र उ॑त्तर॒तः प्राज्व॑लद्भूम्य॒न्तेन॑। ए॒ष वाव स स॑मु॒द्रः। यच्चात्वा॑लः। ए॒ष उ॑वे॒व स भूम्य॒न्तः। यद्वेद्य॒न्तः। तदे॒तत्त्रि॑श॒लन्त्रि॑पूरु॒षम्। तस्मा॒त्तं त्रि॑वित॒ तं ख॑नन्ति। स सु॑वर्णरज॒ताभ्यां कु॒शीभ्यां॒ परि॑गृहीत आसीत्। तं यद॒स्या अध्य॒जन॑यन्। तस्मा॑दादि॒त्यः॥४४॥

%1.5.10.2
अथ॒ यत्सु॑वर्णरज॒ताभ्यां कु॒शीभ्यां॒ परि॑गृहीत॒ आसीत्। साऽस्य॑ कौशि॒कता। तं त्रि॒वृता॒ऽभि प्रास्तु॑वत। तं त्रि॒वृताऽद॑दत। तं त्रि॒वृताऽह॑रन्। याव॑ती त्रि॒वृतो॒ मात्रा। तं प॑ञ्चद॒शेना॒भि प्रास्तु॑वत। तं प॑ञ्चद॒शेनाद॑दत। तं प॑ञ्चद॒शेनाह॑रन्। याव॑ती पञ्चद॒शस्य॒ मात्रा॥४५॥

%1.5.10.3
त स॑प्तद॒शेना॒भि प्रास्तु॑वत। त स॑प्तद॒शेनाद॑दत। त स॑प्तद॒शेनाह॑रन्। याव॑ती सप्तद॒शस्य॒ मात्रा। तस्य॑ सप्तद॒शेन॑ ह्रि॒यमा॑णस्य॒ तेजो॒ हरो॑ऽपतत्। तमे॑कवि॒शेना॒भि प्रास्तु॑वत। तमे॑कवि॒शेनाद॑दत। तमे॑कवि॒शेनाह॑रन्। याव॑त्येकवि॒शस्य॒ मात्रा। ते यत्त्रि॒वृता स्तु॒वते॥४६॥

%1.5.10.4
त्रि॒वृतै॒व तद्यज॑मान॒माद॑दते। तन्त्रि॒वृतै॒व ह॑रन्ति। याव॑ती त्रि॒वृतो॒ मात्रा। अ॒ग्निर्वै त्रि॒वृत्। याव॒द्वा अ॒ग्नेर्दह॑तो धू॒म उ॒देत्यानु॒ व्येति॑। ताव॑ती त्रि॒वृतो॒ मात्रा। अ॒ग्नेरे॒वैनं॒ तत्। मात्रा॒ सायु॑ज्य सलो॒कतां गमयन्ति। अथ॒ यत्प॑ञ्चद॒शेन॑ स्तु॒वते। प॒ञ्च॒द॒शेनै॒व तद्यज॑मान॒माद॑दते॥४७॥

%1.5.10.5
तं प॑ञ्चद॒शेनै॒व ह॑रन्ति। याव॑ती पञ्चद॒शस्य॒ मात्रा। च॒न्द्रमा॒ वै प॑ञ्चद॒शः। ए॒ष हि प॑ञ्चद॒श्याम॑पक्षी॒यते। प॒ञ्च॒द॒श्यामा॑पू॒र्यते। च॒न्द्रम॑स ए॒वैनं॒ तत्। मात्रा॒ सायु॑ज्य सलो॒कतां गमयन्ति। अथ॒ यत्स॑प्तद॒शेन॑ स्तु॒वते। स॒प्त॒द॒शेनै॒व तद्यज॑मान॒माद॑दते। त स॑प्तद॒शेनै॒व ह॑रन्ति॥४८॥

%1.5.10.6
याव॑ती सप्तद॒शस्य॒ मात्रा। प्र॒जाप॑ति॒र्वै स॑प्तद॒शः। प्र॒जाप॑तेरे॒वैनं॒ तत्। मात्रा॒ सायु॑ज्य सलो॒कतां गमयन्ति। अथ॒ यदे॑कवि॒शेन॑ स्तु॒वते। ए॒क॒वि॒शेनै॒व तद्यज॑मान॒माद॑दते। तमे॑कवि॒शेनै॒व ह॑रन्ति। याव॑त्येकवि॒शस्य॒ मात्रा। अ॒सौ वा आ॑दि॒त्य ए॑कवि॒शः। आ॒दि॒त्यस्यै॒वैनं॒ तत्॥४९॥

%1.5.10.7
मात्रा॒ सायु॑ज्य सलो॒कतां गमयन्ति। ते कु॒श्यौ। व्य॑घ्नन्। ते अ॑होरा॒त्रे अ॑भवताम्। अह॑रे॒व सु॒वर्णा॑ऽभवत्। र॒ज॒ता रात्रि॑। स यदा॑दि॒त्य उ॒देति॑। ए॒तामे॒व तत्सु॒वर्णां कु॒शीमनु॒ समे॑ति। अथ॒ यद॑स्त॒मेति॑। ए॒तामे॒व तद्र॑ज॒तां कु॒शीमनु॒संवि॑शति। प्र॒ह्रादो॑ ह॒ वै का॑याध॒वः। वि॒रोच॑न॒ स्वं पु॒त्रमुदास्यत्। स प्र॑द॒रो॑ऽभवत्। तस्मात्प्रद॒रादु॑द॒कं नाचा॑मेत्॥५०॥\anuvakamend[आ॒दि॒त्यः प॑ञ्चद॒शस्य॒ मात्रा स्तु॒वते॑ पञ्चद॒शेनै॒व तद्यज॑मान॒माद॑दते सप्तद॒शेनै॒व ह॑रन्त्यादि॒त्यस्यै॒वैनं॒ तद्वि॑शति च॒त्वारि॑ च]

%1.5.11.1
ये वै च॒त्वार॒ स्तोमा। कृ॒तन्तत्। अथ॒ ये पञ्च॑। कलि॒ सः। तस्मा॒च्चतु॑ष्टोमः। तच्चतु॑ष्टोमस्य चतुष्टोम॒त्वम्। तदा॑हुः। क॒त॒मानि॒ तानि॒ ज्योतीषि। य ए॒तस्य॒ स्तोमा॒ इति॑। त्रि॒वृत्प॑ञ्चद॒शः स॑प्तद॒श ए॑कवि॒शः॥५१॥

%1.5.11.2
ए॒तानि॒ वाव तानि॒ ज्योतीषि। य ए॒तस्य॒ स्तोमा। सोऽब्रवीत्। स॒प्त॒द॒शेन॑ ह्रि॒यमा॑णो॒ व्य॑लेशिषि। भि॒षज्य॑त॒ मेति॑। तम॒श्विनौ॑ धा॒नाभि॑रभिषज्यताम्। पू॒षा क॑र॒म्भेण॑। भार॑ती परिवा॒पेण॑। मि॒त्रावरु॑णौ पय॒स्य॑या। तदा॑हुः॥५२॥

%1.5.11.3
यद॒श्विभ्यान्धा॒नाः। पू॒ष्णः क॑र॒म्भः। भार॑त्यै परिवा॒पः। मि॒त्रावरु॑णयोः पय॒स्याऽथ॑। कस्मा॑दे॒तेषा ह॒विषा॒मिन्द्र॑मे॒व य॑ज॒न्तीति॑। ए॒ता ह्ये॑नं दे॒वता॒ इति॑ ब्रूयात्। ए॒तैर्\mbox{}ह॒विर्भि॒\-रभि॑षज्य॒स्तस्मा॒दिति॑। तं वस॑वो॒ऽष्टाक॑पालेन प्रातः सव॒ने॑ऽभिषज्यन्। रु॒द्रा एका॑दशकपालेन॒ माध्य॑न्दिने॒ सव॑ने। विश्वे॑ दे॒वा द्वाद॑शकपालेन तृतीयसव॒ने॥५३॥

%1.5.11.4
स यद॒ष्टाक॑पालान्प्रातः सव॒ने कु॒र्यात्। एका॑दशकपाला॒न्माध्यं॑ दिने॒ सव॑ने। द्वाद॑शकपालास्तृतीयसव॒ने। विलो॑म॒ तद्य॒ज्ञस्य॑ क्रियेत। एका॑दशकपालाने॒व प्रा॑तः सव॒ने कु॑र्यात्। एका॑दशकपाला॒न्माध्य॑न्दिने॒ सव॑ने। एका॑दशकपालास्तृतीयसव॒ने। य॒ज्ञस्य॑ सलोम॒त्वाय॑। तदा॑हुः। यद्वसू॑नां प्रातः सव॒नम्। रु॒द्राणां॒ माध्य॑न्दिन॒ सव॑नम्। विश्वे॑षां दे॒वानां तृतीयसव॒नम्। अथ॒ कस्मा॑दे॒तेषा ह॒विषा॒मिन्द्र॑मे॒व य॑ज॒न्तीति॑। ए॒ता ह्ये॑नं दे॒वता॒ इति॑ ब्रूयात्। ए॒तैर्\mbox{}ह॒विर्भि॒रभि॑षज्य॒ स्तस्मा॒दिति॑॥५४॥\anuvakamend[ए॒क॒वि॒श आ॑हुस्तृतीयसव॒ने प्रा॑तः सव॒नं पञ्च॑ च]

%1.5.12.1
तस्यावा॑चोऽवपा॒दाद॑बिभयुः। तमे॒तेषु॑ स॒प्तसु॒ छन्द॑ स्वश्रयन्। यदश्र॑यन्। तच्छ्रा॑य॒न्तीय॑स्य श्रायन्तीय॒त्वम्। यदवा॑रयन्। तद्वा॑रव॒न्तीय॑स्य वारवन्तीय॒त्वम्। तस्यावा॑च ए॒वाव॑पा॒दाद॑बिभयुः। तस्मा॑ ए॒तानि॑ स॒प्त च॑तुरुत्त॒राणि॒ छन्दा॒स्युपा॑दधुः। तेषा॒मति॒ त्रीण्य॑रिच्यन्त। न त्रीण्युद॑भवन्॥५५॥

%1.5.12.2
स बृ॑ह॒तीमे॒वास्पृ॑शत्। द्वाभ्या॑म॒क्षराभ्याम्। अ॒हो॒रा॒त्राभ्या॑मे॒व। तदा॑हुः। क॒त॒मा सा दे॒वाक्ष॑रा बृह॒ती। यस्या॒न्तत्प्र॒त्यति॑ष्ठत्। द्वाद॑श पौर्णमा॒स्य॑। द्वाद॒शाष्ट॑काः। द्वाद॑शामावा॒स्या। ए॒षा वाव सा दे॒वाक्ष॑रा बृह॒ती॥५६॥

%1.5.12.3
यस्या॒न्तत्प्र॒त्यति॑ष्ठ॒दिति॑। यानि॑ च॒ छन्दास्य॒त्यरि॑च्यन्त। यानि॑ च॒ नोदभ॑वन्। तानि॒ निर्वीर्याणि ही॒नान्य॑मन्यन्त। साऽब्र॑वीद्बृह॒ती। मामे॒व भू॒त्वा। मामुप॒ सश्र॑य॒तेति॑। च॒तुर्भि॑र॒क्षरै॑रनु॒ष्टुग्बृ॑ह॒तीन्नोद॑भवत्। च॒तुर्भि॑र॒क्षरै प॒ङ्क्तिर्बृ॑ह॒ती\-मत्य॑रिच्यत। तस्या॑मे॒तानि॑ च॒त्वार्य॒क्षराण्यप॒च्छिद्या॑\-दधात्॥५७॥

%1.5.12.4
ते बृ॑ह॒ती ए॒व भू॒त्वा। बृ॒ह॒तीमुप॒ सम॑श्रयताम्। अ॒ष्टा॒भि\-र॒क्षरै॑रु॒ष्णिग्बृ॑ह॒तीन्नोद॑भवत्। अ॒ष्टा॒भि\-र॒क्षरैस्त्रि॒ष्टुग्बृ॑ह॒ती\-मत्य॑\-रिच्यत। तस्या॑मे॒तान्य॒ष्टाव॒क्षराण्यप॒च्छिद्या॑\-दधात्। ते बृ॑ह॒ती ए॒व भू॒त्वा। बृ॒ह॒तीमुप॒ सम॑श्रयताम्। द्वा॒द॒शभि॑र॒क्षरैर्गाय॒त्री बृ॑ह॒तीन्नोद॑भवत्। द्वा॒द॒शभि॑र॒क्षरै॒र्जग॑ती बृह॒तीमत्य॑रिच्यत। तस्या॑मे॒तानि॒ द्वाद॑शा॒क्षराण्यप॒च्छिद्या॑\-दधात्॥५८॥

%1.5.12.5
ते बृ॑ह॒ती ए॒व भू॒त्वा। बृ॒ह॒तीमुप॒ सम॑श्रयताम्। सोऽब्रवीत्प्र॒जाप॑तिः। छन्दासि॒ रथो॑ मे भवत। यु॒ष्माभि॑र॒हमे॒तमध्वा॑न॒मनु॒ सञ्च॑रा॒णीति॑। तस्य॑ गाय॒त्री च॒ जग॑ती च प॒क्षाव॑भवताम्। उ॒ष्णिक्च॑ त्रि॒ष्टुप्च॒ प्रष्ट्यौ। अ॒नु॒ष्टुप्च॑ प॒ङ्क्तिश्च॒ धुर्यौ। बृ॒ह॒त्ये॑वोद्धिर॑भवत्। स ए॒तञ्छ॑न्दोर॒थमा॒स्थाय॑। ए॒तमध्वा॑न॒मनु॒ सम॑चरत्। ए॒त ह॒ वै छ॑न्दोर॒थमा॒स्थाय॑। ए॒तमध्वा॑न॒मनु॒ सञ्च॑रति। येनै॒ष ए॒तत्स॒ञ्चर॑ति। य ए॒वं वि॒द्वान्त्सोमे॑न॒ यज॑ते। य उ॑ चैनमे॒वं वेद॑॥५९॥\anuvakamend[अ॒भ॒व॒न्वाव सा दे॒वाक्ष॑रा बृह॒त्य॑दधा॒द्द्वाद॑शा॒क्षराण्यप॒च्छिद्या॑दधादा॒स्थाय॒ षट्च॑]






\prashnaend{अ॒ग्नेः कृत्ति॑का॒ यत्पुण्यं॑ दे॒वस्य॑ सवि॒तुर्ब्र॑ह्मवा॒दिन॒ कत्यृ॒तमे॒व दे॒वा वा आयु॑षः प्रा॒णमिन्द्रो॑ दधी॒चो दे॑वासु॒राः स प्र॒जाप॑ति॒ स स॑मु॒द्रो ये वै च॒त्वार॒स्तस्यावा॑चो॒ द्वाद॑श॥१२॥}{अ॒ग्नेः कृत्ति॑का देवगृ॒हा ऋ॒तमे॒वर्ध्यामे॒व ति॒स्रः परा॑ची॒र्ये वै च॒त्वारो॒ नव॑पञ्चा॒शत्॥५९॥}{अ॒ग्नेः कृत्ति॑का॒ य उ॑ चैनमे॒वं वेद॑॥}{हरि॑ ओम्॥}{इति श्रीकृष्णयजुर्वेदीयतैत्तिरीयब्राह्मणे प्रथमाष्टके पञ्चमः प्रपाठकः समाप्तः॥}
\clearpage
