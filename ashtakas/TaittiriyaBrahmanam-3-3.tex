\sect{तृतीयः प्रश्नः}
\setcounter{anuvakam}{0}
\dnsub{तैत्तिरीयब्राह्मणे तृतीयाष्टके तृतीयः प्रपाठकः}

%3.3.1.1
प्रत्यु॑ष्ट॒ रक्ष॒ प्रत्यु॑ष्टा॒ अरा॑तय॒ इत्या॑ह। रक्ष॑सा॒मप॑हत्यै। अ॒ग्नेर्व॒स्तेजि॑ष्ठेन॒ तेज॑सा॒ निष्ट॑पा॒मीत्या॑ह मेध्य॒त्वाय॑। स्रुच॒ संमार्ष्टि। स्रु॒वमग्रे। पुमासमे॒वाभ्य॒ सश्य॑ति मिथुन॒त्वाय॑। अथ॑ जु॒हूम्। अथो॑प॒भृतम्। अथ॑ ध्रु॒वाम्। अ॒सौ वै जु॒हूः॥१॥

%3.3.1.2
अ॒न्तरि॑क्षमुप॒भृत्। पृ॒थि॒वी ध्रु॒वा। इ॒मे वै लो॒काः स्रुच॑। वृष्टि॑ स॒म्मार्ज॑नानि। वृष्टि॒र्वा इ॒माल्लोँ॒कान॑नुपू॒र्वं क॑ल्पयति। ते तत॑ कॢ॒प्ताः समे॑धन्ते। समे॑धन्तेऽस्मा इ॒मे लो॒काः प्र॒जया॑ प॒शुभि॑। य ए॒वं वेद॑। यदि॑ का॒मये॑त॒ वर्‌षु॑कः प॒र्जन्य॑ स्या॒दिति॑। अ॒ग्र॒तः संमृ॑ज्यात्॥२॥

%3.3.1.3
वृष्टि॑मे॒व नि य॑च्छति। अ॒वा॒चीनाग्रा॒ हि वृष्टि॑। यदि॑ का॒मये॒ताव॑र्‌षुकः स्या॒दिति॑। मू॒ल॒तः संमृ॑ज्यात्। वृष्टि॑मे॒वोद्य॑च्छति। तदु॒ वा आ॑हुः। अ॒ग्र॒त ए॒वोपरि॑ष्टा॒त्संमृ॑ज्यात्। मू॒ल॒तो॑ऽधस्तात्। तद॑नुपू॒र्वं क॑ल्पते। वर्‌षु॑को भव॒तीति॑॥३॥

%3.3.1.4
प्राची॑मभ्या॒कारम्। अग्रै॑रन्तर॒तः। ए॒वमि॑व॒ ह्यन्न॑म॒द्यते। अथो॒ अग्रा॒द्वा ओष॑धीना॒मूर्जं॑ प्र॒जा उप॑जीवन्ति। ऊ॒र्ज ए॒वान्नाद्य॒स्याव॑रुद्ध्यै। अ॒धस्तात्प्र॒तीचीम्। द॒ण्डमु॑त्तम॒तः। मूले॑न॒ मूलं॒ प्रति॑ष्ठित्यै। तस्मा॑दर॒त्नौ प्राञ्च्यु॒परि॑ष्टा॒ल्लोमा॑नि। प्र॒त्यञ्च्य॒धस्तात्॥४॥

%3.3.1.5
स्रुग्घ्ये॑षा। प्रा॒णो वै स्रु॒वः। जु॒हूर्दक्षि॑णो॒ हस्त॑। उ॒प॒भृत्स॒व्यः। आ॒त्मा ध्रु॒वा। अन्न सं॒मार्ज॑नानि। मु॒ख॒तो वै प्रा॒णो॑ऽपा॒नो भू॒त्वा। आ॒त्मान॒मन्नं॑ प्र॒विश्य॑। बा॒ह्य॒तस्त॒नुव शुभयति। तस्मात्स्रु॒वमे॒वाग्रे॒ संमार्ष्टि। मु॒ख॒तो हि प्रा॒णो॑ऽपा॒नो भू॒त्वा। आ॒त्मान॒मन्न॑मावि॒शति॑। तौ प्रा॑णापा॒नौ। अव्य॑र्धुकः प्राणापा॒नाभ्यां भवति। य ए॒वं वेद॑॥५॥\anuvakamend[जु॒हूर्मृ॑ज्याद्भव॒तीति॑ प्र॒त्यञ्च्य॒धस्तान्मार्ष्टि॒ पञ्च॑ च]

%3.3.2.1
दि॒वः शिल्प॒मव॑ततम्। पृ॒थि॒व्याः क॒कुभि॑ श्रि॒तम्। तेन॑ व॒य स॒हस्र॑वल्‌शेन। स॒पत्नं॑ नाशयामसि॒ स्वाहेति॑ स्रुख्सं॒मार्ज॑नान्य॒ग्नौ प्र ह॑रति। आपो॒ वै द॒र्भाः। रू॒पमे॒वैषा॑मे॒तन्म॑हि॒मानं॒ व्याच॑ष्टे। अ॒नु॒ष्टुभ॒र्चा। आनु॑ष्टुभः प्र॒जाप॑तिः। प्रा॒जा॒प॒त्यो वे॒दः। वे॒दस्याग्र स्रुख्सं॒मार्ज॑नानि॥६॥

%3.3.2.2
स्वेनै॒वैना॑नि॒ छन्द॑सा। स्वया॑ दे॒वत॑या॒ सम॑र्धयति। अथो॒ ऋग्वाव योषा। द॒र्भो वृषा। तन्मि॑थु॒नम्। मि॒थु॒नमे॒वास्य॒ तद्य॒ज्ञे क॑रोति प्र॒जन॑नाय। प्रजा॑यते प्र॒जया॑ प॒शुभि॒र्यज॑मानः। तान्येके॒ वृथै॒वापास्यन्ति। तत्तथा॒ न का॒र्यम्। आर॑ब्धस्य य॒ज्ञिय॑स्य॒ कर्म॑ण॒ सवि॑दो॒हः॥७॥

%3.3.2.3
यद्ये॑नानि प॒शवो॑ऽभि॒ तिष्ठे॑युः। न तत्प॒शुभ्य॒ कम्। अ॒द्भिर्मार्जयि॒त्वोत्क॒रे न्य॑स्येत्। यद्वै य॒ज्ञिय॑स्य॒ कर्म॑णो॒ऽन्यत्राहु॑तीभ्यः सं॒तिष्ठ॑ते। उ॒त्क॒रो वाव तस्य॑ प्रति॒ष्ठा। ए॒ता हि तस्मै प्रति॒ष्ठां दे॒वाः स॒मभ॑रन्। यद॒द्भिर्मा॒र्जय॑ति। तेन॑ शा॒न्तम्। यदु॑त्क॒रे न्य॒स्यति॑। प्र॒ति॒ष्ठामे॒वैना॑नि॒ तद्ग॑मयति॥८॥

%3.3.2.4
प्रति॑तिष्ठति प्र॒जया॑ प॒शुभि॒र्यज॑मानः। अथो स्त॒म्बस्य॒ वा ए॒तद्रू॒पम्। यत्स्रु॑ख्सं॒मार्ज॑नानि। स्त॒म्ब॒शो वा ओष॑धयः। तासां जरत्क॒क्षे प॒शवो॒ न र॑मन्ते। अप्रि॑यो॒ ह्ये॑षां जरत्क॒क्षः। याव॑दप्रियो ह॒ वै ज॑रत्क॒क्षः प॑शू॒नाम्। ताव॑दप्रियः पशू॒नां भ॑वति। यस्यै॒तान्य॒न्यत्रा॒ग्नेर्दध॑ति। न॒व॒दाव्या॑सु॒ वा ओष॑धीषु प॒शवो॑ रमन्ते॥९॥

%3.3.2.5
न॒व॒दा॒वो ह्ये॑षां प्रि॒यः। याव॑त्प्रियो ह॒ वै न॑वदा॒वः प॑शू॒नाम्। ताव॑त्प्रियः पशू॒नां भ॑वति। यस्यै॒तान्य॒ग्नौ प्र॒हर॑न्ति। तस्मा॑दे॒तान्य॒ग्नावे॒व प्रह॑रेत्। य॒त॒रस्मिन्त्संमृ॒ज्यात्। प॒शू॒नां धृत्यै। यो भू॒ताना॒मधि॑पतिः। रु॒द्रस्त॑न्तिच॒रो वृषा। प॒शून॒स्माकं॒ मा हिसीः। ए॒तद॑स्तु हु॒तं तव॒ स्वाहेत्य॑ग्निसं॒मार्ज॑न्य॒ग्नौ प्रह॑रति। ए॒षा वा ए॒तेषां॒ योनि॑। ए॒षा प्र॑ति॒ष्ठा। स्वामे॒वैना॑नि॒ योनिम्। स्वां प्र॑ति॒ष्ठां ग॑मयति। प्रति॑तिष्ठति प्र॒जया॑ प॒शुभि॒र्यज॑मानः॥१०॥\anuvakamend[वे॒दस्याग्र स्रुख्सं॒मार्ज॑नानि विदो॒हो ग॑मयति प॒शवो॑ रमन्ते हिसी॒ष्षट् च॑]

%3.3.3.1
अय॑ज्ञो॒ वा ए॒षः। यो॑ऽप॒त्नीक॑। न प्र॒जाः प्रजा॑येरन्। पत्न्यन्वास्ते। य॒ज्ञमे॒वाक॑। प्र॒जानां प्र॒जन॑नाय। यत्तिष्ठ॑न्ती स॒न्नह्ये॑त। प्रि॒यं ज्ञा॒ति रु॑न्ध्यात्। आसी॑ना॒ सन्न॑ह्यते। आसी॑ना॒ ह्ये॑षा वी॒र्य॑॑ क॒रोति॑॥११॥

%3.3.3.2
यत्प॒श्चात्प्राच्य॒न्वासी॑त। अ॒नया॑ स॒मद॑न्दधीत। दे॒वानां॒ पत्नि॑या स॒मद॑न्दधीत। देशाद्दक्षिण॒त उदी॒च्यन्वास्ते। आ॒त्मनो॑ गोपी॒थाय॑। आ॒शासा॑ना सौमन॒समित्या॑ह। मेध्या॑मे॒वैना॒ङ्केव॑लीं कृ॒त्वा। आ॒शिषा॒ सम॑र्धयति। अ॒ग्नेरनु॑व्रता भू॒त्वा सन्न॑ह्ये सुकृ॒ताय॒ कमित्या॑ह। ए॒तद्वै पत्नि॑यै व्रतोप॒नय॑नम्॥१२॥

%3.3.3.3
तेनै॒वैनां व्र॒तमुप॑नयति। तस्मा॑दाहुः। यश्चै॒वं वेद॒ यश्च॒ न। योक्त्र॑मे॒व यु॑ते। यम॒न्वास्ते। तस्या॒मुष्मिल्लोँ॒के भ॑व॒तीति॒ योक्त्रे॑ण। यद्योक्त्रम्। स योग॑। यदास्ते। स क्षेम॑॥१३॥

%3.3.3.4
यो॒ग॒क्षे॒मस्य॒ कॢप्त्यै। यु॒क्तङ्क्रि॑याता आ॒शीः कामे॑ युज्याता॒ इति॑। आ॒शिष॒ समृ॑द्ध्यै। ग्र॒न्थिङ्ग्र॑थ्नाति। आ॒शिष॑ ए॒वास्यां॒ परि॑ गृह्णाति। पुमा॒न्॒ वै ग्र॒न्थिः। स्त्री पत्नी। तन्मि॑थु॒नम्। मि॒थु॒नमे॒वास्य॒ तद्य॒ज्ञे क॑रोति प्र॒जन॑नाय। प्र जा॑यते प्र॒जया॑ प॒शुभि॒र्यज॑मानः॥१४॥

%3.3.3.5
अथो॑ अ॒र्धो वा ए॒ष आ॒त्मन॑। यत्पत्नी। य॒ज्ञस्य॒ धृत्या॒ अशि॑थिलम्भावाय। सु॒प्र॒जस॑स्त्वा व॒य सु॒पत्नी॒रुप॑ सेदि॒मेत्या॑ह। य॒ज्ञमे॒व तन्मि॑थु॒नीक॑रोति। ऊ॒नेऽति॑रिक्तन्धीयाता॒ इति॒ प्रजात्यै। म॒ही॒नां पयो॒ऽस्योष॑धीना॒ रस॒ इत्या॑ह। रू॒पमे॒वास्यै॒तन्म॑हि॒मानं॒ व्याच॑ष्टे। तस्य॒ तेऽक्षी॑यमाणस्य॒ निर्व॑पामि देवय॒ज्याया॒ इत्या॑ह। आ॒शिष॑मे॒वैतामा शास्ते॥१५॥\anuvakamend[क॒रोति॑ व्रतोप॒नय॑नं॒ क्षेमो॒ यज॑मानः शास्ते]

%3.3.4.1
घृ॒तं च॒ वै मधु॑ च प्र॒जाप॑तिरासीत्। यतो॒ मध्वा॑सीत्। तत॑ प्र॒जा अ॑सृजत। तस्मा॒न्मधु॑षि प्र॒जन॑नमिवास्ति। तस्मा॒न्मधु॑षा॒ न प्रच॑रन्ति। या॒तया॑म॒ हि। आज्ये॑न॒ प्रच॑रन्ति। य॒ज्ञो वा आज्यम्। य॒ज्ञेनै॒व य॒ज्ञं प्रच॑र॒न्त्यया॑तयामत्वाय। पत्न्यवेक्षते॥१६॥

%3.3.4.2
मि॒थु॒न॒त्वाय॒ प्रजात्यै। यद्वै पत्नी॑ य॒ज्ञस्य॑ क॒रोति॑। मि॒थु॒नं तत्। अथो॒ पत्नि॑या ए॒वैष य॒ज्ञस्यान्वार॒म्भोऽन॑वच्छित्त्यै। अ॒मे॒ध्यं वा ए॒तत्क॑रोति। यत्पत्न्य॒वेक्ष॑ते। गार्‌ह॑प॒त्येऽधि॑ श्रयति मेध्य॒त्वाय॑। आ॒ह॒व॒नीय॑म॒भ्युद्द्र॑वति। य॒ज्ञस्य॒ सन्त॑त्यै। तेजो॑ऽसि॒ तेजोऽनु॒ प्रेहीत्या॑ह॥१७॥

%3.3.4.3
तेजो॒ वा अ॒ग्निः। तेज॒ आज्यम्। तेज॑सै॒व तेज॒ सम॑र्धयति। अ॒ग्निस्ते॒ तेजो॒ मा विनै॒दित्या॒हाहिसायै। स्प्यस्य॒ वर्त्मन्त्सादयति। य॒ज्ञस्य॒ सन्त॑त्यै। अ॒ग्नेर्जि॒ह्वाऽसि॑ सु॒भूर्दे॒वाना॒मित्या॑ह। य॒था॒य॒जुरे॒वैतत्। धाम्ने॑धाम्ने दे॒वेभ्यो॒ यजु॑षेयजुषे भ॒वेत्या॑ह। आ॒शिष॑मे॒वैतामा शास्ते॥१८॥

%3.3.4.4
तद्वा अत॑ प॒वित्राभ्यामे॒वोत्पु॑नाति। यज॑मानो॒ वा आज्यम्। प्रा॒णा॒पा॒नौ प॒वित्रे। यज॑मान ए॒व प्रा॑णापा॒नौ द॑धाति। पु॒न॒रा॒हारम्। ए॒वमि॑व॒ हि प्रा॑णापा॒नौ सं॒चर॑तः। शु॒क्रम॑सि॒ ज्योति॑रसि॒ तेजो॒ऽसीत्या॑ह। रू॒पमे॒वास्यै॒तन्म॑हि॒मानं॒ व्याच॑ष्टे। त्रिर्यजु॑षा। त्रय॑ इ॒मे लो॒काः॥१९॥

%3.3.4.5
ए॒षां लो॒काना॒माप्त्यै। त्रिः। त्र्या॑वृ॒द्धि य॒ज्ञः। अथो॑ मेध्य॒त्वाय॑। अथाज्य॑वतीभ्याम॒पः। रू॒पमे॒वासा॑मे॒तद्वर्णं॑ दधाति। अपि॒ वा उ॒ताहु॑। यथा॑ ह॒ वै योषा॑ सु॒वर्ण॒ हिर॑ण्यं पेश॒लं बिभ्र॑ती रू॒पाण्यास्ते। ए॒वमे॒ता ए॒तर्\mbox{}हीति॑। आपो॒ वै सर्वा दे॒वता॥२०॥

%3.3.4.6
ए॒षा हि विश्वे॑षां दे॒वानां त॒नूः। यदाज्यम्। तत्रो॒भयोर्मीमा॒सा। जा॒मि स्यात्। यद्यजु॒षाऽऽज्यं॒ यजु॑षा॒ऽप उ॑त्पुनी॒यात्। छन्द॑सा॒ऽप उत्पु॑ना॒त्यजा॑मित्वाय। अथो॑ मिथुन॒त्वाय॑। सा॒वि॒त्रि॒यर्चा। स॒वि॒तृप्र॑सूतं मे॒ कर्मा॑स॒दिति॑। स॒वि॒तृप्र॑सूतमे॒वास्य॒ कर्म॑ भवति। प॒च्छो गा॑यत्रि॒या त्रि॑ष्षमृद्ध॒त्वाय॑। अ॒द्भिरे॒वौष॑धी॒ सं न॑यति। ओष॑धीभिः प॒शून्। प॒शुभि॒र्यज॑मानम्। शु॒क्रं त्वा॑ शु॒क्रायां॒ ज्योति॑स्त्वा॒ ज्योति॑ष्य॒र्चिस्त्वा॒ऽर्चिषीत्या॑ह सर्व॒त्वाय॑। पर्याप्त्या॒ अन॑न्तरायाय॥२१॥\anuvakamend[ई॒क्ष॒त॒ आ॒ह॒ शा॒स्ते॒ लो॒का दे॒वता॑ भवति॒ षट् च॑]

%3.3.5.1
दे॒वा॒सु॒राः संय॑त्ता आसन्। स ए॒तमिन्द्र॒ आज्य॑स्यावका॒शम॑पश्यत्। तेनावैक्षत। ततो॑ दे॒वा अभ॑वन्। पराऽसु॑राः। य ए॒वं वि॒द्वानाज्य॑म॒वेक्ष॑ते। भव॑त्या॒त्मना। पराऽस्य॒ भ्रातृ॑व्यो भवति। ब्र॒ह्म॒वा॒दिनो॑ वदन्ति। यदाज्ये॑ना॒न्यानि॑ ह॒वीष्य॑भिघा॒रय॑ति॥२२॥

%3.3.5.2
अथ॒ केनाज्य॒मिति॑। स॒त्येनेति॑ ब्रूयात्। चक्षु॒र्वै स॒त्यम्। स॒त्येनै॒वैन॑द॒भि घा॑रयति। ई॒श्व॒रो वा ए॒षोऽन्धो भवि॑तोः। यश्चक्षु॒षाऽऽज्य॑म॒वेक्ष॑ते। नि॒मील्यावेक्षेत। दा॒धारा॒त्मञ्चक्षु॑। अ॒भ्याज्य॑ङ्घारयति। आज्य॑ङ्गृह्णाति॥२३॥

%3.3.5.3
छन्दासि॒ वा आज्यम्। छन्दास्ये॒व प्री॑णाति। च॒तुर्जु॒ह्वां गृ॑ह्णाति। चतु॑ष्पादः प॒शव॑। प॒शूने॒वाव॑ रुन्धे। अ॒ष्टावु॑प॒भृति॑। अ॒ष्टाक्ष॑रा गाय॒त्री। गा॒य॒त्रः प्रा॒णः। प्रा॒णमे॒व प॒शुषु॑ दधाति। च॒तुर्ध्रु॒वायाम्॥२४॥

%3.3.5.4
चतु॑ष्पादः प॒शव॑। प॒शुष्वे॒वोपरि॑ष्टा॒त्प्रति॑ तिष्ठति। य॒ज॒मा॒न॒दे॒व॒त्या॑ वै जु॒हूः। भ्रा॒तृ॒व्य॒दे॒व॒त्यो॑प॒भृत्। च॒तुर्जु॒ह्वाङ्गृ॒ह्णन्भूयो॑ गृह्णीयात्। अ॒ष्टावु॑प॒भृति॑ गृ॒ह्णन्कनी॑यः। यज॑मानायै॒व भ्रातृ॑व्य॒मुप॑स्तिं करोति। गौर्वै स्रुच॑। च॒तुर्जु॒ह्वां गृ॑ह्णाति। तस्मा॒च्चतु॑ष्पदी॥२५॥

%3.3.5.5
अ॒ष्टावु॑प॒भृति॑। तस्मा॑द॒ष्टाश॑फा। च॒तुर्ध्रु॒वायाम्। तस्मा॒च्चतु॑ स्तना। गामे॒व तत्सस्क॑रोति। सास्मै॒ सस्कृ॒तेष॒मूर्ज॑न्दुहे। यज्जु॒ह्वाङ्गृ॒ह्णाति॑। प्र॒या॒जेभ्य॒स्तत्। यदु॑प॒भृति॑। प्र॒या॒जा॒नू॒या॒जेभ्य॒स्तत्। सर्व॑स्मै॒ वा ए॒तद्य॒ज्ञाय॑ गृह्यते। य॒द्ध्रु॒वाया॒माज्यम्॥२६॥\anuvakamend[अ॒भि॒घा॒रय॑ति गृह्णाति ध्रु॒वाया॒ञ्चतु॑ष्पदी प्रयाजानूया॒जेभ्य॒स्तद्द्वे च॑]

%3.3.6.1
आपो॑ देवीरग्रेपुवो अग्रेगुव॒ इत्या॑ह। रू॒पमे॒वासा॑मे॒तन्म॑हि॒मान॒व्व्याँच॑ष्टे। अग्र॑ इ॒मं य॒ज्ञन्न॑य॒ताग्रे॑ य॒ज्ञप॑ति॒मित्या॑ह। अग्र॑ ए॒व य॒ज्ञन्न॑यन्ति। अग्रे॑ य॒ज्ञप॑तिम्। यु॒ष्मानिन्द्रो॑ऽवृणीत वृत्र॒तूर्ये॑ यू॒यमिन्द्र॑मवृणीध्वं वृत्र॒तूर्य॒ इत्या॑ह। वृ॒त्र ह॑ हनि॒ष्यन्निन्द्र॒ आपो॑ वव्रे। आपो॒ हेन्द्रं॑ वव्रिरे। सं॒ज्ञामे॒वासा॑मे॒तत्सामा॑न॒व्व्याँच॑ष्टे। प्रोक्षि॑ता॒ स्थेत्या॑ह॥२७॥

%3.3.6.2
तेनाप॒ प्रोक्षि॑ताः। अ॒ग्निर्दे॒वेभ्यो॒ निला॑यत। कृष्णो॑ रू॒पं कृ॒त्वा। स वन॒स्पती॒न्प्रावि॑शत्। कृष्णोऽस्याखरे॒ष्ठोऽग्नये त्वा॒ स्वाहेत्या॑ह। अ॒ग्नय॑ ए॒वैनं॒ जुष्टं॑ करोति। अथो॑ अ॒ग्नेरे॒व मेध॒मव॑ रुन्धे। वेदि॑रसि ब॒र्॒हिषे त्वा॒ स्वाहेत्या॑ह। प्र॒जा वै ब॒र्॒हिः। पृ॒थि॒वी वेदि॑॥२८॥

%3.3.6.3
प्र॒जा ए॒व पृ॑थि॒व्यां प्रति॑ष्ठापयति। ब॒र्॒हिर॑सि स्रु॒ग्भ्यस्त्वा॒ स्वाहेत्या॑ह। प्र॒जा वै ब॒र्॒हिः। यज॑मान॒ स्रुच॑। यज॑मानमे॒व प्र॒जासु॒ प्रति॑ष्ठापयति। दि॒वे त्वा॒ऽन्तरि॑क्षाय त्वा पृथि॒व्यै त्वेति॑ ब॒र्॒हिरा॒साद्य॒ प्रोक्ष॑ति। ए॒भ्य ए॒वैन॑ल्लो॒केभ्य॒ प्रोक्ष॑ति। अथ॒ तत॑ स॒ह स्रु॒चा पु॒रस्तात्प्र॒त्यञ्च॑ङ्ग्र॒न्थिं प्रत्यु॑क्षति। प्र॒जा वै ब॒र्॒हिः। यथा॒ सूत्यै॑ का॒ल आप॑ पु॒रस्ता॒द्यन्ति॑॥२९॥

%3.3.6.4
ता॒दृगे॒व तत्। स्व॒धा पि॒तृभ्य॒ इत्या॑ह। स्व॒धा॒का॒रो हि पि॑तृ॒णाम्। ऊर्ग्भ॑व बर्‌हि॒षद्भ्य॒ इति॒ दक्षि॑णायै॒ श्रोणे॒रोत्त॑रस्यै॒ निन॑यति॒ सन्त॑त्यै। मासा॒ वै पि॒तरो॑ बर्‌हि॒षद॑। मासा॑ने॒व प्री॑णाति। मासा॒ वा ओष॑धीर्व॒र्धय॑न्ति। मासा पचन्ति॒ समृ॑द्ध्यै। अन॑तिस्कन्दन् ह प॒र्जन्यो॑ वर्‌षति। यत्रै॒तदे॒वङ्क्रि॒यते॥३०॥

%3.3.6.5
ऊ॒र्जा पृ॑थि॒वीङ्ग॑च्छ॒तेत्या॑ह। पृ॒थि॒व्यामे॒वोर्ज॑न्दधाति। तस्मात्पृथि॒व्या ऊ॒र्जा भु॑ञ्जते। ग्र॒न्थिं वि स्रसयति। प्रज॑नयत्ये॒व तत्। ऊ॒र्ध्वं प्राञ्च॒मुद्गू॑ढं प्र॒त्यञ्च॒मा य॑च्छति। तस्मात्प्रा॒चीन॒ रेतो॑ धीयते। प्र॒तीची प्र॒जा जा॑यन्ते। विष्णो॒ स्तूपो॒ऽसीत्या॑ह। य॒ज्ञो वै विष्णु॑॥३१॥

%3.3.6.6
य॒ज्ञस्य॒ धृत्यै। पु॒रस्तात्प्रस्त॒रं गृ॑ह्णाति। मुख्य॑मे॒वैनं॑ करोति। इय॑न्तङ्गृह्णाति। प्र॒जाप॑तिना यज्ञमु॒खेन॒ सम्मि॑तम्। इय॑न्तङ्गृह्णाति। य॒ज्ञ॒प॒रुषा॒ सम्मि॑तम्। इय॑न्तङ्गृह्णाति। ए॒ताव॒द्वै पुरु॑षे वी॒र्यम्। वी॒र्य॑संमितम्॥३२॥

%3.3.6.7
अप॑रिमितङ्गृह्णाति। अप॑रिमित॒स्याव॑रुद्ध्यै। तस्मि॑न्प॒वित्रे॒ अपि॑ सृजति। यज॑मानो॒ वै प्र॑स्त॒रः। प्रा॒णा॒पा॒नौ प॒वित्रे। यज॑मान ए॒व प्रा॑णापा॒नौ द॑धाति। ऊर्णाम्रदसन्त्वा स्तृणा॒मीत्या॑ह। य॒था॒य॒जुरे॒वैतत्। स्वा॒स॒स्थन्दे॒वेभ्य॒ इत्या॑ह। दे॒वेभ्य॑ ए॒वैन॑त्स्वास॒स्थं क॑रोति॥३३॥

%3.3.6.8
ब॒र्॒हिः स्तृ॑णाति। प्र॒जा वै ब॒र्॒हिः। पृ॒थि॒वी वेदि॑। प्र॒जा ए॒व पृ॑थि॒व्यां प्रति॑ष्ठापयति। अन॑तिदृश्ञ स्तृणाति। प्र॒जयै॒वैनं॑ प॒शुभि॒रन॑तिदृश्ञं करोति। धा॒रय॑न्प्रस्त॒रं प॑रि॒धीन्परि॑ दधाति। यज॑मानो॒ वै प्र॑स्त॒रः। यज॑मान ए॒व तत्स्व॒यं प॑रि॒धीन्परि॑ दधाति। ग॒न्ध॒र्वो॑ऽसि वि॒श्वाव॑सु॒रित्या॑ह॥३४॥

%3.3.6.9
विश्व॑मे॒वायु॒र्यज॑माने दधाति। इन्द्र॑स्य बा॒हुर॑सि॒ दक्षि॑ण॒ इत्या॑ह। इ॒न्द्रि॒यमे॒व यज॑माने दधाति। मि॒त्रावरु॑णौ त्वोत्तर॒तः परि॑धत्ता॒मित्या॑ह। प्रा॒णा॒पा॒नौ मि॒त्रावरु॑णौ। प्रा॒णा॒पा॒नावे॒वास्मि॑न्दधाति। सूर्य॑स्त्वा पु॒रस्तात् पा॒त्वित्या॑ह। रक्ष॑सा॒मप॑हत्यै। कस्याश्चिद॒भिश॑स्त्या॒ इत्या॑ह। अप॑रिमितादे॒वैनं॑ पाति॥३५॥

%3.3.6.10
वी॒तिहोत्रन्त्वा कव॒ इत्या॑ह। अ॒ग्निमे॒व हो॒त्रेण॒ सम॑र्धयति। द्यु॒मन्त॒ समि॑धीम॒हीत्या॑ह॒ समि॑द्ध्यै। अग्ने॑ बृ॒हन्त॑मध्व॒र इत्या॑ह॒ वृद्ध्यै। वि॒शो य॒न्त्रे स्थ॒ इत्या॑ह। वि॒शां यत्यै। उ॒दी॒चीनाग्रे॒ नि द॑धाति॒ प्रति॑ष्ठित्यै। वसू॑ना रु॒द्राणा॑मादि॒त्याना॒ सद॑सि सी॒देत्या॑ह। दे॒वता॑नामे॒व सद॑ने प्रस्त॒र सा॑दयति। जु॒हूर॑सि घृ॒ताची॒ नाम्नेत्या॑ह॥३६॥

%3.3.6.11
अ॒सौ वै जु॒हूः। अ॒न्तरि॑क्षमुप॒भृत्। पृ॒थि॒वी ध्रु॒वा। तासा॑मे॒तदे॒व प्रि॒यन्नाम॑। यद्घृ॒ताचीति॑। यद्घृ॒ताचीत्याह॑। प्रि॒येणै॒वैना॒ नाम्ना॑ सादयति। ए॒ता अ॑सदन्त्सुकृ॒तस्य॑ लो॒क इत्या॑ह। स॒त्यं वै सु॑कृ॒तस्य॑ लो॒कः। स॒त्य ए॒वैना सुकृ॒तस्य॑ लो॒के सा॑दयति। ता वि॑ष्णो पा॒हीत्या॑ह। य॒ज्ञो वै विष्णु॑। य॒ज्ञस्य॒ धृत्यै। पा॒हि य॒ज्ञं पा॒हि य॒ज्ञप॑तिं पा॒हि मां य॑ज्ञ॒निय॒मित्या॑ह। य॒ज्ञाय॒ यज॑मानाया॒त्मने। तेभ्य॑ ए॒वाशिष॒माशा॒स्तेऽनार्त्यै॥३७॥\anuvakamend[स्थेत्या॑ह पृथि॒वी वेदि॒र्यन्ति॑ क्रि॒यते॒ वीणु॑र्वी॒र्य॑सम्मितं करोत्याह पाति॒ नाम्नेत्या॑ह लो॒के सा॑दयति॒ षट् च॑]

%3.3.7.1
अ॒ग्निना॒ वै होत्रा। दे॒वा असु॑रान॒भ्य॑भवन्। अ॒ग्नये॑ समि॒ध्यमा॑ना॒यानु॑ब्रू॒हीत्या॑ह॒ भ्रातृ॑व्याभिभूत्यै। एक॑विशतिमिध्मदा॒रूणि॑ भवन्ति। ए॒क॒वि॒शो वै पुरु॑षः। पुरु॑ष॒स्याप्त्यै। पञ्च॑दशेध्मदा॒रूण्य॒भ्या द॑धाति। पञ्च॑दश॒ वा अ॑र्धमा॒सस्य॒ रात्र॑यः। अ॒र्ध॒मा॒स॒शः सं॑वत्स॒र आप्यते। त्रीन्प॑रि॒धीन्परि॑ दधाति॥३८॥

%3.3.7.2
ऊ॒र्ध्वे स॒मिधा॒वा द॑धाति। अ॒नू॒या॒जेभ्य॑ स॒मिध॒मति॑ शिनष्टि। षट्त्संप॑द्यन्ते। षड्वा ऋ॒तव॑। ऋ॒तूने॒व प्री॑णाति। वे॒देनोप॑ वाजयति। प्रा॒जा॒प॒त्यो वै वे॒दः। प्रा॒जा॒प॒त्यः प्रा॒णः। यज॑मान आहव॒नीय॑। यज॑मान ए॒व प्रा॒णन्द॑धाति॥३९॥

%3.3.7.3
त्रिरुप॑ वाजयति। त्रयो॒ वै प्रा॒णाः। प्रा॒णाने॒वास्मि॑न्दधाति। वे॒देनो॑प॒यत्य॑ स्रु॒वेण॑ प्राजाप॒त्यमा॑घा॒रमा घा॑रयति। य॒ज्ञो वै प्र॒जाप॑तिः। य॒ज्ञमे॒व प्र॒जाप॑तिं मुख॒त आर॑भते। अथो प्र॒जाप॑ति॒ सर्वा॑ दे॒वता। सर्वा॑ ए॒व दे॒वता प्रीणाति। अ॒ग्निम॑ग्नी॒त्रिस्त्रि॒ सं मृ॒ड्ढीत्या॑ह। त्र्या॑वृ॒द्धि य॒ज्ञः॥४०॥

%3.3.7.4
अथो॒ रक्ष॑सा॒मप॑हत्यै। प॒रि॒धीन्त्सं मार्ष्टि। पु॒नात्ये॒वैनान्॑। त्रिस्त्रि॒ सं मार्ष्टि। त्र्या॑वृ॒द्धि य॒ज्ञः। अथो॑ मेध्य॒त्वाय॑। अथो॑ ए॒ते वै दे॑वा॒श्वाः। दे॒वा॒श्वाने॒व तत्सं मार्ष्टि। सु॒व॒र्गस्य॑ लो॒कस्य॒ सम॑ष्ट्यै। आसी॑नो॒ऽन्यमा॑घा॒रमा घा॑रयति॥४१॥

%3.3.7.5
तिष्ठ॑न्न॒न्यम्। यथाऽनो॑ वा॒ रथं॑ वा यु॒ञ्ज्यात्। ए॒वमे॒व तद॑ध्व॒र्युर्य॒ज्ञं यु॑नक्ति। सु॒व॒र्गस्य॑ लो॒कस्या॒भ्यूढ्यै। वह॑न्त्येनङ्ग्रा॒म्याः प॒शव॑। य ए॒वं वेद॑। भुव॑नमसि॒ वि प्र॑थ॒स्वेत्या॑ह। य॒ज्ञो वै भुव॑नम्। य॒ज्ञ ए॒व यज॑मानं प्र॒जया॑ प॒शुभि॑ प्रथयति। अग्ने॒ यष्ट॑रि॒दन्नम॒ इत्या॑ह॥४२॥

%3.3.7.6
अ॒ग्निर्वै दे॒वानां॒ यष्टा। य ए॒व दे॒वानां॒ यष्टा। तस्मा॑ ए॒व नम॑स्करोति। जुह्वेह्य॒ग्निस्त्वा ह्वयति देवय॒ज्याया॒ उप॑भृ॒देहि॑ दे॒वस्त्वा॑ सवि॒ता ह्व॑यति देवय॒ज्याया॒ इत्या॑ह। आ॒ग्ने॒यी वै जु॒हूः। सा॒वि॒त्र्यु॑प॒भृत्। ताभ्या॑मे॒वैने॒ प्रसू॑त॒ आद॑त्ते। अग्ना॑विष्णू॒ मा वा॒मव॑ क्रमिष॒मित्या॑ह। अ॒ग्निः पु॒रस्तात्। विष्णु॑र्य॒ज्ञः प॒श्चात्॥४३॥

%3.3.7.7
ताभ्या॑मे॒व प्र॑ति॒प्रोच्या॒त्या क्रा॑मति। विजि॑हाथां॒ मा मा॒ सन्ताप्त॒मित्या॒हाहिसायै। लो॒कं मे॑ लोककृतौ कृणुत॒मित्या॑ह। आ॒शिष॑मे॒वैतामा शास्ते। विष्णो॒ स्थान॑म॒सीत्या॑ह। य॒ज्ञो वै विष्णु॑। ए॒तत्खलु॒ वै दे॒वाना॒मप॑राजितमा॒यत॑नम्। यद्य॒ज्ञः। दे॒वाना॑मे॒वाप॑राजित आ॒यत॑ने तिष्ठति। इ॒त इन्द्रो॑ अकृणोद्वी॒र्या॑णीत्या॑ह॥४४॥

%3.3.7.8
इ॒न्द्रि॒यमे॒व यज॑माने दधाति। स॒मा॒रभ्यो॒र्ध्वो अ॑ध्व॒रो दि॑वि॒स्पृश॒मित्या॑ह॒ वृद्ध्यै। आ॒घा॒रमा॑घा॒र्यमा॑ण॒मनु॑ समा॒रभ्य॑। ए॒तस्मि॑न्का॒ले दे॒वाः सु॑व॒र्गं लो॒कमा॑यन्। सा॒क्षादे॒व यज॑मानः सुव॒र्गं लो॒कमे॑ति। अथो॒ समृ॑द्धेनै॒व य॒ज्ञेन॒ यज॑मानः सुव॒र्गं लो॒कमे॑ति। अह्रु॑तो य॒ज्ञो य॒ज्ञप॑ते॒रित्या॒हानार्त्यै। इन्द्रा॑वा॒न्त्स्वाहेत्या॑ह। इ॒न्द्रि॒यमे॒व यज॑माने दधाति। बृ॒हद्भा इत्या॑ह॥४५॥

%3.3.7.9
सु॒व॒र्गो वै लो॒को बृ॒हद्भाः। सु॒व॒र्गस्य॑ लो॒कस्य॒ सम॑ष्ट्यै। य॒ज॒मा॒न॒दे॒व॒त्या॑ वै जु॒हूः। भ्रा॒तृ॒व्य॒दे॒व॒त्यो॑प॒भृत्। प्रा॒ण आ॑घा॒रः। यत्सस्प॒र्॒शयेत्। भ्रातृ॑व्येऽस्य प्रा॒णन्द॑ध्यात्। असस्पर्‌शयन्न॒त्या क्रा॑मति। यज॑मान ए॒व प्रा॒णन्द॑धाति। पा॒हि माऽग्ने॒ दुश्च॑रिता॒दा मा॒ सुच॑रिते भ॒जेत्या॑ह॥४६॥

%3.3.7.10
अ॒ग्निर्वाव प॒वित्रम्। वृ॒जि॒नमनृ॑त॒न्दुश्च॑रितम्। ऋ॒जु॒क॒र्म स॒त्य सुच॑रितम्। अ॒ग्निरे॒वैनं॑ वृजि॒नादनृ॑ता॒द्दुश्च॑रितात्पाति। ऋ॒जु॒क॒र्मे स॒त्ये सुच॑रिते भजति। तस्मा॑दे॒वमा शास्ते। आ॒त्मनो॑ गोपी॒थाय॑। शिरो॒ वा ए॒तद्य॒ज्ञस्य॑। यदा॑घा॒रः। आ॒त्मा ध्रु॒वा॥४७॥

%3.3.7.11
आ॒घा॒रमा॒घार्य॑ ध्रु॒वा सम॑नक्ति। आ॒त्मन्ने॒व य॒ज्ञस्य॒ शिर॒ प्रति॑ दधाति। द्विः सम॑नक्ति। द्वौ हि प्रा॑णापा॒नौ। तदा॑हुः। त्रिरे॒व सम॑ञ्ज्यात्। त्रिधा॑तु॒ हि शिर॒ इति॑। शिर॑ इवै॒तद्य॒ज्ञस्य॑। अथो॒ त्रयो॒ वै प्रा॒णाः। प्रा॒णाने॒वास्मि॑न्दधाति। म॒खस्य॒ शिरो॑ऽसि॒ सञ्ज्योति॑षा॒ ज्योति॑रङ्क्ता॒मित्या॑ह। ज्योति॑रे॒वास्मा॑ उ॒परि॑ष्टाद्दधाति। सु॒व॒र्गस्य॑ लो॒कस्यानु॑ख्यात्यै॥४८॥\anuvakamend[परि॑दधाति प्रा॒णन्द॑धाति॒ हि य॒ज्ञो घा॑रयति॒ नम॒ इत्या॑ह प॒श्चाद्वी॒र्या॑णीत्या॑ह॒ भा इत्या॑ह भ॒जेत्या॑ह ध्रु॒वैवास्मि॑न्दधाति॒ त्रीणि॑ च]

%3.3.8.1
धिष्णि॑या॒ वा ए॒ते न्यु॑प्यन्ते। यद्ब्र॒ह्मा। यद्धोता। यद॑ध्व॒र्युः। यद॒ग्नीत्। यद्यज॑मानः। तान् यद॑न्तरे॒यात्। यज॑मानस्य प्रा॒णान्त्सङ्क॑र्‌षेत्। प्र॒मायु॑कः स्यात्। पु॒रो॒डाश॑मप॒गृह्य॒ सञ्च॑रत्यध्व॒र्युः॥४९॥

%3.3.8.2
यज॑मानायै॒व तल्लो॒क शिषति। नास्य॑ प्रा॒णान्त्सङ्क॑र्‌षति। न प्र॒मायु॑को भवति। पु॒रस्तात् प्र॒त्यङ्ङासी॑नः। इडा॑या॒ इडा॒मा द॑धाति। हस्त्या॒ होत्रे। प॒शवो॒ वा इडा। प॒शव॒ पुरु॑षः। प॒शुष्वे॒व प॒शून्प्रति॑ष्ठापयति। इडा॑यै॒ वा ए॒षा प्रजा॑तिः॥५०॥

%3.3.8.3
तां प्रजा॑तिं॒ यज॑मा॒नोऽनु॒ प्र जा॑यते। द्विर॒ङ्गुला॑वनक्ति॒ पर्व॑णोः। द्वि॒पाद्यज॑मान॒ प्रति॑ष्ठित्यै। स॒कृदुप॑ स्तृणाति। द्विरा द॑धाति। स॒कृद॒भि घा॑रयति। च॒तुः संप॑द्यते। च॒त्वारि॒ वै प॒शोः प्र॑ति॒ष्ठाना॑नि। यावा॑ने॒व प॒शुः। तमुप॑ह्वयते॥५१॥

%3.3.8.4
मुख॑मिव॒ प्रत्युप॑ह्वयेत। सं॒मु॒खाने॒व प॒शूनुप॑ ह्वयते। प॒शवो॒ वा इडा। तस्मा॒त्साऽन्वा॒रभ्या। अ॒ध्व॒र्युणा॑ च॒ यज॑मानेन च। उप॑हूतः पशु॒मान॑सा॒नीत्या॑ह। उप॒ ह्ये॑नौ॒ ह्वय॑ते॒ होता। इडा॑यै दे॒वता॑नामुपह॒वे। उप॑हूतः पशु॒मान्भ॑वति। य ए॒वं वेद॑॥५२॥

%3.3.8.5
यां वै हस्त्या॒मिडा॑मा॒दधा॑ति। वा॒चः सा भा॑ग॒धेयम्। यामु॑प॒ह्वय॑ते। प्रा॒णाना॒ सा। वाचं॑ चै॒व प्रा॒णाश्चाव॑ रुन्धे। अथ॒ वा ए॒तर्ह्युप॑हूताया॒मिडा॑याम्। पु॒रो॒डाश॑स्यै॒व ब॑र्‌हि॒षदो॑ मीमा॒सा। यज॑मानन्दे॒वा अ॑ब्रुवन्। ह॒विर्नो॒ निर्व॒पेति॑। नाहम॑भा॒गो निर्व॑प्स्या॒मीत्य॑ब्रवीत्॥५३॥

%3.3.8.6
न मया॑ऽभा॒गयाऽनु॑वक्ष्य॒थेति॒ वाग॑ब्रवीत्। नाहम॑भा॒गा पु॑रोनुवा॒क्या॑ भविष्या॒मीति॑ पुरोनुवा॒क्या। नाहम॑भा॒गा या॒ज्या॑ भविष्या॒मीति॑ या॒ज्या। न मया॑ऽभा॒गेन॒ वष॑ट्करिष्य॒थेति॑ वषट्का॒रः। यद्य॑जमानभा॒गन्नि॒धाय॑ पुरो॒डाशं॑ बर्‌हि॒षदं॑ क॒रोति॑। ताने॒व तद्भा॒गिन॑ करोति। च॒तु॒र्धा क॑रोति। चत॑स्रो॒ दिश॑। दि॒क्ष्वे॑व प्रति॑तिष्ठति। ब॒र्॒हि॒षदं॑ करोति॥५४॥

%3.3.8.7
यज॑मानो॒ वै पु॑रो॒डाश॑। प्र॒जा ब॒र्॒हिः। यज॑मानमे॒व प्र॒जासु॒ प्रति॑ष्ठापयति। तस्मा॑द॒स्थ्नाऽन्याः प्र॒जाः प्र॑ति॒तिष्ठ॑न्ति। मा॒सेना॒न्याः। अथो॒ खल्वा॑हुः। दक्षि॑णा॒ वा ए॒ता ह॑विर्य॒ज्ञस्यान्तर्वे॒द्यव॑ रुध्यन्ते। यत्पु॑रो॒डाशं॑ बर्‌हि॒षदं॑ क॒रोतीति॑। च॒तु॒र्धा क॑रोति। च॒त्वारो॒ ह्ये॑ते ह॑विर्य॒ज्ञस्य॒र्त्विज॑॥५५॥

%3.3.8.8
ब्र॒ह्मा होताऽध्व॒र्युर॒ग्नीत्। तम॒भि मृ॑शेत्। इ॒दं ब्र॒ह्मण॑। इ॒द होतु॑। इ॒दम॑ध्व॒र्योः। इ॒दम॒ग्नीध॒ इति॑। यथै॒वादः सौ॒म्येऽध्व॒रे। आ॒देश॑मृ॒त्विग्भ्यो॒ दक्षि॑णा नी॒यन्ते। ता॒दृगे॒व तत्। अ॒ग्नीधे प्रथ॒माया द॑धाति॥५६॥

%3.3.8.9
अ॒ग्निमु॑खा॒ ह्यृद्धि॑। अ॒ग्निमु॑खामे॒वर्द्धिं॒ यज॑मान ऋध्नोति। स॒कृदु॑प॒स्तीर्य॒ द्विरा॒दध॑त्। उ॒प॒स्तीर्य॒ द्विर॒भि घा॑रयति। षट्त्संप॑द्यन्ते। षड्वा ऋ॒तव॑। ऋ॒तूने॒व प्री॑णाति। वे॒देन॑ ब्र॒ह्मणे ब्रह्मभा॒गं परि॑हरति। प्रा॒जा॒प॒त्यो वै वे॒दः। प्रा॒जा॒प॒त्यो ब्र॒ह्मा॥५७॥

%3.3.8.10
स॒वि॒ता य॒ज्ञस्य॒ प्रसूत्यै। अथ॒ काम॑म॒न्येन॑। ततो॒ होत्रे। मध्यं॒ वा ए॒तद्य॒ज्ञस्य॑। यद्धोता। म॒ध्य॒त ए॒व य॒ज्ञं प्री॑णाति। अथाध्व॒र्यवे। प्र॒ति॒ष्ठा वा ए॒षा य॒ज्ञस्य॑। यद॑ध्व॒र्युः। तस्माद्धविर्य॒ज्ञस्यै॒तामे॒वावृत॒मनु॑॥५८॥

%3.3.8.11
अ॒न्या दक्षि॑णा नीयन्ते। य॒ज्ञस्य॒ प्रति॑ष्ठित्यै। अ॒ग्निम॑ग्नीत्स॒कृत्स॑कृ॒त्सं मृ॒ड्ढीत्या॑ह। परा॑ङिव॒ ह्ये॑तर्‌हि॑ य॒ज्ञः। इ॒षि॒ता दैव्या॒ होता॑र॒ इत्या॑ह। इ॒षि॒त हि कर्म॑ क्रि॒यते। भ॒द्र॒वाच्या॑य॒ प्रेषि॑तो॒ मानु॑षः सूक्तवा॒काय॑ सू॒क्ता ब्रू॒हीत्या॑ह। आ॒शिष॑मे॒वैतामा शास्ते। स्व॒गा दैव्या॒ होतृ॑भ्य॒ इत्या॑ह। य॒ज्ञमे॒व तत्स्व॒गा क॑रोति। स्व॒स्तिर्मानु॑षेभ्य॒ इत्या॑ह। आ॒शिष॑मे॒वैतामा शास्ते। श॒य्योँर्ब्रू॒हीत्या॑ह। श॒य्युँमे॒व बा॑र्‌हस्प॒त्यं भा॑ग॒धेये॑न॒ सम॑र्धयति॥५९॥\anuvakamend[च॒र॒त्य॒ध्व॒र्युः प्रजा॑तिर्ह्वयते॒ वेदाब्रवीद्बर्‌हि॒षदं॑ करोत्यृ॒त्विजो॑ दधाति ब्र॒ह्माऽनु॑करोति च॒त्वारि॑ च]

%3.3.9.1
अथ॒ स्रुचा॑वनु॒ष्टुग्भ्यां॒ वाज॑वतीभ्या॒व्व्यूँ॑हति। प्र॒ति॒ष्ठा वा अ॑नु॒ष्टुक्। अन्नं॒ वाज॒ प्रति॑ष्ठित्यै। अ॒न्नाद्य॒स्याव॑रुद्ध्यै। प्राचीं जु॒हूमू॑हति। जा॒ताने॒व भ्रातृ॑व्या॒न्प्रणु॑दते। प्र॒तीची॑मुप॒भृतम्। ज॒नि॒ष्यमा॑णाने॒व प्रति॑नुदते। सविषू॑च ए॒वापोह्य॑ स॒पत्ना॒न्॒ यज॑मानः। अ॒स्मिल्लोँ॒के प्रति॑तिष्ठति॥६०॥

%3.3.9.2
द्वाभ्याम्। द्विप्र॑तिष्ठो॒ हि। वसु॑भ्यस्त्वा रु॒द्रेभ्य॑स्त्वाऽऽदि॒त्येभ्य॒स्त्वेत्या॑ह। य॒था॒य॒जुरे॒वैतत्। स्रु॒क्षु प्र॑स्त॒रम॑नक्ति। इ॒मे वै लो॒काः स्रुच॑। यज॑मानः प्रस्त॒रः। यज॑मानमे॒व तेज॑साऽनक्ति। त्रे॒धाऽन॑क्ति। त्रय॑ इ॒मे लो॒काः॥६१॥

%3.3.9.3
ए॒भ्य ए॒वैनं॑ लो॒केभ्यो॑ऽनक्ति। अ॒भि॒पू॒र्वम॑नक्ति। अ॒भि॒पू॒र्वमे॒व यज॑मान॒न्तेज॑साऽनक्ति। अ॒क्त रिहा॑णा॒ इत्या॑ह। तेजो॒ वा आज्यम्। यज॑मानः प्रस्त॒रः। यज॑मानमे॒व तेज॑साऽनक्ति। वि॒यन्तु॒ वय॒ इत्या॑ह। वय॑ ए॒वैनं॑ कृ॒त्वा। सु॒व॒र्गं लो॒कं ग॑मयति॥६२॥

%3.3.9.4
प्र॒जाय्योँनिं॒ मा निर्मृ॑क्ष॒मित्या॑ह। प्र॒जायै॑ गोपी॒थाय॑। आप्या॑यन्ता॒माप॒ ओष॑धय॒ इत्या॑ह। आप॑ ए॒वौष॑धी॒रा प्या॑ययति। म॒रुतां॒ पृष॑तय॒ स्थेत्या॑ह। म॒रुतो॒ वै वृष्ट्या॑ ईशते। वृष्टि॑मे॒वाव॑ रुन्धे। दिवं॑ गच्छ॒ ततो॑ नो॒ वृष्टि॒मेर॒येत्या॑ह। वृष्टि॒र्वै द्यौः। वृष्टि॑मे॒वाव॑ रुन्धे॥६३॥

%3.3.9.5
याव॒द्वा अ॑ध्व॒र्युः प्र॑स्त॒रं प्र॒हर॑ति। ताव॑द॒स्यायु॑र्मीयते। आ॒यु॒ष्पा अ॑ग्ने॒ऽस्यायु॑र्मे पा॒हीत्या॑ह। आयु॑रे॒वात्मन्ध॑त्ते। याव॒द्वा अ॑ध्व॒र्युः प्र॑स्त॒रं प्र॒हर॑ति। ताव॑दस्य॒ चक्षु॑र्मीयते। च॒क्षु॒ष्पा अ॑ग्नेऽसि॒ चक्षु॑र्मे पा॒हीत्या॑ह। चक्षु॑रे॒वात्मन्ध॑त्ते। ध्रु॒वाऽसीत्या॑ह॒ प्रति॑ष्ठित्यै। यं प॑रि॒धिं प॒र्यध॑त्था॒ इत्या॑ह॥६४॥

%3.3.9.6
य॒था॒य॒जुरे॒वैतत्। अग्ने॑ देव प॒णिभि॑र्वी॒यमा॑ण॒ इत्या॑ह। अ॒ग्नय॑ ए॒वैनं॒ जुष्टं॑ करोति। तन्त॑ ए॒तमनु॒ जोषं॑ भरा॒मीत्या॑ह। स॒जा॒ताने॒वास्मा॒ अनु॑कान्करोति। नेदे॒ष त्वद॑पचे॒तया॑ता॒ इत्या॒हानु॑ख्यात्यै। य॒ज्ञस्य॒ पाथ॒ उप॒ समि॑त॒मित्या॑ह। भू॒मान॑मे॒वोपै॑ति। प॒रि॒धीन्प्र ह॑रति। य॒ज्ञस्य॒ समि॑ष्ट्यै॥६५॥

%3.3.9.7
स्रुचौ॒ सं प्रस्रा॑वयति। यदे॒व तत्र॑ क्रू॒रम्। तत्तेन॑ शमयति। जु॒ह्वामु॑प॒भृतम्। य॒ज॒मा॒न॒दे॒व॒त्या॑ वै जु॒हूः। भ्रा॒तृ॒व्य॒दे॒व॒त्यो॑प॒भृत्। यज॑मानायै॒व भ्रातृ॑व्य॒मुप॑स्तिं करोति। स॒स्रा॒वभा॑गा॒ स्थेत्या॑ह। वस॑वो॒ वै रु॒द्रा आ॑दि॒त्याः सस्रा॒वभा॑गाः। तेषा॒न्तद्भा॑ग॒धेयम्॥६६॥

%3.3.9.8
ताने॒व तेन॑ प्रीणाति। वै॒श्व॒दे॒व्यर्चा। ए॒ते हि विश्वे॑ दे॒वाः। त्रि॒ष्टुग्भ॑वति। इ॒न्द्रि॒यं वै त्रि॒ष्टुक्। इ॒न्द्रि॒यमे॒व यज॑माने दधाति। अ॒ग्नेर्वा॒मप॑न्नगृहस्य॒ सद॑सि सादया॒मीत्या॑ह। इ॒यं वा अ॒ग्निरप॑न्नगृहः। अ॒स्या ए॒वैने॒ सद॑ने सादयति। सु॒म्नाय॑ सुम्निनी सु॒म्ने मा॑ धत्त॒मित्या॑ह॥६७॥

%3.3.9.9
प्र॒जा वै प॒शव॑ सु॒म्नम्। प्र॒जामे॒व प॒शूना॒त्मन्ध॑त्ते। धु॒रि धु॒र्यौ॑ पात॒मित्या॑ह। जा॒या॒प॒त्योर्गो॑पी॒थाय॑। अग्ने॑ऽदब्धायोऽशीततनो॒ इत्या॑ह। य॒था॒य॒जुरे॒वैतत्। पा॒हि मा॒ऽद्य दि॒वः पा॒हि प्रसि॑त्यै पा॒हि दुरि॑ष्ट्यै पा॒हि दु॑रद्म॒न्यै पा॒हि दुश्च॑रिता॒दित्या॑ह। आ॒शिष॑मे॒वैतामा शास्ते। अवि॑षन्नः पि॒तुं कृ॑णु सु॒षदा॒ योनि॒ स्वाहेतीध्मसं॒वृश्च॑नान्यन्वाहार्य॒पच॑नेऽभ्या॒धाय॑ फलीकरणहो॒मं जु॑होति। अति॑रिक्तानि॒ वा इ॑ध्मसं॒ वृश्च॑नानि॥६८॥

%3.3.9.10
अति॑रिक्ताः फली॒कर॑णाः। अति॑रिक्तमाज्योच्छेष॒णम्। अति॑रिक्त ए॒वाति॑रिक्तन्दधाति। अथो॒ अति॑रिक्तेनै॒वाति॑रिक्तमा॒प्त्वाऽव॑ रुन्धे। वेदि॑र्दे॒वेभ्यो॒ निला॑यत। तां वे॒देनान्व॑विन्दन्। वे॒देन॒ वेदिं॑ विविदुः पृथि॒वीम्। सा प॑प्रथे पृथि॒वी पार्थि॑वानि। गर्भं॑ बिभर्ति॒ भुव॑नेष्व॒न्तः। ततो॑ य॒ज्ञो जा॑यते विश्व॒दानि॒रिति॑ पु॒रस्तात्स्तम्बय॒जुषो॑ वे॒देन॒ वेदि॒ संमा॒र्ष्ट्यनु॑वित्त्यै॥६९॥

%3.3.9.11
अथो॒ यद्वे॒दश्च॒ वेदि॑श्च॒ भव॑तः। मि॒थु॒न॒त्वाय॒ प्रजात्यै। प्र॒जाप॑ते॒र्वा ए॒तानि॒ श्मश्रू॑णि। यद्वे॒दः। पत्नि॑या उ॒पस्थ॒ आस्य॑ति। मि॒थु॒नमे॒व क॑रोति। वि॒न्दते प्र॒जाम्। वे॒द होताऽऽह॑व॒नीयात्स्तृ॒णन्ने॑ति। य॒ज्ञमे॒व तत्सन्त॑नो॒त्योत्त॑रस्मादर्धमा॒सात्। त सन्त॑त॒मुत्त॑रेऽर्धमा॒स आल॑भते॥७०॥

%3.3.9.12
तङ्का॒लेका॑ल॒ आग॑ते यजते। ब्र॒ह्म॒वा॒दिनो॑ वदन्ति। स त्वा अ॑ध्व॒र्युः स्यात्। यो यतो॑ य॒ज्ञं प्र॑यु॒ङ्क्ते। तदे॑नं प्रतिष्ठा॒पय॒तीति॑। वाता॒द्वा अ॑ध्व॒र्युर्य॒ज्ञं प्रयु॑ङ्क्ते। देवा॑ गातुविदो गा॒तुं वि॒त्वा गा॒तुमि॒तेत्या॑ह। यत॑ ए॒व य॒ज्ञं प्र॑यु॒ङ्क्ते। तदे॑नं॒ प्रति॑ष्ठापयति। प्रति॑ तिष्ठति प्र॒जया॑ प॒शुभि॒र्यज॑मानः॥७१॥\anuvakamend[ति॒ष्ठ॒ती॒मे लो॒का ग॑मयति॒ द्यौर्वृष्टि॑मे॒वाव॑रुन्धे प॒र्यध॑त्था॒ इत्या॑ह॒ समि॑ष्ट्यै भाग॒धेय॑न्धत्त॒मित्या॑ह॒ वा इ॑ध्मसं॒ वृश्च॑ना॒न्यनु॑वित्त्यै लभते॒ यज॑मानः]

%3.3.10.1
यो वा अय॑थादेवतं य॒ज्ञमु॑प॒चर॑ति। आ दे॒वताभ्यो वृश्च्यते। पापी॑यान्भवति। यो य॑थादेव॒तम्। न दे॒वताभ्य॒ आवृ॑श्च्यते। वसी॑यान्भवति। वा॒रु॒णो वै पाश॑। इ॒मं विष्या॑मि॒ वरु॑णस्य॒ पाश॒मित्या॑ह। व॒रु॒ण॒पा॒शादे॒वैनां मुञ्चति। स॒वि॒तृप्र॑सूतो यथादेव॒तम्॥७२॥

%3.3.10.2
न दे॒वताभ्य॒ आवृ॑श्च्यते। वसी॑यान्भवति। धा॒तुश्च॒ योनौ॑ सुकृ॒तस्य॑ लो॒क इत्या॑ह। अ॒ग्निर्वै धा॒ता। पुण्य॒ङ्कर्म॑ सुकृ॒तस्य॑ लो॒कः। अ॒ग्निरे॒वैनान्धा॒ता। पुण्ये॒ कर्म॑णि सुकृ॒तस्य॑ लो॒के द॑धाति। स्यो॒नं मे॑ स॒ह पत्या॑ करो॒मीत्या॑ह। आ॒त्मन॑श्च॒ यज॑मानस्य॒ चानात्यै स॒न्त्वाय॑। समायु॑षा॒ सं प्र॒जयेत्या॑ह॥७३॥

%3.3.10.3
आ॒शिष॑मे॒वैतामा शास्ते पूर्णपा॒त्रे। अ॒न्त॒तो॑ऽनु॒ष्टुभा। चतु॑ष्प॒द्वा ए॒तच्छन्द॒ प्रति॑ष्ठितं॒ पत्नि॑यै पूर्णपा॒त्रे भ॑वति। अ॒स्मिल्लोँ॒के प्रति॑तिष्ठा॒नीति॑। अ॒स्मिन्ने॒व लो॒के प्रति॑तिष्ठति। अथो॒ वाग्वा अ॑नु॒ष्टुक्। वाङ्मि॑थु॒नम्। आपो॒ रेत॑ प्र॒जन॑नम्। ए॒तस्मा॒द्वै मि॑थु॒नाद्वि॒द्योत॑मानः स्त॒नय॑न्वर्‌षति। रेत॑ सि॒ञ्चन्॥७४॥

%3.3.10.4
प्र॒जाः प्र॑ज॒नय\sn{}। यद्वै य॒ज्ञस्य॒ ब्रह्म॑णा यु॒ज्यते। ब्रह्म॑णा॒ वै तस्य॑ विमो॒कः। अ॒द्भिः शान्ति॑। विमु॑क्तं॒ वा ए॒तर्‌हि॒ योक्त्रं॒ ब्रह्म॑णा। आ॒दायै॑न॒त्पत्नी॑ स॒हाप उप॑गृह्णीते॒ शान्त्यै। अ॒ञ्ज॒लौ पूर्णपा॒त्रमा न॑यति। रेत॑ ए॒वास्यां प्र॒जान्द॑धाति। प्र॒जया॒ हि म॑नु॒ष्य॑ पू॒र्णः। मुखं॒ वि मृ॑ष्टे। अ॒व॒भृ॒थस्यै॒व रू॒पं कृ॒त्वोत्ति॑ष्ठति॥७५॥\anuvakamend[स॒वि॒तृप्र॑सूतो यथादेव॒तं प्र॒जयेत्या॑ह सि॒ञ्चन्मृ॑ष्ट॒ एकं च]

%3.3.11.1
प॒रि॒वे॒षो वा ए॒ष वन॒स्पती॑नाम्। यदु॑पवे॒षः। य ए॒वं वेद॑। वि॒न्दते॑ परिवे॒ष्टारम्। तमु॑त्क॒रे। यन्दे॒वा म॑नु॒ष्ये॑षु। उ॒प॒वे॒षमधा॑रयन्। ये अ॒स्मदप॑ चेतसः। तान॒स्मभ्य॑मि॒हा कु॑रु। उप॑वे॒षोप॑ विड्ढि नः॥७६॥

%3.3.11.2
प्र॒जां पुष्टि॒मथो॒ धनम्। द्वि॒पदो॑ न॒श्चतु॑ष्पदः। ध्रु॒वानन॑पगान्कु॒र्विति॑ पु॒रस्तात्प्र॒त्यञ्च॒मुप॑ गूहति। तस्मात्पु॒रस्तात्प्र॒त्यञ्च॑ शू॒द्रा अव॑स्यन्ति। स्थ॒वि॒म॒त उप॑गूहति। अप्र॑तिवादिन ए॒वैनान्कुरुते। धृष्टि॒र्वा उ॑पवे॒षः। शु॒चर्तो वज्रो॒ ब्रह्म॑णा॒ सशि॑तः। योप॑वे॒षे शुक्। साऽमुमृ॑च्छतु॒ यं द्वि॒ष्म इति॑॥७७॥

%3.3.11.3
अथास्मै नाम॒ गृह्य॒ प्रह॑रति। निर॒मुन्नु॑द॒ ओक॑सः। स॒पत्नो॒ यः पृ॑त॒न्यति॑। नि॒र्बा॒ध्ये॑न ह॒विषा। इन्द्र॑ एणं॒ परा॑शरीत्। इ॒हि ति॒स्रः प॑रा॒वत॑। इ॒हि पञ्च॒ जना॒ अति॑। इ॒हि ति॒स्रोऽति॑ रोच॒नायाव॑त्। सूर्यो॒ अस॑द्दि॒वि। प॒र॒मान्त्वा॑ परा॒वतम्॥७८॥

%3.3.11.4
इन्द्रो॑ नयतु वृत्र॒हा। यतो॒ न पुन॒राय॑सि। श॒श्व॒तीभ्य॒ समाभ्य॒ इति॑। त्रि॒वृद्वा ए॒ष वज्रो॒ ब्रह्म॑णा॒ सशि॑तः। शु॒चैवैनं॑ वि॒ध्वा। ए॒भ्यो लो॒केभ्यो॑ नि॒र्णुद्य॑। वज्रे॑ण॒ ब्रह्म॑णा स्तृणुते। ह॒तो॑ऽसावव॑धिष्मा॒मुमित्या॑ह॒ स्तृत्यै। यं द्वि॒ष्यात्तन्ध्या॑येत्। शु॒चैवैन॑मर्पयति॥७९॥




\prashnaend{प्रत्यु॑ष्टन्दि॒वः शिल्प॒मय॑ज्ञो घृ॒तं च॑ देवासु॒राः स ए॒तमिन्द्र आपो॑ देवीर॒ग्निना॒ धिष्णि॑या॒ अथ॒ स्रुचौ॒ यो वा अय॑थादेवतं परिवे॒षो वा एका॑दश॥११॥}{प्रत्यु॑ष्ट॒मय॑ज्ञ ए॒षा हि विश्वे॑षान्दे॒वाना॑मू॒र्जा पृ॑थि॒वीमथो॒ रक्ष॑सा॒न्तां प्रजा॑ति॒न्द्वाभ्या॒न्तङ्का॒लेका॑ले॒ नव॑सप्ततिः॥७९॥}{प्रत्यु॑ष्टमर्पयति॥}{हरि॑ ओम्॥}{इति श्रीकृष्णयजुर्वेदीयतैत्तिरीयब्राह्मणे तृतीयाष्टके तृतीयः प्रपाठकः समाप्तः॥}
\clearpage
