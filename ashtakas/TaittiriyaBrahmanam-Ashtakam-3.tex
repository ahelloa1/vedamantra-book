\chapt{अष्टकम् ३}
\sect{प्रथमः प्रश्नः}
\setcounter{anuvakam}{0}
\dnsub{तैत्तिरीयब्राह्मणे तृतीयाष्टके प्रथमः प्रपाठकः}

%3.1.1.1
अ॒ग्निर्न॑ पातु॒ कृत्ति॑काः। नक्ष॑त्रन्दे॒वमि॑न्द्रि॒यम्। इ॒दमा॑सां विचक्ष॒णम्। ह॒विरा॒सं जु॑होतन। यस्य॒ भान्ति॑ र॒श्मयो॒ यस्य के॒तव॑। यस्ये॒मा विश्वा॒ भुव॑नानि॒ सर्वा। स कृत्ति॑काभिर॒भि सं॒वसा॑नः। अ॒ग्निर्नो॑ दे॒वः सु॑वि॒ते द॑धातु। प्र॒जाप॑ते रोहि॒णी वे॑तु॒ पत्नी। वि॒श्वरू॑पा बृह॒ती चि॒त्रभा॑नुः॥१॥

%3.1.1.2
सा नो॑ य॒ज्ञस्य॑ सुवि॒ते द॑धातु। यथा॒ जीवे॑म श॒रद॒ सवी॑राः। रो॒हि॒णी दे॒व्युद॑गात्पु॒रस्तात्। विश्वा॑ रू॒पाणि॑ प्रति॒मोद॑माना। प्र॒जाप॑ति ह॒विषा॑ व॒र्धय॑न्ती। प्रि॒या दे॒वाना॒मुप॑यातु य॒ज्ञम्। सोमो॒ राजा॑ मृगशी॒र्॒षेण॒ आग\sn{}। शि॒वं नक्ष॑त्रं प्रि॒यम॑स्य॒ धाम॑। आ॒प्याय॑मानो बहु॒धा जने॑षु। रेत॑ प्र॒जां यज॑माने दधातु॥२॥

%3.1.1.3
यत्ते॒ न॑क्षत्रं मृगशी॒र्‌षमस्ति॑। प्रि॒य रा॑जन्प्रि॒यत॑मं प्रि॒याणाम्। तस्मै॑ ते सोम ह॒विषा॑ विधेम। शन्न॑ एधि द्वि॒पदे॒ शञ्चतु॑ष्पदे। आ॒र्द्रया॑ रु॒द्रः प्रथ॑मान एति। श्रेष्ठो॑ दे॒वानां॒ पति॑रघ्नि॒यानाम्। नक्ष॑त्रमस्य ह॒विषा॑ विधेम। मा न॑ प्र॒जा री॑रिष॒न्मोत वी॒रान्। हे॒ती रु॒द्रस्य॒ परि॑ णो वृणक्तु। आ॒र्द्रा नक्ष॑त्रं जुषता ह॒विर्न॑॥३॥

%3.1.1.4
प्र॒मु॒ञ्चमा॑नौ दुरि॒तानि॒ विश्वा। अपा॒घशसन्नुदता॒मरा॑तिम्। पुन॑र्नो दे॒व्यदि॑तिः स्पृणोतु। पुन॑र्वसू न॒ पुन॒रेतां य॒ज्ञम्। पुन॑र्नो दे॒वा अ॒भिय॑न्तु॒ सर्वे। पुन॑ पुनर्वो ह॒विषा॑ यजामः। ए॒वा न दे॒व्यदि॑तिरन॒र्वा। विश्व॑स्य भ॒र्त्री जग॑तः प्रति॒ष्ठा। पुन॑र्वसू ह॒विषा॑ व॒र्धय॑न्ती। प्रि॒यन्दे॒वाना॒मप्ये॑तु॒ पाथ॑॥४॥

%3.1.1.5
बृह॒स्पति॑ प्रथ॒मं जाय॑मानः। ति॒ष्यं नक्ष॑त्रम॒भिसम्ब॑भूव। श्रेष्ठो॑ दे॒वानां॒ पृत॑नासु जि॒ष्णुः। दिशोऽनु॒ सर्वा॒ अभ॑यन्नो अस्तु। ति॒ष्य॑ पु॒रस्ता॑दु॒त म॑ध्य॒तो न॑। बृह॒स्पति॑र्न॒ परि॑ पातु प॒श्चात्। बाधे॑ता॒न्द्वेषो॒ अभ॑यङ्कृणुताम्। सु॒वीर्य॑स्य॒ पत॑यः स्याम। इ॒द स॒र्पेभ्यो॑ ह॒विर॑स्तु॒ जुष्टम्। आ॒श्रे॒षा येषा॑मनु॒यन्ति॒ चेत॑॥५॥

%3.1.1.6
ये अ॒न्तरि॑क्षं पृथि॒वीङ्क्षि॒यन्ति॑। ते न॑ स॒र्पासो॒ हव॒माग॑मिष्ठाः। ये रो॑च॒ने सूर्य॒स्यापि॑ स॒र्पाः। ये दिवं॑ दे॒वीमनु॑ स॒ञ्चर॑न्ति। येषा॑माश्रे॒षा अ॑नु॒यन्ति॒ कामम्। तेभ्य॑ स॒र्पेभ्यो॒ मधु॑मज्जुहोमि। उप॑हूताः पि॒तरो॒ ये म॒घासु॑। मनो॑जवसः सु॒कृत॑ सुकृ॒त्याः। ते नो॒ नक्ष॑त्रे॒ हव॒माग॑मिष्ठाः। स्व॒धाभि॑र्य॒ज्ञं प्रय॑तं जुषन्ताम्॥६॥

%3.1.1.7
ये अ॑ग्निद॒ग्धा येऽन॑ग्निदग्धाः। ये॑ऽमुं लो॒कं पि॒तर॑ क्षि॒यन्ति॑। याश्च॑ वि॒द्म या उ॑ च॒ न प्र॑वि॒द्म। म॒घासु॑ य॒ज्ञ सुकृ॑तं जुषन्ताम्। गवां॒ पति॒ फल्गु॑नीनामसि॒ त्वम्। तद॑र्यमन्वरुण मित्र॒ चारु॑। तन्त्वा॑ व॒य स॑नि॒तार सनी॒नाम्। जी॒वा जीव॑न्त॒मुप॒ संवि॑शेम। येने॒मा विश्वा॒ भुव॑नानि॒ सञ्जि॑ता। यस्य॑ दे॒वा अ॑नु सं॒ यन्ति॒ चेत॑॥७॥

%3.1.1.8
अ॒र्य॒मा राजा॒ऽजर॒स्तुवि॑ष्मान्। फल्गु॑नीनामृष॒भो रो॑रवीति। श्रेष्ठो॑ दे॒वानां भगवो भगासि। तत्त्वा॑ विदु॒ फल्गु॑नी॒स्तस्य॑ वित्तात्। अ॒स्मभ्यं॑ क्ष॒त्रम॒जर सु॒वीर्यम्। गोम॒दश्व॑व॒दुप॒ सन्नु॑दे॒ह। भगो॑ ह दा॒ता भग॒ इत्प्र॑दा॒ता। भगो॑ दे॒वीः फल्गु॑नी॒रा वि॑वेश। भग॒स्येत्तं प्र॑स॒वं ग॑मेम। यत्र॑ दे॒वैः स॑ध॒मादं॑ मदेम॥८॥

%3.1.1.9
आया॑तु दे॒वः स॑वि॒तोप॑यातु। हि॒र॒ण्यये॑न सु॒वृता॒ रथे॑न। वह॒न्॒ हस्त सु॒भगं॑ विद्म॒नाप॑सम्। प्र॒यच्छ॑न्तं॒ पपु॑रिं॒ पुण्य॒मच्छ॑। हस्त॒ प्रय॑च्छत्व॒मृतं॒ वसी॑यः। दक्षि॑णेन॒ प्रति॑गृभ्णीम एनत्। दा॒तार॑म॒द्य स॑वि॒ता वि॑देय। यो नो॒ हस्ता॑य प्रसु॒वाति॑ य॒ज्ञम्। त्वष्टा॒ नक्ष॑त्रम॒भ्ये॑ति चि॒त्राम्। सु॒भ स॑सं युव॒ति रोच॑मानाम्॥९॥

%3.1.1.10
नि॒वे॒शय॑न्न॒मृता॒न्मर्त्याश्च। रू॒पाणि॑ पि॒शन्भुव॑नानि॒ विश्वा। तन्न॒स्त्वष्टा॒ तदु॑ चि॒त्रा विच॑ष्टाम्। तन्नक्ष॑त्रं भूरि॒दा अ॑स्तु॒ मह्यम्। तन्न॑ प्र॒जां वी॒रव॑ती सनोतु। गोभि॑र्नो॒ अश्वै॒ सम॑नक्तु य॒ज्ञम्। वा॒युर्नक्ष॑त्रम॒भ्ये॑ति॒ निष्ट्याम्। ति॒ग्मशृ॑ङ्गो वृष॒भो रोरु॑वाणः। स॒मी॒रय॒न्भुव॑ना मात॒रिश्वा। अप॒ द्वेषासि नुदता॒मरा॑तीः॥१०॥

%3.1.1.11
तन्नो॑ वा॒युस्तदु॒ निष्ट्या॑ शृणोतु। तन्नक्ष॑त्रं भूरि॒दा अ॑स्तु॒ मह्यम्। तन्नो॑ दे॒वासो॒ अनु॑जानन्तु॒ कामम्। यथा॒ तरे॑म दुरि॒तानि॒ विश्वा। दू॒रम॒स्मच्छत्र॑वो यन्तु भी॒ताः। तदि॑न्द्रा॒ग्नी कृ॑णुता॒न्तद्विशा॑खे। तन्नो॑ दे॒वा अनु॑मदन्तु य॒ज्ञम्। प॒श्चात्पु॒रस्ता॒दभ॑यन्नो अस्तु। नक्ष॑त्राणा॒मधि॑पत्नी॒ विशा॑खे। श्रेष्ठा॑विन्द्रा॒ग्नी भुव॑नस्य गो॒पौ॥११॥

%3.1.1.12
विषू॑च॒ शत्रू॑नप॒ बाध॑मानौ। अप॒ क्षुध॑न्नुदता॒मरा॑तिम्। पू॒र्णा प॒श्चादु॒त पू॒र्णा पु॒रस्तात्। उन्म॑ध्य॒तः पौर्णमा॒सी जि॑गाय। तस्यान्दे॒वा अधि॑ सं॒वस॑न्तः। उ॒त्त॒मे नाक॑ इ॒ह मा॑दयन्ताम्। पृ॒थ्वी सु॒वर्चा॑ युव॒तिः स॒जोषा। पौ॒र्ण॒मा॒स्युद॑गा॒च्छोभ॑माना। आ॒प्या॒यय॑न्ती दुरि॒तानि॒ विश्वा। उ॒रुन्दुहां॒ यज॑मानाय य॒ज्ञम्॥१२॥\anuvakamend[चि॒त्रभा॑नु॒र्यज॑माने दधातु ह॒विर्न॒ पाथ॒श्चेतो॑ जुषन्ता॒ञ्चेतो॑ मदेम॒ रोच॑माना॒मरा॑तीर्गो॒पौ य॒ज्ञम्]

%3.1.2.1
ऋ॒द्ध्यास्म॑ ह॒व्यैर्नम॑सोप॒सद्य॑। मि॒त्रन्दे॒वं मि॑त्र॒धेय॑न्नो अस्तु। अ॒नू॒रा॒धान् ह॒विषा॑ व॒र्धय॑न्तः। श॒तञ्जी॑वेम श॒रद॒ सवी॑राः। चि॒त्रं नक्ष॑त्र॒मुद॑गात्पु॒रस्तात्। अ॒नू॒रा॒धास॒ इति॒ यद्वद॑न्ति। तन्मि॒त्र ए॑ति प॒थिभि॑र्देव॒यानै। हि॒र॒ण्ययै॒र्वित॑तैर॒न्तरि॑क्षे। इन्द्रो ज्ये॒ष्ठामनु॒ नक्ष॑त्रमेति। यस्मि॑न्वृ॒त्रं वृ॑त्र॒तूर्ये॑ त॒तार॑॥१३॥

%3.1.2.2
तस्मि॑न्व॒यम॒मृत॒न्दुहा॑नाः। क्षुध॑न्तरेम॒ दुरि॑ति॒न्दुरि॑ष्टिम्। पु॒र॒न्द॒राय॑ वृष॒भाय॑ धृ॒ष्णवे। अषा॑ढाय॒ सह॑मानाय मी॒ढुषे। इन्द्रा॑य ज्ये॒ष्ठा मधु॑म॒द्दुहा॑ना। उ॒रुं कृ॑णोतु॒ यज॑मानाय लो॒कम्। मूलं॑ प्र॒जां वी॒रव॑तीं विदेय। पराच्येतु॒ निर्‌ऋ॑तिः परा॒चा। गोभि॒र्नक्ष॑त्रं प॒शुभि॒ सम॑क्तम्। अह॑र्भूया॒द्यज॑मानाय॒ मह्यम्॥१४॥

%3.1.2.3
अह॑र्नो अ॒द्य सु॑वि॒ते द॑धातु। मूलं॒ नक्ष॑त्र॒मिति॒ यद्वद॑न्ति। परा॑चीं वा॒चा निर्‌ऋ॑तिन्नुदामि। शि॒वं प्र॒जायै॑ शि॒वम॑स्तु॒ मह्यम्। या दि॒व्या आप॒ पय॑सा सम्बभू॒वुः। या अ॒न्तरि॑क्ष उ॒त पार्थि॑वी॒र्याः। यासा॑मषा॒ढा अ॑नु॒यन्ति॒ कामम्। ता न॒ आप॒ श स्यो॒ना भ॑वन्तु। याश्च॒ कूप्या॒ याश्च॑ ना॒द्या समु॒द्रिया। याश्च॑ वैश॒न्तीरु॒त प्रा॑स॒चीर्याः॥१५॥

%3.1.2.4
यासा॑मषा॒ढा मधु॑ भ॒क्षय॑न्ति॒। ता न॒ आप॒ श स्यो॒ना भ॑वन्तु। तन्नो॒ विश्वे॒ उप॑ शृण्वन्तु दे॒वाः। तद॑षा॒ढा अ॒भिसंय॑न्तु य॒ज्ञम्। तन्नक्ष॑त्रं प्रथतां प॒शुभ्य॑। कृ॒षिर्वृ॒ष्टिर्यज॑मानाय कल्पताम्। शु॒भ्राः क॒न्या॑ युव॒तय॑ सु॒पेश॑सः। क॒र्म॒कृत॑ सु॒कृतो॑ वी॒र्या॑वतीः। विश्वान्दे॒वान् ह॒विषा॑ व॒र्धय॑न्तीः। अ॒षा॒ढाः काम॒मुप॑ यान्तु य॒ज्ञम्॥१६॥

%3.1.2.5
यस्मि॒न्ब्रह्मा॒ऽभ्यज॑य॒त्सर्व॑मे॒तत्। अ॒मुं च॑ लो॒कमि॒दमू॑ च॒ सर्वम्। तन्नो॒ नक्ष॑त्रमभि॒जिद्वि॒जित्य॑। श्रिय॑न्दधा॒त्वहृ॑णीयमानम्। उ॒भौ लो॒कौ ब्रह्म॑णा॒ सञ्जि॑ते॒मौ। तन्नो॒ नक्ष॑त्रमभि॒जिद्विच॑ष्टाम्। तस्मि॑न्व॒यं पृत॑ना॒ सञ्ज॑येम। तन्नो॑ दे॒वासो॒ अनु॑जानन्तु॒ कामम्। शृ॒ण्वन्ति॑ श्रो॒णाम॒मृत॑स्य गो॒पाम्। पुण्या॑मस्या॒ उप॑शृणोमि॒ वाचम्॥१७॥

%3.1.2.6
म॒हीं देवीं विष्णु॑पत्नीमजू॒र्याम्। प्र॒तीची॑मेना ह॒विषा॑ यजामः। त्रे॒धा विष्णु॑रुरुगा॒यो विच॑क्रमे। म॒हीन्दिवं॑ पृथि॒वीम॒न्तरि॑क्षम्। तच्छ्रो॒णैति॒ श्रव॑ इ॒च्छमा॑ना। पुण्य॒ श्लोकं॒ यज॑मानाय कृण्व॒ती। अ॒ष्टौ दे॒वा वस॑वः सो॒म्यास॑। चत॑स्रो दे॒वीर॒जरा॒ श्रवि॑ष्ठाः। ते य॒ज्ञं पान्तु॒ रज॑सः प॒रस्तात्। सं॒व॒त्स॒रीण॑म॒मृत स्व॒स्ति॥१८॥

%3.1.2.7
य॒ज्ञं न॑ पान्तु॒ वस॑वः पु॒रस्तात्। द॒क्षि॒ण॒तो॑ऽभिय॑न्तु॒ श्रवि॑ष्ठाः। पुण्यं॒ नक्ष॑त्रम॒भि संवि॑शाम। मा नो॒ अरा॑तिर॒घश॒साऽग\sn{}। क्ष॒त्रस्य॒ राजा॒ वरु॑णोऽधिरा॒जः। नक्ष॑त्राणा श॒तभि॑ष॒ग्वसि॑ष्ठः। तौ दे॒वेभ्य॑ कृणुतो दी॒र्घमायु॑। श॒त स॒हस्रा॑ भेष॒जानि॑ धत्तः। य॒ज्ञन्नो॒ राजा॒ वरु॑ण॒ उप॑यातु। तन्नो॒ विश्वे॑ अ॒भि संय॑न्तु दे॒वाः॥१९॥

%3.1.2.8
तन्नो॒ नक्ष॑त्र श॒तभि॑षग्जुषा॒णम्। दी॒र्घमायु॒ प्रति॑रद्भेष॒जानि॑। अ॒ज एक॑पा॒दुद॑गात्पु॒रस्तात्। विश्वा॑ भू॒तानि॑ प्रति॒मोद॑मानः। तस्य॑ दे॒वाः प्र॑स॒वं य॑न्ति॒ सर्वे। प्रो॒ष्ठ॒प॒दासो॑ अ॒मृत॑स्य गो॒पाः। वि॒भ्राज॑मानः समिधा॒न उ॒ग्रः। आऽन्तरि॑क्षमरुह॒दग॒न्द्याम्। त सूर्यं॑ दे॒वम॒जमेक॑पादम्। प्रो॒ष्ठ॒प॒दासो॒ अनु॑यन्ति॒ सर्वे॥२०॥

%3.1.2.9
अहि॑र्बु॒ध्निय॒ प्रथ॑मान एति। श्रेष्ठो॑ दे॒वाना॑मु॒त मानु॑षाणाम्। तं ब्राह्म॒णाः सो॑म॒पाः सो॒म्यास॑। प्रो॒ष्ठ॒प॒दासो॑ अ॒भि र॑क्षन्ति॒ सर्वे। च॒त्वार॒ एक॑म॒भिकर्म॑ दे॒वाः। प्रो॒ष्ठ॒प॒दास॒ इति॒ यान् वद॑न्ति। ते बु॒ध्नियं॑ परि॒षद्य स्तु॒वन्त॑। अहि रक्षन्ति॒ नम॑सोप॒सद्य॑। पू॒षा रे॒वत्यन्वे॑ति॒ पन्थाम्। पु॒ष्टि॒पती॑ पशु॒पा वाज॑बस्त्यौ॥२१॥

%3.1.2.10
इ॒मानि॑ ह॒व्या प्रय॑ता जुषा॒णा। सु॒गैर्नो॒ यानै॒रुप॑यातां य॒ज्ञम्। क्षु॒द्रान्प॒शून्र॑क्षतु रे॒वती॑ नः। गावो॑ नो॒ अश्वा॒ अन्वे॑तु पू॒षा। अन्न॒ रक्ष॑न्तौ बहु॒धा विरू॑पम्। वाज सनुतां॒ यज॑मानाय य॒ज्ञम्। तद॒श्विना॑वश्व॒युजोप॑याताम्। शुभ॒ङ्गमि॑ष्ठौ सु॒यमे॑भि॒रश्वै। स्वं नक्ष॑त्र ह॒विषा॒ यज॑न्तौ। मध्वा॒ संपृ॑क्तौ॒ यजु॑षा॒ सम॑क्तौ॥२२॥

%3.1.2.11
यौ दे॒वानां भि॒षजौ॑ हव्यवा॒हौ। विश्व॑स्य दू॒ताव॒मृत॑स्य गो॒पौ। तौ नक्ष॑त्रं जुजुषा॒णोप॑याताम्। नमो॒ऽश्विभ्याङ्कृणुमोऽश्व॒युग्भ्याम्। अप॑ पा॒प्मानं॒ भर॑णीर्भरन्तु। तद्य॒मो राजा॒ भग॑वा॒न्॒ विच॑ष्टाम्। लो॒कस्य॒ राजा॑ मह॒तो म॒हान् हि। सु॒गन्नः पन्था॒मभ॑यङ्कृणोतु। यस्मि॒न्नक्ष॑त्रे य॒म एति॒ राजा। यस्मि॑न्नेनम॒भ्यषि॑ञ्चन्त दे॒वाः। तद॑स्य चि॒त्र ह॒विषा॑ यजाम। अप॑ पा॒प्मानं॒ भर॑णीर्भरन्तु। नि॒वेश॑नी॒ यत्ते॑ दे॒वा अद॑धुः॥२३॥\anuvakamend[त॒तार॒ मह्यं॑ प्रास॒चीर्या यान्तु य॒ज्ञं वाच स्व॒स्ति दे॒वा अनु॑यन्ति॒ सर्वे॒ वाज॑बस्त्यौ॒ सम॑क्तौ दे॒वास्त्रीणि॑ च]

%3.1.3.1
नवो॑नवो भवति॒ जाय॑मानो॒ यमा॑दि॒त्या अ॒शुमाप्या॒यय॑न्ति। ये विरू॑पे॒ सम॑नसा स॒व्व्यँय॑न्ती। स॒मा॒नन्तन्तुं॑ परितात॒ना ते। वि॒भू प्र॒भू अ॑नु॒भू वि॒श्वतो॑ हुवे। ते नो॒ नक्ष॑त्रे॒ हव॒माग॑मेतम्। व॒यन्दे॒वी ब्रह्म॑णा संविदा॒नाः। सु॒रत्ना॑सो दे॒ववी॑ति॒न्दधा॑नाः। अ॒हो॒रा॒त्रे ह॒विषा॑ व॒र्धय॑न्तः। अति॑ पा॒प्मान॒मति॑ मुक्त्या गमेम। प्रत्यु॑वदृश्याय॒ती॥२४॥

%3.1.3.2
व्यु॒च्छन्ती॑ दुहि॒ता दि॒वः। अ॒पो म॒ही वृ॑णुते॒ चक्षु॑षा। तमो॒ ज्योति॑ष्कृणोति सू॒नरी। उदु॒स्रिया सचते॒ सूर्य॑। सचा॑ उ॒द्यन्नक्ष॑त्रमर्चि॒मत्। तवेदु॑षो॒ व्युषि॒ सूर्य॑स्य च। सं भ॒क्तेन॑ गमेमहि। तन्नो॒ नक्ष॑त्रमर्चि॒मत्। भा॒नु॒मत्तेज॑ उ॒च्चर॑त्। उप॑य॒ज्ञमि॒हाग॑मत्॥२५॥

%3.1.3.3
प्र नक्ष॑त्राय दे॒वाय॑। इन्द्रा॒येन्दु हवामहे। सन॑ सवि॒ता सु॑वत्स॒निम्। पु॒ष्टि॒दां वी॒रव॑त्तमम्। उदु॒त्यञ्चि॒त्रम्। अदि॑तिर्न उरुष्यतु म॒हीमू॒ षु मा॒तरम्। इ॒दं विष्णु॒ प्रतद्विष्णु॑। अ॒ग्निर्मू॒र्धा भुव॑। अनु॑नो॒ऽद्यानु॑मति॒रन्विद॑नुमते॒ त्वम्। ह॒व्य॒वाह॒ स्वि॑ष्टम्॥२६॥\anuvakamend[आ॒य॒त्य॑गम॒त्स्वि॑ष्टम्]

%3.1.4.1
अ॒ग्निर्वा अ॑कामयत। अ॒न्ना॒दो दे॒वानास्या॒मिति॑। स ए॒तम॒ग्नये॒ कृत्ति॑काभ्यः पुरो॒डाश॑म॒ष्टाक॑पालं॒ निर॑वपत्। ततो॒ वै सोऽन्ना॒दो दे॒वाना॑मभवत्। अ॒ग्निर्वै दे॒वाना॑मन्ना॒दः। यथा॑ ह॒ वा अ॒ग्निर्दे॒वाना॑मन्ना॒दः। ए॒व ह॒ वा ए॒ष म॑नु॒ष्या॑णां भवति। य ए॒तेन॑ ह॒विषा॒ यज॑ते। य उ॑ चैनदे॒वं वेद॑। सोऽत्र॑ जुहोति। अ॒ग्नये॒ स्वाहा॒ कृत्ति॑काभ्य॒ स्वाहा। अ॒म्बायै॒ स्वाहा॑ दु॒लायै॒ स्वाहा। नि॒त॒त्न्यै स्वाहा॒ऽभ्रय॑न्त्यै॒ स्वाहा। मे॒घय॑न्त्यै॒ स्वाहा॑ व॒र्॒षय॑न्त्यै॒ स्वाहा। चु॒पु॒णीका॑यै॒ स्वाहेति॑॥२७॥

%3.1.4.2
प्र॒जाप॑तिः प्र॒जा अ॑सृजत। ता अ॑स्मात्सृ॒ष्टाः परा॑चीरायन्। तासा रोहि॒णीम॒भ्य॑ध्यायत्। सो॑ऽकामयत। उप॒ मा व॑र्तेत। समे॑नया गच्छे॒येति॑। स ए॒तं प्र॒जाप॑तये रोहि॒ण्यै च॒रुन्निर॑वपत्। ततो॒ वै सा तमु॒पाव॑र्तत। समे॑नयागच्छत। उप॑ ह॒ वा ए॑नं प्रि॒यमाव॑र्तते। सं प्रि॒येण॑ गच्छते। य ए॒तेन॑ ह॒विषा॒ यज॑ते। य उ॑चैनदे॒वं वेद॑। सोऽत्र॑ जुहोति। प्र॒जाप॑तये॒ स्वाहा॑ रोहि॒ण्यै स्वाहा। रोच॑मानायै॒ स्वाहा प्र॒जाभ्य॒ स्वाहेति॑॥२८॥

%3.1.4.3
सोमो॒ वा अ॑कामयत। ओष॑धीना रा॒ज्यम॒भिज॑येय॒मिति॑। स ए॒त सोमा॑य मृगशी॒र्॒षाय॑ श्यामा॒कञ्च॒रुं पय॑सि॒ निर॑वपत्। ततो॒ वै स ओष॑धीना रा॒ज्यम॒भ्य॑जयत्। स॒मा॒नाना ह॒ वै रा॒ज्यम॒भिज॑यति। य ए॒तेन॑ ह॒विषा॒ यज॑ते। य उ॑ चैनदे॒वं वेद॑। सोऽत्र॑ जुहोति। सोमा॑य॒ स्वाहा॑ मृगशी॒र्‌षाय॒ स्वाहा। इ॒न्व॒काभ्य॒ स्वाहौष॑धीभ्य॒ स्वाहा। रा॒ज्याय॒ स्वाहा॒ऽभिजि॑त्यै॒ स्वाहेति॑॥२९॥

%3.1.4.4
रु॒द्रो वा अ॑कामयत। प॒शु॒मान्त्स्या॒मिति॑। स ए॒त रु॒द्राया॒र्द्रायै॒ प्रैय्य॑ङ्गवञ्च॒रुं पय॑सि॒ निर॑वपत्। ततो॒ वै स प॑शु॒मान॑भवत्। प॒शु॒मान् ह॒ वै भ॑वति। य ए॒तेन॑ ह॒विषा॒ यज॑ते। य उ॑ चैनदे॒वं वेद॑। सोऽत्र॑ जुहोति। रु॒द्राय॒ स्वाहा॒ऽर्द्रायै॒ स्वाहा। पिन्व॑मानायै॒ स्वाहा॑ प॒शुभ्य॒ स्वाहेति॑॥३०॥

%3.1.4.5
ऋ॒क्षा वा इ॒यम॑लो॒मका॑ऽऽसीत्। साऽका॑मयत। ओष॑धीभि॒र्वन॒स्पति॑भि॒ प्रजा॑ये॒येति॑। सैतमदि॑त्यै॒ पुन॑र्वसुभ्याञ्च॒रुन्निर॑वपत्। ततो॒ वा इ॒यमोष॑धीभि॒र्वन॒स्पति॑भि॒ प्राजा॑यत। प्रजा॑यते ह॒ वै प्र॒जया॑ प॒शुभि॑। य ए॒तेन॑ ह॒विषा॒ यज॑ते। य उ॑ चैनदे॒वं वेद॑। सोऽत्र॑ जुहोति। अदि॑त्यै॒ स्वाहा॒ पुन॑र्वसुभ्याम्। स्वाहा भूत्यै॒ स्वाहा॒ प्रजात्यै॒ स्वाहेति॑॥३१॥

%3.1.4.6
बृह॒स्पति॒र्वा अ॑कामयत। ब्र॒ह्म॒व॒र्च॒सी स्या॒मिति॑। स ए॒तं बृह॒स्पत॑ये ति॒ष्या॑य नैवा॒रञ्च॒रुं पय॑सि॒ निर॑वपत्। ततो॒ वै स ब्र॑ह्मवर्च॒स्य॑भवत्। ब्र॒ह्म॒व॒र्च॒सी ह॒ वै भ॑वति। य ए॒तेन॑ ह॒विषा॒ यज॑ते। य उ॑ चैनदे॒वं वेद॑। सोऽत्र॑ जुहोति। बृह॒स्पत॑ये॒ स्वाहा॑ ति॒ष्या॑य॒ स्वाहा। ब्र॒ह्म॒व॒र्च॒साय॒ स्वाहेति॑॥३२॥

%3.1.4.7
दे॒वा॒सु॒राः संय॑त्ता आसन्। ते दे॒वाः स॒र्पेभ्य॑ आश्रे॒षाभ्य॒ आज्ये॑ कर॒म्भन्निर॑वपन्। ताने॒ताभि॑रे॒वदे॒वता॑भि॒रुपा॑नयन्। ए॒ताभि॑र्‌ह॒ वै दे॒वता॑भिर्द्वि॒षन्तं॒ भ्रातृ॑व्य॒मुप॑नयति। य ए॒तेन॑ ह॒विषा॒ यज॑ते। य उ॑ चैनदे॒वं वेद॑। सोऽत्र॑ जुहोति। स॒र्पेभ्य॒ स्वाहाऽऽश्रे॒षाभ्य॒ स्वाहा। द॒न्द॒शूकेभ्य॒ स्वाहेति॑॥३३॥

%3.1.4.8
पि॒तरो॒ वा अ॑कामयन्त। पि॒तृ॒लो॒क ऋ॑ध्नुया॒मेति॑। त ए॒तं पि॒तृभ्यो॑ म॒घाभ्य॑ पुरो॒डाश॒ षट्क॑पालं॒ निर॑वपन्। ततो॒ वै ते पि॑तृलो॒क आर्ध्नुवन्। पि॒तृ॒लो॒के ह॒ वा ऋ॑ध्नोति। य ए॒तेन॑ ह॒विषा॒ यज॑ते। य उ॑ चैनदे॒वं वेद॑। सोऽत्र॑ जुहोति। पि॒तृभ्य॒ स्वाहा॑ म॒घाभ्य॑। स्वाहा॑ऽन॒घाभ्य॒ स्वाहा॑ग॒दाभ्य॑। स्वाहा॑ऽरुन्ध॒तीभ्य॒ स्वाहेति॑॥३४॥

%3.1.4.9
अ॒र्य॒मा वा अ॑कामयत। प॒शु॒मान्त्स्या॒मिति॑। स ए॒तम॑र्य॒म्णे फल्गु॑नीभ्याञ्च॒रुन्निर॑वपत्। ततो॒ वै स प॑शु॒मान॑भवत्। प॒शुमान् ह॒ वै भ॑वति। य ए॒तेन॑ ह॒विषा॒ यज॑ते। य उ॑ चैनदे॒वं वेद॑। सोऽत्र॑ जुहोति। अ॒र्य॒म्णे स्वाहा॒ फल्गु॑नीभ्या॒ स्वाहा। प॒शुभ्य॒ स्वाहेति॑॥३५॥

%3.1.4.10
भगो॒ वा अ॑कामयत। भ॒गी श्रे॒ष्ठी दे॒वानास्या॒मिति॑। स ए॒तं भगा॑य॒ फल्गु॑नीभ्याञ्च॒रुन्निर॑वपत्। ततो॒ वै स भ॒गी श्रे॒ष्ठी दे॒वाना॑मभवत्। भ॒गी ह॒ वै श्रे॒ष्ठी स॑मा॒नानां भवति। य ए॒तेन॑ ह॒विषा॒ यज॑ते। य उ॑ चैनदे॒वं वेद॑। सोऽत्र॑ जुहोति। भगा॑य॒ स्वाहा॒ फल्गु॑नीभ्या॒ स्वाहा। श्रैष्ठ्या॑य॒ स्वाहेति॑॥३६॥

%3.1.4.11
स॒वि॒ता वा अ॑कामयत। श्रन्मे॑ दे॒वा दधी॑रन्। स॒वि॒ता स्या॒मिति॑। स ए॒त स॑वि॒त्रे हस्ता॑य पुरो॒डाशं॒ द्वाद॑शकपालं॒ निर॑वपदाशू॒नां व्री॑ही॒णाम्। ततो॒ वै तस्मै॒ श्रद्दे॒वा अद॑धत। स॒वि॒ताऽभ॑वत्। श्रद्ध॒वा अ॑स्मै मनु॒ष्या॑ दधते। स॒वि॒ता स॑मा॒नानां भवति। य ए॒तेन॑ ह॒विषा॒ यज॑ते। य उ॑ चैनदे॒वं वेद॑। सोऽत्र॑ जुहोति। स॒वि॒त्रे स्वाहा॒ हस्ता॑य। स्वाहा॑ दद॒ते स्वाहा॑ पृण॒ते। स्वाहा प्र॒यच्छ॑ते॒ स्वाहा प्रतिगृभ्ण॒ते स्वाहेति॑ ॥३७॥

%3.1.4.12
त्वष्टा॒ वा अ॑कामयत। चि॒त्रं प्र॒जां वि॑न्दे॒येति॑। स ए॒तन्त्वष्ट्रे॑ चि॒त्रायै॑ पुरो॒डाश॑म॒ष्टाक॑पालं॒ निर॑वपत्। ततो॒ वै स चि॒त्रं प्र॒जाम॑विन्दत। चि॒त्र ह॒ वै प्र॒जां वि॑न्दते। य ए॒तेन॑ ह॒विषा॒ यज॑ते। य उ॑ चैनदे॒वं वेद॑। सोऽत्र॑ जुहोति। त्वष्ट्रे॒ स्वाहा॑ चि॒त्रायै॒ स्वाहा। चैत्रा॑य॒ स्वाहा प्र॒जायै॒ स्वाहेति॑॥३८॥

%3.1.4.13
वा॒युर्वा अ॑कामयत। का॒म॒चार॑मे॒षु लो॒केष्व॒भिज॑येय॒मिति॑। स ए॒तद्वा॒यवे॒ निष्ट्या॑यै गृ॒ष्ट्यै दु॒ग्धं पयो॒ निर॑वपत्। ततो॒ वै स का॑म॒चार॑मे॒षु लो॒केष्व॒भ्य॑जयत्। का॒म॒चार ह॒ वा ए॒षु लो॒केष्व॒भिज॑यति। य ए॒तेन॑ ह॒विषा॒ यज॑ते। य उ॑ चैनदे॒वं वेद॑। सोऽत्र॑ जुहोति। वा॒यवे॒ स्वाहा॒ निष्ट्या॑यै॒ स्वाहा। का॒म॒चारा॑य॒ स्वाहा॒ऽभिजि॑त्यै॒ स्वाहेति॑॥३९॥

%3.1.4.14
इ॒न्द्रा॒ग्नी वा अ॑कामयेताम्। श्रैष्ठ्यं॑ दे॒वाना॑म॒भिज॑ये॒वेति॑। तावे॒तमि॑न्द्रा॒ग्निभ्यां॒ विशा॑खाभ्यां पुरो॒डाश॒मेका॑दशकपालं॒ निर॑वपताम्। ततो॒ वै तौ श्रैष्ठ्यं॑ दे॒वाना॑म॒भ्य॑जयताम्। श्रैष्ठ्य ह॒ वै स॑मा॒नाना॑म॒भि ज॑यति। य ए॒तेन॑ ह॒विषा॒ यज॑ते। य उ॑ चैनदे॒वं वेद॑। सोऽत्र॑ जुहोति। इ॒न्द्रा॒ग्निभ्या॒ स्वाहा॒ विशा॑खाभ्या॒ स्वाहा। श्रैष्ठ्या॑य॒ स्वाहा॒ऽभिजि॑त्यै॒ स्वाहेति॑॥४०॥

%3.1.4.15
अथै॒तत्पौर्णमा॒स्या आज्यं॒ निर्व॑पति। कामो॒ वै पौर्णमा॒सी। काम॒ आज्यम्। कामे॑नै॒व काम॒ सम॑र्धयति। क्षि॒प्रमे॑न॒ सकाम॒ उप॑नमति। येन॒ कामे॑न॒ यज॑ते। सोऽत्र॑ जुहोति। पौ॒र्ण॒मा॒स्यै स्वाहा॒ कामा॑य॒ स्वाहाऽऽग॑त्यै॒ स्वाहेति॑॥४१॥\anuvakamend[अ॒ग्निः पञ्च॑दश प्र॒जाप॑ति॒ष्षोड॑श॒ सोम॒ एका॑दश रु॒द्रो दश॒र्क्षैका॑दश॒ बृह॒स्पति॒र्दश॑ देवासु॒रा नव॑ पि॒तर॒ एका॑दशार्य॒मा भगो॒ दश॑ दश सवि॒ता चतु॑र्दश॒ त्वष्टा॑ वा॒युरि॑न्द्रा॒ग्नी दश॑ द॒शाथै॒तत्पौर्णमा॒स्या अ॒ष्टौ पञ्च॑दश]

%3.1.5.1
मि॒त्रो वा अ॑कामयत। मि॒त्र॒धेय॑मे॒षु लो॒केष्व॒भिज॑येय॒मिति॑। स ए॒तं मि॒त्राया॑नूरा॒धेभ्य॑श्च॒रुन्निर॑वपत्। ततो॒ वै स मि॑त्र॒धेय॑मे॒षुलो॒केष्व॒भ्य॑जयत्। मि॒त्र॒धेय ह॒ वा ए॒षु लो॒केष्व॒भिज॑यति। य ए॒तेन॑ ह॒विषा॒ यज॑ते। य उ॑ चैनदे॒वं वेद॑। सोऽत्र॑ जुहोति। मि॒त्राय॒ स्वाहा॑ऽनूरा॒धेभ्य॒ स्वाहा। मि॒त्र॒धेया॑य॒ स्वाहा॒ऽभिजि॑त्यै॒ स्वाहेति॑॥४२॥

%3.1.5.2
इन्द्रो॒ वा अ॑कामयत। ज्यैष्ठ्यं॑ दे॒वाना॑म॒भिज॑येय॒मिति॑। स ए॒तमिन्द्रा॑य ज्ये॒ष्ठायै॑ पुरो॒डाश॒मेका॑दशकपालं॒ निर॑वपन्म॒हाव्री॑हीणाम्। ततो॒ वै स ज्यैष्ठ्यं॑ दे॒वाना॑म॒भ्य॑जयत्। ज्यैष्ठ्य ह॒ वै स॑मा॒नाना॑म॒भिज॑यति। य ए॒तेन॑ ह॒विषा॒ यज॑ते। य उ॑ चैनदे॒वं वेद॑। सोऽत्र॑ जुहोति। इन्द्रा॑य॒ स्वाहा ज्ये॒ष्ठायै॒ स्वाहा। ज्यैष्ठ्या॑य॒ स्वाहा॒भिजि॑त्यै॒ स्वाहेति॑॥४३॥

%3.1.5.3
प्र॒जाप॑ति॒र्वा अ॑कामयत। मूलं॑ प्र॒जां वि॑न्दे॒येति॑। स ए॒तं प्र॒जाप॑तये॒ मूला॑य च॒रुन्निर॑वपत्। ततो॒ वै स मूलं॑ प्र॒जाम॑विन्दत। मूल ह॒ वै प्र॒जां वि॑न्दते। य ए॒तेन॑ ह॒विषा॒ यज॑ते। य उ॑ चैनदे॒वं वेद॑। सोऽत्र॑ जुहोति। प्र॒जाप॑तये॒ स्वाहा॒ मूला॑य॒ स्वाहा। प्र॒जायै॒ स्वाहेति॑॥४४॥

%3.1.5.4
आपो॒ वा अ॑कामयन्त। स॒मु॒द्रङ्काम॑म॒भिज॑ये॒मेति॑। ता ए॒तम॒द्भ्यो॑ऽषा॒ढाभ्य॑श्च॒रुन्निर॑वपन्। ततो॒ वै ताः स॑मु॒द्रङ्काम॑म॒भ्य॑जयन्। स॒मु॒द्र ह॒ वै काम॑म॒भिज॑यति। य ए॒तेन॑ ह॒विषा॒ यज॑ते। य उ॑ चैनदे॒वं वेद॑। सोऽत्र॑ जुहोति। अ॒द्भ्यः स्वाहा॑ऽषा॒ढाभ्य॒ स्वाहा। स॒मु॒द्राय॒ स्वाहा॒ कामा॑य॒ स्वाहा। अ॒भिजि॑त्यै॒ स्वाहेति॑॥४५॥

%3.1.5.5
विश्वे॒ वै दे॒वा अ॑कामयन्त। अ॒न॒प॒ज॒य्यं ज॑ये॒मेति॑। त ए॒तं विश्वेभ्यो दे॒वेभ्यो॑ऽषा॒ढाभ्य॑श्च॒रुन्निर॑वपन्। ततो॒ वै ते॑ऽनपज॒य्यम॑जयन्। अ॒न॒प॒ज॒य्य ह॒ वै ज॑यति। य ए॒तेन॑ ह॒विषा॒ यज॑ते। य उ॑ चैनदे॒वं वेद॑। सोऽत्र॑ जुहोति। विश्वेभ्यो दे॒वेभ्य॒ स्वाहा॑ऽषा॒ढाभ्य॒ स्वाहा। अ॒न॒प॒ज॒य्याय॒ स्वाहा॒ जित्यै॒ स्वाहेति॑॥४६॥

%3.1.5.6
ब्रह्म॒ वा अ॑कामयत। ब्र॒ह्म॒लो॒कम॒भिज॑येय॒मिति॑। तदे॒तं ब्रह्म॑णेऽभि॒जिते॑ च॒रुन्निर॑वपत्। ततो॒ वै तद्ब्र॑ह्मलो॒कम॒भ्य॑जयत्। ब्र॒ह्म॒लो॒क ह॒ वा अ॒भिज॑यति। य ए॒तेन॑ ह॒विषा॒ यज॑ते। य उ॑ चैनदे॒वं वेद॑। सोऽत्र॑ जुहोति। ब्रह्म॑णे॒ स्वाहा॑ऽभि॒जिते॒ स्वाहा। ब्र॒ह्म॒लो॒काय॒ स्वाहा॒ऽभिजि॑त्यै॒ स्वाहेति॑॥४७॥

%3.1.5.7
विष्णु॒र्वा अ॑कामयत। पुण्य॒ श्लोक शृण्वीय। न मा॑ पा॒पी की॒र्तिराग॑च्छे॒दिति॑। स ए॒तं विष्ण॑वे श्रो॒णायै॑ पुरो॒डाश॑न्त्रिकपा॒लन्निर॑वपत्। ततो॒ वै स पुण्य॒ श्लोक॑मशृणुत। नैनं॑ पा॒पी की॒र्तिराग॑च्छत्। पुण्य ह॒ वै श्लोक शृणुते। नैनं॑ पा॒पी की॒र्तिराग॑च्छति। य ए॒तेन॑ ह॒विषा॒ यज॑ते। य उ॑ चैनदे॒वं वेद॑। सोऽत्र॑ जुहोति। विष्ण॑वे॒ स्वाहा श्रो॒णायै॒ स्वाहा। श्लोका॑य॒ स्वाहा श्रु॒ताय॒ स्वाहेति॑॥४८॥

%3.1.5.8
वस॑वो॒ वा अ॑कामयन्त। अग्रं॑ दे॒वता॑नां॒ परी॑या॒मेति॑। त ए॒तं वसु॑भ्य॒ श्रवि॑ष्ठाभ्यः पुरो॒डाश॑म॒ष्टाक॑पालं॒ निर॑वपन्। ततो॒ वै तेऽग्रं॑ दे॒वता॑नां॒ पर्या॑यन्। अग्र ह॒ वै स॑मा॒नानां॒ पर्ये॑ति। य ए॒तेन॑ ह॒विषा॒ यज॑ते। य उ॑ चैनदे॒वं वेद॑। सोऽत्र॑ जुहोति। वसु॑भ्य॒ स्वाहा॒ श्रवि॑ष्ठाभ्य॒ स्वाहा। अग्रा॑य॒ स्वाहा॒ परीत्यै॒ स्वाहेति॑॥४९॥

%3.1.5.9
इन्द्रो॒ वा अ॑कामयत। दृ॒ढोऽशि॑थिलः स्या॒मिति॑। स ए॒तं वरु॑णाय श॒तभि॑षजे भेष॒जेभ्य॑ पुरो॒डाशं॒ दश॑कपालं॒ निर॑वपत्कृ॒ष्णानां व्रीही॒णाम्। ततो॒ वै स दृ॒ढोऽशि॑थिलोऽभवत्। दृ॒ढो ह॒ वा अशि॑थिलो भवति। य ए॒तेन॑ ह॒विषा॒ यज॑ते। य उ॑ चैनदे॒वं वेद॑। सोऽत्र॑ जुहोति। वरु॑णाय॒ स्वाहा॑ श॒तभि॑षजे॒ स्वाहा। भे॒ष॒जेभ्य॒ स्वाहेति॑॥५०॥

%3.1.5.10
अ॒जो वा एक॑पादकामयत। ते॒ज॒स्वी ब्र॑ह्मवर्च॒सी स्या॒मिति॑। स ए॒तम॒जायैक॑पदे प्रोष्ठप॒देभ्य॑श्च॒रुन्निर॑वपत्। ततो॒ वै स ते॑ज॒स्वी ब्र॑ह्मवर्च॒स्य॑भवत्। ते॒ज॒स्वी ह॒ वै ब्र॑ह्मवर्च॒सी भ॑वति। य ए॒तेन॑ ह॒विषा॒ यजते। य उ॑ चैनदे॒वं वेद॑। सोऽत्र॑ जुहोति। अ॒जायैक॑पदे॒ स्वाहा प्रोष्ठप॒देभ्य॒ स्वाहा। तेज॑से॒ स्वाहा ब्रह्मवर्च॒साय॒ स्वाहेति॑॥५१॥

%3.1.5.11
अहि॒र्वै बु॒ध्नियो॑ऽकामयत। इ॒मां प्र॑ति॒ष्ठां वि॑न्दे॒येति॑। स ए॒तमह॑ये बु॒ध्निया॑य प्रोष्ठप॒देभ्य॑ पुरो॒डाशं॒ भूमि॑कपालं॒ निर॑वत्। ततो॒ वै स इ॒मां प्र॑ति॒ष्ठाम॑विन्दत। इ॒मा ह॒ वै प्र॑ति॒ष्ठां वि॑न्दते। य ए॒तेन॑ ह॒विषा॒ यज॑ते। य उ॑ चैनदे॒वं वेद॑। सोऽत्र॑ जुहोति। अह॑ये बु॒ध्निया॑य॒ स्वाहा प्रोष्ठप॒देभ्य॒ स्वाहा। प्र॒ति॒ष्ठायै॒ स्वाहेति॑॥५२॥

%3.1.5.12
पू॒षा वा अ॑कामयत। प॒शु॒मान्त्स्या॒मिति॑। स ए॒तं पू॒ष्णे रे॒वत्यै॑ च॒रुन्निर॑वपत्। ततो॒ वै स प॑शु॒मान॑भवत्। प॒शु॒मान् ह॒ वै भ॑वति। य ए॒तेन॑ ह॒विषा॒ यज॑ते। य उ॑ चैनदे॒वं वेद॑। सोऽत्र॑ जुहोति। पू॒ष्णे स्वाहा॑ रे॒वत्यै॒ स्वाहा। प॒शुभ्य॒ स्वाहेति॑॥५३॥

%3.1.5.13
अ॒श्विनौ॒ वा अ॑कामयेताम्। श्रो॒त्र॒स्विना॒वब॑धिरौ स्या॒वेति॑। तावे॒तम॒श्विभ्या॑मश्व॒युग्भ्यां पुरो॒डाश॑न्द्विकपा॒लन्निर॑वपताम्। ततो॒ वै तौ श्रोत्र॒स्विना॒वब॑धिरावभवताम्। श्रो॒त्र॒स्वी ह॒ वा अब॑धिरो भवति। य एते॒न॑ ह॒विषा॒ यज॑ते। य उ॑ चैनदे॒वं वेद॑। सोऽत्र॑ जुहोति। अ॒श्विभ्या॒ स्वाहाऽश्व॒युग्भ्या॒ स्वाहा। श्रोत्रा॑य॒ स्वाहा॒ श्रुत्यै॒ स्वाहेति॑॥५४॥

%3.1.5.14
य॒मो वा अ॑कामयत। पि॒तृ॒णा रा॒ज्यम॒भिज॑येय॒मिति॑। स ए॒तं य॒माया॑प॒भर॑णीभ्यश्च॒रुन्निर॑पवत्। ततो॒ वै स पि॑तृ॒णा रा॒ज्यम॒भ्य॑जयत्। स॒मा॒नाना ह॒ वै रा॒ज्यम॒भि ज॑यति। य ए॒तेन॑ ह॒विषा॒ यज॑ते। य उ॑ चैनदे॒वं वेद॑। सोऽत्र॑ जुहोति। य॒माय॒ स्वाहा॑ऽप॒भर॑णीभ्य॒ स्वाहा। रा॒ज्याय॒ स्वाहा॒भिजि॑त्यै॒ स्वाहेति॑॥५५॥

%3.1.5.15
अथै॒तद॑मावा॒स्या॑या॒ आज्यं॒ निर्व॑पति। कामो॒ वा अ॑मावा॒स्या। काम॒ आज्यम्। कामे॑नै॒व काम॒ सम॑र्धयति। क्षि॒प्रमे॑न॒ सकाम॒ उप॑नमति। येन॒ कामे॑न॒ यज॑ते। सोऽत्र॑ जुहोति। अ॒मा॒वा॒स्या॑यै॒ स्वाहा॒ कामा॑य॒ स्वाहाऽऽग॑त्यै॒ स्वाहेति॑॥५६॥\anuvakamend[मि॒त्र इन्द्र॑ प्र॒जाप॑ति॒र्दश॑ द॒शाप॒ एका॑दश॒ विश्वे॒ ब्रह्म॒ दश॑दश॒ विष्णु॒स्त्रयो॑दश॒ वस॑व॒ इन्द्रो॒ऽजोऽहि॒र्वै बु॒ध्निय॑ पू॒षाऽश्विनौ॑ य॒मो दश॑ द॒शाथै॒तद॑मावा॒स्या॑या अ॒ष्टौ पञ्च॑दश]

%3.1.6.1
च॒न्द्रमा॒ वा अ॑कामयत। अ॒होरा॒त्रान॑र्धमा॒सान्मासा॑नृ॒तून्त्सं॑वत्स॒रमा॒प्त्वा। च॒न्द्रम॑स॒ सायु॑ज्य सलो॒कता॑माप्नुया॒मिति॑। स ए॒तञ्च॒न्द्रम॑से प्रती॒दृश्या॑यै पुरो॒डाशं॒ पञ्च॑दशकपालं॒ निर॑वपत्। ततो॒ वै सो॑ऽहोरा॒त्रान॑र्धमा॒सान्मासा॑नृ॒तून्त्सं॑वत्स॒रमा॒प्त्वा। च॒न्द्रम॑स॒ सायु॑ज्य सलो॒कता॑माप्नोत्। अ॒हो॒रा॒त्रान् ह॒ वा अ॑र्धमा॒सान्मासा॑नृ॒तून्त्सं॑वत्स॒रमा॒प्त्वा। च॒न्द्रम॑स॒ सायु॑ज्य सलो॒कता॑माप्नोति। य ए॒तेन॑ ह॒विषा॒ यज॑ते। य उ॑ चैनदे॒वं वेद॑। सोऽत्र॑ जुहोति। च॒न्द्रम॑से॒ स्वाहा प्रती॒दृश्या॑यै॒ स्वाहा। अ॒हो॒रा॒त्रेभ्य॒ स्वाहाऽर्धमा॒सेभ्य॒ स्वाहा। मासेभ्य॒ स्वाह॒र्तुभ्य॒ स्वाहा। सं॒व॒त्स॒राय॒ स्वाहेति॑॥५७॥

%3.1.6.2
अ॒हो॒रा॒त्रे वा अ॑कामयेताम्। अत्य॑होरा॒त्रे मु॑च्येवहि। न ना॑वहोरा॒त्रे आप्नुयाता॒मिति॑। ते ए॒तम॑होरा॒त्राभ्यां च॒रुन्निर॑वपताम्। द्व॒यानाव्व्रीँही॒णाम्। शु॒क्लानां च कृ॒ष्णानां च। स॒वा॒त्योर्दु॒ग्धे। श्वे॒तायै॑ च कृ॒ष्णायै॑ च। ततो॒ वै ते अत्य॑होरा॒त्रे अ॑मुच्येते। नैने॑ अहोरा॒त्रे आप्नुताम्। अति॑ ह॒ वा अ॑होरा॒त्रे मु॑च्यते। नैन॑महोरा॒त्रे आप्नुतः। य ए॒तेन॑ ह॒विषा॒ यज॑ते। य उ॑ चैनदे॒वं वेद॑। सोऽत्र॑ जुहोति। अह्ने॒ स्वाहा॒ रात्रि॑यै॒ स्वाहा। अति॑मुक्त्यै॒ स्वाहेति॑॥५८॥

%3.1.6.3
उ॒षा वा अ॑कामयत। प्रि॒याऽऽदि॒त्यस्य॑ सु॒भगा स्या॒मिति॑। सैतमु॒षसे॑ च॒रुन्निर॑वपत्। ततो॒ वै सा प्रि॒याऽऽदि॒त्यस्य॑ सु॒भगा॑ऽभवत्। प्रि॒यो ह॒ वै स॑मा॒नाना सु॒भगो॑ भवति। य ए॒तेन॑ ह॒विषा॒ यज॑ते। य उ॑ चैनदे॒वं वेद॑। सोऽत्र॑ जुहोति। उ॒षसे॒ स्वाहा॒ व्यु॑ष्ट्यै॒ स्वाहा। व्यू॒षुष्यै॒ स्वाहा व्यु॒च्छन्त्यै॒ स्वाहा। व्यु॑ष्टायै॒ स्वाहेति॑॥५९॥

%3.1.6.4
अथै॒तस्मै॒ नक्ष॑त्राय च॒रुनिर्व॑पति। यथा॒ त्वन्दे॒वाना॒मसि॑। ए॒वम॒हं म॑नु॒ष्या॑णां भूयास॒मिति॑। यथा॑ ह॒ वा ए॒तद्दे॒वानाम्। ए॒व ह॒ वा ए॒ष म॑नु॒ष्या॑णां भवति। य ए॒तेन॑ ह॒विषा॒ यज॑ते। य उ॑ चैनदे॒वं वेद॑। सोऽत्र॑ जुहोति। नक्ष॑त्राय॒ स्वाहो॑देष्य॒ते स्वाहा। उ॒द्य॒ते स्वाहोदि॑ताय॒ स्वाहा। हर॑से॒ स्वाहा॒ भर॑से॒ स्वाहा। भ्राज॑से॒ स्वाहा॒ तेज॑से॒ स्वाहा। तप॑से॒ स्वाहा ब्रह्मवर्च॒साय॒ स्वाहेति॑॥६०॥

%3.1.6.5
सूर्यो॒ वा अ॑कामयत। नक्ष॑त्राणां प्रति॒ष्ठा स्या॒मिति॑। स ए॒त सूर्या॑य॒ नक्ष॑त्रेभ्यश्च॒रुन्निर॑वपत्। ततो॒ वै स नक्ष॑त्राणां प्रति॒ष्ठाऽभ॑वत्। प्र॒ति॒ष्ठा ह॒ वै स॑मा॒नानां भवति। य ए॒तेन॑ ह॒विषा॒ यज॑ते। य उ॑ चैनदे॒वं वेद॑। सोऽत्र॑ जुहोति। सूर्या॑य॒ स्वाहा॒ नक्ष॑त्रेभ्य॒ स्वाहा। प्र॒ति॒ष्ठायै॒ स्वाहेति॑॥६१॥

%3.1.6.6
अथै॒तमदि॑त्यै च॒रुन्निर्व॑पति। इ॒यं वा अदि॑तिः। अ॒स्यामे॒व प्रति॑तिष्ठति। सोऽत्र॑ जुहोति। अदि॑त्यै॒ स्वाहा प्रति॒ष्ठायै॒ स्वाहेति॑॥६२॥

%3.1.6.7
अथै॒तं विष्ण॑वे च॒रुन्निर्व॑पति। य॒ज्ञो वै विष्णु॑। य॒ज्ञ ए॒वान्त॒तः प्रति॑तिष्ठति। सोऽत्र॑ जुहोति। विष्ण॑वे॒ स्वाहा॑ य॒ज्ञाय॒ स्वाहा। प्र॒ति॒ष्ठायै॒ स्वाहेति॑॥६३॥\anuvakamend[च॒न्द्रमा॒ पञ्च॑दशाहोरा॒त्रे स॒प्तद॑शो॒षा एका॑द॒शाथै॒तस्मै॒ नक्ष॑त्राय॒ त्रयो॑दश॒ सूर्यो॒ दशाथै॒तमदि॑त्यै॒ पञ्चाथै॒तं विष्ण॑वे॒ षट्त्स॒प्त (स॒वि॒ताऽऽशू॒नाव्व्रीँ॑ही॒णामिन्द्रो॑ म॒हाव्री॑हीणा॒मिन्द्र॑ कृ॒ष्णानाव्व्रीँही॒णाम॑होरा॒त्रे द्व॒यानाव्व्रीँही॒णाम्। पि॒तर॒ष्षट्क॑पाल सवि॒ता द्वाद॑शकपालमिन्द्रा॒ग्नी एका॑दशकपाल॒मिन्द्र॒ एका॑दशकपाल॒मिन्द्रो॒ दश॑कपालं॒ विष्णु॑स्त्रिकपा॒लमहि॒र्भूमि॑कपालम॒श्विनौ द्विकपा॒लञ्च॒न्द्रमा॒ पञ्च॑दशकपालम॒ग्निस्त्वष्टा॒ वस॑वो॒ऽष्टाक॑पालम॒न्यत्र॑ च॒रुम्। रु॒द्रोऽर्य॒मा पू॒षा प॑शु॒मान्त्स्या॒ सोमो॑ रु॒द्रो बृह॒स्पति॒ पय॑सि वा॒युः पय॒ सोमो॑ वा॒युरि॑न्द्रा॒ग्नी मि॒त्र इन्द्र॒ आपो॒ ब्रह्म॑ य॒मो॑ऽभिजि॑त्यै॒ त्वष्टा प्र॒जाप॑तिः प्र॒जायै॑ पौर्णमा॒स्या अ॑मावा॒स्या॑या॒ अग॑त्यै॒ विश्वे॒ जित्या॑ अ॒श्विनौ॒ श्रुत्यै। ब्रह्म॒ तदे॒तं विष्णु॒ स ए॒तं वा॒युः स ए॒तदाप॒स्ताः। पि॒तरो॒ विश्वे॒ वस॑वोऽकामयन्त॒ मेति॒ त ए॒तन्निर॑वपन्। आपो॑ऽकामयन्त॒ मेति॒ ता ए॒तन्निर॑वपन्। इ॒न्द्रा॒ग्नी अ॒श्विना॑वकामयेतां॒ वेति॒ तावे॒तन्निर॑वपताम्। अ॒हो॒रा॒त्रे वा अ॑कामयेता॒मिति॒ ते ए॒तन्निर॑वपताम्। अ॒न्यत्रा॑कामय॒तेति॒ स ए॒तन्निर॑वपत्। इ॒न्द्रा॒ग्नी श्रैष्ठ्य॒मिन्द्रो॒ ज्यैष्ठ्य॒मिन्द्रो॑ दृ॒ढः। अहि॒ सूर्योऽदि॑त्यै॒ विष्ण॑वे प्रति॒ष्ठायै। सोमो॑ य॒मः स॑मा॒नानाम्। अ॒ग्निर्नो॑ रीरिषद॒न्यत्र॑ रीरिषः ॥ )]




\prashnaend{अ॒ग्निर्न॑ ऋ॒ध्यास्म॒ नवो॑नवो॒ऽग्निर्मि॒त्रश्च॒न्द्रमा॒ष्षट्॥६॥}{अ॒ग्निर्न॒स्तन्नो॑ वा॒युरहि॑र्बु॒ध्निय॑ ऋ॒क्षा वा इ॒यमथै॒तत्पौर्णमा॒स्या अ॒जो वा एक॑पा॒त्सूर्य॒स्त्रिष॑ष्टिः॥६३॥}{अ॒ग्निर्न॑ पातु प्रति॒ष्ठायै॒ स्वाहेति॑॥}{हरि॑ ओम्॥}{इति श्रीकृष्णयजुर्वेदीयतैत्तिरीयब्राह्मणे तृतीयाष्टके प्रथमः प्रपाठकः समाप्तः॥}
\clearpage
\sect{द्वितीयः प्रश्नः}
\setcounter{anuvakam}{0}
\dnsub{तैत्तिरीयब्राह्मणे तृतीयाष्टके द्वितीयः प्रपाठकः}

%3.2.1.1
तृ॒तीय॑स्यामि॒तो दि॒वि सोम॑ आसीत्। तङ्गा॑य॒त्र्याऽह॑रत्। तस्य॑ प॒र्णम॑च्छिद्यत। तत्प॒र्णो॑ऽभवत्। तत्प॒र्णस्य॑ पर्ण॒त्वम्। ब्रह्म॒ वै प॒र्णः। यत्प॑र्णशा॒खया॑ व॒त्सान॑पाक॒रोति॑। ब्रह्म॑णै॒वैना॑न॒पाक॑रोति। गा॒य॒त्रो वै प॒र्णः। गा॒य॒त्राः प॒शव॑॥१॥

%3.2.1.2
तस्मा॒त्रीणि॑त्रीणि प॒र्णस्य॑ पला॒शानि॑। त्रि॒पदा॑ गाय॒त्री। यत्प॑र्णशा॒खया॒ गाः प्रा॒र्पय॑ति। स्वयै॒वैना॑ दे॒वत॑या॒ प्रार्प॑यति। यङ्का॒मये॑ताप॒शुः स्या॒दिति॑। अ॒प॒र्णान्तस्मै॒ शुष्काग्रा॒माह॑रेत्। अ॒प॒शुरे॒व भ॑वति। यङ्का॒मये॑त पशु॒मान्त्स्या॒दिति॑। ब॒हु॒प॒र्णान्तस्मै॑ बहुशा॒खामाह॑रेत्। प॒शु॒मन्त॑मे॒वैनं॑ करोति॥२॥

%3.2.1.3
यत्प्राची॑मा॒ हरेत्। दे॒व॒लो॒कम॒भि ज॑येत्। यदुदी॑चीं मनुष्यलो॒कम्। प्राची॒मुदी॑ची॒मा ह॑रति। उ॒भयोर्लो॒कयो॑र॒भिजि॑त्यै। इ॒षे त्वो॒र्जे त्वेत्या॑ह। इष॑मे॒वोर्जं॒ यज॑माने दधाति। वा॒यव॒ स्थेत्या॑ह। वा॒युर्वा अ॒न्तरि॑क्ष॒स्याध्य॑क्षाः। अ॒न्त॒रि॒क्ष॒दे॒व॒त्या खलु॒ वै प॒शव॑॥३॥

%3.2.1.4
वा॒यव॑ ए॒वैना॒न्परि॑ ददाति। प्र वा ए॑नाने॒तदा क॑रोति। यदाह॑। वा॒यव॒ स्थेत्यु॑पा॒यव॒ स्थेत्या॑ह। यज॑मानायै॒व प॒शूनुप॑ ह्वयते। दे॒वो व॑ सवि॒ता प्रार्प॑य॒त्वित्या॑ह॒ प्रसूत्यै। श्रेष्ठ॑तमाय॒ कर्म॑ण॒ इत्या॑ह। य॒ज्ञो हि श्रेष्ठ॑तम॒ङ्कर्म॑। तस्मा॑दे॒वमा॑ह। आप्या॑यध्वमघ्निया देवभा॒गमित्या॑ह॥४॥

%3.2.1.5
व॒त्सेभ्य॑श्च॒ वा ए॒ताः पु॒रा म॑नु॒ष्येभ्य॒श्चाप्या॑यन्त। दे॒वेभ्य॑ ए॒वैना॒ इन्द्रा॒याप्या॑ययति। ऊर्ज॑स्वती॒ पय॑स्वती॒रित्या॑ह। ऊर्ज॒ हि पय॑ स॒म्भर॑न्ति। प्र॒जाव॑तीरनमी॒वा अ॑य॒क्ष्मा इत्या॑ह॒ प्रजात्यै। मा व॑ स्ते॒न ई॑शत॒ माऽघशस॒ इत्या॑ह॒ गुप्त्यै। रु॒द्रस्य॑ हे॒तिः परि॑ वो वृण॒क्त्वित्या॑ह। रु॒द्रादे॒वैनास्त्रायते। ध्रु॒वा अ॒स्मिन्गोप॑तौ स्यात ब॒ह्वीरित्या॑ह। ध्रु॒वा ए॒वास्मि॑न्ब॒ह्वीः क॑रोति॥५॥

%3.2.1.6
यज॑मानस्य प॒शून्पा॒हीत्या॑ह। प॒शू॒नाङ्गो॑पी॒थाय॑। तन्मात्सा॒यं प॒शव॒ उप॑स॒माव॑र्तन्ते। अन॑धः सादयति। गर्भा॑णा॒न्धृत्या॒ अप्र॑पादाय। तस्मा॒द्गर्भा प्र॒जाना॒मप्र॑पादुकाः। उ॒परी॑व॒ निद॑धाति। उ॒परी॑व॒ हि सु॑व॒र्गो लो॒कः। सु॒व॒र्गस्य॑ लो॒कस्य॒ सम॑ष्ट्यै॥६॥\anuvakamend[प॒शव॑ करोति प॒शवो॑ देवभा॒गमित्या॑ह करोति॒ नव॑ च]

%3.2.2.1
दे॒वस्य॑ त्वा सवि॒तुः प्र॑स॒व इत्य॑श्वप॒र्॒शुमाद॑त्ते॒ प्रसूत्यै। अ॒श्विनोर्बा॒हुभ्या॒मित्या॑ह। अ॒श्विनौ॒ हि दे॒वाना॑मध्व॒र्यू आस्ताम्। पू॒ष्णो हस्ताभ्या॒मित्या॑ह॒ यत्यै। यो वा ओष॑धीः पर्व॒शो वेद॑। नैना॒ स हि॑नस्ति। प्र॒जाप॑ति॒र्वा ओष॑धीः पर्व॒शो वे॑द। स ए॑ना॒ न हि॑नस्ति। अ॒श्व॒प॒र्श्वा ब॒र्॒हिरच्छै॑ति। प्रा॒जा॒प॒त्यो वा अश्व॑ सयोनि॒त्वाय॑॥७॥

%3.2.2.2
ओष॑धीना॒महिसायै। य॒ज्ञस्य॑ घो॒षद॒सीत्या॑ह। यज॑मान ए॒व र॒यिन्द॑धाति। प्रत्यु॑ष्ट॒ रक्ष॒ प्रत्यु॑ष्टा॒ अरा॑तय॒ इत्या॑ह। रक्ष॑सा॒मप॑हत्यै। प्रेयम॑गाद्धि॒षणा॑ ब॒र्॒हिरच्छेत्या॑ह। वि॒द्या वै धि॒षणा। वि॒द्ययै॒वैन॒दच्छै॑ति। मनु॑ना कृ॒ता स्व॒धया॒ वित॒ष्टेत्या॑ह। मा॒न॒वी हि पर्\mbox{}शु॑ स्व॒धाकृ॑ता॥८॥

%3.2.2.3
त आव॑हन्ति क॒वय॑ पु॒रस्ता॒दित्या॑ह। शु॒श्रु॒वासो॒ वै क॒वय॑। य॒ज्ञः पु॒रस्तात्। मु॒ख॒त ए॒व य॒ज्ञमा र॑भते। अथो॒ यदे॒तदु॒क्त्वा यत॒ कुत॑श्चा॒ हर॑ति। तत्प्राच्या॑ ए॒व दि॒शो भ॑वति। दे॒वेभ्यो॒ जुष्ट॑मि॒ह ब॒र्॒हिरा॒सद॒ इत्या॑ह। ब॒र्\mbox{}हिष॒ समृ॑द्ध्यै। कर्म॒णोऽन॑पराधाय। दे॒वानां परिषू॒तम॒सीत्या॑ह॥९॥

%3.2.2.4
यद्वा इ॒दङ्किं च॑। तद्दे॒वानां परिषू॒तम्। अथो॒ यथा॒ वस्य॑से प्रति॒प्रोच्याहे॒दङ्क॑रिष्या॒मीति॑। ए॒वमे॒व तद॑ध्व॒र्युर्दे॒वेभ्य॑ प्रति॒प्रोच्य॑ ब॒र्॒हिर्दा॑ति। आ॒त्मनोऽहिसायै। याव॑तः स्त॒म्बान्प॑रिदि॒शेत्। यत्तेषा॑मुच्छि॒ष्यात्। अति॒ तद्य॒ज्ञस्य॑ रेचयेत्। एक स्त॒म्बं परि॑दिशेत्। त सर्व॑न्दायात्॥१०॥

%3.2.2.5
य॒ज्ञस्यान॑तिरेकाय। व॒र्॒षवृ॑द्धम॒सीत्या॑ह। व॒र्॒षवृ॑द्धा॒ वा ओष॑धयः। देव॑बर्\mbox{}हि॒रित्या॑ह। दे॒वेभ्य॑ ए॒वैन॑त्करोति। मा त्वा॒ऽन्वङ्मा ति॒र्यगित्या॒हाहिसायै। पर्व॑ ते राध्यास॒मित्या॒हर्ध्यै। आ॒च्छे॒त्ता ते॒ मा रि॑ष॒मित्या॑ह। नास्या॒त्मनो॑ मीयते। य ए॒वं वेद॑॥११॥

%3.2.2.6
देव॑बर्\mbox{}हिः श॒तव॑ल्\mbox{}शं॒ विरो॒हेत्या॑ह। प्र॒जा वै ब॒र्॒हिः। प्र॒जानां प्र॒जन॑नाय। स॒हस्र॑वल्‌शा॒ वि व॒य रु॑हे॒मेत्या॑ह। आ॒शिष॑मे॒वैतामा शास्ते। पृ॒थि॒व्याः सं॒पृच॑ पा॒हीत्या॑ह॒ प्रति॑ष्ठित्यै। अयु॑ङ्गायुङ्गान्मु॒ष्टील्लुँ॑नोति। मि॒थु॒न॒त्वाय॒ प्रजात्यै। सु॒स॒म्भृता त्वा॒ सम्भ॑रा॒मीत्या॑ह। ब्रह्म॑णै॒वैन॒त्सम्भ॑रति॥१२॥

%3.2.2.7
अदि॑त्यै॒ रास्ना॒ऽसीत्या॑ह। इ॒यं वा अदि॑तिः। अ॒स्या ए॒वैन॒द्रास्नां करोति। इ॒न्द्रा॒ण्यै स॒न्नह॑न॒मित्या॑ह। इ॒न्द्रा॒णी वा अग्रे॑ दे॒वता॑ना॒ सम॑नह्यत। साऽऽर्ध्नोत्। ऋद्ध्यै॒ सन्न॑ह्यति। प्र॒जा वै ब॒र्॒हिः। प्र॒जाना॒मप॑रावापाय। तस्मा॒त्स्नाव॑सन्तताः प्र॒जा जा॑यन्ते॥१३॥

%3.2.2.8
पू॒षा ते ग्र॒न्थिङ्ग्र॑थ्ना॒त्वित्या॑ह। पुष्टि॑मे॒व यज॑माने दधाति। स ते॒ मास्था॒दित्या॒हाहिसायै। प॒श्चात्प्राञ्च॒मुप॑गूहति। प॒श्चाद्वै प्रा॒चीन॒ रेतो॑ धीयते। प॒श्चादे॒वास्मै प्रा॒चीन॒ रेतो॑ दधाति। इन्द्र॑स्य त्वा बा॒हुभ्या॒मुद्य॑च्छ॒ इत्या॑ह। इ॒न्द्रि॒यमे॒व यज॑माने दधाति। बृह॒स्पतेर्मू॒र्ध्ना ह॑रा॒मीत्या॑ह। ब्रह्म॒ वै दे॒वानां॒ बृह॒स्पति॑॥१४॥

%3.2.2.9
ब्रह्म॑णै॒वैन॑द्धरति। उ॒र्व॑न्तरि॑क्ष॒मन्वि॒हीत्या॑ह॒ गत्यै। दे॒व॒ङ्ग॒मम॒सीत्या॑ह। दे॒वाने॒वैन॑द्गमयति। अन॑धः सादयति। गर्भा॑णा॒न्धृत्या॒ अप्र॑पादाय। तस्मा॒द्गर्भा प्र॒जाना॒मप्र॑पादुकाः। उ॒परी॑व॒ नि द॑धाति। उ॒परी॑व॒ हि सु॑व॒र्गो लो॒कः। सु॒व॒र्गस्य॑ लो॒कस्य॒ सम॑ष्ट्यै॥१५॥\anuvakamend[स॒यो॒नि॒त्वाय॑ स्व॒धाकृ॑ता॒ऽसीत्या॑ह दाया॒द्वेद॑ भरति जायन्ते॒ बृह॒स्पति॒ सम॑ष्ट्यै]

%3.2.3.1
पू॒र्वे॒द्युरि॒ध्माब॒र्॒हिः क॑रोति। य॒ज्ञमे॒वारभ्य॑ गृही॒त्वोप॑वसति। प्र॒जाप॑तिर्य॒ज्ञम॑सृजत। तस्यो॒खे अ॑स्रसेताम्। य॒ज्ञो वै प्र॒जाप॑तिः। यत्सान्नाय्यो॒खे भव॑तः। य॒ज्ञस्यै॒व तदु॒खे उप॑दधा॒त्यप्र॑स्रसाय। शुन्ध॑ध्व॒न्दैव्या॑य॒ कर्म॑णे देवय॒ज्याया॒ इत्या॑ह। दे॒व॒य॒ज्याया॑ ए॒वैना॑नि शुन्धति। मा॒त॒रिश्व॑नो घ॒र्मो॑ऽसीत्या॑ह॥१६॥

%3.2.3.2
अ॒न्तरि॑क्षं॒ वै मा॑त॒रिश्व॑नो घ॒र्मः। ए॒षां लो॒कानां॒ विधृ॑त्यै। द्यौर॑सि पृथि॒व्य॑सीत्या॑ह। दि॒वश्च॒ ह्ये॑षा पृ॑थि॒व्याश्च॒ सम्भृ॑ता। यदु॒खा। तस्मा॑दे॒वमा॑ह। वि॒श्वधा॑या असि पर॒मेण॒ धाम्नेत्या॑ह। वृष्टि॒र्वै वि॒श्वधा॑याः। वृष्टि॑मे॒वाव॑रुन्धे। दृह॑स्व॒ मा ह्वा॒रित्या॑ह॒ धृत्यै॥१७॥

%3.2.3.3
वसू॑नां प॒वित्र॑म॒सीत्या॑ह। प्रा॒णा वै वस॑वः। तेषां॒ वा ए॒तद्भा॑ग॒धेयम्। यत्प॒वित्रम्। तेभ्य॑ ए॒वैन॑त्करोति। श॒तधा॑र स॒हस्र॑धार॒मित्या॑ह। प्रा॒णेष्वे॒वायु॑र्दधाति सर्व॒त्वाय॑। त्रि॒वृत्प॑लाशशा॒खायान्दर्भ॒मयं॑ भवति। त्रि॒वृद्वै प्रा॒णः। त्रि॒वृत॑मे॒व प्रा॒णं म॑ध्य॒तो यज॑माने दधाति॥१८॥

%3.2.3.4
सौ॒म्यः प॒र्णः स॑योनि॒त्वाय॑। सा॒क्षात्प॒वित्र॑न्द॒र्भाः। प्राख्सा॒यमधि॒नि द॑धाति। तत्प्रा॑णापा॒नयो॑ रू॒पम्। ति॒र्यक्प्रा॒तः। तद्दर्श॑स्य रू॒पम्। दा॒र्श्य ह्ये॑तदह॑। अन्नं॒ वै च॒न्द्रमा। अन्नं॑ प्रा॒णाः। उ॒भय॑मे॒वोपै॒त्यजा॑मित्वाय॥१९॥

%3.2.3.5
तस्मा॑द॒य स॒र्वत॑ पवते। हु॒तः स्तो॒को हु॒तो द्र॒प्स इत्या॑ह॒ प्रति॑ष्ठित्यै। ह॒विषोऽस्क॑न्दाय। न हि हु॒त स्वाहा॑कृत॒ स्कन्द॑ति। दि॒वि नाको॒ नामा॒ग्निः। तस्य॑ वि॒प्रुषो॑ भाग॒धेयम्। अ॒ग्नये॑ बृह॒ते नाका॒येत्या॑ह। नाक॑मे॒वाग्निं भा॑ग॒धेये॑न॒ सम॑र्धयति। स्वाहा॒ द्यावा॑पृथि॒वीभ्या॒मित्या॑ह। द्यावा॑पृथि॒व्योरे॒वैन॒त्प्रति॑ष्ठापयति॥२०॥

%3.2.3.6
प॒वित्र॑व॒त्यान॑यति। अ॒पाञ्चै॒वौष॑धीनां च॒ रस॒ ससृ॑जति। अथो॒ ओष॑धीष्वे॒व प॒शून्प्रति॑ष्ठापयति। अ॒न्वा॒रभ्य॒ वाचं॑ यच्छति। य॒ज्ञस्य॒ धृत्यै। धा॒रय॑न्नास्ते। धा॒रय॑न्त इव॒ हि दु॒हन्ति॑। काम॑धुक्ष॒ इत्या॒हातृ॒तीय॑स्यै। त्रय॑ इ॒मे लो॒काः। इ॒माने॒व लो॒कान्‌यज॑मानो दुहे॥२१॥

%3.2.3.7
अ॒मूमिति॒ नाम॑ गृह्णाति। भ॒द्रमे॒वासा॒ङ्कर्मा॒ विष्क॑रोति। सा वि॒श्वायु॒ सा वि॒श्वव्य॑चा॒ सा वि॒श्वक॒र्मेत्या॑ह। इ॒यं वै वि॒श्वायु॑। अ॒न्तरि॑क्षं वि॒श्वव्य॑चाः। अ॒सौ वि॒श्वक॑र्मा। इ॒माने॒वैताभि॑र्लो॒कान्‌ य॑थापू॒र्वन्दु॑हे। अथो॒ यथा प्रदा॒त्रे पुण्य॑मा॒शास्ते। ए॒वमे॒वैना॑ ए॒तदुप॑स्तौति। तस्मा॒त्प्रादा॒दित्यु॒न्नीय॒ वन्द॑माना उपस्तु॒वन्त॑ प॒शून्दु॑हन्ति॥२२॥

%3.2.3.8
ब॒हु दु॒ग्धीन्द्रा॑य दे॒वेभ्यो॑ ह॒विरिति॒ वाचं॒ विसृ॑जते। य॒था॒दे॒व॒तमे॒व प्रसौ॑ति। दैव्य॑स्य च मानु॒षस्य॑ च॒ व्यावृ॑त्यै। त्रिरा॑ह। त्रिष॑त्या॒ हि दे॒वाः। अवा॑चं य॒मोऽन॑न्वार॒भ्योत्त॑राः। अप॑रिमितमे॒वाव॑ रुन्धे। न दा॑रुपा॒त्रेण॑ दुह्यात्। अ॒ग्नि॒वद्वै दा॑रुपा॒त्रम्। यद्दा॑रुपा॒त्रेण॑ दु॒ह्यात्॥२३॥

%3.2.3.9
या॒तयाम्ना ह॒विषा॑ यजेत। अथो॒ खल्वा॑हुः। पु॒रो॒डाश॑मुखानि॒ वै ह॒वीषि॑। नेत इ॑तः पुरो॒डाश ह॒विषो॒ यामो॒ऽस्तीति॑। काम॑मे॒व दा॑रुपा॒त्रेण॑ दुह्यात्। शू॒द्र ए॒व न दु॑ह्यात्। अस॑तो॒ वा ए॒ष सम्भू॑तः। यच्छू॒द्रः। अह॑विरे॒व तदित्या॑हुः। यच्छू॒द्रो दोग्धीति॑॥२४॥

%3.2.3.10
अ॒ग्नि॒हो॒त्रमे॒व न दु॑ह्याच्छू॒द्रः। तद्धि नोत्पु॒नन्ति॑। य॒दा खलु॒ वै प॒वित्र॑म॒त्येति॑। अथ॒ तद्ध॒विरिति॑। संपृ॑च्यध्वमृतावरी॒रित्या॑ह। अ॒पाञ्चै॒वौष॑धीनां च॒ रस॒ स सृ॑जति। तस्मा॑द॒पाञ्चौष॑धीनां च॒ रस॒मुप॑जीवामः। म॒न्द्रा धन॑स्य सा॒तय॒ इत्या॑ह। पुष्टि॑मे॒व यज॑माने दधाति। सोमे॑न॒ त्वात॑न॒च्मीन्द्रा॑य॒ दधीत्या॑ह॥२५॥

%3.2.3.11
सोम॑मे॒वैन॑त्करोति। यो वै सोमं॑ भक्षयि॒त्वा। सं॒व॒त्स॒र सोम॒न्न पिब॑ति। पु॒न॒र्भक्ष्योऽस्य सोमपी॒थो भ॑वति। सोम॒ खलु॒ वै सान्ना॒य्यम्। य ए॒वं वि॒द्वान्त्सान्ना॒य्यं पिब॑ति। अ॒पु॒न॒र्भक्ष्योऽस्य सामपी॒थो भ॑वति। न मृ॒न्मये॒नापि॑ दध्यात्। यन्मृ॒न्मयो॑नापिद॒ध्यात्। पि॒तृ॒दे॒व॒त्य स्यात्॥२६॥

%3.2.3.12
अ॒य॒स्पा॒त्रेण॑ वा दारुपा॒त्रेण॒ वाऽपि॑ दधाति। तद्धि सदे॑वम्। उ॒द॒न्वद्भ॑वति। आपो॒ वै र॑क्षो॒घ्नीः। रक्ष॑सा॒मप॑हत्यै। अद॑स्तमसि॒ विष्ण॑वे॒ त्वे॒त्या॑ह। य॒ज्ञो वै विष्णु॑। य॒ज्ञायै॒वैन॒दद॑स्तं करोति। विष्णो॑ ह॒व्य र॑क्ष॒स्वेत्या॑ह॒ गुप्त्यै। अन॑धः सादयति। गर्भा॑णा॒न्धृत्या॒ अप्र॑पादाय। तस्मा॒द्गर्भा प्र॒जाना॒मप्र॑पादुकाः। उ॒परी॑व॒ निद॑धाति। उ॒परी॑व॒ हि सु॑व॒र्गो लो॒कः। सु॒व॒र्गस्य॑ लो॒कस्य॒ सम॑ष्ट्यै॥२७॥\anuvakamend[अ॒सीत्या॑ह॒ धृत्यै॒ यज॑माने दधा॒त्यजा॑मित्वाय स्थापयति दुहे दुहन्ति दु॒ह्याद्दोग्धीति॒ दधीत्या॑ह स्यात्सादयति॒ पञ्च॑ च]

%3.2.4.1
कर्म॑णे वान्दे॒वेभ्य॑ शकेय॒मित्या॑ह॒ शक्त्यै। य॒ज्ञस्य॒ वै सन्त॑ति॒मनु॑ प्र॒जाः प॒शवो॒ यज॑मानस्य॒ सन्ता॑यन्ते। य॒ज्ञस्य॒ विच्छि॑त्ति॒मनु॑ प्र॒जाः प॒शवो॒ यज॑मानस्य॒ विच्छि॑द्यन्ते। य॒ज्ञस्य॒ सन्त॑तिरसि य॒ज्ञस्य॑ त्वा॒ सन्त॑त्यै स्तृणामि॒ सन्त॑त्यै त्वा य॒ज्ञस्येत्याह॑व॒नीया॒त्सन्त॑नोति। यज॑मानस्य प्र॒जायै॑ पशू॒ना सन्त॑त्यै। अ॒पः प्रण॑यति। श्र॒द्धा वा आप॑। श्र॒द्धामे॒वारभ्य॑ प्र॒णीय॒ प्रच॑रति। अ॒पः प्रण॑यति। य॒ज्ञो वा आप॑॥२८॥

%3.2.4.2
य॒ज्ञमे॒वारभ्य॑ प्र॒णीय॒ प्रच॑रति। अ॒पः प्रण॑यति। वज्रो॒ वा आप॑। वज्र॑मे॒व भ्रातृ॑व्येभ्यः प्र॒हृत्य॑ प्र॒णीय॒ प्रच॑रति। अ॒पः प्रण॑यति। आपो॒ वै र॑क्षो॒घ्नीः। रक्ष॑सा॒मप॑हत्यै। अ॒पः प्रण॑यति। आपो॒ वै दे॒वानां प्रि॒यन्धाम॑। दे॒वाना॑मे॒व प्रि॒यन्धाम॑ प्र॒णीय॒ प्रच॑रति॥२९॥

%3.2.4.3
अ॒पः प्रण॑यति। आपो॒ वै सर्वा॑ दे॒वता। दे॒वता॑ ए॒वारभ्य॑ प्र॒णीय॒ प्रच॑रति। वेषा॑य॒ त्वेत्या॑ह। वेषा॑य॒ ह्ये॑नदाद॒त्ते। प्रत्यु॑ष्ट॒ रक्ष॒ प्रत्यु॑ष्टा॒ अरा॑तय॒ इत्या॑ह। रक्ष॑सा॒मप॑हत्यै। धूर॒सीत्या॑ह। ए॒ष वै धुर्यो॒ऽग्निः। तं यदनु॑पस्पृश्याती॒यात्॥३०॥

%3.2.4.4
अ॒ध्व॒र्युं च॒ यज॑मानं च॒ प्रद॑हेत्। उ॒प॒स्पृश्यात्ये॑ति। अ॒ध्व॒र्योश्च॒ यज॑मानस्य॒ चाप्र॑दाहाय। धूर्व॒ तय्योँस्मान्धूर्व॑ति॒ तन्धूर्व॒ यं व॒यन्धूर्वा॑म॒ इत्या॑ह। द्वौ वाव पुरु॑षौ। यञ्चै॒व धूर्व॑ति। यश़्चै॑न॒न्धूर्व॑ति। तावु॒भौ शु॒चाऽर्प॑यति। त्वन्दे॒वाना॑मसि॒ सस्नि॑तमं॒ पप्रि॑तमं॒ जुष्ट॑तमं॒ वह्नि॑तमन्देव॒हूत॑म॒मित्या॑ह। य॒था॒य॒जुरे॒वैतत्॥३१॥

%3.2.4.5
अह्रु॑तमसि हवि॒र्धान॒मित्या॒हानार्त्यै। दृह॑स्व॒ मा ह्वा॒रित्या॑ह॒ धृत्यै। मि॒त्रस्य॑ त्वा॒ चक्षु॑षा॒ प्रेक्ष॒ इत्या॑ह मित्र॒त्वाय॑। मा भेर्मा संवि॑क्था॒ मा त्वा॑ हिसिष॒मित्या॒हाहिसायै। यद्वै किं च॒ वातो॒ नाभि॒ वाति॑। तत्सर्वं॑ वरुणदेव॒त्यम्। उ॒रु वाता॒येत्या॑ह। अवा॑रुणमे॒वैन॑त्करोति। दे॒वस्य॑ त्वा सवि॒तुः प्र॑स॒व इत्या॑ह॒ प्रसूत्यै। अ॒श्विनोर्बा॒हुभ्या॒मित्या॑ह॥३२॥

%3.2.4.6
अ॒श्विनौ॒ हि दे॒वाना॑मध्व॒र्यू आस्ताम्। पू॒ष्णो हस्ताभ्या॒मित्या॑ह॒ यत्यै। अ॒ग्नये॒ जुष्टं॒ निर्व॑पा॒मीत्या॑ह। अ॒ग्नय॑ ए॒वैनां॒ जुष्टं॒ निर्व॑पति। त्रिर्यजु॑षा। त्रय॑ इ॒मे लो॒काः। ए॒षां लो॒काना॒माप्त्यै। तू॒ष्णीञ्च॑तु॒र्थम्। अप॑रिमितमे॒वाव॑रुन्धे। स ए॒वमे॒वानु॑पू॒र्व ह॒वीषि॒ निर्व॑पति॥३३॥

%3.2.4.7
इ॒दन्दे॒वाना॑मि॒दमु॑ नः स॒हेत्या॑ह॒ व्यावृ॑त्यै। स्फा॒त्यै त्वा॒ नारात्या॒ इत्या॑ह॒ गुप्त्यै। तम॑सीव॒ वा ए॒षोऽन्तश्च॑रति। यः प॑री॒णहि॑। सुव॑र॒भि वि ख्ये॑षं वैश्वान॒रञ्ज्योति॒रित्या॑ह। सुव॑रे॒वाभि वि प॑श्यति वैश्वान॒रञ्ज्योति॑। द्यावा॑पृथि॒वी ह॒विषि॑ गृही॒त उद॑वेपेताम्। दृह॑न्ता॒न्दुर्या॒ द्यावा॑पृथि॒व्योरित्या॑ह। गृ॒हाणा॒न्द्यावा॑पृथि॒व्योर्धृत्यै। उ॒र्व॑न्तरि॑क्ष॒मन्वि॒हीत्या॑ह॒ गत्यै। अदि॑त्यास्त्वो॒पस्थे॑ सादया॒मीत्या॑ह। इ॒यं वा अदि॑तिः। अ॒स्या ए॒वैन॑दु॒पस्थे॑ सादयति। अग्ने॑ ह॒व्य र॑क्ष॒स्वेत्या॑ह॒ गुप्त्यै॥३४॥\anuvakamend[य॒ज्ञो वा आपो॒ धाम॑ प्र॒णीय॒ प्रच॑रत्यती॒यादे॒तद्बा॒हुभ्या॒मित्या॑ह ह॒वीषि॒ निर्व॑पति॒ गत्यै॑ च॒त्वारि॑ च]

%3.2.5.1
इन्द्रो॑ वृ॒त्रम॑हन्। सो॑ऽपः। अ॒भ्य॑म्रियत। तासां॒ यन्मेध्यं॑ य॒ज्ञिय॒ सदे॑व॒मासीत्। तदपोद॑क्रामत्। ते द॒र्भा अ॑भवन्। यद्द॒र्भैर॒प उ॑त्पु॒नाति॑। या ए॒व मेध्या॑ य॒ज्ञिया॒ सदे॑वा॒ आप॑। ताभि॑रे॒वैना॒ उत्पु॑नाति। द्वाभ्या॒मुत्पु॑नाति॥३५॥

%3.2.5.2
द्वि॒पाद्यज॑मान॒ प्रति॑ष्ठित्यै। दे॒वो व॑ सवि॒तोत्पु॑ना॒त्वित्या॑ह। स॒वि॒तृप्र॑सूत ए॒वैना॒ उत्पु॑नाति। अच्छि॑द्रेण प॒वित्रे॒णेत्या॑ह। अ॒सौ वा आ॑दि॒त्योऽच्छि॑द्रं प॒वित्रम्। तेनै॒वैना॒ उत्पु॑नाति। वसो॒ सूर्य॑स्य र॒श्मिभि॒रित्या॑ह। प्रा॒णा वा आप॑। प्रा॒णा वस॑वः। प्रा॒णा र॒श्मय॑॥३६॥

%3.2.5.3
प्रा॒णैरे॒व प्रा॒णान्त्सं पृ॑णक्ति। सा॒वि॒त्रि॒यर्चा। स॒वि॒तृप्र॑सूतं मे॒ कर्मा॑स॒दिति॑। स॒वि॒तृप्र॑सूतमे॒वास्य॒ कर्म॑ भवति। प॒च्छो गा॑यत्रि॒या त्रि॑ष्षमृद्ध॒त्वाय॑। आपो॑ देवीरग्रेपुवो अग्रेगुव॒ इत्या॑ह। रू॒पमे॒वासा॑मे॒तन्म॑हि॒मानं॒ व्याच॑ष्टे। अग्र॑ इ॒मं य॒ज्ञं न॑य॒ताग्रे॑ य॒ज्ञप॑ति॒मित्या॑ह। अग्र॑ ए॒व य॒ज्ञं न॑यन्ति। अग्रे॑ य॒ज्ञप॑तिम्॥३७॥

%3.2.5.4
यु॒ष्मानिन्द्रो॑ऽवृणीत वृत्र॒तूर्ये॑ यू॒यमिन्द्र॑मवृणीध्वं वृत्र॒तूर्य॒ इत्या॑ह। वृ॒त्र ह॑ हनि॒ष्यन्निन्द्र॒ आपो॑ वव्रे। आपो॒ हेन्द्रं॑ वव्रिरे। सं॒ज्ञामे॒वासा॑मे॒तत्सामा॑नं॒ व्याच॑ष्टे। प्रोक्षि॑ता॒ स्थेत्या॑ह। तेनाप॒ प्रोक्षि॑ताः। अ॒ग्नये॑ वो॒ जुष्टं॒ प्रोक्षाम्य॒ग्नीषोमाभ्या॒मित्या॑ह। य॒था॒दे॒व॒तमे॒वैना॒न्प्रोक्ष॑ति। त्रिः प्रोक्ष॑ति। त्र्या॑वृ॒द्धि य॒ज्ञः॥३८॥

%3.2.5.5
अथो॒ रक्ष॑सा॒मप॑हत्यै। शुन्ध॑ध्वं॒ दैव्या॑य॒ कर्म॑णे देवय॒ज्याया॒ इत्या॑ह। दे॒व॒य॒ज्याया॑ ए॒वैना॑नि शुन्धति। त्रिः प्रोक्ष॑ति। त्र्या॑वृ॒द्धि य॒ज्ञः। अथो॑ मेध्य॒त्वाय॑। अव॑धूत॒ रक्षोऽव॑धूता॒ आरा॑तय॒ इत्या॑ह। रक्ष॑सा॒मप॑हत्यै। अदि॑त्या॒स्त्वग॒सीत्या॑ह। इ॒यं वा अदि॑तिः॥३९॥

%3.2.5.6
अ॒स्या ए॒वैन॒त्त्वचं॑ करोति। प्रति॑ त्वा पृथि॒वी वे॒त्त्वित्या॑ह॒ प्रति॑ष्ठित्यै। पु॒रस्तात्प्रती॒चीन॑ग्रीव॒मुत्त॑रलो॒मोप॑स्तृणाति मेध्य॒त्वाय॑। तस्मात्पु॒रस्तात्प्र॒त्यञ्च॑ प॒शवो॒ मेध॒मुप॑तिष्ठन्ते। तस्मात्प्र॒जा मृ॒गं ग्राहु॑काः। य॒ज्ञो दे॒वेभ्यो॒ निला॑यत। कृष्णो॑ रू॒पं कृ॒त्वा। यत्कृ॑ष्णाजि॒ने ह॒विर॑ध्यव॒हन्ति॑। य॒ज्ञादे॒व तद्य॒ज्ञं प्रयु॑ङ्क्ते। ह॒विषोऽस्क॑न्दाय॥४०॥

%3.2.5.7
अ॒धि॒षव॑णमसि वानस्प॒त्यमित्या॑ह। अ॒धि॒षव॑णमे॒वैन॑त्करोति। प्रति॒ त्वाऽदि॑त्या॒स्त्वग्वे॒त्त्वित्या॑ह सय॒त्वाय॑। अ॒ग्नेस्त॒नूर॒सीत्या॑ह। अ॒ग्नेर्वा ए॒षा त॒नूः। यदोष॑धयः। वा॒चो वि॒सर्ज॑न॒मित्या॑ह। य॒दा हि प्र॒जा ओष॑धीनाम॒श्ञन्ति॑। अथ॒ वाचं॒ विसृ॑जन्ते। दे॒ववी॑तये त्वा गृह्णा॒मीत्या॑ह॥४१॥

%3.2.5.8
दे॒वता॑भिरे॒वैन॒त्सम॑र्धयति। अद्रि॑रसि वानस्प॒त्य इत्या॑ह। ग्रावा॑णमे॒वैन॑त्करोति। स इ॒दं दे॒वेभ्यो॑ ह॒व्य सु॒शमि॑ शमि॒ष्वेत्या॑ह॒ शान्त्यै। हवि॑ष्कृ॒देहीत्या॑ह। य ए॒व दे॒वाना हवि॒ष्कृत॑। तान्‌ ह्व॑यति। त्रिर्ह्व॑यति। त्रिष॑त्या॒ हि दे॒वाः। इष॒माव॒दोर्ज॒माव॒देत्या॑ह॥४२॥

%3.2.5.9
इष॑मे॒वोर्जं॒ यज॑माने दधाति। द्यु॒मद्व॑दत व॒य सं॑घा॒तं जे॒ष्मेत्या॑ह॒ भ्रातृ॑व्याभिभूत्यै। मनो श्र॒द्धादे॑वस्य॒ यज॑मानस्यासुर॒घ्नी वाक्। य॒ज्ञा॒यु॒धेषु॒ प्रवि॑ष्टाऽऽसीत्। तेऽसु॑रा॒ याव॑न्तो यज्ञायु॒धाना॑मु॒द्वद॑तामु॒पाशृ॑ण्वन्। ते परा॑भवन्। तस्मा॒त्स्वानां॒ मध्ये॑ऽव॒साय॑ यजेत। याव॑न्तोऽस्य॒ भ्रातृ॑व्या यज्ञायु॒धाना॑मु॒द्वद॑तामुपशृ॒ण्वन्ति॑। ते परा॑ भवन्ति। उ॒च्चैः स॒माह॑न्त॒ वा आ॑ह॒ विजि॑त्यै॥४३॥

%3.2.5.10
वृ॒ङ्क्त ए॑षामिन्द्रि॒यं वी॒र्यम्। श्रेष्ठ॑ एषां भवति। व॒र्\mbox{}॒षवृ॑द्धमसि॒ प्रति॑ त्वा व॒र्\mbox{}॒षवृ॑द्धं वे॒त्त्वित्या॑ह। व॒र्\mbox{}॒षवृ॑द्धा॒ वा ओष॑धयः। व॒र्\mbox{}॒षवृ॑द्धा इ॒षीका॒ समृ॑द्ध्यै। य॒ज्ञ रक्षा॒स्यनु॒ प्रावि॑शन्। तान्य॒स्ना प॒शुभ्यो॑ नि॒रवा॑दयन्त। तुषै॒रोष॑धीभ्यः। परा॑पूत॒ रक्ष॒ परा॑पूता॒ अरा॑तय॒ इत्या॑ह। रक्ष॑सा॒मप॑हत्यै॥४४॥

%3.2.5.11
रक्ष॑सां भा॒गो॑ऽसीत्या॑ह। तुषै॑रे॒व रक्षासि नि॒रव॑दयते। अ॒प उप॑स्पृशति मेध्य॒त्वाय॑। वा॒युर्वो॒ विवि॑न॒क्त्वित्या॑ह। प॒वित्रं॒ वै वा॒युः। पु॒नात्ये॒वैनान्॑। अ॒न्तरि॑क्षादिव॒ वा ए॒ते प्रस्क॑न्दन्ति। ये शूर्पात्। दे॒वो व॑ सवि॒ता हिर॑ण्यपाणि॒ प्रति॑गृह्णा॒त्वित्या॑ह॒ प्रति॑ष्ठित्यै। ह॒विषोऽस्क॑न्दाय। त्रिष्फ॒लीक॑र्त॒वा आ॑ह। त्र्या॑वृ॒द्धि य॒ज्ञः। अथो॑ मेध्य॒त्वाय॑॥४५॥\anuvakamend[द्वाभ्या॒मुत्पु॑नाति र॒श्मयो॑ नय॒न्त्यग्रे॑ य॒ज्ञप॑तिं य॒ज्ञोऽदि॑ति॒रस्क॑न्दाय गृह्णा॒मीत्या॑ह व॒देत्या॑ह॒ विजि॑त्या॒ अप॑हत्या॒ अस्क॑न्दाय॒ त्रीणि॑ च]

%3.2.6.1
अव॑धूत॒ रक्षोऽव॑धूता॒ अरा॑तय॒ इत्या॑ह। रक्ष॑सा॒मप॑हत्यै। अदि॑त्या॒स्त्वग॒सीत्या॑ह। इ॒यं वा अदि॑तिः। अ॒स्या ए॒वैन॒त्त्वचं॑ करोति। प्रति॑ त्वा पृथि॒वी वे॒त्त्वित्या॑ह॒ प्रति॑ष्ठित्यै। पु॒रस्तात्प्रती॒चीन॑ग्रीव॒मुत्त॑रलो॒मोप॑स्तृणाति मेध्य॒त्वाय॑। तस्मात्पु॒रस्तात्प्र॒त्यञ्च॑ प॒शवो॒ मेध॒मुप॑तिष्ठन्ते। तस्मात्प्र॒जा मृ॒गं ग्राहु॑काः। य॒ज्ञो दे॒वेभ्यो॒ निला॑यत॥४६॥

%3.2.6.2
कृष्णो॑ रू॒पं कृ॒त्वा। यत्कृ॑ष्णाजि॒ने ह॒विर॑धिपि॒नष्टि॑। य॒ज्ञादे॒व तद्य॒ज्ञं प्रयु॑ङ्क्ते। ह॒विषोऽस्क॑न्दाय। द्यावा॑पृथि॒वी स॒हास्ताम्। ते श॑म्यामा॒त्रमेक॒मह॒र्व्यैता शम्यामा॒त्रमेक॒मह॑। दि॒वः स्क॑म्भ॒निर॑सि॒ प्रति॒ त्वाऽदि॑त्या॒स्त्वग्वे॒त्त्वित्या॑ह। द्यावा॑पृथि॒व्योर्वीत्यै। धि॒षणा॑ऽसि पर्व॒त्या प्रति॑ त्वा दि॒वः स्क॑म्भ॒निर्वे॒त्त्वित्या॑ह। द्यावा॑पृथि॒व्योर्विधृ॑त्यै॥४७॥

%3.2.6.3
धि॒षणा॑ऽसि पार्वते॒यी प्रति॑ त्वा पर्व॒तिर्वे॒त्त्वित्या॑ह। द्यावा॑पृथि॒व्योर्धृत्यै। दे॒वस्य॑ त्वा सवि॒तुः प्र॑स॒व इत्या॑ह॒ प्रसूत्यै। अ॒श्विनोर्बा॒हुभ्या॒मित्या॑ह। अ॒श्वि॒नौ हि दे॒वाना॑मध्व॒र्यू आस्ताम्। पू॒ष्णो हस्ताभ्या॒मित्या॑ह॒ यत्त्ये। अधि॑वपा॒मीत्या॑ह। य॒था॒दे॒व॒तमे॒वैना॒नधि॑ वपति। धा॒न्य॑मसि धिनु॒हि दे॒वानित्या॑ह। ए॒तस्य॒ यजु॑षो वी॒र्ये॑ण॥४८॥

%3.2.6.4
याव॒देका॑ दे॒वता॑ का॒मय॑ते॒ याव॒देका। ताव॒दाहु॑तिः प्रथते। न हि तदस्ति॑। यत्ताव॑दे॒व स्यात्। याव॑ज्जु॒होति॑। प्रा॒णाय॑ त्वाऽपा॒नाय॒ त्वेत्या॑ह। प्रा॒णाने॒व यज॑माने दधाति। दी॒र्घामनु॒ प्रसि॑ति॒मायु॑षे धा॒मित्या॑ह। आयु॑रे॒वास्मि॑न्दधाति। अ॒न्तरि॑क्षादिव॒ वा ए॒तानि॒ प्रस्क॑न्दन्ति। यानि॑ दृ॒षद॑। दे॒वो व॑ सवि॒ता हिर॑ण्यपाणि॒ प्रति॑गृह्णा॒त्वित्या॑ह॒ प्रति॑ष्ठित्यै। ह॒विषोऽस्क॑न्दाय। असं॑वपन्ती पिषा॒णूनि॑ कुरुता॒दित्या॑ह मेध्य॒त्वाय॑॥४९॥\anuvakamend[निला॑यत॒ विधृ॑त्यै वी॒र्ये॑ण स्कन्दन्ति च॒त्वारि॑ च]

%3.2.7.1
धृष्टि॑रसि॒ ब्रह्म॑ य॒च्छेत्या॑ह॒ धृत्यै। अपाग्ने॒ऽग्निमा॒मादं॑ जहि॒ निष्क्र॒व्याद से॒धा दे॑व॒यजं॑ व॒हेत्या॑ह। य ए॒वामात्क्र॒व्यात्। तम॑प॒हत्य॑। मेध्ये॒ऽग्नौ क॒पाल॒मुप॑दधाति। निर्द॑ग्ध॒ रक्षो॒ निर्द॑ग्धा॒ अरा॑तय॒ इत्या॑ह। रक्षास्ये॒व निर्द॑हति। अ॒ग्नि॒वत्युप॑दधाति। अ॒स्मिन्ने॒व लो॒के ज्योति॑र्धत्ते। अङ्गा॑र॒मधि॑ वर्तयति॥५०॥

%3.2.7.2
अ॒न्तरि॑क्ष ए॒व ज्योति॑र्धत्ते। आ॒दि॒त्यमे॒वामुष्मि॑ल्लोँ॒के ज्योति॑र्धत्ते। ज्योति॑ष्मन्तोऽस्मा इ॒मे लो॒का भ॑वन्ति। य ए॒वं वेद॑। ध्रु॒वम॑सि पृथि॒वीं दृ॒हेत्या॑ह। पृ॒थि॒वीमे॒वैतेन॑ दृ॒हति। ध॒र्त्रम॑स्य॒न्तरि॑क्षं दृ॒हेत्या॑ह। अ॒न्तरि॑क्षमे॒वैतेन॑ दृहति। ध॒रुण॑मसि॒ दिवं॑ दृ॒हेत्या॑ह। दिव॑मे॒वैतेन॑ दृहति॥५१॥

%3.2.7.3
धर्मा॑सि॒ दिशो॑ दृ॒हेत्या॑ह। दिश॑ ए॒वैतेन॑ दृहति। इ॒माने॒वैतैर्लो॒कान्दृहति। दृह॑न्तेऽस्मा इ॒मे लो॒काः प्र॒जया॑ प॒शुभि॑। य ए॒वं वेद॑। त्रीण्यग्रे॑ क॒पाला॒न्युप॑दधाति। त्रय॑ इ॒मे लो॒काः। ए॒षां लो॒काना॒माप्त्यै। एक॒मग्रे॑ क॒पाल॒मुप॑ दधाति। एकं॒ वा अग्रे॑ क॒पालं॒ पुरु॑षस्य स॒म्भव॑ति॥५२॥

%3.2.7.4
अथ॒ द्वे। अथ॒ त्रीणि॑। अथ॑ च॒त्वारि॑। अथा॒ष्टौ। तस्मा॑द॒ष्टाक॑पालं॒ पुरु॑षस्य॒ शिर॑। यदे॒वं क॒पालान्युप॒दधा॑ति। य॒ज्ञो वै प्र॒जाप॑तिः। य॒ज्ञमे॒व प्र॒जाप॑ति॒ सस्क॑रोति। आ॒त्मान॑मे॒व तत्सस्क॑रोति। त सस्कृ॑तमा॒त्मानम्॥५३॥

%3.2.7.5
अ॒मुष्मि॑ल्लोँ॒केऽनु॒ परै॑ति। यद॒ष्टावु॑प॒दधा॑ति। गा॒य॒त्रि॒या तत्सम्मि॑तम्। यन्नव॑। त्रि॒वृता॒ तत्। यद्दश॑। वि॒राजा॒ तत्। यदेका॑दश। त्रि॒ष्टुभा॒ तत्। यद्द्वाद॑श॥५४॥

%3.2.7.6
जग॑त्या॒ तत्। छन्द॑ सम्मितानि॒ स उ॑प॒दध॑त्क॒पाला॑नि। इ॒माल्लोँ॒कान॑नुपू॒र्वन्दिशो॒ विधृ॑त्यै दृहति। अथायु॑ प्रा॒णान्प्र॒जां प॒शून् यज॑माने दधाति। स॒जा॒तान॑स्मा अ॒भितो॑ बहु॒लान्क॑रोति। चित॒ स्थेत्या॑ह। य॒था॒य॒जुरे॒वैतत्। भृगू॑णा॒मङ्गि॑रसा॒न्तप॑सा तप्यध्व॒मित्या॑ह। दे॒वता॑नामे॒वैना॑नि॒ तप॑सा तपति। तानि॒ तत॒ सस्थि॑ते। यानि॑ घ॒र्मे क॒पालान्युपचि॒न्वन्ति॑ वे॒धस॒ इति॒ चतु॑ष्पदय॒र्चा वि मु॑ञ्चति। चतु॑ष्पादः प॒शव॑। प॒शुष्वे॒वोपरि॑ष्टा॒त्प्रति॑ तिष्ठति॥५५॥\anuvakamend[व॒र्त॒य॒ति॒ दिव॑मे॒वैतेन॑ दृहति स॒म्भव॑ति॒ त सस्कृ॑तमा॒त्मान॒न्द्वाद॑श॒ सस्थि॑ते॒ त्रीणि॑ च]

%3.2.8.1
दे॒वस्य॑ त्वा सवि॒तुः प्र॑स॒व इत्या॑ह॒ प्रसूत्यै। अ॒श्विनोर्बा॒हुभ्या॒मित्या॑ह। अ॒श्विनौ॒ हि दे॒वाना॑मध्व॒र्यू आस्ताम्। पू॒ष्णो हस्ताभ्या॒मित्या॑ह॒ यत्यै। सं व॑पा॒मीत्या॑ह। य॒था॒दे॒व॒तमे॒वैना॑नि॒ संव॑पति। समापो॑ अ॒द्भिर॑ग्मत॒ समोष॑धयो॒ रसे॒नेत्या॑ह। आपो॒ वा ओष॑धीर्जिन्वन्ति। ओष॑धयो॒ऽपो जि॑न्वन्ति। अ॒न्या वा ए॒तासा॑म॒न्या जि॑न्वन्ति॥५६॥

%3.2.8.2
तस्मा॑दे॒वमा॑ह। स रे॒वती॒र्जग॑तीभि॒र्मधु॑मती॒र्मधु॑मतीभिः सृज्यध्व॒मित्या॑ह। आपो॒ वै रे॒वती। प॒शवो॒ जग॑तीः। ओष॑धयो॒ मधु॑मतीः। आप॒ ओष॑धीः प॒शून्। ताने॒वास्मा॑ एक॒धा स॒सृज्य॑। मधु॑मतः करोति। अ॒द्भ्यः परि॒ प्रजा॑ताः स्थ॒ सम॒द्भिः पृ॑च्यध्व॒मिति॑ प॒र्याप्ला॑वयति। यथा॒ सुवृ॑ष्ट इ॒माम॑नुवि॒सृत्य॑॥५७॥

%3.2.8.3
आप॒ ओष॑धीर्म॒हय॑न्ति। ता॒दृगे॒व तत्। जन॑यत्यै त्वा॒ संयौ॒मीत्या॑ह। प्र॒जा ए॒वैतेन॑ दाधार। अ॒ग्नये त्वा॒ऽग्नीषोमाभ्या॒मित्या॑ह॒ व्यावृ॑त्त्यै। म॒खस्य॒ शिरो॒ऽसीत्या॑ह। य॒ज्ञो वै म॒खः। तस्यै॒तच्छिर॑। यत्पु॑रो॒डाश॑। तस्मा॑दे॒वमा॑ह॥५८॥

%3.2.8.4
घ॒र्मो॑ऽसि वि॒श्वायु॒रित्या॑ह। विश्व॑मे॒वायु॒र्यज॑माने दधाति। उ॒रु प्र॑थस्वो॒रु ते॑ य॒ज्ञप॑तिः प्रथता॒मित्या॑ह। यज॑मानमे॒व प्र॒जया॑ प॒शुभि॑ प्रथयति। त्वचं॑ गृह्णी॒ष्वेत्या॑ह। सर्व॑मे॒वैन॒ सत॑नुं करोति। अथा॒प आ॒नीय॒ परि॑मार्ष्टि। मा॒स ए॒व तत्त्वचं॑ दधाति। तस्मात्त्व॒चा मा॒सं छ॒न्नम्। घ॒र्मो वा ए॒षोऽशान्तः ॥५९॥

%3.2.8.5
अ॒र्ध॒मा॒सेऽर्धमासे॒ प्रवृ॑ज्यते। यत्पु॑रो॒डाश॑। स ईश्व॒रो यज॑मान शु॒चा प्र॒दह॑। पर्य॑ग्नि करोति। प॒शुमे॒वैन॑मकः। शान्त्या॒ अप्र॑दाहाय। त्रिः पर्य॑ग्नि करोति। त्र्या॑वृ॒द्धि य॒ज्ञः। अथो॒ रक्ष॑सा॒मप॑हत्यै। अ॒न्तरि॑त॒ रक्षो॒ऽन्तरि॑ता॒ अरा॑तय॒ इत्या॑ह॥६०॥

%3.2.8.6
रक्ष॑साम॒न्तर्\mbox{}हि॑त्यै। पु॒रो॒डाशं॒ वा अधि॑श्रित॒ रक्षास्यजिघासन्। दि॒वि नाको॒ नामा॒ग्नी र॑क्षो॒हा। स ए॒वास्मा॒द्रक्षा॒स्यपा॑हन्। दे॒वस्त्वा॑ सवि॒ता श्र॑पय॒त्वित्या॑ह। स॒वि॒तृप्र॑सूत ए॒वैन श्रपयति। वर्\mbox{}षि॑ष्ठे॒ अधि॒ नाक॒ इत्या॑ह। रक्ष॑सा॒मप॑हत्यै। अ॒ग्निस्ते॑ त॒नुवं॒ माऽति॑धा॒गित्या॒हाऽन॑तिदाहाय। अग्ने॑ ह॒व्य र॑क्ष॒स्वेत्या॑ह॒ गुप्त्यै॥६१॥

%3.2.8.7
अवि॑दहन्तः श्रपय॒तेति॒ वाचं॒ विसृ॑जते। य॒ज्ञमे॒व ह॒वीष्य॑भिव्या॒हृत्य॒ प्रत॑नुते। पु॒रो॒रुच॒मवि॑दाहाय॒ शृत्त्यै॑ करोति। म॒स्तिष्को॒ वै पु॑रो॒डाश॑। तं यन्नाभि॑ वा॒सयेत्। आ॒विर्म॒स्तिष्क॑ स्यात्। अ॒भिवा॑सयति। तस्मा॒द्गुहा॑ म॒स्तिष्क॑। भस्म॑ना॒ऽभिवा॑सयति। तस्मान्मा॒सेनास्थि॑ छ॒न्नम्॥६२॥

%3.2.8.8
वे॒देना॒भिवा॑सयति। तस्मा॒त्केशै॒ शिर॑श्छ॒न्नम्। अख॑लतिभावुको भवति। य ए॒वं वेद॑। प॒शोर्वै प्र॑ति॒मा पु॑रो॒डाश॑। स नाय॒जुष्क॑मभि॒वास्य॑। वृथे॑व स्यात्। ई॒श्व॒रा यज॑मानस्य प॒शव॒ प्रमे॑तोः। सं ब्रह्म॑णा पृच्य॒स्वेत्या॑ह। प्रा॒णा वै ब्रह्म॑॥६३॥

%3.2.8.9
प्रा॒णाः प॒शव॑। प्रा॒णैरे॒व प॒शून्त्संपृ॑णक्ति। न प्र॒मायु॑का भवन्ति। यज॑मानो॒ वै पु॑रो॒डाश॑। प्र॒जा प॒शव॒ पुरी॑षम्। यदे॒वम॑भिघा॒रय॑ति। यज॑मानमे॒व प्र॒जया॑ प॒शुभि॒ सम॑र्धयति। दे॒वा वै ह॒विर्भृ॒त्वाऽब्रु॑वन्। कस्मि॑न्नि॒दं म्र॑क्ष्यामह॒ इति॑। सोऽग्निर॑ब्रवीत्॥६४॥

%3.2.8.10
मयि॑ त॒नूः सं निध॑ध्वम्। अ॒हं व॒स्तं ज॑नयिष्यामि। यस्मि॑न्म्र॒क्ष्यध्व॒ इति॑। ते दे॒वा अ॒ग्नौ त॒नूः संन्य॑दधत। तस्मा॑दाहुः। अ॒ग्निः सर्वा॑ दे॒वता॒ इति॑। सोऽङ्गा॑रेणा॒पः। अ॒भ्य॑पातयत्। तत॑ एक॒तो॑ऽजायत। स द्वि॒तीय॑म॒भ्य॑पातयत्॥६५॥

%3.2.8.11
ततो द्वि॒तो॑ऽजायत। स तृ॒तीय॑म॒भ्य॑पातयत्। तत॑स्त्रि॒तो॑ऽजायत। यद॒द्भ्योऽजा॑यन्त। तदा॒प्याना॑माप्य॒त्वम्। यदा॒त्मभ्योऽजा॑यन्त। तदा॒त्म्याना॑मात्म्य॒त्वम्। ते दे॒वा आ॒प्येष्व॑मृजत। आ॒प्या अ॑मृजत॒ सूर्याभ्युदिते। सूर्याभ्युदित॒ सूर्या॑भिनिम्रुक्ते॥६६॥

%3.2.8.12
सूर्या॑भिनिम्रुक्तः कुन॒खिनि॑। कु॒न॒खी श्या॒वद॑ति। श्या॒वद॑न्नग्रदिधि॒षौ। अ॒ग्र॒दि॒धि॒षुः प॑रिवि॒त्ते। प॒रि॒वि॒त्तो वी॑र॒हणि॑। वी॒र॒हा ब्र॑ह्म॒हणि॑। तद्ब्र॑ह्म॒हणं॒ नात्य॑च्यवत। अ॒न्त॒र्वे॒दि निन॑य॒त्यव॑रुध्यै। उल्मु॑केना॒भि गृ॑ह्णाति शृत॒त्वाय॑। शृ॒तका॑मा इव॒ हि दे॒वाः॥६७॥\anuvakamend[अ॒न्या जि॑न्वन्त्यनु वि॒सृत्यै॒वमा॒हाशान्त आह॒ गुप्त्यै॑ छ॒न्नं ब्रह्माब्रवीद्द्वि॒तीय॑म॒भ्य॑पातय॒त्सूर्या॑भिनिम्रुक्ते दे॒वाः]

%3.2.9.1
दे॒वस्य॑ त्वा सवि॒तुः प्र॑स॒व इति॒ स्प्यमाद॑त्ते॒ प्रसूत्यै। अ॒श्विनोर्बा॒हुभ्या॒मित्या॑ह। अ॒श्विनौ॒ हि दे॒वाना॑मध्व॒र्यू आस्ताम्। पू॒ष्णो हस्ताभ्या॒मित्या॑ह॒ यत्यै। आद॑द॒ इन्द्र॑स्य बा॒हुर॑सि॒ दक्षि॑ण॒ इत्या॑ह। इ॒न्द्रि॒यमे॒व यज॑माने दधाति। स॒हस्र॑भृष्टिः श॒तते॑जा॒ इत्या॑ह। रू॒पमे॒वास्यै॒तन्म॑हि॒मानं॒ व्याच॑ष्टे। वा॒युर॑सि ति॒ग्मते॑जा॒ इत्या॑ह। तेजो॒ वै वा॒युः॥६८॥

%3.2.9.2
तेज॑ ए॒वास्मि॑न्दधाति। वि॒षाद्वै नामा॑सु॒र आ॑सीत्। सो॑ऽबिभेत्। य॒ज्ञेन॑ मा दे॒वा अ॒भिभ॑विष्य॒न्तीति॑। स पृ॑थि॒वीम॒भ्य॑वमीत्। सा मे॒ध्याऽभ॑वत्। अथो॒ यदिन्द्रो॑ वृ॒त्रमह\sn{}। तस्य॒ लोहि॑तं पृथि॒वीमनु॒ व्य॑धावत्। सा मे॒ध्याऽभ॑वत्। पृथि॑वि देवयज॒नीत्या॑ह॥६९॥

%3.2.9.3
मेध्या॑मे॒वैनां देव॒यज॑नीं करोति। ओष॑ध्यास्ते॒ मूलं॒ मा हिसिष॒मित्या॑ह। ओष॑धीना॒महिसायै। व्र॒जं ग॑च्छ गो॒स्थान॒मित्या॑ह। छन्दासि॒ वै व्र॒जो गो॒स्थान॑। छन्दास्ये॒वास्मै व्र॒जं गो॒स्थानं॑ करोति। वर्\mbox{}ष॑तु ते॒ द्यौरित्या॑ह। वृष्टि॒र्वै द्यौः। वृष्टि॑मे॒वाव॑ रुन्धे। ब॒धा॒न दे॑व सवितः पर॒मस्यां परा॒वतीत्या॑ह॥७०॥

%3.2.9.4
द्वौ वाव पुरु॑षौ। यं चै॒व द्वेष्टि॑। यश्चै॑नं॒ द्वेष्टि॑। तावु॒भौ ब॑ध्नाति पर॒मस्यां परा॒वति॑ श॒तेन॒ पाशै। योऽस्मान्द्वेष्टि॒ यं च॑ व॒यं द्वि॒ष्मस्तमतो॒ मा मौ॒गित्या॒हानि॑म्रुक्त्ये। अ॒ररु॒र्वै नामा॑सु॒र आ॑सीत्। स पृ॑थि॒व्यामुप॑म्लुप्तोऽशयत्। तं दे॒वा अप॑हतो॒ऽररु॑ पृथि॒व्या इति॑ पृथि॒व्या अपाघ्नन्। भ्रातृ॑व्यो॒ वा अ॒ररु॑। अप॑हतो॒ऽररु॑ पृथि॒व्या इति॒ यदाह॑॥७१॥

%3.2.9.5
भ्रातृ॑व्यमे॒व पृ॑थि॒व्या अप॑हन्ति। ते॑ऽमन्यन्त। दिवं॒ वा अ॒यमि॒तः प॑तिष्य॒तीति॑। तम॒ररु॑स्ते॒ दिवं॒ माऽस्का॒निति॑ दि॒वः पर्य॑बाधन्त। भ्रातृ॑व्यो॒ वा अ॒ररु॑। अ॒ररु॑स्ते॒ दिवं॒ मा स्का॒निति॒ यदाह॑। भ्रातृ॑व्यमे॒व दि॒वः परि॑बाधते। स्त॒म्ब॒य॒जुर्‌ह॑रति। पृ॒थि॒व्या ए॒व भ्रातृ॑व्य॒मप॑हन्ति। द्वि॒तीय हरति॥७२॥

%3.2.9.6
अ॒न्तरि॑क्षादे॒वैन॒मप॑हन्ति। तृ॒तीय हरति। दि॒व ए॒वैन॒मप॑हन्ति। तू॒ष्णीं च॑तु॒र्थ ह॑रति। अप॑रिमितादे॒वैन॒मप॑हन्ति। असु॑राणां॒ वा इ॒यमग्र॑ आसीत्। याव॒दासी॑नः परा॒पश्य॑ति। ताव॑द्दे॒वानाम्। ते दे॒वा अ॑ब्रुवन्। अस्त्वे॒व नो॒ऽस्यामपीति॑॥७३॥

%3.2.9.7
क्य॑न्नो दास्य॒थेति॑। याव॑त्स्व॒यं प॑रिगृह्णी॒थेति॑। ते वस॑व॒स्त्वेति॑ दक्षिण॒तः पर्य॑गृह्णन्। रु॒द्रास्त्वेति॑ प॒श्चात्। आ॒दि॒त्यास्त्वेत्यु॑त्तर॒तः। तेऽग्निना॒ प्राञ्चो॑ऽजयन्। वसु॑भिर्दक्षि॒णा। रु॒द्रैः प्र॒त्यञ्च॑। आ॒दि॒त्यैरुद॑ञ्चः। यस्यै॒वं वि॒दुषो॒ वेदिं॑ परिगृ॒ह्णन्ति॑॥७४॥

%3.2.9.8
भव॑त्या॒त्मना। पराऽस्य॒ भ्रातृ॑व्यो भवति। दे॒वस्य॑ सवि॒तुः स॒व इत्या॑ह॒ प्रसूत्यै। कर्म॑ कृण्वन्ति वे॒धस॒ इत्या॑ह। इ॒षि॒त हि कर्म॑ क्रि॒यते। पृ॒थि॒व्यै मेध्यं॑ चामे॒ध्यं च॒ व्युद॑क्रामताम्। प्रा॒चीन॑मुदी॒चीनं॒ मेध्यम्। प्र॒ती॒चीनं॑ दक्षि॒णाऽमे॒ध्यम्। प्राची॒मुदी॑चीं प्रव॒णां क॑रोति। मेध्या॑मे॒वैनां देव॒यज॑नीं करोति॥७५॥

%3.2.9.9
प्राञ्चौ॑ वेद्य॒सावुन्न॑यति। आ॒ह॒व॒नीय॑स्य॒ परि॑गृहीत्यै। प्र॒तीची॒ श्रोणी। गार्ह॑पत्यस्य॒ परि॑गृहीत्यै। अथो॑ मिथुन॒त्वाय॑। उद्ध॑न्ति। यदे॒वास्या॑ अमे॒ध्यम्। तदप॑हन्ति। उद्ध॑न्ति। तस्मा॒दोष॑धय॒ परा॑भवन्ति॥७६॥

%3.2.9.10
मूलं॑ छिनत्ति। भ्रातृ॑व्यस्यै॒व मूलं॑ छिनत्ति। मूलं॒ वा अ॑ति॒तिष्ठ॒द्रक्षा॒स्यनूत्पि॑पते। यद्धस्ते॑न छि॒न्द्यात्। कु॒न॒खिनी प्र॒जाः स्यु॑। स्प्येन॑ छिनत्ति। वज्रो॒ वै स्प्यः। वज्रे॑णै॒व य॒ज्ञाद्रक्षा॒स्यप॑हन्ति। पि॒तृ॒दे॒व॒त्याऽति॑खाता। इय॑तीं खनति॥७७॥

%3.2.9.11
प्र॒जाप॑तिना यज्ञमु॒खेन॒ सम्मि॑ताम्। वेदि॑र्दे॒वेभ्यो॒ निला॑यत। तां च॑तुरङ्गु॒लेऽन्व॑विन्दन्। तस्माच्चतुरङ्गु॒लं खेया। च॒तु॒र॒ङ्गु॒लं ख॑नति। च॒तु॒र॒ङ्गु॒ले ह्योष॑धयः प्रति॒तिष्ठ॑न्ति। आ प्र॑ति॒ष्ठायै॑ खनति। यज॑मानमे॒व प्र॑ति॒ष्ठां ग॑मयति। द॒क्षि॒ण॒तो वर्\mbox{}षी॑यसीं करोति। दे॒व॒यज॑नस्यै॒व रू॒पम॑कः॥७८॥

%3.2.9.12
पुरी॑षवतीं करोति। प्र॒जा वै प॒शव॒ पुरी॑षम्। प्र॒जयै॒वैनं॑ प॒शुभि॒ पुरी॑षवन्तं करोति। उत्त॑रं परिग्रा॒हं परि॑गृह्णाति। ए॒ताव॑ती॒ वै पृ॑थि॒वी। याव॑ती॒ वेदि॑। तस्या॑ ए॒तावत॑ ए॒व भ्रातृ॑व्यं नि॒र्भज्य॑। आ॒त्मन॒ उत्त॑रं परिग्रा॒हं परि॑गृह्णाति। ऋ॒तम॑स्यृत॒सद॑नमस्यृत॒श्रीर॒सीत्या॑ह। य॒था॒य॒जुरे॒वैतत्॥७९॥

%3.2.9.13
क्रू॒रमि॑व॒ वा ए॒तत्क॑रोति। यद्वेदिं॑ क॒रोति॑। धा अ॑सि स्व॒धा अ॒सीति॑ योयुप्यते॒ शान्त्यै। उ॒र्वी चासि॒ वस्वी॑ चा॒सीत्या॑ह। उ॒र्वीमे॒वैनां॒ वस्वीं करोति। पु॒रा क्रू॒रस्य॑ वि॒सृपो॑ विरप्शि॒न्नित्या॑ह मेध्य॒त्वाय॑। उ॒दा॒दाय॑ पृथि॒वीं जी॒रदा॑नु॒र्यामैर॑यञ्च॒न्द्रम॑सि स्व॒धाभि॒रित्या॑ह। यदे॒वास्या॑ अमे॒ध्यम्। तद॑प॒हत्य॑। मेध्यां देव॒यज॑नीं कृ॒त्वा॥८०॥

%3.2.9.14
यद॒दश्च॒न्द्रम॑सि॒ मेध्यम्। तद॒स्यामेर॑यति। तां धीरा॑सो अनु॒दृश्य॑ यजन्त॒ इत्या॒हानु॑ख्यात्यै। प्रोक्ष॑णी॒रा सा॑दय। इ॒ध्माब॒र्\mbox{}हिरुप॑सादय। स्रु॒वं च॒ स्रुच॑श्च॒ संमृ॑ड्ढि। पत्नी॒ संन॑ह्य। आज्ये॑नो॒देहीत्या॑हानुपू॒र्वता॑यै। प्रोक्ष॑णी॒रा सा॑दयति। आपो॒ वै र॑क्षो॒घ्नीः॥८१॥

%3.2.9.15
रक्ष॑सा॒मप॑हत्यै। स्प्यस्य॒ वर्त्मन्त्सादयति। य॒ज्ञस्य॒ संत॑त्यै। उ॒वाच॒ हासि॑तो दैव॒लः। ए॒ताव॑ती॒र्वा अ॒मुष्मि॑ल्लोँ॒क आप॑ आसन्। याव॑ती॒ प्रोक्ष॑णी॒रिति॑। तस्माद्ब॒ह्वीरा॒साद्या। स्प्यमु॒दस्य\sn{}। यं द्वि॒ष्यात्तं ध्या॑येत्। शु॒चैवैन॑मर्पयति॥८२॥\anuvakamend[वै वा॒युरा॑ह परा॒वतीत्या॒हाह॑ द्वि॒तीय हर॒तीति॑ परिगृ॒ह्णन्ति॑ देव॒यज॑नीं करोति भवन्ति खनत्यकरे॒तत्कृ॒त्वा र॑क्षो॒घ्नीर॑र्पयति]

%3.2.10.1
वज्रो॒ वै स्प्यः। यद॒न्वञ्चं॑ धा॒रयेत्। वज्रेऽध्व॒र्युः क्ष॑ण्वीत। पु॒रस्तात्ति॒र्यञ्चं॑ धारयति। वज्रो॒ वै स्प्यः। वज्रे॑णै॒व य॒ज्ञस्य॑ दक्षिण॒तो रक्षा॒स्यप॑हन्ति। अ॒ग्निभ्यां॒ प्राच॑श्च प्र॒तीच॑श्च। स्प्येनोदी॑चश्चाध॒राच॑श्च। स्प्येन॒ वा ए॒ष वज्रे॑णा॒स्यै पा॒प्मानं॒ भ्रातृ॑व्यमप॒हत्य॑। उ॒त्क॒रेऽधि॒ प्रवृ॑श्चति॥८३॥

%3.2.10.2
यथो॑प॒धाय॑ वृ॒श्चन्त्ये॒वम्। हस्ता॒वव॑ नेनिक्ते। आ॒त्मान॑मे॒व प॑वयते। स्प्यं प्रक्षा॑लयति मेध्य॒त्वाय॑। अथो॑ पा॒प्मन॑ ए॒व भ्रातृ॑व्यस्य न्य॒ङ्गं छि॑नत्ति। इ॒ध्माब॒र्\mbox{}हिरुप॑सादयति॒ युक्त्यै। य॒ज्ञस्य॑ मिथुन॒त्वाय॑। अथो॑ पुरो॒रुच॑मे॒वैतां द॑धाति। उत्त॑रस्य॒ कर्म॒णोऽनु॑ख्यात्यै। न पु॒रस्तात्प्र॒त्यगुप॑सादयेत्॥८४॥

%3.2.10.3
यत्पु॒रस्तात्प्र॒त्यगु॑पसा॒दयेत्। अ॒न्यत्रा॑हुतिप॒थादि॒ध्मं प्रति॑पादयेत्। प्र॒जा वै ब॒र्॒हिः। अप॑राध्नुयाद्ब॒र्॒हिषा प्र॒जानां प्र॒जन॑नम्। प॒श्चात्प्रागुप॑सादयति। आ॒हु॒ति॒प॒थेने॒ध्मं प्रति॑पादयति। सं॒प्र॒त्ये॑व ब॒र्॒हिषा प्र॒जानां प्र॒जन॑न॒मुपै॑ति। दक्षि॑णमि॒ध्मम्। उत्त॑रं ब॒र्॒हिः। आ॒त्मा वा इ॒ध्मः। प्र॒जा ब॒र्॒हिः। प्र॒जा ह्यात्मन॒ उत्त॑रतरा ती॒र्थे। ततो॒ मेध॑मुप॒नीय॑। य॒था॒दे॒व॒तमे॒वैन॒त्प्रति॑ष्ठापयति। प्रति॑तिष्ठति प्र॒जया॑ प॒शुभि॒र्यज॑मानः॥८५॥\anuvakamend[वृ॒श्च॒ति॒ सा॒द॒ये॒दि॒ध्मः पञ्च॑ च]




\prashnaend{तृ॒तीय॑स्यां दे॒वस्याश्वप॒र्\mbox{}शुं यो वै पूर्वे॒द्युः कर्म॑णे वा॒मिन्द्रो॑ वृ॒त्रम॑ह॒न्त्सो॑ऽपोऽव॑धूत॒न्धृष्टि॑र्दे॒वस्येत्या॑ह॒ सं व॑पामि दे॒वस्य॒ स्प्यमा द॑दे॒ वज्रो॒ वै स्प्यो दश॑॥१०॥}{तृ॒तीय॑स्यां य॒ज्ञस्यान॑तिरेकाय प॒वित्र॑वत्यध्व॒र्युं चा॑धि॒षव॑णमस्य॒न्तरि॑क्ष ए॒व रक्ष॑साम॒न्तर्\mbox{}हि॑त्यै॒ द्वौ वाव पुरु॑षौ॒ यद॒दश्च॒न्द्रम॑सि॒ मेध्यं॒ पञ्चाशी॑तिः॥८५॥}{तृ॒तीय॑स्यां॒ यज॑मानः॥}{हरि॑ ओम्॥}{इति श्रीकृष्णयजुर्वेदीयतैत्तिरीयब्राह्मणे तृतीयाष्टके द्वितीयः प्रपाठकः समाप्तः॥}
\clearpage
\sect{तृतीयः प्रश्नः}
\setcounter{anuvakam}{0}
\dnsub{तैत्तिरीयब्राह्मणे तृतीयाष्टके तृतीयः प्रपाठकः}

%3.3.1.1
प्रत्यु॑ष्ट॒ रक्ष॒ प्रत्यु॑ष्टा॒ अरा॑तय॒ इत्या॑ह। रक्ष॑सा॒मप॑हत्यै। अ॒ग्नेर्व॒स्तेजि॑ष्ठेन॒ तेज॑सा॒ निष्ट॑पा॒मीत्या॑ह मेध्य॒त्वाय॑। स्रुच॒ संमार्ष्टि। स्रु॒वमग्रे। पुमासमे॒वाभ्य॒ सश्य॑ति मिथुन॒त्वाय॑। अथ॑ जु॒हूम्। अथो॑प॒भृतम्। अथ॑ ध्रु॒वाम्। अ॒सौ वै जु॒हूः॥१॥

%3.3.1.2
अ॒न्तरि॑क्षमुप॒भृत्। पृ॒थि॒वी ध्रु॒वा। इ॒मे वै लो॒काः स्रुच॑। वृष्टि॑ स॒म्मार्ज॑नानि। वृष्टि॒र्वा इ॒माल्लोँ॒कान॑नुपू॒र्वं क॑ल्पयति। ते तत॑ कॢ॒प्ताः समे॑धन्ते। समे॑धन्तेऽस्मा इ॒मे लो॒काः प्र॒जया॑ प॒शुभि॑। य ए॒वं वेद॑। यदि॑ का॒मये॑त॒ वर्‌षु॑कः प॒र्जन्य॑ स्या॒दिति॑। अ॒ग्र॒तः संमृ॑ज्यात्॥२॥

%3.3.1.3
वृष्टि॑मे॒व नि य॑च्छति। अ॒वा॒चीनाग्रा॒ हि वृष्टि॑। यदि॑ का॒मये॒ताव॑र्‌षुकः स्या॒दिति॑। मू॒ल॒तः संमृ॑ज्यात्। वृष्टि॑मे॒वोद्य॑च्छति। तदु॒ वा आ॑हुः। अ॒ग्र॒त ए॒वोपरि॑ष्टा॒त्संमृ॑ज्यात्। मू॒ल॒तो॑ऽधस्तात्। तद॑नुपू॒र्वं क॑ल्पते। वर्‌षु॑को भव॒तीति॑॥३॥

%3.3.1.4
प्राची॑मभ्या॒कारम्। अग्रै॑रन्तर॒तः। ए॒वमि॑व॒ ह्यन्न॑म॒द्यते। अथो॒ अग्रा॒द्वा ओष॑धीना॒मूर्जं॑ प्र॒जा उप॑जीवन्ति। ऊ॒र्ज ए॒वान्नाद्य॒स्याव॑रुद्ध्यै। अ॒धस्तात्प्र॒तीचीम्। द॒ण्डमु॑त्तम॒तः। मूले॑न॒ मूलं॒ प्रति॑ष्ठित्यै। तस्मा॑दर॒त्नौ प्राञ्च्यु॒परि॑ष्टा॒ल्लोमा॑नि। प्र॒त्यञ्च्य॒धस्तात्॥४॥

%3.3.1.5
स्रुग्घ्ये॑षा। प्रा॒णो वै स्रु॒वः। जु॒हूर्दक्षि॑णो॒ हस्त॑। उ॒प॒भृत्स॒व्यः। आ॒त्मा ध्रु॒वा। अन्न सं॒मार्ज॑नानि। मु॒ख॒तो वै प्रा॒णो॑ऽपा॒नो भू॒त्वा। आ॒त्मान॒मन्नं॑ प्र॒विश्य॑। बा॒ह्य॒तस्त॒नुव शुभयति। तस्मात्स्रु॒वमे॒वाग्रे॒ संमार्ष्टि। मु॒ख॒तो हि प्रा॒णो॑ऽपा॒नो भू॒त्वा। आ॒त्मान॒मन्न॑मावि॒शति॑। तौ प्रा॑णापा॒नौ। अव्य॑र्धुकः प्राणापा॒नाभ्यां भवति। य ए॒वं वेद॑॥५॥\anuvakamend[जु॒हूर्मृ॑ज्याद्भव॒तीति॑ प्र॒त्यञ्च्य॒धस्तान्मार्ष्टि॒ पञ्च॑ च]

%3.3.2.1
दि॒वः शिल्प॒मव॑ततम्। पृ॒थि॒व्याः क॒कुभि॑ श्रि॒तम्। तेन॑ व॒य स॒हस्र॑वल्‌शेन। स॒पत्नं॑ नाशयामसि॒ स्वाहेति॑ स्रुख्सं॒मार्ज॑नान्य॒ग्नौ प्र ह॑रति। आपो॒ वै द॒र्भाः। रू॒पमे॒वैषा॑मे॒तन्म॑हि॒मानं॒ व्याच॑ष्टे। अ॒नु॒ष्टुभ॒र्चा। आनु॑ष्टुभः प्र॒जाप॑तिः। प्रा॒जा॒प॒त्यो वे॒दः। वे॒दस्याग्र स्रुख्सं॒मार्ज॑नानि॥६॥

%3.3.2.2
स्वेनै॒वैना॑नि॒ छन्द॑सा। स्वया॑ दे॒वत॑या॒ सम॑र्धयति। अथो॒ ऋग्वाव योषा। द॒र्भो वृषा। तन्मि॑थु॒नम्। मि॒थु॒नमे॒वास्य॒ तद्य॒ज्ञे क॑रोति प्र॒जन॑नाय। प्रजा॑यते प्र॒जया॑ प॒शुभि॒र्यज॑मानः। तान्येके॒ वृथै॒वापास्यन्ति। तत्तथा॒ न का॒र्यम्। आर॑ब्धस्य य॒ज्ञिय॑स्य॒ कर्म॑ण॒ सवि॑दो॒हः॥७॥

%3.3.2.3
यद्ये॑नानि प॒शवो॑ऽभि॒ तिष्ठे॑युः। न तत्प॒शुभ्य॒ कम्। अ॒द्भिर्मार्जयि॒त्वोत्क॒रे न्य॑स्येत्। यद्वै य॒ज्ञिय॑स्य॒ कर्म॑णो॒ऽन्यत्राहु॑तीभ्यः सं॒तिष्ठ॑ते। उ॒त्क॒रो वाव तस्य॑ प्रति॒ष्ठा। ए॒ता हि तस्मै प्रति॒ष्ठां दे॒वाः स॒मभ॑रन्। यद॒द्भिर्मा॒र्जय॑ति। तेन॑ शा॒न्तम्। यदु॑त्क॒रे न्य॒स्यति॑। प्र॒ति॒ष्ठामे॒वैना॑नि॒ तद्ग॑मयति॥८॥

%3.3.2.4
प्रति॑तिष्ठति प्र॒जया॑ प॒शुभि॒र्यज॑मानः। अथो स्त॒म्बस्य॒ वा ए॒तद्रू॒पम्। यत्स्रु॑ख्सं॒मार्ज॑नानि। स्त॒म्ब॒शो वा ओष॑धयः। तासां जरत्क॒क्षे प॒शवो॒ न र॑मन्ते। अप्रि॑यो॒ ह्ये॑षां जरत्क॒क्षः। याव॑दप्रियो ह॒ वै ज॑रत्क॒क्षः प॑शू॒नाम्। ताव॑दप्रियः पशू॒नां भ॑वति। यस्यै॒तान्य॒न्यत्रा॒ग्नेर्दध॑ति। न॒व॒दाव्या॑सु॒ वा ओष॑धीषु प॒शवो॑ रमन्ते॥९॥

%3.3.2.5
न॒व॒दा॒वो ह्ये॑षां प्रि॒यः। याव॑त्प्रियो ह॒ वै न॑वदा॒वः प॑शू॒नाम्। ताव॑त्प्रियः पशू॒नां भ॑वति। यस्यै॒तान्य॒ग्नौ प्र॒हर॑न्ति। तस्मा॑दे॒तान्य॒ग्नावे॒व प्रह॑रेत्। य॒त॒रस्मिन्त्संमृ॒ज्यात्। प॒शू॒नां धृत्यै। यो भू॒ताना॒मधि॑पतिः। रु॒द्रस्त॑न्तिच॒रो वृषा। प॒शून॒स्माकं॒ मा हिसीः। ए॒तद॑स्तु हु॒तं तव॒ स्वाहेत्य॑ग्निसं॒मार्ज॑न्य॒ग्नौ प्रह॑रति। ए॒षा वा ए॒तेषां॒ योनि॑। ए॒षा प्र॑ति॒ष्ठा। स्वामे॒वैना॑नि॒ योनिम्। स्वां प्र॑ति॒ष्ठां ग॑मयति। प्रति॑तिष्ठति प्र॒जया॑ प॒शुभि॒र्यज॑मानः॥१०॥\anuvakamend[वे॒दस्याग्र स्रुख्सं॒मार्ज॑नानि विदो॒हो ग॑मयति प॒शवो॑ रमन्ते हिसी॒ष्षट् च॑]

%3.3.3.1
अय॑ज्ञो॒ वा ए॒षः। यो॑ऽप॒त्नीक॑। न प्र॒जाः प्रजा॑येरन्। पत्न्यन्वास्ते। य॒ज्ञमे॒वाक॑। प्र॒जानां प्र॒जन॑नाय। यत्तिष्ठ॑न्ती स॒न्नह्ये॑त। प्रि॒यं ज्ञा॒ति रु॑न्ध्यात्। आसी॑ना॒ सन्न॑ह्यते। आसी॑ना॒ ह्ये॑षा वी॒र्य॑॑ क॒रोति॑॥११॥

%3.3.3.2
यत्प॒श्चात्प्राच्य॒न्वासी॑त। अ॒नया॑ स॒मद॑न्दधीत। दे॒वानां॒ पत्नि॑या स॒मद॑न्दधीत। देशाद्दक्षिण॒त उदी॒च्यन्वास्ते। आ॒त्मनो॑ गोपी॒थाय॑। आ॒शासा॑ना सौमन॒समित्या॑ह। मेध्या॑मे॒वैना॒ङ्केव॑लीं कृ॒त्वा। आ॒शिषा॒ सम॑र्धयति। अ॒ग्नेरनु॑व्रता भू॒त्वा सन्न॑ह्ये सुकृ॒ताय॒ कमित्या॑ह। ए॒तद्वै पत्नि॑यै व्रतोप॒नय॑नम्॥१२॥

%3.3.3.3
तेनै॒वैनां व्र॒तमुप॑नयति। तस्मा॑दाहुः। यश्चै॒वं वेद॒ यश्च॒ न। योक्त्र॑मे॒व यु॑ते। यम॒न्वास्ते। तस्या॒मुष्मिल्लोँ॒के भ॑व॒तीति॒ योक्त्रे॑ण। यद्योक्त्रम्। स योग॑। यदास्ते। स क्षेम॑॥१३॥

%3.3.3.4
यो॒ग॒क्षे॒मस्य॒ कॢप्त्यै। यु॒क्तङ्क्रि॑याता आ॒शीः कामे॑ युज्याता॒ इति॑। आ॒शिष॒ समृ॑द्ध्यै। ग्र॒न्थिङ्ग्र॑थ्नाति। आ॒शिष॑ ए॒वास्यां॒ परि॑ गृह्णाति। पुमा॒न्॒ वै ग्र॒न्थिः। स्त्री पत्नी। तन्मि॑थु॒नम्। मि॒थु॒नमे॒वास्य॒ तद्य॒ज्ञे क॑रोति प्र॒जन॑नाय। प्र जा॑यते प्र॒जया॑ प॒शुभि॒र्यज॑मानः॥१४॥

%3.3.3.5
अथो॑ अ॒र्धो वा ए॒ष आ॒त्मन॑। यत्पत्नी। य॒ज्ञस्य॒ धृत्या॒ अशि॑थिलम्भावाय। सु॒प्र॒जस॑स्त्वा व॒य सु॒पत्नी॒रुप॑ सेदि॒मेत्या॑ह। य॒ज्ञमे॒व तन्मि॑थु॒नीक॑रोति। ऊ॒नेऽति॑रिक्तन्धीयाता॒ इति॒ प्रजात्यै। म॒ही॒नां पयो॒ऽस्योष॑धीना॒ रस॒ इत्या॑ह। रू॒पमे॒वास्यै॒तन्म॑हि॒मानं॒ व्याच॑ष्टे। तस्य॒ तेऽक्षी॑यमाणस्य॒ निर्व॑पामि देवय॒ज्याया॒ इत्या॑ह। आ॒शिष॑मे॒वैतामा शास्ते॥१५॥\anuvakamend[क॒रोति॑ व्रतोप॒नय॑नं॒ क्षेमो॒ यज॑मानः शास्ते]

%3.3.4.1
घृ॒तं च॒ वै मधु॑ च प्र॒जाप॑तिरासीत्। यतो॒ मध्वा॑सीत्। तत॑ प्र॒जा अ॑सृजत। तस्मा॒न्मधु॑षि प्र॒जन॑नमिवास्ति। तस्मा॒न्मधु॑षा॒ न प्रच॑रन्ति। या॒तया॑म॒ हि। आज्ये॑न॒ प्रच॑रन्ति। य॒ज्ञो वा आज्यम्। य॒ज्ञेनै॒व य॒ज्ञं प्रच॑र॒न्त्यया॑तयामत्वाय। पत्न्यवेक्षते॥१६॥

%3.3.4.2
मि॒थु॒न॒त्वाय॒ प्रजात्यै। यद्वै पत्नी॑ य॒ज्ञस्य॑ क॒रोति॑। मि॒थु॒नं तत्। अथो॒ पत्नि॑या ए॒वैष य॒ज्ञस्यान्वार॒म्भोऽन॑वच्छित्त्यै। अ॒मे॒ध्यं वा ए॒तत्क॑रोति। यत्पत्न्य॒वेक्ष॑ते। गार्‌ह॑प॒त्येऽधि॑ श्रयति मेध्य॒त्वाय॑। आ॒ह॒व॒नीय॑म॒भ्युद्द्र॑वति। य॒ज्ञस्य॒ सन्त॑त्यै। तेजो॑ऽसि॒ तेजोऽनु॒ प्रेहीत्या॑ह॥१७॥

%3.3.4.3
तेजो॒ वा अ॒ग्निः। तेज॒ आज्यम्। तेज॑सै॒व तेज॒ सम॑र्धयति। अ॒ग्निस्ते॒ तेजो॒ मा विनै॒दित्या॒हाहिसायै। स्प्यस्य॒ वर्त्मन्त्सादयति। य॒ज्ञस्य॒ सन्त॑त्यै। अ॒ग्नेर्जि॒ह्वाऽसि॑ सु॒भूर्दे॒वाना॒मित्या॑ह। य॒था॒य॒जुरे॒वैतत्। धाम्ने॑धाम्ने दे॒वेभ्यो॒ यजु॑षेयजुषे भ॒वेत्या॑ह। आ॒शिष॑मे॒वैतामा शास्ते॥१८॥

%3.3.4.4
तद्वा अत॑ प॒वित्राभ्यामे॒वोत्पु॑नाति। यज॑मानो॒ वा आज्यम्। प्रा॒णा॒पा॒नौ प॒वित्रे। यज॑मान ए॒व प्रा॑णापा॒नौ द॑धाति। पु॒न॒रा॒हारम्। ए॒वमि॑व॒ हि प्रा॑णापा॒नौ सं॒चर॑तः। शु॒क्रम॑सि॒ ज्योति॑रसि॒ तेजो॒ऽसीत्या॑ह। रू॒पमे॒वास्यै॒तन्म॑हि॒मानं॒ व्याच॑ष्टे। त्रिर्यजु॑षा। त्रय॑ इ॒मे लो॒काः॥१९॥

%3.3.4.5
ए॒षां लो॒काना॒माप्त्यै। त्रिः। त्र्या॑वृ॒द्धि य॒ज्ञः। अथो॑ मेध्य॒त्वाय॑। अथाज्य॑वतीभ्याम॒पः। रू॒पमे॒वासा॑मे॒तद्वर्णं॑ दधाति। अपि॒ वा उ॒ताहु॑। यथा॑ ह॒ वै योषा॑ सु॒वर्ण॒ हिर॑ण्यं पेश॒लं बिभ्र॑ती रू॒पाण्यास्ते। ए॒वमे॒ता ए॒तर्\mbox{}हीति॑। आपो॒ वै सर्वा दे॒वता॥२०॥

%3.3.4.6
ए॒षा हि विश्वे॑षां दे॒वानां त॒नूः। यदाज्यम्। तत्रो॒भयोर्मीमा॒सा। जा॒मि स्यात्। यद्यजु॒षाऽऽज्यं॒ यजु॑षा॒ऽप उ॑त्पुनी॒यात्। छन्द॑सा॒ऽप उत्पु॑ना॒त्यजा॑मित्वाय। अथो॑ मिथुन॒त्वाय॑। सा॒वि॒त्रि॒यर्चा। स॒वि॒तृप्र॑सूतं मे॒ कर्मा॑स॒दिति॑। स॒वि॒तृप्र॑सूतमे॒वास्य॒ कर्म॑ भवति। प॒च्छो गा॑यत्रि॒या त्रि॑ष्षमृद्ध॒त्वाय॑। अ॒द्भिरे॒वौष॑धी॒ सं न॑यति। ओष॑धीभिः प॒शून्। प॒शुभि॒र्यज॑मानम्। शु॒क्रं त्वा॑ शु॒क्रायां॒ ज्योति॑स्त्वा॒ ज्योति॑ष्य॒र्चिस्त्वा॒ऽर्चिषीत्या॑ह सर्व॒त्वाय॑। पर्याप्त्या॒ अन॑न्तरायाय॥२१॥\anuvakamend[ई॒क्ष॒त॒ आ॒ह॒ शा॒स्ते॒ लो॒का दे॒वता॑ भवति॒ षट् च॑]

%3.3.5.1
दे॒वा॒सु॒राः संय॑त्ता आसन्। स ए॒तमिन्द्र॒ आज्य॑स्यावका॒शम॑पश्यत्। तेनावैक्षत। ततो॑ दे॒वा अभ॑वन्। पराऽसु॑राः। य ए॒वं वि॒द्वानाज्य॑म॒वेक्ष॑ते। भव॑त्या॒त्मना। पराऽस्य॒ भ्रातृ॑व्यो भवति। ब्र॒ह्म॒वा॒दिनो॑ वदन्ति। यदाज्ये॑ना॒न्यानि॑ ह॒वीष्य॑भिघा॒रय॑ति॥२२॥

%3.3.5.2
अथ॒ केनाज्य॒मिति॑। स॒त्येनेति॑ ब्रूयात्। चक्षु॒र्वै स॒त्यम्। स॒त्येनै॒वैन॑द॒भि घा॑रयति। ई॒श्व॒रो वा ए॒षोऽन्धो भवि॑तोः। यश्चक्षु॒षाऽऽज्य॑म॒वेक्ष॑ते। नि॒मील्यावेक्षेत। दा॒धारा॒त्मञ्चक्षु॑। अ॒भ्याज्य॑ङ्घारयति। आज्य॑ङ्गृह्णाति॥२३॥

%3.3.5.3
छन्दासि॒ वा आज्यम्। छन्दास्ये॒व प्री॑णाति। च॒तुर्जु॒ह्वां गृ॑ह्णाति। चतु॑ष्पादः प॒शव॑। प॒शूने॒वाव॑ रुन्धे। अ॒ष्टावु॑प॒भृति॑। अ॒ष्टाक्ष॑रा गाय॒त्री। गा॒य॒त्रः प्रा॒णः। प्रा॒णमे॒व प॒शुषु॑ दधाति। च॒तुर्ध्रु॒वायाम्॥२४॥

%3.3.5.4
चतु॑ष्पादः प॒शव॑। प॒शुष्वे॒वोपरि॑ष्टा॒त्प्रति॑ तिष्ठति। य॒ज॒मा॒न॒दे॒व॒त्या॑ वै जु॒हूः। भ्रा॒तृ॒व्य॒दे॒व॒त्यो॑प॒भृत्। च॒तुर्जु॒ह्वाङ्गृ॒ह्णन्भूयो॑ गृह्णीयात्। अ॒ष्टावु॑प॒भृति॑ गृ॒ह्णन्कनी॑यः। यज॑मानायै॒व भ्रातृ॑व्य॒मुप॑स्तिं करोति। गौर्वै स्रुच॑। च॒तुर्जु॒ह्वां गृ॑ह्णाति। तस्मा॒च्चतु॑ष्पदी॥२५॥

%3.3.5.5
अ॒ष्टावु॑प॒भृति॑। तस्मा॑द॒ष्टाश॑फा। च॒तुर्ध्रु॒वायाम्। तस्मा॒च्चतु॑ स्तना। गामे॒व तत्सस्क॑रोति। सास्मै॒ सस्कृ॒तेष॒मूर्ज॑न्दुहे। यज्जु॒ह्वाङ्गृ॒ह्णाति॑। प्र॒या॒जेभ्य॒स्तत्। यदु॑प॒भृति॑। प्र॒या॒जा॒नू॒या॒जेभ्य॒स्तत्। सर्व॑स्मै॒ वा ए॒तद्य॒ज्ञाय॑ गृह्यते। य॒द्ध्रु॒वाया॒माज्यम्॥२६॥\anuvakamend[अ॒भि॒घा॒रय॑ति गृह्णाति ध्रु॒वाया॒ञ्चतु॑ष्पदी प्रयाजानूया॒जेभ्य॒स्तद्द्वे च॑]

%3.3.6.1
आपो॑ देवीरग्रेपुवो अग्रेगुव॒ इत्या॑ह। रू॒पमे॒वासा॑मे॒तन्म॑हि॒मान॒व्व्याँच॑ष्टे। अग्र॑ इ॒मं य॒ज्ञन्न॑य॒ताग्रे॑ य॒ज्ञप॑ति॒मित्या॑ह। अग्र॑ ए॒व य॒ज्ञन्न॑यन्ति। अग्रे॑ य॒ज्ञप॑तिम्। यु॒ष्मानिन्द्रो॑ऽवृणीत वृत्र॒तूर्ये॑ यू॒यमिन्द्र॑मवृणीध्वं वृत्र॒तूर्य॒ इत्या॑ह। वृ॒त्र ह॑ हनि॒ष्यन्निन्द्र॒ आपो॑ वव्रे। आपो॒ हेन्द्रं॑ वव्रिरे। सं॒ज्ञामे॒वासा॑मे॒तत्सामा॑न॒व्व्याँच॑ष्टे। प्रोक्षि॑ता॒ स्थेत्या॑ह॥२७॥

%3.3.6.2
तेनाप॒ प्रोक्षि॑ताः। अ॒ग्निर्दे॒वेभ्यो॒ निला॑यत। कृष्णो॑ रू॒पं कृ॒त्वा। स वन॒स्पती॒न्प्रावि॑शत्। कृष्णोऽस्याखरे॒ष्ठोऽग्नये त्वा॒ स्वाहेत्या॑ह। अ॒ग्नय॑ ए॒वैनं॒ जुष्टं॑ करोति। अथो॑ अ॒ग्नेरे॒व मेध॒मव॑ रुन्धे। वेदि॑रसि ब॒र्॒हिषे त्वा॒ स्वाहेत्या॑ह। प्र॒जा वै ब॒र्॒हिः। पृ॒थि॒वी वेदि॑॥२८॥

%3.3.6.3
प्र॒जा ए॒व पृ॑थि॒व्यां प्रति॑ष्ठापयति। ब॒र्॒हिर॑सि स्रु॒ग्भ्यस्त्वा॒ स्वाहेत्या॑ह। प्र॒जा वै ब॒र्॒हिः। यज॑मान॒ स्रुच॑। यज॑मानमे॒व प्र॒जासु॒ प्रति॑ष्ठापयति। दि॒वे त्वा॒ऽन्तरि॑क्षाय त्वा पृथि॒व्यै त्वेति॑ ब॒र्॒हिरा॒साद्य॒ प्रोक्ष॑ति। ए॒भ्य ए॒वैन॑ल्लो॒केभ्य॒ प्रोक्ष॑ति। अथ॒ तत॑ स॒ह स्रु॒चा पु॒रस्तात्प्र॒त्यञ्च॑ङ्ग्र॒न्थिं प्रत्यु॑क्षति। प्र॒जा वै ब॒र्॒हिः। यथा॒ सूत्यै॑ का॒ल आप॑ पु॒रस्ता॒द्यन्ति॑॥२९॥

%3.3.6.4
ता॒दृगे॒व तत्। स्व॒धा पि॒तृभ्य॒ इत्या॑ह। स्व॒धा॒का॒रो हि पि॑तृ॒णाम्। ऊर्ग्भ॑व बर्‌हि॒षद्भ्य॒ इति॒ दक्षि॑णायै॒ श्रोणे॒रोत्त॑रस्यै॒ निन॑यति॒ सन्त॑त्यै। मासा॒ वै पि॒तरो॑ बर्‌हि॒षद॑। मासा॑ने॒व प्री॑णाति। मासा॒ वा ओष॑धीर्व॒र्धय॑न्ति। मासा पचन्ति॒ समृ॑द्ध्यै। अन॑तिस्कन्दन् ह प॒र्जन्यो॑ वर्‌षति। यत्रै॒तदे॒वङ्क्रि॒यते॥३०॥

%3.3.6.5
ऊ॒र्जा पृ॑थि॒वीङ्ग॑च्छ॒तेत्या॑ह। पृ॒थि॒व्यामे॒वोर्ज॑न्दधाति। तस्मात्पृथि॒व्या ऊ॒र्जा भु॑ञ्जते। ग्र॒न्थिं वि स्रसयति। प्रज॑नयत्ये॒व तत्। ऊ॒र्ध्वं प्राञ्च॒मुद्गू॑ढं प्र॒त्यञ्च॒मा य॑च्छति। तस्मात्प्रा॒चीन॒ रेतो॑ धीयते। प्र॒तीची प्र॒जा जा॑यन्ते। विष्णो॒ स्तूपो॒ऽसीत्या॑ह। य॒ज्ञो वै विष्णु॑॥३१॥

%3.3.6.6
य॒ज्ञस्य॒ धृत्यै। पु॒रस्तात्प्रस्त॒रं गृ॑ह्णाति। मुख्य॑मे॒वैनं॑ करोति। इय॑न्तङ्गृह्णाति। प्र॒जाप॑तिना यज्ञमु॒खेन॒ सम्मि॑तम्। इय॑न्तङ्गृह्णाति। य॒ज्ञ॒प॒रुषा॒ सम्मि॑तम्। इय॑न्तङ्गृह्णाति। ए॒ताव॒द्वै पुरु॑षे वी॒र्यम्। वी॒र्य॑संमितम्॥३२॥

%3.3.6.7
अप॑रिमितङ्गृह्णाति। अप॑रिमित॒स्याव॑रुद्ध्यै। तस्मि॑न्प॒वित्रे॒ अपि॑ सृजति। यज॑मानो॒ वै प्र॑स्त॒रः। प्रा॒णा॒पा॒नौ प॒वित्रे। यज॑मान ए॒व प्रा॑णापा॒नौ द॑धाति। ऊर्णाम्रदसन्त्वा स्तृणा॒मीत्या॑ह। य॒था॒य॒जुरे॒वैतत्। स्वा॒स॒स्थन्दे॒वेभ्य॒ इत्या॑ह। दे॒वेभ्य॑ ए॒वैन॑त्स्वास॒स्थं क॑रोति॥३३॥

%3.3.6.8
ब॒र्॒हिः स्तृ॑णाति। प्र॒जा वै ब॒र्॒हिः। पृ॒थि॒वी वेदि॑। प्र॒जा ए॒व पृ॑थि॒व्यां प्रति॑ष्ठापयति। अन॑तिदृश्ञ स्तृणाति। प्र॒जयै॒वैनं॑ प॒शुभि॒रन॑तिदृश्ञं करोति। धा॒रय॑न्प्रस्त॒रं प॑रि॒धीन्परि॑ दधाति। यज॑मानो॒ वै प्र॑स्त॒रः। यज॑मान ए॒व तत्स्व॒यं प॑रि॒धीन्परि॑ दधाति। ग॒न्ध॒र्वो॑ऽसि वि॒श्वाव॑सु॒रित्या॑ह॥३४॥

%3.3.6.9
विश्व॑मे॒वायु॒र्यज॑माने दधाति। इन्द्र॑स्य बा॒हुर॑सि॒ दक्षि॑ण॒ इत्या॑ह। इ॒न्द्रि॒यमे॒व यज॑माने दधाति। मि॒त्रावरु॑णौ त्वोत्तर॒तः परि॑धत्ता॒मित्या॑ह। प्रा॒णा॒पा॒नौ मि॒त्रावरु॑णौ। प्रा॒णा॒पा॒नावे॒वास्मि॑न्दधाति। सूर्य॑स्त्वा पु॒रस्तात् पा॒त्वित्या॑ह। रक्ष॑सा॒मप॑हत्यै। कस्याश्चिद॒भिश॑स्त्या॒ इत्या॑ह। अप॑रिमितादे॒वैनं॑ पाति॥३५॥

%3.3.6.10
वी॒तिहोत्रन्त्वा कव॒ इत्या॑ह। अ॒ग्निमे॒व हो॒त्रेण॒ सम॑र्धयति। द्यु॒मन्त॒ समि॑धीम॒हीत्या॑ह॒ समि॑द्ध्यै। अग्ने॑ बृ॒हन्त॑मध्व॒र इत्या॑ह॒ वृद्ध्यै। वि॒शो य॒न्त्रे स्थ॒ इत्या॑ह। वि॒शां यत्यै। उ॒दी॒चीनाग्रे॒ नि द॑धाति॒ प्रति॑ष्ठित्यै। वसू॑ना रु॒द्राणा॑मादि॒त्याना॒ सद॑सि सी॒देत्या॑ह। दे॒वता॑नामे॒व सद॑ने प्रस्त॒र सा॑दयति। जु॒हूर॑सि घृ॒ताची॒ नाम्नेत्या॑ह॥३६॥

%3.3.6.11
अ॒सौ वै जु॒हूः। अ॒न्तरि॑क्षमुप॒भृत्। पृ॒थि॒वी ध्रु॒वा। तासा॑मे॒तदे॒व प्रि॒यन्नाम॑। यद्घृ॒ताचीति॑। यद्घृ॒ताचीत्याह॑। प्रि॒येणै॒वैना॒ नाम्ना॑ सादयति। ए॒ता अ॑सदन्त्सुकृ॒तस्य॑ लो॒क इत्या॑ह। स॒त्यं वै सु॑कृ॒तस्य॑ लो॒कः। स॒त्य ए॒वैना सुकृ॒तस्य॑ लो॒के सा॑दयति। ता वि॑ष्णो पा॒हीत्या॑ह। य॒ज्ञो वै विष्णु॑। य॒ज्ञस्य॒ धृत्यै। पा॒हि य॒ज्ञं पा॒हि य॒ज्ञप॑तिं पा॒हि मां य॑ज्ञ॒निय॒मित्या॑ह। य॒ज्ञाय॒ यज॑मानाया॒त्मने। तेभ्य॑ ए॒वाशिष॒माशा॒स्तेऽनार्त्यै॥३७॥\anuvakamend[स्थेत्या॑ह पृथि॒वी वेदि॒र्यन्ति॑ क्रि॒यते॒ वीणु॑र्वी॒र्य॑सम्मितं करोत्याह पाति॒ नाम्नेत्या॑ह लो॒के सा॑दयति॒ षट् च॑]

%3.3.7.1
अ॒ग्निना॒ वै होत्रा। दे॒वा असु॑रान॒भ्य॑भवन्। अ॒ग्नये॑ समि॒ध्यमा॑ना॒यानु॑ब्रू॒हीत्या॑ह॒ भ्रातृ॑व्याभिभूत्यै। एक॑विशतिमिध्मदा॒रूणि॑ भवन्ति। ए॒क॒वि॒शो वै पुरु॑षः। पुरु॑ष॒स्याप्त्यै। पञ्च॑दशेध्मदा॒रूण्य॒भ्या द॑धाति। पञ्च॑दश॒ वा अ॑र्धमा॒सस्य॒ रात्र॑यः। अ॒र्ध॒मा॒स॒शः सं॑वत्स॒र आप्यते। त्रीन्प॑रि॒धीन्परि॑ दधाति॥३८॥

%3.3.7.2
ऊ॒र्ध्वे स॒मिधा॒वा द॑धाति। अ॒नू॒या॒जेभ्य॑ स॒मिध॒मति॑ शिनष्टि। षट्त्संप॑द्यन्ते। षड्वा ऋ॒तव॑। ऋ॒तूने॒व प्री॑णाति। वे॒देनोप॑ वाजयति। प्रा॒जा॒प॒त्यो वै वे॒दः। प्रा॒जा॒प॒त्यः प्रा॒णः। यज॑मान आहव॒नीय॑। यज॑मान ए॒व प्रा॒णन्द॑धाति॥३९॥

%3.3.7.3
त्रिरुप॑ वाजयति। त्रयो॒ वै प्रा॒णाः। प्रा॒णाने॒वास्मि॑न्दधाति। वे॒देनो॑प॒यत्य॑ स्रु॒वेण॑ प्राजाप॒त्यमा॑घा॒रमा घा॑रयति। य॒ज्ञो वै प्र॒जाप॑तिः। य॒ज्ञमे॒व प्र॒जाप॑तिं मुख॒त आर॑भते। अथो प्र॒जाप॑ति॒ सर्वा॑ दे॒वता। सर्वा॑ ए॒व दे॒वता प्रीणाति। अ॒ग्निम॑ग्नी॒त्रिस्त्रि॒ सं मृ॒ड्ढीत्या॑ह। त्र्या॑वृ॒द्धि य॒ज्ञः॥४०॥

%3.3.7.4
अथो॒ रक्ष॑सा॒मप॑हत्यै। प॒रि॒धीन्त्सं मार्ष्टि। पु॒नात्ये॒वैनान्॑। त्रिस्त्रि॒ सं मार्ष्टि। त्र्या॑वृ॒द्धि य॒ज्ञः। अथो॑ मेध्य॒त्वाय॑। अथो॑ ए॒ते वै दे॑वा॒श्वाः। दे॒वा॒श्वाने॒व तत्सं मार्ष्टि। सु॒व॒र्गस्य॑ लो॒कस्य॒ सम॑ष्ट्यै। आसी॑नो॒ऽन्यमा॑घा॒रमा घा॑रयति॥४१॥

%3.3.7.5
तिष्ठ॑न्न॒न्यम्। यथाऽनो॑ वा॒ रथं॑ वा यु॒ञ्ज्यात्। ए॒वमे॒व तद॑ध्व॒र्युर्य॒ज्ञं यु॑नक्ति। सु॒व॒र्गस्य॑ लो॒कस्या॒भ्यूढ्यै। वह॑न्त्येनङ्ग्रा॒म्याः प॒शव॑। य ए॒वं वेद॑। भुव॑नमसि॒ वि प्र॑थ॒स्वेत्या॑ह। य॒ज्ञो वै भुव॑नम्। य॒ज्ञ ए॒व यज॑मानं प्र॒जया॑ प॒शुभि॑ प्रथयति। अग्ने॒ यष्ट॑रि॒दन्नम॒ इत्या॑ह॥४२॥

%3.3.7.6
अ॒ग्निर्वै दे॒वानां॒ यष्टा। य ए॒व दे॒वानां॒ यष्टा। तस्मा॑ ए॒व नम॑स्करोति। जुह्वेह्य॒ग्निस्त्वा ह्वयति देवय॒ज्याया॒ उप॑भृ॒देहि॑ दे॒वस्त्वा॑ सवि॒ता ह्व॑यति देवय॒ज्याया॒ इत्या॑ह। आ॒ग्ने॒यी वै जु॒हूः। सा॒वि॒त्र्यु॑प॒भृत्। ताभ्या॑मे॒वैने॒ प्रसू॑त॒ आद॑त्ते। अग्ना॑विष्णू॒ मा वा॒मव॑ क्रमिष॒मित्या॑ह। अ॒ग्निः पु॒रस्तात्। विष्णु॑र्य॒ज्ञः प॒श्चात्॥४३॥

%3.3.7.7
ताभ्या॑मे॒व प्र॑ति॒प्रोच्या॒त्या क्रा॑मति। विजि॑हाथां॒ मा मा॒ सन्ताप्त॒मित्या॒हाहिसायै। लो॒कं मे॑ लोककृतौ कृणुत॒मित्या॑ह। आ॒शिष॑मे॒वैतामा शास्ते। विष्णो॒ स्थान॑म॒सीत्या॑ह। य॒ज्ञो वै विष्णु॑। ए॒तत्खलु॒ वै दे॒वाना॒मप॑राजितमा॒यत॑नम्। यद्य॒ज्ञः। दे॒वाना॑मे॒वाप॑राजित आ॒यत॑ने तिष्ठति। इ॒त इन्द्रो॑ अकृणोद्वी॒र्या॑णीत्या॑ह॥४४॥

%3.3.7.8
इ॒न्द्रि॒यमे॒व यज॑माने दधाति। स॒मा॒रभ्यो॒र्ध्वो अ॑ध्व॒रो दि॑वि॒स्पृश॒मित्या॑ह॒ वृद्ध्यै। आ॒घा॒रमा॑घा॒र्यमा॑ण॒मनु॑ समा॒रभ्य॑। ए॒तस्मि॑न्का॒ले दे॒वाः सु॑व॒र्गं लो॒कमा॑यन्। सा॒क्षादे॒व यज॑मानः सुव॒र्गं लो॒कमे॑ति। अथो॒ समृ॑द्धेनै॒व य॒ज्ञेन॒ यज॑मानः सुव॒र्गं लो॒कमे॑ति। अह्रु॑तो य॒ज्ञो य॒ज्ञप॑ते॒रित्या॒हानार्त्यै। इन्द्रा॑वा॒न्त्स्वाहेत्या॑ह। इ॒न्द्रि॒यमे॒व यज॑माने दधाति। बृ॒हद्भा इत्या॑ह॥४५॥

%3.3.7.9
सु॒व॒र्गो वै लो॒को बृ॒हद्भाः। सु॒व॒र्गस्य॑ लो॒कस्य॒ सम॑ष्ट्यै। य॒ज॒मा॒न॒दे॒व॒त्या॑ वै जु॒हूः। भ्रा॒तृ॒व्य॒दे॒व॒त्यो॑प॒भृत्। प्रा॒ण आ॑घा॒रः। यत्सस्प॒र्॒शयेत्। भ्रातृ॑व्येऽस्य प्रा॒णन्द॑ध्यात्। असस्पर्‌शयन्न॒त्या क्रा॑मति। यज॑मान ए॒व प्रा॒णन्द॑धाति। पा॒हि माऽग्ने॒ दुश्च॑रिता॒दा मा॒ सुच॑रिते भ॒जेत्या॑ह॥४६॥

%3.3.7.10
अ॒ग्निर्वाव प॒वित्रम्। वृ॒जि॒नमनृ॑त॒न्दुश्च॑रितम्। ऋ॒जु॒क॒र्म स॒त्य सुच॑रितम्। अ॒ग्निरे॒वैनं॑ वृजि॒नादनृ॑ता॒द्दुश्च॑रितात्पाति। ऋ॒जु॒क॒र्मे स॒त्ये सुच॑रिते भजति। तस्मा॑दे॒वमा शास्ते। आ॒त्मनो॑ गोपी॒थाय॑। शिरो॒ वा ए॒तद्य॒ज्ञस्य॑। यदा॑घा॒रः। आ॒त्मा ध्रु॒वा॥४७॥

%3.3.7.11
आ॒घा॒रमा॒घार्य॑ ध्रु॒वा सम॑नक्ति। आ॒त्मन्ने॒व य॒ज्ञस्य॒ शिर॒ प्रति॑ दधाति। द्विः सम॑नक्ति। द्वौ हि प्रा॑णापा॒नौ। तदा॑हुः। त्रिरे॒व सम॑ञ्ज्यात्। त्रिधा॑तु॒ हि शिर॒ इति॑। शिर॑ इवै॒तद्य॒ज्ञस्य॑। अथो॒ त्रयो॒ वै प्रा॒णाः। प्रा॒णाने॒वास्मि॑न्दधाति। म॒खस्य॒ शिरो॑ऽसि॒ सञ्ज्योति॑षा॒ ज्योति॑रङ्क्ता॒मित्या॑ह। ज्योति॑रे॒वास्मा॑ उ॒परि॑ष्टाद्दधाति। सु॒व॒र्गस्य॑ लो॒कस्यानु॑ख्यात्यै॥४८॥\anuvakamend[परि॑दधाति प्रा॒णन्द॑धाति॒ हि य॒ज्ञो घा॑रयति॒ नम॒ इत्या॑ह प॒श्चाद्वी॒र्या॑णीत्या॑ह॒ भा इत्या॑ह भ॒जेत्या॑ह ध्रु॒वैवास्मि॑न्दधाति॒ त्रीणि॑ च]

%3.3.8.1
धिष्णि॑या॒ वा ए॒ते न्यु॑प्यन्ते। यद्ब्र॒ह्मा। यद्धोता। यद॑ध्व॒र्युः। यद॒ग्नीत्। यद्यज॑मानः। तान् यद॑न्तरे॒यात्। यज॑मानस्य प्रा॒णान्त्सङ्क॑र्‌षेत्। प्र॒मायु॑कः स्यात्। पु॒रो॒डाश॑मप॒गृह्य॒ सञ्च॑रत्यध्व॒र्युः॥४९॥

%3.3.8.2
यज॑मानायै॒व तल्लो॒क शिषति। नास्य॑ प्रा॒णान्त्सङ्क॑र्‌षति। न प्र॒मायु॑को भवति। पु॒रस्तात् प्र॒त्यङ्ङासी॑नः। इडा॑या॒ इडा॒मा द॑धाति। हस्त्या॒ होत्रे। प॒शवो॒ वा इडा। प॒शव॒ पुरु॑षः। प॒शुष्वे॒व प॒शून्प्रति॑ष्ठापयति। इडा॑यै॒ वा ए॒षा प्रजा॑तिः॥५०॥

%3.3.8.3
तां प्रजा॑तिं॒ यज॑मा॒नोऽनु॒ प्र जा॑यते। द्विर॒ङ्गुला॑वनक्ति॒ पर्व॑णोः। द्वि॒पाद्यज॑मान॒ प्रति॑ष्ठित्यै। स॒कृदुप॑ स्तृणाति। द्विरा द॑धाति। स॒कृद॒भि घा॑रयति। च॒तुः संप॑द्यते। च॒त्वारि॒ वै प॒शोः प्र॑ति॒ष्ठाना॑नि। यावा॑ने॒व प॒शुः। तमुप॑ह्वयते॥५१॥

%3.3.8.4
मुख॑मिव॒ प्रत्युप॑ह्वयेत। सं॒मु॒खाने॒व प॒शूनुप॑ ह्वयते। प॒शवो॒ वा इडा। तस्मा॒त्साऽन्वा॒रभ्या। अ॒ध्व॒र्युणा॑ च॒ यज॑मानेन च। उप॑हूतः पशु॒मान॑सा॒नीत्या॑ह। उप॒ ह्ये॑नौ॒ ह्वय॑ते॒ होता। इडा॑यै दे॒वता॑नामुपह॒वे। उप॑हूतः पशु॒मान्भ॑वति। य ए॒वं वेद॑॥५२॥

%3.3.8.5
यां वै हस्त्या॒मिडा॑मा॒दधा॑ति। वा॒चः सा भा॑ग॒धेयम्। यामु॑प॒ह्वय॑ते। प्रा॒णाना॒ सा। वाचं॑ चै॒व प्रा॒णाश्चाव॑ रुन्धे। अथ॒ वा ए॒तर्ह्युप॑हूताया॒मिडा॑याम्। पु॒रो॒डाश॑स्यै॒व ब॑र्‌हि॒षदो॑ मीमा॒सा। यज॑मानन्दे॒वा अ॑ब्रुवन्। ह॒विर्नो॒ निर्व॒पेति॑। नाहम॑भा॒गो निर्व॑प्स्या॒मीत्य॑ब्रवीत्॥५३॥

%3.3.8.6
न मया॑ऽभा॒गयाऽनु॑वक्ष्य॒थेति॒ वाग॑ब्रवीत्। नाहम॑भा॒गा पु॑रोनुवा॒क्या॑ भविष्या॒मीति॑ पुरोनुवा॒क्या। नाहम॑भा॒गा या॒ज्या॑ भविष्या॒मीति॑ या॒ज्या। न मया॑ऽभा॒गेन॒ वष॑ट्करिष्य॒थेति॑ वषट्का॒रः। यद्य॑जमानभा॒गन्नि॒धाय॑ पुरो॒डाशं॑ बर्‌हि॒षदं॑ क॒रोति॑। ताने॒व तद्भा॒गिन॑ करोति। च॒तु॒र्धा क॑रोति। चत॑स्रो॒ दिश॑। दि॒क्ष्वे॑व प्रति॑तिष्ठति। ब॒र्॒हि॒षदं॑ करोति॥५४॥

%3.3.8.7
यज॑मानो॒ वै पु॑रो॒डाश॑। प्र॒जा ब॒र्॒हिः। यज॑मानमे॒व प्र॒जासु॒ प्रति॑ष्ठापयति। तस्मा॑द॒स्थ्नाऽन्याः प्र॒जाः प्र॑ति॒तिष्ठ॑न्ति। मा॒सेना॒न्याः। अथो॒ खल्वा॑हुः। दक्षि॑णा॒ वा ए॒ता ह॑विर्य॒ज्ञस्यान्तर्वे॒द्यव॑ रुध्यन्ते। यत्पु॑रो॒डाशं॑ बर्‌हि॒षदं॑ क॒रोतीति॑। च॒तु॒र्धा क॑रोति। च॒त्वारो॒ ह्ये॑ते ह॑विर्य॒ज्ञस्य॒र्त्विज॑॥५५॥

%3.3.8.8
ब्र॒ह्मा होताऽध्व॒र्युर॒ग्नीत्। तम॒भि मृ॑शेत्। इ॒दं ब्र॒ह्मण॑। इ॒द होतु॑। इ॒दम॑ध्व॒र्योः। इ॒दम॒ग्नीध॒ इति॑। यथै॒वादः सौ॒म्येऽध्व॒रे। आ॒देश॑मृ॒त्विग्भ्यो॒ दक्षि॑णा नी॒यन्ते। ता॒दृगे॒व तत्। अ॒ग्नीधे प्रथ॒माया द॑धाति॥५६॥

%3.3.8.9
अ॒ग्निमु॑खा॒ ह्यृद्धि॑। अ॒ग्निमु॑खामे॒वर्द्धिं॒ यज॑मान ऋध्नोति। स॒कृदु॑प॒स्तीर्य॒ द्विरा॒दध॑त्। उ॒प॒स्तीर्य॒ द्विर॒भि घा॑रयति। षट्त्संप॑द्यन्ते। षड्वा ऋ॒तव॑। ऋ॒तूने॒व प्री॑णाति। वे॒देन॑ ब्र॒ह्मणे ब्रह्मभा॒गं परि॑हरति। प्रा॒जा॒प॒त्यो वै वे॒दः। प्रा॒जा॒प॒त्यो ब्र॒ह्मा॥५७॥

%3.3.8.10
स॒वि॒ता य॒ज्ञस्य॒ प्रसूत्यै। अथ॒ काम॑म॒न्येन॑। ततो॒ होत्रे। मध्यं॒ वा ए॒तद्य॒ज्ञस्य॑। यद्धोता। म॒ध्य॒त ए॒व य॒ज्ञं प्री॑णाति। अथाध्व॒र्यवे। प्र॒ति॒ष्ठा वा ए॒षा य॒ज्ञस्य॑। यद॑ध्व॒र्युः। तस्माद्धविर्य॒ज्ञस्यै॒तामे॒वावृत॒मनु॑॥५८॥

%3.3.8.11
अ॒न्या दक्षि॑णा नीयन्ते। य॒ज्ञस्य॒ प्रति॑ष्ठित्यै। अ॒ग्निम॑ग्नीत्स॒कृत्स॑कृ॒त्सं मृ॒ड्ढीत्या॑ह। परा॑ङिव॒ ह्ये॑तर्‌हि॑ य॒ज्ञः। इ॒षि॒ता दैव्या॒ होता॑र॒ इत्या॑ह। इ॒षि॒त हि कर्म॑ क्रि॒यते। भ॒द्र॒वाच्या॑य॒ प्रेषि॑तो॒ मानु॑षः सूक्तवा॒काय॑ सू॒क्ता ब्रू॒हीत्या॑ह। आ॒शिष॑मे॒वैतामा शास्ते। स्व॒गा दैव्या॒ होतृ॑भ्य॒ इत्या॑ह। य॒ज्ञमे॒व तत्स्व॒गा क॑रोति। स्व॒स्तिर्मानु॑षेभ्य॒ इत्या॑ह। आ॒शिष॑मे॒वैतामा शास्ते। श॒य्योँर्ब्रू॒हीत्या॑ह। श॒य्युँमे॒व बा॑र्‌हस्प॒त्यं भा॑ग॒धेये॑न॒ सम॑र्धयति॥५९॥\anuvakamend[च॒र॒त्य॒ध्व॒र्युः प्रजा॑तिर्ह्वयते॒ वेदाब्रवीद्बर्‌हि॒षदं॑ करोत्यृ॒त्विजो॑ दधाति ब्र॒ह्माऽनु॑करोति च॒त्वारि॑ च]

%3.3.9.1
अथ॒ स्रुचा॑वनु॒ष्टुग्भ्यां॒ वाज॑वतीभ्या॒व्व्यूँ॑हति। प्र॒ति॒ष्ठा वा अ॑नु॒ष्टुक्। अन्नं॒ वाज॒ प्रति॑ष्ठित्यै। अ॒न्नाद्य॒स्याव॑रुद्ध्यै। प्राचीं जु॒हूमू॑हति। जा॒ताने॒व भ्रातृ॑व्या॒न्प्रणु॑दते। प्र॒तीची॑मुप॒भृतम्। ज॒नि॒ष्यमा॑णाने॒व प्रति॑नुदते। सविषू॑च ए॒वापोह्य॑ स॒पत्ना॒न्॒ यज॑मानः। अ॒स्मिल्लोँ॒के प्रति॑तिष्ठति॥६०॥

%3.3.9.2
द्वाभ्याम्। द्विप्र॑तिष्ठो॒ हि। वसु॑भ्यस्त्वा रु॒द्रेभ्य॑स्त्वाऽऽदि॒त्येभ्य॒स्त्वेत्या॑ह। य॒था॒य॒जुरे॒वैतत्। स्रु॒क्षु प्र॑स्त॒रम॑नक्ति। इ॒मे वै लो॒काः स्रुच॑। यज॑मानः प्रस्त॒रः। यज॑मानमे॒व तेज॑साऽनक्ति। त्रे॒धाऽन॑क्ति। त्रय॑ इ॒मे लो॒काः॥६१॥

%3.3.9.3
ए॒भ्य ए॒वैनं॑ लो॒केभ्यो॑ऽनक्ति। अ॒भि॒पू॒र्वम॑नक्ति। अ॒भि॒पू॒र्वमे॒व यज॑मान॒न्तेज॑साऽनक्ति। अ॒क्त रिहा॑णा॒ इत्या॑ह। तेजो॒ वा आज्यम्। यज॑मानः प्रस्त॒रः। यज॑मानमे॒व तेज॑साऽनक्ति। वि॒यन्तु॒ वय॒ इत्या॑ह। वय॑ ए॒वैनं॑ कृ॒त्वा। सु॒व॒र्गं लो॒कं ग॑मयति॥६२॥

%3.3.9.4
प्र॒जाय्योँनिं॒ मा निर्मृ॑क्ष॒मित्या॑ह। प्र॒जायै॑ गोपी॒थाय॑। आप्या॑यन्ता॒माप॒ ओष॑धय॒ इत्या॑ह। आप॑ ए॒वौष॑धी॒रा प्या॑ययति। म॒रुतां॒ पृष॑तय॒ स्थेत्या॑ह। म॒रुतो॒ वै वृष्ट्या॑ ईशते। वृष्टि॑मे॒वाव॑ रुन्धे। दिवं॑ गच्छ॒ ततो॑ नो॒ वृष्टि॒मेर॒येत्या॑ह। वृष्टि॒र्वै द्यौः। वृष्टि॑मे॒वाव॑ रुन्धे॥६३॥

%3.3.9.5
याव॒द्वा अ॑ध्व॒र्युः प्र॑स्त॒रं प्र॒हर॑ति। ताव॑द॒स्यायु॑र्मीयते। आ॒यु॒ष्पा अ॑ग्ने॒ऽस्यायु॑र्मे पा॒हीत्या॑ह। आयु॑रे॒वात्मन्ध॑त्ते। याव॒द्वा अ॑ध्व॒र्युः प्र॑स्त॒रं प्र॒हर॑ति। ताव॑दस्य॒ चक्षु॑र्मीयते। च॒क्षु॒ष्पा अ॑ग्नेऽसि॒ चक्षु॑र्मे पा॒हीत्या॑ह। चक्षु॑रे॒वात्मन्ध॑त्ते। ध्रु॒वाऽसीत्या॑ह॒ प्रति॑ष्ठित्यै। यं प॑रि॒धिं प॒र्यध॑त्था॒ इत्या॑ह॥६४॥

%3.3.9.6
य॒था॒य॒जुरे॒वैतत्। अग्ने॑ देव प॒णिभि॑र्वी॒यमा॑ण॒ इत्या॑ह। अ॒ग्नय॑ ए॒वैनं॒ जुष्टं॑ करोति। तन्त॑ ए॒तमनु॒ जोषं॑ भरा॒मीत्या॑ह। स॒जा॒ताने॒वास्मा॒ अनु॑कान्करोति। नेदे॒ष त्वद॑पचे॒तया॑ता॒ इत्या॒हानु॑ख्यात्यै। य॒ज्ञस्य॒ पाथ॒ उप॒ समि॑त॒मित्या॑ह। भू॒मान॑मे॒वोपै॑ति। प॒रि॒धीन्प्र ह॑रति। य॒ज्ञस्य॒ समि॑ष्ट्यै॥६५॥

%3.3.9.7
स्रुचौ॒ सं प्रस्रा॑वयति। यदे॒व तत्र॑ क्रू॒रम्। तत्तेन॑ शमयति। जु॒ह्वामु॑प॒भृतम्। य॒ज॒मा॒न॒दे॒व॒त्या॑ वै जु॒हूः। भ्रा॒तृ॒व्य॒दे॒व॒त्यो॑प॒भृत्। यज॑मानायै॒व भ्रातृ॑व्य॒मुप॑स्तिं करोति। स॒स्रा॒वभा॑गा॒ स्थेत्या॑ह। वस॑वो॒ वै रु॒द्रा आ॑दि॒त्याः सस्रा॒वभा॑गाः। तेषा॒न्तद्भा॑ग॒धेयम्॥६६॥

%3.3.9.8
ताने॒व तेन॑ प्रीणाति। वै॒श्व॒दे॒व्यर्चा। ए॒ते हि विश्वे॑ दे॒वाः। त्रि॒ष्टुग्भ॑वति। इ॒न्द्रि॒यं वै त्रि॒ष्टुक्। इ॒न्द्रि॒यमे॒व यज॑माने दधाति। अ॒ग्नेर्वा॒मप॑न्नगृहस्य॒ सद॑सि सादया॒मीत्या॑ह। इ॒यं वा अ॒ग्निरप॑न्नगृहः। अ॒स्या ए॒वैने॒ सद॑ने सादयति। सु॒म्नाय॑ सुम्निनी सु॒म्ने मा॑ धत्त॒मित्या॑ह॥६७॥

%3.3.9.9
प्र॒जा वै प॒शव॑ सु॒म्नम्। प्र॒जामे॒व प॒शूना॒त्मन्ध॑त्ते। धु॒रि धु॒र्यौ॑ पात॒मित्या॑ह। जा॒या॒प॒त्योर्गो॑पी॒थाय॑। अग्ने॑ऽदब्धायोऽशीततनो॒ इत्या॑ह। य॒था॒य॒जुरे॒वैतत्। पा॒हि मा॒ऽद्य दि॒वः पा॒हि प्रसि॑त्यै पा॒हि दुरि॑ष्ट्यै पा॒हि दु॑रद्म॒न्यै पा॒हि दुश्च॑रिता॒दित्या॑ह। आ॒शिष॑मे॒वैतामा शास्ते। अवि॑षन्नः पि॒तुं कृ॑णु सु॒षदा॒ योनि॒ स्वाहेतीध्मसं॒वृश्च॑नान्यन्वाहार्य॒पच॑नेऽभ्या॒धाय॑ फलीकरणहो॒मं जु॑होति। अति॑रिक्तानि॒ वा इ॑ध्मसं॒ वृश्च॑नानि॥६८॥

%3.3.9.10
अति॑रिक्ताः फली॒कर॑णाः। अति॑रिक्तमाज्योच्छेष॒णम्। अति॑रिक्त ए॒वाति॑रिक्तन्दधाति। अथो॒ अति॑रिक्तेनै॒वाति॑रिक्तमा॒प्त्वाऽव॑ रुन्धे। वेदि॑र्दे॒वेभ्यो॒ निला॑यत। तां वे॒देनान्व॑विन्दन्। वे॒देन॒ वेदिं॑ विविदुः पृथि॒वीम्। सा प॑प्रथे पृथि॒वी पार्थि॑वानि। गर्भं॑ बिभर्ति॒ भुव॑नेष्व॒न्तः। ततो॑ य॒ज्ञो जा॑यते विश्व॒दानि॒रिति॑ पु॒रस्तात्स्तम्बय॒जुषो॑ वे॒देन॒ वेदि॒ संमा॒र्ष्ट्यनु॑वित्त्यै॥६९॥

%3.3.9.11
अथो॒ यद्वे॒दश्च॒ वेदि॑श्च॒ भव॑तः। मि॒थु॒न॒त्वाय॒ प्रजात्यै। प्र॒जाप॑ते॒र्वा ए॒तानि॒ श्मश्रू॑णि। यद्वे॒दः। पत्नि॑या उ॒पस्थ॒ आस्य॑ति। मि॒थु॒नमे॒व क॑रोति। वि॒न्दते प्र॒जाम्। वे॒द होताऽऽह॑व॒नीयात्स्तृ॒णन्ने॑ति। य॒ज्ञमे॒व तत्सन्त॑नो॒त्योत्त॑रस्मादर्धमा॒सात्। त सन्त॑त॒मुत्त॑रेऽर्धमा॒स आल॑भते॥७०॥

%3.3.9.12
तङ्का॒लेका॑ल॒ आग॑ते यजते। ब्र॒ह्म॒वा॒दिनो॑ वदन्ति। स त्वा अ॑ध्व॒र्युः स्यात्। यो यतो॑ य॒ज्ञं प्र॑यु॒ङ्क्ते। तदे॑नं प्रतिष्ठा॒पय॒तीति॑। वाता॒द्वा अ॑ध्व॒र्युर्य॒ज्ञं प्रयु॑ङ्क्ते। देवा॑ गातुविदो गा॒तुं वि॒त्वा गा॒तुमि॒तेत्या॑ह। यत॑ ए॒व य॒ज्ञं प्र॑यु॒ङ्क्ते। तदे॑नं॒ प्रति॑ष्ठापयति। प्रति॑ तिष्ठति प्र॒जया॑ प॒शुभि॒र्यज॑मानः॥७१॥\anuvakamend[ति॒ष्ठ॒ती॒मे लो॒का ग॑मयति॒ द्यौर्वृष्टि॑मे॒वाव॑रुन्धे प॒र्यध॑त्था॒ इत्या॑ह॒ समि॑ष्ट्यै भाग॒धेय॑न्धत्त॒मित्या॑ह॒ वा इ॑ध्मसं॒ वृश्च॑ना॒न्यनु॑वित्त्यै लभते॒ यज॑मानः]

%3.3.10.1
यो वा अय॑थादेवतं य॒ज्ञमु॑प॒चर॑ति। आ दे॒वताभ्यो वृश्च्यते। पापी॑यान्भवति। यो य॑थादेव॒तम्। न दे॒वताभ्य॒ आवृ॑श्च्यते। वसी॑यान्भवति। वा॒रु॒णो वै पाश॑। इ॒मं विष्या॑मि॒ वरु॑णस्य॒ पाश॒मित्या॑ह। व॒रु॒ण॒पा॒शादे॒वैनां मुञ्चति। स॒वि॒तृप्र॑सूतो यथादेव॒तम्॥७२॥

%3.3.10.2
न दे॒वताभ्य॒ आवृ॑श्च्यते। वसी॑यान्भवति। धा॒तुश्च॒ योनौ॑ सुकृ॒तस्य॑ लो॒क इत्या॑ह। अ॒ग्निर्वै धा॒ता। पुण्य॒ङ्कर्म॑ सुकृ॒तस्य॑ लो॒कः। अ॒ग्निरे॒वैनान्धा॒ता। पुण्ये॒ कर्म॑णि सुकृ॒तस्य॑ लो॒के द॑धाति। स्यो॒नं मे॑ स॒ह पत्या॑ करो॒मीत्या॑ह। आ॒त्मन॑श्च॒ यज॑मानस्य॒ चानात्यै स॒न्त्वाय॑। समायु॑षा॒ सं प्र॒जयेत्या॑ह॥७३॥

%3.3.10.3
आ॒शिष॑मे॒वैतामा शास्ते पूर्णपा॒त्रे। अ॒न्त॒तो॑ऽनु॒ष्टुभा। चतु॑ष्प॒द्वा ए॒तच्छन्द॒ प्रति॑ष्ठितं॒ पत्नि॑यै पूर्णपा॒त्रे भ॑वति। अ॒स्मिल्लोँ॒के प्रति॑तिष्ठा॒नीति॑। अ॒स्मिन्ने॒व लो॒के प्रति॑तिष्ठति। अथो॒ वाग्वा अ॑नु॒ष्टुक्। वाङ्मि॑थु॒नम्। आपो॒ रेत॑ प्र॒जन॑नम्। ए॒तस्मा॒द्वै मि॑थु॒नाद्वि॒द्योत॑मानः स्त॒नय॑न्वर्‌षति। रेत॑ सि॒ञ्चन्॥७४॥

%3.3.10.4
प्र॒जाः प्र॑ज॒नय\sn{}। यद्वै य॒ज्ञस्य॒ ब्रह्म॑णा यु॒ज्यते। ब्रह्म॑णा॒ वै तस्य॑ विमो॒कः। अ॒द्भिः शान्ति॑। विमु॑क्तं॒ वा ए॒तर्‌हि॒ योक्त्रं॒ ब्रह्म॑णा। आ॒दायै॑न॒त्पत्नी॑ स॒हाप उप॑गृह्णीते॒ शान्त्यै। अ॒ञ्ज॒लौ पूर्णपा॒त्रमा न॑यति। रेत॑ ए॒वास्यां प्र॒जान्द॑धाति। प्र॒जया॒ हि म॑नु॒ष्य॑ पू॒र्णः। मुखं॒ वि मृ॑ष्टे। अ॒व॒भृ॒थस्यै॒व रू॒पं कृ॒त्वोत्ति॑ष्ठति॥७५॥\anuvakamend[स॒वि॒तृप्र॑सूतो यथादेव॒तं प्र॒जयेत्या॑ह सि॒ञ्चन्मृ॑ष्ट॒ एकं च]

%3.3.11.1
प॒रि॒वे॒षो वा ए॒ष वन॒स्पती॑नाम्। यदु॑पवे॒षः। य ए॒वं वेद॑। वि॒न्दते॑ परिवे॒ष्टारम्। तमु॑त्क॒रे। यन्दे॒वा म॑नु॒ष्ये॑षु। उ॒प॒वे॒षमधा॑रयन्। ये अ॒स्मदप॑ चेतसः। तान॒स्मभ्य॑मि॒हा कु॑रु। उप॑वे॒षोप॑ विड्ढि नः॥७६॥

%3.3.11.2
प्र॒जां पुष्टि॒मथो॒ धनम्। द्वि॒पदो॑ न॒श्चतु॑ष्पदः। ध्रु॒वानन॑पगान्कु॒र्विति॑ पु॒रस्तात्प्र॒त्यञ्च॒मुप॑ गूहति। तस्मात्पु॒रस्तात्प्र॒त्यञ्च॑ शू॒द्रा अव॑स्यन्ति। स्थ॒वि॒म॒त उप॑गूहति। अप्र॑तिवादिन ए॒वैनान्कुरुते। धृष्टि॒र्वा उ॑पवे॒षः। शु॒चर्तो वज्रो॒ ब्रह्म॑णा॒ सशि॑तः। योप॑वे॒षे शुक्। साऽमुमृ॑च्छतु॒ यं द्वि॒ष्म इति॑॥७७॥

%3.3.11.3
अथास्मै नाम॒ गृह्य॒ प्रह॑रति। निर॒मुन्नु॑द॒ ओक॑सः। स॒पत्नो॒ यः पृ॑त॒न्यति॑। नि॒र्बा॒ध्ये॑न ह॒विषा। इन्द्र॑ एणं॒ परा॑शरीत्। इ॒हि ति॒स्रः प॑रा॒वत॑। इ॒हि पञ्च॒ जना॒ अति॑। इ॒हि ति॒स्रोऽति॑ रोच॒नायाव॑त्। सूर्यो॒ अस॑द्दि॒वि। प॒र॒मान्त्वा॑ परा॒वतम्॥७८॥

%3.3.11.4
इन्द्रो॑ नयतु वृत्र॒हा। यतो॒ न पुन॒राय॑सि। श॒श्व॒तीभ्य॒ समाभ्य॒ इति॑। त्रि॒वृद्वा ए॒ष वज्रो॒ ब्रह्म॑णा॒ सशि॑तः। शु॒चैवैनं॑ वि॒ध्वा। ए॒भ्यो लो॒केभ्यो॑ नि॒र्णुद्य॑। वज्रे॑ण॒ ब्रह्म॑णा स्तृणुते। ह॒तो॑ऽसावव॑धिष्मा॒मुमित्या॑ह॒ स्तृत्यै। यं द्वि॒ष्यात्तन्ध्या॑येत्। शु॒चैवैन॑मर्पयति॥७९॥




\prashnaend{प्रत्यु॑ष्टन्दि॒वः शिल्प॒मय॑ज्ञो घृ॒तं च॑ देवासु॒राः स ए॒तमिन्द्र आपो॑ देवीर॒ग्निना॒ धिष्णि॑या॒ अथ॒ स्रुचौ॒ यो वा अय॑थादेवतं परिवे॒षो वा एका॑दश॥११॥}{प्रत्यु॑ष्ट॒मय॑ज्ञ ए॒षा हि विश्वे॑षान्दे॒वाना॑मू॒र्जा पृ॑थि॒वीमथो॒ रक्ष॑सा॒न्तां प्रजा॑ति॒न्द्वाभ्या॒न्तङ्का॒लेका॑ले॒ नव॑सप्ततिः॥७९॥}{प्रत्यु॑ष्टमर्पयति॥}{हरि॑ ओम्॥}{इति श्रीकृष्णयजुर्वेदीयतैत्तिरीयब्राह्मणे तृतीयाष्टके तृतीयः प्रपाठकः समाप्तः॥}
\clearpage
\sect{चतुर्थः प्रश्नः}
\setcounter{anuvakam}{0}
\dnsub{तैत्तिरीयब्राह्मणे तृतीयाष्टके चतुर्थः प्रपाठकः}

%3.4.1.1
ब्रह्म॑णे ब्राह्म॒णमाल॑भते। क्ष॒त्राय॑ राज॒न्यम्। म॒रुद्भ्यो॒ वैश्यम्। तप॑से शू॒द्रम्। तम॑से॒ तस्क॑रम्। नार॑काय वीर॒हणम्। पा॒प्मने क्ली॒बम्। आ॒क्र॒याया॑यो॒गूम्। कामा॑य पुश्च॒लूम्। अति॑क्रुष्टाय माग॒धम्॥१॥

%3.4.2.1
गी॒ताय॑ सू॒तम्। नृ॒त्ताय॑ शैलू॒षम्। धर्मा॑य सभाच॒रम्। न॒र्माय॑ रे॒भम्। नरि॑ष्ठायै भीम॒लम्। हसा॑य॒ कारिम्। आ॒न॒न्दाय॑ स्त्रीष॒खम्। प्रमुदे॑ कुमारीपु॒त्रम्। मे॒धायै॑ रथका॒रम्। धैर्या॑य॒ तक्षा॑णम्॥२॥

%3.4.3.1
श्रमा॑य कौला॒लम्। मा॒यायै॑ कार्मा॒रम्। रू॒पाय॑ मणिका॒रम्। शुभे॑ व॒पम्। श॒र॒व्या॑या इषुका॒रम्। हे॒त्यै ध॑न्वका॒रम्। कर्म॑णे ज्याका॒रम्। दि॒ष्टाय॑ रज्जुस॒र्गम्। मृ॒त्य॑वे मृग॒युम्। अन्त॑काय श्व॒नितम्॥३॥

%3.4.4.1
स॒न्धये॑ जा॒रम्। गे॒हायो॑पप॒तिम्। निर्\mbox{}ऋ॑त्यै परिवि॒त्तम्। आर्त्यै॑ परिविविदा॒नम्। अराध्यै दिधिषू॒पतिम्। प॒वित्रा॑य भि॒षजम्। प्र॒ज्ञाना॑य नक्षत्रद॒र्॒शम्। निष्कृ॑त्यै पेशस्का॒रीम्। बला॑योप॒दाम्। वर्णा॑यानू॒रुधम्॥४॥

%3.4.5.1
न॒दीभ्य॑ पौञ्जि॒ष्टम्। ऋ॒क्षीकाभ्यो॒ नैषा॑दम्। पु॒रु॒ष॒व्या॒घ्राय॑ दु॒र्मदम्। प्र॒युद्भ्य॒ उन्म॑त्तम्। ग॒न्ध॒र्वा॒प्स॒राभ्यो॒ व्रात्यम्। स॒र्प॒दे॒व॒ज॒नेभ्योऽप्र॑तिपदम्। अवेभ्यः कित॒वम्। इ॒र्यता॑या॒ अकि॑तवम्। पि॒शा॒चेभ्यो॑ बिदलका॒रम्। या॒तु॒धानेभ्यः कण्टकका॒रम्॥५॥

%3.4.6.1
उ॒त्सा॒देभ्य॑ कु॒ब्जम्। प्र॒मुदे॑ वाम॒नम्। द्वा॒र्भ्यः स्रा॒मम्। स्वप्ना॑या॒न्धम्। अध॑र्माय बधि॒रम्। सं॒ज्ञाना॑य स्मरका॒रीम्। प्र॒का॒मोद्या॑योप॒सदम्। आ॒शि॒क्षायै प्र॒श्ञिनम्। उ॒प॒शि॑क्षाया॑ अभिप्र॒श्ञिनम्। म॒र्यादा॑यै प्रश्ञविवा॒कम्॥६॥

%3.4.7.1
ऋत्यै स्ते॒नहृ॑दयम्। वैर॑हत्याय॒ पिशु॑नम्। विवि॑त्त्यै क्ष॒त्तारम्। औप॑द्रष्टाय सङ्ग्रही॒तारम्। बला॑यानुच॒रम्। भू॒म्ने प॑रिष्क॒न्दम्। प्रि॒याय॑ प्रियवा॒दिनम्। अरि॑ष्ट्या अश्वसा॒दम्। मेधा॑य वासः पल्पू॒लीम्। प्र॒का॒माय॑ रजयि॒त्रीम्॥७॥

%3.4.8.1
भायै॑ दार्वाहा॒रम्। प्र॒भाया॑ आग्ने॒न्धम्। नाक॑स्य पृ॒ष्ठाया॑भिषे॒क्तारम्। ब्र॒ध्नस्य॑ वि॒ष्टपा॑य पात्रनिर्णे॒गम्। दे॒व॒लो॒काय॑ पेशि॒तारम्। म॒नु॒ष्य॒लो॒काय॑ प्रकरि॒तारम्। सर्वेभ्यो लो॒केभ्य॑ उपसे॒क्तारम्। अव॑र्त्यै व॒धायो॑पमन्थि॒तारम्। सु॒व॒र्गाय॑ लो॒काय॑ भाग॒दुघम्। वर्\mbox{}षि॑ष्ठाय॒ नाका॑य परिवे॒ष्टारम्॥८॥

%3.4.9.1
अर्मेभ्यो हस्ति॒पम्। ज॒वायाश्व॒पम्। पुष्ट्यै॑ गोपा॒लम्। तेज॑सेऽजपा॒लम्। वी॒र्या॑याविपा॒लम्। इरा॑यै की॒नाशम्। की॒लाला॑य सुराका॒रम्। भ॒द्राय॑ गृह॒पम्। श्रेय॑से वित्त॒धम्। अध्य॑क्षायानुक्ष॒त्तारम्॥९॥

%3.4.10.1
म॒न्यवे॑ऽयस्ता॒पम्। क्रोधा॑य निस॒रम्। शोका॑याभिस॒रम्। उ॒त्कू॒ल॒वि॒कू॒लाभ्यान्त्रि॒स्थिनम्। योगा॑य यो॒क्तारम्। क्षेमा॑य विमो॒क्तारम्। वपु॑षे मानस्कृ॒तम्। शीला॑याञ्जनीका॒रम्। निर्\mbox{}ऋ॑त्यै कोशका॒रीम्। य॒माया॒सूम्॥१०॥

%3.4.11.1
य॒म्यै॑ यम॒सूम्। अथ॑र्व॒भ्योऽव॑तोकाम्। सं॒व॒त्स॒राय॑ पर्या॒रिणीम्। प॒रि॒व॒त्स॒रायावि॑जाताम्। इ॒दा॒व॒त्स॒राया॑प॒स्कद्व॑रीम्। इ॒द्व॒त्स॒राया॒तीत्व॑वरीम्। व॒त्स॒राय॒ विज॑र्जराम्। स॒र्व॒न्त्स॒राय॒ पलि॑क्नीम्। वना॑य वन॒पम्। अ॒न्यतो॑रण्याय दाव॒पम्॥११॥

%3.4.12.1
सरोभ्यो धैव॒रम्। वेश॑न्ताभ्यो॒ दाशम्। उ॒प॒स्थाव॑रीभ्यो॒ बैन्दम्। न॒ड्व॒लाभ्य॑ शौष्क॒लम्। पा॒र्या॑य कैव॒र्तम्। अ॒वा॒र्या॑य मार्गा॒रम्। ती॒र्थेभ्य॑ आ॒न्दम्। विष॑मेभ्यो मैना॒लम्। स्वनेभ्य॒ पर्ण॑कम्। गुहाभ्य॒ किरा॑तम्। सानु॑भ्यो॒ जम्भ॑कम्। पर्व॑तेभ्य॒ किंपू॑रुषम्॥१२॥

%3.4.13.1
प्र॒ति॒श्रुत्का॑या ऋतु॒लम्। घोषा॑य भ॒षम्। अन्ता॑य बहुवा॒दिनम्। अ॒न॒न्ताय॒ मूकम्। मह॑से वीणावा॒दम्। क्रोशा॑य तूणव॒ध्मम्। आ॒क्र॒न्दाय॑ दुन्दुभ्याघा॒तम्। अ॒व॒र॒स्प॒राय॑ शङ्ख॒ध्मम्। ऋ॒भुभ्यो॑जिनसन्धा॒यम्। सा॒ध्येभ्य॑श्चर्म॒म्णम्।॥१३॥

%3.4.14.1
बी॒भ॒त्सायै॑ पौल्क॒सम्। भूत्यै॑ जागर॒णम्। अभूत्यै स्वप॒नम्। तु॒लायै॑ वाणि॒जम्। वर्णा॑य हिरण्यका॒रम्। विश्वेभ्यो दे॒वेभ्य॑ सिध्म॒लम्। प॒श्चा॒द्दो॒षाय॑ ग्ला॒वम्। ऋत्यै॑ जनवा॒दिनम्। व्यृ॑द्ध्या अपग॒ल्भम्। स॒श॒राय॑ प्र॒च्छिदम्॥१४॥

%3.4.15.1
हसा॑य पुश्च॒लूमा ल॑भते। वी॒णा॒वा॒दङ्गण॑कङ्गी॒ताय॑। याद॑से शाबु॒ल्याम्। न॒र्माय॑ भद्रव॒तीम्। तू॒ण॒व॒ध्मङ्ग्रा॑म॒ण्यं॑ पाणिसङ्घा॒तन्नृ॒त्ताय॑। मोदा॑यानु॒क्रोश॑कम्। आ॒न॒न्दाय॑ तल॒वम्॥१५॥

%3.4.16.1
अ॒क्ष॒रा॒जाय॑ कित॒वम्। कृ॒ताय॑ सभा॒विनम्। त्रेता॑या आदिनवद॒र्॒शम्। द्वा॒प॒राय॑ बहि॒ सदम्। कल॑ये सभास्था॒णुम्। दु॒ष्कृ॒ताय॑ च॒रका॑चार्यम्। अध्व॑ने ब्रह्मचा॒रिणम्। पि॒शा॒चेभ्य॑ सैल॒गम्। पि॒पा॒सायै॑ गोव्य॒च्छम्। निर्\mbox{}ऋ॑त्यै गोघा॒तम्। क्षु॒धे गो॑विक॒र्तम्। क्षु॒त्तृ॒ष्णाभ्या॒न्तम्। यो गां वि॒कृन्त॑न्तं मा॒सं भिक्ष॑माण उप॒तिष्ठ॑ते॥१६॥

%3.4.17.1
भूम्यै॑ पीठस॒र्पिण॒मा ल॑भते। अ॒ग्नयेऽस॒लम्। वा॒यवे॑ चाण्डा॒लम्। अ॒न्तरि॑क्षाय वशन॒र्तिनम्। दि॒वे ख॑ल॒तिम्। सूर्या॑य हर्य॒क्षम्। च॒न्द्रम॑से मिर्मि॒रम्। नक्ष॑त्रेभ्यः कि॒लासम्। अह्ने॑ शु॒क्लं पि॑ङ्ग॒लम्। रात्रि॑यै कृ॒ष्णं पि॑ङ्गा॒क्षम्॥१७॥

%3.4.18.1
वा॒चे पुरु॑ष॒मा ल॑भते। प्रा॒णम॑पा॒नव्व्याँ॒नमु॑दा॒न स॑मा॒नन्तान् वा॒यवे। सूर्या॑य॒ चक्षु॒रा ल॑भते। मन॑श्च॒न्द्रम॑से। दि॒ग्भ्यः श्रोत्रम्। प्र॒जाप॑तये॒ पुरु॑षम्॥१८॥

%3.4.19.1
अथै॒तानरू॑पेभ्य॒ आल॑भते। अति॑ह्रस्व॒मति॑दीर्घम्। अतिकृ॑श॒मत्यसलम्। अति॑शुक्ल॒मति॑कृष्णम्। अति॑श्लक्ष्ण॒मति॑लोमशम्। अति॑किरिट॒मति॑दन्तुरम्। अति॑मिर्मिर॒मति॑मेमिषम्। आ॒शायै॑ जा॒मिम्। प्र॒ती॒क्षायै॑ कुमा॒रीम्॥१९॥%\anuvakamend[नो॒ द्वि॒ष्म इति॑ परा॒वत॑मर्पयति]




\prashnaend{ब्रह्म॑णे गी॒ताय॒ श्रमा॑य स॒न्धये॑ न॒दीभ्य॑ उत्सा॒देभ्य॒ ऋत्यै॒ भाया॒ अर्मेभ्यो म॒न्यवे॑ य॒म्यै॑ दश॑दश॒ सरोभ्यो॒ द्वाद॑श प्रति॒श्रुत्का॑यै बीभ॒त्सायै॒ दश॑दश॒ हसा॑य स॒प्ताक्ष॑रा॒जाय॒ त्रयो॑दश॒ भूम्यै॒ दश॑ वा॒चे षडथ॒ नवैका॒न्नविशतिः॥१९॥}{ब्रह्म॑णे य॒म्यै॑ नव॑दश॥१९॥}{ब्रह्म॑णे कुमा॒रीम्॥}{हरि॑ ओम्॥}{इति श्रीकृष्णयजुर्वेदीयतैत्तिरीयब्राह्मणे तृतीयाष्टके चतुर्थः प्रपाठकः समाप्तः॥}
\clearpage
\sect{पञ्चमः प्रश्नः}
\setcounter{anuvakam}{0}
\dnsub{तैत्तिरीयब्राह्मणे तृतीयाष्टके पञ्चमः प्रपाठकः}

%3.5.1.1
स॒त्यं प्रप॑द्ये। ऋ॒तं प्रप॑द्ये। अ॒मृतं॒ प्रप॑द्ये। प्र॒जाप॑तेः प्रि॒यान्त॒नुव॒मनार्तां॒ प्रप॑द्ये। इ॒दम॒हं प॑ञ्चद॒शेन॒ वज्रे॑ण। द्वि॒षन्तं॒ भ्रातृ॑व्य॒मव॑ क्रामामि। योऽस्मान्द्वेष्टि॑। यं च॑ व॒यं द्वि॒ष्मः। भूर्भुव॒ सुव॑। हिम्॥१॥\anuvakamend[स॒त्यन्दश॑]

%3.5.2.1
प्र वो॒ वाजा॑ अ॒भिद्य॑वः। ह॒विष्म॑न्तो घृ॒ताच्या। दे॒वाञ्जि॑गाति सुम्न॒युः। अग्न॒ आया॑हि वी॒तये। गृ॒णा॒नो ह॒व्यदा॑तये। नि होता॑ सत्सि ब॒र्॒हिषि॑। तन्त्वा॑ स॒मिद्भि॑रङ्गिरः। घृ॒तेन॑ वर्धयामसि। बृ॒हच्छो॑चा यविष्ठ्य। स न॑ पृ॒थुः श्र॒वाय्यम्॥२॥

%3.5.2.2
अच्छा॑ देव विवाससि। बृ॒हद॑ग्ने सु॒वीर्यम्। ई॒डेन्यो॑ नम॒स्य॑स्ति॒रः। तमासि दर्‌श॒तः। सम॒ग्निरि॑ध्यते॒ वृषा। वृषो॑ अ॒ग्निः समि॑ध्यते। अश्वो॒ न दे॑व॒वाह॑नः। त ह॒विष्म॑न्त ईडते। वृष॑णन्त्वा व॒यं वृष\sn{}। वृषा॑ण॒ समि॑धीमहि॥३॥

%3.5.2.3
अग्ने॒ दीद्य॑तं बृ॒हत्। अ॒ग्निं दू॒तं वृ॑णीमहे। होता॑रं वि॒श्ववे॑दसम्। अ॒स्य य॒ज्ञस्य॑ सु॒क्रतुम्। स॒मि॒ध्यमा॑नो अध्व॒रे। अ॒ग्निः पा॑व॒क ईड्य॑। शो॒चिष्के॑श॒स्तमी॑महे। समि॑द्धो अग्न आहुत। दे॒वान् य॑क्षि स्वध्वर। त्व हि ह॑व्य॒वाडसि॑। आ जु॑होत दुव॒स्यत॑। अ॒ग्निं प्र॑य॒त्य॑ध्व॒रे। वृ॒णीध्व ह॑व्य॒वाह॑नम्। त्वं वरु॑ण उ॒त मि॒त्रो अ॑ग्ने। त्वां व॑र्धन्ति म॒तिभि॒र्वसि॑ष्ठाः। त्वे वसु॑ सुषण॒नानि॑ सन्तु। यू॒यं पा॑त स्व॒स्तिभि॒ सदा॑ नः ॥४॥\anuvakamend[श्र॒वाय्य॑मिधीम॒ह्यसि॑ स॒प्त च॑]

%3.5.3.1
अग्ने॑ म॒हा अ॑सि ब्राह्मण भारत। असा॒वसौ। दे॒वेद्धो॒ मन्वि॑द्धः। ऋषि॑ष्टुतो॒ विप्रा॑नुमदितः। क॒वि॒श॒स्तो ब्रह्म॑सशितो घृ॒ताह॑वनः। प्र॒णीर्य॒ज्ञानाम्। र॒थीर॑ध्व॒राणाम्। अ॒तूर्तो॒ होता। तूर्णि॑र्‌हव्य॒वाट्। आस्पात्रं॑ जु॒हूर्दे॒वानाम्॥५॥

%3.5.3.2
च॒म॒सो दे॑व॒पान॑। अ॒रा इ॑वाग्ने ने॒मिर्दे॒वास्त्वं प॑रि॒भूर॑सि। आ व॑ह दे॒वान् यज॑मानाय। अ॒ग्निम॑ग्न॒ आव॑ह। सोम॒माव॑ह। अ॒ग्निमाव॑ह। प्र॒जाप॑ति॒माव॑ह। अ॒ग्नीषोमा॒वाव॑ह। इ॒न्द्रा॒ग्नी आव॑ह। इन्द्र॒माव॑ह। म॒हे॒न्द्रमाव॑ह। दे॒वा आज्य॒पा आव॑ह। अ॒ग्नि हो॒त्रायाव॑ह। स्वं म॑हि॒मान॒मा व॑ह। आ चाग्ने दे॒वान् वह॑। सु॒यजा॑ च यज जातवेदः॥६॥\anuvakamend[दे॒वाना॒मिन्द्र॒मा व॑ह॒ षट् च॑]

%3.5.4.1
अ॒ग्निर्\mbox{}होता॒ वेत्व॒ग्निः। हो॒त्रं वेत्तु प्रावि॒त्रम्। स्मो व॒यम्। सा॒धु ते॑ यजमान दे॒वता। घृ॒तव॑तीमध्वर्यो॒ स्रुच॒मास्य॑स्व। दे॒वा॒युवं॑ वि॒श्ववा॑राम्। ईडा॑महै दे॒वा ई॒डेन्यान्॑। न॒म॒स्याम॑ नम॒स्यान्॑। यजा॑म य॒ज्ञियान्॑॥७॥\anuvakamend[अ॒ग्निर्‌होता॒ नव॑]

%3.5.5.1
स॒मिधो॑ अग्न॒ आज्य॑स्य वियन्तु। तनू॒नपा॑दग्न॒ आज्य॑स्य वेतु। इ॒डो अ॑ग्न॒ आज्य॑स्य वियन्तु। ब॒र्॒हिर॑ग्न॒ आज्य॑स्य वेतु। स्वाहा॒ऽग्निम्। स्वाहा॒ सोमम्। स्वाहा॒ऽग्निम्। स्वाहा प्र॒जाप॑तिम्। स्वाहा॒ऽग्नीषोमौ। स्वाहेन्द्रा॒ग्नी। स्वाहेन्द्रम्। स्वाहा॑ महे॒न्द्रम्। स्वाहा॑ दे॒वा आज्य॒पान्। स्वाहा॒ऽग्नि हो॒त्राज्जु॑षा॒णाः। अग्न॒ आज्य॑स्य वियन्तु॥८॥\anuvakamend[इ॒न्द्रा॒ग्नी पञ्च॑ च]

%3.5.6.1
अ॒ग्निर्वृ॒त्राणि॑ जङ्घनत्। द्र॒वि॒ण॒स्युर्वि॑प॒न्यया। समि॑द्धः शु॒क्र आहु॑तः। जु॒षा॒णो अ॒ग्निराज्य॑स्य वेतु। त्व सो॑मासि॒ सत्प॑तिः। त्व राजो॒त वृ॑त्र॒हा। त्वं भ॒द्रो अ॑सि॒ क्रतु॑। जु॒षा॒णः सोम॒ आज्य॑स्य ह॒विषो॑ वेतु। अ॒ग्निः प्र॒त्नेन॒ जन्म॑ना। शुम्भा॑नस्त॒नुव॒ स्वाम्। क॒विर्विप्रे॑ण वावृधे। जु॒षा॒णो अ॒ग्निराज्य॑स्य वेतु। सोम॑ गी॒र्भिष्ट्वा॑ व॒यम्। व॒र्धया॑मो वचो॒विद॑। सु॒मृ॒डी॒को न॒ आवि॑श। जु॒षा॒णः सोम॒ आज्य॑स्य ह॒विषो॑ वेतु॥९॥\anuvakamend[स्वा षट् च॑]

%3.5.7.1
अ॒ग्निर्मू॒र्धा दि॒वः क॒कुत्। पति॑ पृथि॒व्या अ॒यम्। अ॒पा रेतासि जिन्वति। भुवो॑ य॒ज्ञस्य॒ रज॑सश्च ने॒ता। यत्रा॑ नि॒युद्भि॒ सच॑से शि॒वाभि॑। दि॒वि मू॒र्धान॑न्दधिषे सुव॒र्॒षाम्। जि॒ह्वाम॑ग्ने चकृषे हव्य॒वाहम्। प्रजा॑पते॒ न त्वदे॒तान्य॒न्यः। विश्वा॑ जा॒तानि॒ परि॒ ता ब॑भूव। यत्का॑मास्ते जहु॒मस्तन्नो॑ अस्तु॥१०॥

%3.5.7.2
व॒य स्या॑म॒ पत॑यो रयी॒णाम्। स वे॑द पु॒त्रः पि॒तर॒ स मा॒तरम्। स सू॒नुर्भु॑व॒त्स भु॑व॒त्पुन॑र्मघः। स द्यामौर्णो॑द॒न्तरि॑क्ष॒ स सुव॑। स विश्वा॒ भुवो॑ अभव॒त्स आभ॑वत्। अग्नी॑षोमा॒ सवे॑दसा। सहू॑ती वनत॒ङ्गिर॑। सन्दे॑व॒त्रा ब॑भूवथुः। यु॒वमे॒तानि॑ दि॒वि रो॑च॒नानि॑। अ॒ग्निश्च॑ सोम॒ सक्र॑तू अधत्तम्॥११॥

%3.5.7.3
यु॒व सिन्धू र॒भिश॑स्तेरव॒द्यात्। अग्नी॑षोमा॒वमु॑ञ्चतङ्गृभी॒तान्। इन्द्राग्नी रोच॒ना दि॒वः। परि॒ वाजे॑षु भूषथः। तद्वाञ्चेति॒ प्रवी॒र्यम्। श्ञथ॑द्वृ॒त्रमु॒त स॑नोति॒ वाजम्। इन्द्रा॒ यो अ॒ग्नी सहु॑री सप॒र्यात्। इ॒र॒ज्यन्ता॑ वस॒व्य॑स्य॒ भूरे। सह॑स्तमा॒ सह॑सा वाज॒यन्ता। एन्द्र॑ सान॒सि र॒यिम्॥१२॥

%3.5.7.4
स॒जित्वा॑न सदा॒सहम्। वर्‌षि॑ष्ठमू॒तये॑ भर। प्रस॑साहिषे पुरुहूत॒ शत्रून्॑। ज्येष्ठ॑स्ते॒ शुष्म॑ इ॒ह रा॒तिर॑स्तु। इन्द्रा भ॑र॒ दक्षि॑णेना॒ वसू॑नि। पति॒ सिन्धू॑नामसि रे॒वती॑नाम्। म॒हा इन्द्रो॒ य ओज॑सा। प॒र्जन्यो॑ वृष्टि॒मा इ॑व। स्तोमैर्व॒त्सस्य॑ वावृधे। म॒हा इन्द्रो॑ नृ॒वदाच॑र्‌षणि॒प्राः॥१३॥

%3.5.7.5
उ॒त द्वि॒बर्‌हा॑ अमि॒नः सहो॑भिः। अ॒स्म॒द्रिय॑ग्वावृधे वी॒र्या॑य। उ॒रुः पृ॒थुः सुकृ॑तः क॒र्तृभि॑र्भूत्। पि॒प्री॒हि दे॒वा उ॑श॒तो य॑विष्ठ। वि॒द्वा ऋ॒तूर्\mbox{}ऋ॑तुपते यजे॒ह। ये दैव्या॑ ऋ॒त्विज॒स्तेभि॑रग्ने। त्व होतॄ॑णाम॒स्याय॑जिष्ठः। अ॒ग्नि स्वि॑ष्ट॒कृतम्। अया॑ड॒ग्निर॒ग्नेः प्रि॒या धामा॑नि। अया॒ट्त्सोम॑स्य प्रि॒या धामा॑नि॥१४॥

%3.5.7.6
अया॑ड॒ग्नेः प्रि॒या धामा॑नि। अयाट्प्र॒जाप॑तेः प्रि॒या धामा॑नि। अया॑ड॒ग्नीषोम॑योः प्रि॒या धामा॑नि। अया॑डिन्द्राग्नि॒योः प्रि॒या धामा॑नि। अया॒डिन्द्र॑स्य प्रि॒या धामा॑नि। अयाण्महे॒न्द्रस्य॑ प्रि॒या धामा॑नि। अयाड्दे॒वाना॑माज्य॒पानां प्रि॒या धामा॑नि। यक्ष॑द॒ग्नेर्‌होतु॑ प्रि॒या धामा॑नि। यक्ष॒त्स्वं म॑हि॒मानम्। आय॑जता॒मेज्या॒ इष॑। कृ॒णोतु॒ सो अ॑ध्व॒रा जा॒तवे॑दाः। जु॒षता ह॒विः। अग्ने॒ यद॒द्य वि॒शो अ॑ध्वरस्य होतः। पाव॑क शोचे॒ वेष्ट्व हि यज्वा। ऋ॒ता य॑जासि महि॒ना वियद्भूः। ह॒व्या व॑ह यविष्ठ॒ या ते॑ अ॒द्य॥१५॥\anuvakamend[अ॒स्त्व॒ध॒त्त॒ र॒यिञ्च॑र्‌षणि॒प्राः सोम॑स्य प्रि॒या धामा॒नीष॒ष्षट्च॑]

%3.5.8.1
उप॑हूत रथन्त॒र स॒ह पृ॑थि॒व्या। उप॑ मा रथन्त॒र स॒ह पृ॑थि॒व्या ह्व॑यताम्। उप॑हूतं वामदे॒व्य स॒हान्तरि॑क्षेण। उप॑ मा वामदे॒व्य स॒हान्तरि॑क्षेण ह्वयताम्। उप॑हूतं बृ॒हत्स॒ह दि॒वा। उप॑ मा बृ॒हत्स॒ह दि॒वा ह्व॑यताम्। उप॑हूताः स॒प्त होत्रा। उप॑ मा स॒प्त होत्रा ह्वयन्ताम्। उप॑हूता धे॒नुः स॒हर्‌ष॑भा। उप॑ मा धे॒नुः स॒हर्‌ष॑भा ह्वयताम्॥१६॥

%3.5.8.2
उप॑हूतो भ॒क्षः सखा। उप॑ मा भ॒क्षः सखा ह्वयताम्। उप॑हू॒ताँ(४)हो। इडोप॑हूता। उप॑हू॒तेडा। उपो॑ अ॒स्मा इडा ह्वयताम्। इडोप॑हूता। उप॑हू॒तेडा। मा॒न॒वी घृ॒तप॑दी मैत्रावरु॒णी। ब्रह्म॑ दे॒वकृ॑त॒मुप॑हूतम्॥१७॥

%3.5.8.3
दैव्या॑ अध्व॒र्यव॒ उप॑हूताः। उप॑हूता मनु॒ष्या। य इ॒मं य॒ज्ञमवान्॑। ये य॒ज्ञप॑तिं॒ वर्धान्॑। उप॑हूते॒ द्यावा॑पृथि॒वी। पू॒र्व॒जे ऋ॒ताव॑री। दे॒वी दे॒वपु॑त्रे। उप॑हूतो॒ऽयं यज॑मानः। उत्त॑रस्यान्देवय॒ज्याया॒मुप॑हूतः। भूय॑सि हवि॒ष्कर॑ण॒ उप॑हूतः। दि॒व्ये धाम॒न्नुप॑हूतः। इ॒दं मे॑ दे॒वा ह॒विर्जु॑षन्ता॒मिति॒ तस्मि॒न्नुप॑हूतः। विश्व॑मस्य प्रि॒यमुप॑हूतम्। विश्व॑स्य प्रि॒यस्योप॑हूत॒स्योप॑हूतः॥१८॥\anuvakamend[स॒हर्‌ष॑भा ह्वयता॒मुप॑हूत हवि॒ष्कर॑ण॒ उप॑हूतश्च॒त्वारि॑ च]

%3.5.9.1
दे॒वं ब॒र्‌हिः। व॒सु॒वने॑ वसु॒धेय॑स्य वेतु। दे॒वो नरा॒शस॑। व॒सु॒वने॑ वसु॒धेय॑स्य वेतु। दे॒वो अ॒ग्निः स्वि॑ष्ट॒कृत्। सु॒द्रवि॑णा म॒न्द्रः क॒विः। स॒त्यम॑न्माय॒जी होता। होतु॑र्‌होतु॒राय॑जीयान्। अग्ने॒ यान् दे॒वानयाट्। या अपि॑प्रेः। ये ते॑ हो॒त्रे अम॑त्सत। ता स॑स॒नुषी॒ होत्रान्देवङ्ग॒माम्। दि॒वि दे॒वेषु॑ य॒ज्ञमेर॑ये॒मम्। स्वि॒ष्ट॒कृच्चाग्ने॒ होताऽभू। व॒सु॒वने॑ वसु॒धेय॑स्य नमोवा॒के वीहि॑॥१९॥\anuvakamend[अपि॑प्रे॒ पञ्च॑ च]

%3.5.10.1
इ॒दन्द्या॑वापृथिवी भ॒द्रम॑भूत्। आर्ध्म॑ सूक्तवा॒कम्। उ॒त न॑मोवा॒कम्। ऋ॒ध्यास्म॑ सू॒क्तोच्य॑मग्ने। त्व सूक्त॒वाग॑सि। उप॑श्रितो दि॒वः पृ॑थि॒व्योः। ओम॑न्वती ते॒ऽस्मिन् य॒ज्ञे य॑जमान॒ द्यावा॑पृथि॒वी स्ताम्। श॒ङ्ग॒ये जी॒रदा॑नू। अत्र॑स्नू॒ अप्र॑वेदे। उ॒रुग॑व्यूती अभय॒ङ्कृतौ॥२०॥

%3.5.10.2
वृ॒ष्टिद्या॑वा री॒त्या॑पा। श॒म्भुवौ॑ मयो॒भुवौ। ऊर्ज॑स्वती च॒ पय॑स्वती च। सू॒प॒च॒र॒णा च॑ स्वधिचर॒णा च॑। तयो॑रा॒विदि॑। अ॒ग्निरि॒द ह॒विर॑जुषत। अवी॑वृधत॒ महो॒ ज्यायो॑ऽकृत। सोम॑ इ॒दह॒विर॑जुषत। अवी॑वृधत॒ महो॒ ज्यायो॑ऽकृत। अ॒ग्निरि॒द ह॒विर॑जुषत॥२१॥

%3.5.10.3
अवी॑वृधत॒ महो॒ ज्यायो॑ऽकृत। प्र॒जाप॑तिरि॒द ह॒विर॑जुषत। अवी॑वृधत॒ महो॒ ज्यायो॑ऽकृत। अ॒ग्नीषोमा॑वि॒द ह॒विर॑जुषेताम्। अवी॑वृधेतां॒ महो॒ ज्यायोऽक्राताम्। इ॒न्द्रा॒ग्नी इ॒द ह॒विर॑जुषेताम्। अवी॑वृधेतां॒ महो॒ ज्यायोऽक्राताम्। इन्द्र॑ इ॒द ह॒विर॑जुषत। अवी॑वृधत॒ महो॒ ज्यायो॑ऽकृत। म॒हे॒न्द्र इ॒द ह॒विर॑जुषत॥२२॥

%3.5.10.4
अवी॑वृधत॒ महो॒ ज्यायो॑ऽकृत। दे॒वा आज्य॒पा आज्य॑मजुषन्त। अवी॑वृधन्त॒ महो॒ ज्यायोऽक्रत। अ॒ग्निर्‌हो॒त्रेणे॒द ह॒विर॑जुषत। अवी॑वृधत॒ महो॒ ज्यायो॑ऽकृत। अ॒स्यामृध॒द्धोत्रा॑यान्देवङ्ग॒मायाम्। आशास्ते॒ऽयं यज॑मानो॒ऽसौ। आयु॒रा शास्ते। सु॒प्र॒जा॒स्त्वमा शास्ते। स॒जा॒त॒व॒न॒स्यामा शास्ते॥२३॥

%3.5.10.5
उत्त॑रान्देवय॒ज्यामा शास्ते। भूयो॑ हवि॒ष्कर॑ण॒मा शास्ते। दि॒व्यन्धामा शास्ते। विश्वं॑ प्रि॒यमा शास्ते। यद॒नेन॑ ह॒विषाऽऽशास्ते। तद॑श्या॒त्तदृ॑ध्यात्। तद॑स्मै दे॒वा रा॑सन्ताम्। तद॒ग्निर्दे॒वो दे॒वेभ्यो॒ वन॑ते। व॒यम॒ग्नेर्मानु॑षाः। इ॒ष्टं च॑ वी॒तं च॑। उ॒भे च॑ नो॒ द्यावा॑पृथि॒वी अह॑सस्पाताम्। इ॒ह गति॑र्वा॒मस्ये॒दं च॑। नमो॑ दे॒वेभ्य॑॥२४॥\anuvakamend[अ॒भ॒य॒ङ्कृता॑वकृता॒ग्निरि॒द ह॒विर॑जुषत महे॒न्द्र इ॒द ह॒विर॑जुषत सजातवन॒स्यामा शास्ते वी॒तं च॒ त्रीणि॑ च]

%3.5.11.1
तच्छ॒य्योँरा वृ॑णीमहे। गा॒तुं य॒ज्ञाय॑। गा॒तुं य॒ज्ञप॑तये। दैवी स्व॒स्तिर॑स्तु नः। स्व॒स्तिर्मानु॑षेभ्यः। ऊ॒र्ध्वञ्जि॑गातु भेष॒जम्। शन्नो॑ अस्तु द्वि॒पदे। शञ्चतु॑ष्पदे॥२५॥\anuvakamend[तच्छ॒य्योँर॒ष्टौ]

%3.5.12.1
आप्या॑यस्व॒ सन्ते। इ॒ह त्वष्टा॑रमग्रि॒यन्तन्न॑स्तु॒रीपम्। दे॒वानां॒ पत्नी॑रुश॒तीर॑वन्तु नः। प्राव॑न्तु नस्तु॒जये॒ वाज॑सातये। याः पार्थि॑वासो॒ या अ॒पामपि॑ व्र॒ते। ता नो॑ देवीः सुहवा॒ शर्म॑ यच्छत। उ॒त ग्ना वि॑यन्तु दे॒वप॑त्नीः। इ॒न्द्रा॒ण्य॑ग्नाय्य॒श्विनी॒ राट्। आ रोद॑सी वरुणा॒नी शृ॑णोतु। वि॒यन्तु॑ दे॒वीर्य ऋ॒तुर्जनी॑नाम्॥२६॥

%3.5.12.2
अ॒ग्निर्‌होता॑ गृ॒हप॑ति॒ स राजा। विश्वा॑ वेद॒ जनि॑मा जा॒तवे॑दाः। दे॒वाना॑मु॒त यो मर्त्या॑नाम्। यजि॑ष्ठ॒ स प्र य॑जतामृ॒तावा। व॒यमु॑ त्वा गृहपते॒ जना॑नाम्। अग्ने॒ अक॑र्म स॒मिधा॑ बृ॒हन्तम्। अ॒स्थू॒रि णो॒ गार्‌ह॑पत्यानि सन्तु। ति॒ग्मेन॑ न॒स्तेज॑सा॒ सशि॑शाधि॥२७॥\anuvakamend[जनी॑नाम॒ष्टौ च॑]

%3.5.13.1
उप॑हूत रथन्त॒र स॒ह पृ॑थि॒व्या। उप॑ मा रथन्त॒र स॒ह पृ॑थि॒व्या ह्व॑यताम्। उप॑हूतं वामदे॒व्य स॒हान्तरि॑क्षेण। उप॑ मा वामदे॒व्य स॒हान्तरि॑क्षेण ह्वयताम्। उप॑हूतं बृ॒हत्स॒ह दि॒वा। उप॑ मा बृ॒हत्स॒ह दि॒वा ह्व॑यताम्। उप॑हूताः स॒प्त होत्रा। उप॑ मा स॒प्त होत्रा ह्वयन्ताम्। उप॑हूता धे॒नुः स॒हर्ष॑भा। उप॑ मा धे॒नुः स॒हर्‌ष॑भा ह्वयताम्॥२८॥

%3.5.13.2
उप॑हूतो भ॒क्षः सखा। उप॑ मा भ॒क्षः सखा ह्वयताम्। उप॑हू॒ताँ(४)हो। इडोप॑हूता। उप॑हू॒तेडा। उपो॑ अ॒स्मा इडा ह्वयताम्। इडोप॑हूता। उप॑हू॒तेडा। मा॒न॒वी घृ॒तप॑दी मैत्रावरु॒णी। ब्रह्म॑ दे॒वकृ॑त॒मुप॑हूतम्॥२९॥

%3.5.13.3
दैव्या॑ अध्व॒र्यव॒ उप॑हूताः। उप॑हूता मनु॒ष्या। य इ॒मं य॒ज्ञमवान्॑। ये य॒ज्ञप॑त्नीं॒ वर्धान्॑। उप॑हूते॒ द्यावा॑पृथि॒वी। पू॒र्व॒जे ऋ॒ताव॑री। दे॒वी दे॒वपु॑त्रे। उप॑हूते॒यं यज॑माना। इ॒न्द्राणीवा॑ऽविध॒वा। अदि॑तिरिव सुपु॒त्रा। उत्त॑रस्यान्देवय॒ज्याया॒मुप॑हूता। भूय॑सि हवि॒ष्कर॑ण॒ उप॑हूता। दि॒व्ये धाम॒न्नुप॑हूता। इ॒दं मे॑ दे॒वा ह॒विर्जु॑षन्ता॒मिति॒ तस्मि॒न्नुप॑हूता। विश्व॑मस्याः प्रि॒यमुप॑हूतम्। विश्व॑स्य प्रि॒यस्योप॑हूत॒स्योप॑हूता॥३०॥\anuvakamend[स॒हर्\mbox{}ष॑भा ह्वयता॒मुप॑हूत सुपु॒त्रा षट्च॑]




\prashnaend{स॒त्यं प्रवोऽग्ने॑ म॒हान॒ग्निर्\mbox{}होता॑ स॒मिधो॒ऽग्निर्वृ॒त्राण्य॒ग्निर्मू॒र्धोप॑हूतन्दे॒वं ब॒र्॒हिरि॒दन्द्या॑वापृथिवी॒ तच्छ॒य्योँरा प्या॑य॒स्वोप॑हूत॒न्त्रयो॑दश॥१३॥}{स॒त्यं व॒य स्या॑म वृ॒ष्टिद्या॑वा त्रि॒शत्॥३०॥}{स॒त्यमुप॑हूता॥}{हरि॑ ओम्॥}{इति श्रीकृष्णयजुर्वेदीयतैत्तिरीयब्राह्मणे तृतीयाष्टके पञ्चमः प्रपाठकः समाप्तः॥}
\clearpage
\sect{षष्ठमः प्रश्नः}
\setcounter{anuvakam}{0}
\dnsub{तैत्तिरीयब्राह्मणे तृतीयाष्टके षष्ठः प्रपाठकः}

%3.6.1.1
अ॒ञ्जन्ति॒ त्वाम॑ध्व॒रे दे॑व॒यन्त॑। वन॑स्पते॒ मधु॑ना॒ दैव्ये॑न। यदू॒र्ध्वस्ति॑ष्ठा॒द्द्रवि॑णे॒ह ध॑त्तात्। यद्वा॒ क्षयो॑ मा॒तुर॒स्या उ॒पस्थे। उच्छ्र॑यस्व वनस्पते। वर्ष्म॑न्पृथि॒व्या अधि॑। सुमि॑ती मी॒यमा॑नः। वर्चो॑धा य॒ज्ञवा॑हसे। समि॑द्धस्य॒ श्रय॑माणः पु॒रस्तात्। ब्रह्म॑ वन्वा॒नो अ॒जर सु॒वीरम्॥१॥

%3.6.1.2
आ॒रे अ॒स्मदम॑तिं॒ बाध॑मानः। उच्छ्र॑यस्व मह॒ते सौभ॑गाय। ऊ॒र्ध्व ऊ॒षुण॑ ऊ॒तये। तिष्ठा॑ दे॒वो न स॑वि॒ता। ऊ॒र्ध्वो वाज॑स्य॒ सनि॑ता॒ यद॒ञ्जिभि॑। वा॒घद्भि॑र्वि॒ह्वया॑महे। ऊ॒र्ध्वो न॑ पा॒ह्यह॑सो॒ नि के॒तुना। विश्व॒ सम॒त्रिण॑न्दह। कृ॒धी न॑ ऊ॒र्ध्वां च॒ रथा॑य जी॒वसे। वि॒दा दे॒वेषु॑ नो॒ दुव॑॥२॥

%3.6.1.3
जा॒तो जा॑यते सुदिन॒त्वे अह्नाम्। सम॒र्य आ वि॒दथे॒ वर्ध॑मानः। पु॒नन्ति॒ धीरा॑ अ॒पसो॑ मनी॒षा। दे॒व॒या विप्र॒ उदि॑यर्ति॒ वाचम्। युवा॑ सु॒वासा॒ परि॑वीत॒ आगात्। स उ॒ श्रेयान्भवति॒ जाय॑मानः। तन्धीरा॑सः क॒वय॒ उन्न॑यन्ति। स्वा॒धियो॒ मन॑सा देव॒यन्त॑। पृ॒थु॒पाजा॒ अम॑र्त्यः। घृ॒तनि॑र्णि॒ख्स्वा॑हुतः। अ॒ग्निर्य॒ज्ञस्य॑ हव्य॒वाट्। त स॒बाधो॑ य॒तः स्रु॑चः। इ॒त्था धि॒या य॒ज्ञव॑न्तः। आच॑क्रुर॒ग्निमू॒तये। त्वं वरु॑ण उ॒त मि॒त्रो अ॑ग्ने। त्वां व॑र्धन्ति म॒तिभि॒र्वसि॑ष्ठाः। त्वे वसु॑ सुषण॒नानि॑ सन्तु। यू॒यं पा॑त स्व॒स्तिभि॒ सदा॑ नः॥३॥\anuvakamend[सु॒वीर॒न्दुव॒ स्वा॑हुतो॒ऽष्टौ च॑]

%3.6.2.1
होता॑ यक्षद॒ग्नि स॒मिधा॑ सुष॒मिधा॒ समि॑द्धं॒ नाभा॑ पृथि॒व्याः स॑ङ्ग॒थे वा॒मस्य॑। वर्ष्म॑न्दि॒व इ॒डस्प॒दे वेत्वाज्य॑स्य॒ होत॒र्यज॑। होता॑ यक्ष॒त्तनू॒नपा॑त॒मदि॑ते॒र्गर्भं॒ भुव॑नस्य गो॒पाम्। मध्वा॒द्य दे॒वो दे॒वेभ्यो॑ देव॒यानान्प॒थो अ॑नक्तु॒ वेत्वाज्य॑स्य॒ होत॒र्यज॑। होता॑ यक्ष॒न्नरा॒शसं॑ नृश॒स्त्रं नॄः प्र॑णेत्रम्। गोभि॑र्व॒पावा॒न्त्स्याद्वी॒रैः शक्ती॑वा॒न्रथै प्रथम॒या वा॒ हिर॑ण्यैश्च॒न्द्री वेत्वाज्य॑स्य॒ होत॒र्यज॑। होता॑ यक्षद॒ग्निमि॒ड ई॑डि॒तो दे॒वो दे॒वा आव॑क्षद्दू॒तो ह॑व्य॒वाडमू॑रः। उपे॒मं य॒ज्ञमुपे॒मां दे॒वो दे॒वहू॑तिमवतु॒ वेत्वाज्य॑स्य॒ होत॒र्यज॑। होता॑ यक्षद्ब॒र्‌हिः सु॒ष्टरी॒मोर्णं॑म्रदा अ॒स्मिन् य॒ज्ञे वि च॒ प्र च॑ प्रथता स्वास॒स्थं दे॒वेभ्य॑। एमे॑नद॒द्य वस॑वो रु॒द्रा आ॑दि॒त्याः स॑दन्तु प्रि॒यमिन्द्र॑स्यास्तु॒ वेत्याज्य॑स्य॒ होत॒र्यज॑॥४॥

%3.6.2.2
होता॑ यक्ष॒द्दुर॑ ऋ॒ष्वाः क॑व॒ष्यो को॑षधावनी॒रुदाता॑भी॒र्जिह॑तां॒ विपक्षो॑भिः श्रयन्ताम्। सु॒प्रा॒य॒णा अ॒स्मिन् य॒ज्ञे विश्र॑यन्तामृता॒वृधो॑ वि॒यन्त्वाज्य॑स्य॒ होत॒र्यज॑। होता॑ यक्षदु॒षासा॒नक्ता॑ बृह॒ती सु॒पेश॑सा॒ नॄः पति॑भ्यो॒ योनि॑ङ्कृण्वा॒ने। स॒स्मय॑माने॒ इन्द्रे॑ण दे॒वैरेदं ब॒र्॒हिः सी॑दतां वी॒तामाज्य॑स्य॒ होत॒र्यज॑। होता॑ यक्ष॒द्दैव्या॒ होता॑रा म॒न्द्रा पोता॑रा क॒वी प्रचे॑तसा। स्वि॑ष्टम॒द्यान्यः क॑रदि॒षा स्व॑भिगूर्तम॒न्य ऊ॒र्जा सत॑वसे॒मं य॒ज्ञं दि॒वि दे॒वेषु॑ धत्तां वी॒तामाज्य॑स्य॒ होत॒र्यज॑। होता॑ यक्षत्ति॒स्रो दे॒वीर॒पसा॑म॒पस्त॑मा॒ अच्छि॑द्रम॒द्येदमप॑स्तन्वताम्। दे॒वेभ्यो॑ दे॒वीर्दे॒वमपो॑ वि॒यन्त्वाज्य॑स्य॒ होत॒र्यज॑। होता॑ यक्ष॒त्त्वष्टा॑र॒मचि॑ष्टु॒मपा॑क रेतो॒धां विश्र॑वसं यशो॒धाम्। पु॒रु॒रूप॒मका॑मकर्‌शन सु॒पोष॒ पोषै॒ स्यात्सु॒वीरो॑ वी॒रैर्वेत्वाज्य॑स्य॒ होत॒र्यज॑। होता॑ यक्ष॒द्वन॒स्पति॑मु॒पाव॑स्रक्षद्धि॒यो जो॒ष्टार श॒शम॒न्नर॑। स्वदा॒त्स्वधि॑तिर्\mbox{}ऋतु॒थाद्य दे॒वो दे॒वेभ्यो॑ ह॒व्यावा॒ड्वेत्वाज्य॑स्य॒ होत॒र्यज॑। होता॑ यक्षद॒ग्नि स्वाहाऽऽज्य॑स्य॒ स्वाहा॒ मेद॑स॒ स्वाहा स्तो॒काना॒ स्वाहा॒ स्वाहा॑कृतीना॒ स्वाहा॑ ह॒व्यसूक्तीनाम्। स्वाहा॑ दे॒वा आज्य॒पान्त्स्वाहा॒ऽग्नि हो॒त्राज्जु॑षा॒णा अग्न॒ आज्य॑स्य वियन्तु॒ होत॒र्यज॑॥५॥\anuvakamend[प्रि॒यमिन्द्र॑स्यास्तु॒ वेत्वाज्य॑स्य॒ होत॒र्यज॑ सु॒वीरो॑ वी॒रैर्वेत्वाज्य॑स्य॒ होत॒र्यज॑ च॒त्वारि॑ च (अ॒ग्निन्तनू॒नपा॑त॒न्नरा॒शस॑म॒ग्निमि॒ड ई॑डि॒तो ब॒र्‌हिर्दुर॑ उ॒षासा॒नक्ता॒ दैव्या॑ ति॒स्रस्त्वष्टा॑रं॒ वन॒स्पति॑म॒ग्निम्। पञ्च॒ वेत्वेको॑ वि॒यन्तु॒ द्विर्वी॒तामेको॑ वि॒यन्तु॒ द्विर्वेत्वेको॑ वियन्तु॒ होत॒र्यज॑ ॥ )]

%3.6.3.1
समि॑द्धो अ॒द्य मनु॑षो दुरो॒णे। दे॒वो दे॒वान् य॑जसि जातवेदः। आ च॒ वह॑ मित्रमहश्चिकि॒त्वान्। त्वन्दू॒तः क॒विर॑सि॒ प्रचे॑ताः। तनू॑नपात्प॒थ ऋ॒तस्य॒ यानान्॑। मध्वा॑ सम॒ञ्जन्त्स्व॑दया सुजिह्व। मन्मा॑नि धी॒भिरु॒त य॒ज्ञमृ॒न्धन्। दे॒व॒त्रा च॑ कृणुह्यध्व॒रन्न॑। नरा॒शस॑स्य महि॒मान॑मेषाम्। उप॑ स्तोषाम यज॒तस्य॑ य॒ज्ञैः॥६॥

%3.6.3.2
ते सु॒क्रत॑व॒ शुच॑यो धिय॒न्धाः। स्वद॑न्तु दे॒वा उ॒भया॑नि ह॒व्या। आ॒जुह्वा॑न॒ ईड्यो॒ वन्द्य॑श्च। आयाह्यग्ने॒ वसु॑भिः स॒जोषा। त्वं दे॒वाना॑मसि यह्व॒ होता। स ए॑नान् यक्षीषि॒तो यजी॑यान्। प्रा॒चीनं॑ ब॒र्‌हिः प्र॒दिशा॑ पृथि॒व्याः। वस्तो॑र॒स्या वृ॑ज्यते॒ अग्रे॒ अह्नाम्। व्यु॑ प्रथते वित॒रं वरी॑यः। दे॒वेभ्यो॒ अदि॑तये स्यो॒नम्॥७॥

%3.6.3.3
व्यच॑स्वतीरुर्वि॒या विश्र॑यन्ताम्। पति॑भ्यो॒ न जन॑य॒ शुम्भ॑मानाः। देवीर्द्वारो बृहतीर्विश्वमिन्वाः। दे॒वेभ्यो॑ भवथ सुप्राय॒णाः। आसु॒ष्वय॑न्ती यज॒ते उपा॑के। उ॒षासा॒नक्ता॑ सदतां॒ नि योनौ। दि॒व्ये योष॑णे बृह॒ती सु॑रु॒क्मे। अधि॒ श्रिय शुक्र॒पिश॒न्दधा॑ने। दैव्या॒ होता॑रा प्रथ॒मा सु॒वाचा। मिमा॑ना य॒ज्ञं मनु॑षो॒ यज॑ध्यै॥८॥

%3.6.3.4
प्र॒चो॒दय॑न्ता वि॒दथे॑षु का॒रू। प्रा॒चीनं॒ ज्योति॑ प्र॒दिशा॑ दि॒शन्ता। आ नो॑ य॒ज्ञं भार॑ती॒ तूय॑मेतु। इडा॑ मनु॒ष्वदि॒ह चे॒तय॑न्ती। ति॒स्रो दे॒वीर्ब॒र्‌हिरेद स्यो॒नम्। सर॑स्वती॒ स्वप॑सः सदन्तु। य इ॒मे द्यावा॑पृथि॒वी जनि॑त्री। रू॒पैरपिश॒द्भुव॑नानि॒ विश्वा। तम॒द्य हो॑तरिषि॒तो यजी॑यान्। दे॒वन्त्वष्टा॑रमि॒ह य॑क्षि वि॒द्वान्॥९॥

%3.6.3.5
उ॒पाव॑सृज॒त्मन्या॑ सम॒ञ्जन्। दे॒वानां॒ पाथ॑ ऋतु॒था ह॒वीषि॑। वन॒स्पति॑ शमि॒ता दे॒वो अ॒ग्निः। स्वद॑न्तु ह॒व्यं मधु॑ना घृ॒तेन॑। स॒द्यो जा॒तो व्य॑मिमीत य॒ज्ञम्। अ॒ग्निर्दे॒वाना॑मभवत्पुरो॒गाः। अ॒स्य होतु॑ प्र॒दिश्यृ॒तस्य॑ वा॒चि। स्वाहा॑कृत ह॒विर॑दन्तु दे॒वाः॥१०॥\anuvakamend[य॒ज्ञैः स्यो॒नं यज॑ध्यै वि॒द्वान॒ष्टौ च॑]

%3.6.4.1
अ॒ग्निर्‌होता॑ नो अध्व॒रे। वा॒जी सन्परि॑णीयते। दे॒वो दे॒वेषु॑ य॒ज्ञिय॑। परि॑त्रिवि॒ष्ट्य॑ध्व॒रम्। यात्य॒ग्नी र॒थीरि॑व। आ दे॒वेषु॒ प्रयो॒ दध॑त्। परि॒ वाज॑पतिः क॒विः। अ॒ग्निर्‌ह॒व्यान्य॑क्रमीत्। दध॒द्रत्ना॑नि दा॒शुषे॥११॥\anuvakamend[अ॒ग्निर्‌होता॑ नो॒ नव॑]

%3.6.5.1
अजै॑द॒ग्निः। अस॑न॒द्वाज॒न्नि। दे॒वो दे॒वेभ्यो॑ ह॒व्यावाट्। प्राञ्जो॑भिर्‌हिन्वा॒नः। धेना॑भि॒ कल्प॑मानः। य॒ज्ञस्यायु॑ प्रति॒रन्। उप॒ प्रेष्य॑ होतः। ह॒व्या दे॒वेभ्य॑॥१२॥\anuvakamend[अजै॑द॒ष्टौ]

%3.6.6.1
दैव्या शमितार उ॒त म॑नुष्या॒ आर॑भध्वम्। उप॑नयत॒ मेध्या॒ दुर॑। आ॒शासा॑ना॒ मेध॑पतिभ्यां॒ मेधम्। प्रास्मा॑ अ॒ग्निं भ॑रत। स्तृ॒णी॒त ब॒र्॒हिः। अन्वे॑नं मा॒ता म॑न्यताम्। अनु॑ पि॒ता। अनु॒ भ्राता॒ सग॑र्भ्यः। अनु॒ सखा॒ सयूथ्यः। उ॒दी॒चीना अस्य प॒दो निध॑त्तात्॥१३॥

%3.6.6.2
सूर्य॒ञ्चक्षु॑र्गमयतात्। वातं॑ प्रा॒णम॒न्वव॑सृजतात्। दिश॒ श्रोत्रम्। अ॒न्तरि॑क्ष॒मसुम्। पृ॒थि॒वी शरी॑रम्। ए॒क॒धाऽस्य॒ त्वच॒माच्छ्य॑तात्। पु॒रा नाभ्या॑ अपि॒शसो॑ व॒पामुत्खि॑दतात्। अ॒न्तरे॒वोष्माणं॑ वारयतात्। श्ये॒नम॑स्य॒ वक्ष॑ कृणुतात्। प्र॒शसा॑ बा॒हू॥१४॥

%3.6.6.3
श॒ला दो॒षणी। क॒श्यपे॒वासा। अच्छि॑द्रे॒ श्रोणी। क॒वषो॒रू स्रे॒कप॑र्णाष्ठी॒वन्ता। षड्विशतिरस्य॒ वङ्क्र॑यः। ता अ॑नु॒ष्ठ्योच्च्या॑वयतात्। गात्र॑ङ्गात्रम॒स्यानू॑नङ्कृणुतात्। ऊ॒व॒ध्य॒गो॒हं पार्थि॑वङ्खनतात्। अ॒स्ना रक्ष॒ ससृ॑जतात्। व॒नि॒ष्ठुम॑स्य॒ मा रा॑विष्ट॥१५॥

%3.6.6.4
उरू॑कं॒ मन्य॑मानाः। नेद्व॑स्तो॒के तन॑ये। रवि॑ता॒रव॑च्छमितारः। अध्रि॑गो शमी॒ध्वम्। सु॒शमि॑ शमीध्वम्। श॒मी॒ध्वम॑ध्रिगो। अध्रि॑गु॒श्चापा॑पश्च। उ॒भौ दे॒वाना शमि॒तारौ। तावि॒मं प॒शु श्र॑पयतां प्रवि॒द्वासौ। यथा॑यथाऽस्य॒ श्रप॑ण॒न्तथा॑तथा॥१६॥\anuvakamend[ध॒त्ता॒द्बा॒हू मा रा॑विष्ट॒ तथा॑तथा]

%3.6.7.1
जु॒षस्व॑ स॒प्रथ॑स्तमम्। वचो॑ दे॒वप्स॑रस्तमम्। ह॒व्या जुह्वा॑न आ॒सनि॑। इ॒मन्नो॑ य॒ज्ञम॒मृते॑षु धेहि। इ॒मा ह॒व्या जा॑तवेदो जुषस्व। स्तो॒काना॑मग्ने॒ मेद॑सो घृ॒तस्य॑। होत॒ प्राशा॑न प्रथ॒मो नि॒षद्य॑। घृ॒तव॑न्तः पावक ते। स्तो॒काः श्चो॑तन्ति॒ मेद॑सः। स्वध॑र्मन्दे॒ववी॑तये॥१७॥

%3.6.7.2
श्रेष्ठ॑न्नो धेहि॒ वार्यम्। तुभ्य स्तो॒का घृ॑त॒श्चुत॑। अग्ने॒ विप्रा॑य सन्त्य। ऋषि॒ श्रेष्ठ॒ समि॑ध्यसे। य॒ज्ञस्य॑ प्रावि॒ता भ॑व। तुभ्य श्चोतन्त्यध्रिगो शचीवः। स्तो॒कासो॑ अग्ने॒ मेद॑सो घृ॒तस्य॑। क॒वि॒श॒स्तो बृ॑ह॒ता भा॒नुनागा। ह॒व्या जु॑षस्व मेधिर। ओजि॑ष्ठन्ते मध्य॒तो मेद॒ उद्भृ॑तम्। प्र ते॑ व॒यन्द॑दामहे। श्चोत॑न्ति ते वसो स्तो॒का अधि॑त्व॒चि। प्रति॒ तान्दे॑व॒शोवि॑हि॥१८॥\anuvakamend[दे॒ववी॑तय॒ उद्भृ॑त॒न्त्रीणि॑ च]

%3.6.8.1
आवृ॑त्रहणा वृत्र॒हभि॒ शुष्मै। इन्द्र॑ या॒तन्नमो॑भिरग्ने अ॒र्वाक्। यु॒व राधो॑भि॒रक॑वेभिरिन्द्र। अग्ने॑ अ॒स्मे भ॑वतमुत्त॒मेभि॑। होता॑ यक्षदिन्द्रा॒ग्नी। छाग॑स्य व॒पाया॒ मेद॑सः। जु॒षेता ह॒विः। होत॒र्यज॑। विह्यख्य॒न्मन॑सा॒ वस्य॑ इ॒च्छन्। इन्द्राग्नी ज्ञा॒स उ॒त वा॑ सजा॒तान्॥१९॥

%3.6.8.2
नान्या यु॒वत्प्रम॑तिरस्ति॒ मह्यम्। स वा॒न्धियं॑ वाज॒यन्ती॑मतक्षम्। होता॑ यक्षदिन्द्रा॒ग्नी। पु॒रो॒डाश॑स्य जु॒षेता ह॒विः। होत॒र्यज॑। त्वामी॑डते अजि॒रन्दू॒त्या॑य। ह॒विष्म॑न्त॒ सद॒मिन्मानु॑षासः। यस्य॑ दे॒वैरास॑दो ब॒र्‌हि॒र॑ग्ने। अहान्यस्मै सु॒दिना॑ भवन्तु। होता॑ यक्षद॒ग्निम्। पु॒रो॒डाश॑स्य जु॒षता ह॒विः। होत॒र्यज॑॥२०॥\anuvakamend[स॒जा॒तान॒ग्निन्द्वे च॑]

%3.6.9.1
गी॒र्भिर्विप्र॒ प्रम॑तिमि॒च्छमा॑नः। ईट्टे॑ र॒यिं य॒शसं॑ पूर्व॒भाजम्। इन्द्राग्नी वृत्रहणा सुवज्रा। प्र णो॒ नव्ये॑भिस्तिरतन्दे॒ष्णैः। माच्छेद्म र॒श्मीरिति॒ नाध॑मानाः। पि॒तृ॒णा शक्ती॑रनु॒यच्छ॑मानाः। इ॒न्द्रा॒ग्निभ्या॒ङ्कं वृष॑णो मदन्ति। ताह्यद्री॑ धि॒षणा॑या उ॒पस्थे। अ॒ग्नि सु॑दी॒ति सु॒दृशं॑ गृ॒णन्त॑। न॒म॒स्याम॒स्त्वेड्यं॑ जातवेदः। त्वान्दू॒तम॑र॒ति ह॑व्य॒वाहम्। दे॒वा अ॑कृण्वन्न॒मृत॑स्य॒ नाभिम्॥२१॥\anuvakamend[जा॒त॒वे॒दो॒ द्वे च॑]

%3.6.10.1
त्व ह्य॑ग्ने प्रथ॒मो म॒नोता। अ॒स्या धि॒यो अभ॑वो दस्म॒होता। त्व सीव्वृँषन्नकृणोर्दु॒ष्टरी॑तु। सहो॒ विश्व॑स्मै॒ सह॑से॒ सह॑ध्यै। अधा॒ होता॒ न्य॑सीदो॒ यजी॑यान्। इ॒डस्प॒द इ॒षय॒न्नीड्य॒ सन्। तन्त्वा॒ नर॑ प्रथ॒मन्दे॑व॒यन्त॑। म॒हो रा॒ये चि॒तय॑न्तो॒ अनु॑ग्मन्। वृ॒तेव॒ यन्तं॑ ब॒हुभि॑र्वस॒व्यै। त्वे र॒यिञ्जा॑गृ॒वासो॒ अनु॑ग्मन्॥२२॥

%3.6.10.2
रुश॑न्तम॒ग्निन्द॑र्‌श॒तम्बृ॒हन्तम्। व॒पाव॑न्तं वि॒श्वहा॑ दीदि॒वासम्। प॒दन्दे॒वस्य॒ नम॑सा वि॒यन्त॑। श्र॒व॒स्यव॒ श्रव॑ आप॒न्नमृ॑क्तम्। नामा॑नि चिद्दधिरे य॒ज्ञिया॑नि। भ॒द्रायान्ते रणयन्त॒ सन्दृ॑ष्टौ। त्वां व॑र्धन्ति क्षि॒तय॑ पृथि॒व्याम्। त्व राय॑ उ॒भया॑सो॒ जना॑नाम्। त्वन्त्रा॒ता त॑रणे॒ चेत्यो॑भूः। पि॒ता मा॒ता सद॒मिन्मानु॑षाणाम्॥२३॥

%3.6.10.3
सप॒र्येण्य॒ स प्रि॒यो वि॒क्ष्व॑ग्निः। होता॑ म॒न्द्रो निष॑सादा॒ यजी॑यान्। तन्त्वा॑ व॒यन्दम॒ आ दी॑दि॒वासम्। उप॑ज्ञु॒बाधो॒ नम॑सा सदेम। तन्त्वा॑ व॒य सु॒धियो॒ नव्य॑मग्ने। सु॒म्ना॒यव॑ ईमहे देव॒यन्त॑। त्वं विशो॑ अनयो॒ दीद्या॑नः। दि॒वो अ॑ग्ने बृह॒ता रो॑च॒नेन॑। वि॒शां क॒विं वि॒श्पति॒ शश्व॑तीनाम्। नि॒तोश॑नं वृष॒भं च॑र्‌षणी॒नाम्॥२४॥

%3.6.10.4
प्रेती॑षणि मि॒षय॑न्तं पाव॒कम्। राज॑न्तम॒ग्निं य॑ज॒त र॑यी॒णाम्। सो अ॑ग्न ईजे शश॒मे च॒ मर्त॑। यस्त॒ आन॑ट्त्स॒मिधा॑ ह॒व्यदा॑तिम्। य आहु॑तिं॒ परि॒ वेदा॒ नमो॑भिः। विश्वेत्सवा॒मा द॑धते॒ त्वोत॑। अ॒स्मा उ॑ ते॒ महि॑ म॒हे वि॑धेम। नमो॑भिरग्ने स॒मिधो॒त ह॒व्यैः। वेदी॑सूनो सहसो गी॒र्भिरु॒क्थैः। आ ते भ॒द्राया सुम॒तौ य॑तेम॥२५॥

%3.6.10.5
आ॒ यस्त॒तन्थ॒ रोद॑सी॒ विभा॒सा। श्रवो॑भिश्च श्रव॒स्य॑स्तरु॑त्रः। बृ॒हद्भि॒र्वाजै॒ स्थवि॑रेभिर॒स्मे। रे॒वद्भि॑रग्ने वित॒रं वि भा॑हि। नृ॒वद्व॑सो॒ सद॒मिद्धेह्य॒स्मे। भूरि॑तो॒काय॒ तन॑याय प॒श्वः। पू॒र्वीरिषो॑ बृह॒तीरा॒रे अ॑घाः। अ॒स्मे भ॒द्रा सौश्रव॒सानि॑ सन्तु। पु॒रूण्य॑ग्ने पुरु॒धा त्वा॒या। वसू॑नि राजन्व॒सुता॑ते अश्याम्। पु॒रूणि॒ हि त्वे पु॑रुवार॒ सन्ति॑। अग्ने॒ वसु॑ विध॒ते राज॑नि॒त्वे॥२६॥\anuvakamend[जा॒गृ॒वासो॒ अनु॑ग्म॒न्मानु॑षाणाञ्चर्‌षणी॒नां य॑तेमाश्या॒न्द्वे च॑]

%3.6.11.1
आभ॑रत शिक्षतं वज्रबाहू। अ॒स्मा इ॑न्द्राग्नी अवत॒ शची॑भिः। इ॒मे नु ते र॒श्मय॒ सूर्य॑स्य। येभि॑ सपि॒त्वं पि॒तरो॑ न॒ आय\sn{}। होता॑ यक्षदिन्द्रा॒ग्नी। छाग॑स्य ह॒विष॒ आत्ता॑म॒द्य। म॒ध्य॒तो मेद॒ उद्भृ॑तम्। पु॒रा द्वेषोभ्यः। पु॒रा पौरु॑षेय्या गृ॒भः। घस्तान्नू॒नम्॥२७॥

%3.6.11.2
घा॒से अ॑ज्राणां॒ यव॑सप्रथमानाम्। सु॒मत्क्ष॑राणा श॒तरु॑द्रियाणाम्। अ॒ग्नि॒ष्वा॒त्तानां॒ पीवो॑पवसनानाम्। पा॒र्श्व॒तः श्रो॑णि॒तः शि॑ताम॒त उ॑त्साद॒तः। अङ्गा॑दङ्गा॒दव॑त्तानाम्। कर॑त ए॒वेन्द्रा॒ग्नी। जु॒षेता ह॒विः। होत॒र्यज॑। दे॒वेभ्यो॑ वनस्पते ह॒वीषि॑। हिर॑ण्यपर्ण प्र॒दिव॑स्ते॒ अर्थम्।॥२८॥

%3.6.11.3
प्र॒द॒क्षि॒णिद्र॑श॒नया॑ नि॒यूय॑। ऋ॒तस्य॑ वक्षि प॒थिभी॒ रजि॑ष्ठैः। होता॑ यक्षद्वन॒स्पति॑म॒भिहि। पि॒ष्टत॑मया॒ रभि॑ष्ठया रश॒नयाधि॑त। यत्रेन्द्राग्नि॒योश्छाग॑स्य ह॒विष॑ प्रि॒या धामा॑नि। यत्र॒ वन॒स्पते प्रि॒या पाथासि। यत्र॑ दे॒वाना॑माज्य॒पानां प्रि॒या धामा॑नि। यत्रा॒ग्नेर्‌होतु॑ प्रि॒या धामा॑नि। तत्रै॒तं प्र॒स्तुत्ये॑वोप॒स्तुत्ये॑ वो॒पाव॑स्रक्षत्। रभी॑यासमिव कृ॒त्वी॥२९॥

%3.6.11.4
कर॑दे॒वन्दे॒वो वन॒स्पति॑। जु॒षता ह॒विः। होत॒र्यज॑। पि॒प्री॒हि दे॒वा उ॑श॒तो य॑विष्ठ। वि॒द्वा ऋ॒तूर्\mbox{}ऋ॑तुपते यजे॒ह। ये दैव्या॑ ऋ॒त्विज॒स्तेभि॑रग्ने। त्व होतॄ॑णाम॒स्याय॑जिष्ठः। होता॑ यक्षद॒ग्नि स्वि॑ष्ट॒कृतम्। अया॑ड॒ग्निरि॑न्द्राग्नि॒योश्छाग॑स्य ह॒विष॑ प्रि॒या धामा॑नि। अया॒ड्वन॒स्पते प्रि॒या पाथासि। अयाड्दे॒वाना॑माज्य॒पानां प्रि॒या धामा॑नि। यक्ष॑द॒ग्नेर्‌होतु॑ प्रि॒या धामा॑नि। यक्ष॒त्स्वं म॑हि॒मानम्। आय॑जता॒मेज्या॒ इष॑। कृ॒णोतु॒ सो अ॑ध्व॒रा जा॒तवे॑दाः। जु॒षता ह॒विः। होत॒र्यज॑ ॥३०॥\anuvakamend[नू॒नमर्थं॑ कृ॒त्वी पाथासि स॒प्त च॑]

%3.6.12.1
उपो॑ ह॒ यद्वि॒दथं॑ वा॒जिनो॒ गूः। गी॒र्भिर्विप्रा॒ प्रम॑तिमि॒च्छमा॑नाः। अ॒र्वन्तो॒ न काष्ठा॒न्नक्ष॑माणाः। इ॒न्द्रा॒ग्नी जोहु॑वतो॒ नर॒स्ते। वन॑स्पते रश॒नया॑ऽभि॒धाय॑। पि॒ष्टत॑मया व॒युना॑नि वि॒द्वान्। वह॑ देव॒त्रा दि॑धिषो ह॒वीषि॑। प्र च॑दा॒तार॑म॒मृते॑षु वोचः। अ॒ग्नि स्वि॑ष्ट॒कृतम्। अया॑ड॒ग्निरि॑न्द्राग्नि॒योश्छग॑स्य ह॒विष॑ प्रि॒या धामा॑नि॥३१॥

%3.6.12.2
अया॒ड्वन॒स्पते प्रि॒या पाथासि। अयाड्दे॒वाना॑माज्य॒पानां प्रि॒या धामा॑नि। यक्ष॑द॒ग्नेर्‌होतु॑ प्रि॒या धामा॑नि। यक्ष॒त्स्वं म॑हि॒मानम्। आय॑जता॒मेज्या॒ इष॑। कृ॒णोतु॒ सो अ॑ध्व॒रा जा॒तवे॑दाः। जु॒षता ह॒विः। अग्ने॒ यद॒द्य वि॒शो अ॑ध्वरस्य होतः। पाव॑क शोचे॒ वेष्ट्व हि यज्वा। ऋ॒ता य॑जासि महि॒ना वियद्भूः। ह॒व्या व॑ह यविष्ठ॒ या ते॑ अ॒द्य॥३२॥\anuvakamend[धामा॑नि॒ भूरेकं च]

%3.6.13.1
दे॒वं ब॒र्॒हिः सु॑दे॒वन्दे॒वैः स्यात्सु॒वीरं॑ वी॒रैर्वस्तोर्वृ॒ज्येता॒क्तोः प्रभ्रि॑ये॒तात्य॒न्यान्रा॒या ब॒र्॒हिष्म॑तो मदेम वसु॒वने॑ वसु॒धेय॑स्य वेतु॒ यज॑। दे॒वीर्द्वार॑ सङ्घा॒ते वि॒ड्वीर्याम॑ञ्छिथि॒रा ध्रु॒वा दे॒वहू॑तौ व॒त्स ई॑मेना॒स्तरु॑ण॒ आमि॑मीयात्कुमा॒रो वा॒ नव॑जातो॒ मैना॒ अर्वा॑ रे॒णुक॑काट॒ पृण॑ग्वसु॒वने॑ वसु॒धेय॑स्य वियन्तु यज॑। दे॒वी उ॒षासा॒नक्ताऽद्या॒स्मिन्‌य॒ज्ञे प्र॑य॒त्य॑ह्वेता॒मपि॑ नू॒नन्दैवी॒र्विश॒ प्राया॑सिष्टा॒ सुप्री॑ते॒ सुधि॑ते वसु॒वने॑ वसु॒धेय॑स्य वीतां॒ यज॑। दे॒वी जोष्ट्री॒ वसु॑धिती॒ ययो॑र॒न्याऽघाद्द्वेषासि यू॒यव॒दान्याव॑क्ष॒द्वसु॒ वार्या॑णि॒ यज॑मानाय वसु॒वने॑ वसु॒धेय॑स्य वीतां॒ यज॑। दे॒वी ऊ॒र्जाहु॑ती॒ इष॒मूर्ज॑म॒न्याव॑क्ष॒त्सग्धि॒ सपी॑तिम॒न्या नवे॑न॒ पूर्व॒न्दय॑माना॒ स्याम॑ पुरा॒णेन॒ नव॒न्तामूर्ज॑मू॒र्जाहु॑ती ऊ॒र्जय॑माने अधातां वसु॒वने॑ वसु॒धेय॑स्य वीतां॒ यज॑। दे॒वा दैव्या॒ होता॑रा॒ नेष्टा॑रा॒ पोता॑रा ह॒ताघ॑शसावाभ॒रद्व॑सू वसु॒वने वसु॒धेय॑स्य वीतां॒ यज॑। दे॒वीस्ति॒स्रस्ति॒स्रो दे॒वीरिडा॒ सर॑स्वती॒ भार॑ती॒ द्यां भार॑त्यादि॒त्यैर॑स्पृक्ष॒त्सर॑स्वती॒म रु॒द्रैर्य॒ज्ञमा॑वीदि॒हैवेड॑या॒ वसु॑मत्या सध॒मादं॑ मदेम वसु॒वने॑ वसु॒धेय॑स्य वियन्तु॒ यज॑। दे॒वो नरा॒शस॑स्त्रिशी॒र्॒षा ष॑ड॒क्षः श॒तमिदे॑नशितिपृ॒ष्ठा आद॑धति स॒हस्र॑मीं॒ प्रव॑हन्ति मि॒त्रावरु॒णेद॑स्य हो॒त्रमर्\mbox{}ह॑तो॒ बृह॒स्पति॑ स्तो॒त्रम॒श्विनाऽऽध्व॑र्यवं वसु॒वने॑वसु॒धेयस्य॑ वेतु॒ यज॑। दे॒वो वन॒स्पति॑र्व॒र्॒षप्रा॑वा घृ॒तनि॑र्णि॒ग्द्यामग्रे॒णास्पृ॑क्ष॒दान्तरि॑क्षं॒ मध्ये॑नाप्राः पृथि॒वीमुप॑रेणादृहीद्वसु॒वने॑ वसु॒धेय॑स्य वेतु॒ यज॑। दे॒वं ब॒र्॒हिर्वारि॑तीनान्नि॒धेधा॑ऽसि॒ प्रच्यु॑तीना॒मप्र॑च्युतन्निकाम॒धर॑णं पुरुस्पा॒र्॒हं यश॑स्वदे॒ना ब॒र्॒हिषा॒ऽन्या ब॒र्॒हीष्य॒भि ष्या॑म वसु॒वने॑ वसु॒धेय॑स्य वेतु॒ यज॑। दे॒वो अ॒ग्निः स्वि॑ष्ट॒कृत्सु॒द्रवि॑णा म॒न्द्रः क॒विः स॒त्यम॑न्माऽऽय॒जी होता॒ होतु॑र्॒होतु॒राय॑जीया॒नग्ने॒ यान्दे॒वानया॒ड्या अपि॑प्रे॒र्ये ते॑ हो॒त्रे अम॑त्सत॒ ता स॑स॒नुषी॒ होत्रान्देवङ्ग॒मान्दि॒वि दे॒वेषु॑ य॒ज्ञमेर॑ये॒म स्वि॑ष्ट॒कृच्चाग्ने॒ होताऽभूर्वसु॒वने॑ वसु॒धेय॑स्य नमोवा॒के वीहि॒ यज॑ ॥३३॥\anuvakamend[यजैकं च]

%3.6.14.1
दे॒वं ब॒र्॒हिः। व॒सु॒वने॑ वसु॒धेय॑स्य वेतु। दे॒वीर्द्वार॑। व॒सु॒वने॑ वसु॒धेय॑स्य वियन्तु। दे॒वी उ॒षासा॒नक्ता। व॒सु॒वने॑ वसु॒धेय॑स्य वीताम्। दे॒वी जोष्ट्री। व॒सु॒वने॑ वसु॒धेय॑स्य वीताम्। दे॒वी ऊ॒र्जाहु॑ती। व॒सु॒वने॑ वसु॒धेयस्य॑ वीताम्॥३४॥

%3.6.14.2
दे॒वा दैव्या॒ होता॑रा। व॒सु॒वने॑ वसु॒धेय॑स्य वीताम्। दे॒वीस्ति॒स्रस्ति॒स्रो दे॒वीः। व॒सु॒वने॑ वसु॒धेय॑स्य वियन्तु। दे॒वो नरा॒शस॑। व॒सु॒वने॑ वसु॒धेय॑स्य वेतु। दे॒वो वन॒स्पति॑। व॒सु॒वने॑ वसु॒धेय॑स्य वेतु। दे॒वं ब॒र्॒हिर्वारि॑तीनाम्। व॒सु॒वने॑ वसु॒धेय॑स्य वेतु॥३५॥

%3.6.14.3
दे॒वो अ॒ग्निः स्वि॑ष्ट॒कृत्। सु॒द्रवि॑णा म॒न्द्रः क॒विः। स॒त्यम॑न्माय॒जी होता। होतु॑र्‌होतु॒राय॑जीयान्। अग्ने॒ यान्दे॒वानयाट्। या अपि॑प्रेः। ये ते॑ हो॒त्रे अम॑त्सत। ता स॑स॒नुषी॒ होत्रान्देवङ्ग॒माम्। दि॒वि दे॒वेषु॑ य॒ज्ञमेर॑ये॒मम्। स्वि॒ष्ट॒कृच्चाग्ने॒ होताऽभू। व॒सु॒वने॑ वसु॒धेय॑स्य नमोवा॒के वीहि॑॥३६॥\anuvakamend[वी॒तां॒ वे॒त्वभू॒रेकं च]

%3.6.15.1
अ॒ग्निम॒द्य होता॑रमवृणीता॒यं यज॑मान॒ पच॑न्प॒क्तीः पच॑न्पुरो॒डाशं॑ ब॒ध्नन्नि॑न्द्रा॒ग्निभ्या॒ञ्छाग सूप॒स्था अ॒द्य दे॒वो वन॒स्पति॑रभवदिन्द्रा॒ग्निभ्यां॒ छागे॒नाघ॑स्ता॒न्तं मे॑द॒स्तः प्रति॑पच॒ताग्र॑भीष्टा॒मवी॑वृधेतां पुरो॒डाशे॑न॒ त्वाम॒द्यर्\mbox{}ष॑ आर्\mbox{}षेय ऋषीणान्नपादवृणीता॒यं यज॑मानो ब॒हुभ्य॒ आ सङ्ग॑तेभ्य ए॒ष मे॑ दे॒वेषु॒ वसु॒ वार्या य॑क्ष्यत॒ इति॒ ता या दे॒वा दे॑व॒दाना॒न्यदु॒स्तान्य॑स्मा॒ आ च॒ शास्वा च॑ गुरस्वेषि॒तश्च॑ होत॒रसि॑ भद्र॒वाच्या॑य॒ प्रेषि॑तो॒ मानु॑षः सूक्तवा॒काय॑ सू॒क्ता ब्रू॑हि ॥३७॥\anuvakamend[अ॒ग्निम॒द्यैकम्]






\prashnaend{अ॒ञ्जन्ति॒ होता॑ यक्ष॒त्समि॑द्धो अ॒द्याग्निरजै॒द्दैव्या॑ जु॒षस्वा वृ॑त्रहणा गी॒र्भिस्त्व ह्याभ॑रत॒मुपो॑ह॒ यद्दे॒वं ब॒र्॒हिः सु॑दे॒वन्दे॒वं ब॒र्॒हिर॒ग्निम॒द्य पञ्च॑दश॥१५॥}{अ॒ञ्जन्त्य॒ग्निर्\mbox{}होता॑ नो गी॒र्भिरुपो॑ ह॒ यद्वि॒दथं॑ वा॒जिन॑ स॒प्तत्रिशत्॥३७॥}{अ॒ञ्जन्ति॑ सू॒क्ताब्रू॑हि॥}{हरि॑ ओम्॥}{इति श्रीकृष्णयजुर्वेदीयतैत्तिरीयब्राह्मणे तृतीयाष्टके षष्ठः प्रपाठकः समाप्तः॥}
\clearpage
\sect{सप्तमः प्रश्नः}
\setcounter{anuvakam}{0}
\dnsub{तैत्तिरीयब्राह्मणे तृतीयाष्टके सप्तमः प्रपाठकः}

%3.7.1.1
सर्वा॒न्॒ वा ए॒षोऽग्नौ कामा॒न्प्रवे॑शयति। योऽग्नीन॑न्वा॒धाय॑ व्र॒तमु॒पैति॑। सयदनि॑ष्ट्वा प्रया॒यात्। अका॑मप्रीता एन॒ङ्कामा॒ नानु॒प्रया॑युः। अ॒ते॒जा अ॑वी॒र्य॑ स्यात्। स जु॑हुयात्। तुभ्य॒न्ता अ॑ङ्गिरस्तम। विश्वा सुक्षि॒तय॒ पृथ॑क्। अग्ने॒ कामा॑य येमिर॒ इति॑। कामा॑ने॒वास्मि॑न्दधाति॥१॥

%3.7.1.2
काम॑प्रीता एन॒ङ्कामा॒ अनु॒ प्रयान्ति। ते॒ज॒स्वी वी॒र्या॑वान्भवति। सन्त॑ति॒र्वा ए॒षा य॒ज्ञस्य॑। योऽग्नीन॑न्वा॒धाय॑ व्र॒तमु॒पैति॑। स यदु॒द्वाय॑ति। विच्छि॑त्तिरे॒वास्य॒ सा। तं प्राञ्च॑मु॒द्धृत्य॑। मन॒सोप॑तिष्ठेत। मनो॒ वै प्र॒जाप॑तिः। प्रा॒जा॒प॒त्यो य॒ज्ञः॥२॥

%3.7.1.3
मन॑सै॒व य॒ज्ञ सन्त॑नोति। भूरित्या॑ह। भू॒तो वै प्र॒जाप॑तिः। भूति॑मे॒वोपै॑ति। वि वा ए॒ष इ॑न्द्रि॒येण॑ वी॒र्ये॑णर्ध्यते। यस्याहि॑ताग्नेर॒ग्निर॑प॒क्षाय॑ति। याव॒च्छम्य॑या प्र॒विध्येत्। यदि॒ ताव॑दप॒क्षायेत्। त सम्भ॑रेत्। इ॒दन्त॒ एकं॑ प॒र उ॑त॒ एकम्॥३॥

%3.7.1.4
तृ॒तीये॑न॒ ज्योति॑षा॒ संवि॑शस्व। सं॒वेश॑नस्त॒नुवै॒ चारु॑रेधि। प्रि॒ये दे॒वानां पर॒मे ज॒नित्र॒ इति॑। ब्रह्म॑णै॒वैन॒ सम्भ॑रति। सैव तत॒ प्राय॑श्चित्तिः। यदि॑ परस्त॒राम॑प॒क्षायेत्। अ॒नु॒प्र॒यायाव॑स्येत्। सो ए॒व तत॒ प्राय॑श्चित्तिः। ओष॑धी॒र्वा ए॒तस्य॑ प॒शून्पय॒ प्रवि॑शति। यस्य॑ ह॒विषे॑ व॒त्सा अ॒पाकृ॑ता॒ धय॑न्ति॥४॥

%3.7.1.5
तान् यद्दु॒ह्यात्। या॒तयाम्ना ह॒विषा॑ यजेत। यन्न दु॒ह्यात्। य॒ज्ञ॒प॒रुर॒न्तरि॑यात्। वा॒य॒व्यां यवा॒गून्निर्व॑पेत्। वा॒युर्वै पय॑सः प्रदापयि॒ता। स ए॒वास्मै॒ पय॒ प्रदा॑पयति। पयो॒ वा ओष॑धयः। पय॒ पय॑। पय॑सै॒वास्मै॒ पयोऽव॑रुन्धे॥५॥

%3.7.1.6
अथोत्त॑रस्मै ह॒विषे॑ व॒त्सान॒पाकु॑र्यात्। सैव तत॒ प्राय॑श्चित्तिः। अ॒न्य॒त॒रान् वा ए॒ष दे॒वान्भा॑ग॒धेये॑न॒ व्य॑र्धयति। ये यज॑मानस्य सा॒यङ्गृ॒हमा॒ गच्छ॑न्ति। यस्य॑ सायन्दु॒ग्ध ह॒विरार्ति॑मा॒र्च्छति॑। इन्द्रा॑य व्री॒हीन्नि॒रुप्योप॑ वसेत्। पयो॒ वा ओष॑धयः। पय॑ ए॒वारभ्य॑ गृही॒त्वोप॑ वसति। यत्प्रा॒तः स्यात्। तच्छृ॒तं कु॑र्यात्॥६॥

%3.7.1.7
अथेत॑र ऐ॒न्द्रः पु॑रो॒डाश॑ स्यात्। इ॒न्द्रि॒ये ए॒वास्मै॑ स॒मीची॑ दधाति। पयो॒ वा ओष॑धयः। पय॒ पय॑। पय॑सै॒वास्मै॒ पयोऽव॑रुन्धे। अथोत्त॑रस्मै ह॒विषे॑ व॒त्सान॒पाकु॑र्यात्। सैव तत॒ प्राय॑श्चित्तिः। उ॒भया॒न्॒ वा ए॒ष दे॒वान्भा॑ग॒धेये॑न॒ व्य॑र्धयति। ये यज॑मानस्य सा॒यं च॑ प्रा॒तश्च॑ गृ॒हमा॒ गच्छ॑न्ति। यस्यो॒भय ह॒विरार्ति॑मा॒र्च्छति॑॥७॥

%3.7.1.8
ऐ॒न्द्रं पञ्च॑शरावमोद॒नन्निर्व॑पेत्। अ॒ग्निं दे॒वता॑नां प्रथ॒मं य॑जेत्। अ॒ग्निमु॑खा ए॒व दे॒वता प्रीणाति। अ॒ग्निं वा अन्व॒न्या दे॒वता। इन्द्र॒मन्व॒न्याः। ता ए॒वोभयी प्रीणाति। पयो॒ वा ओष॑धयः। पय॒ पय॑। पय॑सै॒वास्मै॒ पयोऽव॑रुन्धे। अथोत्तर॑स्मै ह॒विषे॑ व॒त्सान॒पाकु॑र्यात्॥८॥

%3.7.1.9
सैव तत॒ प्राय॑श्चित्तिः। अ॒र्धो वा ए॒तस्य॑ य॒ज्ञस्य॑ मीयते। यस्य॒ व्रत्येऽह॒न्पत्न्य॑नालम्भु॒का भव॑ति। ताम॑प॒रुध्य॑ यजेत। सर्वे॑णै॒व य॒ज्ञेन॑ यजते। तामि॒ष्ट्वोप॑ ह्वयेत। अमू॒हम॑स्मि। सा त्वम्। द्यौर॒हम्। पृ॒थि॒वी त्वम्। सामा॒हम्। ऋक्त्वम्। तावेहि॒ सम्भ॑वाव। स॒ह रेतो॑ दधावहै। पु॒से पु॒त्राय॒ वेत्त॑वै। रा॒यस्पोषा॑य सुप्रजा॒स्त्वाय॑ सु॒वीर्या॒येति॑। अ॒र्ध ए॒वैना॒मुप॑ ह्वयते। सैव तत॒ प्राय॑श्चित्तिः॥९॥\anuvakamend[द॒धा॒ति॒ य॒ज्ञ उ॑त॒ एक॒न्धय॑न्ति रुन्धे कुर्यादा॒र्च्छत्य॒पाकु॑र्यात्पृथि॒वी त्वम॒ष्टौ च॑ (सर्वा॒न्॒ वि वै यदि॑ परस्त॒रामोष॑धीरन्यत॒रानु॒भया॑न॒र्धो वै ॥ )]

%3.7.2.1
यद्विष्ष॑ण्णेन जुहु॒यात्। अप्र॑जा अप॒शुर्यज॑मानः स्यात्। यदना॑यतने नि॒नयेत्। अ॒ना॒य॒त॒नः स्यात्। प्रा॒जा॒प॒त्यय॒र्चा व॑ल्मीकव॒पाया॒मव॑ नयेत्। प्रा॒जा॒प॒त्यो वै व॒ल्मीक॑। य॒ज्ञः प्र॒जाप॑तिः। प्र॒जाप॑तावे॒व य॒ज्ञं प्रति॑ष्ठापयति। भूरित्या॑ह। भू॒तो वै प्र॒जाप॑तिः॥१०॥

%3.7.2.2
भूति॑मे॒वोपै॑ति। तत्कृ॒त्वा। अ॒न्यां दु॒ग्ध्वा पुन॑र्‌होत॒व्यम्। सैव तत॒ प्राय॑श्चित्तिः। यत्की॒टाव॑पन्नेन जुहु॒यात्। अप्र॑जा अप॒शुर्यज॑मानः स्यात्। यदना॑यतने नि॒नयेत्। अ॒ना॒य॒त॒नः स्यात्। म॒ध्य॒मेन॑ प॒र्णेन॑ द्यावापृथि॒व्य॑य॒र्चाऽन्त॑ परि॒धि निन॑येत्। द्यावा॑पृथि॒व्योरे॒वैन॒त्प्रति॑ष्ठापयति॥११॥

%3.7.2.3
तत्कृ॒त्वा। अ॒न्यां दु॒ग्ध्वा पुन॑र्\mbox{}होत॒व्यम्। सैव तत॒ प्राय॑श्चित्तिः। यदव॑वृष्टेन जुहु॒यात्। अप॑रूपमस्या॒त्मञ्जा॑येत। कि॒लासो॑ वा॒स्याद॑र्\mbox{}श॒सो वा। यत्प्रत्ये॒यात्। य॒ज्ञं विच्छि॑न्द्यात्। स जु॑हुयात्। मि॒त्रो जनान्कल्पयति प्रजा॒नन्॥१२॥

%3.7.2.4
मि॒त्रो दा॑धार पृथि॒वीमु॒त द्याम्। मि॒त्रः कृ॒ष्टीरनि॑मिषा॒ऽभि च॑ष्टे। स॒त्याय॑ ह॒व्यङ्घृ॒तव॑ज्जुहो॒तेति॑। मि॒त्रेणै॒वैन॑त्कल्पयति। तत्कृ॒त्वा। अ॒न्यां दु॒ग्ध्वा पुन॑र्\mbox{}होत॒व्यम्। सैव तत॒ प्राय॑श्चित्तिः। यत्पूर्व॑स्या॒माहु॑त्या हु॒ताया॒मुत्त॒राऽऽहु॑ति॒ स्कन्देत्। द्वि॒पाद्भि॑ प॒शुभि॒र्यज॑मानो॒ व्यृ॑ध्येत। यदुत्त॑रया॒ऽभि जु॑हु॒यात्॥१३॥

%3.7.2.5
चतु॑ष्पाद्भिः प॒शुभि॒र्यज॑मानो॒ व्यृ॑ध़्येत। यत्र॒ वेत्थ॑ वनस्पते दे॒वाना॒ङ्गुह्या॒ नामा॑नि। तत्र॑ ह॒व्यानि॑ गाम॒येति॑ वानस्प॒त्यय॒र्चा स॒मिध॑मा॒धाय॑। तू॒ष्णीमे॒व पुन॑र्जुहुयात्। वन॒स्पति॑नै॒व य॒ज्ञस्यार्ता॒ञ्चानार्ता॒ञ्चाहु॑ती॒ वि दा॑धार। तत्कृ॒त्वा। अ॒न्यां दु॒ग्ध्वा पुन॑र्\mbox{}होत॒व्यम्। सैव तत॒ प्राय॑श्चित्तिः। यत्पु॒रा प्र॑या॒जेभ्य॒ प्राङङ्गा॑र॒ स्कन्देत्। अ॒ध्व॒र्यवे॑ च॒ यज॑मानाय॒ चाक स्यात्॥१४॥

%3.7.2.6
यद्द॑क्षि॒णा। ब्र॒ह्मणे॑ च॒ यज॑मानाय॒ चाक स्यात्। यत्प्र॒त्यक्। होत्रे॑ च॒ पत्नि॑यै च॒ यज॑मानाय॒ चाक स्यात्। यदुदङ्ङ्॑। अ॒ग्नीधे॑ च प॒शुभ्य॑श्च॒ यज॑मानाय॒ चाक स्यात्। यद॑भिजुहु॒यात्। रु॒द्रोस्य प॒शून्घातु॑कः स्यात्। यन्नाभि॑जुहु॒यात्। अशान्त॒ प्रह्रि॑येत॥१५॥

%3.7.2.7
स्रु॒वस्य॒ बुध्ने॑नाभि॒निद॑ध्यात्। मा त॑मो॒ मा य॒ज्ञस्त॑म॒न्मा यज॑मानस्तमत्। नम॑स्ते अस्त्वाय॒ते। नमो॑ रुद्र पराय॒ते। नमो॒ यत्र॑ नि॒षीद॑सि। अ॒मुं मा हिसीर॒मुं मा हिसी॒रिति॒ येन॒ स्कन्देत्। तं प्रह॑रेत्। स॒हस्र॑शृङ्गो वृष॒भो जा॒तवे॑दाः। स्तोम॑पृष्ठो घृ॒तवान्त्सु॒प्रती॑कः। मा नो॑ हासीन्मेत्थि॒तो नेत्त्वा॒ जहा॑म। गो॒पो॒षन्नो॑ वीरपो॒षं च॑ य॒च्छेति॑। ब्रह्म॑णै॒वैनं॒ प्र ह॑रति। सैव तत॒ प्राय॑श्चित्तिः॥१६॥\anuvakamend[वै प्र॒जाप॑तिः स्थापयति प्रजा॒नन्न॒भि जु॑हु॒यात्स्याद्ध्रियेत॒ जहा॑म॒ त्रीणि॑ च (यद्विष्ष॑ण्णेन प्राजाप॒त्यया॒ यत्की॒टा म॑ध्य॒मेन॒ यदव॑वृष्टेन॒ यत्पूर्व॑स्यां॒ यत्पु॒रा प्र॑या॒जेभ्य॒ प्राङङ्गा॑रो॒ यद्द॑क्षि॒णा यत्प्र॒त्यग्यदुदङ्ङ्॑ ॥ )]

%3.7.3.1
वि वा ए॒ष इ॑न्द्रि॒येण॑ वी॒र्ये॑णर्ध्यते। यस्याहि॑ताग्नेर॒ग्निर्म॒थ्यमा॑नो॒ न जाय॑ते। यत्रा॒न्यं पश्येत्। तत॑ आ॒हृत्य॑ होत॒व्यम्। अ॒ग्नावे॒वास्याग्निहो॒त्र हु॒तं भ॑वति। यद्य॒न्यन्न वि॒न्देत्। अ॒जाया होत॒व्यम्। आ॒ग्ने॒यी वा ए॒षा। यद॒जा। अ॒ग्नावे॒वास्याग्निहो॒त्र हु॒तं भ॑वति॥१७॥

%3.7.3.2
अ॒जस्य॒ तु नाश्ञी॑यात्। यद॒जस्याश्ञी॒यात्। यामे॒वाग्नावाहु॑तिं जुहु॒यात्। ताम॑द्यात्। तस्मा॑द॒जस्य॒ नाश्यम्। यद्य॒जान्न वि॒न्देत्। ब्रा॒ह्म॒णस्य॒ दक्षि॑णे॒ हस्ते॑ होत॒व्यम्। ए॒ष वा अ॒ग्निर्वैश्वान॒रः। यद्ब्राह्म॒णः। अ॒ग्नावे॒वास्याग्निहो॒त्र हु॒तं भ॑वति॥१८॥

%3.7.3.3
ब्रा॒ह्म॒णन्तु व॑स॒त्यै॑ नाप॑ रुन्ध्यात्। यद्ब्राह्म॒णं व॑स॒त्या अ॑परु॒न्ध्यात्। यस्मि॑न्ने॒वाग्नावाहु॑तिं जुहु॒यात्। तम्भा॑ग॒धेये॑न॒ व्य॑र्धयेत्। तस्माद्ब्राह्म॒णो व॑स॒त्यै॑ नाप॒रुध्य॑। यदि॑ ब्राह्म॒णन्न वि॒न्देत्। द॒र्भ॒स्त॒म्बे हो॑त॒व्यम्। अ॒ग्नि॒वाऩ््वै द॑र्भस्त॒म्बः। अ॒ग्नावे॒वास्याग्निहो॒त्र हु॒तं भ॑वति। द॒र्भास्तु नाध्या॑सीत॥१९॥

%3.7.3.4
यद्द॒र्भान॒ध्यासी॑त। यामे॒वाग्नावाहु॑तिं जुहु॒यात्। तामध्या॑सीत। तस्माद्द॒र्भा नाध्या॑सित॒व्या। यदि॑ द॒र्भान्न वि॒न्देत्। अ॒प्सु हो॑त॒व्यम्। आपो॒ वै सर्वा॑ दे॒वता। दे॒वतास्वे॒वास्याग्निहो॒त्र हु॒तं भ॑वति। आप॒स्तु न परि॑चक्षीत। यदाप॑ परि॒चक्षी॑त॥२०॥

%3.7.3.5
यामे॒वाप्स्वाहु॑तिं जुहु॒यात्। तां परि॑चक्षीत। तस्मा॒दापो॒ न प॑रि॒चक्ष्या। मेध्या॑ च॒ वा ए॒तस्या॑मे॒ध्या च॑ त॒नुवौ॒ स सृ॑ज्येते। यस्याहि॑ताग्नेर॒न्यैर॒ग्निभि॑र॒ग्नय॑ ससृ॒ज्यन्ते। अ॒ग्नये॒ विवि॑चये पुरो॒डाश॑म॒ष्टाक॑पालं॒ निर्व॑पेत्। मेध्याञ्चै॒वास्या॑मे॒ध्यां च॑ त॒नुवौ॒ व्याव॑र्तयति। अ॒ग्नये व्र॒तप॑तये पु॒रो॒डाश॑म॒ष्टाक॑पालं॒ निर्व॑पेत्। अ॒ग्निमे॒व व्र॒तप॑ति॒ स्वेन॑ भाग॒धेये॒नोप॑ धावति। स ए॒वैनं॑ व्र॒तमा ल॑म्भयति॥२१॥

%3.7.3.6
गर्भ॒ स्रव॑न्तमग॒दम॑कः। अ॒ग्निरिन्द्र॒स्त्वष्टा॒ बृह॒स्पति॑। पृ॒थि॒व्यामव॑ चुश्चोतै॒तत्। नाभिप्राप्नो॑ति॒ निर्‌ऋ॑तिं परा॒चैः। रेतो॒ वा ए॒तद्वाजि॑न॒माहि॑ताग्नेः। यद॑ग्निहो॒त्रम्। तद्यत्स्रवेत्। रेतोऽस्य॒ वाजि॑न स्रवेत्। गर्भ॒ स्रव॑न्तमग॒दम॑क॒रित्या॑ह। रेत॑ ए॒वास्मि॒न्वाजि॑नन्दधाति॥२२॥

%3.7.3.7
अ॒ग्निरित्या॑ह। अ॒ग्निर्वै रे॑तो॒धाः। रेत॑ ए॒व तद्द॑धाति। इन्द्र॒ इत्या॑ह। इ॒न्द्रि॒यमे॒वास्मि॑न्दधाति। त्वष्टेत्या॑ह। त्वष्टा॒ वै प॑शू॒नां मि॑थु॒नाना रूप॒कृत्। रू॒पमे॒व प॒शुषु॑ दधाति। बृह॒स्पति॒रित्या॑ह। ब्रह्म॒ वै दे॒वानां॒ बृह॒स्पति॑। ब्रह्म॑णै॒वास्मै प्र॒जाः प्र ज॑नयति। पृ॒थि॒व्यामव॑ चुश्चोतै॒तदित्या॑ह। अ॒स्यामे॒वैन॒त्प्रति॑ष्ठापयति। नाभिप्राप्नो॑ति॒ निर्‌ऋ॑तिं पराचै॒रित्या॑ह। रक्ष॑सा॒मप॑हत्यै॥२३॥\anuvakamend[अ॒जाऽग्नावे॒वास्याग्निहो॒त्र हु॒तं भ॑वति भवत्यासीत परि॒चक्षी॑त लम्भयति दधाति दे॒वानां॒ बृह॒स्पति॒ पञ्च॑ च  (वि वै यद्य॒न्यम॒जायां ब्राह्म॒णस्य॑ दर्भस्त॒म्बेऽप्सु हो॑त॒व्यम्॥ )]

%3.7.4.1
याः पु॒रस्तात्प्र॒स्रव॑न्ति। उ॒परि॑ष्टात्स॒र्वत॑श्च॒ याः। ताभी॑ र॒श्मिप॑वित्राभिः। श्र॒द्धां य॒ज्ञमा र॑भे। देवा॑ गातुविदः। गा॒तुं य॒ज्ञाय॑ विन्दत। मन॑स॒स्पति॑ना दे॒वेन॑। वाताद्य॒ज्ञः प्र यु॑ज्यताम्। तृ॒तीय॑स्यै दि॒वः। गा॒य॒त्रि॒या सोम॒ आभृ॑तः॥२४॥

%3.7.4.2
सो॒म॒पी॒थाय॒ सन्न॑यितुम्। वक॑ल॒मन्त॑र॒मा द॑दे। आपो॑ देवीः शु॒द्धाः स्थ॑। इ॒मा पात्रा॑णि शुन्धत। उ॒पा॒त॒ङ्क्या॑य दे॒वानाम्। प॒र्ण॒व॒ल्कमु॒त शु॑न्धत। पयो॑ गृ॒हेषु॒ पयो॑ अघ्नि॒यासु॑। पयो॑ व॒त्सेषु॒ पय॒ इन्द्रा॑य ह॒विषे ध्रियस्व। गा॒य॒त्री प॑र्णव॒ल्केन॑। पय॒ सोमं॑ करोत्वि॒मम्॥२५॥

%3.7.4.3
अ॒ग्निं गृ॑ह्णामि सु॒रथ॒य्योँ म॑यो॒भूः। य उ॒द्यन्त॑मा॒रोह॑ति॒ सूर्य॒मह्ने। आ॒दि॒त्यञ्ज्योति॑षां॒ ज्योति॑रुत्त॒मम्। श्वो य॒ज्ञाय॑ रमतान्दे॒वताभ्यः। वसून्रु॒द्राना॑दि॒त्यान्। इन्द्रे॑ण स॒ह दे॒वता। ताः पूर्व॒ परि॑ गृह्णामि। स्व आ॒यत॑ने मनी॒षया। इ॒मामूर्जं॑ पञ्चद॒शींय्येँ प्रवि॑ष्टाः। तान्दे॒वान्परि॑ गृह्णामि॒ पूर्व॑॥२६॥

%3.7.4.4
अ॒ग्निर्‌ह॑व्य॒वाडि॒ह ताना व॑हतु। पौ॒र्णमा॒स ह॒विरि॒दमे॑षां॒ मयि॑। आ॒मा॒वा॒स्य ह॒विरि॒दमे॑षां॒ मयि॑। अ॒न्त॒राऽग्नी प॒शव॑। दे॒व॒स॒सद॒मा ग॑मन्। तान्पूर्व॒ परि॑ गृह्णामि। स्व आ॒यत॑ने मनी॒षया। इ॒ह प्र॒जा वि॒श्वरू॑पा रमन्ताम्। अ॒ग्निङ्गृ॒हप॑तिम॒भि सं॒वसा॑नाः। ताः पूर्व॒ परि॑ गृह्णामि॥२७॥

%3.7.4.5
स्व आ॒यत॑ने मनी॒षया। इ॒ह प॒शवो॑ वि॒श्वरू॑पा रमन्ताम्। अ॒ग्निङ्गृ॒हप॑तिम॒भि सं॒वसा॑नाः। तान्पूर्व॒ परि॑ गृह्णामि। स्व आ॒यत॑ने मनी॒षया। अ॒यं पि॑तृ॒णाम॒ग्निः। अवाड्ढ॒व्या पि॒तृभ्य॒ आ। तं पूर्व॒ परि॑ गृह्णामि। अवि॑षन्नः पि॒तुङ्क॑रत्। अज॑स्र॒न्त्वा स॑भापा॒लाः॥२८॥

%3.7.4.6
वि॒ज॒यभा॑ग॒ समि॑न्धताम्। अग्ने॑ दी॒दा॑य मे सभ्य। विजि॑त्यै श॒रद॑ श॒तम्। अन्न॑मावस॒थीयम्। अ॒भि ह॑राणि श॒रद॑ श॒तम्। आ॒व॒स॒थे श्रियं॒ मन्त्रम्। अहि॑र्बु॒ध्नियो॒ नि य॑च्छतु। इ॒दम॒हम॒ग्निज्येष्ठेभ्यः। वसु॑भ्यो य॒ज्ञं प्रब्र॑वीमि। इ॒दम॒हमिन्द्र॑ज्येष्ठेभ्यः॥२९॥

%3.7.4.7
रु॒द्रेभ्यो॑ य॒ज्ञं प्र ब्र॑वीमि। इ॒दम॒हं वरु॑णज्येष्ठेभ्यः। आ॒दि॒त्येभ्यो॑ य॒ज्ञं प्र ब्र॑वीमि। पय॑स्वती॒रोष॑धयः। पय॑स्वद्वी॒रुधां॒ पय॑। अ॒पां पय॑सो॒ यत्पय॑। तेन॒ मामि॑न्द्र॒ स सृ॑ज। अग्ने व्रतपते व्र॒तञ्च॑रिष्यामि। तच्छ॑केय॒न्तन्मे॑ राध्यताम्। वायो व्रतपत॒ आदि॑त्य व्रतपते॥३०॥

%3.7.4.8
व्र॒तानां व्रतपते व्र॒तं च॑रिष्यामि। तच्छ॑केय॒न्तन्मे॑ राध्यताम्। इ॒मां प्राची॒मुदी॑चीम्। इष॒मूर्ज॑म॒भि सस्कृ॑ताम्। ब॒हु॒प॒र्णामशु॑ष्काग्राम्। हरा॑मि पशु॒पाम॒हम्। यत्कृष्णो॑ रू॒पं कृ॒त्वा। प्रावि॑श॒स्त्वं वन॒स्पतीन्॑। तत॒स्त्वामे॑कविशति॒धा। सम्भ॑रामि सुस॒म्भृता॥३१॥

%3.7.4.9
त्रीन्प॑रि॒धी स्ति॒स्रः स॒मिध॑। य॒ज्ञायु॑रनुसञ्च॒रान्। उ॒प॒वे॒षं मेक्ष॑णं॒ धृष्टिम्। सं भ॑रामि सुस॒म्भृता। या जा॒ता ओष॑धयः। दे॒वेभ्य॑स्त्रियु॒गं पु॒रा। तासां॒ पर्व॑ राध्यासम्। प॒रि॒स्त॒रमा॒हर\sn{}॑। अ॒पां मेध्यं॑ य॒ज्ञियम्। सदे॑व शि॒वम॑स्तु मे॥३२॥

%3.7.4.10
आ॒च्छे॒त्ता वो॒ मा रि॑षम्। जीवा॑नि श॒रद॑ श॒तम्। अप॑रिमितानां॒ परि॑मिताः। सन्न॑ह्ये सुकृ॒ताय॒ कम्। एनो॒ मा निगाङ्कत॒मच्च॒नाहम्। पुन॑रु॒त्थाय॑ बहु॒ला भ॑वन्तु। स॒कृ॒दा॒च्छि॒न्नं ब॒र्॒हिरूर्णा॑मृदु। स्यो॒नं पि॒तृभ्य॑स्त्वा भराम्य॒हम्। अ॒स्मिन्त्सी॑दन्तु मे पि॒तर॑ सो॒म्याः। पि॒ता॒म॒हाः प्रपि॑तामहाश्चानु॒गैः स॒ह॥३३॥

%3.7.4.11
त्रि॒वृत्प॑ला॒शे द॒र्भः। इयान्प्रादे॒शस॑म्मितः। य॒ज्ञे प॒वित्रं॒ पोतृ॑तमम्। पयो॑ ह॒व्यं क॑रोतु मे। इ॒मौ प्रा॑णापा॒नौ। य॒ज्ञस्याङ्गा॑नि सर्व॒शः। आ॒प्या॒यय॑न्तौ॒ सञ्च॑रताम्। प॒वित्रे॑ हव्य॒शोध॑ने। प॒वित्रे स्थो वैष्ण॒वी। वा॒युर्वां॒ मन॑सा पुनातु॥३४॥

%3.7.4.12
अ॒यं प्रा॒णश्चा॑पा॒नश्च॑। यज॑मान॒मपि॑ गच्छताम्। य॒ज्ञे ह्यभू॑तां॒ पोता॑रौ। प॒वित्रे॑ हव्य॒शोध॑ने। त्वया॒ वेदिं॑ विविदुः पृथि॒वीम्। त्वया॑ य॒ज्ञो जा॑यते विश्व॒दानि॑। अच्छि॑द्रं य॒ज्ञमन्वे॑षि वि॒द्वान्। त्वया॒ होता॒ सन्त॑नोत्यर्धमा॒सान्। त्र॒य॒स्त्रि॒शो॑ऽसि॒ तन्तू॑नाम्। प॒वित्रे॑ण स॒हाग॑हि॥३५॥

%3.7.4.13
शि॒वेय रज्जु॑रभि॒धानी। अ॒घ्नि॒यामुप॑ सेवताम्। अप्र॑स्रसाय य॒ज्ञस्य॑। उ॒खे उप॑दधाम्य॒हम्। प॒शुभि॒ सन्नी॑तं बिभृताम्। इन्द्रा॑य शृ॒तन्दधि॑। उ॒प॒वे॒षो॑ऽसि य॒ज्ञाय॑। त्वां प॑रिवे॒षम॑धारयन्। इन्द्रा॑य ह॒विः कृ॒ण्वन्त॑। शि॒वः श॒ग्मो भ॑वासि नः॥३६॥

%3.7.4.14
अमृ॑न्मयन्देवपा॒त्रम्। य॒ज्ञस्यायु॑षि॒ प्र यु॑ज्यताम्। ति॒र॒ प॒वि॒त्रमति॑नीताः। आपो॑ धारय॒ माति॑गुः। दे॒वेन॑ सवि॒त्रोत्पू॑ताः। वसो॒ सूर्य॑स्य र॒श्मिभि॑। गान्दो॑हपवि॒त्रे रज्जुम्। सर्वा॒ पात्रा॑णि शुन्धत। ए॒ता आ च॑रन्ति॒ मधु॑म॒द्दुहा॑नाः। प्र॒जाव॑तीर्य॒शसो॑ वि॒श्वरू॑पाः॥३७॥

%3.7.4.15
ब॒ह्वीर्भव॑न्ती॒रुप॒जाय॑मानाः। इ॒ह व॒ इन्द्रो॑ रमयतु गावः। पू॒षा स्थ॑। अ॒य॒क्ष्मा व॑ प्र॒जया॒ स सृ॑जामि। रा॒यस्पोषे॑ण बहु॒लाभव॑न्तीः। ऊर्जं॒ पय॒ पिन्व॑माना घृ॒तं च॑। जी॒वो जीव॑न्ती॒रुप॑वः सदेयम्। द्यौश्चे॒मं य॒ज्ञं पृ॑थि॒वी च॒ सन्दु॑हाताम्। धा॒ता सोमे॑न स॒ह वाते॑न वा॒युः। यज॑मानाय॒ द्रवि॑णन्दधातु॥३८॥

%3.7.4.16
उत्स॑न्दुहन्ति क॒लश॒ञ्चतु॑र्बिलम्। इडान्दे॒वीम्मधु॑मती सुव॒र्विदम्। तदि॑न्द्रा॒ग्नी जि॑न्वत सू॒नृता॑वत्। तद्यज॑मानममृत॒त्वे द॑धातु। काम॑धुक्ष॒ प्र णो ब्रूहि। इन्द्रा॑य ह॒विरि॑न्द्रि॒यम्। अ॒मूं यस्यां दे॒वानाम्। म॒नु॒ष्या॑णां॒ पयो॑ हि॒तम्। ब॒हु दु॒ग्धीन्द्रा॑य दे॒वेभ्य॑। ह॒व्यमा प्या॑यतां॒ पुन॑॥३९॥

%3.7.4.17
व॒त्सेभ्यो॑ मनु॒ष्येभ्यः। पु॒न॒र्दो॒हाय॑ कल्पताम्। य॒ज्ञस्य॒ सन्त॑तिरसि। य॒ज्ञस्य॑ त्वा॒ सन्त॑ति॒मनु॒ सन्त॑नोमि। अद॑स्तमसि॒ विष्ण॑वे त्वा। य॒ज्ञायापि॑ दधाम्य॒हम्। अ॒द्भिररि॑क्तेन॒ पात्रे॑ण। याः पू॒ताः प॑रि॒शेर॑ते। अ॒यं पय॒ सोमं॑ कृ॒त्वा। स्वाय्योँनि॒मपि॑ गच्छतु॥४०॥

%3.7.4.18
प॒र्ण॒व॒ल्कः प॒वित्रम्। सौ॒म्यः सोमा॒द्धि निर्मि॑तः। इ॒मौ प॒र्णं च॑ द॒र्भं च॑। दे॒वाना हव्य॒शोध॑नौ। प्रा॒त॒र्वे॒षाय॑ गोपाय। विष्णो॑ ह॒व्य हि रक्ष॑सि। उ॒भाव॒ग्नी उ॑पस्तृण॒ते। दे॒वता॒ उप॑वसन्तु मे। अ॒हङ्ग्रा॒म्यानुप॑ वसामि। मह्य॒ङ्गोप॑तये प॒शून्॥४१॥\anuvakamend[आभृ॑त इ॒मं गृ॑ह्णामि॒ पूर्व॒स्ताः पूर्व॒ परि॑गृह्णामि सभापा॒ला इन्द्र॑ज्येष्ठेभ्य॒ आदि॑त्य व्रतपते सुस॒म्भृता॑ मे स॒ह पु॑नातु गहि नो वि॒श्वरू॑पा दधातु॒ पुन॑र्गच्छतु प॒शून् (याः पु॒रस्ता॑दि॒मामूर्ज॑मि॒ह प्र॒जा इ॒ह प॒शवो॒ऽयं पि॑तृ॒णाम॒ग्निः। )]

%3.7.5.1
देवा॑ दे॒वेषु॒ पराक्रमध्वम्। प्रथ॑मा द्वि॒तीये॑षु। द्विती॑यास्तृ॒तीये॑षु। त्रिरे॑कादशा इ॒ह मा॑ऽवत। इ॒द श॑केयं॒ यदि॒दं क॒रोमि॑। आ॒त्मा क॑रोत्वा॒त्मने। इ॒दङ्क॑रिष्ये भेष॒जम्। इ॒दम्मे॑ विश्वभेषजा। अश्वि॑ना॒ प्राव॑तं यु॒वम्। इ॒दम॒ह सेना॑या अ॒भीत्व॑र्यै॥४२॥

%3.7.5.2
मुख॒मपो॑हामि। सूर्य॑ ज्योति॒र्वि भा॑हि। म॒ह॒त इ॑न्द्रि॒याय॑। आ प्या॑यताङ्घृ॒तयो॑निः। अ॒ग्निर्‌ह॒व्याऽनु॑ मन्यताम्। खम॑ङ्क्ष्व॒ त्वच॑मङ्क्ष्व। सु॒रू॒पन्त्वा॑ वसु॒विदम्। प॒शू॒नान्तेज॑सा। अ॒ग्नये॒ जुष्ट॑म॒भि घा॑रयामि। स्यो॒नन्ते॒ सद॑नं करोमि॥४३॥

%3.7.5.3
घृ॒तस्य॒ धार॑या सु॒शेव॑ङ्कल्पयामि। तस्मिन्त्सीदा॒मृते॒ प्रति॑तिष्ठ। व्री॒ही॒णाम्मे॑ध सुमन॒स्यमा॑नः। आ॒र्द्रः प्र॑थस्नु॒र्भुव॑नस्य गो॒पाः। शृ॒त उत्स्ना॑ति जनि॒ता म॑ती॒नाम्। यस्त॑ आ॒त्मा प॒शुषु॒ प्रवि॑ष्टः। दे॒वानां वि॒ष्ठामनु॒ यो वि॑त॒स्थे। आ॒त्म॒न्वान्त्सो॑म घृ॒तवा॒न्॒ हि भू॒त्वा। दे॒वान्ग॑च्छ॒ सुव॑र्विन्द॒ यज॑मानाय॒ मह्यम्। इरा॒ भूति॑ पृथि॒व्यै रसो॒ मोत्क्र॑मीत्॥४४॥

%3.7.5.4
देवा पितर॒ पित॑रो देवाः। यो॑ऽहम॑स्मि॒ स सन् य॑जे। यस्यास्मि॒ न तम॒न्तरे॑मि। स्वं म॑ इ॒ष्ट स्वन्द॒त्तम्। स्वं पू॒र्त स्व श्रा॒न्तम्। स्व हु॒तम्। तस्य॑ मे॒ऽग्निरु॑पद्र॒ष्टा। वा॒युरु॑पश्रो॒ता। आ॒दि॒त्यो॑ऽनुख्या॒ता। द्यौः पि॒ता॥४५॥

%3.7.5.5
पृ॒थि॒वी मा॒ता। प्र॒जाप॑ति॒र्बन्धु॑। य ए॒वास्मि॒ स सन् य॑जे। मा भेर्मा संवि॑क्था॒ मा त्वा॑ हिसिषम्। मा ते॒ तेजोऽप॑ क्रमीत्। भ॒र॒तमुद्ध॑रे॒मनु॑ षिञ़्च। अ॒व॒दाना॑नि ते प्र॒त्यव॑दास्यामि। नम॑स्ते अस्तु॒ मा मा॑ हिसीः। यद॑व॒दाना॑नि तेऽव॒द्यन्। विलो॒माका॑र्‌षमा॒त्मन॑॥४६॥

%3.7.5.6
आज्ये॑न॒ प्रत्य॑नज्म्येनत्। तत्त॒ आ प्या॑यतां॒ पुन॑। अज्या॑यो यवमा॒त्रात्। आ॒व्या॒धात्कृ॑त्यतामि॒दम्। मा रू॑रुपाम य॒ज्ञस्य॑। शु॒द्ध स्वि॑ष्टमि॒द ह॒विः। मनु॑ना दृ॒ष्टाङ्घृतप॑दीम्। मि॒त्रावरु॑णसमीरिताम्। द॒क्षि॒णा॒र्धादसं॑भिन्दन्। अव॑द्याम्येक॒तोमु॑खाम्॥४७॥

%3.7.5.7
इडे॑ भा॒गं जु॑षस्व नः। जिन्व॒ गा जिन्वार्व॑तः। तस्यास्ते भक्षि॒वाण॑ स्याम। स॒र्वात्मा॑नः स॒र्वग॑णाः। ब्रध्न॒ पिन्व॑स्व। दद॑तो मे॒ मा क्षा॑यि। कु॒र्व॒तो मे॒ मोप॑दसत्। दि॒शाङ्कॢप्ति॑रसि। दिशो॑ मे कल्पन्ताम्। कल्प॑न्ताम्मे॒ दिश॑॥४८॥

%3.7.5.8
दैवीश्च॒ मानु॑षीश्च। अ॒हो॒रा॒त्रे मे॑ कल्पेताम्। अ॒र्ध॒मा॒सा मे॑ कल्पन्ताम्। मासा॑ मे कल्पन्ताम्। ऋ॒तवो॑ मे कल्पन्ताम्। सं॒व॒त्स॒रो मे॑ कल्पताम्। कॢप्ति॑रसि॒ कल्प॑तां मे। आशा॑नां त्वाऽऽशापा॒लेभ्य॑। च॒तुर्भ्यो॑ अ॒मृतेभ्यः। इ॒दं भू॒तस्याध्य॑क्षेभ्यः॥४९॥

%3.7.5.9
वि॒धेम॑ ह॒विषा॑ व॒यम्। भज॑तां भा॒गी भा॒गम्। मा भा॒गोऽभ॑क्त। निर॑भा॒गं भ॑जामः। अ॒पस्पि॑न्व। ओष॑धीर्जिन्व। द्वि॒पात्पा॑हि। चतु॑ष्पादव। दि॒वो वृष्टि॒मेर॑य। ब्रा॒ह्म॒णाना॑मि॒द ह॒विः॥५०॥

%3.7.5.10
सो॒म्याना सोमपी॒थिनाम्। निर्भ॒क्तो ब्राह्मणः। नेहा ब्राह्मणस्यास्ति। सम॑ङ्क्तां ब॒र्॒हिर्‌ह॒विषा॑ घृ॒तेन॑। समा॑दि॒त्यैर्वसु॑भि॒ सम्म॒रुद्भि॑। समिन्द्रे॑ण॒ विश्वे॑भिर्दे॒वेभि॑रङ्क्ताम्। दि॒व्यं नभो॑ गच्छतु॒ यत्स्वाहा। इ॒न्द्रा॒णीवा॑ऽविध॒वा भू॑यासम्। अदि॑तिरिव सुपु॒त्रा। अ॒स्थू॒रि त्वा॑ गार्‌हपत्य॥५१॥

%3.7.5.11
उप॒निष॑दे सुप्रजा॒स्त्वाय॑। सं पत्नी॒ पत्या॑ सुकृ॒तेन॑ गच्छताम्। य॒ज्ञस्य॑ यु॒क्तौ धुर्या॑वभूताम्। सं॒जा॒ना॒नौ विज॑हता॒मरा॑तीः। दि॒वि ज्योति॑र॒जर॒मा र॑भेताम्। दश॑ते त॒नुवो॑ यज्ञ य॒ज्ञिया। ताः प्री॑णातु॒ यज॑मानो घृ॒तेन॑। ना॒रि॒ष्ठयो प्र॒शिष॒मीड॑मानः। दे॒वानां॒ दैव्येऽपि॒ यज॑मानो॒ऽमृतो॑ऽभूत्। यं वान्दे॒वा अ॑कल्पयन्॥५२॥

%3.7.5.12
ऊ॒र्जो भा॒ग श॑तक्रतू। ए॒तद्वां॒ तेन॑ प्रीणानि। तेन॑ तृप्यतमहहौ। अ॒हन्दे॒वाना सु॒कृता॑मस्मि लो॒के। ममे॒दमि॒ष्टन्न मिथु॑र्भवाति। अ॒हन्ना॑रि॒ष्ठावनु॑ यजामि वि॒द्वान्। यदाभ्या॒मिन्द्रो॒ अद॑धाद्भाग॒धेय॑म्। अदा॑रसृद्भवत देवसोम। अ॒स्मिन् य॒ज्ञे म॑रुतो मृडता नः। मा नो॑ विदद॒भिभा॒मो अश॑स्तिः॥५३॥

%3.7.5.13
मा नो॑ विदद्वृ॒जना॒ द्वेष्या॒ या। ऋ॒ष॒भं वा॒जिनं॑ व॒यम्। पू॒र्णमा॑सं यजामहे। स नो॑ दोहता सु॒वीर्यम्। रा॒यस्पोष सह॒स्रिणम्। प्रा॒णाय॑ सु॒राध॑से। पू॒र्णमा॑साय॒ स्वाहा। अ॒मा॒वा॒स्या॑ सु॒भगा॑ सु॒शेवा। धे॒नुरि॑व॒ भूय॑ आ॒प्याय॑माना। सा नो॑ दोहता सु॒वीर्यम्। रा॒यस्पोष सह॒स्रिणम्। अ॒पा॒नाय॑ सु॒राध॑से। अ॒मा॒वा॒स्या॑यै॒ स्वाहा। अ॒भि स्तृ॑णीहि॒ परि॑ धेहि॒ वेदिम्। जा॒मिम्मा हिसीरमु॒या शया॑ना। हो॒तृ॒षद॑ना॒ हरि॑ताः सु॒वर्णा। नि॒ष्का इ॒मे यज॑मानस्य ब्र॒ध्ने॥५४॥\anuvakamend[अ॒भीत्व॑र्यै करोमि क्रमीत्पि॒ताऽऽत्मन॑ एक॒तो मु॑खां मे॒ दिशोऽध्य॑क्षेभ्यो ह॒विर्गा॑र्‌हपत्या कल्पय॒न्नश॑स्ति॒ सा नो॑ दोहता सु॒वीर्य स॒प्त च॑]

%3.7.6.1
परि॑स्तृणीत॒ परि॑धत्ता॒ग्निम्। परि॑हितो॒ऽग्निर्यज॑मानं भुनक्तु। अ॒पा रस॒ ओष॑धीना सु॒वर्ण॑। नि॒ष्का इ॒मे यज॑मानस्य सन्तु काम॒दुघा। अ॒मुत्रा॒मुष्मि॑ल्लोँ॒के। भूप॑ते॒ भुव॑नपते। म॒ह॒तो भू॒तस्य॑ पते। ब्र॒ह्माण॑न्त्वा वृणीमहे। अ॒हं भूप॑तिर॒हं भुव॑नपतिः। अ॒हं म॑ह॒तो भू॒तस्य॒ पति॑॥५५॥

%3.7.6.2
दे॒वेन॑ सवि॒त्रा प्रसू॑त॒ आर्त्वि॑ज्यङ्करिष्यामि। देव॑ सवितरे॒तन्त्वा॑ वृणते। बृह॒स्पतिं॒ दैव्यं॑ ब्र॒ह्माणम्। तद॒हं मन॑से॒ प्र ब्र॑वीमि। मनो॑ गायत्रि॒यै। गा॒य॒त्री त्रि॒ष्टुभे। त्रि॒ष्टुब्जग॑त्यै। जग॑त्यनु॒ष्टुभे। अ॒नु॒ष्टुक्प॒ङ्क्त्यै। प॒ङ्क्तिः प्र॒जाप॑तये॥५६॥

%3.7.6.3
प्र॒जाप॑ति॒र्विश्वेभ्यो दे॒वेभ्य॑। विश्वे॑ देवा॒ बृह॒स्पत॑ये। बृह॒स्पति॒र्ब्रह्म॑णे। ब्रह्म॒ भूर्भुव॒ सुव॑। बृह॒स्पति॑र्दे॒वानां ब्र॒ह्मा। अ॒हं म॑नु॒ष्या॑णाम्। बृह॑स्पते य॒ज्ञङ्गो॑पाय। इ॒दं तस्मै॑ ह॒र्म्यं क॑रोमि। यो वो॑ देवा॒श्चर॑ति ब्रह्म॒चर्यम्। मे॒धा॒वी दि॒क्षु मन॑सा तप॒स्वी॥५७॥

%3.7.6.4
अ॒न्तर्दू॒तश्च॑रति॒ मानु॑षीषु। चतु॑ शिखण्डा युव॒तिः सु॒पेशा। घृ॒तप्र॑तीका॒ भुव॑नस्य॒ मध्ये। म॒र्मृ॒ज्यमा॑ना मह॒ते सौभ॑गाय। मह्य॑न्धुक्ष्व॒ यज॑मानाय॒ कामान्॑। भूमि॑र्भू॒त्वा म॑हि॒मानं॑ पुपोष। ततो॑ दे॒वी व॑र्धयते॒ पयासि। य॒ज्ञिया॑ य॒ज्ञं वि च॒ यन्ति॒ शं च॑। ओष॑धी॒राप॑ इ॒ह शक्व॑रीश्च। यो मा॑ हृ॒दा मन॑सा॒ यश्च॑ वा॒चा॥५८॥

%3.7.6.5
यो ब्रह्म॑णा॒ कर्म॑णा॒ द्वेष्टि॑ देवाः। यः श्रु॒तेन॒ हृद॑येनेष्ण॒ता च॑। तस्येन्द्र॒ वज्रे॑ण॒ शिर॑श्छिनद्मि। ऊर्णा॑मृदु॒ प्रथ॑मान स्यो॒नम्। दे॒वेभ्यो॒ जुष्ट॒ सद॑नाय ब॒र्॒हिः। सु॒व॒र्गे लो॒के यज॑मान॒ हि धे॒हि। मां नाक॑स्य पृ॒ष्ठे प॑र॒मे व्यो॑मन्। चतु॑ शिखण्डा युव॒तिः सु॒पेशा। घृ॒तप्र॑तीका व॒युना॑नि वस्ते। साऽऽस्ती॒र्यमा॑णा मह॒ते सौभ॑गाय ॥५९॥

%3.7.6.6
सा मे॑ धुक्ष्व॒ यज॑मानाय॒ कामान्॑। शि॒वा च॑ मे श॒ग्मा चै॑धि। स्यो॒ना च॑ मे सु॒षदा॑ चैधि। ऊर्ज॑स्वती च मे॒ पय॑स्वती चैधि। इष॒मूर्जं॑ मे पिन्वस्व। ब्रह्म॒ तेजो॑ मे पिन्वस्व। क्ष॒त्रमोजो॑ मे पिन्वस्व। विशं॒ पुष्टिं॑ मे पिन्वस्व। आयु॑र॒न्नाद्य॑म्मे पिन्वस्व। प्र॒जां प॒शून्मे॑ पिन्वस्व॥६०॥

%3.7.6.7
अ॒स्मिन् य॒ज्ञ उप॒ भूय॒ इन्नु मे। अवि॑क्षोभाय परि॒धीन्द॑धामि। ध॒र्ता ध॒रुणो॒ धरी॑यान्। अ॒ग्निर्द्वेषासि॒ निरि॒तो नु॑दातै। विच्छि॑नद्मि॒ विधृ॑तीभ्या स॒पत्नान्॑। जा॒तान्भ्रातृ॑व्या॒न्॒ ये च॑ जनि॒ष्यमा॑णाः। वि॒शो य॒न्त्राभ्यां॒ विध॑माम्येनान्। अ॒ह स्वाना॑मुत्त॒मो॑ऽसानि देवाः। वि॒शो य॒न्त्रे नु॒दमा॑ने॒ अरा॑तिम्। विश्वं॑ पा॒प्मान॒मम॑तिन्दुर्मरा॒युम्॥६१॥

%3.7.6.8
सीद॑न्ती दे॒वी सु॑कृ॒तस्य॑ लो॒के। धृती स्थो॒ विधृ॑ती॒ स्वधृ॑ती। प्रा॒णान्मयि॑ धारयतम्। प्र॒जाम्मयि॑ धारयतम्। प॒शून्मयि॑ धारयतम्। अ॒यं प्र॑स्त॒र उ॒भय॑स्य ध॒र्ता। ध॒र्ता प्र॑या॒जाना॑मु॒तानू॑या॒जानाम्। स दा॑धार स॒मिधो॑ वि॒श्वरू॑पाः। तस्मि॒न्त्स्रुचो॒ अध्या सा॑दयामि। आ रो॑ह प॒थो जु॑हु देव॒यानान्॑॥६२॥

%3.7.6.9
यत्रर्‌ष॑यः प्रथम॒जा ये पु॑रा॒णाः। हिर॑ण्यपक्षाऽजि॒रा सम्भृ॑ताङ्गा। वहा॑सि मा सु॒कृतां॒ यत्र॑ लो॒काः। अवा॒हं बा॑ध उप॒भृता॑ स॒पत्नान्॑। जा॒तान्भ्रातृ॑व्या॒न्॒ ये च॑ जनि॒ष्यमा॑णाः। दोहै॑ य॒ज्ञ सु॒दुघा॑मिव धे॒नुम्। अ॒हमुत्त॑रो भूयासम्। अध॑रे॒ मत्स॒पत्ना। यो मा॑ वा॒चा मन॑सा दुर्मरा॒युः। हृ॒दाऽरा॑ती॒याद॑भि॒दास॑दग्ने॥६३॥

%3.7.6.10
इ॒दम॑स्य चि॒त्तमध॑रन्ध्रु॒वाया। अ॒हमुत्त॑रो भूयासम्। अध॑रे॒ मत्स॒पत्ना। ऋ॒ष॒भो॑ऽसि शाक्व॒रः। घृ॒ताची॑ना सू॒नुः। प्रि॒येण॒ नाम्ना प्रि॒ये सद॑सि सीद। स्यो॒नो मे॑ सीद सु॒षद॑ पृथि॒व्याम्। प्रथ॑यि प्र॒जया॑ प॒शुभि॑ सुव॒र्गे लो॒के। दि॒वि सी॑द पृथि॒व्याम॒न्तरि॑क्षे। अ॒हमुत्त॑रो भूयासम्॥६४॥

%3.7.6.11
अध॑रे॒ मत्स॒पत्ना। इ॒य स्था॒ली घृ॒तस्य॑ पू॒र्णा। अच्छि॑न्नपयाः श॒तधा॑र॒ उत्स॑। मा॒रु॒तेन॒ शर्म॑णा॒ दैव्ये॑न। य॒ज्ञो॑ऽसि स॒र्वत॑ श्रि॒तः। स॒र्वतो॒ मां भू॒तं भ॑वि॒ष्यच्छ्र॑यताम्। श॒तम्मे॑ सन्त्वा॒शिष॑। स॒हस्र॑म्मे सन्तु सू॒नृता। इरा॑वतीः पशु॒मती। प्र॒जाप॑तिरसि स॒र्वत॑ श्रि॒तः॥६५॥

%3.7.6.12
स॒र्वतो॒ मां भू॒तं भ॑वि॒ष्यच्छ्र॑यताम्। श॒तं मे॑ सन्त्वा॒शिष॑। स॒हस्रं॑ मे सन्तु सू॒नृता। इरा॑वतीः पशु॒मती। इ॒दमि॑न्द्रि॒यम॒मृतं॑ वी॒र्यम्। अ॒नेनेन्द्रा॑य प॒शवो॑ऽचिकित्सन्। तेन॑ देवा अव॒तोप॒ माम्। इ॒हेष॒मूर्जं॒ यश॒ सह॒ ओज॑ सनेयम्। शृ॒तं मयि॑ श्रयताम्। यत्पृ॑थि॒वीमच॑र॒त्तत्प्रवि॑ष्टम्॥६६॥

%3.7.6.13
येनासि॑ञ्च॒द्बल॒मिन्द्रे प्र॒जाप॑तिः। इ॒दन्तच्छु॒क्रं मधु॑ वा॒जिनी॑वत्। येनो॒परि॑ष्टा॒दधि॑नोन्महे॒न्द्रम्। दधि॒ मान्धि॑नोतु। अ॒यं वे॒दः पृ॑थि॒वीमन्व॑विन्दत्। गुहा॑ स॒तीङ्गह॑ने॒ गह्व॑रेषु। स वि॑न्दतु॒ यज॑मानाय लो॒कम्। अच्छि॑द्रं य॒ज्ञं भूरि॑कर्मा करोतु। अ॒यं य॒ज्ञः सम॑सदद्ध॒विष्मान्॑। ऋ॒चा साम्ना॒ यजु॑षा दे॒वता॑भिः॥६७॥

%3.7.6.14
तेन॑ लो॒कान्त्सूर्य॑वतो जयेम। इन्द्र॑स्य स॒ख्यम॑मृत॒त्वम॑श्याम्। यो न॒ कनी॑य इ॒ह का॒मया॑तै। अ॒स्मिन् य॒ज्ञे यज॑मानाय॒ मह्यम्। अप॒ तमि॑न्द्रा॒ग्नी भुव॑नान्नुदेताम्। अ॒हं प्र॒जां वी॒रव॑तीं विदेय। अग्ने॑ वाजजित्। वाज॑न्त्वा सरि॒ष्यन्तम्। वाजं॑ जे॒ष्यन्तम्। वा॒जिनं॑ वाज॒जितम्॥६८॥

%3.7.6.15
वा॒ज॒जि॒त्यायै॒ सं मार्ज्मि। अ॒ग्निम॑न्ना॒दम॒न्नाद्या॑य। उप॑हूतो॒ द्यौः पि॒ता। उप॒ मान्द्यौः पि॒ता ह्व॑यताम्। अ॒ग्निराग्नीध्रात्। आयु॑षे॒ वर्च॑से। जी॒वात्वै पुण्या॑य। उप॑हूता पृथि॒वी मा॒ता। उप॒ मां मा॒ता पृ॑थि॒वी ह्व॑यताम्। अ॒ग्निराग्नीध्रात्॥६९॥

%3.7.6.16
आयु॑षे॒ वर्च॑से। जी॒वात्वै पुण्या॑य। मनो॒ ज्योति॑र्जुषता॒माज्यम्। विच्छि॑न्नं य॒ज्ञ समि॒मन्द॑धातु। बृह॒स्पति॑स्तनुतामि॒मन्न॑। विश्वे॑ दे॒वा इ॒ह मा॑दयन्ताम्। यन्ते॑ अग्न आवृ॒श्चामि॑। अ॒हं वा क्षिपि॒तश्चर\sn{}। प्र॒जां च॒ तस्य॒ मूलं॑ च। नी॒चैर्दे॑वा॒ नि वृ॑श्चत॥७०॥

%3.7.6.17
अग्ने॒ यो नो॑ऽभि॒दास॑ति। स॒मा॒नो यश्च॒ निष्ट्य॑। इ॒ध्मस्ये॑व प्र॒क्षाय॑तः। मा तस्योच्छे॑षि॒ किञ्च॒न। यो मान्द्वेष्टि॑ जातवेदः। यञ्चा॒हन्द्वेष्मि॒ यश्च॒ माम्। सर्वा॒ स्तान॑ग्ने॒ सन्द॑ह। या श्चा॒हन्द्वेष्मि॒ ये च॒ माम्। अग्ने॑ वाजजित्। वाज॑न्त्वा ससृ॒वासम्॥७१॥

%3.7.6.18
वाजं॑ जिगि॒वासम्। वा॒जिनं॑ वाज॒जितम्। वा॒ज॒जि॒त्यायै॒ सम्मार्ज्मि। अ॒ग्निम॑न्ना॒दम॒न्नाद्या॑य। वेदि॑र्ब॒र्॒हिः शृ॒त ह॒विः। इ॒ध्मः प॑रि॒धय॒ स्रुच॑। आज्यं॑ य॒ज्ञ ऋचो॒ यजु॑। या॒ज्याश्च वषट्का॒राः। सम्मे॒ सन्न॑तयो नमन्ताम्। इ॒ध्म॒स॒न्नह॑ने हु॒ते॥७२॥

%3.7.6.19
दि॒वः खीलोऽव॑ततः। पृ॒थि॒व्या अध्युत्थि॑तः। तेना॑ स॒हस्र॑काण्डेन। द्वि॒षन्त शोचयामसि। द्वि॒षन्मे॑ ब॒हु शो॑चतु। ओष॑धे॒ मो अ॒ह शु॑चम्। यज्ञ॒ नम॑स्ते यज्ञ। नमो॒ नम॑श्च ते यज्ञ। शि॒वेन॑ मे॒ सन्ति॑ष्ठस्व। स्यो॒नेन॑ मे॒ सन्ति॑ष्ठस्व॥७३॥

%3.7.6.20
सु॒भू॒तेन॑ मे॒ सन्ति॑ष्ठस्व। ब्र॒ह्म॒व॒र्च॒सेन॑ मे॒ सन्ति॑ष्ठस्व। य॒ज्ञस्यर्द्धि॒मनु॒ सन्ति॑ष्ठस्व। उप॑ ते यज्ञ॒ नम॑। उप॑ ते॒ नम॑। उप॑ ते॒ नम॑। त्रिष्फ॒लीक्रि॒यमा॑णानाम्। यो न्य॒ङ्गो अ॑व॒शिष्य॑ते। रक्ष॑सां भाग॒धेयम्। आप॒स्तत्प्र व॑हतादि॒तः॥७४॥

%3.7.6.21
उ॒लूख॑ले॒ मुस॑ले॒ यच्च॒ शूर्पे। आ॒शि॒श्लेष॑ दृ॒षदि॒ यत्क॒पाले। अ॒व॒प्रुषो॑ वि॒प्रुष॒ संय॑जामि। विश्वे॑ दे॒वा ह॒विरि॒दं जु॑षन्ताम्। य॒ज्ञे या वि॒प्रुष॒ सन्ति॑ ब॒ह्वीः। अ॒ग्नौ ताः सर्वा॒ स्वि॑ष्टा॒ सुहु॑ता जुहोमि। उ॒द्यन्न॒द्यमि॑त्र महः। स॒पत्नान्मे अनीनशः। दिवै॑नान् वि॒द्युता॑ जहि। नि॒म्रोच॒न्नध॑रान्कृधि॥७५॥

%3.7.6.22
उ॒द्यन्न॒द्य वि नो॑ भज। पि॒ता पु॒त्रेभ्यो॒ यथा। दी॒र्घा॒यु॒त्वस्य॑ हेशिषे। तस्य॑ नो देहि सूर्य। उ॒द्यन्न॒द्य मि॑त्रमहः। आ॒रोह॒न्नुत्त॑रा॒न्दिवम्। हृ॒द्रो॒गम्मम॑ सूर्य। ह॒रि॒माणं॑ च नाशय। शुके॑षु मे हरि॒माणम्। रो॒प॒णाका॑सु दध्मसि ॥७६॥

%3.7.6.23
अथो॑ हारिद्र॒वेषु॑ मे। ह॒रि॒माणं॒ नि द॑ध्मसि। उद॑गाद॒यमा॑दि॒त्यः। विश्वे॑न॒ सह॑सा स॒ह। द्वि॒षन्तं॒ मम॑ र॒न्धय\sn{}। मो अ॒हन्द्वि॑ष॒तो र॑धम्। यो न॒ शपा॒दश॑पतः। यश्च॑ न॒ शप॑त॒ शपात्। उ॒षाश्च॒ तस्मै॑ नि॒म्रुक्च॑। सर्वं॑ पा॒प समू॑हताम्॥७७॥

%3.7.6.24
यो न॑ स॒पत्नो॒ यो रण॑। मर्तो॑ऽभि॒दास॑ति देवाः। इ॒ध्मस्ये॑व प्र॒क्षाय॑तः। मा तस्योच्छे॑षि॒ किञ्च॒न। अव॑सृष्ट॒ परा॑पत। श॒रो ब्रह्म॑सशितः। गच्छा॒ऽमित्रा॒न्प्र वि॑श। मैषा॒ङ्कञ्च॒नोच्छि॑षः॥७८॥\anuvakamend[पति॑ प्र॒जाप॑तये तप॒स्वी वा॒चा सौभ॑गाय प॒शून्मे॑ पिन्वस्व दुर्मरा॒युं दे॑व॒याना॑नग्ने॒ऽन्तरि॑क्षे॒ऽहमुत्त॑रो भूयासं प्र॒जाप॑तिरसि स॒र्वत॑ श्रि॒तः प्रवि॑ष्टन्दे॒वता॑भिर्वाज॒जितं॑ पृथि॒वी ह्व॑यताम॒ग्निराग्नीध्राद्वृश्चत ससृ॒वास हु॒ते स्यो॒नेन॑ मे॒ सन्ति॑ष्ठस्वे॒तः कृ॑धि दध्मस्यूहताम॒ष्टौ च॑]

%3.7.7.1
सक्षे॒दं प॑श्य। विध॑र्तरि॒दं प॑श्य। नाके॒दं प॑श्य। र॒मति॒ पनि॑ष्ठा। ऋ॒तं वर्‌षि॑ष्ठम्। अ॒मृता॒यान्या॒हुः। सूर्यो॒ वरि॑ष्ठो अ॒क्षभि॒र्विभा॑ति। अनु॒ द्यावा॑पृथि॒वी दे॒वपु॑त्रे। दी॒क्षाऽसि॒ तप॑सो॒ योनि॑। तपो॑ऽसि॒ ब्रह्म॑णो॒ योनि॑ ॥७९॥

%3.7.7.2
ब्रह्मा॑सि क्ष॒त्रस्य॒ योनि॑। क्ष॒त्रम॑स्यृ॒तस्य॒ योनि॑। ऋ॒तम॑सि॒ भूरा र॑भे। श्र॒द्धां मन॑सा। दी॒क्षान्तप॑सा। विश्व॑स्य॒ भुव॑न॒स्याधि॑पत्नीम्। सर्वे॒ कामा॒ यज॑मानस्य सन्तु। वातं॑ प्रा॒णं मन॑सा॒ऽन्वा र॑भामहे। प्र॒जाप॑ति॒य्योँ भुव॑नस्य गो॒पाः। स नो॑ मृ॒त्योस्त्रा॑यतां॒ पात्वह॑सः॥८०॥

%3.7.7.3
ज्योग्जी॒वा ज॒राम॑शीमहि। इन्द्र॑ शाक्वर गाय॒त्रीं प्र प॑द्ये। तान्ते॑ युनज्मि। इन्द्र॑ शाक्वर त्रि॒ष्टुभं॒ प्र प॑द्ये। तान्ते॑ युनज्मि। इन्द्र॑ शाक्वर॒ जग॑तीं॒ प्र प॑द्ये। तान्ते॑ युनज्मि। इन्द्र॑ शाक्वरानु॒ष्टुभं॒ प्र प॑द्ये। तान्ते॑ युनज्मि। इन्द्र॑ शाक्वर प॒ङ्क्तिं प्रप॑द्ये॥८१॥

%3.7.7.4
तान्ते॑ युनज्मि। आऽहन्दी॒क्षाम॑रुहमृ॒तस्य॒ पत्नीम्। गा॒य॒त्रेण॒ छन्द॑सा॒ ब्रह्म॑णा च। ऋ॒त स॒त्ये॑ऽधायि। स॒त्यमृ॒ते॑ऽधायि। ऋ॒तं च॑ मे स॒त्यञ्चा॑भूताम्। ज्योति॑रभूव॒ सुव॑रगमम्। सु॒व॒र्गं लो॒कं नाक॑स्य पृ॒ष्ठम्। ब्र॒ध्नस्य॑ वि॒ष्टप॑मगमम्। पृ॒थि॒वी दी॒क्षा॥८२॥

%3.7.7.5
तया॒ऽग्निर्दी॒क्षया॑ दीक्षि॒तः। यया॒ऽग्निर्दी॒क्षया॑ दीक्षि॒तः। तया त्वा दी॒क्षया॑ दीक्षयामि। अ॒न्तरि॑क्षन्दी॒क्षा। तया॑ वा॒युर्दी॒क्षया॑ दीक्षि॒तः। यया॑ वा॒युर्दी॒क्षया॑ दीक्षि॒तः। तया त्वा दी॒क्षया॑ दीक्षयामि। द्यौर्दी॒क्षा। तया॑ऽऽदि॒त्यो दी॒क्षया॑ दीक्षि॒तः। यया॑ऽऽदि॒त्यो दी॒क्षया॑ दीक्षि॒तः॥८३॥

%3.7.7.6
तया त्वा दी॒क्षया॑ दीक्षयामि। दिशो॑ दी॒क्षा। तया॑ च॒न्द्रमा॑ दी॒क्षया॑ दीक्षि॒तः। यया॑ च॒न्द्रमा॑ दी॒क्षया॑ दीक्षि॒तः। तया त्वा दी॒क्षया॑ दीक्षयामि। आपो॑ दी॒क्षा। तया॒ वरु॑णो॒ राजा॑ दी॒क्षया॑ दीक्षि॒तः। यया॒ वरु॑णो॒ राजा॑ दी॒क्षया॑ दीक्षि॒तः। तया त्वा दी॒क्षया॑ दीक्षयामि। ओष॑धयो दी॒क्षा॥८४॥

%3.7.7.7
तया॒ सोमो॒ राजा॑ दी॒क्षया॑ दीक्षि॒तः। यया॒ सोमो॒ राजा॑ दी॒क्षया॑ दीक्षि॒तः। तया त्वा दी॒क्षया॑ दीक्षयामि। वाग्दी॒क्षा। तया प्रा॒णो दी॒क्षया॑ दीक्षि॒तः। यया प्रा॒णो दी॒क्षया॑ दीक्षि॒तः। तया त्वा दी॒क्षया॑ दीक्षयामि। पृ॒थि॒वी त्वा॒ दीक्ष॑माण॒मनु॑ दीक्षताम्। अ॒न्तरि॑क्षन्त्वा॒ दीक्ष॑माण॒मनु॑ दीक्षताम्। द्यौस्त्वा॒ दीक्ष॑माण॒मनु॑ दीक्षताम्॥८५॥

%3.7.7.8
दिश॑स्त्वा॒ दीक्ष॑माण॒मनु॑ दीक्षन्ताम्। आप॑स्त्वा॒ दीक्ष॑माण॒मनु॑ दीक्षन्ताम्। ओष॑धयस्त्वा॒ दीक्ष॑माण॒मनु॑ दीक्षन्ताम्। वाक्त्वा॒ दीक्ष॑माण॒मनु॑ दीक्षताम्। ऋच॑स्त्वा॒ दीक्ष॑माण॒मनु॑ दीक्षन्ताम्। सामा॑नि त्वा॒ दीक्ष॑माण॒मनु॑ दीक्षन्ताम्। यजूषि त्वा॒ दीक्ष॑माण॒मनु॑ दीक्षन्ताम्। अह॑श्च॒ रात्रि॑श्च। कृ॒षिश्च॒ वृष्टि॑श्च। त्विषि॒श्चाप॑चितिश्च॥८६॥

%3.7.7.9
अप॒श्चौष॑धयश्च। ऊर्क्च॑ सू॒नृता॑ च। तास्त्वा॒ दीक्ष॑माण॒मनु॑ दीक्षन्ताम्। स्वे दक्षे॒ दक्ष॑पिते॒ह सी॑द। दे॒वाना सु॒म्नो म॑ह॒ते रणा॑य। स्वा॒स॒स्थस्त॒नुवा॒ संवि॑शस्व। पि॒तेवै॑धि सू॒नव॒ आ सु॒शेव॑। शि॒वो मा॑ शि॒वमा वि॑श। स॒त्यम्म॑ आ॒त्मा। श्र॒द्धा मेऽक्षि॑तिः॥८७॥

%3.7.7.10
तपो॑ मे प्रति॒ष्ठा। स॒वि॒तृप्र॑सूता मा॒ दिशो॑ दीक्षयन्तु। स॒त्यम॑स्मि। अ॒हन्त्वद॑स्मि॒ मद॑सि॒ त्वमे॒तत्। ममा॑सि॒ योनि॒स्तव॒ योनि॑रस्मि। ममै॒व सन्वह॑ ह॒व्यान्य॑ग्ने। पु॒त्रः पि॒त्रे लो॑क॒कृज्जा॑तवेदः। आ॒जुह्वा॑नः सु॒प्रती॑कः पु॒रस्तात्। अग्ने॒ स्वाय्योँनि॒मा सी॑द सा॒ध्या। अ॒स्मिन्त्स॒धस्थे॒ अध्युत्त॑रस्मिन्॥८८॥

%3.7.7.11
विश्वे॑ देवा॒ यज॑मानश्च सीदत। एक॑मि॒षे विष्णु॒स्त्वाऽन्वे॑तु। द्वे ऊ॒र्जे विष्णु॒स्त्वाऽन्वे॑तु। त्रीणि॑ व्र॒ताय॒ विष्णु॒स्त्वाऽन्वे॑तु। च॒त्वारि॒ मायो॑भवाय॒ विष्णु॒स्त्वाऽन्वे॑तु। पञ्च॑ प॒शुभ्यो॒ विष्णु॒स्त्वाऽन्वे॑तु। षड्रा॒यस्पोषा॑य॒ विष्णु॒स्त्वाऽन्वे॑तु। स॒प्त स॒प्तभ्यो॒ होत्राभ्यो॒ विष्णु॒स्त्वाऽन्वे॑तु। सखा॑यः स॒प्तप॑दा अभूम। स॒ख्यन्ते॑ गमेयम् ॥८९॥

%3.7.7.12
स॒ख्यात्ते॒ मा यो॑षम्। स॒ख्यान्मे॒ मा योष्ठाः। साऽसि॑ सुब्रह्मण्ये। तस्यास्ते पृथि॒वी पाद॑। साऽसि॑ सुब्रह्मण्ये। तस्यास्ते॒ऽन्तरि॑क्षं॒ पाद॑। साऽसि॑ सुब्रह्मण्ये। तस्यास्ते॒ द्यौः पाद॑। साऽसि॑ सुब्रह्मण्ये। तस्यास्ते॒ दिश॒ पाद॑॥९०॥

%3.7.7.13
प॒रोर॑जास्ते पञ्च॒मः पाद॑। सा न॒ इष॒मूर्ज॑न्धुक्ष्व। तेज॑ इन्द्रि॒यम्। ब्र॒ह्म॒व॒र्च॒सम॒न्नाद्यम्। वि मि॑मे त्वा॒ पय॑स्वतीम्। दे॒वानान्धे॒नु सु॒दुघा॒मन॑पस्फुरन्तीम्। इन्द्र॒ सोमं॑ पिबतु। क्षेमो॑ अस्तु नः। इ॒मान्न॑राः कृणुत॒ वेदि॒मेत्य॑। वसु॑मती रु॒द्रव॑तीमादि॒त्यव॑तीम्॥९१॥

%3.7.7.14
वर्ष्म॑न्दि॒वः। नाभा॑ पृथि॒व्याः। यथा॒ऽयं यज॑मानो॒ न रिष्येत्। दे॒वस्य॑ सवि॒तुः स॒वे। चतु॑ शिखण्डा युव॒तिः सु॒पेशा। घृ॒तप्र॑तीका॒ भुव॑नस्य॒ मध्ये। तस्या सुप॒र्णावधि॒ यौ निवि॑ष्टौ। तयोर्दे॒वाना॒मधि॑ भाग॒धेयम्। अ॒प जन्य॑म्भ॒यन्नु॑द। अप॑ च॒क्राणि॑ वर्तय। गृ॒ह सोम॑स्य गच्छतम्। न वा उ॑ वे॒तन्म्रि॑यसे॒ न रि॑ष्यसि। दे॒वा इदे॑षि प॒थिभि॑ सु॒गेभि॑। यत्र॒ यन्ति॑ सु॒कृतो॒ नापि॑ दु॒ष्कृत॑। तत्र॑ त्वा दे॒वः स॑वि॒ता द॑धातु॥९२॥\anuvakamend[ब्रह्म॑णो॒ योनि॒रह॑सः प॒ङ्क्तिं प्रप॑द्ये दी॒क्षा यया॑ऽऽदि॒त्यो दी॒क्षया॑ दीक्षि॒तस्तया त्वा दी॒क्षया॑ दीक्षया॒म्योष॑धयो दी॒क्षा द्यौस्त्वा॒ दीक्ष॑माण॒मनु॑ दीक्षता॒मप॑चिति॒श्चाक्षि॑ति॒रुत्त॑रस्मिन्गमेयं॒ दिश॒ पाद॑ आदि॒त्यव॑तीं वर्तय॒ पञ्च॑ च]

%3.7.8.1
यद॒स्य पा॒रे रज॑सः। शु॒क्रञ्ज्योति॒रजा॑यत। तन्न॑ पर्‌ष॒दति॒ द्विष॑। अग्ने॑ वैश्वानर॒ स्वाहा। यस्माद्भी॒षाऽवा॑शिष्ठाः। ततो॑ नो॒ अभ॑यङ्कृधि। प्र॒जाभ्य॒ सर्वाभ्यो मृड। नमो॑ रु॒द्राय॑ मी॒ढुषे। यस्माद्भी॒षा न्यष॑दः। ततो॑ नो॒ अभ॑यङ्कृधि॥९३॥

%3.7.8.2
प्र॒जाभ्य॒ सर्वाभ्यो मृड। नमो॑ रु॒द्राय॑ मी॒ढुषे। उदु॑स्र तिष्ठ॒ प्रति॑तिष्ठ॒ मारि॑षः। मेमं य॒ज्ञं यज॑मानं च रीरिषः। सु॒व॒र्गे लो॒के यज॑मान॒ हि धे॒हि। शन्न॑ एधि द्वि॒पदे॒ शञ्चतु॑ष्पदे। यस्माद्भी॒षाऽवे॑पिष्ठाः प॒लायि॑ष्ठाः स॒मज्ञास्थाः। ततो॑ नो॒ अभ॑यङ्कृधि। प्र॒जाभ्य॒ सर्वाभ्यो मृड। नमो॑ रु॒द्राय॑ मी॒ढुषे॥९४॥

%3.7.8.3
य इ॒दमक॑। तस्मै॒ नम॑। तस्मै॒ स्वाहा। न वा उ॑वे॒तन्म्रि॑यसे। आशा॑नान्त्वा॒ विश्वा॒ आशा। य॒ज्ञस्य॒ हि स्थ ऋ॒त्वियौ। इन्द्राग्नी॒ चेत॑नस्य च। हु॒ता॒हु॒तस्य॑ तृप्यतम्। अहु॑तस्य हु॒तस्य॑ च। हु॒तस्य॒ चाहु॑तस्य च। अहु॑तस्य हु॒तस्य॑ च। इन्द्राग्नी अ॒स्य सोम॑स्य। वी॒तं पि॑बतं जु॒षेथाम्। मा यज॑मान॒न्तमो॑ विदत्। मर्त्विजो॒ मो इ॒माः प्र॒जाः। मा यः सोम॑मि॒मं पिबात्। ससृ॑ष्टमु॒भयं॑ कृ॒तम्॥९५॥\anuvakamend[कृ॒धि॒ मी॒ढुषेऽहु॑तस्य च स॒प्त च॑]

%3.7.9.1
अ॒ना॒गस॑स्त्वा व॒यम्। इन्द्रे॑ण॒ प्रेषि॑ता॒ उप॑। वा॒युष्टे॑ अस्त्वश॒भूः। मि॒त्रस्ते॑ अस्त्वश॒भूः। वरु॑णस्ते अस्त्वश॒भूः। अपांक्षया॒ ऋत॑स्य गर्भाः। भुव॑नस्य गोपा॒ श्येना॑ अतिथयः। पर्व॑तानाङ्ककुभः प्र॒युतो॑ नपातारः। व॒ग्नुनेन्द्र ह्वयत। घोषे॒णामी॑वा श्चातयत॥९६॥

%3.7.9.2
यु॒क्ताः स्थ॒ वह॑त। दे॒वा ग्रावा॑ण॒ इन्दु॒रिन्द्र॒ इत्य॑वादिषुः। एन्द्र॑मचुच्यवुः पर॒मस्या परा॒वत॑। आऽस्मात्स॒धस्थात्। ओरोर॒न्तरि॑क्षात्। आ सु॑भू॒तम॑सुषवुः। ब्र॒ह्म॒व॒र्च॒सम्म॒ आसु॑षवुः। स॒म॒रे रक्षास्यवधिषुः। अप॑हतं ब्रह्म॒ज्यस्य॑। वाक्च॑ त्वा॒ मन॑श्च श्रीणीताम्॥९७॥

%3.7.9.3
प्रा॒णश्च॑ त्वाऽपा॒नश्च॑ श्रीणीताम्। चक्षु॑श्च त्वा॒ श्रोत्रं॑ च श्रीणीताम्। दक्ष॑श्चत्वा॒ बलं॑ च श्रीणीताम्। ओज॑श्च त्वा॒ सह॑श्च श्रीणीताम्। आयु॑श्च त्वाऽज॒रा च॑ श्रीणीताम्। आ॒त्मा च॑ त्वा त॒नूश्च॑ श्रीणीताम्। शृ॒तो॑ऽसि शृ॒तं कृ॑तः। शृ॒ताय॑ त्वा शृ॒तेभ्य॑स्त्वा। यमिन्द्र॑मा॒हुर्वरु॑णं॒ यमा॒हुः। यम्मि॒त्रमा॒हुर्यमु॑ स॒त्यमा॒हुः॥९८॥

%3.7.9.4
यो दे॒वानान्दे॒वत॑मस्तपो॒जाः। तस्मै त्वा॒ तेभ्य॑स्त्वा। मयि॒ त्यदि॑न्द्रि॒यम्मह॒त्। मयि॒ दक्षो॒ मयि॒ क्रतु॑। मयि॑ धायि सु॒वीर्यम्। त्रिशु॑ग्घ॒र्मो वि भा॑तु मे। आकूत्या॒ मन॑सा स॒ह। वि॒राजा॒ ज्योति॑षा स॒ह। य॒ज्ञेन॒ पय॑सा स॒ह। तस्य॒ दोह॑मशीमहि॥९९॥

%3.7.9.5
तस्य॑ सु॒म्नम॑शीमहि। तस्य॑ भ॒क्षम॑शीमहि। वाग्जु॑षा॒णा सोम॑स्य तृप्यतु। मि॒त्रो जना॒न्प्र स मि॑त्र। यस्मा॒न्न जा॒तः परो॑ अ॒न्यो अस्ति॑। य आ॑वि॒वेश॒ भुव॑नानि॒ विश्वा। प्र॒जाप॑तिः प्र॒जया॑ संविदा॒नः। त्रीणि॒ ज्योतीषि सचते॒ स षो॑ड॒शी। ए॒ष ब्र॒ह्मा य ऋ॒त्विय॑। इन्द्रो॒ नाम॑ श्रु॒तो ग॒णे॥१००॥

%3.7.9.6
प्र ते॑ म॒हे वि॒दथे॑ शसिष॒ हरी। य ऋ॒त्विय॒ प्र ते॑ वन्वे। व॒नुषो॑ हर्य॒तम्मदम्। इन्द्रो॒ नाम॑ घृ॒तन्नयः। हरि॑भि॒श्चारु॒ सेच॑ते। श्रु॒तो ग॒ण आ त्वा॑ विशन्तु। हरि॑वर्पस॒ङ्गिर॑। इन्द्राधि॑प॒तेऽधि॑पति॒स्त्वन्दे॒वाना॑मसि। अधि॑पति॒म्माम्। आयु॑ष्मन्तं॒ वर्च॑स्वन्तम्मनु॒ष्ये॑षु कुरु॥१०१॥

%3.7.9.7
इन्द्र॑श्च स॒म्राड्वरु॑णश्च॒ राजा। तौ ते॑ भ॒क्षं च॑क्रतु॒रग्र॑ ए॒तम्। तयो॒रनु॑ भ॒क्षं भ॑क्षयामि। वाग्जु॑षा॒णा सोम॑स्य तृप्यतु। प्र॒जाप॑तिर्वि॒श्वक॑र्मा। तस्य॒ मनो॑ दे॒वं य॒ज्ञेन॑ राध्यासम्। अ॒र्थे॒गा अ॒स्य ज॑हितः। अ॒व॒सान॑पतेऽव॒सान॑म्मे विन्द। नमो॑ रु॒द्राय॑ वास्तो॒ष्पत॑ये। आय॑ने वि॒द्रव॑णे॥१०२॥

%3.7.9.8
उ॒द्याने॒ यत्प॒राय॑णे। आ॒वर्त॑ने वि॒वर्त॑ने। यो गो॑पा॒यति॒ त हु॑वे। यान्य॑पा॒मित्या॒न्यप्र॑तीत्ता॒न्यस्मि॑। य॒मस्य॑ ब॒लिना॒ चरा॑मि। इ॒हैव सन्त॒ प्रति॒ तद्या॑तयामः। जी॒वा जी॒वेभ्यो॒ नि ह॑राम एनत्। अ॒नृ॒णा अ॒स्मिन्न॑नृ॒णाः पर॑स्मिन्। तृ॒तीये॑ लो॒के अ॑नृ॒णाः स्या॑म। ये दे॑व॒याना॑ उ॒त पि॑तृ॒याणा॥१०३॥

%3.7.9.9
सर्वान्प॒थो अ॑नृ॒णा आक्षी॑येम। इ॒दमू॒नु श्रेयो॑ऽव॒सान॒मा ग॑न्म। शि॒वे नो॒ द्यावा॑पृथि॒वी उ॒भे इ॒मे। गोम॒द्धन॑व॒दश्व॑व॒दूर्ज॑स्वत्। सु॒वीरा॑ वी॒रैरनु॒ सञ्च॑रेम। अ॒र्कः प॒वि॒त्र॒ रज॑सो वि॒मान॑। पु॒नाति॑ दे॒वाना॒म्भुव॑नानि॒ विश्वा। द्यावा॑पृथि॒वी पय॑सा संविदा॒ने। घृ॒तन्दु॑हाते अ॒मृतं॒ प्रपी॑ने। प॒वित्र॑म॒र्को रज॑सो वि॒मान॑। पु॒नाति॑ दे॒वाना॒म्भुव॑नानि॒ विश्वा। सुव॒र्ज्योति॒र्यशो॑ म॒हत्। अ॒शी॒महि॑ गा॒धमु॒त प्र॑ति॒ष्ठाम्॥१०४॥\anuvakamend[चा॒त॒य॒त॒ श्री॒णी॒ता॒ स॒त्यमा॒हुर॑शीमहि ग॒णे कु॑रु वि॒द्रव॑णे पितृ॒याणा॑ अ॒र्को रज॑सो वि॒मान॒स्त्रीणि॑ च]

%3.7.10.1
उद॑स्तांप्सीत्सवि॒ता मि॒त्रो अ॑र्य॒मा। सर्वा॑न॒मित्रा॑नवधीद्यु॒गेन॑। बृ॒हन्त॒म्माम॑करद्वी॒रव॑न्तम्। र॒थ॒न्त॒रे श्र॑यस्व॒ स्वाहा॑ पृथि॒व्याम्। वा॒म॒दे॒व्ये श्र॑यस्व॒ स्वाहा॒ऽन्तरि॑क्षे। बृ॒ह॒ति श्र॑यस्व॒ स्वाहा॑ दि॒वि। बृ॒ह॒ता त्वोप॑स्तभ्नोमि। आ त्वा॑ ददे॒ यश॑से वी॒र्या॑य च। अ॒स्मास्व॑घ्निया यू॒यन्द॑धाथेन्द्रि॒यं पय॑। यस्ते द्र॒प्सो यस्त॑ उद॒र्॒षः ॥१०५॥

%3.7.10.2
दैव्य॑ के॒तुर्विश्व॒म्भुव॑नमावि॒वेश॑। स न॑ पा॒ह्यरि॑ष्ट्यै॒ स्वाहा। अनु॑ मा॒ सर्वो॑ य॒ज्ञो॑ऽयमे॑तु। विश्वे॑ दे॒वा म॒रुत॒ सामा॒र्कः। आ॒प्रिय॒श्छन्दासि नि॒विदो॒ यजूषि। अ॒स्यै पृ॑थि॒व्यै यद्य॒ज्ञियम्। प्र॒जाप॑तेर्वर्त॒निमनु॑ वर्तस्व। अनु॑वी॒रैरनु॑ राध्याम॒ गोभि॑। अन्वश्वै॒रनु॒ सर्वै॑रु पु॒ष्टैः। अनु॑ प्र॒जयाऽन्वि॑न्द्रि॒येण॑॥१०६॥

%3.7.10.3
दे॒वा नो॑ य॒ज्ञमृ॑जु॒धा न॑यन्तु। प्रति॑क्ष॒त्रे प्रति॑तिष्ठामि रा॒ष्ट्रे। प्रत्यश्वे॑षु॒ प्रति॑तिष्ठामि॒ गोषु॑। प्रति॑ प्र॒जायां॒ प्रति॑तिष्ठामि॒ भव्ये। विश्व॑म॒न्याऽभि॑ वावृ॒धे। तद॒न्यस्या॒मधि॑श्रि॒तम्। दि॒वे च॑ वि॒श्वक॑र्मणे। पृ॒थि॒व्यै चा॑कर॒न्नम॑। अस्का॒न्द्यौः पृ॑थि॒वीम्। अस्का॑नृष॒भो युवा॒गाः॥१०७॥

%3.7.10.4
स्क॒न्नेमा विश्वा॒ भुव॑ना। स्क॒न्नो य॒ज्ञः प्र ज॑नयतु। अस्का॒नज॑नि॒ प्राज॑नि। आ स्क॒न्नाज्जा॑यते॒ वृषा। स्क॒न्नात्प्र ज॑निषीमहि। ये दे॒वा येषा॑मि॒दम्भा॑ग॒धेय॑म्ब॒भूव॑। येषां प्रया॒जा उ॒तानू॑या॒जाः। इन्द्र॑ज्येष्ठेभ्यो॒ वरु॑णराजभ्यः। अ॒ग्निहो॑तृभ्यो दे॒वेभ्य॒ स्वाहा। उ॒त त्या नो॒ दिवा॑ म॒तिः॥१०८॥

%3.7.10.5
अदि॑तिरू॒त्या ग॑मत्। सा शन्ता॑ची॒ मय॑स्करत्। अप॒ स्रिध॑। उ॒त त्या दैव्या॑ भि॒षजा। शन्न॑स्करतो अ॒श्विना। यू॒याता॑म॒स्मद्रप॑। अप॒ स्रिध॑। शम॒ग्निर॒ग्निभि॑स्करत्। शन्न॑स्तपतु॒ सूर्य॑। शं वातो॑ वात्वर॒पाः॥१०९॥

%3.7.10.6
अप॒ स्रिध॑। तदित्प॒दन्न विचि॑केत वि॒द्वान्। यन्मृ॒तः पुन॑र॒प्येति॑ जी॒वान्। त्रि॒वृद्यद्भुव॑नस्य रथ॒वृत्। जी॒वो गर्भो॒ न मृ॒तः स जी॑वात्। प्रत्य॑स्मै॒ पिपी॑षते। विश्वा॑नि वि॒दुषे॑ भर। अ॒र॒ङ्ग॒माय॒ जग्म॑वे। अप॑श्चाद्दघ्वने॒ नरे। इन्दु॒रिन्दु॒मवा॑गात्। इन्दो॒रिन्द्रो॑ऽपात्। तस्य॑ त इन्द॒विन्द्र॑पीतस्य॒ मधु॑मतः। उप॑हूत॒स्योप॑हूतो भक्षयामि ॥११०॥\anuvakamend[उ॒द॒र्॒ष इ॑न्द्रि॒येण॒ गा म॒तिर॑र॒पा अ॑गा॒त्रीणि॑ च]

%3.7.11.1
ब्रह्म॑ प्रति॒ष्ठा मन॑सो॒ ब्रह्म॑ वा॒चः। ब्रह्म॑ य॒ज्ञाना ह॒विषा॒माज्य॑स्य। अति॑रिक्त॒ङ्कर्म॑णो॒ यच्च॑ ही॒नम्। य॒ज्ञः पर्वा॑णि प्रति॒रन्ने॑ति क॒ल्पय\sn{}। स्वाहा॑कृ॒ताऽऽहु॑तिरेतु दे॒वान्। आश्रा॑वितम॒त्याश्रा॑वितम्। वष॑ट्कृतम॒त्यनूक्तं च य॒ज्ञे। अति॑रिक्त॒ङ्कर्म॑णो॒ यच्च॑ ही॒नम्। य॒ज्ञः पर्वा॑णि प्रति॒रन्ने॑ति क॒ल्पय\sn{}। स्वाहा॑कृ॒ताऽऽहु॑तिरेतु दे॒वान्॥१११॥

%3.7.11.2
यद्वो॑ देवा अतिपा॒दया॑नि। वा॒चा चि॒त्प्रय॑तन्देव॒हेड॑नम्। अ॒रा॒यो अ॒स्मा अ॒भिदु॑च्छुना॒यते। अ॒न्यत्रा॒स्मन्म॑रुत॒स्तन्निधे॑तन। त॒तम्म॒ आप॒स्तदु॑ तायते॒ पुन॑। स्वादि॑ष्ठा धी॒तिरु॒चथा॑य शस्यते। अ॒य स॑मु॒द्र उ॒त वि॒श्वभे॑षजः। स्वाहा॑कृतस्य॒ समु॑तृप्णुतर्भुवः। उद्व॒यन्तम॑स॒स्परि॑। उदु॒त्यञ्चि॒त्रम्॥११२॥

%3.7.11.3
इ॒मम्मे॑ वरुण॒ तत्त्वा॑ यामि। त्वन्नो॑ अग्ने॒ स त्वन्नो॑ अग्ने। त्वम॑ग्ने अ॒यासि॒ प्रजा॑पते। इ॒मञ्जी॒वेभ्य॑ परि॒धिन्द॑धामि। मैषान्नु॑गा॒दप॑रो॒ अर्ध॑मे॒तम्। श॒तञ्जी॑वन्तु श॒रद॑ पुरू॒चीः। ति॒रो मृ॒त्युन्द॑धतां॒ पर्व॑तेन। इ॒ष्टेभ्य॒ स्वाहा॒ वष॒डनि॑ष्टेभ्य॒ स्वाहा। भे॒ष॒जन्दुरि॑ष्ट्यै॒ स्वाहा॒ निष्कृ॑त्यै॒ स्वाहा। दौरार्ध्यै॒ स्वाहा॒ दैवीभ्यस्त॒नूभ्य॒ स्वाहा॥११३॥

%3.7.11.4
ऋद्ध्यै॒ स्वाहा॒ समृ॑द्ध्यै॒ स्वाहा। यत॑ इन्द्र॒ भया॑महे। ततो॑ नो॒ अभ॑यङ्कृधि। मघ॑वञ्छ॒ग्धि तव॒ तन्न॑ ऊ॒तये। वि द्विषो॒ वि मृधो॑ जहि। स्व॒स्ति॒दा वि॒शस्पति॑। वृ॒त्र॒हा वि मृधो॑ व॒शी। वृषेन्द्र॑ पु॒र ए॑तु नः। स्व॒स्ति॒दा अ॑भयङ्क॒रः। आ॒भिर्गी॒र्भिर्यदतो॑ न ऊ॒नम्॥११४॥

%3.7.11.5
आप्या॑यय हरिवो॒ वर्ध॑मानः। य॒दा स्तो॒तृभ्यो॒ महि॑ गो॒त्रा रु॒जासि॑। भू॒यि॒ष्ठ॒भाजो॒ अध॑ ते स्याम। अनाज्ञातं॒ यदाज्ञा॑तम्। य॒ज्ञस्य॑ क्रि॒यते॒ मिथु॑। अग्ने॒ तद॑स्य कल्पय। त्व हि वेत्थ॑ यथात॒थम्। पुरु॑षसम्मितो य॒ज्ञः। य॒ज्ञः पुरु॑षसम्मितः। अग्ने॒ तद॑स्य कल्पय। त्व हि वेत्थ॑ यथात॒थम्। यत्पा॑क॒त्रा मन॑सा दी॒नद॑क्षा॒ न। य॒ज्ञस्य॑ म॒न्वते॒ मर्ता॑सः। अ॒ग्निष्टद्धोता॑ क्रतु॒विद्वि॑जा॒नन्। यजि॑ष्ठो दे॒वा ऋ॑तु॒शो य॑जाति॥११५॥\anuvakamend[दे॒वा श्चि॒त्रं त॒नूभ्य॒ स्वाहो॒नं पुरु॑षसम्मि॒तोऽग्ने॒ तद॑स्य कल्पय॒ पञ्च॑ च]

%3.7.12.1
यद्दे॑वा देव॒हेड॑नम्। देवा॑सश्चकृ॒मा व॒यम्। आदि॑त्या॒स्तस्मान्मा मुञ्चत। ऋ॒तस्य॒र्तेन॒ मामु॒त। देवा॑ जीवनका॒म्या यत्। वा॒चाऽनृ॑तमूदि॒म। अ॒ग्निर्मा॒ तस्मा॒देन॑सः। गार्‌ह॑पत्य॒ प्रमु॑ञ्चतु। दु॒रि॒ता यानि॑ चकृ॒म। क॒रोतु॒ माम॑ने॒नसम्॥११६॥

%3.7.12.2
ऋ॒तेन॑ द्यावापृथिवी। ऋ॒तेन॒ त्व स॑रस्वति। ऋ॒तान्मा॑ मुञ्च॒ताह॑सः। यद॒न्यकृ॑तमारि॒म। स॒जा॒त॒श॒सादु॒त वा॑ जामिश॒सात्। ज्याय॑स॒ शसा॑दु॒त वा॒ कनी॑यसः। अनाज्ञातन्दे॒वकृ॑तं॒ यदेन॑। तस्मा॒त्त्वम॒स्माञ्जा॑तवेदो मुमुग्धि। यद्वा॒चा यन्मन॑सा। बा॒हुभ्या॑मू॒रुभ्या॑मष्ठी॒वद्भ्याम्॥११७॥

%3.7.12.3
शि॒श्ञैर्यदनृ॑तञ्चकृ॒मा व॒यम्। अ॒ग्निर्मा॒ तस्मा॒देन॑सः। यद्धस्ताभ्याञ्च॒कर॒ किल्बि॑षाणि। अ॒क्षाणां व॒ग्नुमु॑प॒जिघ्न॑मानः। दू॒रे॒प॒श्या च॑ राष्ट्र॒भृच्च॑। तान्य॑प्स॒रसा॒वनु॑दत्तामृ॒णानि॑। अदी॑व्यन्नृ॒णं यद॒हञ्च॒कार॑। यद्वादास्यन्त्सञ्ज॒गारा॒ जनेभ्यः। अ॒ग्निर्मा॒ तस्मा॒देन॑सः। यन्मयि॑ मा॒ता गर्भे॑ स॒ति॥११८॥

%3.7.12.4
एन॑श्च॒कार॒ यत्पि॒ता। अ॒ग्निर्मा॒ तस्मा॒देन॑सः। यदा॑ पि॒पेष॑ मा॒तरं॑ पि॒तरम्। पु॒त्रः प्रमु॑दितो॒ धय\sn{}। अहिसितौ पि॒तरौ॒ मया॒ तत्। तद॑ग्ने अनृ॒णो भ॑वामि। यद॒न्तरि॑क्षं पृथि॒वीमु॒त द्याम्। यन्मा॒तरं॑ पि॒तरं॑ वा जिहिसि॒म। अ॒ग्निर्मा॒ तस्मा॒देन॑सः। यदा॒शसा॑ नि॒शसा॒ यत्प॑रा॒शसा॥११९॥

%3.7.12.5
यदेन॑श्चकृ॒मा नूत॑नं॒ यत्पु॑रा॒णम्। अ॒ग्निर्मा॒ तस्मा॒देन॑सः। अति॑ क्रामामि दुरि॒तं यदेन॑। जहा॑मि रि॒प्रं प॑र॒मे स॒धस्थे। यत्र॒ यन्ति॑ सु॒कृतो॒ नापि॑ दु॒ष्कृत॑। तमा रो॑हामि सु॒कृता॒न्नु लो॒कम्। त्रि॒ते दे॒वा अ॑मृजतै॒तदेन॑। त्रि॒त ए॒तन्म॑नु॒ष्ये॑षु मामृजे। ततो॑ मा॒ यदि॒ किञ्चि॑दान॒शे। अ॒ग्निर्मा॒ तस्मा॒देन॑सः॥१२०॥

%3.7.12.6
गार्‌ह॑पत्य॒ प्र मु॑ञ्चतु। दु॒रि॒ता यानि॑ चकृ॒म। क॒रोतु॒ माम॑ने॒नसम्। दि॒वि जा॒ता अ॒प्सु जा॒ताः। या जा॒ता ओष॑धीभ्यः। अथो॒ या अ॑ग्नि॒जा आप॑। ता न॑ शुन्धन्तु॒ शुन्ध॑नीः। यदापो॒ नक्त॑न्दुरि॒तञ्चरा॑म। यद्वा॒ दिवा॒ नूत॑नं॒ यत्पु॑रा॒णम्। हिर॑ण्यवर्णा॒स्तत॒ उत्पु॑नीत नः। इ॒मम्मे॑ वरुण॒ तत्त्वा॑ यामि। त्वन्नो॑ अग्ने॒ स त्वन्नो॑ अग्ने। त्वम॑ग्ने अ॒यासि॑॥१२१॥\anuvakamend[अ॒ने॒नस॑मष्ठी॒वद्भ्या स॒ति प॑रा॒शसा॑ऽऽन॒शेऽग्निर्मा॒ तस्मा॒देन॑सः पुनीत न॒स्त्रीणि॑ च (यद्दे॑वा॒ देवा॑ ऋ॒तेन॑ सजातश॒साद्यद्वा॒चा यद्धस्ताभ्या॒मदीव्यं॒ यन्मयि॑ मा॒ता यदा॑ पि॒पेष॒ यद॒न्तरि॑क्षं॒ यदा॒शसाऽति॑ क्रामामि त्रि॒ते दे॒वा दि॒वि जा॒ता अ॒प्सु जा॒ता यदाप॑ इ॒मम्मे॑ वरुण॒ तत्त्वा॑ यामि॒ त्वन्नो॑ अग्ने॒ स त्वन्नो॑ अग्ने॒ त्वम॑ग्ने अ॒यासि॑। )]

%3.7.13.1
यत्ते॒ ग्राव्ण्णा॑ चिच्छि॒दुः सो॑म राजन्। प्रि॒याण्यङ्गा॑नि॒ स्वधि॑ता॒ परूषि। तत्सन्ध॒त्स्वाज्ये॑नो॒त व॑र्धयस्व। अ॒ना॒गसो॒ अध॒मित्स॒ङ्क्षये॑म। यत्ते॒ ग्रावा॑ बा॒हुच्यु॑तो॒ अचु॑च्यवुः। नरो॒ यत्ते॑ दुदु॒हुर्दक्षि॑णेन। तत्त॒ आप्या॑यता॒न्तत्ते। निष्ट्या॑यतान्देव सोम। यत्ते॒ त्वच॑म्बिभि॒दुर्यच्च॒ योनिम्। यदा॒स्थाना॒त्प्रच्यु॑तो॒ वेन॑सि॒ त्मना ॥१२२॥

%3.7.13.2
त्वया॒ तत्सो॑म गु॒प्तम॑स्तु नः। सा न॑ स॒न्धास॑त्पर॒मे व्यो॑मन्। अहा॒च्छरी॑रं॒ पय॑सा स॒मेत्य॑। अ॒न्योन्यो भवति॒ वर्णो॑ अस्य। तस्मि॑न्व॒यमुप॑हूता॒स्तव॑ स्मः। आ नो॑ भज॒ सद॑सि वि॒श्वरू॑पे। नृ॒चक्षा॒ सोम॑ उ॒त शु॒श्रुग॑स्तु। मा नो॒ वि हा॑सी॒द्गिर॑ आवृणा॒नः। अना॑गास्त॒नुवो॑ वावृधा॒नः। आ नो॑ रू॒पं व॑हतु॒ जाय॑मानः॥१२३॥

%3.7.13.3
उप॑ क्षरन्ति जु॒ह्वो॑ घृ॒तेन॑। प्रि॒याण्यङ्गा॑नि॒ तव॑ व॒र्धय॑न्तीः। तस्मै॑ ते सोम॒ नम॒ इद्वष॑ट्च। उप॑ मा राजन्त्सुकृ॒ते ह्व॑यस्व। सं प्रा॑णापा॒नाभ्या॒ समु॒ चक्षु॑षा॒ त्वम्। स श्रोत्रे॑ण गच्छस्व सोम राजन्। यत्त॒ आस्थि॑त॒ शमु॒ तत्ते॑ अस्तु। जा॒नी॒तान्न॑ स॒ङ्गम॑ने पथी॒नाम्। ए॒तञ्जा॑नीतात्पर॒मे व्यो॑मन्। वृका सधस्था वि॒द रू॒पम॑स्य ॥१२४॥

%3.7.13.4
यदा॒गच्छात्प॒थिभि॑र्देव॒यानै। इ॒ष्टा॒पू॒र्ते कृ॑णुतादा॒विर॑स्मै। अरि॑ष्टो राजन्नग॒दः परे॑हि। नम॑स्ते अस्तु॒ चक्ष॑से रघूय॒ते। नाक॒मारो॑ह स॒ह यज॑मानेन। सूर्यं॑ गच्छतात्पर॒मे व्यो॑मन्। अभूद्दे॒वः स॑वि॒ता वन्द्यो॒नु न॑। इ॒दानी॒मह्न॑ उप॒वाच्यो॒ नृभि॑। वि यो रत्ना॒ भज॑ति मान॒वेभ्य॑। श्रेष्ठ॑न्नो॒ अत्र॒ द्रवि॑णं॒ यथा॒ दध॑त्। उप॑ नो मित्रावरुणावि॒हाव॑तम्। अ॒न्वादीध्याथामि॒ह न॑ सखाया। आ॒दि॒त्यानां॒ प्रसि॑तिर्\mbox{}हे॒तिः। उ॒ग्रा श॒तापाष्ठा घ॒विषा॒ परि॑ णो वृणक्तु। आप्या॑यस्व॒ सन्ते॥१२५॥\anuvakamend[त्मना॒ जाय॑मानोऽस्य॒ दध॒त्पञ्च॑ च]

%3.7.14.1
यद्दि॑दी॒क्षे मन॑सा॒ यच्च॑ वा॒चा। यद्वा प्रा॒णैश्चक्षु॑षा॒ यच्च॒ श्रोत्रे॑ण। यद्रेत॑सा मिथु॒नेनाप्या॒त्मना। अ॒द्भ्यो लो॒का द॑धिरे॒ तेज॑ इन्द्रि॒यम्। शु॒क्रा दी॒क्षायै॒ तप॑सो वि॒मोच॑नीः। आपो॑ विमो॒क्त्रीर्मयि॒ तेज॑ इन्द्रि॒यम्। यदृ॒चा साम्ना॒ यजु॑षा। प॒शू॒नाञ्चर्म॑न् ह॒विषा॑ दिदी॒क्षे। यच्छन्दो॑भि॒रोष॑धीभि॒र्वन॒स्पतौ। अ॒द्भ्यो लो॒का द॑धिरे॒ तेज॑ इन्द्रि॒यम् ॥१२६॥

%3.7.14.2
शु॒क्रा दी॒क्षायै॒ तप॑सो वि॒मोच॑नीः। आपो॑ विमो॒क्त्रीर्मयि॒ तेज॑ इन्द्रि॒यम्। येन॒ ब्रह्म॒ येन॑ क्ष॒त्रम्। येनेन्द्रा॒ग्नी प्र॒जाप॑ति॒ सोमो॒ वरु॑णो॒ येन॒ राजा। विश्वे॑ दे॒वा ऋष॑यो॒ येन॑ प्रा॒णाः। अ॒द्भ्यो लो॒का द॑धिरे॒ तेज॑ इन्द्रि॒यम्। शु॒क्रा दी॒क्षायै॒ तप॑सो वि॒मोच॑नीः। आपो॑ विमो॒क्त्रीर्मयि॒ तेज॑ इन्द्रि॒यम्। अ॒पां पुष्प॑म॒स्योष॑धीना॒ रस॑। सोम॑स्य प्रि॒यन्धाम॑॥१२७॥

%3.7.14.3
अ॒ग्नेः प्रि॒यत॑म ह॒विः स्वाहा। अ॒पां पुष्प॑म॒स्योष॑धीना॒ रस॑। सोम॑स्य प्रि॒यन्धाम॑। इन्द्र॑स्य प्रि॒यत॑म ह॒विः स्वाहा। अ॒पां पुष्प॑म॒स्योष॑धीना॒ रस॑। सोम॑स्य प्रि॒यन्धाम॑। विश्वे॑षान्दे॒वानां प्रि॒यत॑म ह॒विः स्वाहा। व॒य सो॑म व्र॒ते तव॑। मन॑स्त॒नूषु॒ पिप्र॑तः। प्र॒जाव॑न्तो अशीमहि॥१२८॥

%3.7.14.4
दे॒वेभ्य॑ पि॒तृभ्य॒ स्वाहा। सो॒म्येभ्य॑ पि॒तृभ्य॒ स्वाहा। क॒व्येभ्य॑ पि॒तृभ्य॒ स्वाहा। देवा॑स इ॒ह मा॑दयध्वम्। सोम्या॑स इ॒ह मा॑दयध्वम्। कव्या॑स इ॒ह मा॑दयध्वम्। अ॒न॑न्तरिताः पि॒तर॑ सो॒म्याः सो॑मपी॒थात्। अपै॑तु मृ॒त्युर॒मृत॑न्न॒ आग\sn{}। वै॒व॒स्व॒तो नो॒ अभ॑यङ्कृणोतु। प॒र्णं वन॒स्पते॑रिव॥१२९॥

%3.7.14.5
अ॒भि न॑ शीयता र॒यिः। सच॑तान्न॒ शची॒पति॑। पर॑म्मृत्यो॒ अनु॒ परे॑हि॒ पन्थाम्। यस्ते॒ स्व इत॑रो देव॒यानात्। चक्षु॑ष्मते शृण्व॒ते ते ब्रवीमि। मा न॑ प्र॒जा री॑रिषो॒ मोत वी॒रान्। इ॒दमू॒नु श्रेयो॑व॒सान॒माग॑न्म। यद्गो॒जिद्ध॑न॒जिद॑श्व॒जिद्यत्। प॒र्णं वन॒स्पते॑रिव। अ॒भि न॑ शीयता र॒यिः। सच॑तान्न॒ शची॒पति॑॥१३०॥\anuvakamend[वन॒स्पता॑व॒द्भ्यो लो॒का द॑धिरे॒ तेज॑ इन्द्रि॒यन्धामा॑शीमहीवा॒भिन॑ शीयता र॒यिरेकं च]




\prashnaend{सर्वा॒न्॒ यद्विष्ष॑ण्णेन॒ वि वै याः पु॒रस्ता॒द्देवा॑ दे॒वेषु॒ परि॑स्तृणीत॒ सक्षे॒दं यद॒स्य पा॒रे॑ऽना॒गस॒ उद॑स्तांप्सी॒द्ब्रह्म॑ प्रति॒ष्ठा यद्दे॑वा॒ यत्ते॒ ग्राव्ण्णा॒ यद्दि॑दी॒क्षे चतु॑र्दश॥१४॥}{सर्वा॒न्भूति॑मे॒व यामे॒वाप्स्वाहु॑तिं व्र॒तानां पर्णव॒ल्कः सो॒म्याना॑म॒स्मिन्‌य॒ज्ञेऽग्ने॒ यो नो॒ ज्योग्जी॒वाः प॒रोर॑जा॒ प्रते॑महे॒ ब्रह्म॑ प्रति॒ष्ठा गार्‌ह॑पत्यस्त्रि॒शदु॑त्तरश॒तम्॥१३०॥}{सर्वा॒ञ्छची॒पति॑॥}{हरि॑ ओम्॥}{इति श्रीकृष्णयजुर्वेदीयतैत्तिरीयब्राह्मणे तृतीयाष्टके सप्तमः प्रपाठकः समाप्तः॥}
\clearpage
\sect{अष्टमः प्रश्नः}
\setcounter{anuvakam}{0}
\dnsub{तैत्तिरीयब्राह्मणे तृतीयाष्टके अष्टमः प्रपाठकः}

%3.8.1.1
सा॒ङ्ग्र॒ह॒ण्येष्ट्या॑ यजते। इ॒माञ्ज॒नता॒ सङ्गृ॑ह्णा॒नीति॑। द्वाद॑शारत्नी रश॒ना भ॑वति। द्वाद॑श॒ मासा संवत्स॒रः। सं॒व॒त्स॒रमे॒वाव॑ रुन्धे। मौ॒ञ्जी भ॑वति। ऊर्ग्वै मुञ्जा। ऊर्ज॑मे॒वाव॑ रुन्धे। चि॒त्रा नक्ष॑त्रम्भवति। चि॒त्रं वा ए॒तत्कर्म॑॥१॥

%3.8.1.2
यद॑श्वमे॒धः समृ॑द्ध्यै। पुण्य॑नाम देव॒यज॑नम॒ध्यव॑स्यति। पुण्या॑मे॒व तेन॑ की॒र्तिम॒भि ज॑यति। अप॑दातीनृ॒त्विज॑ स॒माव॑ह॒न्त्या सु॑ब्रह्म॒ण्याया। सु॒व॒र्गस्य॑ लो॒कस्य॒ सम॑ष्ट्यै। के॒श॒श्म॒श्रु व॑पते। न॒खानि॒ नि कृ॑न्तते। द॒तो धा॑वते। स्नाति॑। अह॑तं॒ वास॒ परि॑धत्ते। पा॒प्मनोऽप॑हत्यै। वाचं॑ य॒त्वोप॑ वसति। सु॒व॒र्गस्य॑ लो॒कस्य॒ गुप्त्यै। रात्रिं॑ जाग॒रय॑न्त आसते। सु॒व॒र्गस्य॑ लो॒कस्य॒ सम॑ष्ट्यै॥२॥\anuvakamend[कर्म॑ धत्ते॒ पञ्च॑ च]

%3.8.2.1
चतु॑ष्टय्य॒ आपो॑ भवन्ति। चतु॑ शफो॒ वा अश्व॑ प्राजाप॒त्यः समृ॑द्ध्यै। ता दि॒ग्भ्यः स॒माभृ॑ता भवन्ति। दि॒क्षु वा आप॑। अन्नं॒ वा आप॑। अ॒द्भ्यो वा अन्नं॑ जायते। यदे॒वाद्भ्योऽन्नं॒ जाय॑ते। तदव॑ रुन्धे। तासु॑ ब्रह्मौद॒नं प॑चति। रेत॑ ए॒व तद्द॑धाति॥३॥

%3.8.2.2
चतु॑ शरावो भवति। दि॒क्ष्वे॑व प्रति॑तिष्ठति। उ॒भ॒यतो॑रु॒क्मौ भ॑वतः। उ॒भ॒यत॑ ए॒वास्मि॒न्रुच॑न्दधाति। उद्ध॑रति शृत॒त्वाय॑। स॒र्पिष्वान्भवति मेध्य॒त्वाय॑। च॒त्वार॑ आर्\mbox{}षे॒याः प्राश्ञ॑न्ति। दि॒शामे॒व ज्योति॑षि जुहोति। च॒त्वारि॒ हिर॑ण्यानि ददाति। दि॒शामे॒व ज्योती॒ष्यव॑ रुन्धे॥४॥

%3.8.2.3
यदाज्य॑मु॒च्छिष्य॑ते। तस्मि॑न्रश॒नान्यु॑नत्ति। प्र॒जाप॑ति॒र्वा ओ॑द॒नः। रेत॒ आज्यम्। यदाज्ये॑ रश॒नान्यु॒नत्ति॑। प्र॒जाप॑तिमे॒व रेत॑सा॒ सम॑र्धयति। द॒र्भ॒मयी॑ रश॒ना भ॑वति। ब॒हु वा ए॒ष कु॑च॒रो॑ मे॒ध्यमुप॑गच्छति। यदश्व॑। प॒वित्रं॒ वै द॒र्भाः॥५॥

%3.8.2.4
यद्द॑र्भ॒मयी॑ रश॒ना भव॑ति। पु॒नात्ये॒वैनम्। पू॒तमे॑न॒म्मेध्य॒मा ल॑भते। अश्व॑स्य॒ वा आल॑ब्धस्य महि॒मोद॑क्रामत्। स म॒हर्त्वि॑ज॒ प्रावि॑शत्। तन्म॒हर्त्वि॑जाम्महर्त्वि॒क्त्वम्। यन्म॒हर्त्वि॑जः प्रा॒श्ञन्ति॑। म॒हि॒मान॑मे॒वास्मि॒न्तद्द॑धति। अश्व॑स्य॒ वा आल॑ब्धस्य॒ रेत॒ उद॑क्रामत्। तत्सु॒वर्ण॒ हिर॑ण्यमभवत्। यत्सु॒वर्ण॒ हिर॑ण्य॒न्ददा॑ति। रेत॑ ए॒व तद्द॑धाति। ओ॒द॒ने द॑दाति। रेतो॒ वा ओ॑द॒नः। रेतो॒ हिर॑ण्यम्। रेत॑सै॒वास्मि॒न्रेतो॑ दधाति॥६॥\anuvakamend[द॒धा॒ति॒ रु॒न्धे॒ द॒र्भा अ॑भव॒थ्षट् च॑]

%3.8.3.1
यो वै ब्रह्म॑णे दे॒वेभ्य॑ प्र॒जाप॑त॒येऽप्र॑तिप्रो॒च्याश्वं॒ मेध्यं॑ ब॒ध्नाति॑। आ दे॒वताभ्यो वृश्च्यते। पापी॑यान्भवति। यः प्र॑ति॒प्रोच्य॑। न दे॒वताभ्य॒ आवृ॑श्च्यते। वसी॑यान्भवति। यदाह॑। ब्रह्म॒न्नश्व॒म्मेध्य॑म्भन्त्स्यामि दे॒वेभ्य॑ प्र॒जाप॑तये॒ तेन॑ राध्यास॒मिति॑। ब्रह्म॒ वै ब्र॒ह्मा। ब्रह्म॑ण ए॒व दे॒वेभ्य॑ प्र॒जाप॑तये प्रति॒प्रोच्याश्व॒म्मेध्य॑म्बध्नाति॥७॥

%3.8.3.2
न दे॒वताभ्य॒ आ वृ॑श्च्यते। वसी॑यान्भवति। दे॒वस्य॑ त्वा सवि॒तुः प्र॑स॒व इति॑ रश॒नामाद॑त्ते॒ प्रसूत्यै। अ॒श्विनोर्बा॒हुभ्या॒मित्या॑ह। अ॒श्विनौ॒ हि दे॒वाना॑मध्व॒र्यू आस्ताम्। पू॒ष्णो हस्ताभ्या॒मित्या॑ह॒ यत्यै। व्यृ॑द्धं॒ वा ए॒तद्य॒ज्ञस्य॑। यद॑य॒जुष्के॑ण क्रि॒यते। इ॒माम॑गृभ्णन्रश॒नामृ॒तस्येत्यधि॑ वदति॒ यजु॑ष्कृत्यै। य॒ज्ञस्य॒ समृ॑द्ध्यै॥८॥

%3.8.3.3
तदा॑हुः। द्वाद॑शारत्नी रश॒ना क॑र्त॒व्या ३ त्रयो॑दशार॒त्नी ३ रिति॑। ऋ॒ष॒भो वा ए॒ष ऋ॑तू॒नाम्। यत्सं॑वत्स॒रः। तस्य॑ त्रयोद॒शो मासो॑ वि॒ष्टपम्। ऋ॒ष॒भ ए॒ष य॒ज्ञानाम्। यद॑श्वमे॒धः। यथा॒ वा ऋ॑ष॒भस्य॑ वि॒ष्टपम्। ए॒वमे॒तस्य॑ वि॒ष्टपम्। त्र॒यो॒द॒शम॑र॒त्नि र॑श॒नाया॑मु॒पा द॑धाति॥९॥

%3.8.3.4
यथ॑र्\mbox{}ष॒भस्य॑ वि॒ष्टप सस्क॒रोति॑। ता॒दृगे॒व तत्। पूर्व॒ आयु॑षि वि॒दथे॑षु क॒व्येत्या॑ह। आयु॑रे॒वास्मि॑न्दधाति। तया॑ दे॒वाः सु॒तमा ब॑भूवु॒रित्या॑ह। भूति॑मे॒वोपाव॑र्तते। ऋ॒तस्य॒ सामन्त्स॒रमा॒रप॒न्तीत्या॑ह। स॒त्यं वा ऋ॒तम्। स॒त्येनै॒वैन॑मृ॒तेनार॑भते। अ॒भि॒धा अ॒सीत्या॑ह॥१०॥

%3.8.3.5
तस्मा॑दश्वमेधया॒जी सर्वा॑णि भू॒तान्य॒भि भ॑वति। भुव॑नम॒सीत्या॑ह। भू॒मान॑मे॒वोपै॑ति। य॒न्ताऽसीत्या॑ह। य॒न्तार॑मे॒वैनं॑ करोति। ध॒र्तासीत्या॑ह। ध॒र्तार॑मे॒वैनं॑ करोति। सोऽग्निं वैश्वान॒रमित्या॑ह। अ॒ग्नावे॒वैनं॑ वैश्वान॒रे जु॑होति। सप्र॑थस॒मित्या॑ह ॥। 1१॥

%3.8.3.6
प्र॒जयै॒वैनं॑ प॒शुभि॑ प्रथयति। स्वाहा॑कृत॒ इत्या॑ह। होम॑ ए॒वास्यै॒षः। पृ॒थि॒व्यामित्या॑ह। अ॒स्यामे॒वैनं॒ प्रति॑ष्ठापयति। य॒न्ता राड्य॒न्ताऽसि॒ यम॑नो ध॒र्तासि॑ ध॒रुण॒ इत्या॑ह। रू॒पमे॒वास्यै॒तन्म॑हि॒मान॒व्व्याँच॑ष्टे। कृ॒ष्यै त्वा॒ क्षेमा॑य त्वा र॒य्यै त्वा॒ पोषा॑य॒ त्वेत्या॑ह। आ॒शिष॑मे॒वैतामाशास्ते। स्व॒गा त्वा॑ दे॒वेभ्य॒ इत्या॑ह। दे॒वेभ्य॑ ए॒वैन स्व॒गा क॑रोति। स्वाहा त्वा प्र॒जाप॑तय॒ इत्या॑ह। प्रा॒जा॒प॒त्यो वा अश्व॑। यस्या॑ ए॒व दे॒वता॑या आल॒भ्यते। तयै॒वैन॒ सम॑र्धयति॥१२॥\anuvakamend[ब॒ध्ना॒ति॒ समृ॑द्ध्या उ॒पाद॑धात्य॒सीत्या॑ह॒ सप्र॑थस॒मित्या॑ह दे॒वेभ्य॒ इत्या॑ह॒ पञ्च॑ च]

%3.8.4.1
यः पि॒तुर॑नु॒जाया पु॒त्रः। स पु॒रस्तान्नयति। यो मा॒तुर॑नु॒जाया पु॒त्रः। स प॒श्चान्न॑यति। विष्व॑ञ्चमे॒वास्मात्पा॒प्मानं॒ विवृ॑हतः। यो अर्व॑न्तं॒ जिघासति॒ तम॒भ्य॑मीति॒ वरु॑ण॒ इति॒ श्वानं॑ चतुर॒क्षं प्रसौ॑ति। प॒रो मर्त॑ प॒रः श्वेति॒ शुन॑श्चतुर॒क्षस्य॒ प्रह॑न्ति। श्वेव॒ वै पा॒प्मा भ्रातृ॑व्यः। पा॒प्मान॑मे॒वास्य॒ भ्रातृ॑व्य हन्ति। सै॒ध्र॒कम्मुस॑लम्भवति॥१३॥

%3.8.4.2
कर्म॑कर्मै॒वास्मै॑ साधयति। पौ॒श्च॒ले॒यो ह॑न्ति। पु॒श्च॒ल्वां वै दे॒वाः शुच॒न्न्य॑दधुः। शु॒चैवास्य॒ शुच हन्ति। पा॒प्मा वा ए॒तमीप्स॒तीत्या॑हुः। योऽश्वमे॒धेन॒ यज॑त॒ इति॑। अश्व॑स्याधस्प॒दमुपास्यति। व॒ज्री वा अश्व॑ प्राजाप॒त्यः। वज्रे॑णै॒व पा॒प्मान॒म्भ्रातृ॑व्य॒मव॑ क्रामति। द॒क्षि॒णाऽप॑ प्लावयति॥१४॥

%3.8.4.3
पा॒प्मान॑मे॒वास्मा॒च्छम॑ल॒मप॑ प्लावयति। ऐ॒षी॒क उ॑दू॒हो भ॑वति। आयु॒र्वा इ॒षीका। आयु॑रे॒वास्मि॑न्दधति। अ॒मृतं॒ वा इ॒षीका। अ॒मृत॑मे॒वास्मि॑न्दधति। वे॒त॒स॒शा॒खोप॒सम्ब॑द्धा भवति। अ॒प्सुयो॑नि॒र्वा अश्व॑। अ॒प्सु॒जो वे॑त॒सः। स्वादे॒वैन॒य्योँने॒र्निर्मि॑मीते। पु॒रस्तात्प्र॒त्यञ्च॑म॒भ्युदू॑हति। पु॒रस्ता॑दे॒वास्मि॑न्प्र॒तीच्य॒मृत॑न्दधाति। अ॒हं च॒ त्वं च॑ वृत्रह॒न्निति॑ ब्र॒ह्मा यज॑मानस्य॒ हस्त॑ङ्गृह्णाति। ब्र॒ह्म॒क्ष॒त्रे ए॒व सन्द॑धाति। अ॒भिक्रत्वेन्द्र भू॒रध॒ज्मन्नित्य॑ध्व॒र्युर्यज॑मानं वाचयत्य॒भिजि॑त्यै॥१५॥\anuvakamend[भ॒व॒ति॒ प्ला॒व॒य॒ति॒ मि॒मी॒ते॒ पञ्च॑ च]

%3.8.5.1
च॒त्वार॑ ऋ॒त्विज॒ समु॑क्षन्ति। आ॒भ्य ए॒वैनं॑ चत॒सृभ्यो॑ दि॒ग्भ्यो॑ऽभि समी॑रयन्ति। श॒तेन॑ राजपु॒त्रैः स॒हाध्व॒र्युः। पु॒रस्तात्प्र॒त्यङ्तिष्ठ॒न्प्रोक्ष॑ति। अ॒नेनाश्वे॑न॒ मेध्ये॑ने॒ष्ट्वा। अ॒य राजा॑ वृ॒त्रं व॑ध्या॒दिति॑। रा॒ज्यं वा अ॑ध्व॒र्युः। क्ष॒त्र रा॑जपु॒त्रः। रा॒ज्येनै॒वास्मि॑न्क्ष॒त्रन्द॑धाति। श॒तेना॑ रा॒जभि॑रु॒ग्रैः स॒ह ब्र॒ह्मा॥१६॥

%3.8.5.2
द॒क्षि॒ण॒त उद॒ङ्तिष्ठ॒न्प्रोक्ष॑ति। अ॒नेनाश्वे॑न॒ मेध्ये॑ने॒ष्ट्वा। अ॒य राजाप्रतिधृ॒ष्योऽस्त्विति॑। बलं॒ वै ब्र॒ह्मा। बल॑मरा॒जोग्रः। बले॑नै॒वास्मि॒न्बल॑न्दधाति। श॒तेन॑ सूतग्राम॒णिभि॑ स॒ह होता। प॒श्चात्प्राङ्तिष्ठ॒न्प्रोक्ष॑ति। अ॒नेनाश्वे॑न॒ मेध्ये॑ने॒ष्ट्वा। अ॒य राजा॒ऽस्यै वि॒शः॥१७॥

%3.8.5.3
ब॒हु॒ग्वै ब॑ह्व॒श्वायै॑ बह्वजावि॒कायै। ब॒हु॒व्री॒हि॒य॒वायै॑ बहुमाषति॒लायै। ब॒हु॒हि॒र॒ण्यायै॑ बहुह॒स्तिका॑यै। ब॒हु॒दा॒स॒पू॒रु॒षायै॑ रयि॒मत्यै॒ पुष्टि॑मत्यै। ब॒हु॒रा॒य॒स्पो॒षायै॒ राजा॒स्त्विति॑। भू॒मा वै होता। भू॒मा सू॑तग्राम॒ण्य॑। भू॒म्नैवास्मि॑न्भू॒मान॑न्दधाति। श॒तेन॑ क्षत्तसङ्ग्रही॒तृभि॑ स॒होद्गा॒ता। उ॒त्त॒र॒तो द॑क्षि॒णा तिष्ठ॒न्प्रोक्ष॑ति॥१८॥

%3.8.5.4
अ॒नेनाश्वे॑न॒ मेध्ये॑ने॒ष्ट्वा। अ॒य राजा॒ सर्व॒मायु॑रे॒त्विति॑। आयु॒र्वा उ॑द्गा॒ता। आयु॑ क्षत्तसङ्ग्रही॒तार॑। आयु॑षै॒वास्मि॒न्नायु॑र्दधाति। श॒तश॑तम्भवन्ति। श॒तायु॒ पुरु॑षः श॒तेन्द्रि॑यः। आयु॑ष्ये॒वेन्द्रि॒ये प्रति॑तिष्ठति। च॒तु॒ श॒ता भ॑वन्ति। चत॑स्रो॒ दिश॑। दि॒क्ष्वे॑व प्रति॑ तिष्ठति॥१९॥\anuvakamend[ब्र॒ह्मा वि॒श उ॑क्षति॒ दिश॒ एकं च]

%3.8.6.1
यथा॒ वै ह॒विषो॑ गृही॒तस्य॒ स्कन्द॑ति। ए॒वं वा ए॒तदश्व॑स्य स्कन्दति। यन्नि॒क्तमना॑लब्धमुत्सृ॒जन्ति॑। यत्स्तोक्या॑ अ॒न्वाह॑। स॒र्व॒हुत॑मे॒वैनं॑ करो॒त्यस्क॑न्दाय। अस्क॑न्न॒ हि तत्। यद्धु॒तस्य॒ स्कन्द॑ति। स॒हस्र॒मन्वा॑ह। स॒हस्र॑सम्मितः सुव॒र्गो लो॒कः। सु॒व॒र्गस्य॑ लो॒कस्या॒भिजि॑त्यै॥२०॥

%3.8.6.2
यत्परि॑मिता अनुब्रू॒यात्। परि॑मित॒मव॑ रुन्धीत। अप॑रिमिता॒ अन्वा॑ह। अप॑रिमितः सुव॒र्गो लो॒कः। सु॒व॒र्गस्य॑ लो॒कस्य॒ सम॑ष्ट्यै। स्तोक्या॑ जुहोति। या ए॒व वर्ष्या॒ आप॑। ता अव॑ रुन्धे। अ॒स्यां जु॑होति। इ॒यं वा अ॒ग्निर्वैश्वान॒रः॥२१॥

%3.8.6.3
अ॒स्यामे॒वैना॒ प्रति॑ष्ठापयति। उ॒वाच॑ ह प्र॒जाप॑तिः। स्तोक्या॑सु॒ वा अ॒हम॑श्वमे॒ध सस्था॑पयामि। तेन॒ तत॒ सस्थि॑तेन चरा॒मीति॑। अ॒ग्नये॒ स्वाहेत्या॑ह। अ॒ग्नय॑ ए॒वैनं॑ जुहोति। सोमा॑य॒ स्वाहेत्या॑ह। सोमा॑यै॒वैनं॑ जुहोति। स॒वि॒त्रे स्वाहेत्या॑ह। स॒वि॒त्र ए॒वैनं॑ जुहोति॥२२॥

%3.8.6.4
सर॑स्वत्यै॒ स्वाहेत्या॑ह। सर॑स्वत्या ए॒वैनं॑ जुहोति। पू॒ष्णे स्वाहेत्या॑ह। पू॒ष्ण ए॒वैनं॑ जुहोति। बृह॒स्पत॑ये॒ स्वाहेत्या॑ह। बृह॒स्पत॑य ए॒वैनं॑ जुहोति। अ॒पाम्मोदा॑य॒ स्वाहेत्या॑ह। अ॒द्भ्य ए॒वैनं॑ जुहोति। वा॒यवे॒ स्वाहेत्या॑ह। वा॒यव॑ ए॒वैनं॑ जुहोति॥२३॥

%3.8.6.5
मि॒त्राय॒ स्वाहेत्या॑ह। मि॒त्रायै॒वैनं॑ जुहोति। वरु॑णाय॒ स्वाहेत्या॑ह। वरु॑णायै॒वैनं॑ जुहोति। ए॒ताभ्य॑ ए॒वैनं॑ दे॒वताभ्यो जुहोति। दश॑दश सं॒पादं॑ जुहोति। दशाक्षरा वि॒राट्। अन्नं॑ वि॒राट्। वि॒राजै॒वान्नाद्य॒मव॑ रुन्धे। प्र वा ए॒षोऽस्माल्लो॒काच्च्य॑वते। यः परा॑ची॒राहु॑तीर्जु॒होति॑। पुन॑ पुनरभ्या॒वर्तं॑ जुहोति। अ॒स्मिन्ने॒व लो॒के प्रति॑तिष्ठति। ए॒ता ह वाव सोऽश्वमे॒धस्य॒ सस्थि॑तिमुवा॒चास्क॑न्दाय। अस्क॑न्न॒ हि तत्। यद्य॒ज्ञस्य॒ सस्थि॑तस्य॒ स्कन्द॑ति॥२४॥\anuvakamend[अ॒भिजि॑त्यै वैश्वान॒रः स॑वि॒त्र ए॒वैनं॑ जुहोति वा॒यव॑ ए॒वैनं॑ जुहोति च्यवते॒ षट् च॑]

%3.8.7.1
प्र॒जाप॑तये त्वा॒ जुष्टं॒ प्रोक्षा॒मीति॑ पु॒रस्तात्प्र॒त्यङ्तिष्ठ॒न्प्रोक्ष॑ति। प्र॒जाप॑ति॒र्वै दे॒वाना॑मन्ना॒दो वी॒र्या॑वान्। अ॒न्नाद्य॑मे॒वास्मि॑न्वी॒र्यं॑ दधाति। तस्मा॒दश्व॑ पशू॒नाम॑न्ना॒दो वी॒र्या॑वत्तमः। इ॒न्द्रा॒ग्निभ्या॒न्त्वेति॑ दक्षिण॒तः। इ॒न्द्रा॒ग्नी वै दे॒वाना॒मोजि॑ष्ठौ॒ बलि॑ष्ठौ। ओज॑ ए॒वास्मि॒न्बल॑न्दधाति। तस्मा॒दश्व॑ पशू॒नामोजि॑ष्ठो॒ बलि॑ष्ठः। वा॒यवे॒ त्वेति॑ प॒श्चात्। वा॒युर्वै दे॒वाना॑मा॒शुः सा॑रसा॒रित॑मः॥२५॥

%3.8.7.2
ज॒वमे॒वास्मि॑न्दधाति। तस्मा॒दश्व॑ पशू॒नामा॒शुः सा॑रसा॒रित॑मः। विश्वेभ्यस्त्वा दे॒वेभ्य॒ इत्यु॑त्तर॒तः। विश्वे॒ वै दे॒वा दे॒वानां यश॒स्वित॑माः। यश॑ ए॒वास्मि॑न्दधाति। तस्मा॒दश्व॑ पशू॒नां य॑श॒स्वित॑मः। दे॒वेभ्य॒स्त्वेत्य॒धस्तात्। दे॒वा वै दे॒वाना॒मप॑चिततमाः। अप॑चितिमे॒वास्मि॑न्दधाति। तस्मा॒दश्व॑ पशू॒नामप॑चिततमः॥२६॥

%3.8.7.3
सर्वेभ्यस्त्वा दे॒वेभ्य॒ इत्यु॒परि॑ष्टात्। सर्वे॒ वै दे॒वास्त्विषि॑मन्तो हर॒स्विन॑। त्विषि॑मे॒वास्मि॒न्॒ हरो॑ दधाति। तस्मा॒दश्व॑ पशू॒नान्त्विषि॑मान्‌हर॒स्वित॑मः। दि॒वे त्वा॒ऽन्तरि॑क्षाय त्वा पृथि॒व्यै त्वेत्या॑ह। ए॒भ्य ए॒वैनं॑ लो॒केभ्य॒ प्रोक्ष॑ति। स॒ते त्वाऽस॑ते त्वा॒ऽद्भ्यस्त्वौष॑धीभ्यस्त्वा॒ विश्वेभ्यस्त्वा भू॒तेभ्य॒ इत्या॑ह। तस्मा॑दश्वमेधया॒जिन॒ सर्वा॑णि भू॒तान्युप॑जीवन्ति। ब्र॒ह्म॒वा॒दिनो॑ वदन्ति। यत्प्रा॑जाप॒त्योऽश्व॑। अथ॒ कस्मा॑देनम॒न्याभ्यो॑ दे॒वता॒भ्योऽपि॒ प्रोक्ष॒तीति॑। अश्वे॒ वै सर्वा॑ दे॒वता॑ अ॒न्वाय॑त्ताः। तं यद्विश्वेभ्यस्त्वा भू॒तेभ्य॒ इति॑ प्रो॒क्षति॑। दे॒वता॑ ए॒वास्मि॑न्न॒न्वा या॑तयति। तस्मा॒दश्वे॒ सर्वा॑ दे॒वता॑ अ॒न्वाय॑त्ताः॥२७॥\anuvakamend[सा॒र॒सा॒रित॒मोऽप॑चिततमः प्राजाप॒त्योऽश्व॒ पञ्च॑ च]

%3.8.8.1
यथा॒ वै ह॒विषो॑ गृही॒तस्य॒ स्कन्द॑ति। ए॒वं वा ए॒तदश्व॑स्य स्कन्दति। यत्प्रोक्षि॑त॒मना॑लब्धमुत्सृ॒जन्ति॑। यद॑श्वचरि॒तानि॑ जु॒होति॑। स॒र्व॒हुत॑मे॒वैनं॑ करो॒त्यस्क॑न्दाय। अस्क॑न्न॒ हि तत्। यद्धु॒तस्य॒ स्कन्द॑ति। ई॒ङ्का॒राय॒ स्वाहें कृ॑ताय॒ स्वाहेत्या॑ह। ए॒तानि॒ वा अ॑श्वचरि॒तानि॑। च॒रि॒तैरे॒वैन॒ सम॑र्धयति॥२८॥

%3.8.8.2
तदा॑हुः। अना॑हुतयो॒ वा अ॑श्वचरि॒तानि॑। नैता हो॑त॒व्या॑ इति॑। अथो॒ खल्वा॑हुः। हो॒त॒व्या॑ ए॒व। अत्र॒ वावैवं वि॒द्वान॑श्वमे॒ध सस्था॑पयति। यद॑श्वचरि॒तानि॑ जु॒होति॑। तस्माद्धोत॒व्या॑ इति॑। ब॒हि॒र्धा वा ए॑नमे॒तदा॒यत॑नाद्दधाति। भ्रातृ॑व्यमस्मै जनयति॥२९॥

%3.8.8.3
यस्या॑नायत॒नेऽन्यत्रा॒ग्नेराहु॑तीर्जु॒होति॑। सा॒वि॒त्रि॒या इष्ट्या पु॒रस्तात्स्विष्ट॒कृत॑। आ॒ह॒व॒नीयेऽश्वचरि॒तानि॑ जुहोति। आ॒यत॑न ए॒वास्याहु॑तीर्जुहोति। नास्मै॒ भ्रातृ॑व्यञ्जनयति। तदा॑हुः। य॒ज्ञ॒मु॒खेय॑ज्ञमुखे होत॒व्या। य॒ज्ञस्य॒ कॢप्त्यै। सु॒व॒र्गस्य॑ लो॒कस्यानु॑ख्यात्या॒ इति॑। अथो॒ खल्वा॑हुः॥३०॥

%3.8.8.4
यद्य॑ज्ञमु॒खेय॑ज्ञमुखे जुहु॒यात्। प॒शुभि॒र्यज॑मान॒व्व्यँ॑र्धयेत्। अव॑ सुव॒र्गाल्लो॒कात्प॑द्येत। पापी॑यान्त्स्या॒दिति॑। स॒कृदे॒व हो॑त॒व्या। न यज॑मानं प॒शुभि॒र्व्य॑र्धयति। अ॒भि सु॑व॒र्गं लो॒कं ज॑यति। न पापी॑यान्भवति। अ॒ष्टाच॑त्वारिशतमश्वरू॒पाणि॑ जुहोति। अ॒ष्टाच॑त्वारिशदक्षरा॒ जग॑ती। जाग॒तोऽश्व॑ प्राजाप॒त्यः समृ॑द्ध्यै। ए॒कमति॑रिक्तं जुहोति। तस्मा॒देक॑ प्र॒जास्वर्धु॑कः॥३१॥\anuvakamend[अ॒र्ध॒य॒ति॒ ज॒न॒य॒ति॒ खल्वा॑हु॒र्जग॑ती॒ त्रीणि॑ च]

%3.8.9.1
वि॒भूर्मा॒त्रा प्र॒भूः पि॒त्रेत्या॑ह। इ॒यं वै मा॒ता। अ॒सौ पि॒ता। आ॒भ्यामे॒वैनं॒ परि॑ददाति। अश्वो॑ऽसि॒ हयो॒ऽसीत्या॑ह। शास्त्ये॒वैन॑मे॒तत्। तस्माच्छि॒ष्टाः प्र॒जा जा॑यन्ते। अत्यो॒ऽसीत्या॑ह। तस्मा॒दश्व॒ सर्वान्प॒शूनत्ये॑ति। तस्मा॒दश्व॒ सर्वे॑षां पशू॒ना श्रैष्ठ्यं॑ गच्छति॥३२॥

%3.8.9.2
प्र यश॒ श्रैष्ठ्य॑माप्नोति। य ए॒वं वेद॑। नरो॒ऽस्यर्वा॑ऽसि॒ सप्ति॑रसि वा॒ज्य॑सीत्या॑ह। रू॒पमे॒वास्यै॒तन्म॑हि॒मान॒व्व्याँच॑ष्टे। ययु॒र्नामा॒ऽसीत्या॑ह। ए॒तद्वा अश्व॑स्य प्रि॒यन्ना॑म॒धेयम्। प्रि॒येणै॒वैन॑न्नाम॒धेये॑ना॒भि व॑दति। तस्मा॒दप्या॑मि॒त्रौ स॒ङ्गत्य॑। नाम्ना॒ चेद्ध्वये॑ते। मि॒त्रमे॒व भ॑वतः॥३३॥

%3.8.9.3
आ॒दि॒त्यानां॒ पत्वाऽन्वि॒हीत्या॑ह। आ॒दि॒त्याने॒वैन॑ङ्गमयति। अ॒ग्नये॒ स्वाहा॒ स्वाहेन्द्रा॒ग्निभ्या॒मिति॑ पूर्वहो॒मां जु॑होति। पूर्व॑ ए॒व द्वि॒षन्त॒म्भ्रातृ॑व्य॒मति॑ क्रामति। भूर॑सि भु॒वे त्वा॒ भव्या॑य त्वा भविष्य॒ते त्वेत्युत्सृ॑जति सर्व॒त्वाय॑। देवा॑ आशापाला ए॒तन्दे॒वेभ्योऽश्व॒म्मेधा॑य॒ प्रोक्षि॑तङ्गोपाय॒तेत्या॑ह। श॒तं वै तल्प्या॑ राजपु॒त्रा दे॒वा आ॑शापा॒लाः। तेभ्य॑ ए॒वैनं॒ परि॑ ददाति। ई॒श्व॒रो वा अश्व॒ प्रमु॑क्त॒ परां परा॒वत॒ङ्गन्तो। इ॒ह धृति॒ स्वाहे॒ह विधृ॑ति॒ स्वाहे॒ह रन्ति॒ स्वाहे॒ह रम॑ति॒ स्वाहेति॑ चतृ॒षु प॒त्सु जु॑होति॥३४॥

%3.8.9.4
ए॒ता वा अश्व॑स्य॒ बन्ध॑नम्। ताभि॑रे॒वैन॑म्बध्नाति। तस्मा॒दश्व॒ प्रमु॑क्तो॒ बन्ध॑न॒मा ग॑च्छति। तस्मा॒दश्व॒ प्रमु॑क्तो॒ बन्ध॑न॒न्न ज॑हाति। रा॒ष्ट्रं वा अ॑श्वमे॒धः। रा॒ष्ट्रे खलु॒ वा ए॒ते व्याय॑च्छन्ते। येऽश्व॒म्मेध्य॒ रक्ष॑न्ति। तेषां॒ य उ॒दृचं॒ गच्छ॑न्ति। रा॒ष्ट्रादे॒व ते रा॒ष्ट्रङ्ग॑च्छन्ति। अथ॒ य उ॒दृच॒न्न गच्छ॑न्ति॥३५॥

%3.8.9.5
रा॒ष्ट्रादे॒व ते व्यव॑च्छिद्यन्ते। परा॒ वा ए॒ष सि॑च्यते। यो॑ऽब॒लोऽश्वमे॒धेन॒ यज॑ते। यद॒मित्रा॒ अश्वं॑ वि॒न्देर\sn{}। ह॒न्येतास्य य॒ज्ञः। च॒तु॒ श॒ता र॑क्षन्ति। य॒ज्ञस्याघा॑ताय। अथा॒न्यमा॒नीय॒ प्रोक्षे॑युः। सैव तत॒ प्राय॑श्चित्तिः॥३६॥\anuvakamend[ग॒च्छ॒ति॒ भ॒व॒त॒ प॒त्सु जु॑होति॒ न गच्छ॑न्ति॒ नव॑ च]

%3.8.10.1
प्र॒जाप॑तिरकामयताश्वमे॒धेन॑ यजे॒येति॑। स तपो॑ऽतप्यत। तस्य॑ तेपा॒नस्य॑। स॒प्तात्मनो॑ दे॒वता॒ उद॑क्रामन्। सा दी॒क्षाऽभ॑वत्। स ए॒तानि॑ वैश्वदे॒वान्य॑पश्यत्। तान्य॑जुहोत्। तैर्वै स दी॒क्षामवा॑रुन्ध। यद्वैश्वदे॒वानि॑ जु॒होति॑। दी॒क्षामे॒व तैर्यज॑मा॒नोऽव॑ रुन्धे॥३७॥

%3.8.10.2
स॒प्त जु॑होति। स॒प्त हि ता दे॒वता॑ उ॒दक्रा॑मन्। अ॒न्व॒हं जु॑होति। अ॒न्व॒हमे॒व दी॒क्षामव॑ रुन्धे। त्रीणि॑ वैश्वदे॒वानि॑ जुहोति। च॒त्वार्यौद्ग्रह॒णानि॑। स॒प्त संप॑द्यन्ते। स॒प्त वै शी॑र्‌ष॒ण्या प्रा॒णाः। प्रा॒णा दी॒क्षा। प्रा॒णैरे॒व प्रा॒णान्दी॒क्षामव॑ रुन्धे॥३८॥

%3.8.10.3
एक॑विशतिं वैश्वदे॒वानि॑ जुहोति। एक॑विशति॒र्वै दे॑वलो॒काः। द्वाद॑श॒ मासा॒ पञ्च॒र्तव॑। त्रय॑ इ॒मे लो॒काः। अ॒सावा॑दि॒त्य ए॑कवि॒शः। ए॒ष सु॑व॒र्गो लो॒कः। तद्दैव्यं॑ क्ष॒त्रम्। सा श्रीः। तद्ब्र॒ध्नस्य॑ वि॒ष्टपम्। तत्स्वाराज्यमुच्यते॥३९॥

%3.8.10.4
त्रि॒शत॑मौद्ग्रह॒णानि॑ जुहोति। त्रि॒शद॑क्षरा वि॒राट्। अन्नं॑ वि॒राट्। वि॒राजै॒वान्नाद्य॒मव॑ रुन्धे। त्रे॒धा वि॒भज्य॑ दे॒वतां जुहोति। त्र्या॑वृतो॒ वै दे॒वाः। त्र्या॑वृत इ॒मे लो॒काः। ए॒षां लो॒काना॒माप्त्यै। ए॒षां लो॒कानां॒ कॢप्त्यै। अप॒ वा ए॒तस्मात्प्रा॒णाः क्रा॑मन्ति॥४०॥

%3.8.10.5
यो दी॒क्षाम॑तिरे॒चय॑ति। स॒प्ता॒हं प्रच॑रन्ति। स॒प्त वै शी॑र्\mbox{}ष॒ण्या प्रा॒णाः। प्रा॒णा दी॒क्षा। प्रा॒णैरे॒व प्रा॒णान्दी॒क्षामव॑ रुन्धे। पू॒र्णा॒हु॒तिमु॑त्त॒मां जु॑होति। सर्वं॒ वै पूर्णाहु॒तिः। सर्व॑मे॒वाप्नो॑ति। अथो॑ इ॒यं वै पूर्णाहु॒तिः। अ॒स्यामे॒व प्रति॑ तिष्ठति॥४१॥\anuvakamend[रु॒न्धे॒ प्रा॒णान्दी॒क्षामव॑ रुन्ध उच्यते क्रामन्ति तिष्ठति]

%3.8.11.1
प्र॒जाप॑तिरश्वमे॒धम॑सृजत। त सृ॒ष्टं न किञ्च॒नोद॑यच्छत्। तं वैश्वदे॒वान्ये॒वोद॑यच्छन्। यद्वैश्वदे॒वानि॑ जु॒होति॑। य॒ज्ञस्योद्य॑त्यै। स्वाहा॒ऽऽधिमाधी॑ताय॒ स्वाहा। स्वाहाऽधी॑तं॒ मन॑से॒ स्वाहा। स्वाहा॒ मन॑ प्र॒जाप॑तये॒ स्वाहा। काय॒ स्वाहा॒ कस्मै॒ स्वाहा॑ कत॒मस्मै॒ स्वाहेति॑ प्राजाप॒त्ये मुख्ये॑ भवतः। प्र॒जाप॑तिमुखाभिरे॒वैनं॑ दे॒वता॑भि॒रुद्य॑च्छते॥४२॥

%3.8.11.2
अदि॑त्यै॒ स्वाहाऽदि॑त्यै म॒ह्यै स्वाहाऽदि॑त्यै सुमृडी॒कायै॒ स्वाहेत्या॑ह। इ॒यं वा अदि॑तिः। अ॒स्या ए॒वैनं॑ प्रति॒ष्ठायोद्य॑च्छते। सर॑स्वत्यै॒ स्वाहा॒ सर॑स्वत्यै बृह॒त्यै स्वाहा॒ सर॑स्वत्यै पाव॒कायै॒ स्वाहेत्या॑ह। वाग्वै सर॑स्वती। वा॒चैवैन॒मुद्य॑च्छते। पू॒ष्णे स्वाहा॑ पू॒ष्णे प्र॑प॒थ्या॑य॒ स्वाहा॑ पू॒ष्णे न॒रन्धि॑षाय॒ स्वाहेत्या॑ह। प॒शवो॒ वै पू॒षा। प॒शुभि॑रे॒वैन॒मुद्य॑च्छते। त्वष्ट्रे॒ स्वाहा॒ त्वष्ट्रे॑ तु॒रीपा॑य॒ स्वाहा॒ त्वष्ट्रे॑ पुरु॒रूपा॑य॒ स्वाहेत्या॑ह। त्वष्टा॒ वै प॑शू॒नां मि॑थु॒नाना रूप॒कृत्। रू॒पमे॒व प॒शुषु॑ दधाति। अथो॑ रू॒पैरे॒वैन॒मुद्य॑च्छते। विष्ण॑वे॒ स्वाहा॒ विष्ण॑वे निखुर्य॒पाय॒ स्वाहा॒ विष्ण॑वे निभूय॒पाय॒ स्वाहेत्या॑ह। य॒ज्ञो वै विष्णु॑। य॒ज्ञायै॒वैन॒मुद्य॑च्छते। पू॒र्णा॒हु॒तिमु॑त्त॒मां जु॑होति। प्रत्युत्त॑ब्ध्यै सय॒त्वाय॑॥४३॥\anuvakamend[य॒च्छ॒ते॒ पु॒रु॒रूपा॑य॒ स्वाहेत्या॑हा॒ष्टौ च॑]

%3.8.12.1
सा॒वि॒त्रम॒ष्टाक॑पालं प्रा॒तर्निर्व॑पति। अ॒ष्टाक्ष॑रा गाय॒त्री। गा॒य॒त्रं प्रा॑तः सव॒नम्। प्रा॒त॒ स॒व॒नादे॒वैनं॑ गायत्रि॒याश्छन्द॒सोऽधि॒ निर्मि॑मीते। अथो प्रातः सव॒नमे॒व तेनाप्नोति। गा॒य॒त्रीं छन्द॑। स॒वि॒त्रे प्र॑सवि॒त्र एका॑दशकपालं म॒ध्यन्दि॑ने। एका॑दशाक्षरा त्रि॒ष्टुप्। त्रैष्टु॑भं॒ माध्य॑न्दिन॒ सव॑नम्। माध्य॑न्दिनादे॒वैन॒ सव॑नात्रि॒ष्टुभ॒श्छन्द॒सोऽधि॒ निर्मि॑मीते॥४४॥

%3.8.12.2
अथो॒ माध्य॑न्दिनमे॒व सव॑नं॒ तेनाप्नोति। त्रि॒ष्टुभं॒ छन्द॑। स॒वि॒त्र आ॑सवि॒त्रे द्वाद॑शकपालमपरा॒ह्णे। द्वाद॑शाक्षरा॒ जग॑ती। जाग॑तं तृतीयसव॒नम्। तृ॒ती॒य॒स॒व॒नादे॒वैनं॒ जग॑त्या॒श्छन्द॒सोऽधि॒ निर्मि॑मीते। अथो॑ तृतीयसव॒नमे॒व तेनाप्नोति। जग॑तीं॒ छन्द॑। ई॒श्व॒रो वा अश्व॒ प्रमु॑क्त॒ परां परा॒वतं॒ गन्तो। इ॒ह धृति॒ स्वाहे॒ह विधृ॑ति॒ स्वाहे॒ह रन्ति॒ स्वाहे॒ह रम॑ति॒ स्वाहेति॒ चत॑स्र॒ आहु॑तीर्जुहोति॥४५॥

%3.8.12.3
चत॑स्रो॒ दिश॑। दि॒ग्भिरे॒वैनं॒ परि॑गृह्णाति। आश्व॑त्थो व्र॒जो भ॑वति। प्र॒जाप॑तिर्दे॒वेभ्यो॒ निला॑यत। अश्वो॑ रू॒पं कृ॒त्वा। सोऽश्व॒त्थे सं॑वत्स॒रम॑तिष्ठत्। तद॑श्व॒त्थस्याश्वत्थ॒त्वम्। यदाश्व॑त्थो व्र॒जो भव॑ति। स्व ए॒वैनं॒ योनौ॒ प्रति॑ष्ठापयति॥४६॥\anuvakamend[त्रि॒ष्टुभ॒श्छन्द॒सोऽधि॒ निर्मि॑मीते जुहोति॒ नव॑ च]

%3.8.13.1
आ ब्रह्म॑न्ब्राह्म॒णो ब्र॑ह्मवर्च॒सी जा॑यता॒मित्या॑ह। ब्रा॒ह्म॒ण ए॒व ब्र॑ह्मवर्च॒सं द॑धाति। तस्मात्पु॒रा ब्राह्म॒णो ब्र॑ह्मवर्च॒स्य॑जायत। आऽस्मिन्रा॒ष्ट्रे रा॑ज॒न्य॑ इष॒व्य॑ शूरो॑ महार॒थो जा॑यता॒मित्या॑ह। रा॒ज॒न्य॑ ए॒व शौ॒र्यं म॑हि॒मानं॑ दधाति। तस्मात्पु॒रा रा॑ज॒न्य॑ इष॒व्य॑ शूरो॑ महार॒थो॑ऽजायत। दोग्ध्री॑ धे॒नुरित्या॑ह। धे॒न्वामे॒व पयो॑ दधाति। तस्मात्पु॒रा दोग्ध्री॑ धे॒नुर॑जायत। वोढा॑ऽन॒ड्वानित्या॑ह॥४७॥

%3.8.13.2
अ॒न॒डुह्ये॒व वी॒र्यं॑ दधाति। तस्मात्पु॒रा वोढा॑ऽन॒ड्वान॑जायत। आ॒शुः सप्ति॒रित्या॑ह। अश्व॑ ए॒व ज॒वं द॑धाति। तस्मात्पु॒राऽऽशुरश्वो॑ऽजायत। पुर॑न्धि॒र्योषेत्या॑ह। यो॒षित्ये॒व रू॒पं द॑धाति। तस्मा॒त्स्त्री यु॑व॒तिः प्रि॒या भावु॑का। जि॒ष्णू र॑थे॒ष्ठा इत्या॑ह। आ ह॒ वै तत्र॑ जि॒ष्णू र॑थे॒ष्ठा जा॑यते॥४८॥

%3.8.13.3
यत्रै॒तेन॑ य॒ज्ञेन॒ यज॑न्ते। स॒भेयो॒ युवेत्या॑ह। यो वै पूर्ववय॒सी। स स॒भेयो॒ युवा। तस्मा॒द्युवा॒ पुमान्प्रि॒यो भावु॑कः। आऽस्य यज॑मानस्य वी॒रो जा॑यता॒मित्या॑ह। आ ह॒ वै तत्र॒ यज॑मानस्य वी॒रो जा॑यते। यत्रै॒तेन॑ य॒ज्ञेन॒ यज॑न्ते। नि॒का॒मेनि॑कामे नः प॒र्जन्यो॑ वर्\mbox{}ष॒त्वित्या॑ह। नि॒का॒मेनि॑कामे ह॒ वै तत्र॑ प॒र्जन्यो॑ वर्\mbox{}षति। यत्रै॒तेन॑ य॒ज्ञेन॒ यज॑न्ते। फ॒लिन्यो॑ न॒ ओष॑धयः पच्यन्ता॒मित्या॑ह। फ॒लिन्यो॑ ह॒ वै तत्रौष॑धयः पच्यन्ते। यत्रै॒तेन॑ य॒ज्ञेन॒ यज॑न्ते। यो॒ग॒क्षे॒मो न॑ कल्पता॒मित्या॑ह। कल्प॑ते ह॒ वै तत्र॑ प्र॒जाभ्यो॑ योगक्षे॒मः। यत्रै॒तेन॑ य॒ज्ञेन॒ यज॑न्ते ॥४९॥\anuvakamend[अ॒न॒ड्वानित्या॑ह जायते वर्‌षति स॒प्त च॑]

%3.8.14.1
प्र॒जाप॑तिर्दे॒वेभ्यो॑ य॒ज्ञान्व्यादि॑शत्। स आ॒त्मन्न॑श्वमे॒धम॑धत्त। तं दे॒वा अ॑ब्रुवन्। ए॒ष वाव य॒ज्ञः। यद॑श्वमे॒धः। अप्ये॒व नोत्रा॒स्त्विति॑। तेभ्य॑ ए॒तान॑न्नहो॒मान्प्राय॑च्छत्। तान॑जुहोत्। तैर्वै स दे॒वान॑प्रीणात्। यद॑न्नहो॒मां जु॒होति॑॥५०॥

%3.8.14.2
दे॒वाने॒व तैर्यज॑मानः प्रीणाति। आज्ये॑न जुहोति। अ॒ग्नेर्वा ए॒तद्रू॒पम्। यदाज्यम्। यदाज्ये॑न जु॒होति॑। अ॒ग्निमे॒व तत्प्री॑णाति। मधु॑ना जुहोति। म॒ह॒त्यै वा ए॒तद्दे॒वता॑यै रू॒पम्। यन्मधु॑। यन्मधु॑ना जु॒होति॑॥५१॥

%3.8.14.3
म॒ह॒तीमे॒व तद्दे॒वतां प्रीणाति। त॒ण्डु॒लैर्जु॑होति। वसू॑नां॒ वा ए॒तद्रू॒पम्। यत्त॑ण्डु॒लाः। यत्त॑ण्डु॒लैर्जु॒होति॑। वसू॑ने॒व तत्प्री॑णाति। पृथु॑कैर्जुहोति। रु॒द्राणां॒ वा ए॒तद्रू॒पम्। यत्पृथु॑काः। यत्पृथु॑कैर्जु॒होति॑।॥५२॥

%3.8.14.4
रु॒द्राने॒व तत्प्री॑णाति। ला॒जैर्जु॑होति। आ॒दि॒त्यानां॒ वा ए॒तद्रू॒पम्। यल्ला॒जाः। यल्ला॒जैर्जु॒होति॑। आ॒दि॒त्याने॒व तत्प्री॑णाति। क॒रम्बैर्जुहोति। विश्वे॑षां॒ वा ए॒तद्दे॒वाना रू॒पम्। यत्क॒रम्बा। यत्क॒रम्बैर्जु॒होति॑॥५३॥

%3.8.14.5
विश्वा॑ने॒व तद्दे॒वान्प्री॑णाति। धा॒नाभि॑र्जुहोति। नक्ष॑त्राणां॒ वा ए॒तद्रू॒पम्। यद्धा॒नाः। यद्धा॒नाभि॑र्जु॒होति॑। नक्ष॑त्राण्ये॒व तत्प्री॑णाति। सक्तु॑भिर्जुहोति। प्र॒जाप॑ते॒र्वा ए॒तद्रू॒पम्। यत्सक्त॑वः। यत्सक्तु॑भिर्जु॒होति॑॥५४॥

%3.8.14.6
प्र॒जा॑पतिमे॒व तत्प्री॑णाति। म॒सूस्यैर्जुहोति। सर्वा॑सां॒ वा ए॒तद्दे॒वता॑ना रू॒पम्। यन्म॒सूस्या॑नि। यन्म॒सूस्यैर्जु॒होति॑। सर्वा॑ ए॒व तद्दे॒वता प्रीणाति। प्रि॒य॒ङ्गु॒त॒ण्डु॒लैर्जु॑होति। प्रि॒याङ्गा॑ ह॒ वै नामै॒ते। ए॒तैर्वै दे॒वा अश्व॒स्याङ्गा॑नि॒ सम॑दधुः। यत्प्रि॑यङ्गुतण्डु॒लैर्जु॒होति॑। अश्व॑स्यै॒वाङ्गा॑नि॒ संद॑धाति। दशान्ना॑नि जुहोति। दशाक्षरा वि॒राट्। वि॒राट्कृ॒त्स्नस्या॒न्नाद्य॒स्याव॑रुध्यै॥५५॥\anuvakamend[जु॒होति॒ मधु॑ना जु॒होति॒ पृथु॑कैर्जु॒होति॑ क॒रम्बैर्जु॒होति॒ सक्तु॑भिर्जु॒होति॑ प्रियङ्गुतण्डु॒लैर्जु॒होति॑ च॒त्वारि॑ च (अ॒न्नहो॒मानाज्ये॑ना॒ग्नेर्मधु॑ना तण़्डु॒लैः पृथु॑कैर्ला॒जैः क॒रम्बैर्धा॒नाभि॒ सक्तु॑भिर्म॒सूस्यै प्रियङ्गुतण्डु॒लैर्द॒शान्ना॑नि॒ द्वाद॑श। )]

%3.8.15.1
प्र॒जाप॑तिरश्वमे॒धम॑सृजत। त सृ॒ष्ट रक्षास्यजिघासन्। स ए॒तान्प्र॒जाप॑तिर्न॒क्त हो॒मान॑पश्यत्। तान॑जुहोत्। तैर्वै स य॒ज्ञाद्रक्षा॒स्यपा॑हन्। यन्न॑क्त हो॒मां जु॒होति॑। य॒ज्ञादे॒व तैर्यज॑मानो॒ रक्षा॒स्यप॑ हन्ति। आज्ये॑न जुहोति। वज्रो॒ वा आज्यम्। वज्रे॑णै॒व य॒ज्ञाद्रक्षा॒स्यप॑ हन्ति॥५६॥

%3.8.15.2
आज्य॑स्य प्रति॒पदं॑ करोति। प्रा॒णो वा आज्यम्। मु॒ख॒त ए॒वास्य॑ प्रा॒णं द॑धाति। अ॒न्न॒हो॒माञ्जु॑होति। शरी॑रवदे॒वाव॑ रुन्धे। व्य॒त्यासं॑ जुहोति। उ॒भय॒स्याव॑रुध्यै। नक्तं॑ जुहोति। रक्ष॑सा॒मप॑हत्यै। आज्ये॑नान्त॒तो जु॑होति॥५७॥

%3.8.15.3
प्रा॒णो वा आज्यम्। उ॒भ॒यत॑ ए॒वास्य॑ प्रा॒णं द॑धाति। पु॒रस्ताच्चो॒परि॑ष्टाच्च। एक॑स्मै॒ स्वाहेत्या॑ह। अ॒स्मिन्ने॒व लो॒के प्रति॑तिष्ठति। द्वाभ्या॒ स्वाहेत्या॑ह। अ॒मुष्मि॑न्ने॒व लो॒के प्रति॑ तिष्ठति। उ॒भयो॑रे॒व लो॒कयो॒ प्रति॑ तिष्ठति। अ॒स्मिश्चा॒मुष्मिश्च। श॒ताय॒ स्वाहेत्या॑ह। श॒तायु॒र्वै पुरु॑षः श॒तवीर्यः। आयु॑रे॒व वी॒र्य॑मव॑ रुन्धे। स॒हस्रा॑य॒ स्वाहेत्या॑ह। आयु॒र्वै स॒हस्रम्। आयु॑रे॒वाव॑ रुन्धे। सर्व॑स्मै॒ स्वाहेत्या॑ह। अप॑रिमितमे॒वाव॑ रुन्धे॥५८॥\anuvakamend[ए॒व य॒ज्ञाद्रक्षा॒स्यप॑हन्त्यन्त॒तो जु॑होति श॒ताय॒ स्वाहेत्या॑ह स॒प्त च॑]

%3.8.16.1
प्र॒जाप॑तिं॒ वा ए॒ष ईप्स॒तीत्या॑हुः। योऽश्वमे॒धेन॒ यज॑त॒ इति॑। अथो॑ आहुः। सर्वा॑णि भू॒तानीति॑। एक॑स्मै॒ स्वाहेत्या॑ह। प्र॒जाप॑ति॒र्वा एक॑। तमे॒वाप्नो॑ति। एक॑स्मै॒ स्वाहा॒ द्वाभ्या॒ स्वाहेत्य॑भिपू॒र्वमाहु॑तीर्जुहोति। अ॒भि॒पू॒र्वमे॒व सु॑व॒र्गं लो॒कमे॑ति। ए॒को॒त्त॒रं जु॑होति॥५९॥

%3.8.16.2
ए॒क॒वदे॒व सु॑व॒र्गं लो॒कमे॑ति। सन्त॑तं जुहोति। सु॒व॒र्गस्य॑ लो॒कस्य॒ सन्त॑त्यै। श॒ताय॒ स्वाहेत्या॑ह। श॒तायु॒र्वै पुरु॑षः श॒तवीर्यः। आयु॑रे॒व वी॒र्य॑मव॑रुन्धे। स॒हस्रा॑य॒ स्वाहेत्या॑ह। आयु॒र्वै स॒हस्रम्। आयु॑रे॒वाव॑ रुन्धे। अ॒युता॑य॒ स्वाहा॑ नि॒युता॑य॒ स्वाहा प्र॒युता॑य॒ स्वाहेत्या॑ह॥६०॥

%3.8.16.3
त्रय॑ इ॒मे लो॒काः। इ॒माने॒व लो॒कानव॑ रुन्धे। अर्बु॑दाय॒ स्वाहेत्या॑ह। वाग्वा अर्बु॑दम्। वाच॑मे॒वाव॑ रुन्धे। न्य॑र्बुदाय॒ स्वाहेत्या॑ह। यो वै वा॒चो भू॒मा। तन्न्य॑र्बुदम्। वा॒च ए॒व भू॒मान॒मव॑ रुन्धे। स॒मु॒द्राय॒ स्वाहेत्या॑ह ॥६१॥

%3.8.16.4
स॒मु॒द्रमे॒वाप्नो॑ति। मध्या॑य॒ स्वाहेत्या॑ह। मध्य॑मे॒वाप्नो॑ति। अन्ता॑य॒ स्वाहेत्या॑ह। अन्त॑मे॒वाप्नो॑ति। प॒रा॒र्धाय॒ स्वाहेत्या॑ह। प॒रा॒र्धमे॒वाप्नो॑ति। उ॒षसे॒ स्वाहा॒ व्यु॑ष्ट्यै॒ स्वाहेत्या॑ह। रात्रि॒र्वा उ॒षाः। अह॒र्व्यु॑ष्टिः। अ॒हो॒रा॒त्रे ए॒वाव॑रुन्धे। अथो॑ अहोरा॒त्रयो॑रे॒व प्रति॑तिष्ठति। ता यदु॒भयी॒र्दिवा॑ वा॒ नक्तं॑ वा जुहु॒यात्। अ॒हो॒रा॒त्रे मो॑हयेत्। उ॒षसे॒ स्वाहा॒ व्यु॑ष्ट्यै॒ स्वाहो॑देष्य॒ते स्वाहोद्य॒ते स्वाहेत्यनु॑दिते जुहोति। उदि॑ताय॒ स्वाहा॑ सुव॒र्गाय॒ स्वाहा॑ लो॒काय॒ स्वाहेत्युदि॑ते जुहोति। अ॒हो॒रा॒त्रयो॒रव्य॑तिमोहाय॥६२॥\anuvakamend[ए॒को॒त्त॒रं जु॑होति प्र॒युता॑य॒ स्वाहेत्या॑ह समु॒द्राय॒ स्वाहेत्या॒हाह॒र्व्यु॑ष्टिः स॒प्त च॑]

%3.8.17.1
वि॒भूर्मा॒त्रा प्र॒भूः पि॒त्रेत्य॑श्वना॒मानि॑ जुहोति। उ॒भयो॑रे॒वैनं॑ लो॒कयोर्नाम॒धेयं॑ गमयति। आय॑नाय॒ स्वाहा॒ प्राय॑णाय॒ स्वाहेत्यु॑द्द्रा॒वाञ्जु॑होति। सर्व॑मे॒वैन॒मस्क॑न्न सुव॒र्गं लो॒कं ग॑मयति। अ॒ग्नये॒ स्वाहा॒ सोमा॑य॒ स्वाहेति॑ पूर्वहो॒माञ्जु॑होति। पूर्व॑ ए॒व द्वि॒षन्तं॒ भ्रातृ॑व्य॒मति॑ क्रामति। पृ॒थि॒व्यै स्वाहा॒ऽन्तरि॑क्षाय॒ स्वाहेत्या॑ह। य॒था॒य॒जुरे॒वैतत्। अ॒ग्नये॒ स्वाहा॒ सोमा॑य॒ स्वाहेति॑ पूर्वदी॒क्षा जु॑होति। पूर्व॑ ए॒व द्वि॒षन्तं॒ भ्रातृ॑व्य॒मति॑ क्रामति ॥६३॥

%3.8.17.2
पृ॒थि॒व्यै स्वाहा॒ऽन्तरि॑क्षाय॒ स्वाहेत्ये॑कवि॒शिनीं दी॒क्षां जु॑होति। एक॑विशति॒र्वै दे॑वलो॒काः। द्वाद॑श॒ मासा॒ पञ्च॒र्तव॑। त्रय॑ इ॒मे लो॒काः। अ॒सावा॑दि॒त्य ए॑कवि॒शः। ए॒ष सु॑व॒र्गो लो॒कः। सु॒व॒र्गस्य॑ लो॒कस्य॒ सम॑ष्ट्यै। भुवो॑ दे॒वानां॒ कर्म॒णेत्यृ॑तुदी॒क्षा जु॑होति। ऋ॒तूने॒वास्मै॑ कल्पयति। अ॒ग्नये॒ स्वाहा॑ वा॒यवे॒ स्वाहेति॑ जुहो॒त्यन॑न्तरित्यै॥६४॥

%3.8.17.3
अ॒र्वाङ्य॒ज्ञः संक्रा॑म॒त्वित्याप्तीर्जुहोति। सु॒व॒र्गस्य॑ लो॒कस्याप्त्यै। भू॒तं भव्यं॑ भवि॒ष्यदिति॒ पर्याप्तीर्जुहोति। सु॒व॒र्गस्य॑ लो॒कस्य॒ पर्याप्त्यै। आ मे॑ गृ॒हा भ॑व॒न्त्वित्या॒भूर्जु॑होति। सु॒व॒र्गस्य॑ लो॒कस्याभूत्यै। अ॒ग्निना॒ तपोऽन्व॑भव॒दित्य॑नु॒भूर्जु॑होति। सु॒व॒र्गस्य॑ लो॒कस्यानु॑भूत्यै। स्वाहा॒ऽऽधिमाधी॑ताय॒ स्वाहेति॒ सम॑स्तानि वैश्वदे॒वानि॑ जुहोति। सम॑स्तमे॒व द्वि॒षन्तं॒ भ्रातृ॑व्य॒मति॑ क्रामति॥६५॥

%3.8.17.4
द॒द्भ्यः स्वाहा॒ हनूभ्या॒ स्वाहेत्य॑ङ्गहो॒माञ्जु॑होति। अङ्गे॑अङ्गे॒ वै पुरु॑षस्य पा॒प्मोप॑श्लिष्टः। अङ्गा॑दङ्गादे॒वैनं॑ पा॒प्मन॒स्तेन॑ मुञ्चति। अ॒ञ्ज्ये॒ताय॒ स्वाहा॑ कृ॒ष्णाय॒ स्वाहा श्वे॒ताय॒ स्वाहेत्य॑श्वरू॒पाणि॑ जुहोति। रू॒पैरे॒वैन॒ सम॑र्धयति। ओष॑धीभ्य॒ स्वाहा॒ मूलेभ्य॒ स्वाहेत्यो॑षधिहो॒माञ्जु॑होति। द्व॒य्यो वा ओष॑धयः। पुष्पेभ्यो॒ऽन्याः फलं॑ गृ॒ह्णन्ति॑। मूलेभ्यो॒ऽन्याः। ता ए॒वोभयी॒रव॑ रुन्धे॥६६॥

%3.8.17.5
वन॒स्पति॑भ्य॒ स्वाहेति॑ वनस्पतिहो॒माञ्जु॑होति। आ॒र॒ण्यस्या॒न्नाद्य॒स्याव॑रुध्यै। मे॒षस्त्वा॑ पच॒तैर॑व॒त्वित्यपाव्यानि जुहोति। प्रा॒णा वै दे॒वा अपाव्याः। प्रा॒णाने॒वाव॑ रुन्धे। कूप्याभ्य॒ स्वाहा॒द्भ्यः स्वाहेत्य॒पा होमाञ्जुहोति। अ॒प्सु वा आप॑। अन्नं॒ वा आप॑। अ॒द्भ्यो वा अन्नं॑ जायते। यदे॒वाद्भ्योऽन्नं॒ जाय॑ते। तदव॑ रुन्धे॥६७॥\anuvakamend[पू॒र्व॒दी॒क्षा जु॑होति॒ पूर्व॑ ए॒व द्वि॒षन्तं॒ भ्रातृ॑व्य॒मति॑ क्राम॒त्यन॑न्तरित्यै क्रामति रुन्धे॒ जाय॑त॒ एकं॑ च]

%3.8.18.1
अम्भासि जुहोति। अ॒यं वै लो॒कोऽम्भासि। तस्य॒ वस॒वोऽधि॑पतयः। अ॒ग्निर्ज्योति॑। यदम्भासि जु॒होति॑। इ॒ममे॒व लो॒कमव॑ रुन्धे। वसू॑ना॒ सायु॑ज्यं गच्छति। अ॒ग्निं ज्योति॒रव॑ रुन्धे। नभासि जुहोति। अ॒न्तरि॑क्षं॒ वै नभासि॥६८॥

%3.8.18.2
तस्य॑ रु॒द्रा अधि॑पतयः। वा॒युर्ज्योति॑। यन्नभासि जु॒होति॑। अ॒न्तरि॑क्षमे॒वाव॑ रुन्धे। रु॒द्राणा॒ सायु॑ज्यं गच्छति। वा॒युं ज्योति॒रव॑ रुन्धे। महासि जुहोति। अ॒सौ वै लो॒को महासि। तस्या॑दि॒त्या अधि॑पतयः। सूर्यो॒ ज्योति॑॥६९॥

%3.8.18.3
यन्महासि जु॒होति॑। अ॒मुमे॒व लो॒कमव॑ रुन्धे। आ॒दि॒त्याना॒ सायु॑ज्यं गच्छति। सूर्यं॒ ज्योति॒रव॑ रुन्धे। नमो॒ राज्ञे॒ नमो॒ वरु॑णा॒येति॑ य॒व्यानि॑ जुहोति। अ॒न्नाद्य॒स्याव॑रुध्यै। म॒यो॒भूर्वातो॑ अ॒भि वा॑तू॒स्रा इति॑ ग॒व्यानि॑ जुहोति। प॒शू॒नामव॑रुध्यै। प्रा॒णाय॒ स्वाहा व्या॒नाय॒ स्वाहेति॑ सन्ततिहो॒माञ्जु॑होति। सु॒व॒र्गस्य॑ लो॒कस्य॒ संत॑त्यै ॥७०॥

%3.8.18.4
सि॒ताय॒ स्वाहाऽसि॑ताय॒ स्वाहेति॒ प्रमु॑क्तीर्जुहोति। सु॒व॒र्गस्य॑ लो॒कस्य॒ प्रमु॑क्त्यै। पृ॒थि॒व्यै स्वाहा॒ऽन्तरि॑क्षाय॒ स्वाहेत्या॑ह। य॒था॒य॒जुरे॒वैतत्। द॒त्वते॒ स्वाहा॑ऽद॒न्तका॑य॒ स्वाहेति॑ शरीरहो॒माञ्जु॑होति। पि॒तृ॒लो॒कमे॒व तैर्यज॑मानो॒ऽव॑ रुन्धे। कस्त्वा॑ युनक्ति॒ स त्वा॑ युन॒क्त्विति॑ परि॒धीन् यु॑नक्ति। इ॒मे वै लो॒काः प॑रि॒धय॑। इ॒माने॒वास्मै॑ लो॒कान् यु॑नक्ति। सु॒व॒र्गस्य॑ लो॒कस्य॒ सम॑ष्ट्यै॥७१॥

%3.8.18.5
यः प्रा॑ण॒तो य आत्म॒दा इति॑ महि॒मानौ॑ जुहोति। सु॒व॒र्गो वै लो॒को मह॑। सु॒व॒र्गमे॒व ताभ्यां लो॒कं यज॑मा॒नोऽव॑ रुन्धे। आ ब्रह्म॑न्ब्राह्म॒णो ब्र॑ह्मवर्च॒सी जा॑यता॒मिति॒ सम॑स्तानि ब्रह्मवर्च॒सानि॑ जुहोति। ब्र॒ह्म॒व॒र्चसमे॒व तैर्यज॑मा॒नोऽव॑ रुन्धे। जज्ञि॒ बीज॒मिति॑ जुहो॒त्यन॑न्तरित्यै। अ॒ग्नये॒ सम॑नमत्पृथि॒व्यै सम॑नम॒दिति॑ सन्नतिहो॒माञ्जु॑होति। सु॒व॒र्गस्य॑ लो॒कस्य॒ सन्न॑त्यै। भू॒ताय॒ स्वाहा॑ भविष्य॒ते स्वाहेति॑ भूताभ॒व्यौ होमौ॑ जुहोति। अ॒यं वै लो॒को भू॒तम्॥७२॥

%3.8.18.6
अ॒सौ भ॑वि॒ष्यत्। अ॒नयो॑रे॒व लो॒कयो॒ प्रति॑तिष्ठति। सर्व॒स्याप्त्यै। सर्व॒स्याव॑रुध्यै। यदक्र॑न्दः प्रथ॒मं जाय॑मान॒ इत्य॑श्वस्तो॒मीयं॑ जुहोति। सर्व॒स्याप्त्यै। सर्व॑स्य॒ जित्यै। सर्व॑मे॒व तेनाप्नोति। सर्वं॑ जयति। योऽश्वमे॒धेन॒ यज॑ते॥७३॥

%3.8.18.7
य उ॑ चैनमे॒वं वेद॑। य॒ज्ञ रक्षास्यजिघासन्। स ए॒तान्प्र॒जाप॑तिर्नक्तहो॒मान॑पश्यत्। तान॑जुहोत्। तैर्वै स य॒ज्ञाद्रक्षा॒स्यपा॑हन्। यन्न॑क्तहो॒मां जु॒होति॑। य॒ज्ञादे॒व तैर्यज॑मानो॒ रक्षा॒स्यप॑हन्ति। उ॒षसे॒ स्वाहा॒ व्यु॑ष्ट्यै॒ स्वाहेत्य॑न्त॒तो जु॑होति। सु॒व॒र्गस्य॑ लो॒कस्य॒ सम॑ष्ट्यै॥७४॥\anuvakamend[वै नभासि॒ सूर्यो॒ ज्योति॒ सन्त॑त्यै॒ सम॑ष्ट्यै भू॒तं यज॑ते॒ नव॑ च]

%3.8.19.1
ए॒क॒यू॒पो वै॑काद॒शिनी॑ वा। अ॒न्येषां य॒ज्ञानां॒ यूपा॑ भवन्ति। ए॒क॒वि॒शिन्य॑श्वमे॒धस्य॑। सु॒व॒र्गस्य॑ लो॒कस्या॒भिजि॑त्यै। बै॒ल्॒वो वा॑ खादि॒रो वा॑ पाला॒शो वा। अ॒न्येषां यज्ञक्रतू॒नां यूपा॑ भवन्ति। राज्जु॑दाल॒ एक॑विशत्यरत्निरश्वमे॒धस्य॑। सु॒व॒र्गस्य॑ लो॒कस्य॒ सम॑ष्ट्यै। नान्येषां पशू॒नां ते॑ज॒न्या अ॑व॒द्यन्ति॑। अव॑द्य॒न्त्यश्व॑स्य॥७५॥

%3.8.19.2
पा॒प्मा वै ते॑ज॒नी। पा॒प्मनोऽप॑हत्यै। प्ल॒क्ष॒शा॒खाया॑म॒न्येषां पशू॒नाम॑व॒द्यन्ति॑। वे॒त॒स॒शा॒खाया॒मश्व॑स्य। अ॒प्सुयो॑नि॒र्वा अश्व॑। अ॒प्सु॒जो वे॑त॒सः। स्व ए॒वास्य॒ योना॒वव॑ द्यति। यूपे॑षु ग्रा॒म्यान्प॒शून्नि॑यु॒ञ्जन्ति॑। आ॒रो॒केष्वा॑र॒ण्यान्धा॑रयन्ति। प॒शू॒नां व्यावृ॑त्त्यै। आ ग्रा॒म्यान्प॒शूल्लँभ॑न्ते। प्रार॒ण्यान्त्सृ॑जन्ति। पा॒प्मनोऽप॑हत्यै॥७६॥\anuvakamend[अश्व॑स्य॒ व्यावृ॑त्त्यै॒ त्रीणि॑ च]

%3.8.20.1
राज्जु॑दालमग्नि॒ष्ठं मि॑नोति। भ्रू॒ण॒ह॒त्याया॒ अप॑हत्यै। पौतु॑द्रवाव॒भितो॑ भवतः। पुण्य॑स्य ग॒न्धस्याव॑रुध्यै। भ्रू॒ण॒ह॒त्यामे॒वास्मा॑दप॒हत्य॑। पुण्ये॑न ग॒न्धेनो॑भ॒यत॒ परि॑ गृह्णाति। षड्बै॒ल्॒वा भ॑वन्ति। ब्र॒ह्म॒व॒र्च॒सस्याव॑रुध्यै। षट्खा॑दि॒राः। तेज॒सोऽव॑रुध्यै॥७७॥

%3.8.20.2
षट्पा॑ला॒शाः। सो॒म॒पी॒थस्याव॑रुध्यै। एक॑विशति॒ संप॑द्यन्ते। एक॑विशति॒र्वै दे॑वलो॒काः। द्वाद॑श॒ मासा॒ पञ्च॒र्तव॑। त्रय॑ इ॒मे लो॒काः। अ॒सावा॑दि॒त्य एक॑वि॒शः। ए॒ष सु॑व॒र्गो लो॒कः। सु॒व॒र्गस्य॑ लो॒कस्य॒ सम॑ष्ट्यै। श॒तं प॒शवो॑ भवन्ति॥७८॥

%3.8.20.3
श॒तायु॒ पुरु॑षः श॒तेन्द्रि॑यः। आयु॑ष्ये॒वेन्द्रि॒ये प्रति॑ तिष्ठति। सर्वं॒ वा अ॑श्वमे॒ध्याप्नो॑ति। अप॑रिमिता भवन्ति। अप॑रिमित॒स्याव॑रुध्यै। ब्र॒ह्म॒वा॒दिनो॑ वदन्ति। कस्मात्स॒त्यात्। द॒क्षि॒ण॒तोऽन्येषां पशू॒नाम॑व॒द्यन्ति॑। उ॒त्त॒र॒तोऽश्व॒स्येति॑। वा॒रुणो॒ वा अश्व॑॥७९॥

%3.8.20.4
ए॒षा वै वरु॑णस्य॒ दिक्। स्वाया॑मे॒वास्य॑ दि॒श्यव॑द्यति। यदित॑रेषां पशू॒नाम॑व॒द्यति॑। श॒त॒दे॒व॒त्यं॑ तेनाव॑ रुन्धे। चि॒तेऽग्नावधि॑ वैत॒से कटेऽश्वं॑ चिनोति। अ॒प्सुयो॑नि॒र्वा अश्व॑। अ॒प्सु॒जो वे॑त॒सः। स्व ए॒वैनं॒ योनौ॒ प्रति॑ष्ठापयति। पु॒रस्तात्प्र॒त्यञ्चं॑ तूप॒रं चि॑नोति। प॒श्चात्प्रा॒चीनं॑ गोमृ॒गम्॥८०॥

%3.8.20.5
प्रा॒णा॒पा॒नावे॒वास्मिन्त्स॒म्यञ्चौ॑ दधाति। अश्वं॑ तूप॒रं गो॑मृ॒गमिति॑ सर्व॒हुत॑ ए॒ताञ्जु॑होति। ए॒षां लो॒काना॑म॒भिजि॑त्यै। आ॒त्मना॒ऽभि जु॑होति। सात्मा॑नमे॒वैन॒ सत॑नुं करोति। सात्मा॒ऽमुष्मि॑ल्लोँ॒के भ॑वति। य ए॒वं वेद॑। अथो॒ वसो॑रे॒व धारां॒ तेनाव॑ रुन्धे। इ॒लु॒वर्दा॑य॒ स्वाहा॑ बलि॒वर्दा॑य॒ स्वाहेत्या॑ह। सं॒व॒त्स॒रो वा इ॑लु॒वर्द॑। प॒रि॒व॒त्स॒रो ब॑लि॒वर्द॑। सं॒व॒त्स॒रादे॒व प॑रिवत्स॒रादायु॒रव॑ रुन्धे। आयु॑रे॒वास्मि॑न्दधाति। तस्मा॑दश्वमेधया॒जी ज॒रसा॑ वि॒स्रसा॒मुं लो॒कमे॑ति॥८१॥\anuvakamend[तेज॒सोऽव॑रुध्यै भव॒न्त्यश्वो॑ गोमृ॒गमि॑लु॒वर्द॑श्च॒त्वारि॑ च]

%3.8.21.1
ए॒क॒वि॒शोऽग्निर्भ॑वति। ए॒क॒वि॒शः स्तोम॑। एक॑विशति॒र्यूपा। यथा॒ वा अश्वा॑ वर्\mbox{}ष॒भा वा॒ वृषा॑णः सस्फु॒रेर\sn{}। ए॒वमे॒व तत्स्तोमा॒ सस्फु॑रन्ते। यदे॑कवि॒शाः। ते यत्स॑मृ॒च्छेर\sn{}। ह॒न्येतास्य य॒ज्ञः। द्वा॒द॒श ए॒वाग्निः स्या॒दित्या॑हुः। द्वा॒द॒शः स्तोम॑॥८२॥

%3.8.21.2
एका॑दश॒ यूपा। यद्द्वा॑द॒शोऽग्निर्भव॑ति। द्वाद॑श॒ मासा संवत्स॒रः। सं॒व॒त्स॒रेणै॒वास्मा॒ अन्न॒मव॑ रुन्धे। यद्दश॒ यूपा॒ भव॑न्ति। दशाक्षरा वि॒राट्। अन्नं॑ वि॒राट्। वि॒राजै॒वान्नाद्य॒मव॑ रुन्धे। य ए॑काद॒शः। स्तन॑ ए॒वास्यै॒ सः॥८३॥

%3.8.21.3
दु॒ह ए॒वैनां॒ तेन॑। तदा॑हुः। यद्द्वा॑द॒शोऽग्निः स्याद्द्वाद॒शः स्तोम॒ एका॑दश॒ यूपा। यथा॒ स्थूरि॑णा या॒यात्। ता॒दृक्तत्। ए॒क॒वि॒श ए॒वाग्निः स्या॒दित्या॑हुः। ए॒क॒वि॒शः स्तोम॑। एक॑विशति॒र्यूपा। यथा॒ प्रष्टि॑भि॒र्याति॑। ता॒दृगे॒व तत्॥८४॥

%3.8.21.4
यो वा अ॑श्वमे॒धे ति॒स्रः क॒कुभो॒ वेद॑। क॒कुद्ध॒ राज्ञां भवति। ए॒क॒वि॒शोऽग्निर्भ॑वति। ए॒क॒वि॒शः स्तोम॑। एक॑विशति॒र्यूपा। ए॒ता वा अ॑श्वमे॒धे ति॒स्रः क॒कुभ॑। य ए॒वं वेद॑। क॒कुद्ध॒ राज्ञां भवति। यो वा अश्व॑मे॒धे त्रीणि॑ शी॒र्॒षाणि॒ वेद॑। शिरो॑ ह॒ राज्ञां भवति। ए॒क॒वि॒शोऽग्निर्भ॑वति। ए॒क॒वि॒शः स्तोम॑। एक॑विशति॒र्यूपा। ए॒तानि॒ वा अ॑श्वमे॒धे त्रीणि॑ शी॒र्॒षाणि॑। य ए॒वं वेद॑। शिरो॑ ह॒ राज्ञां भवति॥८५॥\anuvakamend[द्वा॒द॒शः स्तोम॒ स ए॒व तच्छिरो॑ ह॒ राज्ञां भवति॒ षट् च॑]

%3.8.22.1
दे॒वा वा अ॑श्वमे॒धे पव॑माने। सु॒व॒र्गं लो॒कं न प्राजा॑नन्। तमश्व॒ प्राजा॑नात्। यद॑श्वमे॒धेऽश्वे॑न॒ मेध्ये॒नोद॑ञ्चो बहिष्पवमा॒न सर्प॑न्ति। सु॒व॒र्गस्य॑ लो॒कस्य॒ प्रज्ञात्यै। न वै म॑नु॒ष्य॑ सुव॒र्गं लो॒कमञ्ज॑सा वेद। अश्वो॒ वै सु॑व॒र्गं लो॒कमञ्ज॑सा वेद। यदु॑द्गा॒तोद्गायेत्। यथा क्षेत्रज्ञो॒ऽन्येन॑ प॒था प्र॑तिपा॒दयेत्। ता॒दृक्तत्॥८६॥

%3.8.22.2
उ॒द्गा॒तार॑मप॒रुध्य॑। अश्व॑मुद्गी॒थाय॑ वृणीते। यथा क्षेत्र॒ज्ञोऽञ्ज॑सा॒ नय॑ति। ए॒वमे॒वैन॒मश्व॑ सुव॒र्गं लो॒कमञ्ज॑सा नयति। पुच्छ॑म॒न्वा र॑भन्ते। सु॒व॒र्गस्य॑ लो॒कस्य॒ सम॑ष्ट्यै। हिं क॑रोति। सामै॒वाक॑। हिं क॑रोति। उ॒द्गी॒थ ए॒वास्य॒ सः॥८७॥

%3.8.22.3
वड॑बा॒ उप॑ रुन्धन्ति। मि॒थु॒न॒त्वाय॒ प्रजात्यै। अथो॒ यथो॑पगा॒तार॑ उप॒गाय॑न्ति। ता॒दृगे॒व तत्। उद॑गासी॒दश्वो॒ मेध्य॒ इत्या॑ह। प्रा॒जा॒प॒त्यो वा अश्व॑। प्र॒जाप॑तिरुद्गी॒थः। उ॒द्गी॒थमे॒वाव॑ रुन्धे। अथो॑ ऋक्सा॒मयो॑रे॒व प्रति॑ तिष्ठति। हिर॑ण्येनो॒पाक॑रोति। ज्योति॒र्वै हिर॑ण्यम्। ज्योति॑रे॒व मु॑ख॒तो द॑धाति। यज॑माने च प्र॒जासु॑ च। अथो॒ हिर॑ण्यज्योतिरे॒व यज॑मानः सुव॒र्गं लो॒कमे॑ति॥८८॥\anuvakamend[तत्स उ॒पाक॑रोति च॒त्वारि॑ च]

%3.8.23.1
पुरु॑षो॒ वै य॒ज्ञः। य॒ज्ञः प्र॒जाप॑तिः। यदश्वे॑ प॒शून्नि॑यु॒ञ्जन्ति॑। य॒ज्ञादे॒व तद्य॒ज्ञं प्रयु॑ङ्क्ते। अश्वं॑ तूप॒रं गो॑मृ॒गम्। तान॑ग्नि॒ष्ठ आल॑भते। से॒ना॒मु॒खमे॒व तत्सश्य॑ति। तस्माद्राजमु॒खं भी॒ष्मं भावु॑कम्। आ॒ग्ने॒यं कृ॒ष्णग्री॑वं पु॒रस्ताल्ल॒लाटे। पू॒र्वा॒ग्निमे॒व तं कु॑रुते॥८९॥

%3.8.23.2
तस्मात्पूर्वा॒ग्निं पु॒रस्तात्स्थापयन्ति। पौ॒ष्णम॒न्वञ्चम्। अन्नं॒ वै पू॒षा। तस्मात्पूर्वा॒ग्नावा॑हा॒र्य॑मा ह॑रन्ति। ऐ॒न्द्रा॒पौ॒ष्णमु॒परि॑ष्टात्। ऐ॒न्द्रो वै रा॑ज॒न्योऽन्नं॑ पू॒षा। अ॒न्नाद्ये॑नै॒वैन॑मुभ॒यत॒ परि॑ गृह्णाति। तस्माद्राज॒न्योऽन्ना॒दो भावु॑कः। आ॒ग्ने॒यौ कृ॒ष्णग्री॑वौ बाहु॒वोः। बा॒हु॒वोरे॒व वी॒र्यं धत्ते॥९०॥

%3.8.23.3
तस्माद्राज॒न्यो॑ बाहुब॒लीभावु॑कः। त्वा॒ष्ट्रौ लो॑मशस॒क्थौ स॒क्थ्योः। स॒क्थ्योरे॒व वी॒र्यं॑ धत्ते। तस्माद्राज॒न्य॑ ऊरुब॒लीभावु॑कः। शि॒ति॒पृ॒ष्ठौ बा॑र्\mbox{}हस्प॒त्यौ पृ॒ष्ठे। ब्र॒ह्म॒व॒र्च॒समे॒वोपरि॑ष्टाद्धत्ते। अथो॑ क॒वचे॑ ए॒वैते अ॒भित॒ पर्यू॑हते। तस्माद्राज॒न्य॑ सन्न॑द्धो वी॒र्यं॑ करोति। धा॒त्रे पृ॑षोद॒रम॒धस्तात्। प्र॒ति॒ष्ठामे॒वैतां कु॑रुते। अथो॑ इ॒यं वै धा॒ता। अ॒स्यामे॒व प्रति॑ तिष्ठति। सौ॒र्यं ब॒लक्षं॒ पुच्छे। उ॒त्से॒धमे॒व तं कु॑रुते। तम्मा॑दुत्से॒धम्भ॒ये प्र॒जा अ॒भिसश्र॑यन्ति॥९१॥\anuvakamend[कु॒रु॒ते॒ ध॒त्ते॒ कु॒रु॒ते॒ पञ्च॑ च]




\prashnaend{सा॒ङ्ग्र॒ह॒ण्या चतु॑ष्टय्यो॒ यो वै यः पि॒तुश्च॒त्वारो॒ यथा॑ नि॒क्तं प्र॒जाप॑तये त्वा॒ यथा॒ प्रोक्षि॑तं वि॒भूरा॑ह प्र॒जाप॑तिरकामयताश्वमे॒धेन॑ प्र॒जाप॑ति॒र्न किञ्च॒न सा॑वि॒त्रमा ब्रह्म॑न्प्र॒जाप॑तिर्दे॒वैभ्य॑ प्र॒जाप॑ती॒ रक्षासि प्र॒जाप॑तिमीप्सति वि॒भूर॑श्वना॒मान्यम्भास्येकयू॒पो राज्जु॑दालमेकवि॒शो दे॒वाः पुरु॑ष॒स्त्रयो॑विशतिः॥२३॥}{सा॒ङ्ग्रह॒ण्या तस्मा॑दश्वमेधया॒जी यत्परि॑मिता॒ यद्य॑ज्ञमु॒खे यो दी॒क्षान्दे॒वाने॒व त्रय॑ इ॒मे सि॒ताय॑ प्राणापा॒नावे॒वास्मि॒न्तस्माद्राज॒न्य॑ एक॑नवतिः॥९१॥}{सा॒ङ्ग्र॒ह॒ण्या सश्र॑यन्ति॥}{हरि॑ ओम्॥}{इति श्रीकृष्णयजुर्वेदीयतैत्तिरीयब्राह्मणे तृतीयाष्टके अष्टमः प्रपाठकः समाप्तः॥}
\clearpage
\sect{नवमः प्रश्नः}
\setcounter{anuvakam}{0}
\dnsub{तैत्तिरीयब्राह्मणे तृतीयाष्टके नवमः प्रपाठकः}

%3.9.1.1
प्र॒जाप॑तिरश्वमे॒धम॑सृजत। सोऽस्मात्सृ॒ष्टोऽपाक्रामत्। तम॑ष्टाद॒शिभि॒रनु॒ प्रायु॑ङ्क्त। तमाप्नोत्। तमा॒प्त्वाऽष्टा॑द॒शिभि॒रवा॑रुन्ध। यद॑ष्टाद॒शिन॑ आल॒भ्यन्ते। य॒ज्ञमे॒व तैरा॒प्त्वा यज॑मा॒नोऽव॑रुन्धे। सं॒व॒त्स॒रस्य॒ वा ए॒षा प्र॑ति॒मा। यद॑ष्टाद॒शिन॑। द्वाद॑श॒ मासा॒ पञ्च॒र्तव॑॥१॥

%3.9.1.2
सं॒व॒त्स॒रोऽष्टाद॒शः। यद॑ष्टाद॒शिन॑ आल॒भ्यन्ते। सं॒व॒त्स॒रमे॒व तैरा॒प्त्वा यज॑मा॒नोऽव॑रुन्धे। अ॒ग्नि॒ष्ठेऽन्यान्प॒शूनु॑पाक॒रोति॑। इत॑रेषु॒ यूपेष्वष्टाद॒शिनोऽजा॑मित्वाय। नव॑न॒वाल॑भ्यन्ते सवीर्य॒त्वाय॑। यदा॑र॒ण्यैः सस्था॒पयेत्। व्यव॑स्येतां पितापु॒त्रौ। व्यध्वा॑नः क्रामेयुः। विदू॑र॒ङ्ग्राम॑योर्ग्रामा॒न्तौ स्या॑ताम्॥२॥

%3.9.1.3
ऋ॒क्षीका पुरुषव्या॒घ्राः प॑रिमो॒षिण॑ आव्या॒धिनी॒स्तस्क॑रा॒ अर॑ण्ये॒ष्वाजा॑येरन्। तदा॑हुः। अप॑शवो॒ वा ए॒ते। यदा॑र॒ण्याः। यदा॑र॒ण्यैः सस्था॒पयेत्। क्षि॒प्रे यज॑मान॒मर॑ण्यं मृ॒त ह॑रेयुः। अर॑ण्यायतना॒ ह्या॑र॒ण्याः प॒शव॒ इति॑। यत्प॒शून्नालभे॑त। अन॑वरुद्धा अस्य प॒शव॑ स्युः। यत्पर्य॑ग्निकृतानुत्सृ॒जेत्॥३॥

%3.9.1.4
य॒ज्ञ॒वे॒श॒सं कु॑र्यात्। यत्प॒शूना॒लभ॑ते। तेनै॒व प॒शूनव॑रुन्धे। यत्पर्य॑ग्निकृतानुत्सृ॒जत्यय॑ज्ञवेशसाय। अव॑रुद्धा अस्य प॒शवो॒ भव॑न्ति। न य॑ज्ञवेश॒सम्भ॑वति। न यज॑मान॒मर॑ण्यम्मृ॒त ह॑रन्ति। ग्रा॒म्यैः स स्था॑पयति। ए॒ते वै प॒शव॒ क्षेमो॒ नाम॑। सं पि॑तापु॒त्रावव॑स्यतः। समध्वा॑नः क्रामन्ति। स॒म॒न्ति॒कङ्ग्राम॑योर्ग्रामा॒न्तौ भ॑वतः। नर्क्षीका पुरुषव्या॒घ्राः प॑रिमो॒षिण॑ आव्या॒धिनी॒स्तस्क॑रा॒ अर॑ण्ये॒ष्वाजा॑यन्ते॥४॥\anuvakamend[ऋ॒तव॑ स्यातामुत्सृ॒जेत्स्य॑त॒स्त्रीणि॑ च]

%3.9.2.1
प्र॒जाप॑तिरकामयतो॒भौ लो॒कावव॑ रुन्धी॒येति॑। स ए॒तानु॒भयान्प॒शून॑पश्यत्। ग्रा॒म्याश्चा॑र॒ण्याश्च॑। तानाल॑भत। तैर्वै स उ॒भौ लो॒काववा॑रुन्ध। ग्रा॒म्यैरे॒व प॒शुभि॑रि॒मं लो॒कमवा॑रुन्ध। आ॒र॒ण्यैर॒मुम्। यद्ग्रा॒म्यान्प॒शूना॒लभ॑ते। इ॒ममे॒व तैर्लो॒कमव॑ रुन्धे। यदा॑र॒ण्यान्॥५॥

%3.9.2.2
अ॒मुन्तैः। अन॑वरुद्धो॒ वा ए॒तस्य॑ संवत्स॒र इत्या॑हुः। य इ॒तइ॑तश्चातुर्मा॒स्यानि॑ संवत्स॒रं प्र॑यु॒ङ्क्त इति॑। ए॒तावा॒न्॒ वै सं॑वत्स॒रः। यच्चा॑तुर्मा॒स्यानि॑। यदे॒ते चा॑तुर्मा॒स्याः प॒शव॑ आल॒भ्यन्ते। प्र॒त्यक्ष॑मे॒व तैः सं॑वत्स॒रं यज॑मा॒नोऽव॑रुन्धे। वि वा ए॒ष प्र॒जया॑ प॒शुभि॑र्‌ऋध्यते। यः सं॑वत्स॒रं प्र॑यु॒ङ्क्ते। सं॒व॒त्स॒रः सु॑व॒र्गो लो॒कः॥६॥

%3.9.2.3
सु॒व॒र्गन्तु लो॒कन्नाप॑राध्नोति। प्र॒जा वै प॒शव॑ एकाद॒शिनी। यदे॒त ऐ॑कादशि॒नाः प॒शव॑ आल॒भ्यन्ते। सा॒क्षादे॒व प्र॒जां प॒शून् यज॑मा॒नोऽव॑रुन्धे। प्र॒जाप॑तिर्वि॒राज॑मसृजत। सा सृ॒ष्टाऽश्व॑मे॒धं प्रावि॑शत्। तान्द॒शिभि॒रनु॒ प्रायु॑ङ्क्त। तामाप्नोत्। तामा॒प्त्वा द॒शिभि॒रवा॑रुन्ध। यद्द॒शिन॑ आल॒भ्यन्ते॥७॥

%3.9.2.4
वि॒राज॑मे॒व तैरा॒प्त्वा यज॑मा॒नोऽव॑रुन्धे। एका॑दश द॒शत॒ आल॑भ्यन्ते। एका॑दशाक्षरा त्रि॒ष्टुप्। त्रैष्टु॑भाः प॒शव॑। प॒शूने॒वाव॑रुन्धे। वै॒श्व॒दे॒वो वा अश्व॑। ना॒ना॒दे॒व॒त्या प॒शवो॑ भवन्ति। अश्व॑स्य सर्व॒त्वाय॑। नाना॑रूपा भवन्ति। तस्मा॒न्नाना॑रूपाः प॒शव॑। ब॒हु॒रू॒पा भ॑वन्ति। तस्माद्बहुरू॒पाः प॒शव॒ समृ॑द्ध्यै॥८॥\anuvakamend[आ॒र॒ण्याल्लोँ॒को द॒शिन॑ आल॒भ्यन्ते॒ नाना॑रूपाः प॒शवो॒ द्वे च॑]

%3.9.3.1
अ॒स्मै वै लो॒काय॑ ग्रा॒म्याः प॒शव॒ आल॑भ्यन्ते। अ॒मुष्मा॑ आर॒ण्याः। यद्ग्रा॒म्यान्प॒शूना॒लभ॑ते। इ॒ममे॒व तैर्लो॒कमव॑रुन्धे। यदा॑र॒ण्यान्। अ॒मुन्तैः। उ॒भयान्प॒शूनाल॑भते। गा॒म्याश्चा॑र॒ण्याश्च॑। उ॒भयोर्लो॒कयो॒रव॑रुद्ध्यै। उ॒भयान्प॒शूनाल॑भते॥९॥

%3.9.3.2
ग्रा॒म्याश्चा॑र॒ण्याश्च॑। उ॒भय॑स्या॒न्नाद्य॒स्याव॑रुद्ध्यै। उ॒भयान्प॒शूनाल॑भते। ग्रा॒म्याश्चा॑र॒ण्याश्च॑। उ॒भये॑षां पशू॒नामव॑रुद्ध्यै। त्रय॑स्त्रयो भवन्ति। त्रय॑ इ॒मे लो॒काः। ए॒षां लो॒काना॒माप्त्यै। ब्र॒ह्म॒वा॒दिनो॑ वदन्ति। कस्मात्स॒त्यात्॥१०॥

%3.9.3.3
अ॒स्मिल्लोँ॒के ब॒हव॒ कामा॒ इति॑। यत्स॑मा॒नीभ्यो॑ दे॒वताभ्यो॒ऽन्येऽन्ये प॒शव॑ आल॒भ्यन्ते। अ॒स्मिन्ने॒व तल्लो॒के कामान्दधाति। तस्मा॑द॒स्मिल्लोँ॒के ब॒हव॒ कामा। त्र॒या॒णान्त्र॑याणा स॒ह व॒पा जु॑होति। त्र्या॑वृतो॒ वै दे॒वाः। त्र्या॑वृत इ॒मे लो॒काः। ए॒षां लो॒काना॒माप्त्यै। ए॒षां लो॒कानां॒ कॢप्त्यै। पर्य॑ग्निकृतानार॒ण्यानुत्सृ॑ज॒न्त्यहिसायै॥११॥\anuvakamend[अव॑रुद्ध्या उ॒भयान्प॒शूनाल॑भते स॒त्यादहिसायै]

%3.9.4.1
यु॒ञ्जन्ति॑ ब्र॒ध्नमित्या॑ह। अ॒सौ वा आ॑दि॒त्यो ब्र॒ध्नः। आ॒दि॒त्यमे॒वास्मै॑ युनक्ति। अ॒रु॒षमित्या॑ह। अ॒ग्निर्वा अ॑रु॒षः। अ॒ग्निमे॒वास्मै॑ युनक्ति। चर॑न्त॒मित्या॑ह। वा॒युर्वै चर\sn{}। वा॒युमे॒वास्मै॑ युनक्ति। परि॑त॒स्थुष॒ इत्या॑ह॥१२॥

%3.9.4.2
इ॒मे वै लो॒काः परि॑त॒स्थुष॑। इ॒माने॒वास्मै॑ लो॒कान् यु॑नक्ति। रोच॑न्ते रोच॒ना दि॒वीत्या॑ह। नक्ष॑त्राणि॒ वै रो॑च॒ना दि॒वि। नक्ष॑त्राण्ये॒वास्मै॑ रोचयति। यु॒ञ्जन्त्य॑स्य॒ काम्येत्या॑ह। कामा॑ने॒वास्मै॑ युनक्ति। हरी॒ विप॑क्ष॒सेत्या॑ह। इ॒मे वै हरी॒ विप॑क्षसा। इ॒मे ए॒वास्मै॑ युनक्ति॥१३॥

%3.9.4.3
शोणा॑ धृ॒ष्णू नृ॒वाह॒सेत्या॑ह। अ॒हो॒रा॒त्रे वै नृ॒वाह॑सा। अ॒हो॒रा॒त्रे ए॒वास्मै॑ युनक्ति। ए॒ता ए॒वास्मै॑ दे॒वता॑ युनक्ति। सु॒व॒र्गस्य॑ लो॒कस्य॒ सम॑ष्ट्यै। के॒तुं कृ॒ण्वन्न॑के॒तव॒ इति॑ ध्व॒जं प्रति॑मुञ्चति। यश॑ ए॒वैन॒ राज्ञाङ्गमयति। जी॒मूत॑स्येव भवति॒ प्रती॑क॒मित्या॑ह। य॒था॒य॒जुरे॒वैतत्। ये ते॒ पन्था॑नः सवितः पू॒र्व्यास॒ इत्य॑ध्व॒र्युर्यज॑मानं वाचयत्य॒भिजि॑त्यै॥१४॥

%3.9.4.4
परा॒ वा ए॒तस्य॑ य॒ज्ञ ए॑ति। यस्य॑ प॒शुरु॒पाकृ॑तो॒ऽन्यत्र॒ वेद्या॒ एति॑। ए॒तस्तो॑तरे॒तेन॑ प॒था पुन॒रश्व॒माव॑र्तयासि न॒ इत्या॑ह। वा॒युर्वै स्तोता। वा॒युमे॒वास्य॑ प॒रस्ताद्दधा॒त्यावृ॑त्त्यै। यथा॒ वै ह॒विषो॑ गृही॒तस्य॒ स्कन्द॑ति। ए॒वं वा ए॒तदश्व॑स्य स्कन्दति। यद॑स्यो॒पाकृ॑तस्य॒ लोमा॑नि॒ शीय॑न्ते। यद्वाले॑षु का॒चाना॒वय॑न्ति। लोमान्ये॒वास्य॒ तत्सम्भ॑रन्ति॥१५॥

%3.9.4.5
भूर्भुव॒ सुव॒रिति॑ प्राजाप॒त्याभि॒राव॑यन्ति। प्रा॒जा॒प॒त्यो वा अश्व॑। स्वयै॒वैनं॑ दे॒वत॑या॒ सम॑र्धयन्ति। भूरिति॒ महि॑षी। भुव॒ इति॑ वा॒वाता। सुव॒रिति॑ परिवृ॒क्ती। ए॒षां लो॒काना॑म॒भिजि॑त्यै। हि॒र॒ण्यया का॒चा भ॑वन्ति। ज्योति॒र्वै हिर॑ण्यम्। रा॒ष्ट्रम॑श्वमे॒धः॥१६॥

%3.9.4.6
ज्योति॑श्चै॒वास्मै॑ रा॒ष्ट्रं च॑ स॒मीची॑ दधाति। स॒हस्र॑म्भवन्ति। स॒हस्र॑सम्मितः सुव॒र्गो लो॒कः। सु॒व॒र्गस्य॑ लो॒कस्या॒भिजि॑त्यै। अप॒ वा ए॒तस्मा॒त्तेज॑ इन्द्रि॒यं प॒शव॒ श्रीः क्रा॑मन्ति। योऽश्वमे॒धेन॒ यज॑ते। वस॑वस्त्वाऽञ्जन्तु गाय॒त्रेण॒ छन्द॒सेति॒ महि॑ष्य॒भ्य॑नक्ति। तेजो॒ वा आज्यम्। तेजो॑ गाय॒त्री। तेज॑सै॒वास्मै॒ तेजोऽव॑रुन्धे॥१७॥

%3.9.4.7
रु॒द्रास्त्वाञ्जन्तु॒ त्रैष्टु॑भेन॒ छन्द॒सेति॑ वा॒वाता। तेजो॒ वा आज्यम्। इ॒न्द्रि॒यन्त्रि॒ष्टुप्। तेज॑सै॒वास्मा॑ इन्द्रि॒यमव॑रुन्धे। आ॒दि॒त्यास्त्वाऽञ्जन्तु॒ जाग॑तेन॒ छन्द॒सेति॑ परिवृ॒क्ती। तेजो॒ वा आज्यम्। प॒शवो॒ जग॑ती। तेज॑सै॒वास्मे॑ प॒शूनव॑रुन्धे। पत्न॑यो॒ऽभ्य॑ञ्जन्ति। श्रि॒या वा ए॒तद्रू॒पम्॥१८॥

%3.9.4.8
यत्पत्न॑यः। श्रिय॑मे॒वास्मि॒न्तद्द॑धति। नास्मा॒त्तेज॑ इन्द्रि॒यं प॒शव॒ श्रीरप॑ क्रामन्ति। लाजी ३ ञ्छाची ३ न् यशो॑म॒माँ(४) इत्यति॑रिक्त॒मन्न॒मश्वा॑यो॒पाह॑रन्ति। प्र॒जामे॒वान्ना॒दीं कु॑र्वते। ए॒तद्दे॑वा॒ अन्न॑मत्तै॒तदन्न॑मद्धि प्रजापत॒ इत्या॑ह। प्र॒जाया॑मे॒वान्नाद्य॑न्दधते। यदि॒ नाव॒जिघ्रेत्। अ॒ग्निः प॒शुरा॑सी॒दित्यव॑घ्रापयेत्। अव॑ है॒व जि॑घ्रति। आक्रान्॑ वा॒जी क्रमै॒रत्य॑क्रमीद्वा॒जी द्यौस्ते॑ पृ॒ष्ठं पृ॑थि॒वी स॒धस्थ॒मित्यश्व॒मनु॑मन्त्रयते। ए॒षां लो॒काना॑म॒भिजि॑त्यै। समि॑द्धो अ॒ञ्जन्कृद॑रं मती॒नामित्यश्व॑स्या॒प्रियो॑ भवन्ति सरूप॒त्याय॑॥१९॥\anuvakamend[परि॑त॒स्थुष॒ इत्या॑हे॒मे ए॒वास्मै॑ युनक्त्य॒भिजि॑त्यै भरन्त्यश्वमे॒धो रु॑न्धे रू॒पञ्जि॑घ्रति॒ त्रीणि॑ च]

%3.9.5.1
तेज॑सा॒ वा ए॒ष ब्र॑ह्मवर्च॒सेन॒ व्यृ॑द्ध्यते। योऽश्वमे॒धेन॒ यज॑ते। होता॑ च ब्र॒ह्मा च॑ ब्र॒ह्मोद्यं॑ वदतः। तेज॑सा चै॒वैनं॑ ब्रह्मवर्च॒सेन॑ च॒ सम॑र्धयतः। द॒क्षि॒ण॒तो ब्र॒ह्मा भ॑वति। द॒क्षि॒ण॒तआ॑यतनो॒ वै ब्र॒ह्मा। बा॒र्॒ह॒स्प॒त्यो वै ब्र॒ह्मा। ब्र॒ह्म॒व॒र्च॒समे॒वास्य॑ दक्षिण॒तो द॑धाति। तस्मा॒द्दक्षि॒णोऽर्धो ब्रह्मवर्च॒सित॑रः। उ॒त्त॒र॒तो होता॑ भवति॥२०॥

%3.9.5.2
उ॒त्त॒र॒तआ॑यतनो॒ वै होता। आ॒ग्ने॒यो वै होता। तेजो॒ वा अ॒ग्निः। तेज॑ ए॒वास्योत्तर॒तो द॑धाति। तस्मा॒दुत्त॒रोऽर्ध॑स्तेज॒स्वित॑रः। यूप॑म॒भितो॑ वदतः। य॒ज॒मा॒न॒दे॒व॒त्यो॑ वै यूप॑। यज॑मानमे॒व तेज॑सा च ब्रह्मवर्च॒सेन॑ च॒ सम॑र्धयतः। कि स्वि॑दासीत्पू॒र्वचि॑त्ति॒रित्या॑ह। द्यौर्वै वृष्टि॑ पू॒र्वचि॑त्तिः॥२१॥

%3.9.5.3
दिव॑मे॒व वृष्टि॒मव॑रुन्धे। कि स्वि॑दासीद्बृ॒हद्वय॒ इत्या॑ह। अश्वो॒ वै बृ॒हद्वय॑। अश्व॑मे॒वाव॑रुन्धे। कि स्वि॑दासीत्पिशङ्गि॒लेत्या॑ह। रात्रि॒र्वै पि॑शङ्गि॒ला। रात्रि॑मे॒वाव॑रुन्धे। कि स्वि॑दासीत्पिलिप्पि॒लेत्या॑ह। श्रीर्वै पि॑लिप्पि॒ला। अ॒न्नाद्य॑मे॒वाव॑रुन्धे॥२२॥

%3.9.5.4
कः स्वि॑देका॒की च॑र॒तीत्या॑ह। अ॒सौ वा आ॑दि॒त्य ए॑का॒की च॑रति। तेज॑ ए॒वाव॑रुन्धे। क उ॑स्विज्जायते॒ पुन॒रित्या॑ह। च॒न्द्रमा॒ वै जा॑यते॒ पुन॑। आयु॑रे॒वाव॑रुन्धे। कि स्वि॑द्धि॒मस्य॑ भेष॒जमित्या॑ह। अ॒ग्निर्वै हि॒मस्य॑ भेष॒जम्। ब्र॒ह्म॒व॒र्च॒समेवाव॑रुन्धे। कि स्वि॑दा॒वप॑नं म॒हदित्या॑ह॥२३॥

%3.9.5.5
अ॒यं वै लो॒क आ॒वप॑नम्म॒हत्। अ॒स्मिन्ने॒व लो॒के प्रति॑तिष्ठति। पृ॒च्छामि॑ त्वा॒ पर॒मन्तं॑ पृथि॒व्या इत्या॑ह। वेदि॒र्वै परोऽन्त॑ पृथि॒व्याः। वेदि॑मे॒वाव॑रुन्धे। पृ॒च्छामि॑ त्वा॒ भुव॑नस्य॒ नाभि॒मित्या॑ह। य॒ज्ञो वै भुव॑नस्य॒ नाभि॑। य॒ज्ञमे॒वाव॑रुन्धे। पृ॒च्छामि॑ त्वा॒ वृष्णो॒ अश्व॑स्य॒ रेत॒ इत्या॑ह। सोमो॒ वै वृष्णो॒ अश्व॑स्य॒ रेत॑। सो॒म॒पी॒थमे॒वाव॑रुन्धे। पृ॒च्छामि॑ वा॒चः प॑र॒मं व्यो॑मेत्या॑ह। ब्रह्म॒ वै वा॒चः प॑र॒मं व्यो॑म। ब्र॒ह्म॒व॒र्च॒समे॒वाव॑रुन्धे॥२४॥\anuvakamend[होता॑ भवति॒ वै वृष्टि॑ पू॒र्वचि॑त्तिर॒न्नाद्य॑मे॒वाव॑रुन्धे म॒हदित्या॑ह॒ सोमो॒ वै वृष्णो॒ अश्व॑स्य॒ रेत॑श्च॒त्वारि॑ च]

%3.9.6.1
अप॒ वा ए॒तस्मात्प्रा॒णाः क्रा॑मन्ति। योऽश्वमे॒धेन॒ यज॑ते। प्रा॒णाय॒ स्वाहा व्या॒नाय॒ स्वाहेति॑ संज्ञ॒प्यमा॑न॒ आहु॑तीर्जुहोति। प्रा॒णाने॒वास्मि॑न्दधाति। नास्मात्प्रा॒णा अप॑क्रामन्ति। अव॑न्ती॒ स्थाव॑न्तीस्त्वाऽवन्तु। प्रि॒यन्त्वा प्रि॒याणाम्। वर्‌षि॑ष्ठ॒माप्या॑नाम्। नि॒धी॒नान्त्वा॑ निधि॒पति हवामहे वसो म॒मेत्या॑ह। अपै॒वास्मै॒ तद्ध्नु॑वते॥२५॥

%3.9.6.2
अथो॑ धु॒वन्त्ये॒वैनम्। अथो॒ न्ये॑वास्मै ह्नुवते। त्रिः परि॑यन्ति। त्रय॑ इ॒मे लो॒काः। ए॒भ्य ए॒वैनं॑ लो॒केभ्यो॑ धुवते। त्रिः पुन॒ परि॑यन्ति। षट्त्संप॑द्यन्ते। षड्वा ऋ॒तव॑। ऋ॒तुभि॑रे॒वैन॑न्धुवते। अप॒ वा ए॒तेभ्य॑ प्रा॒णाः क्रा॑मन्ति॥२६॥

%3.9.6.3
ये य॒ज्ञे धुव॑नन्त॒न्वते। न॒व॒कृत्व॒ परि॑यन्ति। नव॒ वै पुरु॑षे प्रा॒णाः। प्रा॒णाने॒वात्मन्द॑धते। नैभ्य॑ प्रा॒णा अप॑क्रामन्ति। अम्बे॒ अम्बा॒ल्यम्बि॑क॒ इति॒ पत्नी॑मु॒दान॑यति। अह्व॑तै॒वैनाम्। सुभ॑गे॒ काम्पी॑लवासि॒नीत्या॑ह। तप॑ ए॒वैना॒मुप॑नयति। सु॒व॒र्गे लो॒के संप्रोर्ण्वा॑था॒मित्या॑ह॥२७॥

%3.9.6.4
सु॒व॒र्गमे॒वैनां लो॒कं ग॑मयति। आऽहम॑जानि गर्भ॒धमा त्वम॑जाऽसि गर्भ॒धमित्या॑ह। प्र॒जा वै प॒शवो॒ गर्भ॑। प्र॒जामे॒व प॒शूना॒त्मन्ध॑त्ते। दे॒वा वा अ॑श्वमे॒धे पव॑माने। सु॒व॒र्गं लो॒कं न प्राजा॑नन्। तमश्व॒ प्राजा॑नात्। यत्सू॒चीभि॑रसिप॒थान्क॒ल्पय॑न्ति। सु॒व॒र्गस्य॑ लो॒कस्य॒ प्रज्ञात्यै। गा॒य॒त्री त्रि॒ष्टुब्जग॒तीत्या॑ह॥२८॥

%3.9.6.5
य॒था॒य॒जुरे॒वैतत्। त्र॒य्यः सू॒च्यो॑ भवन्ति। अ॒य॒स्मय्यो॑ रज॒ता हरि॑ण्यः। अ॒स्य वै लो॒कस्य॑ रू॒पम॑य॒स्मय्य॑। अ॒न्तरि॑क्षस्य रज॒ताः। दि॒वो हरि॑ण्यः। दिशो॒ वा अ॑य॒स्मय्य॑। अ॒वा॒न्त॒र॒दि॒शा र॑ज॒ताः। ऊ॒र्ध्वा हरि॑ण्यः। दिश॑ ए॒वास्मै॑ कल्पयति। कस्त्वा छ्यति॒ कस्त्वा॒ विशा॒स्तीत्या॒हाहिसायै॥२९॥\anuvakamend[ह्नु॒व॒ते॒ क्रा॒म॒न्त्यू॒र्ण्वा॒था॒मित्या॑ह॒ जग॒तीत्या॑ह कल्पय॒त्येकं च]

%3.9.7.1
अप॒ वा ए॒तस्मा॒च्छ्री रा॒ष्ट्रङ्क्रा॑मति। योऽश्वमे॒धेन॒ यज॑ते। ऊ॒र्ध्वामे॑ना॒मुच्छ्र॑यता॒दित्या॑ह। श्रीर्वै रा॒ष्ट्रम॑श्वमे॒धः। श्रिय॑मे॒वास्मै॑ रा॒ष्ट्रमू॒र्ध्वमुच्छ्र॑यति। वे॒णु॒भा॒रङ्गि॒रावि॒वेत्या॑ह। रा॒ष्ट्रं वै भा॒रः। रा॒ष्ट्रमे॒वास्मै॒ पर्यू॑हति। अथास्या॒ मध्य॑मेधता॒मित्या॑ह। श्रीर्वै रा॒ष्ट्रस्य॒ मध्यम्॥३०॥

%3.9.7.2
श्रिय॑मे॒वाव॑रुन्धे। शी॒ते वाते॑ पु॒नन्नि॒वेत्या॑ह। क्षेमो॒ वै रा॒ष्ट्रस्य॑ शी॒तो वात॑। क्षेम॑मे॒वाव॑रुन्धे। यद्ध॑रि॒णी यव॒मत्तीत्या॑ह। विड्वै ह॑रि॒णी। रा॒ष्ट्रं यव॑। विशं॑ चै॒वास्मै॑ रा॒ष्ट्रं च॑ स॒मीची॑ दधाति। न पु॒ष्टं प॒शु म॑न्यत॒ इत्या॑ह। तस्मा॒द्राजा॑ प॒शून्न पुष्य॑ति॥३१॥

%3.9.7.3
शू॒द्रा यदर्य॑जारा॒ न पोषा॑य धनाय॒तीत्या॑ह। तस्माद्वैशीपु॒त्रन्नाभिषि॑ञ्चन्ते। इ॒यं य॒का श॑कुन्ति॒केत्या॑ह। विड्वै श॑कुन्ति॒का। रा॒ष्ट्रम॑श्वमे॒धः। विशं॑ चै॒वास्मै॑ रा॒ष्ट्रं च॑ स॒मीची॑ दधाति। आ॒हल॒मिति॒ सर्प॒तीत्या॑ह। तस्माद्रा॒ष्ट्राय॒ विश॑ सर्पन्ति। आह॑तङ्ग॒भे पस॒ इत्या॑ह। विड्वै गभ॑॥३२॥

%3.9.7.4
रा॒ष्ट्रं पस॑। रा॒ष्ट्रमे॒व वि॒श्याह॑न्ति। तस्माद्रा॒ष्ट्रं विशं॒ घातु॑कम्। मा॒ता च॑ ते पि॒ता च॑ त॒ इत्या॑ह। इ॒यं वै मा॒ता। अ॒सौ पि॒ता। आ॒भ्यामे॒वैनं॒ परि॑ददाति। अग्रं॑ वृ॒क्षस्य॑ रोहत॒ इत्या॑ह। श्रीर्वै वृ॒क्षस्याग्रम्। श्रि॒यमे॒वाव॑ रुन्धे॥३३॥

%3.9.7.5
प्रसु॑ला॒मीति॑ ते पि॒ता ग॒भे मु॒ष्टिम॑तसय॒दित्या॑ह। विड्वै गभ॑। रा॒ष्ट्रम्मु॒ष्टिः। रा॒ष्ट्रमे॒व वि॒श्याह॑न्ति। तस्माद्रा॒ष्ट्रं विशं॒ घातु॑कम्। अप॒ वा ए॒तेभ्य॑ प्रा॒णाः क्रा॑मन्ति। ये य॒ज्ञेऽपू॑तं॒ वद॑न्ति। द॒धि॒क्राव्ण्णो॑ अकारिष॒मिति॑ सुरभि॒मती॒मृचं॑ वदन्ति। प्रा॒णा वै सु॑र॒भय॑। प्रा॒णाने॒वात्मन्द॑धते। नैभ्य॑ प्रा॒णा अप॑क्रामन्ति। आपो॒ हि ष्ठा म॑यो॒भुव॒ इत्य॒द्भिर्मार्जयन्ते। आपो॒ वै सर्वा॑ दे॒वता। दे॒वता॑भिरे॒वात्मानं॑ पवयन्ते॥३४॥\anuvakamend[रा॒ष्ट्रस्य॒ मध्यं॒ पुष्य॑ति॒ गभो॑ रुन्धे दधते च॒त्वारि॑ च]

%3.9.8.1
प्र॒जाप॑तिः प्र॒जाः सृ॒ष्ट्वा प्रे॒णाऽनु॒ प्रावि॑शत्। ताभ्य॒ पुन॒ सम्भ॑वितु॒न्नाश॑क्नोत्। सोऽब्रवीत्। ऋ॒ध्नव॒दित्सः। यो मे॒तः पुन॑ स॒म्भर॒दिति॑। तन्दे॒वा अ॑श्वमे॒धेनै॒व सम॑भरन्। ततो॒ वै त आर्ध्नुवन्। योऽश्वमे॒धेन॒ यज॑ते। प्र॒जाप॑तिमे॒व सम्भ॑रत्यृ॒ध्नोति॑। पुरु॑ष॒माल॑भते॥३५॥

%3.9.8.2
वै॒रा॒जो वै पुरु॑षः। वि॒राज॑मे॒वाल॑भते। अथो॒ अन्नं॒ वै वि॒राट्। अन्न॑मे॒वाव॑रुन्धे। अश्व॒माल॑भते। प्रा॒जा॒प॒त्यो वा अश्व॑। प्र॒जाप॑तिमे॒वाल॑भते। अथो॒ श्रीर्वा एक॑शफम्। श्रिय॑मे॒वाव॑रुन्धे। गामाल॑भते॥३६॥

%3.9.8.3
य॒ज्ञो वै गौः। य॒ज्ञमे॒वाल॑भते। अथो॒ अन्नं॒ वै गौः। अन्न॑मे॒वाव॑रुन्धे। अ॒जा॒वी आल॑भते भू॒म्ने। अथो॒ पुष्टि॒र्वै भू॒मा। पुष्टि॑मे॒वाव॑रुन्धे। पर्य॑ग्निकृतं॒ पुरु॑षञ्चार॒ण्याश्चोत्सृ॑ज॒न्त्यहिसायै। उ॒भौ वा ए॒तौ प॒शू आल॑भ्येते। यश्चा॑व॒मो यश्च॑ पर॒मः। तेऽस्यो॒भये॑ य॒ज्ञे ब॒द्धाः। अ॒भीष्टा॑ अ॒भिप्री॑ताः। अ॒भिजि॑ता अ॒भिहु॑ता भवन्ति। नैन॑न्द॒ङ्क्ष्णव॑ प॒शवो॑ य॒ज्ञे ब॒द्धाः। अ॒भीष्टा॑ अ॒भिप्री॑ताः। अ॒भिजि॑ता अ॒भिहु॑ता हिसन्ति। योऽश्वमे॒धेन॒ यज॑ते। य उ॑ चैनमे॒वं वेद॑॥३७॥\anuvakamend[ल॒भ॒ते॒ गामाल॑भते पर॒मोऽष्टौ च॑]

%3.9.9.1
प्र॒थ॒मेन॒ वा ए॒ष स्तोमे॑न रा॒ध्वा। च॒तु॒ष्टो॒मेन॑ कृ॒तेनाया॑ना॒मुत्त॒रेह\sn{}। ए॒क॒वि॒शे प्र॑ति॒ष्ठायां॒ प्रति॑ तिष्ठति। ए॒क॒वि॒शात्प्र॑ति॒ष्ठाया॑ ऋ॒तून॒न्वारो॑हति। ऋ॒तवो॒ वै पृ॒ष्ठानि॑। ऋ॒तव॑ संवत्स॒रः। ऋ॒तुष्वे॒व सं॑वत्स॒रे प्र॑ति॒ष्ठाय॑। दे॒वता॑ अ॒भ्यारो॑हति। शक्व॑रयः पृ॒ष्ठम्भ॑वन्त्य॒न्यद॑न्य॒च्छन्द॑। अ॒न्येऽन्ये॒ वा ए॒ते प॒शव॒ आल॑भ्यन्ते॥३८॥

%3.9.9.2
उ॒तेव॑ ग्रा॒म्याः। उ॒तेवा॑र॒ण्याः। अह॑रे॒व रू॒पेण॒ सम॑र्धयति। अथो॒ अह्न॑ ए॒वैष ब॒लिर्‌ह्रि॑यते। तदा॑हुः। अप॑शवो॒ वा ए॒ते। यद॑जा॒वय॑श्चार॒ण्याश्च॑। ए॒ते वै सर्वे॑ प॒शव॑। यद्ग॒व्या इति॑। ग॒व्यान्प॒शूनु॑त्त॒मेऽह॒न्नाल॑भते॥३९॥

%3.9.9.3
तेनै॒वोभयान्प॒शूनव॑रुन्धे। प्रा॒जा॒प॒त्या भ॑वन्ति। अन॑भिजितस्या॒भिजि॑त्यै। सौ॒रीर्नव॑ श्वे॒ता व॒शा अ॑नूब॒न्ध्या॑ भवन्ति। अ॒न्त॒त ए॒व ब्र॑ह्मवर्च॒समव॑रुन्धे। सोमा॑य स्व॒राज्ञे॑ऽनोवा॒हाव॑न॒ड्वाहा॒विति॑ द्व॒न्द्विन॑ प॒शूनाल॑भते। अ॒हो॒रा॒त्राणा॑म॒भिजि॑त्यै। प॒शुभि॒र्वा ए॒ष व्यृ॑ध्यते। योऽश्वमे॒धेन॒ यज॑ते। छ॒ग॒लङ्क॒ल्माष॑ङ्किकिदी॒विं वि॑दी॒गय॒मिति॑ त्वा॒ष्ट्रान्प॒शूना ल॑भते। प॒शुभि॑रे॒वात्मान॒ सम॑र्धयति। ऋ॒तुभि॒र्वा ए॒ष व्यृ॑ध्यते। योऽश्वमे॒धेन॒ यज॑ते। पि॒शङ्गा॒स्त्रयो॑ वास॒न्ता इत्यृ॑तुप॒शूनाल॑भते। ऋ॒तुभि॑रे॒वात्मान॒ सम॑र्धयति। आ वा ए॒ष प॒शुभ्यो॑ वृश्च्यते। योऽश्वमे॒धेन॒ यज॑ते। पर्य॑ग्निकृता॒ उत्सृ॑ज॒न्त्यनाव्रस्काय॥४०॥\anuvakamend[ल॒भ्य॒न्ते॒ ल॒भ॒ते॒ त्वा॒ष्ट्रान्प॒शूनाल॑भते॒ऽष्टौ च॑]

%3.9.10.1
प्र॒जाप॑तिरकामयत म॒हान॑न्ना॒दः स्या॒मिति॑। स ए॒ताव॑श्वमे॒धे म॑हि॒माना॑वपश्यत्। ताव॑गृह्णीत। ततो॒ वै स म॒हान॑न्ना॒दो॑ऽभवत्। यः का॒मये॑त म॒हान॑न्ना॒दः स्या॒मिति॑। स ए॒ताव॑श्वमे॒धे म॑हि॒मानौ॑ गृह्णीत। म॒हाने॒वान्ना॒दो भ॑वति। य॒ज॒मा॒न॒दे॒व॒त्या॑ वै व॒पा। राजा॑ महि॒मा। यद्व॒पाम्म॑हि॒म्नोभ॒यत॑ परि॒यज॑ति। यज॑मानमे॒व रा॒ज्येनो॑भ॒यत॒ परि॑गृह्णाति। पु॒रस्तात्स्वाहाकारा॒ वा अ॒न्ये दे॒वाः। उ॒परि॑ष्टात्स्वाहाकारा अ॒न्ये। ते वा ए॒तेऽश्व॑ ए॒व मेध्य॑ उ॒भयेऽव॑रुध्यन्ते। यद्व॒पाम्म॑हि॒म्नोभ॒यत॑ परि॒यज॑ति। ताने॒वोभयान्प्रीणाति॥४१॥\anuvakamend[प॒रि॒यज॑ति॒ षट्च॑]

%3.9.11.1
वै॒श्व॒दे॒वो वा अश्व॑। तं यत्प्रा॑जाप॒त्यं कु॒र्यात्। या दे॒वता॒ अपि॑भागाः। ता भा॑ग॒धेये॑न॒ व्य॑र्धयेत्। दे॒वताभ्यः स॒मद॑न्दध्यात्। स्ते॒गान्दष्ट्राभ्याम्म॒ण्डूकां॒ जम्भ्ये॑भि॒रिति॑। आज्य॑मव॒दानं॑ कृ॒त्वा प्र॑तिस॒ङ्ख्याय॒माहु॑तीर्जुहोति। या ए॒व दे॒वता॒ अपि॑भागाः। ता भा॑ग॒धेये॑न॒ सम॑र्धयति। न दे॒वताभ्यः स॒मद॑न्दधाति॥४२॥

%3.9.11.2
चतु॑र्दशै॒तान॑नुवा॒काञ्जु॑हो॒त्यन॑न्तरित्यै। प्र॒या॒साय॒ स्वाहेति॑ पञ्चद॒शम्। पञ्च॑दश॒ वा अ॑र्धमा॒सस्य॒ रात्र॑यः। अ॒र्ध॒मा॒स॒शः सं॑वत्स॒र आप्यते। दे॒वा॒सु॒राः संय॑त्ता आसन्। तेऽब्रुवन्न॒ग्नय॑ स्विष्ट॒कृत॑। अश्व॑स्य॒ मेध्य॑स्य व॒यमु॑द्धा॒रमुद्ध॑रामहै। अथै॒तान॒भि भ॑वा॒मेति॑। ते लोहि॑त॒मुद॑हरन्त। ततो॑ दे॒वा अभ॑वन्॥४३॥

%3.9.11.3
पराऽसु॑राः। यत्स्वि॑ष्ट॒कृद्भ्यो॒ लोहि॑तं जु॒होति॒ भ्रातृ॑व्याभिभूत्यै। भव॑त्या॒त्मना। पराऽस्य॒ भ्रातृ॑व्यो भवति। गो॒मृ॒ग॒क॒ण्ठेन॑ प्रथ॒मामाहु॑तिं जुहोति। प॒शवो॒ वै गो॑मृ॒गः। रु॒द्रोऽग्निः स्वि॑ष्ट॒कृत्। रु॒द्रादे॒व प॒शून॒न्तर्द॑धाति। अथो॒ यत्रै॒षाऽऽहु॑तिर्‌हू॒यते। न तत्र॑ रु॒द्रः प॒शून॒भिम॑न्यते॥४४॥

%3.9.11.4
अ॒श्व॒श॒फेन॑ द्वि॒तीया॒माहु॑तिं जुहोति। प॒शवो॒ वा एक॑शफम्। रु॒द्रोऽग्निः स्वि॑ष्ट॒कृत्। रु॒द्रादे॒व प॒शून॒न्तर्द॑धाति। अथो॒ यत्रै॒षाऽऽहु॑तिर्‌हू॒यते। न तत्र॑ रु॒द्रः प॒शून॒भिम॑न्यते। अ॒य॒स्मये॑न कम॒ण्डलु॑ना तृ॒तीयाम्। आहु॑तिं जुहोत्याया॒स्यो॑ वै प्र॒जाः। रु॒द्रोऽग्निः स्वि॑ष्ट॒कृत्। रु॒द्रादे॒व प्र॒जा अ॒न्तर्द॑धाति। अथो॒ यत्रै॒षाऽऽहु॑तिर्‌हू॒यते। न तत्र॑ रु॒द्रः प्र॒जा अ॒भिम॑न्यते॥४५॥\anuvakamend[द॒धा॒त्यभ॑वन्मन्यते प्र॒जा अ॒न्तर्द॑धाति॒ द्वे च॑ ]

%3.9.12.1
अश्व॑स्य॒ वा आल॑ब्धस्य॒ मेध॒ उद॑क्रामत्। तद॑श्वस्तो॒मीय॑मभवत्। यद॑श्वस्तो॒मीयं॑ जु॒होति॑। समे॑धमे॒वैन॒माल॑भते। आज्ये॑न जुहोति। मेधो॒ वा आज्यम्। मेधोऽश्वस्तो॒मीयम्। मेधे॑नै॒वास्मि॒न्मेध॑न्दधाति। षट्त्रिशतं जुहोति। षट्त्रिशदक्षरा बृह॒ती॥४६॥

%3.9.12.2
बार्‌ह॑ताः प॒शव॑। सा प॑शू॒नाम्मात्रा। प॒शूने॒व मात्र॑या॒ सम॑र्धयति। तायद्भूय॑सीर्वा॒ कनी॑यसीर्वा जुहु॒यात्। प॒शून्मात्र॑या॒ व्य॑र्धयेत्। षट्त्रिशतं जुहोति। षट्त्रिशदक्षरा बृह॒ती। बार्‌ह॑ताः प॒शव॑। सा प॑शू॒नाम्मात्रा। प॒शूने॒व मात्र॑या॒ सम॑र्धयति॥४७॥

%3.9.12.3
अ॒श्व॒स्तो॒मीय हु॒त्वा द्वि॒पदा॑ जुहोति। द्वि॒पाद्वै पुरु॑षो॒ द्विप्र॑तिष्ठः। तदे॑नं प्रति॒ष्ठया॒ सम॑र्धयति। तदा॑हुः। अ॒श्व॒स्तो॒मीयं॒ पूर्व होत॒व्याँ ३ न्द्वि॒पदाँ ३ इति॑। अश्वो॒ वा अ॑श्वस्तो॒मीयम्। पुरु॑षो द्वि॒पदा। अ॒श्व॒स्तो॒मीय हु॒त्वा द्वि॒पदा॑ जुहोति। तस्माद्द्वि॒पाच्चतु॑ष्पादमत्ति। अथो द्वि॒पद्ये॒व चतु॑ष्पद॒ प्रति॑ष्ठायपति। द्वि॒पदा॑ हु॒त्वा। नान्यामुत्त॑रा॒माहु॑तिं जुहुयात्। यद॒न्यामुत्त॑रा॒माहु॑तिं जुहु॒यात्। प्र प्र॑ति॒ष्ठायाश्च्यवेत। द्वि॒पदा॑ अन्त॒तो जु॑होति॒ प्रति॑ष्ठित्यै॥४८॥\anuvakamend[बृ॒ह॒त्य॑र्धयति स्थापयति॒ पञ्च॑ च]

%3.9.13.1
प्र॒जाप॑तिरश्वमे॒धम॑सृजत। सोऽस्मात्सृ॒ष्टोऽपाक्रामत्। तं य॑ज्ञक्र॒तुभि॒रन्वैच्छत्। तं य॑ज्ञक्र॒तुभि॒र्नान्व॑विन्दत्। तमिष्टि॑भि॒रन्वैच्छत्। तमिष्टि॑भि॒रन्व॑विन्दत्। तदिष्टी॑नामिष्टि॒त्वम्। यत्सं॑वत्स॒रमिष्टि॑भि॒र्यज॑ते। अश्व॑मे॒व तदन्वि॑च्छति। सा॒वि॒त्रियो॑ भवन्ति॥४९॥

%3.9.13.2
इ॒यं वै स॑वि॒ता। यो वा अ॒स्यान्नश्य॑ति॒ यो नि॒लय॑ते। अ॒स्यां वाव तं वि॑न्दन्ति। न वा इ॒माङ्कश्च॒नेत्या॑हुः। ति॒र्यङ्नोर्ध्वोत्ये॑तुमर्ह॒तीति॑। यत्सा॑वि॒त्रियो॒ भव॑न्ति। स॒वि॒तृप्र॑सूत ए॒वैन॑मिच्छति। ई॒श्व॒रो वा अश्व॒ प्रमु॑क्त॒ परां परा॒वत॒ङ्गन्तो। यत्सा॒यन्धृतीर्जु॒होति॑। अश्व॑स्यै॒व यत्यै॒ धृत्यै॥५०॥ यत्प्रा॒तरिष्टि॑भि॒र्यज॑ते। अश्व॑मे॒व तदन्वि॑च्छति। यत्सा॒यन्धृतीर्जु॒होति॑। अश्व॑स्यै॒व यत्यै॒ धृत्यै। तस्मात्सा॒यं प्र॒जाः क्षे॒म्या॑ भवन्ति। यत्प्रा॒तरिष्टि॑भि॒र्यज॑ते। अश्व॑मे॒व तदन्वि॑च्छति। तस्मा॒द्दिवा॑ नष्टै॒ष ए॑ति। यत्प्रा॒तरिष्टि॑भि॒र्यज॑ते सा॒यन्धृतीर्जु॒होति॑। अ॒हो॒रा॒त्राभ्या॑मे॒वैन॒मन्वि॑च्छति। अथो॑ अहोरा॒त्राभ्या॑मे॒वास्मै॑ योगक्षे॒मङ्क॑ल्पयति॥५१॥\anuvakamend[भ॒व॒न्ति॒ धृत्या॑ एन॒मन्वि॑च्छ॒त्येकं च]

%3.9.14.1
अप॒ वा ए॒तस्मा॒च्छ्री रा॒ष्ट्रङ्क्रा॑मति। योऽश्वमे॒धेन॒ यज॑ते। ब्रा॒ह्म॒णौ वी॑णागा॒थिनौ॑ गायतः। श्रि॒या वा ए॒तद्रू॒पम्। यद्वीणा। श्रिय॑मे॒वास्मि॒न्तद्ध॑त्तः। य॒दा खलु॒ वै पुरु॑ष॒ श्रिय॑मश्ञु॒ते। वीणाऽस्मै वाद्यते। तदा॑हुः। यदु॒भौ ब्राह्म॒णौ गाये॑ताम्॥५२॥

%3.9.14.2
प्र॒भ्रशु॑कास्मा॒च्छ्रीः स्यात्। न वै ब्राह्म॒णे श्री र॑मत॒ इति॑। ब्रा॒ह्म॒णोऽन्यो गायेत्। रा॒ज॒न्योऽन्यः। ब्रह्म॒ वै ब्राह्म॒णः। क्ष॒त्र रा॑ज॒न्य॑। तथा॑ हास्य॒ ब्रह्म॑णा च क्ष॒त्रेण॑ चोभ॒यत॒ श्रीः परि॑गृहीता भवति। तदा॑हुः। यदु॒भौ दिवा॒ गाये॑ताम्। अपास्माद्रा॒ष्ट्रङ्क्रा॑मेत्॥५३॥

%3.9.14.3
न वै ब्रा॑ह्म॒णे रा॒ष्ट्र र॑मत॒ इति॑। य॒दा खलु॒ वै राजा॑ का॒मय॑ते। अथ॑ ब्राह्म॒णञ्जि॑नाति। दिवा ब्राह्म॒णो गा॑येत्। नक्त राज॒न्य॑। ब्रह्म॑णो॒ वै रू॒पमह॑। क्ष॒त्रस्य॒ रात्रि॑। तथा॑ हास्य॒ ब्रह्म॑णा च क्ष॒त्रेण॑ चोभ॒यतो॑ रा॒ष्ट्रं परि॑गृहीतम्भवति। इत्य॑ददा॒ इत्य॑यजथा॒ इत्य॑पच॒ इति॑ ब्राह्म॒णो गायेत्। इ॒ष्टा॒पू॒र्तं वै ब्राह्म॒णस्य॑॥५४॥

%3.9.14.4
इ॒ष्टा॒पू॒र्तेनै॒वैन॒ स सम॑र्धयति। इत्य॑जिना॒ इत्य॑युध्यथा॒ इत्य॒मु स॑ङ्ग्रा॒मम॑ह॒न्निति॑ राज॒न्य॑। यु॒द्धं वै रा॑ज॒न्य॑स्य। यु॒द्धेनै॒वैन॒ स सम॑र्धयति। अकॢ॑प्ता॒ वा ए॒तस्य॒र्तव॒ इत्या॑हुः। योऽश्वमे॒धेन॒ यज॑त॒ इति॑। ति॒स्रोऽन्यो गाय॑ति ति॒स्रोऽन्यः। षट्त्संप॑द्यन्ते। षड्वा ऋ॒तव॑। ऋ॒तूने॒वास्मै॑ कल्पयतः। ताभ्या स॒स्थायाम्। अ॒नो॒यु॒क्ते च॑ श॒ते च॑ ददाति। श॒तायु॒ पुरु॑षः श॒तेन्द्रि॑यः। आयु॑ष्ये॒वेन्द्रि॒ये प्रति॑तिष्ठति॥५५॥\anuvakamend[गाये॑ताङ्क्रामेद्ब्राह्म॒णस्य॑ कल्पयतश्च॒त्वारि॑ च]

%3.9.15.1
सर्वे॑षु॒ वा ए॒षु लो॒केषु॑ मृ॒त्यवो॒ऽन्वाय॑त्ताः। तेभ्यो॒ यदाहु॑ती॒र्न जु॑हु॒यात्। लो॒केलो॑क एनं मृ॒त्युर्वि॑न्देत्। मृ॒त्यवे॒ स्वाहा॑ मृ॒त्यवे॒ स्वाहेत्य॑भिपू॒र्वमाहु॑तीर्जुहोति। लो॒काल्लो॑कादे॒व मृ॒त्युमव॑यजते। नैनं॑ लो॒केलो॑के मृ॒त्युर्वि॑न्दति। यद॒मुष्मै॒ स्वाहा॒ऽमुष्मै॒ स्वाहेति॒ जुह्व॑त्स॒ञ्चक्षी॑त। ब॒हुं मृ॒त्युम॒मित्रं॑ कुर्वीत। मृ॒त्यवे॒ स्वाहेत्येक॑स्मा ए॒वैकां जुहुयात्। एको॒ वा अ॒मुष्मि॑ल्लोँ॒के मृ॒त्युः॥५६॥

%3.9.15.2
अ॒श॒न॒या॒ मृ॒त्युरे॒व। तमे॒वामुष्मि॑ल्लोँ॒केऽव॑यजते। भ्रू॒ण॒ह॒त्यायै स्वाहेत्य॑वभृ॒थ आहु॑तिं जुहोति। भ्रू॒ण॒ह॒त्यामे॒वाव॑ यजते। तदा॑हुः। यद्भ्रू॑णह॒त्या पा॒त्र्याऽथ॑। कस्माद्य॒ज्ञेऽपि॑ क्रियत॒ इति॑। अमृ॑त्यु॒र्वा अ॒न्यो भ्रू॑णह॒त्याया॒ इत्या॑हुः। भ्रू॒ण॒ह॒त्या वाव मृ॒त्युरिति॑। यद्भ्रू॑णह॒त्यायै॒ स्वाहेत्य॑वभृ॒थ आहु॑तिं जु॒होति॑॥५७॥

%3.9.15.3
मृ॒त्युमे॒वाहु॑त्या तर्पयि॒त्वा प॑रि॒पाणं॑ कृ॒त्वा। भ्रू॒ण॒घ्ने भे॑ष॒जं क॑रोति। ए॒ता ह॒ वै मु॑ण्डि॒भ औ॑दन्य॒वः। भ्रू॒ण॒ह॒त्यायै॒ प्राय॑श्चित्तिं वि॒दां च॑कार। यो हा॒स्यापि॑ प्र॒जायां ब्राह्म॒ण हन्ति॑। सर्व॑स्मै॒ तस्मै॑ भेष॒जं क॑रोति। जु॒म्ब॒काय॒ स्वाहेत्य॑वभृ॒थ उ॑त्त॒मामाहु॑तिं जुहोति। वरु॑णो॒ वै जु॑म्ब॒कः। अ॒न्त॒त ए॒व वरु॑ण॒मव॑यजते। ख॒ल॒तेर्वि॑क्लि॒धस्य॑ शु॒क्लस्य॑ पिङ्गा॒क्षस्य॑ मू॒र्धं जु॑होति। ए॒तद्वै वरु॑णस्य रू॒पम्। रू॒पेणै॒व वरु॑ण॒मव॑यजते॥५८॥\anuvakamend[लो॒के मृ॒त्युर्जु॒होति॑ मू॒र्धं जु॑होति॒ द्वे च॑]

%3.9.16.1
वा॒रु॒णो वा अश्व॑। तन्दे॒वत॑या॒ व्य॑र्धयति। यत्प्रा॑जाप॒त्यं क॒रोति॑। नमो॒ राज्ञे॒ नमो॒ वरु॑णा॒येत्या॑ह। वा॒रु॒णो वा अश्व॑। स्वयै॒वैनं॑ दे॒वत॑या॒ सम॑र्धयति। नमोऽश्वा॑य॒ नम॑ प्र॒जाप॑तय॒ इत्या॑ह। प्रा॒जा॒प॒त्यो वा अश्व॑। स्वयै॒वैनं॑ दे॒वत॑या॒ सम॑र्धयति। नमोऽधि॑पतय॒ इत्या॑ह॥५९॥

%3.9.16.2
धर्मो॒ वा अधि॑पतिः। धर्म॑मे॒वाव॑रुन्धे। अधि॑पतिर॒स्यधि॑पतिम्मा कु॒र्वधि॑पतिर॒हं प्र॒जानां भूयास॒मित्या॑ह। अधि॑पतिमे॒वैन समा॒नानां करोति। मान्धे॑हि॒ मयि॑ धे॒हीत्या॑ह। आ॒शिष॑मे॒वैतामाशास्ते। उ॒पाकृ॑ताय॒ स्वाहेत्यु॒पाकृ॑ते जुहोति। आल॑ब्धाय॒ स्वाहेति॒ नियु॑क्ते जुहोति। हु॒ताय॒ स्वाहेति॑ हु॒ते जु॑होति। ए॒षां लो॒काना॑म॒भिजि॑त्यै॥६०॥

%3.9.16.3
प्र वा ए॒ष ए॒भ्यो लो॒केभ्य॑श्च्यवते। योऽश्वमे॒धेन॒ यज॑ते। आ॒ग्ने॒यमैन्द्रा॒ग्नमाश्वि॒नम्। तान्प॒शूल॑भते॒ प्रति॑ष्ठित्यै। यदाग्ने॒यो भव॑ति। अ॒ग्निः सर्वा॑ दे॒वता। दे॒वता॑ ए॒वाव॑रुन्धे। ब्रह्म॒ वा अ॒ग्निः। क्ष॒त्रमिन्द्र॑। यदैन्द्रा॒ग्नो भव॑ति॥६१॥

%3.9.16.4
ब्र॒ह्म॒क्ष॒त्रे ए॒वाव॑रुन्धे। यदाश्वि॒नो भव॑ति। आ॒शिषा॒मव॑रुद्ध्यै। त्रयो॑ भवन्ति। त्रय॑ इ॒मे लो॒काः। ए॒ष्वे॑व लो॒केषु॒ प्रति॑तिष्ठति। अ॒ग्नयेऽहो॒मुचे॒ऽष्टाक॑पाल॒ इति॒ दश॑हविष॒मिष्टिं॒ निर्व॑पति। दशाक्षरा वि॒राट्। अन्नं॑ वि॒राट्। वि॒राजै॒वान्नाद्य॒मव॑रुन्धे। अ॒ग्नेर्म॑न्वे प्रथ॒मस्य॒ प्रचे॑तस॒ इति॑ याज्यानुवा॒क्या॑ भवन्ति सर्व॒त्वाय॑ ॥६२॥\anuvakamend[अधि॑पतय॒ इत्या॑हा॒भि॑जित्या ऐन्द्रा॒ग्नो भव॑ति रुन्ध॒ एकं च]

%3.9.17.1
यद्यश्व॑मुप॒तप॑द्वि॒न्देत्। आ॒ग्ने॒यम॒ष्टाक॑पालं॒ निर्व॑पेत्। सौ॒म्यञ्च॒रुम्। सा॒वि॒त्रम॒ष्टाक॑पालम्। यदाग्ने॒यो भव॑ति। अ॒ग्निः सर्वा॑ दे॒वता। दे॒वता॑भिरे॒वैन॑म्भिषज्यति। यत्सौ॒म्यो भव॑ति। सोमो॒ वा ओष॑धीना॒ राजा। याभ्य॑ ए॒वैनं वि॒न्दति॑॥६३॥

%3.9.17.2
ताभि॑रे॒वैन॑म्भिषज्यति। यत्सा॑वि॒त्रो भव॑ति। स॒वि॒तृप्र॑सूत ए॒वैन॑म्भिषज्यति। ए॒ताभि॑रे॒वैनं॑ दे॒वता॑भिर्भिषज्यति। अ॒ग॒दो है॒व भ॑वति। पौ॒ष्णञ्च॒रुन्निर्व॑पेत्। यदि॑ श्लो॒णः स्यात्। पू॒षा वै श्लौण्य॑स्य भि॒षक्। स ए॒वैन॑म्भिषज्यति। अश्लो॑णो है॒व भ॑वति॥६४॥

%3.9.17.3
रौ॒द्रञ्च॒रुन्निर्व॑पेत्। यदि॑ मह॒ती दे॒वता॑ऽभि॒मन्ये॑त। ए॒त॒द्दे॒व॒त्यो॑ वा अश्व॑। स्वयै॒वैनं॑ दे॒वत॑या भिषज्यति। अ॒ग॒दो है॒व भ॑वति। वै॒श्वा॒न॒रन्द्वाद॑शकपालं॒ निर्व॑पेन्मृगाख॒रे यदि॒ नागच्छेत्। इ॒यं वा अ॒ग्निर्वैश्वान॒रः। इ॒यमे॒वैन॑म॒र्चिभ्यां परि॒रोध॒मान॑यति। आहै॒व सुत्य॒मह॑र्गच्छति। यद्य॑धी॒यात्॥६५॥

%3.9.17.4
अ॒ग्नयेऽहो॒मुचे॒ऽष्टाक॑पालः। सौ॒र्यं पय॑। वा॒य॒व्य॑ आज्य॑भागः। यज॑मानो॒ वा अश्व॑। अह॑सा॒ वा ए॒ष गृ॑ही॒तः। यस्याश्वो॒ मेधा॑य॒ प्रोक्षि॑तो॒ऽध्येति॑। यदहो॒मुचे॑ नि॒र्वप॑ति। अह॑स ए॒व तेन॑ मुच्यते। यज॑मानो॒ वा अश्व॑। रेत॑सा॒ वा ए॒ष व्यृ॑ध्यते॥६६॥

%3.9.17.5
यस्याश्वो॒ मेधा॑य॒ प्रोक्षि॑तो॒ऽध्येति॑। सौ॒र्य रेत॑। यत्सौ॒र्यं पयो॒ भव॑ति। रेत॑सै॒वैन॒ स सम॑र्धयति। यज॑मानो॒ वा अश्व॑। गर्भै॒र्वा ए॒ष व्यृ॑ध्यते। यस्याश्वो॒ मेधा॑य॒ प्रोक्षि॑तो॒ऽध्येति॑। वा॒य॒व्या॑ गर्भा। यद्वा॑य॒व्य॑ आज्य॑भागो॒ भव॑ति। गर्भै॑रे॒वैन॒ स सम॑र्धयति। अथो॒ यस्यै॒षाऽश्व॑मे॒धे प्राय॑श्चित्तिः क्रि॒यते। इ॒ष्ट्वा वसी॑यान्भवति॥६७॥\anuvakamend[वि॒न्दत्यश्लो॑णो है॒व भ॑वत्यधी॒यादृ॑ध्यते॒ गर्भै॑रे॒वैन॒ स सम॑र्धयति॒ द्वे च॑]

%3.9.18.1
तदा॑हुः। द्वाद॑श ब्रह्मौद॒नान्त्सस्थि॑ते॒ निर्व॑पेत्। द्वा॒द॒शभि॒र्वेष्टि॑भिर्यजे॒तेति॑। यदिष्टि॑भि॒र्यजे॑त। उ॒प॒नामु॑क एनं य॒ज्ञः स्यात्। पापी॑या॒स्तु स्यात्। आ॒प्तानि॒ वा ए॒तस्य॒ छन्दासि। य ई॑जा॒नः। तानि॒ क ए॒ताव॑दाशु॒ पुन॒ प्रयुं॑ जी॒तेति॑। सर्वा॒ वै सस्थि॑ते य॒ज्ञे वागाप्यते॥६८॥

%3.9.18.2
साप्ता भ॑वति या॒तयाम्नी। क्रू॒रीकृ॑तेव॒ हि भव॒त्यरु॑ष्कृता। सा न पुन॑ प्र॒युज्येत्या॑हुः। द्वाद॑शै॒व ब्र॑ह्मौद॒नान्त्सस्थि॑ते॒ निर्व॑पेत्। प्र॒जाप॑ति॒र्वा ओ॑द॒नः। य॒ज्ञः प्र॒जाप॑तिः। उ॒प॒नामु॑क एनं य॒ज्ञो भ॑वति। न पापी॑यान्भवति। द्वाद॑श भवन्ति। द्वाद॑श॒मासा संवत्स॒रः। सं॒व॒त्स॒र ए॒व प्रति॑तिष्ठति॥६९॥\anuvakamend[आ॒प्य॒ते॒ सं॒व॒त्स॒र एकं च]

%3.9.19.1
ए॒ष वै वि॒भूर्नाम॑ य॒ज्ञः। सर्व ह॒ वै तत्र॑ वि॒भु भ॑वति। यत्रै॒तेन॑ य॒ज्ञेन॒ यज॑न्ते। ए॒ष वै प्र॒भूर्नाम॑ य॒ज्ञः। सर्व ह॒ वै तत्र॑ प्र॒भु भ॑वति। यत्रै॒तेन॑ य॒ज्ञेन॒ यज॑न्ते। ए॒ष वा ऊर्ज॑स्वा॒न्नाम॑ य॒ज्ञः। सर्व ह॒ वै तत्रोर्ज॑स्वद्भवति। यत्रै॒तेन॑ य॒ज्ञेन॒ यज॑न्ते। ए॒ष वै पय॑स्वा॒न्नाम॑ य॒ज्ञः॥७०॥

%3.9.19.2
सर्व ह॒ वै तत्र॒ पय॑स्वद्भवति। यत्रै॒तेन॑ य॒ज्ञेन॒ यज॑न्ते। ए॒ष वै विधृ॑तो॒ नाम॑ य॒ज्ञः। सर्व ह॒ वै तत्र॒ विधृ॑तम्भवति। यत्रै॒तेन॑ य॒ज्ञेन॒ यज॑न्ते। ए॒ष वै व्यावृ॑त्तो॒ नाम॑ य॒ज्ञः। सर्व ह॒ वै तत्र॒ व्यावृ॑त्तम्भवति। यत्रै॒तेन॑ य॒ज्ञेन॒ यज॑न्ते। ए॒ष वै प्रति॑ष्ठितो॒ नाम॑ य॒ज्ञः। सर्व ह॒ वै तत्र॒ प्रति॑ष्ठितम्भवति॥७१॥

%3.9.19.3
यत्रै॒तेन॑ य॒ज्ञेन॒ यज॑न्ते। ए॒ष वै ते॑ज॒स्वी नाम॑ य॒ज्ञः। सर्व ह॒ वै तत्र॑ तेज॒स्वि भ॑वति। यत्रै॒तेन॑ य॒ज्ञेन॒ यज॑न्ते। ए॒ष वै ब्र॑ह्मवर्च॒सी नाम॑ य॒ज्ञः। आ ह॒ तत्र॑ ब्राह्म॒णो ब्र॑ह्मवर्च॒सी जा॑यते। यत्रै॒तेन॑ य॒ज्ञेन॒ यज॑न्ते। ए॒ष वा अ॑तिव्या॒धी नाम॑ य॒ज्ञः। आ ह॒ वै तत्र॑ राज॒न्यो॑ऽतिव्या॒धी जा॑यते। यत्रै॒तेन॑ य॒ज्ञेन॒ यज॑न्ते। ए॒ष वै दी॒र्घो नाम॑ य॒ज्ञः। दी॒र्घायु॑षो ह॒ वै तत्र॑ मनु॒ष्या॑ भवन्ति। यत्रै॒तेन॑ य॒ज्ञेन॒ यज॑न्ते। ए॒ष वै कॢ॒प्तो नाम॑ य॒ज्ञः। कल्प॑ते ह॒ वै तत्र॑ प्र॒जाभ्यो॑ योगक्षे॒मः। यत्रै॒तेन॑ य॒ज्ञेन॒ यज॑न्ते॥७२॥\anuvakamend[पय॑स्वा॒न्नाम॑ य॒ज्ञः प्रति॑ष्ठितम्भवति॒ यत्रै॒तेन॑ य॒ज्ञेन॒ यज॑न्ते॒ षट्च॑ (ए॒ष वै विभूः प्र॒भूरूर्ज॑स्वा॒न्पय॑स्वा॒न् विधृ॑तो॒ व्यावृ॑त्त॒ प्रति॑ष्ठितस्तेज॒स्वी ब्र॑ह्मवर्च॒स्य॑तिव्या॒धी दी॒र्घः कॢ॒प्तो द्वाद॑श ॥ )]

%3.9.20.1
ता॒र्प्येणाश्व॒ संज्ञ॑पयन्ति। य॒ज्ञो वै ता॒र्प्यम्। य॒ज्ञेनै॒वैन॒ सम॑र्धयन्ति। या॒मेन॒ साम्ना प्रस्तो॒ताऽनूप॑तिष्ठते। य॒म॒लो॒कमे॒वैन॑ङ्गमयति। ता॒र्प्ये च॑ कृत्यधीवा॒से चाश्व॒ संज्ञ॑पयन्ति। ए॒तद्वै प॑शू॒ना रू॒पम्। रू॒पेणै॒व प॒शूनव॑रुन्धे। हि॒र॒ण्य॒क॒शि॒पु भ॑वति। तेज॒सोऽव॑रुद्ध्यै॥७३॥

%3.9.20.2
रु॒क्मो भ॑वति। सु॒व॒र्गस्य॑ लो॒कस्यानु॑ख्यात्यै। अश्वो॑ भवति। प्र॒जाप॑ते॒राप्त्यै। अ॒स्य वै लो॒कस्य॑ रू॒पन्ता॒र्प्यम्। अ॒न्तरि॑क्षस्य कृत्यधीवा॒सः। दि॒वो हि॑रण्यकशि॒पु। आ॒दि॒त्यस्य॑ रु॒क्मः। प्र॒जाप॑ते॒रश्व॑। इ॒ममे॒व लो॒कन्ता॒र्प्येणाप्तोति॥७४॥

%3.9.20.3
अ॒न्तरि॑क्षङ्कृत्यधीवा॒सेन॑। दिव हिरण्यकशि॒पुना। आ॒दि॒त्य रु॒क्मेण॑। अश्वे॑नै॒व मेध्ये॑न प्र॒जाप॑ते॒ सायु॑ज्य सलो॒कता॑माप्नोति। ए॒तासा॑मे॒व दे॒वता॑ना॒ सायु॑ज्यम्। सा॒र्ष्टिता समानलो॒कता॑माप्नोति। योऽश्वमे॒धेन॒ यज॑ते। य उ॑ चैनमे॒वं वेद॑॥७५॥\anuvakamend[अव॑रुध्या आप्नोत्य॒ष्टौ च॑]

%3.9.21.1
आ॒दि॒त्याश्चाङ्गि॑रसश्च सुव॒र्गे लो॒केऽस्पर्धन्त। तेऽङ्गि॑रस आदि॒त्येभ्य॑। अ॒मुमा॑दि॒त्यमश्व श्वे॒तम्भू॒तन्दक्षि॑णामनयन्। तेऽब्रुवन्। यन्नोनेष्ट। स वर्यो॑ भू॒दिति॑। तस्मा॒दश्व॒ सव॒र्येत्याह्व॑यन्ति। तस्माद्य॒ज्ञे वरो॑ दीयते। यत्प्र॒जाप॑ति॒राल॒ब्धोऽश्वोऽभ॑वत्। तस्मा॒दश्वो॒ नाम॑॥७६॥

%3.9.21.2
यच्छ्वय॒दरु॒रासीत्। तस्मा॒दर्वा॒ नाम॑। यत्स॒द्यो वाजान्त्स॒मज॑यत्। तस्माद्वा॒जी नाम॑। यदसु॑राणां लो॒कानाद॑त्त। तस्मा॑दादि॒त्यो नाम॑। अ॒ग्निर्वा अ॑श्वमे॒धस्य॒ योनि॑रा॒यत॑नम्। सूर्यो॒ऽग्नेर्योनि॑रा॒यत॑नम्। यद॑श्वमे॒धेऽग्नौ चित्य॑ उत्तरवे॒दिमु॑प॒वप॑ति। योनि॑मन्तमे॒वैन॑मा॒यत॑नवन्तं करोति॥७७॥

%3.9.21.3
योनि॑माना॒यत॑नवान्भवति। य ए॒वं वेद॑। प्रा॒णा॒पा॒नौ वा ए॒तौ दे॒वानाम्। यद॑र्काश्वमे॒धौ। प्रा॒णा॒पा॒नावे॒वाव॑रुन्धे। ओजो॒ बलं॒ वा ए॒तौ दे॒वानाम्। यद॑र्काश्वमे॒धौ। ओजो॒ बल॑मे॒वाव॑रुन्धे। अ॒ग्निर्वा अ॑श्वमे॒धस्य॒ योनि॑रा॒यत॑नम्। सूर्यो॒ग्नेर्योनि॑रा॒यत॑नम्। यद॑श्वमे॒धेऽग्नौ चित्य॑ उत्तरवे॒दिञ्चि॒नोति॑। ताव॑र्काश्वमे॒धौ। अ॒र्का॒श्व॒मे॒धावे॒वाव॑रुन्धे। अथो॑ अर्काश्वमे॒धयो॑रे॒व प्रति॑तिष्ठति॥७८॥\anuvakamend[नाम॑ करोति॒ सूर्यो॒ऽग्नेर्योनि॑रा॒यत॑नञ्च॒त्वारि॑ च]

%3.9.22.1
प्र॒जाप॑तिं॒ वै दे॒वाः पि॒तरम्। प॒शुम्भू॒तम्मेधा॒याल॑भन्त। तमा॒लभ्योपा॑वसन्। प्रा॒तर्यष्टास्मह॒ इति॑। एकं॒ वा ए॒तद्दे॒वाना॒मह॑। यत्सं॑वत्स॒रः। तस्मा॒दश्व॑ पु॒रस्तात्संवत्स॒र आल॑भ्यते। यत्प्र॒जाप॑ति॒राल॒ब्धोऽश्वोऽभ॑वत्। तस्मा॒दश्व॑। यत्स॒द्यो मेधोऽभ॑वत्॥७९॥

%3.9.22.2
तस्मा॑दश्वमे॒धः। वेदु॒कोऽश्व॑मा॒शुम्भ॑वति। य ए॒वं वेद॑। यद्वै तत्प्र॒जाप॑ति॒राल॒ब्धोऽश्वोऽभ॑वत्। तस्मा॒दश्व॑ प्र॒जाप॑तेः पशू॒नामनु॑रूपतमः। आऽस्य॑ पु॒त्रः प्रति॑रूपो जायते। य ए॒वं वेद॑। सर्वा॑णि भू॒तानि॑ स॒म्भृत्याल॑भते। समे॑नन्दे॒वास्तेज॑से ब्रह्मवर्च॒साय॑ भरन्ति। योऽश्वमे॒धेन॒ यज॑ते॥८०॥

%3.9.22.3
य उ॑ चैनमे॒वं वेद॑। ए॒तद्वै तद्दे॒वा ए॒तान्दे॒वताम्। प॒शुम्भू॒तम्मेधा॒याल॑भन्त। य॒ज्ञमे॒व। य॒ज्ञेन॑ य॒ज्ञम॑यजन्त दे॒वाः। का॒म॒प्रं य॒ज्ञम॑कुर्वत। ते॑ऽमृत॒त्वम॑कामयन्त। ते॑ऽमृत॒त्वम॑गच्छन्। योऽश्वमे॒धेन॒ यज॑ते। दे॒वाना॑मे॒वाय॑नेनैति॥८१॥

%3.9.22.4
प्रा॒जा॒प॒त्येनै॒व य॒ज्ञेन॑ यजते काम॒प्रेण॑। अपु॑नर्मारमे॒व ग॑च्छति। ए॒तस्य॒ वै रू॒पेण॑ पु॒रस्तात्प्राजाप॒त्यमृ॑ष॒भं तू॑प॒रं ब॑हुरू॒पमाल॑भते। सर्वे॑भ्य॒ कामेभ्यः। सर्व॒स्याप्त्यै। सर्व॑स्य॒ जित्यै। सर्व॑मे॒व तेनाप्नोति। सर्वं॑ जयति। योऽश्वमे॒धेन॒ यज॑ते। य उ॑ चैनमे॒वं वेद॑॥८२॥\anuvakamend[मेधोऽभ॑व॒द्यज॑त एति॒ वेद॑]

%3.9.23.1
यो वा अश्व॑स्य॒ मेध्य॑स्य॒ लोम॑नी॒ वेद॑। अश्व॑स्यै॒व मेध्य॑स्य॒ लोमं॑ लोमं जुहोति। अ॒हो॒रा॒त्रे वा अश्व॑स्य॒ मेध्य॑स्य॒ लोम॑नी। यत्सा॒यं प्रा॑तर्जु॒होति॑। अश्व॑स्यै॒व मेध्य॑स्य॒ लोमं॑ लोमं जुहोति। ए॒तद॑नुकृति ह स्म॒ वै पु॒रा। अश्व॑स्य॒ मेध्य॑स्य॒ लोमं॑ लोमं जुह्वति। यो वा अश्व॑स्य॒ मेध्य॑स्य प॒दे वेद॑। अश्व॑स्यै॒व मेध्य॑स्य प॒देप॑दे जुहोति। द॒र्॒श॒पू॒र्ण॒मा॒सौ वा अश्व॑स्य॒ मेध्य॑स्य प॒दे॥८३॥

%3.9.23.2
यद्द॑र्‌शपूर्णमा॒सौ यज॑ते। अश्व॑स्यै॒व मेध्य॑स्य प॒देप॑दे जुहोति। ए॒तद॑नुकृति ह स्म॒ वै पु॒रा। अश्व॑स्य॒ मेध्य॑स्य प॒देप॑दे जुह्वति। यो वा अश्व॑स्य॒ मेध्य॑स्य वि॒वर्त॑नं॒ वेद॑। अश्व॑स्यै॒व मेध्य॑स्य वि॒वर्त॑नेविवर्तने जुहोति। अ॒सौ वा आ॑दि॒त्योऽश्व॑। स आ॑हव॒नीय॒माग॑च्छति। तद्विव॑र्तते। यद॑ग्निहो॒त्रं जु॒होति॑। अश्व॑स्यै॒व मेध्य॑स्य वि॒वर्त॑नेविवर्तने जुहोति। ए॒तद॑नुकृति ह स्म॒ वै पु॒रा। अ॑श्वस्य॒ मेध्य॑स्य वि॒वर्त॑नेविवर्तने जुह्वति॥८४॥\anuvakamend[प॒दे अ॑ग्निहो॒त्रं जु॒होति॒ त्रीणि॑ च]

\prashnaend{प्र॒जाप॑ति॒स्तम॑ष्टादशिभि॑ प्र॒जाप॑तिरकामयतो॒भाव॒स्मै यु॒ञ्जन्ति॒ तेज॒साऽप॑प्राणा अप॒श्रीरू॒र्ध्वां प्र॒जाप॑तिः प्रे॒णाऽनु॑ प्रथ॒मेन॑ प्र॒जाप॑तिरकामयत म॒हान्वैश्वदे॒वो वा अश्वोऽश्व॑स्य प्र॒जाप॑ति॒स्तं य॑ज्ञक्र॒तुभि॒रप॒श्रीर्ब्राह्म॒णौ सर्वे॑षु वारु॒णो यद्यश्व॒न्तदा॑हुरे॒ष वै वि॒भूस्ता॒र्प्येणा॑दि॒त्याः प्र॒जाप॑तिं पि॒तर॒य्योँ वा अश्व॑स्य॒ मेध्य॑स्य॒ लोम॑नी॒ त्रयो॑विशतिः॥२३॥}{प्र॒जाप॑तिर॒स्मिँल्लो॒क उ॑त्तर॒तः श्रिय॑मे॒व प्र॒जाप॑तिरकामयत म॒हान्यत्प्रा॒तः प्र वा ए॒ष ए॒भ्यो लो॒केभ्य॒ सर्व ह॒ वै तत्र॒ पय॑ स्व॒द्य उ॑ चैनमे॒वं वेद॑ च॒त्वार्यशी॑तिः॥८४॥}{प्र॒जाप॑तिरश्वमे॒धं जु॑ह्वति॥}{हरि॑ ओम्॥}{इति श्रीकृष्णयजुर्वेदीयतैत्तिरीयब्राह्मणे तृतीयाष्टके नवमः प्रपाठकः समाप्तः॥}
\clearpage
%%% END ASHTAKAM
