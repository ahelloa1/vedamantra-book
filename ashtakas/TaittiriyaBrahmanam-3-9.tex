\sect{नवमः प्रश्नः}
\setcounter{anuvakam}{0}
\dnsub{तैत्तिरीयब्राह्मणे तृतीयाष्टके नवमः प्रपाठकः}

%3.9.1.1
प्र॒जाप॑तिरश्वमे॒धम॑सृजत। सोऽस्मात्सृ॒ष्टोऽपाक्रामत्। तम॑ष्टाद॒शिभि॒रनु॒ प्रायु॑ङ्क्त। तमाप्नोत्। तमा॒प्त्वाऽष्टा॑द॒शिभि॒रवा॑रुन्ध। यद॑ष्टाद॒शिन॑ आल॒भ्यन्ते। य॒ज्ञमे॒व तैरा॒प्त्वा यज॑मा॒नोऽव॑रुन्धे। सं॒व॒त्स॒रस्य॒ वा ए॒षा प्र॑ति॒मा। यद॑ष्टाद॒शिन॑। द्वाद॑श॒ मासा॒ पञ्च॒र्तव॑॥१॥

%3.9.1.2
सं॒व॒त्स॒रोऽष्टाद॒शः। यद॑ष्टाद॒शिन॑ आल॒भ्यन्ते। सं॒व॒त्स॒रमे॒व तैरा॒प्त्वा यज॑मा॒नोऽव॑रुन्धे। अ॒ग्नि॒ष्ठेऽन्यान्प॒शूनु॑पाक॒रोति॑। इत॑रेषु॒ यूपेष्वष्टाद॒शिनोऽजा॑मित्वाय। नव॑न॒वाल॑भ्यन्ते सवीर्य॒त्वाय॑। यदा॑र॒ण्यैः सस्था॒पयेत्। व्यव॑स्येतां पितापु॒त्रौ। व्यध्वा॑नः क्रामेयुः। विदू॑र॒ङ्ग्राम॑योर्ग्रामा॒न्तौ स्या॑ताम्॥२॥

%3.9.1.3
ऋ॒क्षीका पुरुषव्या॒घ्राः प॑रिमो॒षिण॑ आव्या॒धिनी॒स्तस्क॑रा॒ अर॑ण्ये॒ष्वाजा॑येरन्। तदा॑हुः। अप॑शवो॒ वा ए॒ते। यदा॑र॒ण्याः। यदा॑र॒ण्यैः सस्था॒पयेत्। क्षि॒प्रे यज॑मान॒मर॑ण्यं मृ॒त ह॑रेयुः। अर॑ण्यायतना॒ ह्या॑र॒ण्याः प॒शव॒ इति॑। यत्प॒शून्नालभे॑त। अन॑वरुद्धा अस्य प॒शव॑ स्युः। यत्पर्य॑ग्निकृतानुत्सृ॒जेत्॥३॥

%3.9.1.4
य॒ज्ञ॒वे॒श॒सं कु॑र्यात्। यत्प॒शूना॒लभ॑ते। तेनै॒व प॒शूनव॑रुन्धे। यत्पर्य॑ग्निकृतानुत्सृ॒जत्यय॑ज्ञवेशसाय। अव॑रुद्धा अस्य प॒शवो॒ भव॑न्ति। न य॑ज्ञवेश॒सम्भ॑वति। न यज॑मान॒मर॑ण्यम्मृ॒त ह॑रन्ति। ग्रा॒म्यैः स स्था॑पयति। ए॒ते वै प॒शव॒ क्षेमो॒ नाम॑। सं पि॑तापु॒त्रावव॑स्यतः। समध्वा॑नः क्रामन्ति। स॒म॒न्ति॒कङ्ग्राम॑योर्ग्रामा॒न्तौ भ॑वतः। नर्क्षीका पुरुषव्या॒घ्राः प॑रिमो॒षिण॑ आव्या॒धिनी॒स्तस्क॑रा॒ अर॑ण्ये॒ष्वाजा॑यन्ते॥४॥\anuvakamend[ऋ॒तव॑ स्यातामुत्सृ॒जेत्स्य॑त॒स्त्रीणि॑ च]

%3.9.2.1
प्र॒जाप॑तिरकामयतो॒भौ लो॒कावव॑ रुन्धी॒येति॑। स ए॒तानु॒भयान्प॒शून॑पश्यत्। ग्रा॒म्याश्चा॑र॒ण्याश्च॑। तानाल॑भत। तैर्वै स उ॒भौ लो॒काववा॑रुन्ध। ग्रा॒म्यैरे॒व प॒शुभि॑रि॒मं लो॒कमवा॑रुन्ध। आ॒र॒ण्यैर॒मुम्। यद्ग्रा॒म्यान्प॒शूना॒लभ॑ते। इ॒ममे॒व तैर्लो॒कमव॑ रुन्धे। यदा॑र॒ण्यान्॥५॥

%3.9.2.2
अ॒मुन्तैः। अन॑वरुद्धो॒ वा ए॒तस्य॑ संवत्स॒र इत्या॑हुः। य इ॒तइ॑तश्चातुर्मा॒स्यानि॑ संवत्स॒रं प्र॑यु॒ङ्क्त इति॑। ए॒तावा॒न्॒ वै सं॑वत्स॒रः। यच्चा॑तुर्मा॒स्यानि॑। यदे॒ते चा॑तुर्मा॒स्याः प॒शव॑ आल॒भ्यन्ते। प्र॒त्यक्ष॑मे॒व तैः सं॑वत्स॒रं यज॑मा॒नोऽव॑रुन्धे। वि वा ए॒ष प्र॒जया॑ प॒शुभि॑र्‌ऋध्यते। यः सं॑वत्स॒रं प्र॑यु॒ङ्क्ते। सं॒व॒त्स॒रः सु॑व॒र्गो लो॒कः॥६॥

%3.9.2.3
सु॒व॒र्गन्तु लो॒कन्नाप॑राध्नोति। प्र॒जा वै प॒शव॑ एकाद॒शिनी। यदे॒त ऐ॑कादशि॒नाः प॒शव॑ आल॒भ्यन्ते। सा॒क्षादे॒व प्र॒जां प॒शून् यज॑मा॒नोऽव॑रुन्धे। प्र॒जाप॑तिर्वि॒राज॑मसृजत। सा सृ॒ष्टाऽश्व॑मे॒धं प्रावि॑शत्। तान्द॒शिभि॒रनु॒ प्रायु॑ङ्क्त। तामाप्नोत्। तामा॒प्त्वा द॒शिभि॒रवा॑रुन्ध। यद्द॒शिन॑ आल॒भ्यन्ते॥७॥

%3.9.2.4
वि॒राज॑मे॒व तैरा॒प्त्वा यज॑मा॒नोऽव॑रुन्धे। एका॑दश द॒शत॒ आल॑भ्यन्ते। एका॑दशाक्षरा त्रि॒ष्टुप्। त्रैष्टु॑भाः प॒शव॑। प॒शूने॒वाव॑रुन्धे। वै॒श्व॒दे॒वो वा अश्व॑। ना॒ना॒दे॒व॒त्या प॒शवो॑ भवन्ति। अश्व॑स्य सर्व॒त्वाय॑। नाना॑रूपा भवन्ति। तस्मा॒न्नाना॑रूपाः प॒शव॑। ब॒हु॒रू॒पा भ॑वन्ति। तस्माद्बहुरू॒पाः प॒शव॒ समृ॑द्ध्यै॥८॥\anuvakamend[आ॒र॒ण्याल्लोँ॒को द॒शिन॑ आल॒भ्यन्ते॒ नाना॑रूपाः प॒शवो॒ द्वे च॑]

%3.9.3.1
अ॒स्मै वै लो॒काय॑ ग्रा॒म्याः प॒शव॒ आल॑भ्यन्ते। अ॒मुष्मा॑ आर॒ण्याः। यद्ग्रा॒म्यान्प॒शूना॒लभ॑ते। इ॒ममे॒व तैर्लो॒कमव॑रुन्धे। यदा॑र॒ण्यान्। अ॒मुन्तैः। उ॒भयान्प॒शूनाल॑भते। गा॒म्याश्चा॑र॒ण्याश्च॑। उ॒भयोर्लो॒कयो॒रव॑रुद्ध्यै। उ॒भयान्प॒शूनाल॑भते॥९॥

%3.9.3.2
ग्रा॒म्याश्चा॑र॒ण्याश्च॑। उ॒भय॑स्या॒न्नाद्य॒स्याव॑रुद्ध्यै। उ॒भयान्प॒शूनाल॑भते। ग्रा॒म्याश्चा॑र॒ण्याश्च॑। उ॒भये॑षां पशू॒नामव॑रुद्ध्यै। त्रय॑स्त्रयो भवन्ति। त्रय॑ इ॒मे लो॒काः। ए॒षां लो॒काना॒माप्त्यै। ब्र॒ह्म॒वा॒दिनो॑ वदन्ति। कस्मात्स॒त्यात्॥१०॥

%3.9.3.3
अ॒स्मिल्लोँ॒के ब॒हव॒ कामा॒ इति॑। यत्स॑मा॒नीभ्यो॑ दे॒वताभ्यो॒ऽन्येऽन्ये प॒शव॑ आल॒भ्यन्ते। अ॒स्मिन्ने॒व तल्लो॒के कामान्दधाति। तस्मा॑द॒स्मिल्लोँ॒के ब॒हव॒ कामा। त्र॒या॒णान्त्र॑याणा स॒ह व॒पा जु॑होति। त्र्या॑वृतो॒ वै दे॒वाः। त्र्या॑वृत इ॒मे लो॒काः। ए॒षां लो॒काना॒माप्त्यै। ए॒षां लो॒कानां॒ कॢप्त्यै। पर्य॑ग्निकृतानार॒ण्यानुत्सृ॑ज॒न्त्यहिसायै॥११॥\anuvakamend[अव॑रुद्ध्या उ॒भयान्प॒शूनाल॑भते स॒त्यादहिसायै]

%3.9.4.1
यु॒ञ्जन्ति॑ ब्र॒ध्नमित्या॑ह। अ॒सौ वा आ॑दि॒त्यो ब्र॒ध्नः। आ॒दि॒त्यमे॒वास्मै॑ युनक्ति। अ॒रु॒षमित्या॑ह। अ॒ग्निर्वा अ॑रु॒षः। अ॒ग्निमे॒वास्मै॑ युनक्ति। चर॑न्त॒मित्या॑ह। वा॒युर्वै चर\sn{}। वा॒युमे॒वास्मै॑ युनक्ति। परि॑त॒स्थुष॒ इत्या॑ह॥१२॥

%3.9.4.2
इ॒मे वै लो॒काः परि॑त॒स्थुष॑। इ॒माने॒वास्मै॑ लो॒कान् यु॑नक्ति। रोच॑न्ते रोच॒ना दि॒वीत्या॑ह। नक्ष॑त्राणि॒ वै रो॑च॒ना दि॒वि। नक्ष॑त्राण्ये॒वास्मै॑ रोचयति। यु॒ञ्जन्त्य॑स्य॒ काम्येत्या॑ह। कामा॑ने॒वास्मै॑ युनक्ति। हरी॒ विप॑क्ष॒सेत्या॑ह। इ॒मे वै हरी॒ विप॑क्षसा। इ॒मे ए॒वास्मै॑ युनक्ति॥१३॥

%3.9.4.3
शोणा॑ धृ॒ष्णू नृ॒वाह॒सेत्या॑ह। अ॒हो॒रा॒त्रे वै नृ॒वाह॑सा। अ॒हो॒रा॒त्रे ए॒वास्मै॑ युनक्ति। ए॒ता ए॒वास्मै॑ दे॒वता॑ युनक्ति। सु॒व॒र्गस्य॑ लो॒कस्य॒ सम॑ष्ट्यै। के॒तुं कृ॒ण्वन्न॑के॒तव॒ इति॑ ध्व॒जं प्रति॑मुञ्चति। यश॑ ए॒वैन॒ राज्ञाङ्गमयति। जी॒मूत॑स्येव भवति॒ प्रती॑क॒मित्या॑ह। य॒था॒य॒जुरे॒वैतत्। ये ते॒ पन्था॑नः सवितः पू॒र्व्यास॒ इत्य॑ध्व॒र्युर्यज॑मानं वाचयत्य॒भिजि॑त्यै॥१४॥

%3.9.4.4
परा॒ वा ए॒तस्य॑ य॒ज्ञ ए॑ति। यस्य॑ प॒शुरु॒पाकृ॑तो॒ऽन्यत्र॒ वेद्या॒ एति॑। ए॒तस्तो॑तरे॒तेन॑ प॒था पुन॒रश्व॒माव॑र्तयासि न॒ इत्या॑ह। वा॒युर्वै स्तोता। वा॒युमे॒वास्य॑ प॒रस्ताद्दधा॒त्यावृ॑त्त्यै। यथा॒ वै ह॒विषो॑ गृही॒तस्य॒ स्कन्द॑ति। ए॒वं वा ए॒तदश्व॑स्य स्कन्दति। यद॑स्यो॒पाकृ॑तस्य॒ लोमा॑नि॒ शीय॑न्ते। यद्वाले॑षु का॒चाना॒वय॑न्ति। लोमान्ये॒वास्य॒ तत्सम्भ॑रन्ति॥१५॥

%3.9.4.5
भूर्भुव॒ सुव॒रिति॑ प्राजाप॒त्याभि॒राव॑यन्ति। प्रा॒जा॒प॒त्यो वा अश्व॑। स्वयै॒वैनं॑ दे॒वत॑या॒ सम॑र्धयन्ति। भूरिति॒ महि॑षी। भुव॒ इति॑ वा॒वाता। सुव॒रिति॑ परिवृ॒क्ती। ए॒षां लो॒काना॑म॒भिजि॑त्यै। हि॒र॒ण्यया का॒चा भ॑वन्ति। ज्योति॒र्वै हिर॑ण्यम्। रा॒ष्ट्रम॑श्वमे॒धः॥१६॥

%3.9.4.6
ज्योति॑श्चै॒वास्मै॑ रा॒ष्ट्रं च॑ स॒मीची॑ दधाति। स॒हस्र॑म्भवन्ति। स॒हस्र॑सम्मितः सुव॒र्गो लो॒कः। सु॒व॒र्गस्य॑ लो॒कस्या॒भिजि॑त्यै। अप॒ वा ए॒तस्मा॒त्तेज॑ इन्द्रि॒यं प॒शव॒ श्रीः क्रा॑मन्ति। योऽश्वमे॒धेन॒ यज॑ते। वस॑वस्त्वाऽञ्जन्तु गाय॒त्रेण॒ छन्द॒सेति॒ महि॑ष्य॒भ्य॑नक्ति। तेजो॒ वा आज्यम्। तेजो॑ गाय॒त्री। तेज॑सै॒वास्मै॒ तेजोऽव॑रुन्धे॥१७॥

%3.9.4.7
रु॒द्रास्त्वाञ्जन्तु॒ त्रैष्टु॑भेन॒ छन्द॒सेति॑ वा॒वाता। तेजो॒ वा आज्यम्। इ॒न्द्रि॒यन्त्रि॒ष्टुप्। तेज॑सै॒वास्मा॑ इन्द्रि॒यमव॑रुन्धे। आ॒दि॒त्यास्त्वाऽञ्जन्तु॒ जाग॑तेन॒ छन्द॒सेति॑ परिवृ॒क्ती। तेजो॒ वा आज्यम्। प॒शवो॒ जग॑ती। तेज॑सै॒वास्मे॑ प॒शूनव॑रुन्धे। पत्न॑यो॒ऽभ्य॑ञ्जन्ति। श्रि॒या वा ए॒तद्रू॒पम्॥१८॥

%3.9.4.8
यत्पत्न॑यः। श्रिय॑मे॒वास्मि॒न्तद्द॑धति। नास्मा॒त्तेज॑ इन्द्रि॒यं प॒शव॒ श्रीरप॑ क्रामन्ति। लाजी ३ ञ्छाची ३ न् यशो॑म॒माँ(४) इत्यति॑रिक्त॒मन्न॒मश्वा॑यो॒पाह॑रन्ति। प्र॒जामे॒वान्ना॒दीं कु॑र्वते। ए॒तद्दे॑वा॒ अन्न॑मत्तै॒तदन्न॑मद्धि प्रजापत॒ इत्या॑ह। प्र॒जाया॑मे॒वान्नाद्य॑न्दधते। यदि॒ नाव॒जिघ्रेत्। अ॒ग्निः प॒शुरा॑सी॒दित्यव॑घ्रापयेत्। अव॑ है॒व जि॑घ्रति। आक्रान्॑ वा॒जी क्रमै॒रत्य॑क्रमीद्वा॒जी द्यौस्ते॑ पृ॒ष्ठं पृ॑थि॒वी स॒धस्थ॒मित्यश्व॒मनु॑मन्त्रयते। ए॒षां लो॒काना॑म॒भिजि॑त्यै। समि॑द्धो अ॒ञ्जन्कृद॑रं मती॒नामित्यश्व॑स्या॒प्रियो॑ भवन्ति सरूप॒त्याय॑॥१९॥\anuvakamend[परि॑त॒स्थुष॒ इत्या॑हे॒मे ए॒वास्मै॑ युनक्त्य॒भिजि॑त्यै भरन्त्यश्वमे॒धो रु॑न्धे रू॒पञ्जि॑घ्रति॒ त्रीणि॑ च]

%3.9.5.1
तेज॑सा॒ वा ए॒ष ब्र॑ह्मवर्च॒सेन॒ व्यृ॑द्ध्यते। योऽश्वमे॒धेन॒ यज॑ते। होता॑ च ब्र॒ह्मा च॑ ब्र॒ह्मोद्यं॑ वदतः। तेज॑सा चै॒वैनं॑ ब्रह्मवर्च॒सेन॑ च॒ सम॑र्धयतः। द॒क्षि॒ण॒तो ब्र॒ह्मा भ॑वति। द॒क्षि॒ण॒तआ॑यतनो॒ वै ब्र॒ह्मा। बा॒र्॒ह॒स्प॒त्यो वै ब्र॒ह्मा। ब्र॒ह्म॒व॒र्च॒समे॒वास्य॑ दक्षिण॒तो द॑धाति। तस्मा॒द्दक्षि॒णोऽर्धो ब्रह्मवर्च॒सित॑रः। उ॒त्त॒र॒तो होता॑ भवति॥२०॥

%3.9.5.2
उ॒त्त॒र॒तआ॑यतनो॒ वै होता। आ॒ग्ने॒यो वै होता। तेजो॒ वा अ॒ग्निः। तेज॑ ए॒वास्योत्तर॒तो द॑धाति। तस्मा॒दुत्त॒रोऽर्ध॑स्तेज॒स्वित॑रः। यूप॑म॒भितो॑ वदतः। य॒ज॒मा॒न॒दे॒व॒त्यो॑ वै यूप॑। यज॑मानमे॒व तेज॑सा च ब्रह्मवर्च॒सेन॑ च॒ सम॑र्धयतः। कि स्वि॑दासीत्पू॒र्वचि॑त्ति॒रित्या॑ह। द्यौर्वै वृष्टि॑ पू॒र्वचि॑त्तिः॥२१॥

%3.9.5.3
दिव॑मे॒व वृष्टि॒मव॑रुन्धे। कि स्वि॑दासीद्बृ॒हद्वय॒ इत्या॑ह। अश्वो॒ वै बृ॒हद्वय॑। अश्व॑मे॒वाव॑रुन्धे। कि स्वि॑दासीत्पिशङ्गि॒लेत्या॑ह। रात्रि॒र्वै पि॑शङ्गि॒ला। रात्रि॑मे॒वाव॑रुन्धे। कि स्वि॑दासीत्पिलिप्पि॒लेत्या॑ह। श्रीर्वै पि॑लिप्पि॒ला। अ॒न्नाद्य॑मे॒वाव॑रुन्धे॥२२॥

%3.9.5.4
कः स्वि॑देका॒की च॑र॒तीत्या॑ह। अ॒सौ वा आ॑दि॒त्य ए॑का॒की च॑रति। तेज॑ ए॒वाव॑रुन्धे। क उ॑स्विज्जायते॒ पुन॒रित्या॑ह। च॒न्द्रमा॒ वै जा॑यते॒ पुन॑। आयु॑रे॒वाव॑रुन्धे। कि स्वि॑द्धि॒मस्य॑ भेष॒जमित्या॑ह। अ॒ग्निर्वै हि॒मस्य॑ भेष॒जम्। ब्र॒ह्म॒व॒र्च॒समेवाव॑रुन्धे। कि स्वि॑दा॒वप॑नं म॒हदित्या॑ह॥२३॥

%3.9.5.5
अ॒यं वै लो॒क आ॒वप॑नम्म॒हत्। अ॒स्मिन्ने॒व लो॒के प्रति॑तिष्ठति। पृ॒च्छामि॑ त्वा॒ पर॒मन्तं॑ पृथि॒व्या इत्या॑ह। वेदि॒र्वै परोऽन्त॑ पृथि॒व्याः। वेदि॑मे॒वाव॑रुन्धे। पृ॒च्छामि॑ त्वा॒ भुव॑नस्य॒ नाभि॒मित्या॑ह। य॒ज्ञो वै भुव॑नस्य॒ नाभि॑। य॒ज्ञमे॒वाव॑रुन्धे। पृ॒च्छामि॑ त्वा॒ वृष्णो॒ अश्व॑स्य॒ रेत॒ इत्या॑ह। सोमो॒ वै वृष्णो॒ अश्व॑स्य॒ रेत॑। सो॒म॒पी॒थमे॒वाव॑रुन्धे। पृ॒च्छामि॑ वा॒चः प॑र॒मं व्यो॑मेत्या॑ह। ब्रह्म॒ वै वा॒चः प॑र॒मं व्यो॑म। ब्र॒ह्म॒व॒र्च॒समे॒वाव॑रुन्धे॥२४॥\anuvakamend[होता॑ भवति॒ वै वृष्टि॑ पू॒र्वचि॑त्तिर॒न्नाद्य॑मे॒वाव॑रुन्धे म॒हदित्या॑ह॒ सोमो॒ वै वृष्णो॒ अश्व॑स्य॒ रेत॑श्च॒त्वारि॑ च]

%3.9.6.1
अप॒ वा ए॒तस्मात्प्रा॒णाः क्रा॑मन्ति। योऽश्वमे॒धेन॒ यज॑ते। प्रा॒णाय॒ स्वाहा व्या॒नाय॒ स्वाहेति॑ संज्ञ॒प्यमा॑न॒ आहु॑तीर्जुहोति। प्रा॒णाने॒वास्मि॑न्दधाति। नास्मात्प्रा॒णा अप॑क्रामन्ति। अव॑न्ती॒ स्थाव॑न्तीस्त्वाऽवन्तु। प्रि॒यन्त्वा प्रि॒याणाम्। वर्‌षि॑ष्ठ॒माप्या॑नाम्। नि॒धी॒नान्त्वा॑ निधि॒पति हवामहे वसो म॒मेत्या॑ह। अपै॒वास्मै॒ तद्ध्नु॑वते॥२५॥

%3.9.6.2
अथो॑ धु॒वन्त्ये॒वैनम्। अथो॒ न्ये॑वास्मै ह्नुवते। त्रिः परि॑यन्ति। त्रय॑ इ॒मे लो॒काः। ए॒भ्य ए॒वैनं॑ लो॒केभ्यो॑ धुवते। त्रिः पुन॒ परि॑यन्ति। षट्त्संप॑द्यन्ते। षड्वा ऋ॒तव॑। ऋ॒तुभि॑रे॒वैन॑न्धुवते। अप॒ वा ए॒तेभ्य॑ प्रा॒णाः क्रा॑मन्ति॥२६॥

%3.9.6.3
ये य॒ज्ञे धुव॑नन्त॒न्वते। न॒व॒कृत्व॒ परि॑यन्ति। नव॒ वै पुरु॑षे प्रा॒णाः। प्रा॒णाने॒वात्मन्द॑धते। नैभ्य॑ प्रा॒णा अप॑क्रामन्ति। अम्बे॒ अम्बा॒ल्यम्बि॑क॒ इति॒ पत्नी॑मु॒दान॑यति। अह्व॑तै॒वैनाम्। सुभ॑गे॒ काम्पी॑लवासि॒नीत्या॑ह। तप॑ ए॒वैना॒मुप॑नयति। सु॒व॒र्गे लो॒के संप्रोर्ण्वा॑था॒मित्या॑ह॥२७॥

%3.9.6.4
सु॒व॒र्गमे॒वैनां लो॒कं ग॑मयति। आऽहम॑जानि गर्भ॒धमा त्वम॑जाऽसि गर्भ॒धमित्या॑ह। प्र॒जा वै प॒शवो॒ गर्भ॑। प्र॒जामे॒व प॒शूना॒त्मन्ध॑त्ते। दे॒वा वा अ॑श्वमे॒धे पव॑माने। सु॒व॒र्गं लो॒कं न प्राजा॑नन्। तमश्व॒ प्राजा॑नात्। यत्सू॒चीभि॑रसिप॒थान्क॒ल्पय॑न्ति। सु॒व॒र्गस्य॑ लो॒कस्य॒ प्रज्ञात्यै। गा॒य॒त्री त्रि॒ष्टुब्जग॒तीत्या॑ह॥२८॥

%3.9.6.5
य॒था॒य॒जुरे॒वैतत्। त्र॒य्यः सू॒च्यो॑ भवन्ति। अ॒य॒स्मय्यो॑ रज॒ता हरि॑ण्यः। अ॒स्य वै लो॒कस्य॑ रू॒पम॑य॒स्मय्य॑। अ॒न्तरि॑क्षस्य रज॒ताः। दि॒वो हरि॑ण्यः। दिशो॒ वा अ॑य॒स्मय्य॑। अ॒वा॒न्त॒र॒दि॒शा र॑ज॒ताः। ऊ॒र्ध्वा हरि॑ण्यः। दिश॑ ए॒वास्मै॑ कल्पयति। कस्त्वा छ्यति॒ कस्त्वा॒ विशा॒स्तीत्या॒हाहिसायै॥२९॥\anuvakamend[ह्नु॒व॒ते॒ क्रा॒म॒न्त्यू॒र्ण्वा॒था॒मित्या॑ह॒ जग॒तीत्या॑ह कल्पय॒त्येकं च]

%3.9.7.1
अप॒ वा ए॒तस्मा॒च्छ्री रा॒ष्ट्रङ्क्रा॑मति। योऽश्वमे॒धेन॒ यज॑ते। ऊ॒र्ध्वामे॑ना॒मुच्छ्र॑यता॒दित्या॑ह। श्रीर्वै रा॒ष्ट्रम॑श्वमे॒धः। श्रिय॑मे॒वास्मै॑ रा॒ष्ट्रमू॒र्ध्वमुच्छ्र॑यति। वे॒णु॒भा॒रङ्गि॒रावि॒वेत्या॑ह। रा॒ष्ट्रं वै भा॒रः। रा॒ष्ट्रमे॒वास्मै॒ पर्यू॑हति। अथास्या॒ मध्य॑मेधता॒मित्या॑ह। श्रीर्वै रा॒ष्ट्रस्य॒ मध्यम्॥३०॥

%3.9.7.2
श्रिय॑मे॒वाव॑रुन्धे। शी॒ते वाते॑ पु॒नन्नि॒वेत्या॑ह। क्षेमो॒ वै रा॒ष्ट्रस्य॑ शी॒तो वात॑। क्षेम॑मे॒वाव॑रुन्धे। यद्ध॑रि॒णी यव॒मत्तीत्या॑ह। विड्वै ह॑रि॒णी। रा॒ष्ट्रं यव॑। विशं॑ चै॒वास्मै॑ रा॒ष्ट्रं च॑ स॒मीची॑ दधाति। न पु॒ष्टं प॒शु म॑न्यत॒ इत्या॑ह। तस्मा॒द्राजा॑ प॒शून्न पुष्य॑ति॥३१॥

%3.9.7.3
शू॒द्रा यदर्य॑जारा॒ न पोषा॑य धनाय॒तीत्या॑ह। तस्माद्वैशीपु॒त्रन्नाभिषि॑ञ्चन्ते। इ॒यं य॒का श॑कुन्ति॒केत्या॑ह। विड्वै श॑कुन्ति॒का। रा॒ष्ट्रम॑श्वमे॒धः। विशं॑ चै॒वास्मै॑ रा॒ष्ट्रं च॑ स॒मीची॑ दधाति। आ॒हल॒मिति॒ सर्प॒तीत्या॑ह। तस्माद्रा॒ष्ट्राय॒ विश॑ सर्पन्ति। आह॑तङ्ग॒भे पस॒ इत्या॑ह। विड्वै गभ॑॥३२॥

%3.9.7.4
रा॒ष्ट्रं पस॑। रा॒ष्ट्रमे॒व वि॒श्याह॑न्ति। तस्माद्रा॒ष्ट्रं विशं॒ घातु॑कम्। मा॒ता च॑ ते पि॒ता च॑ त॒ इत्या॑ह। इ॒यं वै मा॒ता। अ॒सौ पि॒ता। आ॒भ्यामे॒वैनं॒ परि॑ददाति। अग्रं॑ वृ॒क्षस्य॑ रोहत॒ इत्या॑ह। श्रीर्वै वृ॒क्षस्याग्रम्। श्रि॒यमे॒वाव॑ रुन्धे॥३३॥

%3.9.7.5
प्रसु॑ला॒मीति॑ ते पि॒ता ग॒भे मु॒ष्टिम॑तसय॒दित्या॑ह। विड्वै गभ॑। रा॒ष्ट्रम्मु॒ष्टिः। रा॒ष्ट्रमे॒व वि॒श्याह॑न्ति। तस्माद्रा॒ष्ट्रं विशं॒ घातु॑कम्। अप॒ वा ए॒तेभ्य॑ प्रा॒णाः क्रा॑मन्ति। ये य॒ज्ञेऽपू॑तं॒ वद॑न्ति। द॒धि॒क्राव्ण्णो॑ अकारिष॒मिति॑ सुरभि॒मती॒मृचं॑ वदन्ति। प्रा॒णा वै सु॑र॒भय॑। प्रा॒णाने॒वात्मन्द॑धते। नैभ्य॑ प्रा॒णा अप॑क्रामन्ति। आपो॒ हि ष्ठा म॑यो॒भुव॒ इत्य॒द्भिर्मार्जयन्ते। आपो॒ वै सर्वा॑ दे॒वता। दे॒वता॑भिरे॒वात्मानं॑ पवयन्ते॥३४॥\anuvakamend[रा॒ष्ट्रस्य॒ मध्यं॒ पुष्य॑ति॒ गभो॑ रुन्धे दधते च॒त्वारि॑ च]

%3.9.8.1
प्र॒जाप॑तिः प्र॒जाः सृ॒ष्ट्वा प्रे॒णाऽनु॒ प्रावि॑शत्। ताभ्य॒ पुन॒ सम्भ॑वितु॒न्नाश॑क्नोत्। सोऽब्रवीत्। ऋ॒ध्नव॒दित्सः। यो मे॒तः पुन॑ स॒म्भर॒दिति॑। तन्दे॒वा अ॑श्वमे॒धेनै॒व सम॑भरन्। ततो॒ वै त आर्ध्नुवन्। योऽश्वमे॒धेन॒ यज॑ते। प्र॒जाप॑तिमे॒व सम्भ॑रत्यृ॒ध्नोति॑। पुरु॑ष॒माल॑भते॥३५॥

%3.9.8.2
वै॒रा॒जो वै पुरु॑षः। वि॒राज॑मे॒वाल॑भते। अथो॒ अन्नं॒ वै वि॒राट्। अन्न॑मे॒वाव॑रुन्धे। अश्व॒माल॑भते। प्रा॒जा॒प॒त्यो वा अश्व॑। प्र॒जाप॑तिमे॒वाल॑भते। अथो॒ श्रीर्वा एक॑शफम्। श्रिय॑मे॒वाव॑रुन्धे। गामाल॑भते॥३६॥

%3.9.8.3
य॒ज्ञो वै गौः। य॒ज्ञमे॒वाल॑भते। अथो॒ अन्नं॒ वै गौः। अन्न॑मे॒वाव॑रुन्धे। अ॒जा॒वी आल॑भते भू॒म्ने। अथो॒ पुष्टि॒र्वै भू॒मा। पुष्टि॑मे॒वाव॑रुन्धे। पर्य॑ग्निकृतं॒ पुरु॑षञ्चार॒ण्याश्चोत्सृ॑ज॒न्त्यहिसायै। उ॒भौ वा ए॒तौ प॒शू आल॑भ्येते। यश्चा॑व॒मो यश्च॑ पर॒मः। तेऽस्यो॒भये॑ य॒ज्ञे ब॒द्धाः। अ॒भीष्टा॑ अ॒भिप्री॑ताः। अ॒भिजि॑ता अ॒भिहु॑ता भवन्ति। नैन॑न्द॒ङ्क्ष्णव॑ प॒शवो॑ य॒ज्ञे ब॒द्धाः। अ॒भीष्टा॑ अ॒भिप्री॑ताः। अ॒भिजि॑ता अ॒भिहु॑ता हिसन्ति। योऽश्वमे॒धेन॒ यज॑ते। य उ॑ चैनमे॒वं वेद॑॥३७॥\anuvakamend[ल॒भ॒ते॒ गामाल॑भते पर॒मोऽष्टौ च॑]

%3.9.9.1
प्र॒थ॒मेन॒ वा ए॒ष स्तोमे॑न रा॒ध्वा। च॒तु॒ष्टो॒मेन॑ कृ॒तेनाया॑ना॒मुत्त॒रेह\sn{}। ए॒क॒वि॒शे प्र॑ति॒ष्ठायां॒ प्रति॑ तिष्ठति। ए॒क॒वि॒शात्प्र॑ति॒ष्ठाया॑ ऋ॒तून॒न्वारो॑हति। ऋ॒तवो॒ वै पृ॒ष्ठानि॑। ऋ॒तव॑ संवत्स॒रः। ऋ॒तुष्वे॒व सं॑वत्स॒रे प्र॑ति॒ष्ठाय॑। दे॒वता॑ अ॒भ्यारो॑हति। शक्व॑रयः पृ॒ष्ठम्भ॑वन्त्य॒न्यद॑न्य॒च्छन्द॑। अ॒न्येऽन्ये॒ वा ए॒ते प॒शव॒ आल॑भ्यन्ते॥३८॥

%3.9.9.2
उ॒तेव॑ ग्रा॒म्याः। उ॒तेवा॑र॒ण्याः। अह॑रे॒व रू॒पेण॒ सम॑र्धयति। अथो॒ अह्न॑ ए॒वैष ब॒लिर्‌ह्रि॑यते। तदा॑हुः। अप॑शवो॒ वा ए॒ते। यद॑जा॒वय॑श्चार॒ण्याश्च॑। ए॒ते वै सर्वे॑ प॒शव॑। यद्ग॒व्या इति॑। ग॒व्यान्प॒शूनु॑त्त॒मेऽह॒न्नाल॑भते॥३९॥

%3.9.9.3
तेनै॒वोभयान्प॒शूनव॑रुन्धे। प्रा॒जा॒प॒त्या भ॑वन्ति। अन॑भिजितस्या॒भिजि॑त्यै। सौ॒रीर्नव॑ श्वे॒ता व॒शा अ॑नूब॒न्ध्या॑ भवन्ति। अ॒न्त॒त ए॒व ब्र॑ह्मवर्च॒समव॑रुन्धे। सोमा॑य स्व॒राज्ञे॑ऽनोवा॒हाव॑न॒ड्वाहा॒विति॑ द्व॒न्द्विन॑ प॒शूनाल॑भते। अ॒हो॒रा॒त्राणा॑म॒भिजि॑त्यै। प॒शुभि॒र्वा ए॒ष व्यृ॑ध्यते। योऽश्वमे॒धेन॒ यज॑ते। छ॒ग॒लङ्क॒ल्माष॑ङ्किकिदी॒विं वि॑दी॒गय॒मिति॑ त्वा॒ष्ट्रान्प॒शूना ल॑भते। प॒शुभि॑रे॒वात्मान॒ सम॑र्धयति। ऋ॒तुभि॒र्वा ए॒ष व्यृ॑ध्यते। योऽश्वमे॒धेन॒ यज॑ते। पि॒शङ्गा॒स्त्रयो॑ वास॒न्ता इत्यृ॑तुप॒शूनाल॑भते। ऋ॒तुभि॑रे॒वात्मान॒ सम॑र्धयति। आ वा ए॒ष प॒शुभ्यो॑ वृश्च्यते। योऽश्वमे॒धेन॒ यज॑ते। पर्य॑ग्निकृता॒ उत्सृ॑ज॒न्त्यनाव्रस्काय॥४०॥\anuvakamend[ल॒भ्य॒न्ते॒ ल॒भ॒ते॒ त्वा॒ष्ट्रान्प॒शूनाल॑भते॒ऽष्टौ च॑]

%3.9.10.1
प्र॒जाप॑तिरकामयत म॒हान॑न्ना॒दः स्या॒मिति॑। स ए॒ताव॑श्वमे॒धे म॑हि॒माना॑वपश्यत्। ताव॑गृह्णीत। ततो॒ वै स म॒हान॑न्ना॒दो॑ऽभवत्। यः का॒मये॑त म॒हान॑न्ना॒दः स्या॒मिति॑। स ए॒ताव॑श्वमे॒धे म॑हि॒मानौ॑ गृह्णीत। म॒हाने॒वान्ना॒दो भ॑वति। य॒ज॒मा॒न॒दे॒व॒त्या॑ वै व॒पा। राजा॑ महि॒मा। यद्व॒पाम्म॑हि॒म्नोभ॒यत॑ परि॒यज॑ति। यज॑मानमे॒व रा॒ज्येनो॑भ॒यत॒ परि॑गृह्णाति। पु॒रस्तात्स्वाहाकारा॒ वा अ॒न्ये दे॒वाः। उ॒परि॑ष्टात्स्वाहाकारा अ॒न्ये। ते वा ए॒तेऽश्व॑ ए॒व मेध्य॑ उ॒भयेऽव॑रुध्यन्ते। यद्व॒पाम्म॑हि॒म्नोभ॒यत॑ परि॒यज॑ति। ताने॒वोभयान्प्रीणाति॥४१॥\anuvakamend[प॒रि॒यज॑ति॒ षट्च॑]

%3.9.11.1
वै॒श्व॒दे॒वो वा अश्व॑। तं यत्प्रा॑जाप॒त्यं कु॒र्यात्। या दे॒वता॒ अपि॑भागाः। ता भा॑ग॒धेये॑न॒ व्य॑र्धयेत्। दे॒वताभ्यः स॒मद॑न्दध्यात्। स्ते॒गान्दष्ट्राभ्याम्म॒ण्डूकां॒ जम्भ्ये॑भि॒रिति॑। आज्य॑मव॒दानं॑ कृ॒त्वा प्र॑तिस॒ङ्ख्याय॒माहु॑तीर्जुहोति। या ए॒व दे॒वता॒ अपि॑भागाः। ता भा॑ग॒धेये॑न॒ सम॑र्धयति। न दे॒वताभ्यः स॒मद॑न्दधाति॥४२॥

%3.9.11.2
चतु॑र्दशै॒तान॑नुवा॒काञ्जु॑हो॒त्यन॑न्तरित्यै। प्र॒या॒साय॒ स्वाहेति॑ पञ्चद॒शम्। पञ्च॑दश॒ वा अ॑र्धमा॒सस्य॒ रात्र॑यः। अ॒र्ध॒मा॒स॒शः सं॑वत्स॒र आप्यते। दे॒वा॒सु॒राः संय॑त्ता आसन्। तेऽब्रुवन्न॒ग्नय॑ स्विष्ट॒कृत॑। अश्व॑स्य॒ मेध्य॑स्य व॒यमु॑द्धा॒रमुद्ध॑रामहै। अथै॒तान॒भि भ॑वा॒मेति॑। ते लोहि॑त॒मुद॑हरन्त। ततो॑ दे॒वा अभ॑वन्॥४३॥

%3.9.11.3
पराऽसु॑राः। यत्स्वि॑ष्ट॒कृद्भ्यो॒ लोहि॑तं जु॒होति॒ भ्रातृ॑व्याभिभूत्यै। भव॑त्या॒त्मना। पराऽस्य॒ भ्रातृ॑व्यो भवति। गो॒मृ॒ग॒क॒ण्ठेन॑ प्रथ॒मामाहु॑तिं जुहोति। प॒शवो॒ वै गो॑मृ॒गः। रु॒द्रोऽग्निः स्वि॑ष्ट॒कृत्। रु॒द्रादे॒व प॒शून॒न्तर्द॑धाति। अथो॒ यत्रै॒षाऽऽहु॑तिर्‌हू॒यते। न तत्र॑ रु॒द्रः प॒शून॒भिम॑न्यते॥४४॥

%3.9.11.4
अ॒श्व॒श॒फेन॑ द्वि॒तीया॒माहु॑तिं जुहोति। प॒शवो॒ वा एक॑शफम्। रु॒द्रोऽग्निः स्वि॑ष्ट॒कृत्। रु॒द्रादे॒व प॒शून॒न्तर्द॑धाति। अथो॒ यत्रै॒षाऽऽहु॑तिर्‌हू॒यते। न तत्र॑ रु॒द्रः प॒शून॒भिम॑न्यते। अ॒य॒स्मये॑न कम॒ण्डलु॑ना तृ॒तीयाम्। आहु॑तिं जुहोत्याया॒स्यो॑ वै प्र॒जाः। रु॒द्रोऽग्निः स्वि॑ष्ट॒कृत्। रु॒द्रादे॒व प्र॒जा अ॒न्तर्द॑धाति। अथो॒ यत्रै॒षाऽऽहु॑तिर्‌हू॒यते। न तत्र॑ रु॒द्रः प्र॒जा अ॒भिम॑न्यते॥४५॥\anuvakamend[द॒धा॒त्यभ॑वन्मन्यते प्र॒जा अ॒न्तर्द॑धाति॒ द्वे च॑ ]

%3.9.12.1
अश्व॑स्य॒ वा आल॑ब्धस्य॒ मेध॒ उद॑क्रामत्। तद॑श्वस्तो॒मीय॑मभवत्। यद॑श्वस्तो॒मीयं॑ जु॒होति॑। समे॑धमे॒वैन॒माल॑भते। आज्ये॑न जुहोति। मेधो॒ वा आज्यम्। मेधोऽश्वस्तो॒मीयम्। मेधे॑नै॒वास्मि॒न्मेध॑न्दधाति। षट्त्रिशतं जुहोति। षट्त्रिशदक्षरा बृह॒ती॥४६॥

%3.9.12.2
बार्‌ह॑ताः प॒शव॑। सा प॑शू॒नाम्मात्रा। प॒शूने॒व मात्र॑या॒ सम॑र्धयति। तायद्भूय॑सीर्वा॒ कनी॑यसीर्वा जुहु॒यात्। प॒शून्मात्र॑या॒ व्य॑र्धयेत्। षट्त्रिशतं जुहोति। षट्त्रिशदक्षरा बृह॒ती। बार्‌ह॑ताः प॒शव॑। सा प॑शू॒नाम्मात्रा। प॒शूने॒व मात्र॑या॒ सम॑र्धयति॥४७॥

%3.9.12.3
अ॒श्व॒स्तो॒मीय हु॒त्वा द्वि॒पदा॑ जुहोति। द्वि॒पाद्वै पुरु॑षो॒ द्विप्र॑तिष्ठः। तदे॑नं प्रति॒ष्ठया॒ सम॑र्धयति। तदा॑हुः। अ॒श्व॒स्तो॒मीयं॒ पूर्व होत॒व्याँ ३ न्द्वि॒पदाँ ३ इति॑। अश्वो॒ वा अ॑श्वस्तो॒मीयम्। पुरु॑षो द्वि॒पदा। अ॒श्व॒स्तो॒मीय हु॒त्वा द्वि॒पदा॑ जुहोति। तस्माद्द्वि॒पाच्चतु॑ष्पादमत्ति। अथो द्वि॒पद्ये॒व चतु॑ष्पद॒ प्रति॑ष्ठायपति। द्वि॒पदा॑ हु॒त्वा। नान्यामुत्त॑रा॒माहु॑तिं जुहुयात्। यद॒न्यामुत्त॑रा॒माहु॑तिं जुहु॒यात्। प्र प्र॑ति॒ष्ठायाश्च्यवेत। द्वि॒पदा॑ अन्त॒तो जु॑होति॒ प्रति॑ष्ठित्यै॥४८॥\anuvakamend[बृ॒ह॒त्य॑र्धयति स्थापयति॒ पञ्च॑ च]

%3.9.13.1
प्र॒जाप॑तिरश्वमे॒धम॑सृजत। सोऽस्मात्सृ॒ष्टोऽपाक्रामत्। तं य॑ज्ञक्र॒तुभि॒रन्वैच्छत्। तं य॑ज्ञक्र॒तुभि॒र्नान्व॑विन्दत्। तमिष्टि॑भि॒रन्वैच्छत्। तमिष्टि॑भि॒रन्व॑विन्दत्। तदिष्टी॑नामिष्टि॒त्वम्। यत्सं॑वत्स॒रमिष्टि॑भि॒र्यज॑ते। अश्व॑मे॒व तदन्वि॑च्छति। सा॒वि॒त्रियो॑ भवन्ति॥४९॥

%3.9.13.2
इ॒यं वै स॑वि॒ता। यो वा अ॒स्यान्नश्य॑ति॒ यो नि॒लय॑ते। अ॒स्यां वाव तं वि॑न्दन्ति। न वा इ॒माङ्कश्च॒नेत्या॑हुः। ति॒र्यङ्नोर्ध्वोत्ये॑तुमर्ह॒तीति॑। यत्सा॑वि॒त्रियो॒ भव॑न्ति। स॒वि॒तृप्र॑सूत ए॒वैन॑मिच्छति। ई॒श्व॒रो वा अश्व॒ प्रमु॑क्त॒ परां परा॒वत॒ङ्गन्तो। यत्सा॒यन्धृतीर्जु॒होति॑। अश्व॑स्यै॒व यत्यै॒ धृत्यै॥५०॥ यत्प्रा॒तरिष्टि॑भि॒र्यज॑ते। अश्व॑मे॒व तदन्वि॑च्छति। यत्सा॒यन्धृतीर्जु॒होति॑। अश्व॑स्यै॒व यत्यै॒ धृत्यै। तस्मात्सा॒यं प्र॒जाः क्षे॒म्या॑ भवन्ति। यत्प्रा॒तरिष्टि॑भि॒र्यज॑ते। अश्व॑मे॒व तदन्वि॑च्छति। तस्मा॒द्दिवा॑ नष्टै॒ष ए॑ति। यत्प्रा॒तरिष्टि॑भि॒र्यज॑ते सा॒यन्धृतीर्जु॒होति॑। अ॒हो॒रा॒त्राभ्या॑मे॒वैन॒मन्वि॑च्छति। अथो॑ अहोरा॒त्राभ्या॑मे॒वास्मै॑ योगक्षे॒मङ्क॑ल्पयति॥५१॥\anuvakamend[भ॒व॒न्ति॒ धृत्या॑ एन॒मन्वि॑च्छ॒त्येकं च]

%3.9.14.1
अप॒ वा ए॒तस्मा॒च्छ्री रा॒ष्ट्रङ्क्रा॑मति। योऽश्वमे॒धेन॒ यज॑ते। ब्रा॒ह्म॒णौ वी॑णागा॒थिनौ॑ गायतः। श्रि॒या वा ए॒तद्रू॒पम्। यद्वीणा। श्रिय॑मे॒वास्मि॒न्तद्ध॑त्तः। य॒दा खलु॒ वै पुरु॑ष॒ श्रिय॑मश्ञु॒ते। वीणाऽस्मै वाद्यते। तदा॑हुः। यदु॒भौ ब्राह्म॒णौ गाये॑ताम्॥५२॥

%3.9.14.2
प्र॒भ्रशु॑कास्मा॒च्छ्रीः स्यात्। न वै ब्राह्म॒णे श्री र॑मत॒ इति॑। ब्रा॒ह्म॒णोऽन्यो गायेत्। रा॒ज॒न्योऽन्यः। ब्रह्म॒ वै ब्राह्म॒णः। क्ष॒त्र रा॑ज॒न्य॑। तथा॑ हास्य॒ ब्रह्म॑णा च क्ष॒त्रेण॑ चोभ॒यत॒ श्रीः परि॑गृहीता भवति। तदा॑हुः। यदु॒भौ दिवा॒ गाये॑ताम्। अपास्माद्रा॒ष्ट्रङ्क्रा॑मेत्॥५३॥

%3.9.14.3
न वै ब्रा॑ह्म॒णे रा॒ष्ट्र र॑मत॒ इति॑। य॒दा खलु॒ वै राजा॑ का॒मय॑ते। अथ॑ ब्राह्म॒णञ्जि॑नाति। दिवा ब्राह्म॒णो गा॑येत्। नक्त राज॒न्य॑। ब्रह्म॑णो॒ वै रू॒पमह॑। क्ष॒त्रस्य॒ रात्रि॑। तथा॑ हास्य॒ ब्रह्म॑णा च क्ष॒त्रेण॑ चोभ॒यतो॑ रा॒ष्ट्रं परि॑गृहीतम्भवति। इत्य॑ददा॒ इत्य॑यजथा॒ इत्य॑पच॒ इति॑ ब्राह्म॒णो गायेत्। इ॒ष्टा॒पू॒र्तं वै ब्राह्म॒णस्य॑॥५४॥

%3.9.14.4
इ॒ष्टा॒पू॒र्तेनै॒वैन॒ स सम॑र्धयति। इत्य॑जिना॒ इत्य॑युध्यथा॒ इत्य॒मु स॑ङ्ग्रा॒मम॑ह॒न्निति॑ राज॒न्य॑। यु॒द्धं वै रा॑ज॒न्य॑स्य। यु॒द्धेनै॒वैन॒ स सम॑र्धयति। अकॢ॑प्ता॒ वा ए॒तस्य॒र्तव॒ इत्या॑हुः। योऽश्वमे॒धेन॒ यज॑त॒ इति॑। ति॒स्रोऽन्यो गाय॑ति ति॒स्रोऽन्यः। षट्त्संप॑द्यन्ते। षड्वा ऋ॒तव॑। ऋ॒तूने॒वास्मै॑ कल्पयतः। ताभ्या स॒स्थायाम्। अ॒नो॒यु॒क्ते च॑ श॒ते च॑ ददाति। श॒तायु॒ पुरु॑षः श॒तेन्द्रि॑यः। आयु॑ष्ये॒वेन्द्रि॒ये प्रति॑तिष्ठति॥५५॥\anuvakamend[गाये॑ताङ्क्रामेद्ब्राह्म॒णस्य॑ कल्पयतश्च॒त्वारि॑ च]

%3.9.15.1
सर्वे॑षु॒ वा ए॒षु लो॒केषु॑ मृ॒त्यवो॒ऽन्वाय॑त्ताः। तेभ्यो॒ यदाहु॑ती॒र्न जु॑हु॒यात्। लो॒केलो॑क एनं मृ॒त्युर्वि॑न्देत्। मृ॒त्यवे॒ स्वाहा॑ मृ॒त्यवे॒ स्वाहेत्य॑भिपू॒र्वमाहु॑तीर्जुहोति। लो॒काल्लो॑कादे॒व मृ॒त्युमव॑यजते। नैनं॑ लो॒केलो॑के मृ॒त्युर्वि॑न्दति। यद॒मुष्मै॒ स्वाहा॒ऽमुष्मै॒ स्वाहेति॒ जुह्व॑त्स॒ञ्चक्षी॑त। ब॒हुं मृ॒त्युम॒मित्रं॑ कुर्वीत। मृ॒त्यवे॒ स्वाहेत्येक॑स्मा ए॒वैकां जुहुयात्। एको॒ वा अ॒मुष्मि॑ल्लोँ॒के मृ॒त्युः॥५६॥

%3.9.15.2
अ॒श॒न॒या॒ मृ॒त्युरे॒व। तमे॒वामुष्मि॑ल्लोँ॒केऽव॑यजते। भ्रू॒ण॒ह॒त्यायै स्वाहेत्य॑वभृ॒थ आहु॑तिं जुहोति। भ्रू॒ण॒ह॒त्यामे॒वाव॑ यजते। तदा॑हुः। यद्भ्रू॑णह॒त्या पा॒त्र्याऽथ॑। कस्माद्य॒ज्ञेऽपि॑ क्रियत॒ इति॑। अमृ॑त्यु॒र्वा अ॒न्यो भ्रू॑णह॒त्याया॒ इत्या॑हुः। भ्रू॒ण॒ह॒त्या वाव मृ॒त्युरिति॑। यद्भ्रू॑णह॒त्यायै॒ स्वाहेत्य॑वभृ॒थ आहु॑तिं जु॒होति॑॥५७॥

%3.9.15.3
मृ॒त्युमे॒वाहु॑त्या तर्पयि॒त्वा प॑रि॒पाणं॑ कृ॒त्वा। भ्रू॒ण॒घ्ने भे॑ष॒जं क॑रोति। ए॒ता ह॒ वै मु॑ण्डि॒भ औ॑दन्य॒वः। भ्रू॒ण॒ह॒त्यायै॒ प्राय॑श्चित्तिं वि॒दां च॑कार। यो हा॒स्यापि॑ प्र॒जायां ब्राह्म॒ण हन्ति॑। सर्व॑स्मै॒ तस्मै॑ भेष॒जं क॑रोति। जु॒म्ब॒काय॒ स्वाहेत्य॑वभृ॒थ उ॑त्त॒मामाहु॑तिं जुहोति। वरु॑णो॒ वै जु॑म्ब॒कः। अ॒न्त॒त ए॒व वरु॑ण॒मव॑यजते। ख॒ल॒तेर्वि॑क्लि॒धस्य॑ शु॒क्लस्य॑ पिङ्गा॒क्षस्य॑ मू॒र्धं जु॑होति। ए॒तद्वै वरु॑णस्य रू॒पम्। रू॒पेणै॒व वरु॑ण॒मव॑यजते॥५८॥\anuvakamend[लो॒के मृ॒त्युर्जु॒होति॑ मू॒र्धं जु॑होति॒ द्वे च॑]

%3.9.16.1
वा॒रु॒णो वा अश्व॑। तन्दे॒वत॑या॒ व्य॑र्धयति। यत्प्रा॑जाप॒त्यं क॒रोति॑। नमो॒ राज्ञे॒ नमो॒ वरु॑णा॒येत्या॑ह। वा॒रु॒णो वा अश्व॑। स्वयै॒वैनं॑ दे॒वत॑या॒ सम॑र्धयति। नमोऽश्वा॑य॒ नम॑ प्र॒जाप॑तय॒ इत्या॑ह। प्रा॒जा॒प॒त्यो वा अश्व॑। स्वयै॒वैनं॑ दे॒वत॑या॒ सम॑र्धयति। नमोऽधि॑पतय॒ इत्या॑ह॥५९॥

%3.9.16.2
धर्मो॒ वा अधि॑पतिः। धर्म॑मे॒वाव॑रुन्धे। अधि॑पतिर॒स्यधि॑पतिम्मा कु॒र्वधि॑पतिर॒हं प्र॒जानां भूयास॒मित्या॑ह। अधि॑पतिमे॒वैन समा॒नानां करोति। मान्धे॑हि॒ मयि॑ धे॒हीत्या॑ह। आ॒शिष॑मे॒वैतामाशास्ते। उ॒पाकृ॑ताय॒ स्वाहेत्यु॒पाकृ॑ते जुहोति। आल॑ब्धाय॒ स्वाहेति॒ नियु॑क्ते जुहोति। हु॒ताय॒ स्वाहेति॑ हु॒ते जु॑होति। ए॒षां लो॒काना॑म॒भिजि॑त्यै॥६०॥

%3.9.16.3
प्र वा ए॒ष ए॒भ्यो लो॒केभ्य॑श्च्यवते। योऽश्वमे॒धेन॒ यज॑ते। आ॒ग्ने॒यमैन्द्रा॒ग्नमाश्वि॒नम्। तान्प॒शूल॑भते॒ प्रति॑ष्ठित्यै। यदाग्ने॒यो भव॑ति। अ॒ग्निः सर्वा॑ दे॒वता। दे॒वता॑ ए॒वाव॑रुन्धे। ब्रह्म॒ वा अ॒ग्निः। क्ष॒त्रमिन्द्र॑। यदैन्द्रा॒ग्नो भव॑ति॥६१॥

%3.9.16.4
ब्र॒ह्म॒क्ष॒त्रे ए॒वाव॑रुन्धे। यदाश्वि॒नो भव॑ति। आ॒शिषा॒मव॑रुद्ध्यै। त्रयो॑ भवन्ति। त्रय॑ इ॒मे लो॒काः। ए॒ष्वे॑व लो॒केषु॒ प्रति॑तिष्ठति। अ॒ग्नयेऽहो॒मुचे॒ऽष्टाक॑पाल॒ इति॒ दश॑हविष॒मिष्टिं॒ निर्व॑पति। दशाक्षरा वि॒राट्। अन्नं॑ वि॒राट्। वि॒राजै॒वान्नाद्य॒मव॑रुन्धे। अ॒ग्नेर्म॑न्वे प्रथ॒मस्य॒ प्रचे॑तस॒ इति॑ याज्यानुवा॒क्या॑ भवन्ति सर्व॒त्वाय॑ ॥६२॥\anuvakamend[अधि॑पतय॒ इत्या॑हा॒भि॑जित्या ऐन्द्रा॒ग्नो भव॑ति रुन्ध॒ एकं च]

%3.9.17.1
यद्यश्व॑मुप॒तप॑द्वि॒न्देत्। आ॒ग्ने॒यम॒ष्टाक॑पालं॒ निर्व॑पेत्। सौ॒म्यञ्च॒रुम्। सा॒वि॒त्रम॒ष्टाक॑पालम्। यदाग्ने॒यो भव॑ति। अ॒ग्निः सर्वा॑ दे॒वता। दे॒वता॑भिरे॒वैन॑म्भिषज्यति। यत्सौ॒म्यो भव॑ति। सोमो॒ वा ओष॑धीना॒ राजा। याभ्य॑ ए॒वैनं वि॒न्दति॑॥६३॥

%3.9.17.2
ताभि॑रे॒वैन॑म्भिषज्यति। यत्सा॑वि॒त्रो भव॑ति। स॒वि॒तृप्र॑सूत ए॒वैन॑म्भिषज्यति। ए॒ताभि॑रे॒वैनं॑ दे॒वता॑भिर्भिषज्यति। अ॒ग॒दो है॒व भ॑वति। पौ॒ष्णञ्च॒रुन्निर्व॑पेत्। यदि॑ श्लो॒णः स्यात्। पू॒षा वै श्लौण्य॑स्य भि॒षक्। स ए॒वैन॑म्भिषज्यति। अश्लो॑णो है॒व भ॑वति॥६४॥

%3.9.17.3
रौ॒द्रञ्च॒रुन्निर्व॑पेत्। यदि॑ मह॒ती दे॒वता॑ऽभि॒मन्ये॑त। ए॒त॒द्दे॒व॒त्यो॑ वा अश्व॑। स्वयै॒वैनं॑ दे॒वत॑या भिषज्यति। अ॒ग॒दो है॒व भ॑वति। वै॒श्वा॒न॒रन्द्वाद॑शकपालं॒ निर्व॑पेन्मृगाख॒रे यदि॒ नागच्छेत्। इ॒यं वा अ॒ग्निर्वैश्वान॒रः। इ॒यमे॒वैन॑म॒र्चिभ्यां परि॒रोध॒मान॑यति। आहै॒व सुत्य॒मह॑र्गच्छति। यद्य॑धी॒यात्॥६५॥

%3.9.17.4
अ॒ग्नयेऽहो॒मुचे॒ऽष्टाक॑पालः। सौ॒र्यं पय॑। वा॒य॒व्य॑ आज्य॑भागः। यज॑मानो॒ वा अश्व॑। अह॑सा॒ वा ए॒ष गृ॑ही॒तः। यस्याश्वो॒ मेधा॑य॒ प्रोक्षि॑तो॒ऽध्येति॑। यदहो॒मुचे॑ नि॒र्वप॑ति। अह॑स ए॒व तेन॑ मुच्यते। यज॑मानो॒ वा अश्व॑। रेत॑सा॒ वा ए॒ष व्यृ॑ध्यते॥६६॥

%3.9.17.5
यस्याश्वो॒ मेधा॑य॒ प्रोक्षि॑तो॒ऽध्येति॑। सौ॒र्य रेत॑। यत्सौ॒र्यं पयो॒ भव॑ति। रेत॑सै॒वैन॒ स सम॑र्धयति। यज॑मानो॒ वा अश्व॑। गर्भै॒र्वा ए॒ष व्यृ॑ध्यते। यस्याश्वो॒ मेधा॑य॒ प्रोक्षि॑तो॒ऽध्येति॑। वा॒य॒व्या॑ गर्भा। यद्वा॑य॒व्य॑ आज्य॑भागो॒ भव॑ति। गर्भै॑रे॒वैन॒ स सम॑र्धयति। अथो॒ यस्यै॒षाऽश्व॑मे॒धे प्राय॑श्चित्तिः क्रि॒यते। इ॒ष्ट्वा वसी॑यान्भवति॥६७॥\anuvakamend[वि॒न्दत्यश्लो॑णो है॒व भ॑वत्यधी॒यादृ॑ध्यते॒ गर्भै॑रे॒वैन॒ स सम॑र्धयति॒ द्वे च॑]

%3.9.18.1
तदा॑हुः। द्वाद॑श ब्रह्मौद॒नान्त्सस्थि॑ते॒ निर्व॑पेत्। द्वा॒द॒शभि॒र्वेष्टि॑भिर्यजे॒तेति॑। यदिष्टि॑भि॒र्यजे॑त। उ॒प॒नामु॑क एनं य॒ज्ञः स्यात्। पापी॑या॒स्तु स्यात्। आ॒प्तानि॒ वा ए॒तस्य॒ छन्दासि। य ई॑जा॒नः। तानि॒ क ए॒ताव॑दाशु॒ पुन॒ प्रयुं॑ जी॒तेति॑। सर्वा॒ वै सस्थि॑ते य॒ज्ञे वागाप्यते॥६८॥

%3.9.18.2
साप्ता भ॑वति या॒तयाम्नी। क्रू॒रीकृ॑तेव॒ हि भव॒त्यरु॑ष्कृता। सा न पुन॑ प्र॒युज्येत्या॑हुः। द्वाद॑शै॒व ब्र॑ह्मौद॒नान्त्सस्थि॑ते॒ निर्व॑पेत्। प्र॒जाप॑ति॒र्वा ओ॑द॒नः। य॒ज्ञः प्र॒जाप॑तिः। उ॒प॒नामु॑क एनं य॒ज्ञो भ॑वति। न पापी॑यान्भवति। द्वाद॑श भवन्ति। द्वाद॑श॒मासा संवत्स॒रः। सं॒व॒त्स॒र ए॒व प्रति॑तिष्ठति॥६९॥\anuvakamend[आ॒प्य॒ते॒ सं॒व॒त्स॒र एकं च]

%3.9.19.1
ए॒ष वै वि॒भूर्नाम॑ य॒ज्ञः। सर्व ह॒ वै तत्र॑ वि॒भु भ॑वति। यत्रै॒तेन॑ य॒ज्ञेन॒ यज॑न्ते। ए॒ष वै प्र॒भूर्नाम॑ य॒ज्ञः। सर्व ह॒ वै तत्र॑ प्र॒भु भ॑वति। यत्रै॒तेन॑ य॒ज्ञेन॒ यज॑न्ते। ए॒ष वा ऊर्ज॑स्वा॒न्नाम॑ य॒ज्ञः। सर्व ह॒ वै तत्रोर्ज॑स्वद्भवति। यत्रै॒तेन॑ य॒ज्ञेन॒ यज॑न्ते। ए॒ष वै पय॑स्वा॒न्नाम॑ य॒ज्ञः॥७०॥

%3.9.19.2
सर्व ह॒ वै तत्र॒ पय॑स्वद्भवति। यत्रै॒तेन॑ य॒ज्ञेन॒ यज॑न्ते। ए॒ष वै विधृ॑तो॒ नाम॑ य॒ज्ञः। सर्व ह॒ वै तत्र॒ विधृ॑तम्भवति। यत्रै॒तेन॑ य॒ज्ञेन॒ यज॑न्ते। ए॒ष वै व्यावृ॑त्तो॒ नाम॑ य॒ज्ञः। सर्व ह॒ वै तत्र॒ व्यावृ॑त्तम्भवति। यत्रै॒तेन॑ य॒ज्ञेन॒ यज॑न्ते। ए॒ष वै प्रति॑ष्ठितो॒ नाम॑ य॒ज्ञः। सर्व ह॒ वै तत्र॒ प्रति॑ष्ठितम्भवति॥७१॥

%3.9.19.3
यत्रै॒तेन॑ य॒ज्ञेन॒ यज॑न्ते। ए॒ष वै ते॑ज॒स्वी नाम॑ य॒ज्ञः। सर्व ह॒ वै तत्र॑ तेज॒स्वि भ॑वति। यत्रै॒तेन॑ य॒ज्ञेन॒ यज॑न्ते। ए॒ष वै ब्र॑ह्मवर्च॒सी नाम॑ य॒ज्ञः। आ ह॒ तत्र॑ ब्राह्म॒णो ब्र॑ह्मवर्च॒सी जा॑यते। यत्रै॒तेन॑ य॒ज्ञेन॒ यज॑न्ते। ए॒ष वा अ॑तिव्या॒धी नाम॑ य॒ज्ञः। आ ह॒ वै तत्र॑ राज॒न्यो॑ऽतिव्या॒धी जा॑यते। यत्रै॒तेन॑ य॒ज्ञेन॒ यज॑न्ते। ए॒ष वै दी॒र्घो नाम॑ य॒ज्ञः। दी॒र्घायु॑षो ह॒ वै तत्र॑ मनु॒ष्या॑ भवन्ति। यत्रै॒तेन॑ य॒ज्ञेन॒ यज॑न्ते। ए॒ष वै कॢ॒प्तो नाम॑ य॒ज्ञः। कल्प॑ते ह॒ वै तत्र॑ प्र॒जाभ्यो॑ योगक्षे॒मः। यत्रै॒तेन॑ य॒ज्ञेन॒ यज॑न्ते॥७२॥\anuvakamend[पय॑स्वा॒न्नाम॑ य॒ज्ञः प्रति॑ष्ठितम्भवति॒ यत्रै॒तेन॑ य॒ज्ञेन॒ यज॑न्ते॒ षट्च॑ (ए॒ष वै विभूः प्र॒भूरूर्ज॑स्वा॒न्पय॑स्वा॒न् विधृ॑तो॒ व्यावृ॑त्त॒ प्रति॑ष्ठितस्तेज॒स्वी ब्र॑ह्मवर्च॒स्य॑तिव्या॒धी दी॒र्घः कॢ॒प्तो द्वाद॑श ॥ )]

%3.9.20.1
ता॒र्प्येणाश्व॒ संज्ञ॑पयन्ति। य॒ज्ञो वै ता॒र्प्यम्। य॒ज्ञेनै॒वैन॒ सम॑र्धयन्ति। या॒मेन॒ साम्ना प्रस्तो॒ताऽनूप॑तिष्ठते। य॒म॒लो॒कमे॒वैन॑ङ्गमयति। ता॒र्प्ये च॑ कृत्यधीवा॒से चाश्व॒ संज्ञ॑पयन्ति। ए॒तद्वै प॑शू॒ना रू॒पम्। रू॒पेणै॒व प॒शूनव॑रुन्धे। हि॒र॒ण्य॒क॒शि॒पु भ॑वति। तेज॒सोऽव॑रुद्ध्यै॥७३॥

%3.9.20.2
रु॒क्मो भ॑वति। सु॒व॒र्गस्य॑ लो॒कस्यानु॑ख्यात्यै। अश्वो॑ भवति। प्र॒जाप॑ते॒राप्त्यै। अ॒स्य वै लो॒कस्य॑ रू॒पन्ता॒र्प्यम्। अ॒न्तरि॑क्षस्य कृत्यधीवा॒सः। दि॒वो हि॑रण्यकशि॒पु। आ॒दि॒त्यस्य॑ रु॒क्मः। प्र॒जाप॑ते॒रश्व॑। इ॒ममे॒व लो॒कन्ता॒र्प्येणाप्तोति॥७४॥

%3.9.20.3
अ॒न्तरि॑क्षङ्कृत्यधीवा॒सेन॑। दिव हिरण्यकशि॒पुना। आ॒दि॒त्य रु॒क्मेण॑। अश्वे॑नै॒व मेध्ये॑न प्र॒जाप॑ते॒ सायु॑ज्य सलो॒कता॑माप्नोति। ए॒तासा॑मे॒व दे॒वता॑ना॒ सायु॑ज्यम्। सा॒र्ष्टिता समानलो॒कता॑माप्नोति। योऽश्वमे॒धेन॒ यज॑ते। य उ॑ चैनमे॒वं वेद॑॥७५॥\anuvakamend[अव॑रुध्या आप्नोत्य॒ष्टौ च॑]

%3.9.21.1
आ॒दि॒त्याश्चाङ्गि॑रसश्च सुव॒र्गे लो॒केऽस्पर्धन्त। तेऽङ्गि॑रस आदि॒त्येभ्य॑। अ॒मुमा॑दि॒त्यमश्व श्वे॒तम्भू॒तन्दक्षि॑णामनयन्। तेऽब्रुवन्। यन्नोनेष्ट। स वर्यो॑ भू॒दिति॑। तस्मा॒दश्व॒ सव॒र्येत्याह्व॑यन्ति। तस्माद्य॒ज्ञे वरो॑ दीयते। यत्प्र॒जाप॑ति॒राल॒ब्धोऽश्वोऽभ॑वत्। तस्मा॒दश्वो॒ नाम॑॥७६॥

%3.9.21.2
यच्छ्वय॒दरु॒रासीत्। तस्मा॒दर्वा॒ नाम॑। यत्स॒द्यो वाजान्त्स॒मज॑यत्। तस्माद्वा॒जी नाम॑। यदसु॑राणां लो॒कानाद॑त्त। तस्मा॑दादि॒त्यो नाम॑। अ॒ग्निर्वा अ॑श्वमे॒धस्य॒ योनि॑रा॒यत॑नम्। सूर्यो॒ऽग्नेर्योनि॑रा॒यत॑नम्। यद॑श्वमे॒धेऽग्नौ चित्य॑ उत्तरवे॒दिमु॑प॒वप॑ति। योनि॑मन्तमे॒वैन॑मा॒यत॑नवन्तं करोति॥७७॥

%3.9.21.3
योनि॑माना॒यत॑नवान्भवति। य ए॒वं वेद॑। प्रा॒णा॒पा॒नौ वा ए॒तौ दे॒वानाम्। यद॑र्काश्वमे॒धौ। प्रा॒णा॒पा॒नावे॒वाव॑रुन्धे। ओजो॒ बलं॒ वा ए॒तौ दे॒वानाम्। यद॑र्काश्वमे॒धौ। ओजो॒ बल॑मे॒वाव॑रुन्धे। अ॒ग्निर्वा अ॑श्वमे॒धस्य॒ योनि॑रा॒यत॑नम्। सूर्यो॒ग्नेर्योनि॑रा॒यत॑नम्। यद॑श्वमे॒धेऽग्नौ चित्य॑ उत्तरवे॒दिञ्चि॒नोति॑। ताव॑र्काश्वमे॒धौ। अ॒र्का॒श्व॒मे॒धावे॒वाव॑रुन्धे। अथो॑ अर्काश्वमे॒धयो॑रे॒व प्रति॑तिष्ठति॥७८॥\anuvakamend[नाम॑ करोति॒ सूर्यो॒ऽग्नेर्योनि॑रा॒यत॑नञ्च॒त्वारि॑ च]

%3.9.22.1
प्र॒जाप॑तिं॒ वै दे॒वाः पि॒तरम्। प॒शुम्भू॒तम्मेधा॒याल॑भन्त। तमा॒लभ्योपा॑वसन्। प्रा॒तर्यष्टास्मह॒ इति॑। एकं॒ वा ए॒तद्दे॒वाना॒मह॑। यत्सं॑वत्स॒रः। तस्मा॒दश्व॑ पु॒रस्तात्संवत्स॒र आल॑भ्यते। यत्प्र॒जाप॑ति॒राल॒ब्धोऽश्वोऽभ॑वत्। तस्मा॒दश्व॑। यत्स॒द्यो मेधोऽभ॑वत्॥७९॥

%3.9.22.2
तस्मा॑दश्वमे॒धः। वेदु॒कोऽश्व॑मा॒शुम्भ॑वति। य ए॒वं वेद॑। यद्वै तत्प्र॒जाप॑ति॒राल॒ब्धोऽश्वोऽभ॑वत्। तस्मा॒दश्व॑ प्र॒जाप॑तेः पशू॒नामनु॑रूपतमः। आऽस्य॑ पु॒त्रः प्रति॑रूपो जायते। य ए॒वं वेद॑। सर्वा॑णि भू॒तानि॑ स॒म्भृत्याल॑भते। समे॑नन्दे॒वास्तेज॑से ब्रह्मवर्च॒साय॑ भरन्ति। योऽश्वमे॒धेन॒ यज॑ते॥८०॥

%3.9.22.3
य उ॑ चैनमे॒वं वेद॑। ए॒तद्वै तद्दे॒वा ए॒तान्दे॒वताम्। प॒शुम्भू॒तम्मेधा॒याल॑भन्त। य॒ज्ञमे॒व। य॒ज्ञेन॑ य॒ज्ञम॑यजन्त दे॒वाः। का॒म॒प्रं य॒ज्ञम॑कुर्वत। ते॑ऽमृत॒त्वम॑कामयन्त। ते॑ऽमृत॒त्वम॑गच्छन्। योऽश्वमे॒धेन॒ यज॑ते। दे॒वाना॑मे॒वाय॑नेनैति॥८१॥

%3.9.22.4
प्रा॒जा॒प॒त्येनै॒व य॒ज्ञेन॑ यजते काम॒प्रेण॑। अपु॑नर्मारमे॒व ग॑च्छति। ए॒तस्य॒ वै रू॒पेण॑ पु॒रस्तात्प्राजाप॒त्यमृ॑ष॒भं तू॑प॒रं ब॑हुरू॒पमाल॑भते। सर्वे॑भ्य॒ कामेभ्यः। सर्व॒स्याप्त्यै। सर्व॑स्य॒ जित्यै। सर्व॑मे॒व तेनाप्नोति। सर्वं॑ जयति। योऽश्वमे॒धेन॒ यज॑ते। य उ॑ चैनमे॒वं वेद॑॥८२॥\anuvakamend[मेधोऽभ॑व॒द्यज॑त एति॒ वेद॑]

%3.9.23.1
यो वा अश्व॑स्य॒ मेध्य॑स्य॒ लोम॑नी॒ वेद॑। अश्व॑स्यै॒व मेध्य॑स्य॒ लोमं॑ लोमं जुहोति। अ॒हो॒रा॒त्रे वा अश्व॑स्य॒ मेध्य॑स्य॒ लोम॑नी। यत्सा॒यं प्रा॑तर्जु॒होति॑। अश्व॑स्यै॒व मेध्य॑स्य॒ लोमं॑ लोमं जुहोति। ए॒तद॑नुकृति ह स्म॒ वै पु॒रा। अश्व॑स्य॒ मेध्य॑स्य॒ लोमं॑ लोमं जुह्वति। यो वा अश्व॑स्य॒ मेध्य॑स्य प॒दे वेद॑। अश्व॑स्यै॒व मेध्य॑स्य प॒देप॑दे जुहोति। द॒र्॒श॒पू॒र्ण॒मा॒सौ वा अश्व॑स्य॒ मेध्य॑स्य प॒दे॥८३॥

%3.9.23.2
यद्द॑र्‌शपूर्णमा॒सौ यज॑ते। अश्व॑स्यै॒व मेध्य॑स्य प॒देप॑दे जुहोति। ए॒तद॑नुकृति ह स्म॒ वै पु॒रा। अश्व॑स्य॒ मेध्य॑स्य प॒देप॑दे जुह्वति। यो वा अश्व॑स्य॒ मेध्य॑स्य वि॒वर्त॑नं॒ वेद॑। अश्व॑स्यै॒व मेध्य॑स्य वि॒वर्त॑नेविवर्तने जुहोति। अ॒सौ वा आ॑दि॒त्योऽश्व॑। स आ॑हव॒नीय॒माग॑च्छति। तद्विव॑र्तते। यद॑ग्निहो॒त्रं जु॒होति॑। अश्व॑स्यै॒व मेध्य॑स्य वि॒वर्त॑नेविवर्तने जुहोति। ए॒तद॑नुकृति ह स्म॒ वै पु॒रा। अ॑श्वस्य॒ मेध्य॑स्य वि॒वर्त॑नेविवर्तने जुह्वति॥८४॥\anuvakamend[प॒दे अ॑ग्निहो॒त्रं जु॒होति॒ त्रीणि॑ च]

\prashnaend{प्र॒जाप॑ति॒स्तम॑ष्टादशिभि॑ प्र॒जाप॑तिरकामयतो॒भाव॒स्मै यु॒ञ्जन्ति॒ तेज॒साऽप॑प्राणा अप॒श्रीरू॒र्ध्वां प्र॒जाप॑तिः प्रे॒णाऽनु॑ प्रथ॒मेन॑ प्र॒जाप॑तिरकामयत म॒हान्वैश्वदे॒वो वा अश्वोऽश्व॑स्य प्र॒जाप॑ति॒स्तं य॑ज्ञक्र॒तुभि॒रप॒श्रीर्ब्राह्म॒णौ सर्वे॑षु वारु॒णो यद्यश्व॒न्तदा॑हुरे॒ष वै वि॒भूस्ता॒र्प्येणा॑दि॒त्याः प्र॒जाप॑तिं पि॒तर॒य्योँ वा अश्व॑स्य॒ मेध्य॑स्य॒ लोम॑नी॒ त्रयो॑विशतिः॥२३॥}{प्र॒जाप॑तिर॒स्मिँल्लो॒क उ॑त्तर॒तः श्रिय॑मे॒व प्र॒जाप॑तिरकामयत म॒हान्यत्प्रा॒तः प्र वा ए॒ष ए॒भ्यो लो॒केभ्य॒ सर्व ह॒ वै तत्र॒ पय॑ स्व॒द्य उ॑ चैनमे॒वं वेद॑ च॒त्वार्यशी॑तिः॥८४॥}{प्र॒जाप॑तिरश्वमे॒धं जु॑ह्वति॥}{हरि॑ ओम्॥}{इति श्रीकृष्णयजुर्वेदीयतैत्तिरीयब्राह्मणे तृतीयाष्टके नवमः प्रपाठकः समाप्तः॥}
\clearpage
