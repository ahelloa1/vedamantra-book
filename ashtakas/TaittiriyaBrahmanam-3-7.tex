\sect{सप्तमः प्रश्नः}
\setcounter{anuvakam}{0}
\dnsub{तैत्तिरीयब्राह्मणे तृतीयाष्टके सप्तमः प्रपाठकः}

%3.7.1.1
सर्वा॒न्॒ वा ए॒षोऽग्नौ कामा॒न्प्रवे॑शयति। योऽग्नीन॑न्वा॒धाय॑ व्र॒तमु॒पैति॑। सयदनि॑ष्ट्वा प्रया॒यात्। अका॑मप्रीता एन॒ङ्कामा॒ नानु॒प्रया॑युः। अ॒ते॒जा अ॑वी॒र्य॑ स्यात्। स जु॑हुयात्। तुभ्य॒न्ता अ॑ङ्गिरस्तम। विश्वा सुक्षि॒तय॒ पृथ॑क्। अग्ने॒ कामा॑य येमिर॒ इति॑। कामा॑ने॒वास्मि॑न्दधाति॥१॥

%3.7.1.2
काम॑प्रीता एन॒ङ्कामा॒ अनु॒ प्रयान्ति। ते॒ज॒स्वी वी॒र्या॑वान्भवति। सन्त॑ति॒र्वा ए॒षा य॒ज्ञस्य॑। योऽग्नीन॑न्वा॒धाय॑ व्र॒तमु॒पैति॑। स यदु॒द्वाय॑ति। विच्छि॑त्तिरे॒वास्य॒ सा। तं प्राञ्च॑मु॒द्धृत्य॑। मन॒सोप॑तिष्ठेत। मनो॒ वै प्र॒जाप॑तिः। प्रा॒जा॒प॒त्यो य॒ज्ञः॥२॥

%3.7.1.3
मन॑सै॒व य॒ज्ञ सन्त॑नोति। भूरित्या॑ह। भू॒तो वै प्र॒जाप॑तिः। भूति॑मे॒वोपै॑ति। वि वा ए॒ष इ॑न्द्रि॒येण॑ वी॒र्ये॑णर्ध्यते। यस्याहि॑ताग्नेर॒ग्निर॑प॒क्षाय॑ति। याव॒च्छम्य॑या प्र॒विध्येत्। यदि॒ ताव॑दप॒क्षायेत्। त सम्भ॑रेत्। इ॒दन्त॒ एकं॑ प॒र उ॑त॒ एकम्॥३॥

%3.7.1.4
तृ॒तीये॑न॒ ज्योति॑षा॒ संवि॑शस्व। सं॒वेश॑नस्त॒नुवै॒ चारु॑रेधि। प्रि॒ये दे॒वानां पर॒मे ज॒नित्र॒ इति॑। ब्रह्म॑णै॒वैन॒ सम्भ॑रति। सैव तत॒ प्राय॑श्चित्तिः। यदि॑ परस्त॒राम॑प॒क्षायेत्। अ॒नु॒प्र॒यायाव॑स्येत्। सो ए॒व तत॒ प्राय॑श्चित्तिः। ओष॑धी॒र्वा ए॒तस्य॑ प॒शून्पय॒ प्रवि॑शति। यस्य॑ ह॒विषे॑ व॒त्सा अ॒पाकृ॑ता॒ धय॑न्ति॥४॥

%3.7.1.5
तान् यद्दु॒ह्यात्। या॒तयाम्ना ह॒विषा॑ यजेत। यन्न दु॒ह्यात्। य॒ज्ञ॒प॒रुर॒न्तरि॑यात्। वा॒य॒व्यां यवा॒गून्निर्व॑पेत्। वा॒युर्वै पय॑सः प्रदापयि॒ता। स ए॒वास्मै॒ पय॒ प्रदा॑पयति। पयो॒ वा ओष॑धयः। पय॒ पय॑। पय॑सै॒वास्मै॒ पयोऽव॑रुन्धे॥५॥

%3.7.1.6
अथोत्त॑रस्मै ह॒विषे॑ व॒त्सान॒पाकु॑र्यात्। सैव तत॒ प्राय॑श्चित्तिः। अ॒न्य॒त॒रान् वा ए॒ष दे॒वान्भा॑ग॒धेये॑न॒ व्य॑र्धयति। ये यज॑मानस्य सा॒यङ्गृ॒हमा॒ गच्छ॑न्ति। यस्य॑ सायन्दु॒ग्ध ह॒विरार्ति॑मा॒र्च्छति॑। इन्द्रा॑य व्री॒हीन्नि॒रुप्योप॑ वसेत्। पयो॒ वा ओष॑धयः। पय॑ ए॒वारभ्य॑ गृही॒त्वोप॑ वसति। यत्प्रा॒तः स्यात्। तच्छृ॒तं कु॑र्यात्॥६॥

%3.7.1.7
अथेत॑र ऐ॒न्द्रः पु॑रो॒डाश॑ स्यात्। इ॒न्द्रि॒ये ए॒वास्मै॑ स॒मीची॑ दधाति। पयो॒ वा ओष॑धयः। पय॒ पय॑। पय॑सै॒वास्मै॒ पयोऽव॑रुन्धे। अथोत्त॑रस्मै ह॒विषे॑ व॒त्सान॒पाकु॑र्यात्। सैव तत॒ प्राय॑श्चित्तिः। उ॒भया॒न्॒ वा ए॒ष दे॒वान्भा॑ग॒धेये॑न॒ व्य॑र्धयति। ये यज॑मानस्य सा॒यं च॑ प्रा॒तश्च॑ गृ॒हमा॒ गच्छ॑न्ति। यस्यो॒भय ह॒विरार्ति॑मा॒र्च्छति॑॥७॥

%3.7.1.8
ऐ॒न्द्रं पञ्च॑शरावमोद॒नन्निर्व॑पेत्। अ॒ग्निं दे॒वता॑नां प्रथ॒मं य॑जेत्। अ॒ग्निमु॑खा ए॒व दे॒वता प्रीणाति। अ॒ग्निं वा अन्व॒न्या दे॒वता। इन्द्र॒मन्व॒न्याः। ता ए॒वोभयी प्रीणाति। पयो॒ वा ओष॑धयः। पय॒ पय॑। पय॑सै॒वास्मै॒ पयोऽव॑रुन्धे। अथोत्तर॑स्मै ह॒विषे॑ व॒त्सान॒पाकु॑र्यात्॥८॥

%3.7.1.9
सैव तत॒ प्राय॑श्चित्तिः। अ॒र्धो वा ए॒तस्य॑ य॒ज्ञस्य॑ मीयते। यस्य॒ व्रत्येऽह॒न्पत्न्य॑नालम्भु॒का भव॑ति। ताम॑प॒रुध्य॑ यजेत। सर्वे॑णै॒व य॒ज्ञेन॑ यजते। तामि॒ष्ट्वोप॑ ह्वयेत। अमू॒हम॑स्मि। सा त्वम्। द्यौर॒हम्। पृ॒थि॒वी त्वम्। सामा॒हम्। ऋक्त्वम्। तावेहि॒ सम्भ॑वाव। स॒ह रेतो॑ दधावहै। पु॒से पु॒त्राय॒ वेत्त॑वै। रा॒यस्पोषा॑य सुप्रजा॒स्त्वाय॑ सु॒वीर्या॒येति॑। अ॒र्ध ए॒वैना॒मुप॑ ह्वयते। सैव तत॒ प्राय॑श्चित्तिः॥९॥\anuvakamend[द॒धा॒ति॒ य॒ज्ञ उ॑त॒ एक॒न्धय॑न्ति रुन्धे कुर्यादा॒र्च्छत्य॒पाकु॑र्यात्पृथि॒वी त्वम॒ष्टौ च॑ (सर्वा॒न्॒ वि वै यदि॑ परस्त॒रामोष॑धीरन्यत॒रानु॒भया॑न॒र्धो वै ॥ )]

%3.7.2.1
यद्विष्ष॑ण्णेन जुहु॒यात्। अप्र॑जा अप॒शुर्यज॑मानः स्यात्। यदना॑यतने नि॒नयेत्। अ॒ना॒य॒त॒नः स्यात्। प्रा॒जा॒प॒त्यय॒र्चा व॑ल्मीकव॒पाया॒मव॑ नयेत्। प्रा॒जा॒प॒त्यो वै व॒ल्मीक॑। य॒ज्ञः प्र॒जाप॑तिः। प्र॒जाप॑तावे॒व य॒ज्ञं प्रति॑ष्ठापयति। भूरित्या॑ह। भू॒तो वै प्र॒जाप॑तिः॥१०॥

%3.7.2.2
भूति॑मे॒वोपै॑ति। तत्कृ॒त्वा। अ॒न्यां दु॒ग्ध्वा पुन॑र्‌होत॒व्यम्। सैव तत॒ प्राय॑श्चित्तिः। यत्की॒टाव॑पन्नेन जुहु॒यात्। अप्र॑जा अप॒शुर्यज॑मानः स्यात्। यदना॑यतने नि॒नयेत्। अ॒ना॒य॒त॒नः स्यात्। म॒ध्य॒मेन॑ प॒र्णेन॑ द्यावापृथि॒व्य॑य॒र्चाऽन्त॑ परि॒धि निन॑येत्। द्यावा॑पृथि॒व्योरे॒वैन॒त्प्रति॑ष्ठापयति॥११॥

%3.7.2.3
तत्कृ॒त्वा। अ॒न्यां दु॒ग्ध्वा पुन॑र्\mbox{}होत॒व्यम्। सैव तत॒ प्राय॑श्चित्तिः। यदव॑वृष्टेन जुहु॒यात्। अप॑रूपमस्या॒त्मञ्जा॑येत। कि॒लासो॑ वा॒स्याद॑र्\mbox{}श॒सो वा। यत्प्रत्ये॒यात्। य॒ज्ञं विच्छि॑न्द्यात्। स जु॑हुयात्। मि॒त्रो जनान्कल्पयति प्रजा॒नन्॥१२॥

%3.7.2.4
मि॒त्रो दा॑धार पृथि॒वीमु॒त द्याम्। मि॒त्रः कृ॒ष्टीरनि॑मिषा॒ऽभि च॑ष्टे। स॒त्याय॑ ह॒व्यङ्घृ॒तव॑ज्जुहो॒तेति॑। मि॒त्रेणै॒वैन॑त्कल्पयति। तत्कृ॒त्वा। अ॒न्यां दु॒ग्ध्वा पुन॑र्\mbox{}होत॒व्यम्। सैव तत॒ प्राय॑श्चित्तिः। यत्पूर्व॑स्या॒माहु॑त्या हु॒ताया॒मुत्त॒राऽऽहु॑ति॒ स्कन्देत्। द्वि॒पाद्भि॑ प॒शुभि॒र्यज॑मानो॒ व्यृ॑ध्येत। यदुत्त॑रया॒ऽभि जु॑हु॒यात्॥१३॥

%3.7.2.5
चतु॑ष्पाद्भिः प॒शुभि॒र्यज॑मानो॒ व्यृ॑ध़्येत। यत्र॒ वेत्थ॑ वनस्पते दे॒वाना॒ङ्गुह्या॒ नामा॑नि। तत्र॑ ह॒व्यानि॑ गाम॒येति॑ वानस्प॒त्यय॒र्चा स॒मिध॑मा॒धाय॑। तू॒ष्णीमे॒व पुन॑र्जुहुयात्। वन॒स्पति॑नै॒व य॒ज्ञस्यार्ता॒ञ्चानार्ता॒ञ्चाहु॑ती॒ वि दा॑धार। तत्कृ॒त्वा। अ॒न्यां दु॒ग्ध्वा पुन॑र्\mbox{}होत॒व्यम्। सैव तत॒ प्राय॑श्चित्तिः। यत्पु॒रा प्र॑या॒जेभ्य॒ प्राङङ्गा॑र॒ स्कन्देत्। अ॒ध्व॒र्यवे॑ च॒ यज॑मानाय॒ चाक स्यात्॥१४॥

%3.7.2.6
यद्द॑क्षि॒णा। ब्र॒ह्मणे॑ च॒ यज॑मानाय॒ चाक स्यात्। यत्प्र॒त्यक्। होत्रे॑ च॒ पत्नि॑यै च॒ यज॑मानाय॒ चाक स्यात्। यदुदङ्ङ्॑। अ॒ग्नीधे॑ च प॒शुभ्य॑श्च॒ यज॑मानाय॒ चाक स्यात्। यद॑भिजुहु॒यात्। रु॒द्रोस्य प॒शून्घातु॑कः स्यात्। यन्नाभि॑जुहु॒यात्। अशान्त॒ प्रह्रि॑येत॥१५॥

%3.7.2.7
स्रु॒वस्य॒ बुध्ने॑नाभि॒निद॑ध्यात्। मा त॑मो॒ मा य॒ज्ञस्त॑म॒न्मा यज॑मानस्तमत्। नम॑स्ते अस्त्वाय॒ते। नमो॑ रुद्र पराय॒ते। नमो॒ यत्र॑ नि॒षीद॑सि। अ॒मुं मा हिसीर॒मुं मा हिसी॒रिति॒ येन॒ स्कन्देत्। तं प्रह॑रेत्। स॒हस्र॑शृङ्गो वृष॒भो जा॒तवे॑दाः। स्तोम॑पृष्ठो घृ॒तवान्त्सु॒प्रती॑कः। मा नो॑ हासीन्मेत्थि॒तो नेत्त्वा॒ जहा॑म। गो॒पो॒षन्नो॑ वीरपो॒षं च॑ य॒च्छेति॑। ब्रह्म॑णै॒वैनं॒ प्र ह॑रति। सैव तत॒ प्राय॑श्चित्तिः॥१६॥\anuvakamend[वै प्र॒जाप॑तिः स्थापयति प्रजा॒नन्न॒भि जु॑हु॒यात्स्याद्ध्रियेत॒ जहा॑म॒ त्रीणि॑ च (यद्विष्ष॑ण्णेन प्राजाप॒त्यया॒ यत्की॒टा म॑ध्य॒मेन॒ यदव॑वृष्टेन॒ यत्पूर्व॑स्यां॒ यत्पु॒रा प्र॑या॒जेभ्य॒ प्राङङ्गा॑रो॒ यद्द॑क्षि॒णा यत्प्र॒त्यग्यदुदङ्ङ्॑ ॥ )]

%3.7.3.1
वि वा ए॒ष इ॑न्द्रि॒येण॑ वी॒र्ये॑णर्ध्यते। यस्याहि॑ताग्नेर॒ग्निर्म॒थ्यमा॑नो॒ न जाय॑ते। यत्रा॒न्यं पश्येत्। तत॑ आ॒हृत्य॑ होत॒व्यम्। अ॒ग्नावे॒वास्याग्निहो॒त्र हु॒तं भ॑वति। यद्य॒न्यन्न वि॒न्देत्। अ॒जाया होत॒व्यम्। आ॒ग्ने॒यी वा ए॒षा। यद॒जा। अ॒ग्नावे॒वास्याग्निहो॒त्र हु॒तं भ॑वति॥१७॥

%3.7.3.2
अ॒जस्य॒ तु नाश्ञी॑यात्। यद॒जस्याश्ञी॒यात्। यामे॒वाग्नावाहु॑तिं जुहु॒यात्। ताम॑द्यात्। तस्मा॑द॒जस्य॒ नाश्यम्। यद्य॒जान्न वि॒न्देत्। ब्रा॒ह्म॒णस्य॒ दक्षि॑णे॒ हस्ते॑ होत॒व्यम्। ए॒ष वा अ॒ग्निर्वैश्वान॒रः। यद्ब्राह्म॒णः। अ॒ग्नावे॒वास्याग्निहो॒त्र हु॒तं भ॑वति॥१८॥

%3.7.3.3
ब्रा॒ह्म॒णन्तु व॑स॒त्यै॑ नाप॑ रुन्ध्यात्। यद्ब्राह्म॒णं व॑स॒त्या अ॑परु॒न्ध्यात्। यस्मि॑न्ने॒वाग्नावाहु॑तिं जुहु॒यात्। तम्भा॑ग॒धेये॑न॒ व्य॑र्धयेत्। तस्माद्ब्राह्म॒णो व॑स॒त्यै॑ नाप॒रुध्य॑। यदि॑ ब्राह्म॒णन्न वि॒न्देत्। द॒र्भ॒स्त॒म्बे हो॑त॒व्यम्। अ॒ग्नि॒वाऩ््वै द॑र्भस्त॒म्बः। अ॒ग्नावे॒वास्याग्निहो॒त्र हु॒तं भ॑वति। द॒र्भास्तु नाध्या॑सीत॥१९॥

%3.7.3.4
यद्द॒र्भान॒ध्यासी॑त। यामे॒वाग्नावाहु॑तिं जुहु॒यात्। तामध्या॑सीत। तस्माद्द॒र्भा नाध्या॑सित॒व्या। यदि॑ द॒र्भान्न वि॒न्देत्। अ॒प्सु हो॑त॒व्यम्। आपो॒ वै सर्वा॑ दे॒वता। दे॒वतास्वे॒वास्याग्निहो॒त्र हु॒तं भ॑वति। आप॒स्तु न परि॑चक्षीत। यदाप॑ परि॒चक्षी॑त॥२०॥

%3.7.3.5
यामे॒वाप्स्वाहु॑तिं जुहु॒यात्। तां परि॑चक्षीत। तस्मा॒दापो॒ न प॑रि॒चक्ष्या। मेध्या॑ च॒ वा ए॒तस्या॑मे॒ध्या च॑ त॒नुवौ॒ स सृ॑ज्येते। यस्याहि॑ताग्नेर॒न्यैर॒ग्निभि॑र॒ग्नय॑ ससृ॒ज्यन्ते। अ॒ग्नये॒ विवि॑चये पुरो॒डाश॑म॒ष्टाक॑पालं॒ निर्व॑पेत्। मेध्याञ्चै॒वास्या॑मे॒ध्यां च॑ त॒नुवौ॒ व्याव॑र्तयति। अ॒ग्नये व्र॒तप॑तये पु॒रो॒डाश॑म॒ष्टाक॑पालं॒ निर्व॑पेत्। अ॒ग्निमे॒व व्र॒तप॑ति॒ स्वेन॑ भाग॒धेये॒नोप॑ धावति। स ए॒वैनं॑ व्र॒तमा ल॑म्भयति॥२१॥

%3.7.3.6
गर्भ॒ स्रव॑न्तमग॒दम॑कः। अ॒ग्निरिन्द्र॒स्त्वष्टा॒ बृह॒स्पति॑। पृ॒थि॒व्यामव॑ चुश्चोतै॒तत्। नाभिप्राप्नो॑ति॒ निर्‌ऋ॑तिं परा॒चैः। रेतो॒ वा ए॒तद्वाजि॑न॒माहि॑ताग्नेः। यद॑ग्निहो॒त्रम्। तद्यत्स्रवेत्। रेतोऽस्य॒ वाजि॑न स्रवेत्। गर्भ॒ स्रव॑न्तमग॒दम॑क॒रित्या॑ह। रेत॑ ए॒वास्मि॒न्वाजि॑नन्दधाति॥२२॥

%3.7.3.7
अ॒ग्निरित्या॑ह। अ॒ग्निर्वै रे॑तो॒धाः। रेत॑ ए॒व तद्द॑धाति। इन्द्र॒ इत्या॑ह। इ॒न्द्रि॒यमे॒वास्मि॑न्दधाति। त्वष्टेत्या॑ह। त्वष्टा॒ वै प॑शू॒नां मि॑थु॒नाना रूप॒कृत्। रू॒पमे॒व प॒शुषु॑ दधाति। बृह॒स्पति॒रित्या॑ह। ब्रह्म॒ वै दे॒वानां॒ बृह॒स्पति॑। ब्रह्म॑णै॒वास्मै प्र॒जाः प्र ज॑नयति। पृ॒थि॒व्यामव॑ चुश्चोतै॒तदित्या॑ह। अ॒स्यामे॒वैन॒त्प्रति॑ष्ठापयति। नाभिप्राप्नो॑ति॒ निर्‌ऋ॑तिं पराचै॒रित्या॑ह। रक्ष॑सा॒मप॑हत्यै॥२३॥\anuvakamend[अ॒जाऽग्नावे॒वास्याग्निहो॒त्र हु॒तं भ॑वति भवत्यासीत परि॒चक्षी॑त लम्भयति दधाति दे॒वानां॒ बृह॒स्पति॒ पञ्च॑ च  (वि वै यद्य॒न्यम॒जायां ब्राह्म॒णस्य॑ दर्भस्त॒म्बेऽप्सु हो॑त॒व्यम्॥ )]

%3.7.4.1
याः पु॒रस्तात्प्र॒स्रव॑न्ति। उ॒परि॑ष्टात्स॒र्वत॑श्च॒ याः। ताभी॑ र॒श्मिप॑वित्राभिः। श्र॒द्धां य॒ज्ञमा र॑भे। देवा॑ गातुविदः। गा॒तुं य॒ज्ञाय॑ विन्दत। मन॑स॒स्पति॑ना दे॒वेन॑। वाताद्य॒ज्ञः प्र यु॑ज्यताम्। तृ॒तीय॑स्यै दि॒वः। गा॒य॒त्रि॒या सोम॒ आभृ॑तः॥२४॥

%3.7.4.2
सो॒म॒पी॒थाय॒ सन्न॑यितुम्। वक॑ल॒मन्त॑र॒मा द॑दे। आपो॑ देवीः शु॒द्धाः स्थ॑। इ॒मा पात्रा॑णि शुन्धत। उ॒पा॒त॒ङ्क्या॑य दे॒वानाम्। प॒र्ण॒व॒ल्कमु॒त शु॑न्धत। पयो॑ गृ॒हेषु॒ पयो॑ अघ्नि॒यासु॑। पयो॑ व॒त्सेषु॒ पय॒ इन्द्रा॑य ह॒विषे ध्रियस्व। गा॒य॒त्री प॑र्णव॒ल्केन॑। पय॒ सोमं॑ करोत्वि॒मम्॥२५॥

%3.7.4.3
अ॒ग्निं गृ॑ह्णामि सु॒रथ॒य्योँ म॑यो॒भूः। य उ॒द्यन्त॑मा॒रोह॑ति॒ सूर्य॒मह्ने। आ॒दि॒त्यञ्ज्योति॑षां॒ ज्योति॑रुत्त॒मम्। श्वो य॒ज्ञाय॑ रमतान्दे॒वताभ्यः। वसून्रु॒द्राना॑दि॒त्यान्। इन्द्रे॑ण स॒ह दे॒वता। ताः पूर्व॒ परि॑ गृह्णामि। स्व आ॒यत॑ने मनी॒षया। इ॒मामूर्जं॑ पञ्चद॒शींय्येँ प्रवि॑ष्टाः। तान्दे॒वान्परि॑ गृह्णामि॒ पूर्व॑॥२६॥

%3.7.4.4
अ॒ग्निर्‌ह॑व्य॒वाडि॒ह ताना व॑हतु। पौ॒र्णमा॒स ह॒विरि॒दमे॑षां॒ मयि॑। आ॒मा॒वा॒स्य ह॒विरि॒दमे॑षां॒ मयि॑। अ॒न्त॒राऽग्नी प॒शव॑। दे॒व॒स॒सद॒मा ग॑मन्। तान्पूर्व॒ परि॑ गृह्णामि। स्व आ॒यत॑ने मनी॒षया। इ॒ह प्र॒जा वि॒श्वरू॑पा रमन्ताम्। अ॒ग्निङ्गृ॒हप॑तिम॒भि सं॒वसा॑नाः। ताः पूर्व॒ परि॑ गृह्णामि॥२७॥

%3.7.4.5
स्व आ॒यत॑ने मनी॒षया। इ॒ह प॒शवो॑ वि॒श्वरू॑पा रमन्ताम्। अ॒ग्निङ्गृ॒हप॑तिम॒भि सं॒वसा॑नाः। तान्पूर्व॒ परि॑ गृह्णामि। स्व आ॒यत॑ने मनी॒षया। अ॒यं पि॑तृ॒णाम॒ग्निः। अवाड्ढ॒व्या पि॒तृभ्य॒ आ। तं पूर्व॒ परि॑ गृह्णामि। अवि॑षन्नः पि॒तुङ्क॑रत्। अज॑स्र॒न्त्वा स॑भापा॒लाः॥२८॥

%3.7.4.6
वि॒ज॒यभा॑ग॒ समि॑न्धताम्। अग्ने॑ दी॒दा॑य मे सभ्य। विजि॑त्यै श॒रद॑ श॒तम्। अन्न॑मावस॒थीयम्। अ॒भि ह॑राणि श॒रद॑ श॒तम्। आ॒व॒स॒थे श्रियं॒ मन्त्रम्। अहि॑र्बु॒ध्नियो॒ नि य॑च्छतु। इ॒दम॒हम॒ग्निज्येष्ठेभ्यः। वसु॑भ्यो य॒ज्ञं प्रब्र॑वीमि। इ॒दम॒हमिन्द्र॑ज्येष्ठेभ्यः॥२९॥

%3.7.4.7
रु॒द्रेभ्यो॑ य॒ज्ञं प्र ब्र॑वीमि। इ॒दम॒हं वरु॑णज्येष्ठेभ्यः। आ॒दि॒त्येभ्यो॑ य॒ज्ञं प्र ब्र॑वीमि। पय॑स्वती॒रोष॑धयः। पय॑स्वद्वी॒रुधां॒ पय॑। अ॒पां पय॑सो॒ यत्पय॑। तेन॒ मामि॑न्द्र॒ स सृ॑ज। अग्ने व्रतपते व्र॒तञ्च॑रिष्यामि। तच्छ॑केय॒न्तन्मे॑ राध्यताम्। वायो व्रतपत॒ आदि॑त्य व्रतपते॥३०॥

%3.7.4.8
व्र॒तानां व्रतपते व्र॒तं च॑रिष्यामि। तच्छ॑केय॒न्तन्मे॑ राध्यताम्। इ॒मां प्राची॒मुदी॑चीम्। इष॒मूर्ज॑म॒भि सस्कृ॑ताम्। ब॒हु॒प॒र्णामशु॑ष्काग्राम्। हरा॑मि पशु॒पाम॒हम्। यत्कृष्णो॑ रू॒पं कृ॒त्वा। प्रावि॑श॒स्त्वं वन॒स्पतीन्॑। तत॒स्त्वामे॑कविशति॒धा। सम्भ॑रामि सुस॒म्भृता॥३१॥

%3.7.4.9
त्रीन्प॑रि॒धी स्ति॒स्रः स॒मिध॑। य॒ज्ञायु॑रनुसञ्च॒रान्। उ॒प॒वे॒षं मेक्ष॑णं॒ धृष्टिम्। सं भ॑रामि सुस॒म्भृता। या जा॒ता ओष॑धयः। दे॒वेभ्य॑स्त्रियु॒गं पु॒रा। तासां॒ पर्व॑ राध्यासम्। प॒रि॒स्त॒रमा॒हर\sn{}॑। अ॒पां मेध्यं॑ य॒ज्ञियम्। सदे॑व शि॒वम॑स्तु मे॥३२॥

%3.7.4.10
आ॒च्छे॒त्ता वो॒ मा रि॑षम्। जीवा॑नि श॒रद॑ श॒तम्। अप॑रिमितानां॒ परि॑मिताः। सन्न॑ह्ये सुकृ॒ताय॒ कम्। एनो॒ मा निगाङ्कत॒मच्च॒नाहम्। पुन॑रु॒त्थाय॑ बहु॒ला भ॑वन्तु। स॒कृ॒दा॒च्छि॒न्नं ब॒र्॒हिरूर्णा॑मृदु। स्यो॒नं पि॒तृभ्य॑स्त्वा भराम्य॒हम्। अ॒स्मिन्त्सी॑दन्तु मे पि॒तर॑ सो॒म्याः। पि॒ता॒म॒हाः प्रपि॑तामहाश्चानु॒गैः स॒ह॥३३॥

%3.7.4.11
त्रि॒वृत्प॑ला॒शे द॒र्भः। इयान्प्रादे॒शस॑म्मितः। य॒ज्ञे प॒वित्रं॒ पोतृ॑तमम्। पयो॑ ह॒व्यं क॑रोतु मे। इ॒मौ प्रा॑णापा॒नौ। य॒ज्ञस्याङ्गा॑नि सर्व॒शः। आ॒प्या॒यय॑न्तौ॒ सञ्च॑रताम्। प॒वित्रे॑ हव्य॒शोध॑ने। प॒वित्रे स्थो वैष्ण॒वी। वा॒युर्वां॒ मन॑सा पुनातु॥३४॥

%3.7.4.12
अ॒यं प्रा॒णश्चा॑पा॒नश्च॑। यज॑मान॒मपि॑ गच्छताम्। य॒ज्ञे ह्यभू॑तां॒ पोता॑रौ। प॒वित्रे॑ हव्य॒शोध॑ने। त्वया॒ वेदिं॑ विविदुः पृथि॒वीम्। त्वया॑ य॒ज्ञो जा॑यते विश्व॒दानि॑। अच्छि॑द्रं य॒ज्ञमन्वे॑षि वि॒द्वान्। त्वया॒ होता॒ सन्त॑नोत्यर्धमा॒सान्। त्र॒य॒स्त्रि॒शो॑ऽसि॒ तन्तू॑नाम्। प॒वित्रे॑ण स॒हाग॑हि॥३५॥

%3.7.4.13
शि॒वेय रज्जु॑रभि॒धानी। अ॒घ्नि॒यामुप॑ सेवताम्। अप्र॑स्रसाय य॒ज्ञस्य॑। उ॒खे उप॑दधाम्य॒हम्। प॒शुभि॒ सन्नी॑तं बिभृताम्। इन्द्रा॑य शृ॒तन्दधि॑। उ॒प॒वे॒षो॑ऽसि य॒ज्ञाय॑। त्वां प॑रिवे॒षम॑धारयन्। इन्द्रा॑य ह॒विः कृ॒ण्वन्त॑। शि॒वः श॒ग्मो भ॑वासि नः॥३६॥

%3.7.4.14
अमृ॑न्मयन्देवपा॒त्रम्। य॒ज्ञस्यायु॑षि॒ प्र यु॑ज्यताम्। ति॒र॒ प॒वि॒त्रमति॑नीताः। आपो॑ धारय॒ माति॑गुः। दे॒वेन॑ सवि॒त्रोत्पू॑ताः। वसो॒ सूर्य॑स्य र॒श्मिभि॑। गान्दो॑हपवि॒त्रे रज्जुम्। सर्वा॒ पात्रा॑णि शुन्धत। ए॒ता आ च॑रन्ति॒ मधु॑म॒द्दुहा॑नाः। प्र॒जाव॑तीर्य॒शसो॑ वि॒श्वरू॑पाः॥३७॥

%3.7.4.15
ब॒ह्वीर्भव॑न्ती॒रुप॒जाय॑मानाः। इ॒ह व॒ इन्द्रो॑ रमयतु गावः। पू॒षा स्थ॑। अ॒य॒क्ष्मा व॑ प्र॒जया॒ स सृ॑जामि। रा॒यस्पोषे॑ण बहु॒लाभव॑न्तीः। ऊर्जं॒ पय॒ पिन्व॑माना घृ॒तं च॑। जी॒वो जीव॑न्ती॒रुप॑वः सदेयम्। द्यौश्चे॒मं य॒ज्ञं पृ॑थि॒वी च॒ सन्दु॑हाताम्। धा॒ता सोमे॑न स॒ह वाते॑न वा॒युः। यज॑मानाय॒ द्रवि॑णन्दधातु॥३८॥

%3.7.4.16
उत्स॑न्दुहन्ति क॒लश॒ञ्चतु॑र्बिलम्। इडान्दे॒वीम्मधु॑मती सुव॒र्विदम्। तदि॑न्द्रा॒ग्नी जि॑न्वत सू॒नृता॑वत्। तद्यज॑मानममृत॒त्वे द॑धातु। काम॑धुक्ष॒ प्र णो ब्रूहि। इन्द्रा॑य ह॒विरि॑न्द्रि॒यम्। अ॒मूं यस्यां दे॒वानाम्। म॒नु॒ष्या॑णां॒ पयो॑ हि॒तम्। ब॒हु दु॒ग्धीन्द्रा॑य दे॒वेभ्य॑। ह॒व्यमा प्या॑यतां॒ पुन॑॥३९॥

%3.7.4.17
व॒त्सेभ्यो॑ मनु॒ष्येभ्यः। पु॒न॒र्दो॒हाय॑ कल्पताम्। य॒ज्ञस्य॒ सन्त॑तिरसि। य॒ज्ञस्य॑ त्वा॒ सन्त॑ति॒मनु॒ सन्त॑नोमि। अद॑स्तमसि॒ विष्ण॑वे त्वा। य॒ज्ञायापि॑ दधाम्य॒हम्। अ॒द्भिररि॑क्तेन॒ पात्रे॑ण। याः पू॒ताः प॑रि॒शेर॑ते। अ॒यं पय॒ सोमं॑ कृ॒त्वा। स्वाय्योँनि॒मपि॑ गच्छतु॥४०॥

%3.7.4.18
प॒र्ण॒व॒ल्कः प॒वित्रम्। सौ॒म्यः सोमा॒द्धि निर्मि॑तः। इ॒मौ प॒र्णं च॑ द॒र्भं च॑। दे॒वाना हव्य॒शोध॑नौ। प्रा॒त॒र्वे॒षाय॑ गोपाय। विष्णो॑ ह॒व्य हि रक्ष॑सि। उ॒भाव॒ग्नी उ॑पस्तृण॒ते। दे॒वता॒ उप॑वसन्तु मे। अ॒हङ्ग्रा॒म्यानुप॑ वसामि। मह्य॒ङ्गोप॑तये प॒शून्॥४१॥\anuvakamend[आभृ॑त इ॒मं गृ॑ह्णामि॒ पूर्व॒स्ताः पूर्व॒ परि॑गृह्णामि सभापा॒ला इन्द्र॑ज्येष्ठेभ्य॒ आदि॑त्य व्रतपते सुस॒म्भृता॑ मे स॒ह पु॑नातु गहि नो वि॒श्वरू॑पा दधातु॒ पुन॑र्गच्छतु प॒शून् (याः पु॒रस्ता॑दि॒मामूर्ज॑मि॒ह प्र॒जा इ॒ह प॒शवो॒ऽयं पि॑तृ॒णाम॒ग्निः। )]

%3.7.5.1
देवा॑ दे॒वेषु॒ पराक्रमध्वम्। प्रथ॑मा द्वि॒तीये॑षु। द्विती॑यास्तृ॒तीये॑षु। त्रिरे॑कादशा इ॒ह मा॑ऽवत। इ॒द श॑केयं॒ यदि॒दं क॒रोमि॑। आ॒त्मा क॑रोत्वा॒त्मने। इ॒दङ्क॑रिष्ये भेष॒जम्। इ॒दम्मे॑ विश्वभेषजा। अश्वि॑ना॒ प्राव॑तं यु॒वम्। इ॒दम॒ह सेना॑या अ॒भीत्व॑र्यै॥४२॥

%3.7.5.2
मुख॒मपो॑हामि। सूर्य॑ ज्योति॒र्वि भा॑हि। म॒ह॒त इ॑न्द्रि॒याय॑। आ प्या॑यताङ्घृ॒तयो॑निः। अ॒ग्निर्‌ह॒व्याऽनु॑ मन्यताम्। खम॑ङ्क्ष्व॒ त्वच॑मङ्क्ष्व। सु॒रू॒पन्त्वा॑ वसु॒विदम्। प॒शू॒नान्तेज॑सा। अ॒ग्नये॒ जुष्ट॑म॒भि घा॑रयामि। स्यो॒नन्ते॒ सद॑नं करोमि॥४३॥

%3.7.5.3
घृ॒तस्य॒ धार॑या सु॒शेव॑ङ्कल्पयामि। तस्मिन्त्सीदा॒मृते॒ प्रति॑तिष्ठ। व्री॒ही॒णाम्मे॑ध सुमन॒स्यमा॑नः। आ॒र्द्रः प्र॑थस्नु॒र्भुव॑नस्य गो॒पाः। शृ॒त उत्स्ना॑ति जनि॒ता म॑ती॒नाम्। यस्त॑ आ॒त्मा प॒शुषु॒ प्रवि॑ष्टः। दे॒वानां वि॒ष्ठामनु॒ यो वि॑त॒स्थे। आ॒त्म॒न्वान्त्सो॑म घृ॒तवा॒न्॒ हि भू॒त्वा। दे॒वान्ग॑च्छ॒ सुव॑र्विन्द॒ यज॑मानाय॒ मह्यम्। इरा॒ भूति॑ पृथि॒व्यै रसो॒ मोत्क्र॑मीत्॥४४॥

%3.7.5.4
देवा पितर॒ पित॑रो देवाः। यो॑ऽहम॑स्मि॒ स सन् य॑जे। यस्यास्मि॒ न तम॒न्तरे॑मि। स्वं म॑ इ॒ष्ट स्वन्द॒त्तम्। स्वं पू॒र्त स्व श्रा॒न्तम्। स्व हु॒तम्। तस्य॑ मे॒ऽग्निरु॑पद्र॒ष्टा। वा॒युरु॑पश्रो॒ता। आ॒दि॒त्यो॑ऽनुख्या॒ता। द्यौः पि॒ता॥४५॥

%3.7.5.5
पृ॒थि॒वी मा॒ता। प्र॒जाप॑ति॒र्बन्धु॑। य ए॒वास्मि॒ स सन् य॑जे। मा भेर्मा संवि॑क्था॒ मा त्वा॑ हिसिषम्। मा ते॒ तेजोऽप॑ क्रमीत्। भ॒र॒तमुद्ध॑रे॒मनु॑ षिञ़्च। अ॒व॒दाना॑नि ते प्र॒त्यव॑दास्यामि। नम॑स्ते अस्तु॒ मा मा॑ हिसीः। यद॑व॒दाना॑नि तेऽव॒द्यन्। विलो॒माका॑र्‌षमा॒त्मन॑॥४६॥

%3.7.5.6
आज्ये॑न॒ प्रत्य॑नज्म्येनत्। तत्त॒ आ प्या॑यतां॒ पुन॑। अज्या॑यो यवमा॒त्रात्। आ॒व्या॒धात्कृ॑त्यतामि॒दम्। मा रू॑रुपाम य॒ज्ञस्य॑। शु॒द्ध स्वि॑ष्टमि॒द ह॒विः। मनु॑ना दृ॒ष्टाङ्घृतप॑दीम्। मि॒त्रावरु॑णसमीरिताम्। द॒क्षि॒णा॒र्धादसं॑भिन्दन्। अव॑द्याम्येक॒तोमु॑खाम्॥४७॥

%3.7.5.7
इडे॑ भा॒गं जु॑षस्व नः। जिन्व॒ गा जिन्वार्व॑तः। तस्यास्ते भक्षि॒वाण॑ स्याम। स॒र्वात्मा॑नः स॒र्वग॑णाः। ब्रध्न॒ पिन्व॑स्व। दद॑तो मे॒ मा क्षा॑यि। कु॒र्व॒तो मे॒ मोप॑दसत्। दि॒शाङ्कॢप्ति॑रसि। दिशो॑ मे कल्पन्ताम्। कल्प॑न्ताम्मे॒ दिश॑॥४८॥

%3.7.5.8
दैवीश्च॒ मानु॑षीश्च। अ॒हो॒रा॒त्रे मे॑ कल्पेताम्। अ॒र्ध॒मा॒सा मे॑ कल्पन्ताम्। मासा॑ मे कल्पन्ताम्। ऋ॒तवो॑ मे कल्पन्ताम्। सं॒व॒त्स॒रो मे॑ कल्पताम्। कॢप्ति॑रसि॒ कल्प॑तां मे। आशा॑नां त्वाऽऽशापा॒लेभ्य॑। च॒तुर्भ्यो॑ अ॒मृतेभ्यः। इ॒दं भू॒तस्याध्य॑क्षेभ्यः॥४९॥

%3.7.5.9
वि॒धेम॑ ह॒विषा॑ व॒यम्। भज॑तां भा॒गी भा॒गम्। मा भा॒गोऽभ॑क्त। निर॑भा॒गं भ॑जामः। अ॒पस्पि॑न्व। ओष॑धीर्जिन्व। द्वि॒पात्पा॑हि। चतु॑ष्पादव। दि॒वो वृष्टि॒मेर॑य। ब्रा॒ह्म॒णाना॑मि॒द ह॒विः॥५०॥

%3.7.5.10
सो॒म्याना सोमपी॒थिनाम्। निर्भ॒क्तो ब्राह्मणः। नेहा ब्राह्मणस्यास्ति। सम॑ङ्क्तां ब॒र्॒हिर्‌ह॒विषा॑ घृ॒तेन॑। समा॑दि॒त्यैर्वसु॑भि॒ सम्म॒रुद्भि॑। समिन्द्रे॑ण॒ विश्वे॑भिर्दे॒वेभि॑रङ्क्ताम्। दि॒व्यं नभो॑ गच्छतु॒ यत्स्वाहा। इ॒न्द्रा॒णीवा॑ऽविध॒वा भू॑यासम्। अदि॑तिरिव सुपु॒त्रा। अ॒स्थू॒रि त्वा॑ गार्‌हपत्य॥५१॥

%3.7.5.11
उप॒निष॑दे सुप्रजा॒स्त्वाय॑। सं पत्नी॒ पत्या॑ सुकृ॒तेन॑ गच्छताम्। य॒ज्ञस्य॑ यु॒क्तौ धुर्या॑वभूताम्। सं॒जा॒ना॒नौ विज॑हता॒मरा॑तीः। दि॒वि ज्योति॑र॒जर॒मा र॑भेताम्। दश॑ते त॒नुवो॑ यज्ञ य॒ज्ञिया। ताः प्री॑णातु॒ यज॑मानो घृ॒तेन॑। ना॒रि॒ष्ठयो प्र॒शिष॒मीड॑मानः। दे॒वानां॒ दैव्येऽपि॒ यज॑मानो॒ऽमृतो॑ऽभूत्। यं वान्दे॒वा अ॑कल्पयन्॥५२॥

%3.7.5.12
ऊ॒र्जो भा॒ग श॑तक्रतू। ए॒तद्वां॒ तेन॑ प्रीणानि। तेन॑ तृप्यतमहहौ। अ॒हन्दे॒वाना सु॒कृता॑मस्मि लो॒के। ममे॒दमि॒ष्टन्न मिथु॑र्भवाति। अ॒हन्ना॑रि॒ष्ठावनु॑ यजामि वि॒द्वान्। यदाभ्या॒मिन्द्रो॒ अद॑धाद्भाग॒धेय॑म्। अदा॑रसृद्भवत देवसोम। अ॒स्मिन् य॒ज्ञे म॑रुतो मृडता नः। मा नो॑ विदद॒भिभा॒मो अश॑स्तिः॥५३॥

%3.7.5.13
मा नो॑ विदद्वृ॒जना॒ द्वेष्या॒ या। ऋ॒ष॒भं वा॒जिनं॑ व॒यम्। पू॒र्णमा॑सं यजामहे। स नो॑ दोहता सु॒वीर्यम्। रा॒यस्पोष सह॒स्रिणम्। प्रा॒णाय॑ सु॒राध॑से। पू॒र्णमा॑साय॒ स्वाहा। अ॒मा॒वा॒स्या॑ सु॒भगा॑ सु॒शेवा। धे॒नुरि॑व॒ भूय॑ आ॒प्याय॑माना। सा नो॑ दोहता सु॒वीर्यम्। रा॒यस्पोष सह॒स्रिणम्। अ॒पा॒नाय॑ सु॒राध॑से। अ॒मा॒वा॒स्या॑यै॒ स्वाहा। अ॒भि स्तृ॑णीहि॒ परि॑ धेहि॒ वेदिम्। जा॒मिम्मा हिसीरमु॒या शया॑ना। हो॒तृ॒षद॑ना॒ हरि॑ताः सु॒वर्णा। नि॒ष्का इ॒मे यज॑मानस्य ब्र॒ध्ने॥५४॥\anuvakamend[अ॒भीत्व॑र्यै करोमि क्रमीत्पि॒ताऽऽत्मन॑ एक॒तो मु॑खां मे॒ दिशोऽध्य॑क्षेभ्यो ह॒विर्गा॑र्‌हपत्या कल्पय॒न्नश॑स्ति॒ सा नो॑ दोहता सु॒वीर्य स॒प्त च॑]

%3.7.6.1
परि॑स्तृणीत॒ परि॑धत्ता॒ग्निम्। परि॑हितो॒ऽग्निर्यज॑मानं भुनक्तु। अ॒पा रस॒ ओष॑धीना सु॒वर्ण॑। नि॒ष्का इ॒मे यज॑मानस्य सन्तु काम॒दुघा। अ॒मुत्रा॒मुष्मि॑ल्लोँ॒के। भूप॑ते॒ भुव॑नपते। म॒ह॒तो भू॒तस्य॑ पते। ब्र॒ह्माण॑न्त्वा वृणीमहे। अ॒हं भूप॑तिर॒हं भुव॑नपतिः। अ॒हं म॑ह॒तो भू॒तस्य॒ पति॑॥५५॥

%3.7.6.2
दे॒वेन॑ सवि॒त्रा प्रसू॑त॒ आर्त्वि॑ज्यङ्करिष्यामि। देव॑ सवितरे॒तन्त्वा॑ वृणते। बृह॒स्पतिं॒ दैव्यं॑ ब्र॒ह्माणम्। तद॒हं मन॑से॒ प्र ब्र॑वीमि। मनो॑ गायत्रि॒यै। गा॒य॒त्री त्रि॒ष्टुभे। त्रि॒ष्टुब्जग॑त्यै। जग॑त्यनु॒ष्टुभे। अ॒नु॒ष्टुक्प॒ङ्क्त्यै। प॒ङ्क्तिः प्र॒जाप॑तये॥५६॥

%3.7.6.3
प्र॒जाप॑ति॒र्विश्वेभ्यो दे॒वेभ्य॑। विश्वे॑ देवा॒ बृह॒स्पत॑ये। बृह॒स्पति॒र्ब्रह्म॑णे। ब्रह्म॒ भूर्भुव॒ सुव॑। बृह॒स्पति॑र्दे॒वानां ब्र॒ह्मा। अ॒हं म॑नु॒ष्या॑णाम्। बृह॑स्पते य॒ज्ञङ्गो॑पाय। इ॒दं तस्मै॑ ह॒र्म्यं क॑रोमि। यो वो॑ देवा॒श्चर॑ति ब्रह्म॒चर्यम्। मे॒धा॒वी दि॒क्षु मन॑सा तप॒स्वी॥५७॥

%3.7.6.4
अ॒न्तर्दू॒तश्च॑रति॒ मानु॑षीषु। चतु॑ शिखण्डा युव॒तिः सु॒पेशा। घृ॒तप्र॑तीका॒ भुव॑नस्य॒ मध्ये। म॒र्मृ॒ज्यमा॑ना मह॒ते सौभ॑गाय। मह्य॑न्धुक्ष्व॒ यज॑मानाय॒ कामान्॑। भूमि॑र्भू॒त्वा म॑हि॒मानं॑ पुपोष। ततो॑ दे॒वी व॑र्धयते॒ पयासि। य॒ज्ञिया॑ य॒ज्ञं वि च॒ यन्ति॒ शं च॑। ओष॑धी॒राप॑ इ॒ह शक्व॑रीश्च। यो मा॑ हृ॒दा मन॑सा॒ यश्च॑ वा॒चा॥५८॥

%3.7.6.5
यो ब्रह्म॑णा॒ कर्म॑णा॒ द्वेष्टि॑ देवाः। यः श्रु॒तेन॒ हृद॑येनेष्ण॒ता च॑। तस्येन्द्र॒ वज्रे॑ण॒ शिर॑श्छिनद्मि। ऊर्णा॑मृदु॒ प्रथ॑मान स्यो॒नम्। दे॒वेभ्यो॒ जुष्ट॒ सद॑नाय ब॒र्॒हिः। सु॒व॒र्गे लो॒के यज॑मान॒ हि धे॒हि। मां नाक॑स्य पृ॒ष्ठे प॑र॒मे व्यो॑मन्। चतु॑ शिखण्डा युव॒तिः सु॒पेशा। घृ॒तप्र॑तीका व॒युना॑नि वस्ते। साऽऽस्ती॒र्यमा॑णा मह॒ते सौभ॑गाय ॥५९॥

%3.7.6.6
सा मे॑ धुक्ष्व॒ यज॑मानाय॒ कामान्॑। शि॒वा च॑ मे श॒ग्मा चै॑धि। स्यो॒ना च॑ मे सु॒षदा॑ चैधि। ऊर्ज॑स्वती च मे॒ पय॑स्वती चैधि। इष॒मूर्जं॑ मे पिन्वस्व। ब्रह्म॒ तेजो॑ मे पिन्वस्व। क्ष॒त्रमोजो॑ मे पिन्वस्व। विशं॒ पुष्टिं॑ मे पिन्वस्व। आयु॑र॒न्नाद्य॑म्मे पिन्वस्व। प्र॒जां प॒शून्मे॑ पिन्वस्व॥६०॥

%3.7.6.7
अ॒स्मिन् य॒ज्ञ उप॒ भूय॒ इन्नु मे। अवि॑क्षोभाय परि॒धीन्द॑धामि। ध॒र्ता ध॒रुणो॒ धरी॑यान्। अ॒ग्निर्द्वेषासि॒ निरि॒तो नु॑दातै। विच्छि॑नद्मि॒ विधृ॑तीभ्या स॒पत्नान्॑। जा॒तान्भ्रातृ॑व्या॒न्॒ ये च॑ जनि॒ष्यमा॑णाः। वि॒शो य॒न्त्राभ्यां॒ विध॑माम्येनान्। अ॒ह स्वाना॑मुत्त॒मो॑ऽसानि देवाः। वि॒शो य॒न्त्रे नु॒दमा॑ने॒ अरा॑तिम्। विश्वं॑ पा॒प्मान॒मम॑तिन्दुर्मरा॒युम्॥६१॥

%3.7.6.8
सीद॑न्ती दे॒वी सु॑कृ॒तस्य॑ लो॒के। धृती स्थो॒ विधृ॑ती॒ स्वधृ॑ती। प्रा॒णान्मयि॑ धारयतम्। प्र॒जाम्मयि॑ धारयतम्। प॒शून्मयि॑ धारयतम्। अ॒यं प्र॑स्त॒र उ॒भय॑स्य ध॒र्ता। ध॒र्ता प्र॑या॒जाना॑मु॒तानू॑या॒जानाम्। स दा॑धार स॒मिधो॑ वि॒श्वरू॑पाः। तस्मि॒न्त्स्रुचो॒ अध्या सा॑दयामि। आ रो॑ह प॒थो जु॑हु देव॒यानान्॑॥६२॥

%3.7.6.9
यत्रर्‌ष॑यः प्रथम॒जा ये पु॑रा॒णाः। हिर॑ण्यपक्षाऽजि॒रा सम्भृ॑ताङ्गा। वहा॑सि मा सु॒कृतां॒ यत्र॑ लो॒काः। अवा॒हं बा॑ध उप॒भृता॑ स॒पत्नान्॑। जा॒तान्भ्रातृ॑व्या॒न्॒ ये च॑ जनि॒ष्यमा॑णाः। दोहै॑ य॒ज्ञ सु॒दुघा॑मिव धे॒नुम्। अ॒हमुत्त॑रो भूयासम्। अध॑रे॒ मत्स॒पत्ना। यो मा॑ वा॒चा मन॑सा दुर्मरा॒युः। हृ॒दाऽरा॑ती॒याद॑भि॒दास॑दग्ने॥६३॥

%3.7.6.10
इ॒दम॑स्य चि॒त्तमध॑रन्ध्रु॒वाया। अ॒हमुत्त॑रो भूयासम्। अध॑रे॒ मत्स॒पत्ना। ऋ॒ष॒भो॑ऽसि शाक्व॒रः। घृ॒ताची॑ना सू॒नुः। प्रि॒येण॒ नाम्ना प्रि॒ये सद॑सि सीद। स्यो॒नो मे॑ सीद सु॒षद॑ पृथि॒व्याम्। प्रथ॑यि प्र॒जया॑ प॒शुभि॑ सुव॒र्गे लो॒के। दि॒वि सी॑द पृथि॒व्याम॒न्तरि॑क्षे। अ॒हमुत्त॑रो भूयासम्॥६४॥

%3.7.6.11
अध॑रे॒ मत्स॒पत्ना। इ॒य स्था॒ली घृ॒तस्य॑ पू॒र्णा। अच्छि॑न्नपयाः श॒तधा॑र॒ उत्स॑। मा॒रु॒तेन॒ शर्म॑णा॒ दैव्ये॑न। य॒ज्ञो॑ऽसि स॒र्वत॑ श्रि॒तः। स॒र्वतो॒ मां भू॒तं भ॑वि॒ष्यच्छ्र॑यताम्। श॒तम्मे॑ सन्त्वा॒शिष॑। स॒हस्र॑म्मे सन्तु सू॒नृता। इरा॑वतीः पशु॒मती। प्र॒जाप॑तिरसि स॒र्वत॑ श्रि॒तः॥६५॥

%3.7.6.12
स॒र्वतो॒ मां भू॒तं भ॑वि॒ष्यच्छ्र॑यताम्। श॒तं मे॑ सन्त्वा॒शिष॑। स॒हस्रं॑ मे सन्तु सू॒नृता। इरा॑वतीः पशु॒मती। इ॒दमि॑न्द्रि॒यम॒मृतं॑ वी॒र्यम्। अ॒नेनेन्द्रा॑य प॒शवो॑ऽचिकित्सन्। तेन॑ देवा अव॒तोप॒ माम्। इ॒हेष॒मूर्जं॒ यश॒ सह॒ ओज॑ सनेयम्। शृ॒तं मयि॑ श्रयताम्। यत्पृ॑थि॒वीमच॑र॒त्तत्प्रवि॑ष्टम्॥६६॥

%3.7.6.13
येनासि॑ञ्च॒द्बल॒मिन्द्रे प्र॒जाप॑तिः। इ॒दन्तच्छु॒क्रं मधु॑ वा॒जिनी॑वत्। येनो॒परि॑ष्टा॒दधि॑नोन्महे॒न्द्रम्। दधि॒ मान्धि॑नोतु। अ॒यं वे॒दः पृ॑थि॒वीमन्व॑विन्दत्। गुहा॑ स॒तीङ्गह॑ने॒ गह्व॑रेषु। स वि॑न्दतु॒ यज॑मानाय लो॒कम्। अच्छि॑द्रं य॒ज्ञं भूरि॑कर्मा करोतु। अ॒यं य॒ज्ञः सम॑सदद्ध॒विष्मान्॑। ऋ॒चा साम्ना॒ यजु॑षा दे॒वता॑भिः॥६७॥

%3.7.6.14
तेन॑ लो॒कान्त्सूर्य॑वतो जयेम। इन्द्र॑स्य स॒ख्यम॑मृत॒त्वम॑श्याम्। यो न॒ कनी॑य इ॒ह का॒मया॑तै। अ॒स्मिन् य॒ज्ञे यज॑मानाय॒ मह्यम्। अप॒ तमि॑न्द्रा॒ग्नी भुव॑नान्नुदेताम्। अ॒हं प्र॒जां वी॒रव॑तीं विदेय। अग्ने॑ वाजजित्। वाज॑न्त्वा सरि॒ष्यन्तम्। वाजं॑ जे॒ष्यन्तम्। वा॒जिनं॑ वाज॒जितम्॥६८॥

%3.7.6.15
वा॒ज॒जि॒त्यायै॒ सं मार्ज्मि। अ॒ग्निम॑न्ना॒दम॒न्नाद्या॑य। उप॑हूतो॒ द्यौः पि॒ता। उप॒ मान्द्यौः पि॒ता ह्व॑यताम्। अ॒ग्निराग्नीध्रात्। आयु॑षे॒ वर्च॑से। जी॒वात्वै पुण्या॑य। उप॑हूता पृथि॒वी मा॒ता। उप॒ मां मा॒ता पृ॑थि॒वी ह्व॑यताम्। अ॒ग्निराग्नीध्रात्॥६९॥

%3.7.6.16
आयु॑षे॒ वर्च॑से। जी॒वात्वै पुण्या॑य। मनो॒ ज्योति॑र्जुषता॒माज्यम्। विच्छि॑न्नं य॒ज्ञ समि॒मन्द॑धातु। बृह॒स्पति॑स्तनुतामि॒मन्न॑। विश्वे॑ दे॒वा इ॒ह मा॑दयन्ताम्। यन्ते॑ अग्न आवृ॒श्चामि॑। अ॒हं वा क्षिपि॒तश्चर\sn{}। प्र॒जां च॒ तस्य॒ मूलं॑ च। नी॒चैर्दे॑वा॒ नि वृ॑श्चत॥७०॥

%3.7.6.17
अग्ने॒ यो नो॑ऽभि॒दास॑ति। स॒मा॒नो यश्च॒ निष्ट्य॑। इ॒ध्मस्ये॑व प्र॒क्षाय॑तः। मा तस्योच्छे॑षि॒ किञ्च॒न। यो मान्द्वेष्टि॑ जातवेदः। यञ्चा॒हन्द्वेष्मि॒ यश्च॒ माम्। सर्वा॒ स्तान॑ग्ने॒ सन्द॑ह। या श्चा॒हन्द्वेष्मि॒ ये च॒ माम्। अग्ने॑ वाजजित्। वाज॑न्त्वा ससृ॒वासम्॥७१॥

%3.7.6.18
वाजं॑ जिगि॒वासम्। वा॒जिनं॑ वाज॒जितम्। वा॒ज॒जि॒त्यायै॒ सम्मार्ज्मि। अ॒ग्निम॑न्ना॒दम॒न्नाद्या॑य। वेदि॑र्ब॒र्॒हिः शृ॒त ह॒विः। इ॒ध्मः प॑रि॒धय॒ स्रुच॑। आज्यं॑ य॒ज्ञ ऋचो॒ यजु॑। या॒ज्याश्च वषट्का॒राः। सम्मे॒ सन्न॑तयो नमन्ताम्। इ॒ध्म॒स॒न्नह॑ने हु॒ते॥७२॥

%3.7.6.19
दि॒वः खीलोऽव॑ततः। पृ॒थि॒व्या अध्युत्थि॑तः। तेना॑ स॒हस्र॑काण्डेन। द्वि॒षन्त शोचयामसि। द्वि॒षन्मे॑ ब॒हु शो॑चतु। ओष॑धे॒ मो अ॒ह शु॑चम्। यज्ञ॒ नम॑स्ते यज्ञ। नमो॒ नम॑श्च ते यज्ञ। शि॒वेन॑ मे॒ सन्ति॑ष्ठस्व। स्यो॒नेन॑ मे॒ सन्ति॑ष्ठस्व॥७३॥

%3.7.6.20
सु॒भू॒तेन॑ मे॒ सन्ति॑ष्ठस्व। ब्र॒ह्म॒व॒र्च॒सेन॑ मे॒ सन्ति॑ष्ठस्व। य॒ज्ञस्यर्द्धि॒मनु॒ सन्ति॑ष्ठस्व। उप॑ ते यज्ञ॒ नम॑। उप॑ ते॒ नम॑। उप॑ ते॒ नम॑। त्रिष्फ॒लीक्रि॒यमा॑णानाम्। यो न्य॒ङ्गो अ॑व॒शिष्य॑ते। रक्ष॑सां भाग॒धेयम्। आप॒स्तत्प्र व॑हतादि॒तः॥७४॥

%3.7.6.21
उ॒लूख॑ले॒ मुस॑ले॒ यच्च॒ शूर्पे। आ॒शि॒श्लेष॑ दृ॒षदि॒ यत्क॒पाले। अ॒व॒प्रुषो॑ वि॒प्रुष॒ संय॑जामि। विश्वे॑ दे॒वा ह॒विरि॒दं जु॑षन्ताम्। य॒ज्ञे या वि॒प्रुष॒ सन्ति॑ ब॒ह्वीः। अ॒ग्नौ ताः सर्वा॒ स्वि॑ष्टा॒ सुहु॑ता जुहोमि। उ॒द्यन्न॒द्यमि॑त्र महः। स॒पत्नान्मे अनीनशः। दिवै॑नान् वि॒द्युता॑ जहि। नि॒म्रोच॒न्नध॑रान्कृधि॥७५॥

%3.7.6.22
उ॒द्यन्न॒द्य वि नो॑ भज। पि॒ता पु॒त्रेभ्यो॒ यथा। दी॒र्घा॒यु॒त्वस्य॑ हेशिषे। तस्य॑ नो देहि सूर्य। उ॒द्यन्न॒द्य मि॑त्रमहः। आ॒रोह॒न्नुत्त॑रा॒न्दिवम्। हृ॒द्रो॒गम्मम॑ सूर्य। ह॒रि॒माणं॑ च नाशय। शुके॑षु मे हरि॒माणम्। रो॒प॒णाका॑सु दध्मसि ॥७६॥

%3.7.6.23
अथो॑ हारिद्र॒वेषु॑ मे। ह॒रि॒माणं॒ नि द॑ध्मसि। उद॑गाद॒यमा॑दि॒त्यः। विश्वे॑न॒ सह॑सा स॒ह। द्वि॒षन्तं॒ मम॑ र॒न्धय\sn{}। मो अ॒हन्द्वि॑ष॒तो र॑धम्। यो न॒ शपा॒दश॑पतः। यश्च॑ न॒ शप॑त॒ शपात्। उ॒षाश्च॒ तस्मै॑ नि॒म्रुक्च॑। सर्वं॑ पा॒प समू॑हताम्॥७७॥

%3.7.6.24
यो न॑ स॒पत्नो॒ यो रण॑। मर्तो॑ऽभि॒दास॑ति देवाः। इ॒ध्मस्ये॑व प्र॒क्षाय॑तः। मा तस्योच्छे॑षि॒ किञ्च॒न। अव॑सृष्ट॒ परा॑पत। श॒रो ब्रह्म॑सशितः। गच्छा॒ऽमित्रा॒न्प्र वि॑श। मैषा॒ङ्कञ्च॒नोच्छि॑षः॥७८॥\anuvakamend[पति॑ प्र॒जाप॑तये तप॒स्वी वा॒चा सौभ॑गाय प॒शून्मे॑ पिन्वस्व दुर्मरा॒युं दे॑व॒याना॑नग्ने॒ऽन्तरि॑क्षे॒ऽहमुत्त॑रो भूयासं प्र॒जाप॑तिरसि स॒र्वत॑ श्रि॒तः प्रवि॑ष्टन्दे॒वता॑भिर्वाज॒जितं॑ पृथि॒वी ह्व॑यताम॒ग्निराग्नीध्राद्वृश्चत ससृ॒वास हु॒ते स्यो॒नेन॑ मे॒ सन्ति॑ष्ठस्वे॒तः कृ॑धि दध्मस्यूहताम॒ष्टौ च॑]

%3.7.7.1
सक्षे॒दं प॑श्य। विध॑र्तरि॒दं प॑श्य। नाके॒दं प॑श्य। र॒मति॒ पनि॑ष्ठा। ऋ॒तं वर्‌षि॑ष्ठम्। अ॒मृता॒यान्या॒हुः। सूर्यो॒ वरि॑ष्ठो अ॒क्षभि॒र्विभा॑ति। अनु॒ द्यावा॑पृथि॒वी दे॒वपु॑त्रे। दी॒क्षाऽसि॒ तप॑सो॒ योनि॑। तपो॑ऽसि॒ ब्रह्म॑णो॒ योनि॑ ॥७९॥

%3.7.7.2
ब्रह्मा॑सि क्ष॒त्रस्य॒ योनि॑। क्ष॒त्रम॑स्यृ॒तस्य॒ योनि॑। ऋ॒तम॑सि॒ भूरा र॑भे। श्र॒द्धां मन॑सा। दी॒क्षान्तप॑सा। विश्व॑स्य॒ भुव॑न॒स्याधि॑पत्नीम्। सर्वे॒ कामा॒ यज॑मानस्य सन्तु। वातं॑ प्रा॒णं मन॑सा॒ऽन्वा र॑भामहे। प्र॒जाप॑ति॒य्योँ भुव॑नस्य गो॒पाः। स नो॑ मृ॒त्योस्त्रा॑यतां॒ पात्वह॑सः॥८०॥

%3.7.7.3
ज्योग्जी॒वा ज॒राम॑शीमहि। इन्द्र॑ शाक्वर गाय॒त्रीं प्र प॑द्ये। तान्ते॑ युनज्मि। इन्द्र॑ शाक्वर त्रि॒ष्टुभं॒ प्र प॑द्ये। तान्ते॑ युनज्मि। इन्द्र॑ शाक्वर॒ जग॑तीं॒ प्र प॑द्ये। तान्ते॑ युनज्मि। इन्द्र॑ शाक्वरानु॒ष्टुभं॒ प्र प॑द्ये। तान्ते॑ युनज्मि। इन्द्र॑ शाक्वर प॒ङ्क्तिं प्रप॑द्ये॥८१॥

%3.7.7.4
तान्ते॑ युनज्मि। आऽहन्दी॒क्षाम॑रुहमृ॒तस्य॒ पत्नीम्। गा॒य॒त्रेण॒ छन्द॑सा॒ ब्रह्म॑णा च। ऋ॒त स॒त्ये॑ऽधायि। स॒त्यमृ॒ते॑ऽधायि। ऋ॒तं च॑ मे स॒त्यञ्चा॑भूताम्। ज्योति॑रभूव॒ सुव॑रगमम्। सु॒व॒र्गं लो॒कं नाक॑स्य पृ॒ष्ठम्। ब्र॒ध्नस्य॑ वि॒ष्टप॑मगमम्। पृ॒थि॒वी दी॒क्षा॥८२॥

%3.7.7.5
तया॒ऽग्निर्दी॒क्षया॑ दीक्षि॒तः। यया॒ऽग्निर्दी॒क्षया॑ दीक्षि॒तः। तया त्वा दी॒क्षया॑ दीक्षयामि। अ॒न्तरि॑क्षन्दी॒क्षा। तया॑ वा॒युर्दी॒क्षया॑ दीक्षि॒तः। यया॑ वा॒युर्दी॒क्षया॑ दीक्षि॒तः। तया त्वा दी॒क्षया॑ दीक्षयामि। द्यौर्दी॒क्षा। तया॑ऽऽदि॒त्यो दी॒क्षया॑ दीक्षि॒तः। यया॑ऽऽदि॒त्यो दी॒क्षया॑ दीक्षि॒तः॥८३॥

%3.7.7.6
तया त्वा दी॒क्षया॑ दीक्षयामि। दिशो॑ दी॒क्षा। तया॑ च॒न्द्रमा॑ दी॒क्षया॑ दीक्षि॒तः। यया॑ च॒न्द्रमा॑ दी॒क्षया॑ दीक्षि॒तः। तया त्वा दी॒क्षया॑ दीक्षयामि। आपो॑ दी॒क्षा। तया॒ वरु॑णो॒ राजा॑ दी॒क्षया॑ दीक्षि॒तः। यया॒ वरु॑णो॒ राजा॑ दी॒क्षया॑ दीक्षि॒तः। तया त्वा दी॒क्षया॑ दीक्षयामि। ओष॑धयो दी॒क्षा॥८४॥

%3.7.7.7
तया॒ सोमो॒ राजा॑ दी॒क्षया॑ दीक्षि॒तः। यया॒ सोमो॒ राजा॑ दी॒क्षया॑ दीक्षि॒तः। तया त्वा दी॒क्षया॑ दीक्षयामि। वाग्दी॒क्षा। तया प्रा॒णो दी॒क्षया॑ दीक्षि॒तः। यया प्रा॒णो दी॒क्षया॑ दीक्षि॒तः। तया त्वा दी॒क्षया॑ दीक्षयामि। पृ॒थि॒वी त्वा॒ दीक्ष॑माण॒मनु॑ दीक्षताम्। अ॒न्तरि॑क्षन्त्वा॒ दीक्ष॑माण॒मनु॑ दीक्षताम्। द्यौस्त्वा॒ दीक्ष॑माण॒मनु॑ दीक्षताम्॥८५॥

%3.7.7.8
दिश॑स्त्वा॒ दीक्ष॑माण॒मनु॑ दीक्षन्ताम्। आप॑स्त्वा॒ दीक्ष॑माण॒मनु॑ दीक्षन्ताम्। ओष॑धयस्त्वा॒ दीक्ष॑माण॒मनु॑ दीक्षन्ताम्। वाक्त्वा॒ दीक्ष॑माण॒मनु॑ दीक्षताम्। ऋच॑स्त्वा॒ दीक्ष॑माण॒मनु॑ दीक्षन्ताम्। सामा॑नि त्वा॒ दीक्ष॑माण॒मनु॑ दीक्षन्ताम्। यजूषि त्वा॒ दीक्ष॑माण॒मनु॑ दीक्षन्ताम्। अह॑श्च॒ रात्रि॑श्च। कृ॒षिश्च॒ वृष्टि॑श्च। त्विषि॒श्चाप॑चितिश्च॥८६॥

%3.7.7.9
अप॒श्चौष॑धयश्च। ऊर्क्च॑ सू॒नृता॑ च। तास्त्वा॒ दीक्ष॑माण॒मनु॑ दीक्षन्ताम्। स्वे दक्षे॒ दक्ष॑पिते॒ह सी॑द। दे॒वाना सु॒म्नो म॑ह॒ते रणा॑य। स्वा॒स॒स्थस्त॒नुवा॒ संवि॑शस्व। पि॒तेवै॑धि सू॒नव॒ आ सु॒शेव॑। शि॒वो मा॑ शि॒वमा वि॑श। स॒त्यम्म॑ आ॒त्मा। श्र॒द्धा मेऽक्षि॑तिः॥८७॥

%3.7.7.10
तपो॑ मे प्रति॒ष्ठा। स॒वि॒तृप्र॑सूता मा॒ दिशो॑ दीक्षयन्तु। स॒त्यम॑स्मि। अ॒हन्त्वद॑स्मि॒ मद॑सि॒ त्वमे॒तत्। ममा॑सि॒ योनि॒स्तव॒ योनि॑रस्मि। ममै॒व सन्वह॑ ह॒व्यान्य॑ग्ने। पु॒त्रः पि॒त्रे लो॑क॒कृज्जा॑तवेदः। आ॒जुह्वा॑नः सु॒प्रती॑कः पु॒रस्तात्। अग्ने॒ स्वाय्योँनि॒मा सी॑द सा॒ध्या। अ॒स्मिन्त्स॒धस्थे॒ अध्युत्त॑रस्मिन्॥८८॥

%3.7.7.11
विश्वे॑ देवा॒ यज॑मानश्च सीदत। एक॑मि॒षे विष्णु॒स्त्वाऽन्वे॑तु। द्वे ऊ॒र्जे विष्णु॒स्त्वाऽन्वे॑तु। त्रीणि॑ व्र॒ताय॒ विष्णु॒स्त्वाऽन्वे॑तु। च॒त्वारि॒ मायो॑भवाय॒ विष्णु॒स्त्वाऽन्वे॑तु। पञ्च॑ प॒शुभ्यो॒ विष्णु॒स्त्वाऽन्वे॑तु। षड्रा॒यस्पोषा॑य॒ विष्णु॒स्त्वाऽन्वे॑तु। स॒प्त स॒प्तभ्यो॒ होत्राभ्यो॒ विष्णु॒स्त्वाऽन्वे॑तु। सखा॑यः स॒प्तप॑दा अभूम। स॒ख्यन्ते॑ गमेयम् ॥८९॥

%3.7.7.12
स॒ख्यात्ते॒ मा यो॑षम्। स॒ख्यान्मे॒ मा योष्ठाः। साऽसि॑ सुब्रह्मण्ये। तस्यास्ते पृथि॒वी पाद॑। साऽसि॑ सुब्रह्मण्ये। तस्यास्ते॒ऽन्तरि॑क्षं॒ पाद॑। साऽसि॑ सुब्रह्मण्ये। तस्यास्ते॒ द्यौः पाद॑। साऽसि॑ सुब्रह्मण्ये। तस्यास्ते॒ दिश॒ पाद॑॥९०॥

%3.7.7.13
प॒रोर॑जास्ते पञ्च॒मः पाद॑। सा न॒ इष॒मूर्ज॑न्धुक्ष्व। तेज॑ इन्द्रि॒यम्। ब्र॒ह्म॒व॒र्च॒सम॒न्नाद्यम्। वि मि॑मे त्वा॒ पय॑स्वतीम्। दे॒वानान्धे॒नु सु॒दुघा॒मन॑पस्फुरन्तीम्। इन्द्र॒ सोमं॑ पिबतु। क्षेमो॑ अस्तु नः। इ॒मान्न॑राः कृणुत॒ वेदि॒मेत्य॑। वसु॑मती रु॒द्रव॑तीमादि॒त्यव॑तीम्॥९१॥

%3.7.7.14
वर्ष्म॑न्दि॒वः। नाभा॑ पृथि॒व्याः। यथा॒ऽयं यज॑मानो॒ न रिष्येत्। दे॒वस्य॑ सवि॒तुः स॒वे। चतु॑ शिखण्डा युव॒तिः सु॒पेशा। घृ॒तप्र॑तीका॒ भुव॑नस्य॒ मध्ये। तस्या सुप॒र्णावधि॒ यौ निवि॑ष्टौ। तयोर्दे॒वाना॒मधि॑ भाग॒धेयम्। अ॒प जन्य॑म्भ॒यन्नु॑द। अप॑ च॒क्राणि॑ वर्तय। गृ॒ह सोम॑स्य गच्छतम्। न वा उ॑ वे॒तन्म्रि॑यसे॒ न रि॑ष्यसि। दे॒वा इदे॑षि प॒थिभि॑ सु॒गेभि॑। यत्र॒ यन्ति॑ सु॒कृतो॒ नापि॑ दु॒ष्कृत॑। तत्र॑ त्वा दे॒वः स॑वि॒ता द॑धातु॥९२॥\anuvakamend[ब्रह्म॑णो॒ योनि॒रह॑सः प॒ङ्क्तिं प्रप॑द्ये दी॒क्षा यया॑ऽऽदि॒त्यो दी॒क्षया॑ दीक्षि॒तस्तया त्वा दी॒क्षया॑ दीक्षया॒म्योष॑धयो दी॒क्षा द्यौस्त्वा॒ दीक्ष॑माण॒मनु॑ दीक्षता॒मप॑चिति॒श्चाक्षि॑ति॒रुत्त॑रस्मिन्गमेयं॒ दिश॒ पाद॑ आदि॒त्यव॑तीं वर्तय॒ पञ्च॑ च]

%3.7.8.1
यद॒स्य पा॒रे रज॑सः। शु॒क्रञ्ज्योति॒रजा॑यत। तन्न॑ पर्‌ष॒दति॒ द्विष॑। अग्ने॑ वैश्वानर॒ स्वाहा। यस्माद्भी॒षाऽवा॑शिष्ठाः। ततो॑ नो॒ अभ॑यङ्कृधि। प्र॒जाभ्य॒ सर्वाभ्यो मृड। नमो॑ रु॒द्राय॑ मी॒ढुषे। यस्माद्भी॒षा न्यष॑दः। ततो॑ नो॒ अभ॑यङ्कृधि॥९३॥

%3.7.8.2
प्र॒जाभ्य॒ सर्वाभ्यो मृड। नमो॑ रु॒द्राय॑ मी॒ढुषे। उदु॑स्र तिष्ठ॒ प्रति॑तिष्ठ॒ मारि॑षः। मेमं य॒ज्ञं यज॑मानं च रीरिषः। सु॒व॒र्गे लो॒के यज॑मान॒ हि धे॒हि। शन्न॑ एधि द्वि॒पदे॒ शञ्चतु॑ष्पदे। यस्माद्भी॒षाऽवे॑पिष्ठाः प॒लायि॑ष्ठाः स॒मज्ञास्थाः। ततो॑ नो॒ अभ॑यङ्कृधि। प्र॒जाभ्य॒ सर्वाभ्यो मृड। नमो॑ रु॒द्राय॑ मी॒ढुषे॥९४॥

%3.7.8.3
य इ॒दमक॑। तस्मै॒ नम॑। तस्मै॒ स्वाहा। न वा उ॑वे॒तन्म्रि॑यसे। आशा॑नान्त्वा॒ विश्वा॒ आशा। य॒ज्ञस्य॒ हि स्थ ऋ॒त्वियौ। इन्द्राग्नी॒ चेत॑नस्य च। हु॒ता॒हु॒तस्य॑ तृप्यतम्। अहु॑तस्य हु॒तस्य॑ च। हु॒तस्य॒ चाहु॑तस्य च। अहु॑तस्य हु॒तस्य॑ च। इन्द्राग्नी अ॒स्य सोम॑स्य। वी॒तं पि॑बतं जु॒षेथाम्। मा यज॑मान॒न्तमो॑ विदत्। मर्त्विजो॒ मो इ॒माः प्र॒जाः। मा यः सोम॑मि॒मं पिबात्। ससृ॑ष्टमु॒भयं॑ कृ॒तम्॥९५॥\anuvakamend[कृ॒धि॒ मी॒ढुषेऽहु॑तस्य च स॒प्त च॑]

%3.7.9.1
अ॒ना॒गस॑स्त्वा व॒यम्। इन्द्रे॑ण॒ प्रेषि॑ता॒ उप॑। वा॒युष्टे॑ अस्त्वश॒भूः। मि॒त्रस्ते॑ अस्त्वश॒भूः। वरु॑णस्ते अस्त्वश॒भूः। अपांक्षया॒ ऋत॑स्य गर्भाः। भुव॑नस्य गोपा॒ श्येना॑ अतिथयः। पर्व॑तानाङ्ककुभः प्र॒युतो॑ नपातारः। व॒ग्नुनेन्द्र ह्वयत। घोषे॒णामी॑वा श्चातयत॥९६॥

%3.7.9.2
यु॒क्ताः स्थ॒ वह॑त। दे॒वा ग्रावा॑ण॒ इन्दु॒रिन्द्र॒ इत्य॑वादिषुः। एन्द्र॑मचुच्यवुः पर॒मस्या परा॒वत॑। आऽस्मात्स॒धस्थात्। ओरोर॒न्तरि॑क्षात्। आ सु॑भू॒तम॑सुषवुः। ब्र॒ह्म॒व॒र्च॒सम्म॒ आसु॑षवुः। स॒म॒रे रक्षास्यवधिषुः। अप॑हतं ब्रह्म॒ज्यस्य॑। वाक्च॑ त्वा॒ मन॑श्च श्रीणीताम्॥९७॥

%3.7.9.3
प्रा॒णश्च॑ त्वाऽपा॒नश्च॑ श्रीणीताम्। चक्षु॑श्च त्वा॒ श्रोत्रं॑ च श्रीणीताम्। दक्ष॑श्चत्वा॒ बलं॑ च श्रीणीताम्। ओज॑श्च त्वा॒ सह॑श्च श्रीणीताम्। आयु॑श्च त्वाऽज॒रा च॑ श्रीणीताम्। आ॒त्मा च॑ त्वा त॒नूश्च॑ श्रीणीताम्। शृ॒तो॑ऽसि शृ॒तं कृ॑तः। शृ॒ताय॑ त्वा शृ॒तेभ्य॑स्त्वा। यमिन्द्र॑मा॒हुर्वरु॑णं॒ यमा॒हुः। यम्मि॒त्रमा॒हुर्यमु॑ स॒त्यमा॒हुः॥९८॥

%3.7.9.4
यो दे॒वानान्दे॒वत॑मस्तपो॒जाः। तस्मै त्वा॒ तेभ्य॑स्त्वा। मयि॒ त्यदि॑न्द्रि॒यम्मह॒त्। मयि॒ दक्षो॒ मयि॒ क्रतु॑। मयि॑ धायि सु॒वीर्यम्। त्रिशु॑ग्घ॒र्मो वि भा॑तु मे। आकूत्या॒ मन॑सा स॒ह। वि॒राजा॒ ज्योति॑षा स॒ह। य॒ज्ञेन॒ पय॑सा स॒ह। तस्य॒ दोह॑मशीमहि॥९९॥

%3.7.9.5
तस्य॑ सु॒म्नम॑शीमहि। तस्य॑ भ॒क्षम॑शीमहि। वाग्जु॑षा॒णा सोम॑स्य तृप्यतु। मि॒त्रो जना॒न्प्र स मि॑त्र। यस्मा॒न्न जा॒तः परो॑ अ॒न्यो अस्ति॑। य आ॑वि॒वेश॒ भुव॑नानि॒ विश्वा। प्र॒जाप॑तिः प्र॒जया॑ संविदा॒नः। त्रीणि॒ ज्योतीषि सचते॒ स षो॑ड॒शी। ए॒ष ब्र॒ह्मा य ऋ॒त्विय॑। इन्द्रो॒ नाम॑ श्रु॒तो ग॒णे॥१००॥

%3.7.9.6
प्र ते॑ म॒हे वि॒दथे॑ शसिष॒ हरी। य ऋ॒त्विय॒ प्र ते॑ वन्वे। व॒नुषो॑ हर्य॒तम्मदम्। इन्द्रो॒ नाम॑ घृ॒तन्नयः। हरि॑भि॒श्चारु॒ सेच॑ते। श्रु॒तो ग॒ण आ त्वा॑ विशन्तु। हरि॑वर्पस॒ङ्गिर॑। इन्द्राधि॑प॒तेऽधि॑पति॒स्त्वन्दे॒वाना॑मसि। अधि॑पति॒म्माम्। आयु॑ष्मन्तं॒ वर्च॑स्वन्तम्मनु॒ष्ये॑षु कुरु॥१०१॥

%3.7.9.7
इन्द्र॑श्च स॒म्राड्वरु॑णश्च॒ राजा। तौ ते॑ भ॒क्षं च॑क्रतु॒रग्र॑ ए॒तम्। तयो॒रनु॑ भ॒क्षं भ॑क्षयामि। वाग्जु॑षा॒णा सोम॑स्य तृप्यतु। प्र॒जाप॑तिर्वि॒श्वक॑र्मा। तस्य॒ मनो॑ दे॒वं य॒ज्ञेन॑ राध्यासम्। अ॒र्थे॒गा अ॒स्य ज॑हितः। अ॒व॒सान॑पतेऽव॒सान॑म्मे विन्द। नमो॑ रु॒द्राय॑ वास्तो॒ष्पत॑ये। आय॑ने वि॒द्रव॑णे॥१०२॥

%3.7.9.8
उ॒द्याने॒ यत्प॒राय॑णे। आ॒वर्त॑ने वि॒वर्त॑ने। यो गो॑पा॒यति॒ त हु॑वे। यान्य॑पा॒मित्या॒न्यप्र॑तीत्ता॒न्यस्मि॑। य॒मस्य॑ ब॒लिना॒ चरा॑मि। इ॒हैव सन्त॒ प्रति॒ तद्या॑तयामः। जी॒वा जी॒वेभ्यो॒ नि ह॑राम एनत्। अ॒नृ॒णा अ॒स्मिन्न॑नृ॒णाः पर॑स्मिन्। तृ॒तीये॑ लो॒के अ॑नृ॒णाः स्या॑म। ये दे॑व॒याना॑ उ॒त पि॑तृ॒याणा॥१०३॥

%3.7.9.9
सर्वान्प॒थो अ॑नृ॒णा आक्षी॑येम। इ॒दमू॒नु श्रेयो॑ऽव॒सान॒मा ग॑न्म। शि॒वे नो॒ द्यावा॑पृथि॒वी उ॒भे इ॒मे। गोम॒द्धन॑व॒दश्व॑व॒दूर्ज॑स्वत्। सु॒वीरा॑ वी॒रैरनु॒ सञ्च॑रेम। अ॒र्कः प॒वि॒त्र॒ रज॑सो वि॒मान॑। पु॒नाति॑ दे॒वाना॒म्भुव॑नानि॒ विश्वा। द्यावा॑पृथि॒वी पय॑सा संविदा॒ने। घृ॒तन्दु॑हाते अ॒मृतं॒ प्रपी॑ने। प॒वित्र॑म॒र्को रज॑सो वि॒मान॑। पु॒नाति॑ दे॒वाना॒म्भुव॑नानि॒ विश्वा। सुव॒र्ज्योति॒र्यशो॑ म॒हत्। अ॒शी॒महि॑ गा॒धमु॒त प्र॑ति॒ष्ठाम्॥१०४॥\anuvakamend[चा॒त॒य॒त॒ श्री॒णी॒ता॒ स॒त्यमा॒हुर॑शीमहि ग॒णे कु॑रु वि॒द्रव॑णे पितृ॒याणा॑ अ॒र्को रज॑सो वि॒मान॒स्त्रीणि॑ च]

%3.7.10.1
उद॑स्तांप्सीत्सवि॒ता मि॒त्रो अ॑र्य॒मा। सर्वा॑न॒मित्रा॑नवधीद्यु॒गेन॑। बृ॒हन्त॒म्माम॑करद्वी॒रव॑न्तम्। र॒थ॒न्त॒रे श्र॑यस्व॒ स्वाहा॑ पृथि॒व्याम्। वा॒म॒दे॒व्ये श्र॑यस्व॒ स्वाहा॒ऽन्तरि॑क्षे। बृ॒ह॒ति श्र॑यस्व॒ स्वाहा॑ दि॒वि। बृ॒ह॒ता त्वोप॑स्तभ्नोमि। आ त्वा॑ ददे॒ यश॑से वी॒र्या॑य च। अ॒स्मास्व॑घ्निया यू॒यन्द॑धाथेन्द्रि॒यं पय॑। यस्ते द्र॒प्सो यस्त॑ उद॒र्॒षः ॥१०५॥

%3.7.10.2
दैव्य॑ के॒तुर्विश्व॒म्भुव॑नमावि॒वेश॑। स न॑ पा॒ह्यरि॑ष्ट्यै॒ स्वाहा। अनु॑ मा॒ सर्वो॑ य॒ज्ञो॑ऽयमे॑तु। विश्वे॑ दे॒वा म॒रुत॒ सामा॒र्कः। आ॒प्रिय॒श्छन्दासि नि॒विदो॒ यजूषि। अ॒स्यै पृ॑थि॒व्यै यद्य॒ज्ञियम्। प्र॒जाप॑तेर्वर्त॒निमनु॑ वर्तस्व। अनु॑वी॒रैरनु॑ राध्याम॒ गोभि॑। अन्वश्वै॒रनु॒ सर्वै॑रु पु॒ष्टैः। अनु॑ प्र॒जयाऽन्वि॑न्द्रि॒येण॑॥१०६॥

%3.7.10.3
दे॒वा नो॑ य॒ज्ञमृ॑जु॒धा न॑यन्तु। प्रति॑क्ष॒त्रे प्रति॑तिष्ठामि रा॒ष्ट्रे। प्रत्यश्वे॑षु॒ प्रति॑तिष्ठामि॒ गोषु॑। प्रति॑ प्र॒जायां॒ प्रति॑तिष्ठामि॒ भव्ये। विश्व॑म॒न्याऽभि॑ वावृ॒धे। तद॒न्यस्या॒मधि॑श्रि॒तम्। दि॒वे च॑ वि॒श्वक॑र्मणे। पृ॒थि॒व्यै चा॑कर॒न्नम॑। अस्का॒न्द्यौः पृ॑थि॒वीम्। अस्का॑नृष॒भो युवा॒गाः॥१०७॥

%3.7.10.4
स्क॒न्नेमा विश्वा॒ भुव॑ना। स्क॒न्नो य॒ज्ञः प्र ज॑नयतु। अस्का॒नज॑नि॒ प्राज॑नि। आ स्क॒न्नाज्जा॑यते॒ वृषा। स्क॒न्नात्प्र ज॑निषीमहि। ये दे॒वा येषा॑मि॒दम्भा॑ग॒धेय॑म्ब॒भूव॑। येषां प्रया॒जा उ॒तानू॑या॒जाः। इन्द्र॑ज्येष्ठेभ्यो॒ वरु॑णराजभ्यः। अ॒ग्निहो॑तृभ्यो दे॒वेभ्य॒ स्वाहा। उ॒त त्या नो॒ दिवा॑ म॒तिः॥१०८॥

%3.7.10.5
अदि॑तिरू॒त्या ग॑मत्। सा शन्ता॑ची॒ मय॑स्करत्। अप॒ स्रिध॑। उ॒त त्या दैव्या॑ भि॒षजा। शन्न॑स्करतो अ॒श्विना। यू॒याता॑म॒स्मद्रप॑। अप॒ स्रिध॑। शम॒ग्निर॒ग्निभि॑स्करत्। शन्न॑स्तपतु॒ सूर्य॑। शं वातो॑ वात्वर॒पाः॥१०९॥

%3.7.10.6
अप॒ स्रिध॑। तदित्प॒दन्न विचि॑केत वि॒द्वान्। यन्मृ॒तः पुन॑र॒प्येति॑ जी॒वान्। त्रि॒वृद्यद्भुव॑नस्य रथ॒वृत्। जी॒वो गर्भो॒ न मृ॒तः स जी॑वात्। प्रत्य॑स्मै॒ पिपी॑षते। विश्वा॑नि वि॒दुषे॑ भर। अ॒र॒ङ्ग॒माय॒ जग्म॑वे। अप॑श्चाद्दघ्वने॒ नरे। इन्दु॒रिन्दु॒मवा॑गात्। इन्दो॒रिन्द्रो॑ऽपात्। तस्य॑ त इन्द॒विन्द्र॑पीतस्य॒ मधु॑मतः। उप॑हूत॒स्योप॑हूतो भक्षयामि ॥११०॥\anuvakamend[उ॒द॒र्॒ष इ॑न्द्रि॒येण॒ गा म॒तिर॑र॒पा अ॑गा॒त्रीणि॑ च]

%3.7.11.1
ब्रह्म॑ प्रति॒ष्ठा मन॑सो॒ ब्रह्म॑ वा॒चः। ब्रह्म॑ य॒ज्ञाना ह॒विषा॒माज्य॑स्य। अति॑रिक्त॒ङ्कर्म॑णो॒ यच्च॑ ही॒नम्। य॒ज्ञः पर्वा॑णि प्रति॒रन्ने॑ति क॒ल्पय\sn{}। स्वाहा॑कृ॒ताऽऽहु॑तिरेतु दे॒वान्। आश्रा॑वितम॒त्याश्रा॑वितम्। वष॑ट्कृतम॒त्यनूक्तं च य॒ज्ञे। अति॑रिक्त॒ङ्कर्म॑णो॒ यच्च॑ ही॒नम्। य॒ज्ञः पर्वा॑णि प्रति॒रन्ने॑ति क॒ल्पय\sn{}। स्वाहा॑कृ॒ताऽऽहु॑तिरेतु दे॒वान्॥१११॥

%3.7.11.2
यद्वो॑ देवा अतिपा॒दया॑नि। वा॒चा चि॒त्प्रय॑तन्देव॒हेड॑नम्। अ॒रा॒यो अ॒स्मा अ॒भिदु॑च्छुना॒यते। अ॒न्यत्रा॒स्मन्म॑रुत॒स्तन्निधे॑तन। त॒तम्म॒ आप॒स्तदु॑ तायते॒ पुन॑। स्वादि॑ष्ठा धी॒तिरु॒चथा॑य शस्यते। अ॒य स॑मु॒द्र उ॒त वि॒श्वभे॑षजः। स्वाहा॑कृतस्य॒ समु॑तृप्णुतर्भुवः। उद्व॒यन्तम॑स॒स्परि॑। उदु॒त्यञ्चि॒त्रम्॥११२॥

%3.7.11.3
इ॒मम्मे॑ वरुण॒ तत्त्वा॑ यामि। त्वन्नो॑ अग्ने॒ स त्वन्नो॑ अग्ने। त्वम॑ग्ने अ॒यासि॒ प्रजा॑पते। इ॒मञ्जी॒वेभ्य॑ परि॒धिन्द॑धामि। मैषान्नु॑गा॒दप॑रो॒ अर्ध॑मे॒तम्। श॒तञ्जी॑वन्तु श॒रद॑ पुरू॒चीः। ति॒रो मृ॒त्युन्द॑धतां॒ पर्व॑तेन। इ॒ष्टेभ्य॒ स्वाहा॒ वष॒डनि॑ष्टेभ्य॒ स्वाहा। भे॒ष॒जन्दुरि॑ष्ट्यै॒ स्वाहा॒ निष्कृ॑त्यै॒ स्वाहा। दौरार्ध्यै॒ स्वाहा॒ दैवीभ्यस्त॒नूभ्य॒ स्वाहा॥११३॥

%3.7.11.4
ऋद्ध्यै॒ स्वाहा॒ समृ॑द्ध्यै॒ स्वाहा। यत॑ इन्द्र॒ भया॑महे। ततो॑ नो॒ अभ॑यङ्कृधि। मघ॑वञ्छ॒ग्धि तव॒ तन्न॑ ऊ॒तये। वि द्विषो॒ वि मृधो॑ जहि। स्व॒स्ति॒दा वि॒शस्पति॑। वृ॒त्र॒हा वि मृधो॑ व॒शी। वृषेन्द्र॑ पु॒र ए॑तु नः। स्व॒स्ति॒दा अ॑भयङ्क॒रः। आ॒भिर्गी॒र्भिर्यदतो॑ न ऊ॒नम्॥११४॥

%3.7.11.5
आप्या॑यय हरिवो॒ वर्ध॑मानः। य॒दा स्तो॒तृभ्यो॒ महि॑ गो॒त्रा रु॒जासि॑। भू॒यि॒ष्ठ॒भाजो॒ अध॑ ते स्याम। अनाज्ञातं॒ यदाज्ञा॑तम्। य॒ज्ञस्य॑ क्रि॒यते॒ मिथु॑। अग्ने॒ तद॑स्य कल्पय। त्व हि वेत्थ॑ यथात॒थम्। पुरु॑षसम्मितो य॒ज्ञः। य॒ज्ञः पुरु॑षसम्मितः। अग्ने॒ तद॑स्य कल्पय। त्व हि वेत्थ॑ यथात॒थम्। यत्पा॑क॒त्रा मन॑सा दी॒नद॑क्षा॒ न। य॒ज्ञस्य॑ म॒न्वते॒ मर्ता॑सः। अ॒ग्निष्टद्धोता॑ क्रतु॒विद्वि॑जा॒नन्। यजि॑ष्ठो दे॒वा ऋ॑तु॒शो य॑जाति॥११५॥\anuvakamend[दे॒वा श्चि॒त्रं त॒नूभ्य॒ स्वाहो॒नं पुरु॑षसम्मि॒तोऽग्ने॒ तद॑स्य कल्पय॒ पञ्च॑ च]

%3.7.12.1
यद्दे॑वा देव॒हेड॑नम्। देवा॑सश्चकृ॒मा व॒यम्। आदि॑त्या॒स्तस्मान्मा मुञ्चत। ऋ॒तस्य॒र्तेन॒ मामु॒त। देवा॑ जीवनका॒म्या यत्। वा॒चाऽनृ॑तमूदि॒म। अ॒ग्निर्मा॒ तस्मा॒देन॑सः। गार्‌ह॑पत्य॒ प्रमु॑ञ्चतु। दु॒रि॒ता यानि॑ चकृ॒म। क॒रोतु॒ माम॑ने॒नसम्॥११६॥

%3.7.12.2
ऋ॒तेन॑ द्यावापृथिवी। ऋ॒तेन॒ त्व स॑रस्वति। ऋ॒तान्मा॑ मुञ्च॒ताह॑सः। यद॒न्यकृ॑तमारि॒म। स॒जा॒त॒श॒सादु॒त वा॑ जामिश॒सात्। ज्याय॑स॒ शसा॑दु॒त वा॒ कनी॑यसः। अनाज्ञातन्दे॒वकृ॑तं॒ यदेन॑। तस्मा॒त्त्वम॒स्माञ्जा॑तवेदो मुमुग्धि। यद्वा॒चा यन्मन॑सा। बा॒हुभ्या॑मू॒रुभ्या॑मष्ठी॒वद्भ्याम्॥११७॥

%3.7.12.3
शि॒श्ञैर्यदनृ॑तञ्चकृ॒मा व॒यम्। अ॒ग्निर्मा॒ तस्मा॒देन॑सः। यद्धस्ताभ्याञ्च॒कर॒ किल्बि॑षाणि। अ॒क्षाणां व॒ग्नुमु॑प॒जिघ्न॑मानः। दू॒रे॒प॒श्या च॑ राष्ट्र॒भृच्च॑। तान्य॑प्स॒रसा॒वनु॑दत्तामृ॒णानि॑। अदी॑व्यन्नृ॒णं यद॒हञ्च॒कार॑। यद्वादास्यन्त्सञ्ज॒गारा॒ जनेभ्यः। अ॒ग्निर्मा॒ तस्मा॒देन॑सः। यन्मयि॑ मा॒ता गर्भे॑ स॒ति॥११८॥

%3.7.12.4
एन॑श्च॒कार॒ यत्पि॒ता। अ॒ग्निर्मा॒ तस्मा॒देन॑सः। यदा॑ पि॒पेष॑ मा॒तरं॑ पि॒तरम्। पु॒त्रः प्रमु॑दितो॒ धय\sn{}। अहिसितौ पि॒तरौ॒ मया॒ तत्। तद॑ग्ने अनृ॒णो भ॑वामि। यद॒न्तरि॑क्षं पृथि॒वीमु॒त द्याम्। यन्मा॒तरं॑ पि॒तरं॑ वा जिहिसि॒म। अ॒ग्निर्मा॒ तस्मा॒देन॑सः। यदा॒शसा॑ नि॒शसा॒ यत्प॑रा॒शसा॥११९॥

%3.7.12.5
यदेन॑श्चकृ॒मा नूत॑नं॒ यत्पु॑रा॒णम्। अ॒ग्निर्मा॒ तस्मा॒देन॑सः। अति॑ क्रामामि दुरि॒तं यदेन॑। जहा॑मि रि॒प्रं प॑र॒मे स॒धस्थे। यत्र॒ यन्ति॑ सु॒कृतो॒ नापि॑ दु॒ष्कृत॑। तमा रो॑हामि सु॒कृता॒न्नु लो॒कम्। त्रि॒ते दे॒वा अ॑मृजतै॒तदेन॑। त्रि॒त ए॒तन्म॑नु॒ष्ये॑षु मामृजे। ततो॑ मा॒ यदि॒ किञ्चि॑दान॒शे। अ॒ग्निर्मा॒ तस्मा॒देन॑सः॥१२०॥

%3.7.12.6
गार्‌ह॑पत्य॒ प्र मु॑ञ्चतु। दु॒रि॒ता यानि॑ चकृ॒म। क॒रोतु॒ माम॑ने॒नसम्। दि॒वि जा॒ता अ॒प्सु जा॒ताः। या जा॒ता ओष॑धीभ्यः। अथो॒ या अ॑ग्नि॒जा आप॑। ता न॑ शुन्धन्तु॒ शुन्ध॑नीः। यदापो॒ नक्त॑न्दुरि॒तञ्चरा॑म। यद्वा॒ दिवा॒ नूत॑नं॒ यत्पु॑रा॒णम्। हिर॑ण्यवर्णा॒स्तत॒ उत्पु॑नीत नः। इ॒मम्मे॑ वरुण॒ तत्त्वा॑ यामि। त्वन्नो॑ अग्ने॒ स त्वन्नो॑ अग्ने। त्वम॑ग्ने अ॒यासि॑॥१२१॥\anuvakamend[अ॒ने॒नस॑मष्ठी॒वद्भ्या स॒ति प॑रा॒शसा॑ऽऽन॒शेऽग्निर्मा॒ तस्मा॒देन॑सः पुनीत न॒स्त्रीणि॑ च (यद्दे॑वा॒ देवा॑ ऋ॒तेन॑ सजातश॒साद्यद्वा॒चा यद्धस्ताभ्या॒मदीव्यं॒ यन्मयि॑ मा॒ता यदा॑ पि॒पेष॒ यद॒न्तरि॑क्षं॒ यदा॒शसाऽति॑ क्रामामि त्रि॒ते दे॒वा दि॒वि जा॒ता अ॒प्सु जा॒ता यदाप॑ इ॒मम्मे॑ वरुण॒ तत्त्वा॑ यामि॒ त्वन्नो॑ अग्ने॒ स त्वन्नो॑ अग्ने॒ त्वम॑ग्ने अ॒यासि॑। )]

%3.7.13.1
यत्ते॒ ग्राव्ण्णा॑ चिच्छि॒दुः सो॑म राजन्। प्रि॒याण्यङ्गा॑नि॒ स्वधि॑ता॒ परूषि। तत्सन्ध॒त्स्वाज्ये॑नो॒त व॑र्धयस्व। अ॒ना॒गसो॒ अध॒मित्स॒ङ्क्षये॑म। यत्ते॒ ग्रावा॑ बा॒हुच्यु॑तो॒ अचु॑च्यवुः। नरो॒ यत्ते॑ दुदु॒हुर्दक्षि॑णेन। तत्त॒ आप्या॑यता॒न्तत्ते। निष्ट्या॑यतान्देव सोम। यत्ते॒ त्वच॑म्बिभि॒दुर्यच्च॒ योनिम्। यदा॒स्थाना॒त्प्रच्यु॑तो॒ वेन॑सि॒ त्मना ॥१२२॥

%3.7.13.2
त्वया॒ तत्सो॑म गु॒प्तम॑स्तु नः। सा न॑ स॒न्धास॑त्पर॒मे व्यो॑मन्। अहा॒च्छरी॑रं॒ पय॑सा स॒मेत्य॑। अ॒न्योन्यो भवति॒ वर्णो॑ अस्य। तस्मि॑न्व॒यमुप॑हूता॒स्तव॑ स्मः। आ नो॑ भज॒ सद॑सि वि॒श्वरू॑पे। नृ॒चक्षा॒ सोम॑ उ॒त शु॒श्रुग॑स्तु। मा नो॒ वि हा॑सी॒द्गिर॑ आवृणा॒नः। अना॑गास्त॒नुवो॑ वावृधा॒नः। आ नो॑ रू॒पं व॑हतु॒ जाय॑मानः॥१२३॥

%3.7.13.3
उप॑ क्षरन्ति जु॒ह्वो॑ घृ॒तेन॑। प्रि॒याण्यङ्गा॑नि॒ तव॑ व॒र्धय॑न्तीः। तस्मै॑ ते सोम॒ नम॒ इद्वष॑ट्च। उप॑ मा राजन्त्सुकृ॒ते ह्व॑यस्व। सं प्रा॑णापा॒नाभ्या॒ समु॒ चक्षु॑षा॒ त्वम्। स श्रोत्रे॑ण गच्छस्व सोम राजन्। यत्त॒ आस्थि॑त॒ शमु॒ तत्ते॑ अस्तु। जा॒नी॒तान्न॑ स॒ङ्गम॑ने पथी॒नाम्। ए॒तञ्जा॑नीतात्पर॒मे व्यो॑मन्। वृका सधस्था वि॒द रू॒पम॑स्य ॥१२४॥

%3.7.13.4
यदा॒गच्छात्प॒थिभि॑र्देव॒यानै। इ॒ष्टा॒पू॒र्ते कृ॑णुतादा॒विर॑स्मै। अरि॑ष्टो राजन्नग॒दः परे॑हि। नम॑स्ते अस्तु॒ चक्ष॑से रघूय॒ते। नाक॒मारो॑ह स॒ह यज॑मानेन। सूर्यं॑ गच्छतात्पर॒मे व्यो॑मन्। अभूद्दे॒वः स॑वि॒ता वन्द्यो॒नु न॑। इ॒दानी॒मह्न॑ उप॒वाच्यो॒ नृभि॑। वि यो रत्ना॒ भज॑ति मान॒वेभ्य॑। श्रेष्ठ॑न्नो॒ अत्र॒ द्रवि॑णं॒ यथा॒ दध॑त्। उप॑ नो मित्रावरुणावि॒हाव॑तम्। अ॒न्वादीध्याथामि॒ह न॑ सखाया। आ॒दि॒त्यानां॒ प्रसि॑तिर्\mbox{}हे॒तिः। उ॒ग्रा श॒तापाष्ठा घ॒विषा॒ परि॑ णो वृणक्तु। आप्या॑यस्व॒ सन्ते॥१२५॥\anuvakamend[त्मना॒ जाय॑मानोऽस्य॒ दध॒त्पञ्च॑ च]

%3.7.14.1
यद्दि॑दी॒क्षे मन॑सा॒ यच्च॑ वा॒चा। यद्वा प्रा॒णैश्चक्षु॑षा॒ यच्च॒ श्रोत्रे॑ण। यद्रेत॑सा मिथु॒नेनाप्या॒त्मना। अ॒द्भ्यो लो॒का द॑धिरे॒ तेज॑ इन्द्रि॒यम्। शु॒क्रा दी॒क्षायै॒ तप॑सो वि॒मोच॑नीः। आपो॑ विमो॒क्त्रीर्मयि॒ तेज॑ इन्द्रि॒यम्। यदृ॒चा साम्ना॒ यजु॑षा। प॒शू॒नाञ्चर्म॑न् ह॒विषा॑ दिदी॒क्षे। यच्छन्दो॑भि॒रोष॑धीभि॒र्वन॒स्पतौ। अ॒द्भ्यो लो॒का द॑धिरे॒ तेज॑ इन्द्रि॒यम् ॥१२६॥

%3.7.14.2
शु॒क्रा दी॒क्षायै॒ तप॑सो वि॒मोच॑नीः। आपो॑ विमो॒क्त्रीर्मयि॒ तेज॑ इन्द्रि॒यम्। येन॒ ब्रह्म॒ येन॑ क्ष॒त्रम्। येनेन्द्रा॒ग्नी प्र॒जाप॑ति॒ सोमो॒ वरु॑णो॒ येन॒ राजा। विश्वे॑ दे॒वा ऋष॑यो॒ येन॑ प्रा॒णाः। अ॒द्भ्यो लो॒का द॑धिरे॒ तेज॑ इन्द्रि॒यम्। शु॒क्रा दी॒क्षायै॒ तप॑सो वि॒मोच॑नीः। आपो॑ विमो॒क्त्रीर्मयि॒ तेज॑ इन्द्रि॒यम्। अ॒पां पुष्प॑म॒स्योष॑धीना॒ रस॑। सोम॑स्य प्रि॒यन्धाम॑॥१२७॥

%3.7.14.3
अ॒ग्नेः प्रि॒यत॑म ह॒विः स्वाहा। अ॒पां पुष्प॑म॒स्योष॑धीना॒ रस॑। सोम॑स्य प्रि॒यन्धाम॑। इन्द्र॑स्य प्रि॒यत॑म ह॒विः स्वाहा। अ॒पां पुष्प॑म॒स्योष॑धीना॒ रस॑। सोम॑स्य प्रि॒यन्धाम॑। विश्वे॑षान्दे॒वानां प्रि॒यत॑म ह॒विः स्वाहा। व॒य सो॑म व्र॒ते तव॑। मन॑स्त॒नूषु॒ पिप्र॑तः। प्र॒जाव॑न्तो अशीमहि॥१२८॥

%3.7.14.4
दे॒वेभ्य॑ पि॒तृभ्य॒ स्वाहा। सो॒म्येभ्य॑ पि॒तृभ्य॒ स्वाहा। क॒व्येभ्य॑ पि॒तृभ्य॒ स्वाहा। देवा॑स इ॒ह मा॑दयध्वम्। सोम्या॑स इ॒ह मा॑दयध्वम्। कव्या॑स इ॒ह मा॑दयध्वम्। अ॒न॑न्तरिताः पि॒तर॑ सो॒म्याः सो॑मपी॒थात्। अपै॑तु मृ॒त्युर॒मृत॑न्न॒ आग\sn{}। वै॒व॒स्व॒तो नो॒ अभ॑यङ्कृणोतु। प॒र्णं वन॒स्पते॑रिव॥१२९॥

%3.7.14.5
अ॒भि न॑ शीयता र॒यिः। सच॑तान्न॒ शची॒पति॑। पर॑म्मृत्यो॒ अनु॒ परे॑हि॒ पन्थाम्। यस्ते॒ स्व इत॑रो देव॒यानात्। चक्षु॑ष्मते शृण्व॒ते ते ब्रवीमि। मा न॑ प्र॒जा री॑रिषो॒ मोत वी॒रान्। इ॒दमू॒नु श्रेयो॑व॒सान॒माग॑न्म। यद्गो॒जिद्ध॑न॒जिद॑श्व॒जिद्यत्। प॒र्णं वन॒स्पते॑रिव। अ॒भि न॑ शीयता र॒यिः। सच॑तान्न॒ शची॒पति॑॥१३०॥\anuvakamend[वन॒स्पता॑व॒द्भ्यो लो॒का द॑धिरे॒ तेज॑ इन्द्रि॒यन्धामा॑शीमहीवा॒भिन॑ शीयता र॒यिरेकं च]




\prashnaend{सर्वा॒न्॒ यद्विष्ष॑ण्णेन॒ वि वै याः पु॒रस्ता॒द्देवा॑ दे॒वेषु॒ परि॑स्तृणीत॒ सक्षे॒दं यद॒स्य पा॒रे॑ऽना॒गस॒ उद॑स्तांप्सी॒द्ब्रह्म॑ प्रति॒ष्ठा यद्दे॑वा॒ यत्ते॒ ग्राव्ण्णा॒ यद्दि॑दी॒क्षे चतु॑र्दश॥१४॥}{सर्वा॒न्भूति॑मे॒व यामे॒वाप्स्वाहु॑तिं व्र॒तानां पर्णव॒ल्कः सो॒म्याना॑म॒स्मिन्‌य॒ज्ञेऽग्ने॒ यो नो॒ ज्योग्जी॒वाः प॒रोर॑जा॒ प्रते॑महे॒ ब्रह्म॑ प्रति॒ष्ठा गार्‌ह॑पत्यस्त्रि॒शदु॑त्तरश॒तम्॥१३०॥}{सर्वा॒ञ्छची॒पति॑॥}{हरि॑ ओम्॥}{इति श्रीकृष्णयजुर्वेदीयतैत्तिरीयब्राह्मणे तृतीयाष्टके सप्तमः प्रपाठकः समाप्तः॥}
\clearpage
