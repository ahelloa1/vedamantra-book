\sect{द्वितीयः प्रश्नः}
\setcounter{anuvakam}{0}
\dnsub{तैत्तिरीयब्राह्मणे द्वितीयाष्टके द्वितीयः प्रपाठकः}

%2.2.1.1
प्र॒जाप॑तिरकामयत प्र॒जाः सृ॑जे॒येति॑। स ए॒तन्दश॑होतारमपश्यत्। तं मन॑साऽनु॒द्रुत्य॑ दर्भस्त॒म्बे॑ऽजुहोत्। ततो॒ वै स प्र॒जा अ॑सृजत। ता अ॑स्मात्सृ॒ष्टा अपाक्रामन्। ता ग्रहे॑णागृह्णात्। तद्ग्रह॑स्य ग्रह॒त्वम्। यः का॒मये॑त॒ प्रजा॑ये॒येति॑। स दश॑होतारं॒ मन॑साऽनु॒द्रुत्य॑ दर्भस्त॒म्बे जु॑हुयात्। प्र॒जाप॑ति॒र्वै दश॑होता॥१॥

%2.2.1.2
प्र॒जाप॑तिरे॒व भू॒त्वा प्रजा॑यते। मन॑सा जुहोति। मन॑ इव॒ हि प्र॒जाप॑तिः। प्र॒जाप॑ते॒राप्त्यै। पू॒र्णया॑ जुहोति। पू॒र्ण इ॑व॒ हि प्र॒जाप॑तिः। प्र॒जाप॑ते॒राप्त्यै। न्यू॑नया जुहोति। न्यू॑ना॒द्धि प्र॒जाप॑तिः प्र॒जा असृ॑जत। प्र॒जाना॒ सृष्ट्यै॥२॥

%2.2.1.3
द॒र्भ॒स्त॒म्बे जु॑होति। ए॒तस्मा॒द्वै योने प्र॒जाप॑तिः प्र॒जा अ॑सृजत। यस्मा॑दे॒व योने प्र॒जाप॑तिः प्र॒जा असृ॑जत। तस्मा॑दे॒व योने॒ प्रजा॑यते। ब्रा॒ह्म॒णो द॑क्षिण॒त उपास्ते। ब्रा॒ह्म॒णो वै प्र॒जाना॑मुपद्र॒ष्टा। उ॒प॒द्र॒ष्टु॒मत्ये॒व प्रजा॑यते। ग्रहो॑ भवति। प्र॒जाना सृ॒ष्टाना॒न्धृत्यै। यं ब्राह्म॒णं वि॒द्यां वि॒द्वासं॒ यशो॒ नर्च्छेत्॥३॥

%2.2.1.4
सोऽर॑ण्यं प॒रेत्य॑। द॒र्भ॒स्त॒म्बमु॒द्ग्रथ्य॑। ब्रा॒ह्म॒णन्द॑क्षिण॒तो नि॒षाद्य॑। चतु॑र्होतॄ॒न्व्याच॑क्षीत। ए॒तद्वै दे॒वानां पर॒मङ्गुह्यं॒ ब्रह्म॑। यच्चतु॑र्‌होतारः। तदे॒व प्र॑का॒शं ग॑मयति। तदे॑नं प्रका॒शङ्ग॒तम्। प्र॒का॒शं प्र॒जानाङ्गमयति। द॒र्भ॒स्त॒म्बमु॒द्ग्रथ्य॒ व्याच॑ष्टे॥४॥

%2.2.1.5
अ॒ग्नि॒वान् वै द॑र्भस्त॒म्बः। अ॒ग्नि॒वत्ये॒व व्याच॑ष्टे। ब्रा॒ह्म॒णो द॑क्षिण॒त उपास्ते। ब्रा॒ह्म॒णो वै प्र॒जाना॑मुपद्र॒ष्टा। उ॒प॒द्र॒ष्टु॒मत्ये॒वैनं॒ यश॑ ऋच्छति। ई॒श्व॒रन्तं यशोर्तो॒रित्या॑हुः। यस्यान्ते व्या॒चष्ट॒ इति॑। वर॒स्तस्मै॒ देय॑। यदे॒वैनं॒ तत्रो॑प॒नम॑ति। तदे॒वाव॑ रुन्धे॥५॥

%2.2.1.6
अ॒ग्निमा॒दधा॑नो॒ दश॑होत्रा॒ऽरणि॒मव॑ दध्यात्। प्रजा॑तमे॒वैन॒मा ध॑त्ते। तेनै॒वोद्द्रुत्याग्निहो॒त्रं जु॑हुयात्। प्रजा॑तमे॒वैन॑ज्जुहोति। ह॒विर्नि॑र्व॒प्स्यन्दश॑होतारं॒ व्याच॑क्षीत। प्रजा॑तमे॒वैनं॒ निर्व॑पति। सा॒मि॒धे॒नीर॑नुव॒क्ष्यन्दश॑होतारं॒ व्याच॑क्षीत। सा॒मि॒धे॒नीरे॒व सृ॒ष्ट्वाऽऽरभ्य॒ प्रत॑नुते। अथो॑ य॒ज्ञो वै दश॑होता। य॒ज्ञमे॒व त॑नुते॥६॥

%2.2.1.7
अ॒भि॒चर॒न्दश॑होतारं जुहुयात्। नव॒ वै पुरु॑षे प्रा॒णाः। नाभि॑र्दश॒मी। सप्रा॑णमे॒वैन॑म॒भि च॑रति। ए॒ताव॒द्वै पुरु॑षस्य॒ स्वम्। याव॑त्प्रा॒णाः। याव॑दे॒वास्यास्ति॑। तद॒भि च॑रति। स्वकृ॑त॒ इरि॑णे जुहोति प्रद॒रे वा। ए॒तद्वा अ॒स्यै निर्‌ऋ॑तिगृहीतम्। निर्‌ऋ॑तिगृहीत ए॒वैनं॒ निर्‌ऋ॑त्या ग्राहयति। यद्वा॒चः क्रू॒रम्। तेन॒ वष॑ट्करोति। वा॒च ए॒वैनं॑ क्रू॒रेण॒ प्र वृ॑श्चति। ता॒जगार्ति॒मार्च्छ॑ति॥७॥\anuvakamend[दश॑होता॒ सृष्ट्या॑ ऋ॒च्छेद्व्याच॑प्टे रुन्ध ए॒व त॑नुते॒ निर्‌ऋ॑तिगृहीतं॒ पञ्च॑ च]

%2.2.2.1
प्र॒जाप॑तिरकामयत दर्‌शपूर्णमा॒सौ सृ॑जे॒येति॑। स ए॒तञ्चतु॑र्‌होतारमपश्यत्। तं मन॑साऽनु॒द्रुत्या॑हव॒नीये॑ऽजुहोत्। ततो॒ वै स द॑र्‌शपूर्णमा॒साव॑सृजत। ताव॑स्मात्सृ॒ष्टावपाक्रामताम्। तौ ग्रहे॑णागृह्णात्। तद्ग्रह॑स्य ग्रह॒त्वम्। द॒र्श॒पू॒र्ण॒मा॒सावा॒लभ॑मानः। चतु॑र्‌होतारं॒ मन॑साऽनु॒द्रुत्या॑हव॒नीये॑ जुहुयात्। द॒र्श॒पू॒र्ण॒मा॒सावे॒व सृ॒ष्ट्वाऽऽरभ्य॒ प्रत॑नुते॥८॥

%2.2.2.2
ग्रहो॑ भवति। द॒र्‌श॒पू॒र्ण॒मा॒सयो सृ॒ष्टयो॒र्धृत्यै। सो॑ऽकामयत चातुर्मा॒स्यानि॑ सृजे॒येति॑। स ए॒तं पञ्च॑होतारमपश्यत्। तं मन॑साऽनु॒द्रुत्या॑हव॒नीये॑ऽजुहोत्। ततो॒ वै स चा॑तुर्मा॒स्यान्य॑सृजत। तान्य॑स्मात्सृ॒ष्टान्यपाक्रामन्। तानि॒ ग्रहे॑णागृह्णात्। तद्ग्रह॑स्य ग्रह॒त्वम्। चा॒तु॒र्मा॒स्यान्या॒लभ॑मानः॥९॥

%2.2.2.3
पञ्च॑होतारं॒ मन॑साऽनु॒द्रुत्या॑हव॒नीये॑ जुहुयात्। चा॒तु॒र्मा॒स्यान्ये॒व सृ॒ष्ट्वाऽऽरभ्य॒ प्रत॑नुते। ग्रहो॑ भवति। चा॒तु॒र्मा॒स्याना सृ॒ष्टाना॒न्धृत्यै। सो॑ऽकामयत पशुब॒न्ध सृ॑जे॒येति॑। स ए॒त षड्ढो॑तारमपश्यत्। तं मन॑साऽनु॒द्रुत्या॑हव॒नीये॑ऽजुहोत्। ततो॒ वै स प॑शुब॒न्धम॑सृजत। सोस्मात्सृ॒ष्टोऽपाक्रामत्। तङ्ग्रहे॑णागृह्णात्॥१०॥

%2.2.2.4
तद्ग्रह॑स्य ग्रह॒त्वम्। प॒शु॒ब॒न्धेन॑ य॒क्ष्यमा॑णः। षड्ढो॑तारं॒ मन॑साऽनु॒द्रुत्या॑हव॒नीये॑ जुहुयात्। प॒शु॒ब॒न्धमे॒व सृ॒ष्ट्वाऽऽरभ्य॒ प्र त॑नुते। ग्रहो॑ भवति। प॒शु॒ब॒न्धस्य॑ सृ॒ष्टस्य॒ धृत्यै। सो॑ऽकामयत सौ॒म्यम॑ध्व॒र सृ॑जे॒येति॑। स ए॒त स॒प्तहो॑तारमपश्यत्। तं मन॑साऽनु॒द्रुत्या॑हव॒नीये॑ऽजुहोत्। ततो॒ वै स सौ॒म्यम॑ध्व॒रम॑सृजत॥११॥

%2.2.2.5
सोऽस्मात्सृ॒ष्टोऽपाक्रामत्। तङ्ग्रहे॑णागृह्णात्। तद्ग्रह॑स्य ग्रह॒त्वम्। दी॒क्षि॒ष्यमा॑णः। स॒प्तहो॑तारं॒ मन॑साऽनु॒द्रुत्या॑हव॒नीये॑ जुहुयात्। सौ॒म्यमे॒वाध्व॒र सृ॒ष्ट्वाऽऽरभ्य॒ प्र त॑नुते। ग्रहो॑ भवति। सौ॒म्यस्याध्व॒रस्य॑ सृ॒ष्टस्य॒ धृत्यै। दे॒वेभ्यो॒ वै य॒ज्ञो न प्राभ॑वत्। तमे॑ताव॒च्छः सम॑भरन्॥१२॥

%2.2.2.6
यत्सं॑भा॒राः। ततो॒ वै तेभ्यो॑ य॒ज्ञः प्राभ॑वत्। यत्सं॑भा॒रा भव॑न्ति। य॒ज्ञस्य॒ प्रभूत्यै। आ॒ति॒थ्यमा॒साद्य॒ व्याच॑ष्टे। य॒ज्ञ॒मु॒खं वा आ॑ति॒थ्यम्। मु॒ख॒त ए॒व य॒ज्ञ स॒म्भृत्य॒ प्र त॑नुते। अय॑ज्ञो॒ वा ए॒षः। यो॑ऽप॒त्नीक॑। न प्र॒जाः प्रजा॑येरन्। पत्नी॒र्व्याच॑ष्टे। य॒ज्ञमे॒वाक॑। प्र॒जानां प्र॒जन॑नाय। उ॒प॒सत्सु॒ व्याच॑ष्टे। ए॒तद्वै पत्नी॑नामा॒यत॑नम्। स्व ए॒वैना॑ आ॒यत॒नेऽव॑कल्पयति॥१३॥\anuvakamend[त॒नु॒त॒ आ॒लभ॑मानोऽगृह्णादसृजताभरञ्जायेर॒न्थ्षट्च॑]

%2.2.3.1
प्र॒जाप॑तिरकामयत॒ प्रजा॑ये॒येति॑। स तपो॑ऽतप्यत। स त्रि॒वृत॒ स्तोम॑मसृजत। तं प॑ञ्चद॒शः स्तोमो॑ मध्य॒त उद॑तृणत्। तौ पूर्वप॒क्षश्चा॑परप॒क्षश्चा॑भवताम्। पू॒र्व॒प॒क्षन्दे॒वा अन्वसृ॑ज्यन्त। अ॒प॒र॒प॒क्षमन्वसु॑राः। ततो॑ दे॒वा अभ॑वन्। पराऽसु॑राः। यङ्का॒मये॑त॒ वसी॑यान्त्स्या॒दिति॑॥१४॥

%2.2.3.2
तं पूर्वप॒क्षे या॑जयेत्। वसी॑याने॒व भ॑वति। यङ्का॒मये॑त॒ पापी॑यान्त्स्या॒दिति॑। तम॑परप॒क्षे या॑जयेत्। पापी॑याने॒व भ॑वति। तस्मात्पूर्वप॒क्षो॑ऽपरप॒क्षात्क॑रु॒ण्य॑तरः। प्र॒जाप॑ति॒र्वै दश॑होता। चतु॑र्‌होता॒ पञ्च॑होता। षड्ढो॑ता स॒प्तहो॑ता। ऋ॒तव॑ संवत्स॒रः॥१५॥

%2.2.3.3
प्र॒जाः प॒शव॑ इ॒मे लो॒काः। य ए॒वं प्र॒जाप॑तिं ब॒होर्भूयासं॒ वेद॑। ब॒होरे॒व भूयान्भवति। प्र॒जाप॑तिर्देवासु॒रान॑सृजत। स इन्द्र॒मपि॒ नासृ॑जत। तन्दे॒वा अ॑ब्रुवन्। इन्द्र॑न्नो जन॒येति॑। सोऽब्रवीत्। यथा॒ऽहय्युँ॒ष्मास्तप॒साऽसृ॑क्षि। ए॒वमिन्द्रं॑ जनयध्व॒मिति॑॥१६॥

%2.2.3.4
ते तपो॑ऽतप्यन्त। त आ॒त्मन्निन्द्र॑मपश्यन्। तम॑ब्रुवन्। जाय॒स्वेति॑। सोऽब्रवीत्। किं भा॑ग॒धेय॑म॒भि ज॑निष्य॒ इति॑। ऋ॒तून्त्सं॑वत्स॒रम्। प्र॒जाः प॒शून्। इ॒माल्लोँ॒कानित्य॑ब्रुवन्। तं वै माऽऽहु॑त्या॒ प्र ज॑नय॒तेत्य॑ब्रवीत्॥१७॥

%2.2.3.5
तञ्चतु॑र्‌होत्रा॒ प्राज॑नयन्। यः का॒मये॑त वी॒रो म॒ आजा॑ये॒तेति॑। स चतु॑र्‌होतारं जुहुयात्। प्र॒जाप॑ति॒र्वै चतु॑र्होता। प्र॒जाप॑तिरे॒व भू॒त्वा प्रजा॑यते। ज॒जन॒दिन्द्र॑मिन्द्रि॒याय॒ स्वाहेति॒ ग्रहे॑ण जुहोति। आऽस्य॑ वी॒रो जा॑यते। वी॒र हि दे॒वा ए॒तयाऽऽहु॑त्या॒ प्राज॑नयन्। आ॒दि॒त्याश्चाङ्गि॑रसश्च सुव॒र्गे लो॒केऽस्पर्धन्त। व॒यं पूर्वे॑ सुव॒र्गं लो॒कमि॑याम व॒यं पूर्व॒ इति॑॥१८॥

%2.2.3.6
त आ॑दि॒त्या ए॒तं पञ्च॑होतारमपश्यन्। तं पु॒रा प्रा॑तरनुवा॒कादाग्नीध्रेऽजुहवुः। ततो॒ वै ते पूर्वे॑ सुव॒र्गं लो॒कमा॑यन्। यः सु॑व॒र्गका॑म॒ स्यात्। स पञ्च॑होतारं पु॒रा प्रा॑तरनुवा॒कादाग्नीध्रे जुहुयात्। सं॒व॒त्स॒रो वै पञ्च॑होता। सं॒व॒त्स॒रः सु॑व॒र्गो लो॒कः। सं॒व॒त्स॒र ए॒वर्तुषु॑ प्रति॒ष्ठाय॑। सु॒व॒र्गं लो॒कमे॑ति। तेऽब्रुव॒न्नङ्गि॑रस आदि॒त्यान्॥१९॥

%2.2.3.7
क्व॑ स्थ। क्व॑ वः स॒द्भ्यो ह॒व्यं व॑क्ष्याम॒ इति॑। छन्द॒ स्वित्य॑ब्रुवन्। गा॒य॒त्रि॒यान्त्रि॒ष्टुभि॒ जग॑त्या॒मिति॑। तस्मा॒च्छन्द॑ सु स॒द्भ्य आ॑दि॒त्येभ्य॑। आ॒ङ्गी॒र॒सीः प्र॒जा ह॒व्यं व॑हन्ति। वह॑न्त्यस्मै प्र॒जा ब॒लिम्। ऐन॒मप्र॑तिख्यातं गच्छति। य ए॒वं वेद॑। द्वाद॑श॒ मासा॒ पञ्च॒र्तव॑। त्रय॑ इ॒मे लो॒काः। अ॒सावा॑दि॒त्य ए॑कवि॒शः। ए॒तस्मि॒न्वा ए॒ष श्रि॒तः। ए॒तस्मि॒न्प्रति॑ष्ठितः। य ए॒वमे॒त श्रि॒तं प्रति॑ष्ठितं॒ वेद॑। प्रत्ये॒व ति॑ष्ठति॥२०॥\anuvakamend[स्या॒दिति॑ संवत्स॒रो ज॑नयध्व॒मितीत्य॑ब्रवी॒त्पूर्व॒ इत्या॑दि॒त्यानृ॒तव॒ष्षट्च॑]

%2.2.4.1
प्र॒जाप॑तिरकामयत॒ प्र जा॑ये॒येति॑। स ए॒तन्दश॑होतारमपश्यत्। तेन॑ दश॒धाऽऽत्मानं॑ वि॒धाय॑। दश॑होत्राऽतप्यत। तस्य॒ चित्ति॒ स्रुगासीत्। चि॒त्तमाज्यम्। तस्यै॒ताव॑त्ये॒व वागासीत्। ए॒तावान्॑ यज्ञक्र॒तुः। स चतु॑र्‌होतारमसृजत। सो॑ऽनन्दत्॥२१॥

%2.2.4.2
असृ॑क्षि॒ वा इ॒ममिति॑। तस्य॒ सामो॑ ह॒विरासीत्। स चतु॑र्‌होत्राऽतप्यत। सो॑ऽताम्यत्। स भूरिति॒ व्याह॑रत्। स भूमि॑मसृजत। अ॒ग्नि॒हो॒त्रन्द॑र्‌शपूर्णमा॒सौ यजूषि। स द्वि॒तीय॑मतप्यत। सो॑ऽताम्यत्। स भुव॒ इति॒ व्याह॑रत्॥२२॥

%2.2.4.3
सोऽन्तरि॑क्षमसृजत। चा॒तु॒र्मा॒स्यानि॒ सामा॑नि। स तृ॒तीय॑मतप्यत। सो॑ऽताम्यत्। स सुव॒रिति॒ व्याह॑रत्। स दिव॑मसृजत। अ॒ग्नि॒ष्टो॒ममु॒क्थ्य॑मतिरा॒त्रमृच॑। ए॒ता वै व्याहृ॑तय इ॒मे लो॒काः। इ॒मान्खलु॒ वै लो॒काननु॑ प्र॒जाः प॒शव॒श्छन्दासि॒ प्राजा॑यन्त। य ए॒वमे॒ताः प्र॒जाप॑तेः प्रथ॒मा व्याहृ॑ती॒ प्रजा॑ता॒ वेद॑॥२३॥

%2.2.4.4
प्र प्र॒जया॑ प॒शुभि॑र्मिथु॒नैर्जा॑यते। स पञ्च॑होतारमसृजत। स ह॒विर्नावि॑न्दत। तस्मै॒ सोम॑स्त॒नुवं॒ प्राय॑च्छत्। ए॒तत्ते॑ ह॒विरिति॑। स पञ्च॑होत्राऽतप्यत। सो॑ऽताम्यत्। स प्र॒त्यङ्ङ॑बाधत। सोऽसु॑रानसृजत। तद॒स्याप्रि॑यमासीत्॥२४॥

%2.2.4.5
तद्दु॒र्वर्ण॒ हिर॑ण्यमभवत्। तद्दु॒र्वर्ण॑स्य॒ हिर॑ण्यस्य॒ जन्म॑। स द्वि॒तीय॑मतप्यत। सो॑ऽताम्यत्। स प्राङ॑बाधत। स दे॒वान॑सृजत। तद॑स्य प्रि॒यमा॑सीत्। तत्सु॒वर्ण॒ हिर॑ण्यमभवत्। तत्सु॒वर्ण॑स्य॒ हिर॑ण्यस्य॒ जन्म॑। य ए॒व सु॒वर्ण॑स्य॒ हिर॑ण्यस्य॒ जन्म॒ वेद॑॥२५॥

%2.2.4.6
सु॒वर्ण॑ आ॒त्मना॑ भवति। दु॒र्वर्णोऽस्य॒ भ्रातृ॑व्यः। तस्मात्सु॒वर्ण॒ हिर॑ण्यं भा॒र्यम्। सु॒वर्ण॑ ए॒व भ॑वति। ऐनं॑ प्रि॒यङ्ग॑च्छति॒ नाप्रि॑यम्। स स॒प्तहो॑तारमसृजत। स स॒प्तहोत्रै॒व सु॑व॒र्गं लो॒कमैत्। त्रि॒ण॒वेन॒ स्तोमे॑नै॒भ्यो लो॒केभ्योऽसु॑रा॒न्प्राणु॑दत। त्र॒य॒स्त्रि॒शेन॒ प्रत्य॑तिष्ठत्। ए॒क॒वि॒शेन॒ रुच॑मधत्त॥२६॥

%2.2.4.7
स॒प्त॒द॒शेन॒ प्राजा॑यत। य ए॒वं वि॒द्वान्त्सोमे॑न॒ यज॑ते। स॒प्तहोत्रै॒व सु॑व॒र्गं लो॒कमे॑ति। त्रि॒ण॒वेन॒ स्तोमे॑नै॒भ्यो लो॒केभ्यो॒ भ्रातृ॑व्या॒न्प्रणु॑दते। त्र॒य॒स्त्रि॒शेन॒ प्रति॑तिष्ठति। ए॒क॒वि॒शेन॒ रुच॑न्धत्ते। स॒प्त॒द॒शेन॒ प्र जा॑यते। तस्मात्सप्तद॒शः स्तोमो॒ न नि॒र्॒हृत्य॑। प्र॒जाप॑ति॒र्वै स॑प्तद॒शः। प्र॒जाप॑तिमे॒व म॑ध्य॒तो ध॑त्ते॒ प्रजात्यै॥२७॥\anuvakamend[अ॒न॒न्द॒द्भुव॒ इति॒ व्याह॑र॒द्वेदा॑सी॒द्वेदा॑धत्त॒ प्रजात्यै]

%2.2.5.1
दे॒वा वै वरु॑णमयाजयन्। स यस्यै॑यस्यै दे॒वता॑यै॒ दक्षि॑णा॒मन॑यत्। ताम॑व्लीनात्। तेऽब्रुवन्। व्या॒वृत्य॒ प्रति॑ गृह्णाम। तथा॑ नो॒ दक्षि॑णा॒ न व्लेष्य॒तीति॑। ते व्या॒वृत्य॒ प्रत्य॑गृह्णन्। ततो॒ वै तान्दक्षि॑णा॒ नाव्ली॑नात्। य ए॒वं वि॒द्वान्व्या॒वृत्य॒ दक्षि॑णां प्रतिगृ॒ह्णाति॑। नैन॒न्दक्षि॑णा व्लीनाति॥२८॥

%2.2.5.2
राजा त्वा॒ वरु॑णो नयतु देवि दक्षिणे॒ऽग्नये॒ हिर॑ण्य॒मित्या॑ह। आ॒ग्ने॒यं वै हिर॑ण्यम्। स्वयै॒वैन॑द्दे॒वत॑या॒ प्रति॑ गृह्णाति। सोमा॑य॒ वास॒ इत्या॑ह। सौ॒म्यं वै वास॑। स्वयै॒वैन॑द्दे॒वत॑या॒ प्रति॑ गृह्णाति। रु॒द्राय॒ गामित्या॑ह। रौ॒द्री वै गौः। स्वयै॒वैनान्दे॒वत॑या॒ प्रति॑गृह्णाति। वरु॑णा॒याश्व॒मित्या॑ह॥२९॥

%2.2.5.3
वा॒रु॒णो वा अश्व॑। स्वयै॒वैनं॑ दे॒वत॑या॒ प्रति॑गृह्णाति। प्र॒जाप॑तये॒ पुरु॑ष॒मित्या॑ह। प्रा॒जा॒प॒त्यो वै पुरु॑षः। स्वयै॒वैनं॑ दे॒वत॑या॒ प्रति॑ गृह्णाति। मन॑वे॒ तल्प॒मित्या॑ह। मा॒न॒वो वै तल्प॑। स्वयै॒वैनं॑ दे॒वत॑या॒ प्रति॑ गृह्णाति। उ॒त्ता॒नायाङ्गीर॒सायान॒ इत्या॑ह। इ॒यं वा उ॑त्ता॒न आङ्गीर॒सः॥३०॥

%2.2.5.4
अ॒नयै॒वैन॒त्प्रति॑ गृह्णाति। वै॒श्वा॒न॒र्यर्चा रथं॒ प्रति॑ गृह्णाति। वै॒श्वा॒न॒रो वै दे॒वत॑या॒ रथ॑। स्वयै॒वैनं॑ दे॒वत॑या॒ प्रति॑ गृह्णाति। तेना॑मृत॒त्वम॑श्या॒मित्या॑ह। अ॒मृत॑मे॒वात्मन्ध॑त्ते। वयो॑ दा॒त्र इत्या॑ह। वय॑ ए॒वैनं॑ कृ॒त्वा। सु॒व॒र्गं लो॒कं ग॑मयति। मयो॒ मह्य॑मस्तु प्रतिग्रही॒त्र इत्या॑ह॥३१॥

%2.2.5.5
यद्वै शि॒वम्। तन्मय॑। आ॒त्मन॑ ए॒वैषा परीत्तिः। क इ॒दङ्कस्मा॑ अदा॒दित्या॑ह। प्र॒जाप॑ति॒र्वै कः। स प्र॒जाप॑तये ददाति। काम॒ कामा॒येत्या॑ह। कामे॑न॒ हि ददा॑ति। कामे॑न प्रतिगृ॒ह्णाति॑। कामो॑ दा॒ता काम॑ प्रतिग्रही॒तेत्या॑ह॥३२॥

%2.2.5.6
कामो॒ हि दा॒ता। काम॑ प्रतिग्रही॒ता। काम समु॒द्रमावि॒शेत्या॑ह। स॒मु॒द्र इ॑व॒ हि काम॑। नेव॒ हि काम॒स्यान्तोऽस्ति॑। न स॑मु॒द्रस्य॑। कामे॑न त्वा॒ प्रति॑गृह्णा॒मीत्या॑ह। येन॒ कामे॑न प्रतिगृ॒ह्णाति॑। स ए॒वैन॑म॒मुष्मि॑ल्लोँ॒के काम॒ आग॑च्छति। कामै॒तत्त॑ ए॒षा ते॑ काम॒ दक्षि॒णेत्या॑ह। काम॑ ए॒व तद्यज॑मानो॒ऽमुष्मि॑ल्लोँ॒के दक्षि॑णामिच्छति। न प्र॑तिग्रही॒तरि॑। य ए॒वं वि॒द्वान्दक्षि॑णां प्रतिगृ॒ह्णाति॑। अ॒नृ॒णामे॒वैनां॒ प्रति॑ गृह्णाति॥३३॥\anuvakamend[व्ली॒ना॒त्यश्व॒मित्या॑हाङ्गीर॒सः प्र॑तिग्रही॒त्र इत्या॑ह प्रतिग्रही॒तेत्या॑ह॒ दक्षि॒णेत्या॑ह च॒त्वारि॑ च]

%2.2.6.1
अन्तो॒ वा ए॒ष य॒ज्ञस्य॑। यद्द॑श॒ममह॑। द॒श॒मेऽहन्त्सर्परा॒ज्ञिया॑ ऋ॒ग्भिः स्तु॑वन्ति। य॒ज्ञस्यै॒वान्त॑ङ्ग॒त्वा। अ॒न्नाद्य॒मव॑ रुन्धते। ति॒सृभि॑ स्तुवन्ति। त्रय॑ इ॒मे लो॒काः। ए॒भ्य ए॒व लो॒केभ्यो॒ऽन्नाद्य॒मव॑ रुन्धते। पृश्ञि॑वतीर्भवन्ति। अन्नं॒ वै पृश्ञि॑॥३४॥

%2.2.6.2
अन्न॑मे॒वाव॑ रुन्धते। मन॑सा॒ प्रस्तौ॑ति। मन॒सोद्गा॑यति। मन॑सा॒ प्रति॑ हरति। मन॑ इव॒ हि प्र॒जाप॑तिः। प्र॒जाप॑ते॒राप्त्यै। दे॒वा वै स॒र्पाः। तेषा॑मि॒य राज्ञी। यत्स॑र्परा॒ज्ञिया॑ ऋ॒ग्भिः स्तु॒वन्ति॑। अ॒स्यामे॒व प्रति॑ तिष्ठन्ति॥३५॥

%2.2.6.3
चतु॑र्‌होतॄ॒न्॒ होता॒ व्याच॑ष्टे। स्तु॒तमनु॑शसति॒ शान्त्यै। अन्तो॒ वा ए॒ष य॒ज्ञस्य॑। यद्द॑श॒ममह॑। ए॒तत्खलु॒ वै दे॒वानां पर॒मङ्गुह्यं॒ ब्रह्म॑। यच्चतु॑र्होतारः। द॒श॒मेऽह॒ श्चतु॑र्‌होतॄ॒न्व्याच॑ष्टे। य॒ज्ञस्यै॒वान्त॑ङ्ग॒त्वा। प॒र॒मन्दे॒वाना॒ङ्गुह्यं॒ ब्रह्माव॑ रुन्धे। तदे॒व प्र॑का॒शं ग॑मयति॥३६॥

%2.2.6.4
तदे॑नं प्रका॒शङ्ग॒तम्। प्र॒का॒शं प्र॒जानाङ्गमयति। वाचं॑ यच्छति। य॒ज्ञस्य॒ धृत्यै। य॒ज॒मा॒न॒दे॒व॒त्यं॑ वा अह॑। भ्रा॒तृ॒व्य॒दे॒व॒त्या॑ रात्रि॑। अह्ना॒ रात्रि॑न्ध्यायेत्। भ्रातृ॑व्यस्यै॒व तल्लो॒कं वृ॑ङ्क्ते। यद्दिवा॒ वाचं॑ विसृ॒जेत्। अह॒र्भ्रातृ॑व्या॒योच्छिषेत्। यन्नक्तं॑ विसृ॒जेत्। रात्रिं॒ भ्रातृ॑व्या॒योच्छिषेत्। अ॒धि॒वृ॒क्ष॒सू॒र्ये वाचं॒ विसृ॑जति। ए॒ताव॑न्तमे॒वास्मै॑ लो॒कमुच्छिषति। याव॑दादि॒त्योऽस्त॒मेति॑॥३७॥\anuvakamend[पृश्ञि॑ तिष्ठन्ति गमयति शिषे॒त्पञ्च॑ च]

%2.2.7.1
प्र॒जाप॑तिः प्र॒जा अ॑सृजत। ताः सृ॒ष्टाः सम॑श्लिष्यन्। ता रू॒पेणानु॒प्रावि॑शत्। तस्मा॑दाहुः। रू॒पं वै प्र॒जाप॑ति॒रिति॑। ता नाम्नाऽनु॒ प्रावि॑शत्। तस्मा॑दाहुः। नाम॒ वै प्र॒जाप॑ति॒रिति॑। तस्मा॒दप्या॑मि॒त्रौ स॒ङ्गत्य॑। नाम्ना॒ चेद्ध्वये॑ते॥३८॥

%2.2.7.2
मि॒त्रमे॒व भ॑वतः। प्र॒जाप॑तिर्देवासु॒रान॑सृजत। स इन्द्र॒मपि॒ नासृ॑जत। तन्दे॒वा अ॑ब्रुवन्। इन्द्र॑न्नो जन॒येति॑। स आ॒त्मन्निन्द्र॑मपश्यत्। तम॑सृजत। तन्त्रि॒ष्टुग्वी॒र्यं॑ भू॒त्वाऽनु॒ प्रावि॑शत्। तस्य॒ वज्र॑ पञ्चद॒शो हस्त॒ आप॑द्यत। तेनो॒दय्यासु॑रान॒भ्य॑भवत्॥३९॥

%2.2.7.3
य ए॒वं वेद॑। अ॒भि भ्रातृ॑व्यान्भवति। ते दे॒वा असु॑रैर्वि॒जित्य॑। सु॒व॒र्गं लो॒कमा॑यन्। ते॑ऽमुष्मि॑ल्लोँ॒के व्य॑क्षुध्यन्। तेऽब्रुवन्। अ॒मुत॑ प्रदानं॒ वा उप॑जिजीवि॒मेति॑। ते स॒प्तहो॑तारं य॒ज्ञं वि॒धाया॒यास्यम्। आ॒ङ्गी॒र॒सं प्राहि॑ण्वन्। ए॒तेना॒मुत्र॑ कल्प॒येति॑॥४०॥

%2.2.7.4
तस्य॒ वा इ॒यङ्कॢप्ति॑। यदि॒दङ्किं च॑। य ए॒वं वेद॑। कल्प॑तेऽस्मै। स वा अ॒यं म॑नु॒ष्ये॑षु य॒ज्ञः स॒प्तहो॑ता। अ॒मुत्र॑ स॒द्भ्यो दे॒वेभ्यो॑ ह॒व्यं व॑हति। य ए॒वं वेद॑। उपै॑नं य॒ज्ञो न॑मति। सो॑ऽमन्यत। अ॒भि वा इ॒मेऽस्माल्लो॒काद॒मुं लो॒कङ्क॑मिष्यन्त॒ इति॑। स वाच॑स्पते॒ हृदिति॒ व्याह॑रत्। तस्मात्पु॒त्रो हृद॑यम्। तस्मा॑द॒स्माल्लो॒काद॒मुं लो॒कन्नाभि का॑मयन्ते। पु॒त्रो हि हृद॑यम्॥४१॥\anuvakamend[ह्वये॑ते अभवत्कल्प॒येतीति॑ च॒त्वारि॑ च]

%2.2.8.1
दे॒वा वै चतु॑र्‌होतृभिर्य॒ज्ञम॑तन्वत। ते वि पा॒प्मना॒ भ्रातृ॑व्ये॒णाज॑यन्त। अ॒भि सु॑व॒र्गं लो॒कम॑जयन्। य ए॒वं वि॒द्वाश्चतु॑र्होतृभिर्य॒ज्ञन्त॑नु॒ते। वि पा॒प्मना॒ भ्रातृ॑व्येण जयते। अ॒भि सु॑व॒र्गं लो॒कं ज॑यति। षड्ढोत्रा प्राय॒णीय॒मा सा॑दयति। अ॒मुष्मै॒ वै लो॒काय॒ षड्ढो॑ता। घ्नन्ति॒ खलु॒ वा ए॒तत्सोमम्। यद॑भिषु॒ण्वन्ति॑॥४२॥

%2.2.8.2
ऋ॒जु॒धैवैन॑म॒मुं लो॒कं ग॑मयति। चतु॑र्‌होत्राऽऽति॒थ्यम्। यशो॒ वै चतु॑र्‌होता। यश॑ ए॒वात्मन्ध॑त्ते। पञ्च॑होत्रा प॒शुमुप॑सादयति। सु॒व॒र्ग्यो॑ वै पञ्च॑होता। यज॑मानः प॒शुः। यज॑मानमे॒व सु॑व॒र्गं लो॒कं ग॑मयति। ग्रहान्गृही॒त्वा स॒प्तहो॑तारं जुहोति। इ॒न्द्रि॒यं वै स॒प्तहो॑ता॥४३॥

%2.2.8.3
इ॒न्द्रि॒यमे॒वात्मन्ध॑त्ते। यो वै चतु॑र्‌होतॄननुसव॒नन्त॒र्पय॑ति। तृप्य॑ति प्र॒जया॑ प॒शुभि॑। उपै॑न सोमपी॒थो न॑मति। ब॒हि॒ष्प॒व॒मा॒ने दश॑होतारं॒ व्याच॑क्षीत। माध्य॑न्दिने॒ पव॑माने॒ चतु॑र्‌होतारम्। आर्भ॑वे॒ पव॑माने॒ पञ्च॑होतारम्। पि॒तृ॒य॒ज्ञे षड्ढो॑तारम्। य॒ज्ञा॒य॒ज्ञिय॑स्य स्तो॒त्रे स॒प्तहो॑तारम्। अ॒नु॒स॒व॒नमे॒वैनास्तर्पयति॥४४॥

%2.2.8.4
तृप्य॑ति प्र॒जया॑ प॒शुभि॑। उपै॑न सोमपी॒थो न॑मति। दे॒वा वै चतु॑र्‌होतृभिः स॒त्रमा॑सत। ऋद्धि॑परिमितं॒ यश॑स्कामाः। तेऽब्रुवन्। यन्न॑ प्रथ॒मं यश॑ ऋ॒च्छात्। सर्वे॑षान्न॒स्तत्स॒हास॒दिति॑। सोम॒श्चतु॑र्‌होत्रा। अ॒ग्निः पञ्च॑होत्रा। धा॒ता षड्ढोत्रा॥४५॥

%2.2.8.5
इन्द्र॑ स॒प्तहोत्रा। प्र॒जाप॑ति॒र्दश॑होत्रा। तेषा॒ सोम॒ राजा॑नं॒ यश॑ आर्च्छत्। तन्न्य॑कामयत। तेनापाक्रामत्। तेन॑ प्र॒लाय॑मचरत्। तन्दे॒वाः प्रै॒षैः प्रैष॑मैच्छन्। तत्प्रै॒षाणां प्रैष॒त्वम्। नि॒विद्भि॒र्न्य॑वेदयन्। तन्नि॒विदान्निवि॒त्त्वम्॥४६॥

%2.2.8.6
आ॒प्रीभि॑राप्नुवन्। तदा॒प्रीणा॑माप्रि॒त्वम्। तम॑घ्नन्। तस्य॒ यशो॒ व्य॑गृह्णत। ते ग्रहा॑ अभवन्। तद्ग्रहा॑णाङ्ग्रह॒त्वम्। यस्यै॒वं वि॒दुषो॒ ग्रहा॑ गृ॒ह्यन्ते। तस्य॒ त्वे॑व गृ॑ही॒ताः। तेऽब्रुवन्। यो वै न॒ श्रेष्ठोऽभूत्॥४७॥

%2.2.8.7
तम॑वधिष्म। पुन॑रि॒म सु॑वामहा॒ इति॑। तञ्छन्दो॑भिरसुवन्त। तच्छन्द॑साञ्छन्द॒स्त्वम्। साम्ना॒ समान॑यन्। तत्साम्न॑ साम॒त्वम्। उ॒क्थैरुद॑स्थापयन्। तदु॒क्थाना॑मुक्थ॒त्वम्। य ए॒वं वेद॑। प्रत्ये॒व ति॑ष्ठति॥४८॥

%2.2.8.8
सर्व॒मायु॑रेति। सोमो॒ वै यश॑। य ए॒वं वि॒द्वान्त्सोम॑मा॒गच्छ॑ति। यश॑ ए॒वैन॑मृच्छति। तस्मा॑दाहुः। यश्चै॒वं वेद॒ यश्च॒ न। तावु॒भौ सोम॒माग॑च्छतः। सोमो॒ हि यश॑। तन्त्वाऽव यश॑ ऋच्छ॒तीत्या॑हुः। यः सोमे॒ सोमं॒ प्राहेति॑। तस्मा॒त्सोमे॒ सोम॒ प्रोच्य॑। यश॑ ए॒वैन॑मृच्छति॥४९॥\anuvakamend[अ॒भि॒षु॒ण्वन्ति॑ स॒प्तहो॑ता तर्पयति॒ षड्ढोत्रा निवि॒त्त्वमभूत्तिष्ठति॒ प्राहेति॒ द्वे च॑]

%2.2.9.1
इ॒दं वा अग्रे॒ नैव किं च॒ नासीत्। न द्यौरा॑सीत्। न पृ॑थि॒वी। नान्तरि॑क्षम्। तदस॑दे॒व सन्मनो॑ऽकुरुत॒ स्यामिति॑। तद॑तप्यत। तस्मात्तेपा॒नाद्धू॒मो॑ऽजायत। तद्भूयो॑ऽतप्यत। तस्मात्तेपा॒नाद॒ग्निर॑जायत। तद्भूयो॑ऽतप्यत॥५०॥

%2.2.9.2
तस्मात्तेपा॒नाज्ज्योति॑रजायत। तद्भूयो॑ऽतप्यत। तन्मात्तेपा॒नाद॒र्चिर॑जायत। तद्भूयो॑ऽतप्यत। तस्मात्तेपा॒नान्मरी॑चयोऽजायन्त। तद्भूयो॑ऽतप्यत। तस्मात्तेपा॒नादु॑दा॒रा अ॑जायन्त। तद्भूयो॑ऽतप्यत। तद॒भ्रमि॑व॒ सम॑हन्यत। तद्व॒स्तिम॑भिनत्॥५१॥

%2.2.9.3
स स॑मु॒द्रो॑ऽभवत्। तस्मात्समु॒द्रस्य॒ न पि॑बन्ति। प्र॒जन॑नमिव॒ हि मन्य॑न्ते। तस्मात्प॒शोर्जाय॑माना॒दाप॑ पु॒रस्ताद्यन्ति। तद्दश॑हो॒ताऽन्व॑सृज्यत। प्र॒जाप॑ति॒र्वै दश॑होता। य ए॒वन्तप॑सो वी॒र्य॑ं वि॒द्वा स्तप्य॑ते। भव॑त्ये॒व। तद्वा इ॒दमाप॑ सलि॒लमा॑सीत्। सो॑ऽरोदीत्प्र॒जाप॑तिः॥५२॥

%2.2.9.4
स कस्मा॑ अज्ञि। यद्य॒स्या अप्र॑तिष्ठाया॒ इति॑। यद॒प्स्व॑वाप॑द्यत। सा पृ॑थि॒व्य॑भवत्। यद्व्यमृ॑ष्ट। तद॒न्तरि॑क्षमभवत्। यदू॒र्ध्वमु॒दमृ॑ष्ट। सा द्यौर॑भवत्। यदरो॑दीत्। तद॒नयो॑ रोद॒स्त्वम्॥५३॥

%2.2.9.5
य ए॒वं वेद॑। नास्य॑ गृ॒हे रु॑दन्ति। ए॒तद्वा ए॒षां लो॒कानां॒ जन्म॑। य ए॒वमे॒षां लो॒कानां॒ जन्म॒ वेद॑। नैषु लो॒केष्वार्ति॒मार्च्छ॑ति। स इ॒मां प्र॑ति॒ष्ठाम॑विन्दत। स इ॒मां प्र॑ति॒ष्ठां वि॒त्वाऽका॑मयत॒ प्रजा॑ये॒येति॑। स तपो॑ऽतप्यत। सोऽन्तर्वा॑नभवत्। स ज॒घना॒दसु॑रानसृजत॥५४॥

%2.2.9.6
तेभ्यो॑ मृ॒न्मये॒ पात्रेऽन्न॑मदुहत्। याऽस्य॒ सा त॒नूरासीत्। तामपा॑हत। सा तमि॑स्राऽभवत्। सो॑कामयत॒ प्रजा॑ये॒येति॑। स तपो॑ऽतप्यत। सोन्तर्वा॑नभवत्। स प्र॒जन॑नादे॒व प्र॒जा अ॑सृजत। तस्मा॑दि॒मा भूयि॑ष्ठाः। प्र॒जन॑ना॒द्ध्ये॑ना॒ असृ॑जत॥५५॥

%2.2.9.7
ताभ्यो॑ दारु॒मये॒ पात्रे॒ पयो॑ऽदुहत्। याऽस्य॒ सा त॒नूरासीत्। तामपा॑हत। सा जोत्स्ना॑ऽभवत्। सो॑ऽकामयत॒ प्रजा॑ये॒येति॑। स तपो॑ऽतप्यत। सोऽन्तर्वा॑नभवत्। स उ॑पप॒क्षाभ्या॑मे॒वर्तून॑सृजत। तेभ्यो॑ रज॒ते पात्रे॑ घृ॒तम॑दुहत्। याऽस्य॒ सा त॒नूरासीत्॥५६॥

%2.2.9.8
तामपा॑हत। सो॑ऽहोरा॒त्रयो स॒न्धिर॑भवत्। सो॑ऽकामयत॒ प्रजा॑ये॒येति॑। स तपो॑ऽतप्यत। सोऽन्तर्वा॑नभवत्। स मुखाद्दे॒वान॑सृजत। तेभ्यो॒ हरि॑ते॒ पात्रे॒ सोम॑मदुहत्। याऽस्य॒ सा त॒नूरासीत्। तामपा॑हत। तदह॑रभवत्॥५७॥

%2.2.9.9
ए॒ते वै प्र॒जाप॑ते॒र्दोहा। य ए॒वं वेद॑। दु॒ह ए॒व प्र॒जाः। दिवा॒ वै नो॑ऽभू॒दिति॑। तद्दे॒वानान्देव॒त्वम्। य ए॒वन्दे॒वानान्देव॒त्वं वेद॑। दे॒ववा॑ने॒व भ॑वति। ए॒तद्वा अ॑होरा॒त्राणां॒ जन्म॑। य ए॒वम॑होरा॒त्राणां॒ जन्म॒ वेद॑। नाहो॑रा॒त्रेष्वार्ति॒मार्च्छ॑ति॥५८॥

%2.2.9.10
अस॒तोऽधि॒ मनो॑ऽसृज्यत। मन॑ प्र॒जाप॑तिमसृजत। प्र॒जाप॑तिः प्र॒जा अ॑सृजत। तद्वा इ॒दं मन॑स्ये॒व प॑र॒मं प्रति॑ष्ठितम्। यदि॒दङ्किं च॑। तदे॒तच्छ्वो॑वस्य॒सन्नाम॒ ब्रह्म॑। व्यु॒च्छन्तीव्युच्छन्त्यस्मै॒ वस्य॑सीवस्यसी॒ व्यु॑च्छति। प्रजा॑यते प्र॒जया॑ प॒शुभि॑। प्र प॑रमे॒ष्ठिनो॒ मात्रा॑माप्नोति। य ए॒वं वेद॑॥५९॥\anuvakamend[अ॒ग्निर॑जायत॒ तद्भूयो॑ऽतप्यताभिनदरोदीत्प्र॒जाप॑तीरोद॒स्त्वम॑सृज॒तासृ॑जत घृ॒तम॑दुह॒द्याऽस्य॒ सा त॒नूरासी॒दह॑रभवदृच्छति॒ वेद॑ (इ॒दं धू॒मोऽग्निर्ज्योति॑र॒र्चिर्मरी॑चय उदा॒रास्तद॒ब्भ्र स ज॒घना॒त्सा तमि॑स्रा॒ स प्र॒जन॑ना॒त्सा जोत्स्ना॒ स उ॑पप॒क्षाभ्या॒ सो॑ऽहोरा॒त्रयो स॒न्धिः स मुखा॒त्तदह॑र्दे॒ववान्मृ॒न्मये॑ दारु॒मये॑ रज॒ते हरि॑ते॒ तेभ्य॒स्ताभ्यो॒ द्वे तेऽन्नं॒ पयो॑ घृ॒त सोमम् ॥ )]

%2.2.10.1
प्र॒जाप॑ति॒रिन्द्र॑मसृजतानुजाव॒रन्दे॒वानाम्। तं प्राहि॑णोत्। परे॑हि। ए॒तेषान्दे॒वाना॒मधि॑पतिरे॒धीति॑। तन्दे॒वा अ॑ब्रुवन्। कस्त्वमसि॑। व॒यं वै त्वच्छ्रेयासः स्म॒ इति॑। सोऽब्रवीत्। कस्त्वमसि॑ व॒यं वै त्वच्छ्रेयासः स्म॒ इति॑ मा दे॒वा अ॑वोचं॒ निति॑। अथ॒ वा इ॒दन्तर्‌हि॑ प्र॒जाप॑तौ॒ हर॑ आसीत्॥६०॥

%2.2.10.2
यद॒स्मिन्ना॑दि॒त्ये। तदे॑नमब्रवीत्। ए॒तन्मे॒ प्रय॑च्छ। अथा॒हमे॒तेषान्दे॒वाना॒मधि॑पतिर्भविष्या॒मीति॑। को॑ऽह स्या॒मित्य॑ब्रवीत्। ए॒तत्प्र॒दायेति॑। ए॒तत्स्या॒ इत्य॑ब्रवीत्। यदे॒तद्ब्रवी॒षीति॑। को ह॒ वै नाम॑ प्र॒जाप॑तिः। य ए॒वं वेद॑॥६१॥

%2.2.10.3
वि॒दुरे॑न॒न्नाम्ना। तद॑स्मै रु॒क्मं कृ॒त्वा प्रत्य॑मुञ्चत्। ततो॒ वा इन्द्रो॑ दे॒वाना॒मधि॑पतिरभवत्। य ए॒वं वेद॑। अधि॑पतिरे॒व स॑मा॒नानां भवति। सो॑ऽमन्यत। किङ्किं॒ वा अ॑कर॒मिति॑। स च॒न्द्रं म॒ आह॒रेति॒ प्राल॑पत्। तच्च॒न्द्रम॑सश्चन्द्रम॒स्त्वम्। य ए॒वं वेद॑॥६२॥

%2.2.10.4
च॒न्द्रवा॑ने॒व भ॑वति। तन्दे॒वा अ॑ब्रुवन्। सु॒वीर्यो॑ मर्या॒ यथा॑ गोपा॒यत॒ इति॑। तत्सूर्य॑स्य सूर्य॒त्वम्। य ए॒वं वेद॑। नैन॑न्दभ्नोति। कश्च॒ नास्मि॒न्वा इ॒दमि॑न्द्रि॒यं प्रत्य॑स्था॒दिति॑। तदिन्द्र॑स्येन्द्र॒त्वम्। य ए॒वं वेद॑। इ॒न्द्रि॒या॒व्ये॑व भ॑वति॥६३॥

%2.2.10.5
अ॒यं वा इ॒दं प॑र॒मो॑ऽभू॒दिति॑। तत्प॑रमे॒ष्ठिन॑ परमेष्ठि॒त्वम्। य ए॒वं वेद॑। प॒र॒मामे॒व काष्ठां गच्छति। तन्दे॒वाः स॑म॒न्तं पर्य॑विशन्। वस॑वः पु॒रस्तात्। रु॒द्रा द॑क्षिण॒तः। आ॒दि॒त्याः प॒श्चात्। विश्वे॑ दे॒वा उ॑त्तर॒तः। अङ्गि॑रसः प्र॒त्यञ्चम्॥६४॥

%2.2.10.6
सा॒ध्याः पराञ्चम्। य ए॒वं वेद॑। उपै॑न समा॒नाः संवि॑शन्ति। स प्र॒जाप॑तिरे॒व भू॒त्वा प्र॒जा आव॑यत्। ता अ॑स्मै॒ नाति॑ष्ठन्ता॒न्नाद्या॑य। ता मुखं॑ पु॒रस्ता॒त्पश्य॑न्तीः। द॒क्षि॒ण॒तः पर्या॑यन्। स द॑क्षिण॒तः पर्य॑वर्तयत। ता मुखं॑ पु॒रस्ता॒त्पश्य॑न्तीः। मुख॑न्दक्षिण॒तः॥६५॥

%2.2.10.7
प॒श्चात्पर्या॑यन्। स प॒श्चात्पर्य॑वर्तयत। ता मुखं॑ पु॒रस्ता॒त्पश्य॑न्तीः। मुख॑न्दक्षिण॒तः। मुखं॑ प॒श्चात्। उ॒त्त॒र॒तः पर्या॑यन्। स उ॑त्तर॒तः पर्य॑वर्तयत। ता मुखं॑ पु॒रस्ता॒त्पश्य॑न्तीः। मुख॑न्दष्खिण॒तः। मुखं॑ प॒श्चात्॥६६॥

%2.2.10.8
मुख॑मुत्तर॒तः। ऊ॒र्ध्वा उदा॑यन्। स उ॒परि॑ष्टा॒न्न्य॑वर्तयत। ताः स॒र्वतो॑मुखो भू॒त्वाऽऽव॑यत्। ततो॒ वै तस्मै प्र॒जा अति॑ष्ठन्ता॒न्नाद्या॑य। य ए॒वं वि॒द्वान्परि॑ च व॒र्तय॑ते॒ नि च॑। प्र॒जाप॑तिरे॒व भू॒त्वा प्र॒जा अ॑त्ति। तिष्ठ॑न्तेऽस्मै प्र॒जा अ॒न्नाद्या॑य। अ॒न्ना॒द ए॒व भव॑ति॥६७॥\anuvakamend[आ॒सी॒द्वेद॑ चन्द्रम॒स्त्वं य ए॒वं वेदेन्द्रिया॒व्ये॑व भ॑वति प्र॒त्यञ्चं॒ मुख॑न्दक्षिण॒तो मुखं॑ प॒श्चान्नव॑ च]

%2.2.11.1
प्र॒जाप॑तिरकामयत ब॒होर्भूयान्त्स्या॒मिति॑। स ए॒तन्दश॑होतारमपश्यत्। तं प्रायु॑ङ्क्त। तस्य॒ प्रयु॑क्ति ब॒होर्भूया॑नभवत्। यः का॒मये॑त ब॒होर्भूयान्त्स्या॒मिति॑। स दश॑होतारं॒ प्रयु॑ञ्जीत। ब॒होरे॒व भूयान्भवति। सो॑ऽकामयत वी॒रो म॒ आजा॑ये॒तेति॑। स दश॑होतु॒श्चतु॑र्‌होतारं॒ निर॑मिमीत। तं प्रायु॑ङ्क्त॥६८॥

%2.2.11.2
तस्य॒ प्रयु॒क्तीन्द्रो॑ऽजायत। यः का॒मये॑त वी॒रो म॒ आजा॑ये॒तेति॑। स चतु॑र्‌होतारं॒ प्रयु॑ञ्जीत। आऽस्य॑ वी॒रो जा॑यते। सो॑ऽकामयत पशु॒मान्त्स्या॒मिति॑। स चतु॑र्‌होतु॒ पञ्च॑होतारं॒ निर॑मिमीत। तं प्रायु॑ङ्क्त। तस्य॒ प्रयु॑क्ति पशु॒मान॑भवत्। यः का॒मये॑त पशु॒मान्त्स्या॒मिति॑। स पञ्च॑होतारं॒ प्रयु॑ञ्जीत॥६९॥

%2.2.11.3
प॒शु॒माने॒व भ॑वति। सो॑ऽकामयत॒र्तवो॑ मे कल्पेर॒न्निति॑। स पञ्च॑होतु॒ षड्ढो॑तारं॒ निर॑मिमीत। तं प्रायु॑ङ्क्त। तस्य॒ प्रयु॑क्त्यृ॒तवोऽस्मा अकल्पन्त। यः का॒मये॑त॒र्तवो॑ मे कल्पेर॒न्निति॑। स षड्ढो॑तारं॒ प्रयु॑ञ्जीत। कल्प॑न्तेऽस्मा ऋ॒तव॑। सो॑ऽकामयत सोम॒पः सो॑मया॒जी स्याम्। आ मे॑ सोम॒पः सो॑मया॒जी जा॑ये॒तेति॑॥७०॥

%2.2.11.4
स षड्ढो॑तुः स॒प्तहो॑तारं॒ निर॑मिमीत। तं प्रायु॑ङ्क्त। तस्य॒ प्रयु॑क्ति सोम॒पः सो॑मया॒ज्य॑भवत्। आऽस्य॑ सोम॒पः सो॑मया॒ज्य॑जायत। यः का॒मये॑त सोम॒पः सो॑मया॒जी स्याम्। आ मे॑ सोम॒पः सो॑मया॒जी जा॑ये॒तेति॑। स स॒प्तहो॑तारं॒ प्रयु॑ञ्जीत। सो॒म॒प ए॒व सो॑मया॒जी भ॑वति। आऽस्य॑ सोम॒पः सो॑मया॒जी जा॑यते। स वा ए॒ष प॒शुः प॑ञ्च॒धा प्रति॑तिष्ठति॥७१॥

%2.2.11.5
प॒द्भिर्मुखे॑न। ते दे॒वाः प॒शून् वि॒त्वा। सु॒व॒र्गं लो॒कमा॑यन्। ते॑ऽमुष्मि॑ल्लोँ॒के व्य॑क्षुध्यन्। तेऽब्रुवन्। अ॒मुत॑ प्रदानं॒ वा उप॑जिजीवि॒मेति॑। ते स॒प्तहो॑तारं य॒ज्ञं वि॒धाया॒यास्यम्। आ॒ङ्गी॒र॒सं प्राहि॑ण्वन्। ए॒तेना॒मुत्र॑ कल्प॒येति॑। तस्य॒ वा इ॒यङ्कॢप्ति॑॥७२॥

%2.2.11.6
यदि॒दङ्किं च॑। य ए॒वं वेद॑। कल्प॑तेऽस्मै। स वा अ॒यं म॑नु॒ष्ये॑षु य॒ज्ञः स॒प्तहो॑ता। अ॒मुत्र॑ स॒द्भ्यो दे॒वेभ्यो॑ ह॒व्यं व॑हति। य ए॒वं वेद॑। उपै॑नं य॒ज्ञो न॑मति। यो वै चतु॑र्‌होतृणान्नि॒दानं॒ वेद॑। नि॒दान॑वान्भवति। अ॒ग्नि॒हो॒त्रं वै दश॑होतुर्नि॒दानम्। द॒र्‌श॒पू॒र्ण॒मासौ चतु॑र्‌होतुः। चा॒तु॒र्मा॒स्यानि॒ पञ्च॑होतुः। प॒शु॒ब॒न्धष्षड्ढो॑तुः। सौ॒म्योऽध्व॒रः स॒प्तहो॑तुः। ए॒तद्वै चतु॑र्‌होतृणान्नि॒दानम्। य ए॒वं वेद॑। नि॒दान॑वान्भवति॥७३॥\anuvakamend[अ॒मि॒मी॒त॒ तं प्रायु॑ङ्क्त॒ पञ्च॑होतारं॒ प्र यु॑ञ्जीत जाये॒तेति॑ तिष्ठति॒ कॢप्ति॒र्दश॑होतुर्नि॒दान स॒प्त च॑]




\prashnaend{प्र॒जाप॑तिरकामयत प्र॒जाः सृ॑जे॒येति॑ प्र॒जाप॑तिरकामयत दर्‌शपूर्णमा॒सौ सृ॑जे॒येति॑ प्र॒जाप॑तिरकामयत॒ प्रजा॑ये॒येति॒ स तप॒ स त्रि॒वृतं॑ प्र॒जाप॑तिरकामयत॒ दश॑होतारं॒ तेन॑ दश॒धाऽऽत्मानं॑ दे॒वा वै वरु॑ण॒मन्तो॒ वै प्र॒जाप॑ति॒स्ताः सृ॒ष्टाः सम॑श्लिष्यन्दे॒वा वै चतु॑र्‌होतृभिरि॒दं वा अग्रे प्र॒जाप॑ति॒रिन्द्रं॑ प्र॒जाप॑तिरकामयत ब॒होर्भूया॒नेका॑दश॥११॥}{प्र॒जाप॑ति॒स्तद्ग्रह॑स्य प्र॒जाप॑तिरकामयता॒नयै॒वैन॒त्तस्य॒ वा इ॒यं कॢप्ति॒स्तस्मात्तेपा॒नाज्ज्योति॒र्यद॒स्मिन्ना॑दि॒त्ये स षड्ढो॑तुः स॒प्तहो॑तार॒न्त्रिस॑प्ततिः॥७३॥}{प्र॒जाप॑तिरकामयत नि॒दान॑वान्भवति॥}{हरि॑ ओम्॥}{इति श्रीकृष्णयजुर्वेदीयतैत्तिरीयब्राह्मणे द्वितीयाष्टके द्वितीयः प्रपाठकः समाप्तः॥}
\clearpage
