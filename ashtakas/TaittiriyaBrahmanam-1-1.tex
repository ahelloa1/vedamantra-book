\sect{प्रथमः प्रश्नः}
\setcounter{anuvakam}{0}
\dnsub{तैत्तिरीयब्राह्मणे प्रथमाष्टके प्रथमः प्रपाठकः}

%1.1.1.1
ब्रह्म॒ सन्ध॑त्तं॒ तन्मे॑ जिन्वतम्। क्ष॒त्र सन्ध॑त्तं॒ तन्मे॑ जिन्वतम्। इष॒ सन्ध॑त्तं॒ तां मे॑ जिन्वतम्। ऊर्ज॒ सन्ध॑त्तं॒ तां मे॑ जिन्वतम्। र॒यि सन्ध॑त्तं॒ तां मे॑ जिन्वतम्। पुष्टि॒ सन्ध॑त्तं॒ तां मे॑ जिन्वतम्। प्र॒जा सन्ध॑त्तं॒ तां मे॑ जिन्वतम्। प॒शून्त्सन्ध॑त्तं॒ तान्मे॑ जिन्वतम्। स्तु॒तो॑ऽसि॒ जन॑धाः। दे॒वास्त्वा॑ शुक्र॒पाः प्रण॑यन्तु॥१॥

%1.1.1.2
सु॒वीरा प्र॒जाः प्र॑ज॒नय॒न्परी॑हि। शु॒क्रः शु॒क्रशो॑चिषा। स्तु॒तो॑ऽसि॒ जन॑धाः। दे॒वास्त्वा॑ मन्थि॒पाः प्रण॑यन्तु। सु॒प्र॒जाः प्र॒जाः प्र॑ज॒नय॒न्परी॑हि। म॒न्थी म॒न्थिशो॑चिषा। स॒ञ्ज॒ग्मा॒नौ दि॒व आपृ॑थि॒व्यायु॑। सन्ध॑त्तं॒ तन्मे॑ जिन्वतम्। प्रा॒ण सन्ध॑त्तं॒ तं मे॑ जिन्वतम्। अ॒पा॒न सन्ध॑त्तं॒ तं मे॑ जिन्वतम्॥२॥

%1.1.1.3
व्या॒न सन्ध॑त्तं॒ तं मे॑ जिन्वतम्। चक्षु॒ सन्ध॑त्तं॒ तन्मे॑ जिन्वतम्। श्रोत्र॒ सन्ध॑त्तं॒ तन्मे॑ जिन्वतम्। मन॒ सन्ध॑त्तं॒ तन्मे॑ जिन्वतम्। वाच॒ सन्ध॑त्तं॒ तां मे॑ जिन्वतम्। आयु॑ स्थ॒ आयु॑र्मे धत्तम्। आयु॑र्य॒ज्ञाय॑ धत्तम्। आयु॑र्य॒ज्ञप॑तये धत्तम्। प्रा॒णः स्थ॑ प्रा॒णं मे॑ धत्तम्। प्रा॒णं य॒ज्ञाय॑ धत्तम्॥३॥

%1.1.1.4
प्रा॒णं य॒ज्ञप॑तये धत्तम्। चक्षु॑ स्थ॒श्चक्षु॑र्मे धत्तम्। चक्षु॑र्य॒ज्ञाय॑ धत्तम्। चक्षु॑र्य॒ज्ञप॑तये धत्तम्। श्रोत्र स्थ॒ श्रोत्रं॑ मे धत्तम्। श्रोत्रं॑ य॒ज्ञाय॑ धत्तम्। श्रोत्रं॑ य॒ज्ञप॑तये धत्तम्। तौ दे॑वौ शुक्रामन्थिनौ। क॒ल्पय॑तं॒ दैवी॒र्विश॑। क॒ल्पय॑तं॒ मानु॑षीः॥४॥

%1.1.1.5
इष॒मूर्ज॑म॒स्मासु॑ धत्तम्। प्रा॒णान्प॒शुषु॑। प्र॒जां मयि॑ च॒ यज॑माने च। निर॑स्त॒ शण्ड॑। निर॑स्तो॒ मर्क॑। अप॑नुत्तौ॒ शण्डा॒मर्कौ॑ स॒हामुना। शु॒क्रस्य॑ स॒मिद॑सि। म॒न्थिन॑ स॒मिद॑सि। स प्र॑थ॒मः संकृ॑तिर्वि॒श्वक॑र्मा। स प्र॑थ॒मो मि॒त्रो वरु॑णो अ॒ग्निः। स प्र॑थ॒मो बृह॒स्पति॑श्चिकि॒त्वान्। तस्मा॒ इन्द्रा॑य सु॒तमा जु॑होमि॥५॥\anuvakamend[न॒य॒न्त्व॒पा॒न सन्ध॑त्तं॒ तं मे॑ जिन्वतं प्रा॒णं य॒ज्ञाय॑ धत्तं॒ मानु॑षीर॒ग्निर्द्वे च॑॥ (ब्रह्म॑ क्ष॒त्रं तदिष॒मूर्ज र॒यिं पुष्टिं॑ प्र॒जां तां प॒शून्तान्त्सन्ध॑त्तं॒ तत्प्रा॒णम॑पा॒नं व्या॒नं तं चक्षु॒ श्रोत्रं॒ मन॒स्तद्वाचं॒ ताम्। इ॒षादि॒पञ्च॑के॒ वाचं॒ तां मे॑ प॒शून्त्सन्ध॑त्तं॒ तान्मे प्रा॒णादि॒त्रित॑ये॒ तं मे॒ऽन्यत्र॒ तन्मे)]

%1.1.2.1
कृत्ति॑कास्व॒ग्निमाद॑धीत। ए॒तद्वा अ॒ग्नेर्नक्ष॑त्रम्। यत्कृत्ति॑काः। स्वाया॑मै॒वैनं॑ दे॒वता॑यामा॒धाय॑। ब्र॒ह्म॒व॒र्च॒सी भ॑वति। मुखं॒ वा ए॒तन्नक्ष॑त्राणाम्। यत्कृत्ति॑काः। यः कृत्ति॑कास्व॒ग्निमा॑ध॒त्ते। मुख्य॑ ए॒व भ॑वति। अथो॒ खलु॑॥६॥

%1.1.2.2
अ॒ग्नि॒न॒क्ष॒त्रमित्यप॑चायन्ति। गृ॒हान् ह॒ दाहु॑को भवति। प्र॒जाप॑ती रोहि॒ण्याम॒ग्निम॑सृजत। तं दे॒वा रो॑हि॒ण्यामाद॑धत। ततो॒ वै ते सर्वा॒न्रोहा॑नरोहन्। तद्रो॑हि॒ण्यै रो॑हिणि॒त्वम्। यो रो॑हि॒ण्याम॒ग्निमा॑ध॒त्ते। ऋ॒ध्नोत्ये॒व। सर्वा॒न्रोहान्रोहति। दे॒वा वै भ॒द्राः सन्तो॒ऽग्निमाधि॑त्सन्त॥७॥

%1.1.2.3
तेषा॒मना॑हितो॒ऽग्निरासीत्। अथैभ्यो वा॒मं वस्वपाक्रामत्। ते पुन॑र्वस्वो॒राद॑धत। ततो॒ वै तान् वा॒मं वसू॒पाव॑र्तत। यः पु॒राऽभ॒द्रः सन्पापी॑या॒न्त्स्यात्। स पुन॑र्वस्वोर॒ग्निमाद॑धीत। पुन॑रे॒वैनं॑ वा॒मं वसू॒पाव॑र्तते। भ॒द्रो भ॑वति। यः का॒मये॑त॒ दानका॑मा मे प्र॒जाः स्यु॒रिति॑। स पूर्व॑यो॒ फल्गु॑न्योर॒ग्निमाद॑धीत॥८॥

%1.1.2.4
अ॒र्य॒म्णो वा ए॒तन्नक्ष॑त्रम्। यत्पूर्वे॒ फल्गु॑नी। अ॒र्य॒मेति॒ तमा॑हु॒र्यो ददा॑ति। दान॑कामा अस्मै प्र॒जा भ॑वन्ति। यः का॒मये॑त भ॒गी स्या॒मिति॑। स उत्त॑रयो॒ फल्गु॑न्योर॒ग्निमाद॑धीत। भग॑स्य॒ वा ए॒तन्नक्ष॑त्रम्। यदुत्त॑रे॒ फल्गु॑नी। भ॒ग्ये॑व भ॑वति। का॒ल॒क॒ञ्जा वै नामासु॑रा आसन्॥९॥

%1.1.2.5
ते सु॑व॒र्गाय॑ लो॒काया॒ग्निम॑चिन्वत। पुरु॑ष॒ इष्ट॑का॒मुपा॑दधा॒त्पुरु॑ष॒ इष्ट॑काम्। स इन्द्रो ब्राह्म॒णो ब्रुवा॑ण॒ इष्ट॑का॒मुपा॑धत्त। ए॒षा मे॑ चि॒त्रा नामेति॑। ते सु॑व॒र्गं लो॒कमा प्रारो॑हन्। स इन्द्र॒ इष्ट॑का॒मावृ॑हत्। तेऽवा॑कीर्यन्त। ये॑ऽवाकीर्यन्त। त ऊर्णा॒वभ॑योऽभवन्। द्वावुद॑पतताम्॥१०॥

%1.1.2.6
तौ दि॒व्यौ श्वाना॑वभवताम्। यो भ्रातृ॑व्यवा॒न्त्स्यात्। स चि॒त्राया॑म॒ग्निमाद॑धीत। अ॒व॒कीर्यै॒व भ्रातृ॑व्यान्। ओजो॒ बल॑मिन्द्रि॒यं वी॒र्य॑मा॒त्मन्ध॑त्ते। व॒सन्ता ब्राह्म॒णोऽग्निमाद॑धीत। व॒स॒न्तो वै ब्राह्म॒णस्य॒र्तुः। स्व ए॒वैन॑मृ॒तावा॒धाय॑। ब्र॒ह्म॒व॒र्च॒सी भ॑वति। मुखं॒ वा ए॒तदृ॑तू॒नाम्॥११॥

%1.1.2.7
यद्व॑स॒न्तः। यो व॒सन्ता॒ऽग्निमा॑ध॒त्ते। मुख्य॑ ए॒व भ॑वति। अथो॒ योनि॑मन्तमे॒वैनं॒ प्रजा॑त॒माध॑त्ते। ग्री॒ष्मे रा॑ज॒न्य॑ आद॑धीत। ग्री॒ष्मो वै रा॑ज॒न्य॑स्य॒र्तुः। स्व ए॒वैन॑मृ॒तावा॒धाय॑। इ॒न्द्रि॒या॒वी भ॑वति। श॒रदि॒ वैश्य॒ आद॑धीत। श॒रद्वै वैश्य॑स्य॒र्तुः॥१२॥

%1.1.2.8
स्व ए॒वैन॑मृ॒तावा॒धाय॑। प॒शु॒मान्भ॑वति। न पूर्व॑यो॒ फल्गु॑न्योर॒ग्निमाद॑धीत। ए॒षा वै ज॑घ॒न्या॑ रात्रि॑ संवत्स॒रस्य॑। यत्पूर्वे॒ फल्गु॑नी। पृ॒ष्टि॒त ए॒व सं॑वत्स॒रस्या॒ग्निमा॒धाय॑। पापी॑यान्भवति। उत्त॑रयो॒रा द॑धीत। ए॒षा वै प्र॑थ॒मा रात्रि॑ संवत्स॒रस्य॑। यदुत्त॑रे॒ फल्गु॑नी। मु॒ख॒त ए॒व सं॑वत्स॒रस्या॒ग्निमा॒धाय॑। वसी॑यान्भवति। अथो॒ खलु॑। य॒दैवैनं॑ य॒ज्ञ उ॑प॒नमेत्। अथाद॑धीत। सैवास्यर्द्धि॑॥१३॥\anuvakamend[खल्वा॑धित्सन्त॒ फल्गु॑न्योर॒ग्निमाद॑धीतासन्नपततामृतू॒नां वैश्य॑स्य॒र्तुरुत्त॑रे॒ फल्गु॑नी॒ षट्च॑]

%1.1.3.1
उद्ध॑न्ति। यदे॒वास्या॑ अमे॒ध्यम्। तदप॑हन्ति। अ॒पोऽवोक्षति॒ शान्त्यै। सिक॑ता॒ निव॑पति। ए॒तद्वा अ॒ग्नेर्वैश्वान॒रस्य॑ रू॒पम्। रू॒पेणै॒व वैश्वान॒रमव॑रुन्धे। ऊषां॒ निव॑पति। पुष्टि॒र्वा ए॒षा प्र॒जन॑नम्। यदूषा ॥१४॥

%1.1.3.2
पुष्ट्या॑मे॒व प्र॒जन॑ने॒ऽग्निमाध॑त्ते। अथो॑ सं॒ज्ञान॑ ए॒व। सं॒ज्ञान॒ ह्ये॑तत्प॑शू॒नाम्। यदूषा। द्यावा॑पृथि॒वी स॒हास्ताम्। ते वि॑य॒ती अ॑ब्रूताम्। अस्त्वे॒व नौ॑ स॒ह य॒ज्ञिय॒मिति॑। यद॒मुष्या॑ य॒ज्ञिय॒मासीत्। तद॒स्याम॑दधात्। त ऊषा॑ अभवन्॥१५॥

%1.1.3.3
यद॒स्या य॒ज्ञिय॒मासीत्। तद॒मुष्या॑मदधात्। तद॒दश्च॒न्द्रम॑सि कृ॒ष्णम्। ऊषान्नि॒वप॑न्न॒दो ध्या॑येत्। द्यावा॑पृथि॒व्योरे॒व य॒ज्ञिये॒ऽग्निमाध॑त्ते। अ॒ग्निर्दे॒वेभ्यो॒ निला॑यत। आ॒खू रू॒पं कृ॒त्वा। स पृ॑थि॒वीं प्रावि॑शत्। स ऊ॒तीः कु॑र्वा॒णः पृ॑थि॒वीमनु॒ सम॑चरत्। तदा॑खुकरी॒षम॑भवत्॥१६॥

%1.1.3.4
यदा॑खुकरी॒ष सं॑भा॒रो भव॑ति। यदे॒वास्य॒ तत्र॒ न्य॑क्तम्। तदे॒वाव॑रुन्धे। ऊर्जं॒ वा ए॒त रसं॑ पृथि॒व्या उ॑प॒दीका॒ उद्दि॑हन्ति। यद्व॒ल्मीकम्। यद्व॑ल्मीकव॒पा सं॑भा॒रो भव॑ति। ऊर्ज॑मे॒व रसं॑ पृथि॒व्या अव॑रुन्धे। अथो॒ श्रोत्र॑मे॒व। श्रोत्र॒ ह्ये॑तत्पृ॑थि॒व्याः। यद्व॒ल्मीक॑॥१७॥

%1.1.3.5
अब॑धिरो भवति। य ए॒वं वेद॑। प्र॒जाप॑तिः प्र॒जा अ॑सृजत। तासा॒मन्न॒मुपाक्षीयत। ताभ्य॒ सूद॒मुप॒प्राभि॑नत्। ततो॒ वै तासा॒मन्नं॒ नाक्षी॑यत। यस्य॒ सूद॑ सम्भा॒रो भव॑ति। नास्य॑ गृ॒हेऽन्नं॑ क्षीयते। आपो॒ वा इ॒दमग्रे॑ सलि॒लमा॑सीत्। तेन॑ प्र॒जाप॑तिरश्राम्यत्॥१८॥

%1.1.3.6
क॒थमि॒द स्या॒दिति॑। सो॑ऽपश्यत्पुष्करप॒र्णं तिष्ठ॑त्। सो॑ऽमन्यत। अस्ति॒ वै तत्। यस्मि॑न्नि॒दमधि॒ तिष्ठ॒तीति॑। स व॑रा॒हो रू॒पं कृ॒त्वोप॒ न्य॑मज्जत्। स पृ॑थि॒वीम॒ध आर्च्छत्। तस्या॑ उप॒हत्योद॑मज्जत्। तत्पु॑ष्करप॒र्णेऽप्रथयत्। यदप्र॑थयत्॥१९॥

%1.1.3.7
तत्पृ॑थि॒व्यै पृ॑थिवि॒त्वम्। अभू॒द्वा इ॒दमिति॑। तद्भूम्यै॑ भूमि॒त्वम्। तां दिशोऽनु॒ वात॒ सम॑वहत्। ता शर्क॑राभिरदृहत्। शं वै नो॑ऽभू॒दिति॑। तच्छर्क॑राणा शर्कर॒त्वम्। यद्व॑रा॒हवि॑हत सम्भा॒रो भव॑ति। अ॒स्यामे॒वाछ॑म्बट्कारम॒ग्निमाध॑त्ते। शर्क॑रा भवन्ति॒ धृत्यै॥२०॥

%1.1.3.8
अथो॑ शं॒त्वाय॑। सरे॑ता अ॒ग्निरा॒धेय॒ इत्या॑हुः। आपो॒ वरु॑णस्य॒ पत्न॑य आसन्। ता अ॒ग्निर॒भ्य॑ध्यायत्। ताः सम॑भवत्। तस्य॒ रेत॒ परा॑ऽपतत्। तद्धिर॑ण्यमभवत्। यद्धिर॑ण्यमु॒पास्य॑ति। सरे॑तसमे॒वाग्निमाध॑त्ते। पुरु॑ष॒ इन्न्वै स्वाद्रेत॑सो बीभत्सत॒ इत्या॑हुः॥२१॥

%1.1.3.9
उ॒त्त॒र॒त उपास्य॒त्यबी॑भत्सायै। अति॒ प्रय॑च्छति। आर्ति॑मे॒वाति॒ प्रय॑च्छति। अ॒ग्निर्दे॒वेभ्यो॒ निला॑यत। अश्वो॑ रू॒पं कृ॒त्वा। सोऽश्व॒त्थे सं॑वत्स॒रम॑तिष्ठत्। तद॑श्व॒त्थस्याश्वत्थ॒त्वम्। यदाश्व॑त्थः सम्भा॒रो भव॑ति। यदे॒वास्य॒ तत्र॒ न्य॑क्तम्। तदे॒वाव॑रुन्धे॥२२॥

%1.1.3.10
दे॒वा वा ऊर्जं॒ व्य॑भजन्त। तत॑ उदु॒म्बर॒ उद॑तिष्ठत्। ऊर्ग्वा उ॑दु॒म्बर॑। यदौदु॑म्बरः सम्भा॒रो भव॑ति। ऊर्ज॑मे॒वाव॑रुन्धे। तृ॒तीय॑स्यामि॒तो दि॒वि सोम॑ आसीत्। तं गा॑य॒त्र्याऽह॑रत्। तस्य॑ प॒र्णम॑च्छिद्यत। तत्प॒र्णो॑ऽभवत्। तत्प॒र्णस्य॑ पर्ण॒त्वम्॥२३॥

%1.1.3.11
यस्य॑ पर्ण॒मय॑ सम्भा॒रो भव॑ति। सो॒म॒पी॒थमे॒वाव॑रुन्धे। दे॒वा वै ब्रह्म॑न्नवदन्त। तत्प॒र्ण उपा॑शृणोत्। सु॒श्रवा॒ वै नाम॑। यत्प॑र्ण॒मय॑ सम्भा॒रो भव॑ति। ब्र॒ह्म॒व॒र्च॒समे॒वाव॑ रुन्धे। प्र॒जाप॑तिर॒ग्निम॑सृजत। सो॑ऽबिभे॒त्प्र मा॑ धक्ष्य॒तीति॑। त श॒म्या॑ऽशमयत्॥२४॥

%1.1.3.12
तच्छ॒म्यै॑ शमि॒त्वम्। यच्छ॑मी॒मय॑ सम्भा॒रो भव॑ति। शान्त्या॒ अप्र॑दाहाय। अ॒ग्नेः सृ॒ष्टस्य॑ य॒तः। विक॑ङ्कतं॒ भा आर्च्छत्। यद्वैक॑ङ्कतः सम्भा॒रो भव॑ति। भा ए॒वाव॑ रुन्धे। सहृ॑दयो॒ऽग्निरा॒धेय॒ इत्या॑हुः। म॒रुतो॒ऽद्भिर॒ग्निम॑तमयन्। तस्य॑ ता॒न्तस्य॒ हृद॑य॒माच्छि॑न्दन्। साऽशनि॑रभवत्। यद॒शनि॑हतस्य वृ॒क्षस्य॑ सम्भा॒रो भव॑ति। सहृ॑दयमे॒वाग्निमा ध॑त्ते॥२५॥\anuvakamend[ऊषा॑ अभवन्नभवद्व॒ल्मीकोऽश्राम्य॒दप्र॑थय॒द्धृत्यै॑ बीभत्सत॒ इत्या॑हू रुन्धे पर्ण॒त्वम॑शमयदच्छिन्द॒ स्त्रीणि॑ च]

%1.1.4.1
द्वा॒द॒शसु॑ विक्रा॒मेष्व॒ग्निमा द॑धीत। द्वाद॑श॒ मासा संवत्स॒रः। सं॒व॒त्स॒रादे॒वैन॑मव॒रुद्ध्या ध॑त्ते। यद्द्वा॑द॒शसु॑ विक्रा॒मेष्वा॒ दधी॑त। परि॑मित॒मव॑ रुन्धीत। चक्षु॑र्निमित॒ आद॑धीत। इय॒द्द्वाद॑श विक्रा॒मा (३) इति॑। परि॑मितं चै॒वाप॑रिमितं॒ चाव॑ रुन्धे। अनृ॑तं॒ वै वा॒चा व॑दति। अनृ॑तं॒ मन॑सा ध्यायति॥२६॥

%1.1.4.2
चक्षु॒र्वै स॒त्यम्। अद्रा(३)गित्या॑ह। अद॑र्\mbox{}श॒मिति॑। तत्स॒त्यम्। यश्चक्षु॑र्निमिते॒ऽग्निमा॑ध॒त्ते। स॒त्य ए॒वैन॒मा ध॑त्ते। तस्मा॒दाहि॑ताग्नि॒र्नानृ॑तं वदेत्। नास्य॑ ब्राह्म॒णोऽनाश्वान्गृ॒हे व॑सेत्। स॒त्ये ह्य॑स्या॒ग्निराहि॑तः। आ॒ग्ने॒यी वै रात्रि॑॥२७॥

%1.1.4.3
आ॒ग्ने॒याः प॒शव॑। ऐ॒न्द्रमह॑। नक्तं॒ गार्\mbox{}ह॑पत्य॒मा द॑धाति। प॒शूने॒वाव॑ रुन्धे। दिवा॑ऽऽहव॒नीयम्। इ॒न्द्रि॒यमे॒वाव॑ रुन्धे। अ॒र्धोदि॑ते॒ सूर्य॑ आहव॒नीय॒मा द॑धाति। ए॒तस्मि॒न्वै लो॒के प्र॒जाप॑तिः प्र॒जा अ॑सृजत। प्र॒जा ए॒व तद्यज॑मानः सृजते। अथो॑ भू॒तं चै॒व भ॑वि॒ष्यच्चाव॑ रुन्धे॥२८॥

%1.1.4.4
इडा॒ वै मा॑न॒वी य॑ज्ञानूका॒शिन्या॑सीत्। साऽशृ॑णोत्। असु॑रा अ॒ग्निमाद॑धत॒ इति॑। तद॑गच्छत्। त आ॑हव॒नीय॒मग्र॒ आद॑धत। अथ॒ गार्\mbox{}ह॑पत्यम्। अथान्वाहार्य॒पच॑नम्। साऽब्र॑वीत्। प्र॒तीच्ये॑षा॒ श्रीर॑गात्। भ॒द्रा भू॒त्वा परा॑ भविष्य॒न्तीति॑॥२९॥

%1.1.4.5
यस्यै॒वम॒ग्निरा॑धी॒यते। प्र॒तीच्य॑स्य॒ श्रीरे॑ति। भ॒द्रो भू॒त्वा परा॑भवति। साऽशृ॑णोत्। दे॒वा अ॒ग्निमाद॑धत॒ इति॑। तद॑गच्छत्। तेऽन्वाहार्य॒पच॑न॒मग्र॒ आद॑धत। अथ॒ गार्\mbox{}ह॑पत्यम्। अथा॑हव॒नीयम्। साऽब्र॑वीत्॥३०॥

%1.1.4.6
प्राच्ये॑षा॒ श्रीर॑गात्। भ॒द्रा भू॒त्वा सु॑व॒र्गंल्लो॒कमेष्यन्ति। प्र॒जां तु न वेत्स्यन्त॒ इति॑। यस्यै॒वम॒ग्निरा॑धी॒यते। प्राच्य॑स्य॒ श्रीरे॑ति। भ॒द्रो भू॒त्वा सु॑व॒र्गं लो॒कमे॑ति। प्र॒जां तु न वि॑न्दते। साऽब्र॑वी॒दिडा॒ मनुम्। तथा॒ वा अ॒हं तवा॒ग्निमाधास्यामि। यथा॒ प्र प्र॒जया॑ प॒शुभि॑र्मिथु॒नैर्ज॑नि॒ष्यसे॥३१॥

%1.1.4.7
प्रत्य॒स्मिल्लोँ॒के स्था॒स्यसि॑। अ॒भि सु॑व॒र्गं लो॒कं जे॒ष्यसीति॑। गार्\mbox{}ह॑पत्य॒मग्र॒ आद॑धात्। गार्\mbox{}ह॑पत्यं॒ वा अनु॑ प्र॒जाः प॒शव॒ प्रजा॑यन्ते। गार्\mbox{}ह॑पत्येनै॒वास्मै प्र॒जां प॒शून्प्राज॑नयत्। अथान्वाहार्य॒पच॑नम्। ति॒र्यङ्ङि॑व॒ वा अ॒यं लो॒कः। अ॒स्मिन्नै॒व तेन॑ लो॒के प्रत्य॑तिष्ठत्। अथा॑हव॒नीयम्। तेनै॒व सु॑व॒र्गं लो॒कम॒भ्य॑जयत्॥३२॥

%1.1.4.8
यस्यै॒वम॒ग्निरा॑धी॒यते। प्र प्र॒जया॑ प॒शुभि॑र्मिथु॒नैर्जा॑यते। प्रत्य॒स्मिल्लोँ॒के ति॑ष्ठति। अ॒भि सु॑व॒र्गं लो॒कं ज॑यति। यस्य॒ वा अय॑थादेवतम॒ग्निरा॑धी॒यते। आ दे॒वताभ्यो वृश्च्यते। पापी॑यान्भवति। यस्य॑ यथादेव॒तम्। न दे॒वताभ्य॒ आवृ॑श्च्यते। वसी॑यान्भवति॥३३॥ भृगू॑णां॒ त्वाऽङ्गि॑रसां व्रतपते व्र॒तेनाद॑धा॒मीति॑ भृग्वङ्गि॒रसा॒माद॑ध्यात्। आ॒दि॒त्यानां त्वा दे॒वानां व्रतपते व्र॒तेनाद॑धा॒मीत्य॒न्यासां॒ ब्राह्म॑णीनां प्र॒जानाम्। वरु॑णस्य त्वा॒ राज्ञो व्रतपते व्र॒तेनाद॑धा॒मीति॒ राज्ञ॑। इन्द्र॑स्य त्वेन्द्रि॒येण॑ व्रतपते व्र॒तेनाद॑धा॒मीति॑ राज॒न्य॑स्य। मनोस्त्वा ग्राम॒ण्यो व्रतपते व्र॒तेनाद॑धा॒मिति॒ वैश्य॑स्य। ऋ॒भू॒णां त्वा॑ दे॒वानां व्रतपते व्र॒तेनाद॑धा॒मीति॑ रथका॒रस्य॑। य॒था॒दे॒व॒तम॒ग्निराधी॑यते। न दे॒वताभ्य॒ आवृ॑श्च्यते। वसी॑यान्भवति॥३४॥\anuvakamend[ध्या॒य॒ति॒ वै रात्रि॒श्चाव॑रुन्धे भविष्य॒न्तीत्य॑ब्रवीज्जनि॒ष्यसे॑ऽजय॒द्वसी॑यान्भवति॒ नव॑ च]

%1.1.5.1
प्र॒जाप॑तिर्वा॒चः स॒त्यम॑पश्यत्। तेना॒ग्निमाध॑त्त। तेन॒ वै स आर्ध्नोत्। भूर्भुव॒ सुव॒रित्या॑ह। ए॒तद्वै वा॒चः स॒त्यम्। य ए॒तेना॒ग्निमा॑ध॒त्ते। ऋ॒ध्नोत्ये॒व। अथो॑ स॒त्यप्रा॑शूरे॒व भ॑वति। अथो॒ य ए॒वं वि॒द्वान॑भि॒चर॑ति। स्तृ॒णु॒त ए॒वैनम्॥३५॥

%1.1.5.2
भूरित्या॑ह। प्र॒जा ए॒व तद्यज॑मानः सृजते। भुव॒ इत्या॑ह। अ॒स्मिन्ने॒व लो॒के प्रति॑तिष्ठति। सुव॒रित्या॑ह। सु॒व॒र्ग ए॒व लो॒के प्रति॑तिष्ठति। त्रि॒भिर॒क्षरै॒र्गार्\mbox{}ह॑पत्य॒मा द॑धाति। त्रय॑ इ॒मे लो॒काः। ए॒ष्वे॑वैनं॑ लो॒केषु॒ प्रति॑ष्ठित॒माध॑त्ते। सर्वै प॒ञ्चभि॑राहव॒नीयम्॥३६॥

%1.1.5.3
सु॒व॒र्गाय॒ वा ए॒ष लो॒कायाधी॑यते। यदा॑हव॒नीय॑। सु॒व॒र्ग ए॒वास्मै॑ लो॒के वा॒चः स॒त्य सर्व॑माप्नोति। त्रि॒भिर्गार्\mbox{}ह॑पत्य॒मा द॑धाति। प॒ञ्चभि॑राहव॒नीयम्। अ॒ष्टौ संप॑द्यन्ते। अ॒ष्टाक्ष॑रा गाय॒त्री। गा॒य॒त्रोऽग्निः। यावा॑ने॒वाग्निः। तमाध॑त्ते॥३७॥

%1.1.5.4
प्र॒जाप॑तिः प्र॒जा अ॑सृजत। ता अ॑स्मात्सृ॒ष्टाः परा॑चीरायन्। ताभ्यो॒ ज्योति॒रुद॑गृह्णात्। तं ज्योति॒ पश्य॑न्तीः प्र॒जा अ॒भि स॒माव॑र्तन्त। उ॒परी॑वा॒ग्निमुद्गृ॑ह्णीयादु॒द्धर\sn{}। ज्योति॑रे॒व पश्य॑न्तीः प्र॒जा यज॑मानम॒भि स॒माव॑र्तन्ते। प्र॒जाप॑ते॒रक्ष्य॑श्वयत्। तत्परा॑ऽपतत्। तदश्वो॑ऽभवत्। तदश्व॑स्याश्व॒त्वम्॥३८॥

%1.1.5.5
ए॒ष वै प्र॒जाप॑तिः। यद॒ग्निः। प्रा॒जा॒प॒त्योऽश्व॑। यदश्वं॑ पु॒रस्ता॒न्नय॑ति। स्वमे॒व चक्षु॒ पश्य॑न्प्र॒जाप॑ति॒रनूदे॑ति। व॒ज्री वा ए॒षः। यदश्व॑। यदश्वं॑ पु॒रस्ता॒न्नय॑ति। जा॒ताने॒व भ्रातृ॑व्या॒न्प्रणु॑दते। पुन॒रा व॑र्तयति॥३९॥

%1.1.5.6
ज॒नि॒ष्यमा॑णाने॒व प्रति॑नुदते। न्या॑हव॒नीयो॒ गार्\mbox{}ह॑पत्य\-मकामयत। निगार्\mbox{}ह॑पत्य आहव॒नीयम्। तौ वि॒भाजं॒ नाश॑क्नोत्। सोऽश्व॑ पूर्व॒वाड्भू॒त्वा। प्राञ्चं॒ पूर्व॒मुद॑वहत्। तत्पूर्व॒वाह॑ पूर्ववा॒ट्त्वम्। यदश्वं॑ पु॒रस्ता॒न्नय॑ति। विभ॑क्तिरे॒वैन॑यो॒ सा। अथो॒ नाना॑वीर्यावे॒वैनौ॑ कुरुते॥४०॥

%1.1.5.7
यदु॒पर्यु॑परि॒ शिरो॒ हरेत्। प्रा॒णान्‌विच्छि॑न्द्यात्। अ॒धो॑ऽध॒ शिरो॑ हरति। प्रा॒णानां गोपी॒थाय॑। इय॒त्यग्रे॑ हरति। अथेय॒त्यथेय॑ति। त्रय॑ इ॒मे लो॒काः। ए॒ष्वे॑वैनं॑ लो॒केषु॒ प्रति॑ष्ठित॒माध॑त्ते। प्र॒जाप॑तिर॒ग्निम॑सृजत। सो॑ऽबिभे॒त्प्र मा॑ धक्ष्य॒तीति॑॥४१॥

%1.1.5.8
तस्य॑ त्रे॒धा म॑हि॒मानं॒ व्यौ॑हत्। शान्त्या॒ अप्र॑दाहाय। यत्रे॒धाऽग्निरा॑धी॒यते। म॒हि॒मान॑मे॒वास्य॒ तद्व्यू॑हति। शान्त्या॒ अप्र॑दाहाय। पुन॒रा व॑र्तयति। म॒हि॒मान॑मे॒वास्य॒ संद॑धाति। प॒शुर्वा ए॒षः। यदश्व॑। ए॒ष रु॒द्रः॥४२॥

%1.1.5.9
यद॒ग्निः। यदश्व॑स्य प॒देऽग्निमा॑द॒ध्यात्। रु॒द्राय॑ प॒शूनपि॑दध्यात्। अ॒प॒शुर्यज॑मानः स्यात्। यन्नाक्र॒मयेत्। अन॑वरुद्धा अस्य प॒शव॑ स्युः। पा॒र्श्व॒त आक्र॑मयेत्। यथाऽऽहि॑तस्या॒ग्नेरङ्गा॑रा अभ्यव॒वर्ते॑रन्। अव॑रुद्धा अस्य प॒शवो॒ भव॑न्ति। न रु॒द्रायापि॑दधाति॥४३॥

%1.1.5.10
त्रीणि॑ ह॒वीषि॒ निर्व॑पति। वि॒राज॑ ए॒व विक्रान्तं॒ यज॑मा॒नोऽनु॒ विक्र॑मते। अ॒ग्नये॒ पव॑मानाय। अ॒ग्नये॑ पाव॒काय॑। अ॒ग्नये॒ शुच॑ये। यद॒ग्नये॒ पव॑मानाय नि॒र्वप॑ति। पु॒नात्ये॒वैनम्। यद॒ग्नये॑ पाव॒काय॑। पू॒त ए॒वास्मि॑न्न॒न्नाद्यं॑ दधाति। यद॒ग्नये॒ शुच॑ये। ब्र॒ह्म॒व॒र्च॒समे॒वास्मि॑न्नु॒परि॑ष्टाद्दधाति॥४४॥\anuvakamend[ए॒न॒मा॒ह॒व॒नीयं॑ धत्तेऽश्व॒त्वं व॑र्तयति कुरुत॒ इति॑ रु॒द्रो द॑धाति॒ य॒दग्नये॒ शुच॑य॒ एकं॑ च]

%1.1.6.1
दे॒वा॒सु॒राः संय॑त्ता आसन्। ते दे॒वा वि॑ज॒यमु॑प॒यन्त॑। अ॒ग्नौ वा॒मं वसु॒ सं न्य॑दधत। इ॒दमु॑ नो भविष्यति। यदि॑ नो जे॒ष्यन्तीति॑। तद॒ग्निर्नोत्सह॑मशक्नोत्। तत् त्रे॒धा विन्य॑दधात्। प॒शुषु॒ तृती॑यम्। अ॒प्सु तृती॑यम्। आ॒दि॒त्ये तृती॑यम्॥४५॥

%1.1.6.2
तद्दे॒वा वि॒जित्य॑। पुन॒रवा॑रुरुत्सन्त। तेऽग्नये॒ पव॑मानाय पुरो॒डाश॑म॒ष्टाक॑पालं॒ निर॑वपन्। प॒शवो॒ वा अ॒ग्निः पव॑मानः। यदे॒व प॒शुष्वासीत्। तत्तेनावा॑रुन्धत। तेऽग्नये॑ पाव॒काय॑। आपो॒ वा अ॒ग्निः पा॑व॒कः। यदे॒वाप्स्वासीत्। तत्तेनावा॑रुन्धत॥४६॥

%1.1.6.3
तेऽग्नये॒ शुच॑ये। अ॒सौ वा आ॑दि॒त्योऽग्निः शुचि॑। यदे॒वादि॒त्य आसीत्। तत्तेनावा॑रुन्धत। ब्र॒ह्म॒वा॒दिनो॑ वदन्ति। त॒नुवो॒ वावैता अ॑ग्न्या॒धेय॑स्य। आ॒ग्ने॒यो वा अ॒ष्टाक॑पालोऽग्न्या॒धेय॒मिति॑। यत्तन्नि॒र्वपेत्। नैतानि॑। यथा॒ऽऽत्मा स्यात्॥४७॥

%1.1.6.4
नाङ्गा॑नि। ता॒दृगे॒व तत्। यदे॒तानि॑ नि॒र्वपेत्। न तम्। यथाऽङ्गा॑नि॒ स्युः। नात्मा। ता॒दृगे॒व तत्। उ॒भया॑नि स॒ह नि॒रुप्या॑णि। य॒ज्ञस्य॑ सात्म॒त्वाय॑। उ॒भयं॒ वा ए॒तस्येन्द्रि॒यं वी॒र्य॑माप्यते॥४८॥

%1.1.6.5
योऽग्निमा॑ध॒त्ते। ऐ॒न्द्रा॒ग्नमेका॑दशकपाल॒मनु॒ निर्व॑पेत्। आ॒दि॒त्यं च॒रुम्। इ॒न्द्रा॒ग्नी वै दे॒वाना॒मया॑तमायामानौ। ये ए॒व दे॒वते॒ अया॑तयाम्नी। ताभ्या॑मे॒वास्मा॑ इन्द्रि॒यं वी॒र्य॑मव॑ रुन्धे। आ॒दि॒त्यो भ॑वति। इ॒यं वा अदि॑तिः। अ॒स्यामे॒व प्रति॑तिष्ठति। धे॒न्वै वा ए॒तद्रेत॑॥४९॥

%1.1.6.6
यदाज्यम्। अ॒न॒डुह॑स्तण्डु॒लाः। मि॒थु॒नमे॒वाव॑रुन्धे। घृ॒ते भ॑वति। य॒ज्ञस्यालूक्षान्तत्वाय। च॒त्वार॑ आर्\mbox{}षे॒याः प्राश्ञ॑न्ति। दि॒शामे॒व ज्योति॑षि जुहोति। प॒शवो॒ वा ए॒तानि॑ ह॒वीषि॑। ए॒ष रु॒द्रः। यद॒ग्निः॥५०॥

%1.1.6.7
यत्स॒द्य ए॒तानि॑ ह॒वीषि॑ नि॒र्वपेत्। रु॒द्राय॑ प॒शूनपि॑ दध्यात्। अ॒प॒शुर्यज॑मानः स्यात्। यन्नानु॑नि॒र्वपेत्। अन॑वरुद्धा अस्य प॒शव॑ स्युः। द्वा॒द॒शसु॒ रात्री॒ष्वनु॒ निर्व॑पेत्। सं॒व॒त्स॒रप्र॑तिमा॒ वै द्वाद॑श॒ रात्र॑यः। सं॒व॒त्स॒रेणै॒वास्मै॑ रु॒द्र श॑मयि॒त्वा। प॒शूनव॑रुन्धे। यदेक॑मेकमे॒तानि॑ ह॒वीषि॑ नि॒र्वपेत्॥५१॥

%1.1.6.8
यथा॒ त्रीण्या॒वप॑नानि पू॒रयेत्। ता॒दृक्तत्। न प्र॒जन॑न॒\-मुच्छिषेत्। एकं॑ नि॒रुप्य॑। उत्त॑रे॒ सम॑स्येत्। तृ॒तीय॑मे॒वास्मै॑ लो॒कमुच्छिषति प्र॒जन॑नाय। तं प्र॒जया॑ प॒शुभि॒रनु॒ प्रजा॑यते। अथो॑ य॒ज्ञस्यै॒वैषाऽभिक्रान्तिः। र॒थ॒च॒क्रं प्रव॑र्तयति। म॒नु॒ष्य॒र॒थेनै॒व दे॑वर॒थं प्र॒त्यव॑रोहति॥५२॥

%1.1.6.9
ब्र॒ह्म॒वा॒दिनो॑ वदन्ति। हो॒त॒व्य॑मग्निहो॒त्राँ(३) न हो॑त॒व्या(३) मिति॑। यद्यजु॑षा जुहु॒यात्। अय॑थापूर्व॒माहु॑ती जुहुयात्। यन्न जु॑हु॒यात्। अ॒ग्निः परा॑ भवेत्। तू॒ष्णीमे॒व हो॑त॒व्यम्। य॒था॒पू॒र्वमाहु॑ती जु॒होति॑। नाग्निः परा॑भवति। अ॒ग्नीधे॑ ददाति॥५३॥

%1.1.6.10
अ॒ग्निमु॑खाने॒वर्तून्प्री॑णाति। उ॒प॒बर्\mbox{}ह॑णं ददाति। रू॒पाणा॒मव॑\-रुद्ध्यै। अश्वं॑ ब्र॒ह्मणे। इ॒न्द्रि॒यमे॒वाव॑रुन्धे। धे॒नु होत्रे। आ॒शिष॑ ए॒वाव॑रुन्धे। अ॒न॒ड्वाह॑मध्व॒र्यवे। वह्नि॒र्वा अ॑न॒ड्वान्। वह्नि॑रध्व॒र्युः॥५४॥

%1.1.6.11
वह्नि॑नै॒व वह्नि॑ य॒ज्ञस्याव॑रुन्धे। मि॒थु॒नौ गावौ॑ ददाति। मि॒थु॒नस्याव॑रुद्ध्यै। वासो॑ ददाति। स॒र्व॒दे॒व॒त्यं॑ वै वास॑। सर्वा॑ ए॒व दे॒वता प्रीणाति। आ द्वा॑द॒शभ्यो॑ ददाति। द्वाद॑श॒ मासा संवत्स॒रः। सं॒व॒त्स॒र ए॒व प्रति॑तिष्ठति। काम॑मू॒र्ध्वं देयम्। अप॑रिमित॒स्याव॑रुद्ध्यै॥५५॥\anuvakamend[आ॒दि॒त्ये तृती॑यम॒प्स्वासी॒त्तत्तेनावा॑रुन्धत॒ स्यादाप्यते॒ रेतो॒ऽग्निरेक॑मेकमे॒तानि॑ ह॒वीषि॑ नि॒र्वपेत्प्र॒त्यव॑रोहति ददात्यध्व॒र्युर्देय॒मेकं॑ च]

%1.1.7.1
घ॒र्मः शिर॒स्तद॒यम॒ग्निः। संप्रि॑यः प॒शुभि॑र्भुवत्। छ॒र्दिस्तो॒काय॒ तन॑याय यच्छ। वात॑ प्रा॒णस्तद॒यम॒ग्निः। संप्रि॑यः प॒शुभि॑र्भुवत्। स्व॒दि॒तं तो॒काय॒ तन॑याय पि॒तुं प॑च। प्राची॒मनु॑ प्र॒दिशं॒ प्रेहि॑ वि॒द्वान्। अ॒ग्नेर॑ग्ने पु॒रो अ॑ग्निर्भवे॒ह। विश्वा॒ आशा॒ दीद्या॑नो॒ विभा॑हि। ऊर्जं॑ नो धेहि द्वि॒पदे॒ चतु॑ष्पदे॥५६॥

%1.1.7.2
अ॒र्कश्चक्षु॒स्तद॒सौ सूर्य॒स्तद॒यम॒ग्निः। संप्रि॑यः प॒शुभि॑र्भुवत्। यत्ते॑ शुक्र शु॒क्रं वर्च॑ शु॒क्रा त॒नूः। शु॒क्रं ज्योति॒रज॑स्रम्। तेन॑ मे दीदिहि॒ तेन॒ त्वाऽऽद॑धे। अ॒ग्निनाऽग्ने॒ ब्रह्म॑णा। आ॒न॒शे व्या॑नशे॒ सर्व॒मायु॒र्व्या॑नशे। ये ते॑ अग्ने शि॒वे त॒नुवौ। वि॒राट्च॑ स्व॒राट्च॑। ते मावि॑शतां॒ ते मा॑ जिन्वताम्॥५७॥

%1.1.7.3
ये ते॑ अग्ने शि॒वे त॒नुवौ। सं॒राट्चा॑भि॒भूश्च॑। ते मावि॑शतां॒ ते मा॑ जिन्वताम्। ये ते॑ अग्ने शि॒वे त॒नुवौ। वि॒भूश्च॑ परि॒भूश्च॑। ते मा वि॑शतां॒ ते मा॑ जिन्वताम्। ये ते॑ अग्ने शि॒वे त॒नुवौ। प्र॒भ्वी च॒ प्रभू॑तिश्च। ते मा वि॑शतां॒ ते मा॑ जिन्वताम्। यास्ते॑ अग्ने शि॒वास्त॒नुव॑। ताभि॒स्त्वाऽऽद॑धे। यास्ते॑ अग्ने घो॒रास्त॒नुव॑। ताभि॑र॒मुं ग॑च्छ॥५८॥\anuvakamend[चतु॑ष्पदे जिन्वतां त॒नुव॒स्त्रीणि॑ च]

%1.1.8.1
इ॒मे वा ए॒ते लो॒का अ॒ग्नय॑। ते यदव्या॑वृत्ता आधी॒येर\sn{}। शो॒चये॑यु॒र्यज॑मानम्। घ॒र्मः शिर॒ इति॒ गार्\mbox{}ह॑पत्य॒मा द॑धाति। वात॑ प्रा॒ण इत्य॑न्वाहार्य॒पच॑नम्। अ॒र्कश्चक्षु॒रित्या॑हव॒नीयम्। तेनै॒वैना॒न्व्याव॑र्तयति। तथा॒ न शो॑चयन्ति॒ यज॑मानम्। र॒थ॒न्त॒रम॒भिगा॑यते॒ गार्\mbox{}ह॑पत्य आधी॒यमा॑ने। राथ॑न्तरो॒ वा अ॒यं लो॒कः॥५९॥

%1.1.8.2
अ॒स्मिन्ने॒वैनं॑ लो॒के प्रति॑ष्ठित॒मा ध॑त्ते। वा॒म॒दे॒व्यम॒भिगा॑यत उद्ध्रि॒यमा॑णे। अ॒न्तरि॑क्षं॒ वै वा॑मदे॒व्यम्। अ॒न्तरि॑क्ष ए॒वैनं॒ प्रति॑ष्ठित॒माध॑त्ते। अथो॒ शान्ति॒र्वै वा॑मदे॒व्यम्। शा॒न्तमे॒वैनं॑ पश॒व्य॑मुद्ध॑रते। बृ॒हद॒भिगा॑यत आहव॒नीय॑ आधी॒यमा॑ने। बार्\mbox{}ह॑तो॒ वा अ॒सौ लो॒कः। अ॒मुष्मि॑न्ने॒वैनं॑ लो॒के प्रति॑ष्ठित॒माध॑त्ते। प्र॒जाप॑तिर॒ग्निम॑सृजत॥६०॥

%1.1.8.3
सोऽश्वो॒ऽवारो॑ भू॒त्वा परा॑ङैत्। तं वा॑रव॒न्तीये॑नावारयत। तद्वा॑रव॒न्तीय॑स्य वारवन्तीय॒त्वम्। श्यै॒तेन॑ श्ये॒ती अ॑कुरुत। तच्छ्यै॒तस्य॑ श्यैत॒त्वम्। यद्वा॑रव॒न्तीय॑मभि॒ गाय॑ते। वा॒र॒यि॒त्वैवैनं॒ प्रति॑ष्ठित॒मा ध॑त्ते। श्यै॒तेन॑ श्ये॒ती कु॑रुते। घ॒र्मः शिर॒ इति॒ गार्\mbox{}ह॑पत्य॒माद॑धाति। सशी॑र्\mbox{}षाणमे॒वैन॒मा ध॑त्ते॥६१॥

%1.1.8.4
उपै॑न॒मुत्त॑रो य॒ज्ञो न॑मति। रु॒द्रो वा ए॒षः। यद॒ग्निः। स आ॑धी॒यमा॑न ईश्व॒रो यज॑मानस्य प॒शून् हिसि॑तोः। संप्रि॑यः प॒शुभि॑र्भुव॒दित्या॑ह। प॒शुभि॑रे॒वैन॒ संप्रि॑यं करोति। प॒शू॒नामहिसायै। छ॒र्दिस्तो॒काय॒ तन॑याय य॒च्छेत्या॑ह। आ॒शिष॑मे॒वैतामा शास्ते। वात॑ प्रा॒ण इत्य॑न्वाहार्य॒पच॑नम्॥६२॥

%1.1.8.5
सप्रा॑णमे॒वैन॒मा ध॑त्ते। स्व॒दि॒तं तो॒काय॒ तन॑याय पि॒तुं प॒चेत्या॑ह। अन्न॑मे॒वास्मै स्वदयति। प्राची॒मनु॑ प्र॒दिशं॒ प्रेहि॑ वि॒द्वानित्या॑ह। विभ॑क्तिरे॒वैन॑यो॒ सा। अथो॒ नाना॑वीर्यावे॒वैनौ॑ कुरुते। ऊर्जं॑ नो धेहि द्वि॒पदे॒ चतु॑ष्पद॒ इत्या॑ह। आ॒शिष॑मे॒वैतामा शास्ते। अ॒र्कश्चक्षु॒रित्या॑हव॒नीयम्। अ॒र्को वै दे॒वाना॒मन्नम्॥६३॥

%1.1.8.6
अन्न॑मे॒वाव॑ रुन्धे। तेन॑ मे दीदि॒हीत्या॑ह। समि॑न्ध ए॒वैनम्। आ॒न॒शे व्या॑नश॒ इति॒ त्रिरुदि॑ङ्गयति। त्रय॑ इ॒मे लो॒काः। ए॒ष्वे॑वैनं॑ लो॒केषु॒ प्रति॑ष्ठित॒मा ध॑त्ते। तत्तथा॒ न का॒र्यम्। वीङ्गि॑त॒मप्र॑तिष्ठित॒मा द॑धीत। उ॒द्धृत्यै॒वाधाया॑भि॒मन्त्रिय॑। अवीङ्गितमे॒वैनं॒ प्रति॑ष्ठित॒माध॑त्ते। वि॒राट्च॑ स्व॒राट्च॒ यास्ते॑ अग्ने शि॒वास्त॒नुव॒स्ताभि॒स्त्वाऽऽद॑ध॒ इत्या॑ह। ए॒ता वा अ॒ग्नेः शि॒वास्त॒नुव॑। ताभि॑रे॒वैन॒ सम॑र्धयति। यास्ते॑ अग्ने घो॒रास्त॒नुव॒स्ताभि॑र॒मुं ग॒च्छेति॑ ब्रूया॒द्यं द्वि॒ष्यात्। ताभि॑रे॒वैनं॒ परा॑भावयति॥६४॥\anuvakamend[लो॒को॑ऽसृजतैन॒माध॑त्तेऽन्वाहार्य॒पच॑नं दे॒वाना॒मन्न॑मेनं॒ प्रति॑ष्ठित॒माध॑त्ते॒ पञ्च॑ च]

%1.1.9.1
श॒मी॒ग॒र्भाद॒ग्निं म॑न्थति। ए॒षा वा अ॒ग्नेर्य॒ज्ञिया॑ त॒नूः। तामे॒वास्मै॑ जनयति। अदि॑तिः पु॒त्रका॑मा। सा॒ध्येभ्यो॑ दे॒वेभ्यो ब्रह्मौद॒नम॑पचत्। तस्या॑ उ॒च्छेष॑णमददुः। तत्प्राश्ञात्। सा रेतो॑ऽधत्त। तस्यै॑ धा॒ता चार्य॒मा चा॑जायेताम्। सा द्वि॒तीय॑मपचत्॥६५॥

%1.1.9.2
तस्या॑ उ॒च्छेष॑णमददुः। तत्प्राश्ञात्। सा रेतो॑ऽधत्त। तस्यै॑ मि॒त्रश्च॒ वरु॑णश्चाजायेताम्। सा तृ॒तीय॑मपचत्। तस्या॑ उ॒च्छेष॑णमददुः। तत्प्राश्ञात्। सा रेतो॑ऽधत्त। तस्या॒ अश॑श्च॒ भग॑श्चाजायेताम्। सा च॑तु॒र्थम॑पचत्॥६६॥

%1.1.9.3
तस्या॑ उ॒च्छेष॑णमददुः। तत्प्राश्ञात्। सा रेतो॑ऽधत्त। तस्या॒ इन्द्र॑श्च॒ विव॑स्वाश्चाजायेताम्। ब्र॒ह्मौ॒द॒नं प॑चति। रेत॑ ए॒व तद्द॑धाति। प्राश्ञ॑न्ति ब्राह्म॒णा ओ॑द॒नम्। यदाज्य॑मु॒च्छिष्य॑ते। तेन॑ स॒मिधो॒ऽभ्यज्या द॑धाति। उ॒च्छेष॑णा॒द्वा अदि॑ती॒ रेतो॑ऽधत्त॥६७॥

%1.1.9.4
उ॒च्छेष॑णादे॒व तद्रेतो॑ धत्ते। अस्थि॒ वा ए॒तत्। यत्स॒मिध॑। ए॒तद्रेत॑। यदाज्यम्। यदाज्ये॑न स॒मिधो॒ऽभ्यज्या॒दधा॑ति। अस्थ्ये॒व तद्रेत॑सि दधाति। ति॒स्र आद॑धाति मिथुन॒त्वाय॑। इय॑तीर्भवन्ति। प्र॒जाप॑तिना यज्ञमु॒खेन॒ सम्मि॑ताः॥६८॥

%1.1.9.5
इय॑तीर्भवन्ति। य॒ज्ञ॒प॒रुषा॒ सम्मि॑ताः। इय॑तीर्भवन्ति। ए॒ताव॒द्वै पुरु॑षे वी॒र्यम्। वी॒र्य॑संमिताः। आ॒र्द्रा भ॑वन्ति। आ॒र्द्रमि॑व॒ हि रेत॑ सि॒च्यते। चित्रि॑यस्याश्व॒त्थस्याद॑धाति। चि॒त्रमे॒व भ॑वति। घृ॒तव॑तीभि॒रा द॑धाति॥६९॥

%1.1.9.6
ए॒तद्वा अ॒ग्नेः प्रि॒यं धाम॑। यद्घृ॒तम्। प्रि॒येणै॒वैनं॒ धाम्ना॒ सम॑र्धयति। अथो॒ तेज॑सा। गा॒य॒त्रीभि॑र्ब्राह्म॒णस्याद॑ध्यात्। गा॒य॒त्रछ॑न्दा॒ वै ब्राह्म॒णः। स्वस्य॒ छन्द॑सः प्रत्ययन॒स्त्वाय॑। त्रि॒ष्टुग्भी॑ राज॒न्य॑स्य। त्रि॒ष्टुप्छ॑न्दा॒ वै रा॑ज॒न्य॑। स्वस्य॒ छन्द॑सः प्रत्ययन॒स्त्वाय॑॥७०॥

%1.1.9.7
जग॑तीभि॒र्वैश्य॑स्य। जग॑तीछन्दा॒ वै वैश्य॑। स्वस्य॒ छन्द॑सः प्रत्ययन॒स्त्वाय॑। त सं॑वत्स॒रं गो॑पायेत्। सं॒व॒त्स॒र हि रेतो॑ हि॒तं वर्ध॑ते। यद्ये॑न संवत्स॒रे नोप॒नमेत्। स॒मिध॒ पुन॒राद॑ध्यात्। रेत॑ ए॒व तद्धि॒तं वर्ध॑मानमेति। न मा॒सम॑श्ञीयात्। न स्त्रिय॒मुपे॑यात्॥७१॥

%1.1.9.8
यन्मा॒सम॑श्ञी॒यात्। यत्स्त्रिय॑मुपे॒यात्। निर्वीर्यः स्यात्। नैन॑म॒ग्निरुप॑नमेत्। श्व आ॑धा॒स्यमा॑नो ब्रह्मौद॒नं प॑चति। आ॒दि॒त्या वा इ॒त उ॑त्त॒माः सु॑व॒र्गं लो॒कमा॑यन्। ते वा इ॒तो यन्तं॒ प्रति॑नुदन्ते। ए॒ते खलु॒ वावादि॒त्याः। यद्ब्राह्म॒णाः। तैरे॒व स॒न्त्वं ग॑च्छति॥७२॥

%1.1.9.9
नैनं॒ प्रति॑नुदन्ते। ब्र॒ह्म॒वा॒दिनो॑ वदन्ति। क्वा॑ सः। अ॒ग्निः का॒र्य॑। योऽस्मै प्र॒जां प॒शून्प्र॑ज॒नय॒तीति॑। शल्कै॒स्तारात्रि॑म॒ग्निमि॑न्धीत। तस्मि॑न्नुपव्यु॒षम॒रणी॒ निष्ट॑पेत्। यथ॑र्\mbox{}ष॒भाय॑ वाशि॒ता न्या॑विच्छा॒यति॑। ता॒दृगे॒व तत्। अ॒पो॒दूह्य॒ भस्मा॒ग्निं म॑न्थति॥७३॥

%1.1.9.10
सैव साऽग्नेः सन्त॑तिः। तं म॑थि॒त्वा प्राञ्च॒मुद्ध॑रति। सं॒व॒त्स॒रमे॒व तद्रेतो॑ हि॒तं प्रज॑नयति। अना॑हित॒स्तस्या॒ग्निरित्या॑हुः। यः स॒मिधोऽना॑धाया॒ग्निमा॑ध॒त्त इति॑। ताः सं॑वत्स॒रे पु॒रस्ता॒दाद॑ध्यात्। सं॒व॒त्सरादे॒वैन॑मव॒रुध्याध॑त्ते। यदि॑ संवत्स॒रेऽनाद॒ध्यात्। द्वा॒द॒श्यां पु॒रस्ता॒दाद॑ध्यात्। सं॒व॒त्स॒रप्र॑तिमा॒ वै द्वाद॑श॒ रात्र॑यः। सं॒व॒त्स॒रमे॒वास्याहि॑ता भवन्ति। यदि॑ द्वाद॒श्यां नाद॒ध्यात्। त्र्य॒हे पु॒रस्ता॒दाद॑ध्यात्। आहि॑ता ए॒वास्य॑ भवन्ति॥७४॥\anuvakamend[द्वि॒तीय॑मपचच्चतु॒र्थम॑पच॒ददि॑ती॒ रेतो॑ऽधत्त॒ सम्मि॑ता घृ॒तव॑तीभि॒राद॑धाति राज॒न्य॑ स्वस्य॒ छन्द॑सः प्रत्ययन॒स्त्वाये॑याद्गच्छति मन्थति॒ रात्र॑यश्च॒त्वारि॑ च]

%1.1.10.1
प्र॒जाप॑तिः प्र॒जा अ॑सृजत। स रि॑रिचा॒नो॑ऽमन्यत। स तपो॑ऽतप्यत। स आ॒त्मन्वी॒र्य॑मपश्यत्। तद॑वर्धत। तद॑स्मा॒त्सह॑सो॒र्ध्वम॑सृज्यत। सा वि॒राड॑भवत्। तां दे॑वासु॒रा व्य॑गृह्णत। सोऽब्रवीत्प्र॒जाप॑तिः। मम॒ वा ए॒षा॥७५॥

%1.1.10.2
दोहा॑ ए॒व यु॒ष्माक॒मिति॑। सा तत॒ प्राच्युद॑क्रामत्। तत्प्र॒जाप॑ति॒ पर्य॑गृह्णात्। अथ॑र्व पि॒तुं मे॑ गोपा॒येति॑। सा द्वि॒तीय॒मुद॑क्रामत्। तत्प्र॒जाप॑ति॒ पर्य॑गृह्णात्। नर्य॑ प्र॒जां मे॑ गोपा॒येति॑। सा तृ॒तीय॒मुद॑क्रामत्। तत्प्र॒जाप॑ति॒ पर्य॑गृह्णात्। शस्य॑ प॒शून्मे॑ गोपा॒येति॑॥७६॥

%1.1.10.3
सा च॑तु॒र्थमुद॑क्रामत्। तत्प्र॒जाप॑ति॒ पर्य॑गृह्णात्। सप्र॑थ स॒भां मे॑ गोपा॒येति॑। सा प॑ञ्च॒ममुद॑क्रामत्। तत्प्र॒जाप॑ति॒ पर्य॑गृह्णात्। अहे॑ बुध्निय॒ मन्त्रं॑ मे गोपा॒येति॑। अ॒ग्नीन् वाव सा तान्व्य॑क्रमत। तान्प्र॒जाप॑ति॒ पर्य॑गृह्णात्। अथो॑ प॒ङ्क्तिमे॒व। प॒ङ्क्तिर्वा ए॒षा ब्राह्म॒णे प्रवि॑ष्टा॥७७॥

%1.1.10.4
तामा॒त्मनोऽधि॒ निर्मि॑मीते। यद॒ग्निरा॑धी॒यते। तस्मा॑दे॒ताव॑न्तो॒ऽग्नय॒ आधी॑यन्ते। पाङ्क्तं॒ वा इ॒द सर्वम्। पाङ्क्ते॑नै॒व पाङ्क्त स्पृणोति। अथ॑र्व पि॒तुं मे॑ गोपा॒येत्या॑ह। अन्न॑मे॒वैतेन॑ स्पृणोति। नर्य॑ प्र॒जां मे॑ गोपा॒येत्या॑ह। प्र॒जामे॒वैतेन॑ स्पृणोति। शस्य॑ प॒शून्मे॑ गोपा॒येत्या॑ह॥७८॥

%1.1.10.5
प॒शूने॒वैतेन॑ स्पृणोति। सप्र॑थ स॒भां मे॑ गोपा॒येत्या॑ह। स॒भामे॒वैतेनेन्द्रि॒य स्पृ॑णोति। अहे॑ बुध्निय॒ मन्त्रं॑ मे गोपा॒येत्या॑ह। मन्त्र॑मे॒वैतेन॒ श्रिय स्पृणोति। यदा॑न्वाहार्य॒पच॑नेऽन्वाहा॒र्यं॑ पच॑न्ति। तेन॒ सोऽस्या॒भीष्ट॑ प्री॒तः। यद्गार्\mbox{}ह॑पत्य॒ आज्य॑मधि॒श्रय॑न्ति॒ संपत्नीर्या॒जय॑न्ति। तेन॒ सोऽस्या॒भीष्ट॑ प्री॒तः। यदा॑हव॒नीये॒ जुह्व॑ति॥७९॥

%1.1.10.6
तेन॒ सोऽस्या॒भीष्ट॑ प्री॒तः। यत्स॒भायां वि॒जय॑न्ते। तेन॒ सोऽस्या॒भीष्ट॑ प्री॒तः। यदा॑वस॒थेऽन्न॒ हर॑न्ति। तेन॒ सोऽस्या॒भीष्ट॑ प्री॒तः। तथाऽस्य॒ सर्वे प्री॒ता अ॒भीष्टा॒ आधी॑यन्ते। प्र॒व॒स॒थमे॒ष्यन्ने॒वमुप॑तिष्ठे॒तैक॑मेकम्। यथा ब्राह्म॒णाय॑ गृहेवा॒सिने॑ परि॒दाय॑ गृ॒हानेति॑। ता॒दृगे॒व तत्। पुन॑रा॒गत्योप॑तिष्ठते। सा भा॑गेयमे॒वैषां॒ तत्। सा तत॑ ऊ॒र्ध्वारो॑हत्। सा रो॑हि॒ण्य॑भवत्। तद्रो॑हि॒ण्यै रो॑हिणि॒त्वम्। रो॒हि॒ण्याम॒ग्निमाद॑धीत। स्व ए॒वैनं॒ योनौ॒ प्रति॑ष्ठित॒माध॑त्ते। ऋ॒ध्नोत्ये॑नेन॥८०॥\anuvakamend[ए॒षा प॒शून्मे॑ गोपा॒येति॒ प्रवि॑ष्टा प॒शून्मे॑ गोपा॒येत्या॑ह॒ जुह्व॑ति तिष्ठते स॒प्त च॑]





\prashnaend{ब्रह्म॒ संध॑त्तं॒ कृत्ति॑का॒सूद्ध॑न्ति द्वाद॒शसु॑ प्र॒जाप॑तिर्वा॒चो दे॑वासु॒रास्तद॒ग्निर्नेद्घ॒र्मश्शिर॑ इ॒मे वै श॑मीग॒र्भात्प्र॒जाप॑ति॒ स रि॑रिचा॒नः स तप॒ स आ॒त्मन्वी॒र्यं॑दश॑॥१०॥}{ब्रह्म॒ सन्ध॑त्तं॒ तौ दि॒व्यावथो॑ शं॒त्वाय॒ प्राच्ये॑षां॒ यदु॒पर्यु॑परि॒ यत्स॒द्यः सोऽश्वो॒ऽवारो॑ भू॒त्वा जग॑तीभि॒रशी॑तिः॥८०॥}{ब्रह्म॒ सन्ध॑त्तमृ॒ध्नोत्ये॑नेन॥}{हरि॑ ओम्॥}{इति श्रीकृष्णयजुर्वेदीयतैत्तिरीयब्राह्मणे प्रथमाष्टके प्रथमः प्रपाठकः समाप्तः॥}
\clearpage
