\sect{पञ्चमः प्रश्नः}
\setcounter{anuvakam}{0}
\dnsub{तैत्तिरीयब्राह्मणे द्वितीयाष्टके पञ्चमः प्रपाठकः}

%2.5.1.1
प्रा॒णो र॑क्षति॒ विश्व॒मेज॑त्। इर्यो॑ भू॒त्वा ब॑हु॒धा ब॒हूनि॑। स इत्सर्वं॒ व्या॑नशे। यो दे॒वो दे॒वेषु॑ वि॒भूर॒न्तः। आवृ॑दू॒दात् क्षेत्रिय॑ध्व॒गद्वृषा। तमित्प्रा॒णं मन॒सोप॑ शिक्षत। अग्रं॑ दे॒वाना॑मि॒दम॑त्तु नो ह॒विः। मन॑स॒श्चित्ते॒दम्। भू॒तं भव्यं॑ च गुप्यते। तद्धि दे॒वेष्व॑ग्रि॒यम्॥१॥

%2.5.1.2
आ न॑ एतु पुरश्च॒रम्। स॒ह दे॒वैरि॒म हवम्। मन॒ श्रेय॑सिश्रेयसि। कर्म॑न् य॒ज्ञप॑ति॒न्दध॑त्। जु॒षतां मे॒ वागि॒द ह॒विः। वि॒राड्दे॒वी पु॒रोहि॑ता। ह॒व्य॒वा़डन॑पायिनी। यया॑ रू॒पाणि॑ बहु॒धा वद॑न्ति। पेशासि दे॒वाः प॑र॒मे ज॒नित्रे। सा नो॑ वि॒राडन॑पस्फुरन्ती॥२॥

%2.5.1.3
वाग्दे॒वी जु॑षतामि॒द ह॒विः। चक्षु॑र्दे॒वानां॒ ज्योति॑र॒मृते॒ न्य॑क्तम्। अ॒स्य वि॒ज्ञाना॑य बहु॒धा निधी॑यते। तस्य॑ सु॒म्नम॑शीमहि। मा नो॑ हासीद्विचक्ष॒णम्। आयु॒रिन्न॒ प्रतीर्यताम्। अन॑न्धा॒श्चक्षु॑षा व॒यम्। जी॒वा ज्योति॑रशीमहि। सुव॒र्ज्योति॑रु॒तामृतम्। श्रोत्रे॑ण भ॒द्रमु॒त शृ॑ण्वन्ति स॒त्यम्। श्रोत्रे॑ण॒ वाचं॑ बहु॒धोद्यमा॑नाम्। श्रोत्रे॑ण॒ मोद॑श्च॒ मह॑श्च श्रूयते। श्रोत्रे॑ण॒ सर्वा॒ दिश॒ आ शृ॑णोमि। येन॒ प्राच्या॑ उ॒त द॑क्षि॒णा। प्र॒तीच्यै॑ दि॒शः शृ॒ण्वन्त्यु॑त्त॒रात्। तदिच्छ्रोत्रं॑ बहु॒धोद्यमा॑नम्। अ॒रान्न ने॒मिः परि॒ सर्वं॑ बभूव॥३॥\anuvakamend[अ॒ग्रि॒यमन॑पस्फुरन्ती स॒त्य स॒प्त च॑]

%2.5.2.1
उ॒देहि॑ वाजि॒न्यो अ॑स्य॒प्स्व॑न्तः। इ॒द रा॒ष्ट्रमा वि॑श सू॒नृता॑वत्। यो रोहि॑तो॒ विश्व॑मि॒दञ्ज॒जान॑। स नो॑ रा॒ष्ट्रेषु॒ सुधि॑तान्दधातु। रोहरोह॒ रोहि॑त॒ आरु॑रोह। प्र॒जाभि॒र्वृद्धिं॑ ज॒नुषा॑मु॒पस्थम्। ताभि॒ सर॑ब्धो अविद॒थ्षडु॒र्वीः। गा॒तुं प्र॒पश्य॑न्नि॒ह रा॒ष्ट्रमाऽहा। आऽहा॑र्\mbox{}षीद्रा॒ष्ट्रमि॒ह रोहि॑तः। मृधो॒ व्यास्थ॒दभ॑यन्नो अस्तु ॥४॥

%2.5.2.2
अ॒स्मभ्य॑न्द्यावापृथिवी॒ शक्व॑रीभिः। रा॒ष्ट्रन्दु॑हाथामि॒ह रे॒वती॑भिः। विम॑मर्\mbox{}श॒ रोहि॑तो वि॒श्वरू॑पः। स॒मा॒च॒क्रा॒णः प्र॒रुहो॒ रुह॑श्च। दिव॑ङ्ग॒त्वाय॑ मह॒ता म॑हि॒म्ना। वि नो॑ रा॒ष्ट्रमु॑नत्तु॒ पय॑सा॒ स्वेन॑। यास्ते॒ विश॒स्तप॑सा सं बभू॒वुः। गा॒य॒त्रं व॒त्समनु॒ तास्त॒ आऽगु॑। तास्त्वा वि॑शन्तु॒ मह॑सा॒ स्वेन॑। सं मा॑ता पु॒त्रो अ॒भ्ये॑तु॒ रोहि॑तः॥५॥

%2.5.2.3
यू॒यमु॑ग्रा मरुतः पृश्ञिमातरः। इन्द्रे॑ण स॒युजा॒ प्रमृ॑णीथ॒ शत्रून्॑। आ वो॒ रोहि॑तो अशृणोदभिद्यवः। त्रिस॑प्तासो मरुतः स्वादुसम्मुदः। रोहि॑तो॒ द्यावा॑पृथि॒वी ज॑जान। तस्मि॒स्तन्तुं॑ परमे॒ष्ठी त॑तान। तस्मि॑ञ्छिश्रिये अ॒ज एक॑पात्। अदृह॒द्द्यावा॑पृथि॒वी बले॑न। रोहि॑तो॒ द्यावा॑पृथि॒वी अ॑दृहत्। तेन॒ सुव॑ स्तभि॒तन्तेन॒ नाक॑॥६॥

%2.5.2.4
सो अ॒न्तरि॑क्षे॒ रज॑सो वि॒मान॑। तेन॑ दे॒वाः सुव॒रन्व॑विन्दन्। सु॒शेव॑न्त्वा भा॒नवो॑ दीदि॒वासम्। सम॑ग्रासो जु॒ह्वो॑ जातवेदः। उ॒क्षन्ति॑ त्वा वा॒जिन॒मा घृ॒तेन॑। सस॑मग्ने युवसे॒ भोज॑नानि। अग्ने॒ शर्ध॑ मह॒ते सौभ॑गाय। तव॑ द्यु॒म्नान्यु॑त्त॒मानि॑ सन्तु। सञ्जास्प॒त्य सु॒यम॒मा कृ॑णुष्व। श॒त्रू॒य॒ताम॒भि ति॑ष्ठा॒ महासि॥७॥\anuvakamend[अ॒स्त्वे॒तु॒ रोहि॑तो॒ नाको॒ महासि]

%2.5.3.1
पुन॑र्न॒ इन्द्रो॑ म॒घवा॑ ददातु। धना॑नि श॒क्रो धन्य॑ सु॒राधा। अ॒र्वा॒चीन॑ङ्कृणुतां याचि॒तो मन॑। श्रु॒ष्टी नो॑ अ॒स्य ह॒विषो॑ जुषा॒णः। यानि॑ नोऽजि॒नन्धना॑नि। ज॒हर्थ॑ शूर म॒न्युना। इन्द्रानु॑विन्द न॒स्तानि॑। अ॒नेन॑ ह॒विषा॒ पुन॑। इन्द्र॒ आशाभ्य॒ परि॑। सर्वा॒भ्योऽभ॑यङ्करत्॥८॥

%2.5.3.2
जेता॒ शत्रू॒न्॒ विच॑र्\mbox{}षणिः। आकूत्यै त्वा॒ कामा॑य त्वा स॒मृधे त्वा। पु॒रो द॑धे अमृत॒त्वाय॑ जी॒वसे। आकू॑तिम॒स्याव॑से। काम॑मस्य॒ समृ॑द्ध्यै। इन्द्र॑स्य युञ्जते॒ धिय॑। आकू॑तिन्दे॒वीं मन॑सः पु॒रो द॑धे। य॒ज्ञस्य॑ मा॒ता सु॒हवा॑ मे अस्तु। यदि॒च्छामि॒ मन॑सा॒ सका॑मः। वि॒देय॑मेन॒द्धृद॑ये॒ निवि॑ष्टम्॥९॥

%2.5.3.3
सेद॒ग्निर॒ग्नीरत्येत्य॒न्यान्। यत्र॑ वा॒जी तन॑यो वी॒डुपा॑णिः। स॒हस्र॑पाथा अ॒क्षरा॑ स॒मेति॑। आशा॑नान्त्वाऽऽशापा॒लेभ्य॑। च॒तुर्भ्यो॑ अ॒मृतेभ्यः। इ॒दं भू॒तस्याध्य॑क्षेभ्यः। वि॒धेम॑ ह॒विषा॑ व॒यम्। विश्वा॒ आशा॒ मधु॑ना॒ स सृ॑जामि। अ॒न॒मी॒वा आप॒ ओष॑धयो भवन्तु। अ॒यं यज॑मानो॒ मृधो॒ व्य॑स्यताम्॥१०॥

%2.5.3.4
अगृ॑भीताः प॒शव॑ सन्तु॒ सर्वे। अ॒ग्निः सोमो॒ वरु॑णो मि॒त्र इन्द्र॑। बृह॒स्पति॑ सवि॒ता यः स॑ह॒स्री। पू॒षा नो॒ गोभि॒रव॑सा॒ सर॑स्वती। त्वष्टा॑ रू॒पाणि॒ सम॑नक्तु य॒ज्ञैः। त्वष्टा॑ रू॒पाणि॒ दध॑ती॒ सर॑स्वती। पू॒षा भग सवि॒ता नो॑ ददातु। बृह॒स्पति॒र्दद॒दिन्द्र॑ स॒हस्रम्। मि॒त्रो दा॒ता वरु॑ण॒ सोमो॑ अ॒ग्निः॥११॥\anuvakamend[क॒र॒न्निवि॑ष्टमस्यता॒न्नव॑ च]

%2.5.4.1
आ नो॑ भर॒ भग॑मिन्द्र द्यु॒मन्तम्। नि ते॑ दे॒ष्णस्य॑ धीमहि प्ररे॒के। उ॒र्व इ॑व पप्रथे॒ कामो॑ अ॒स्मे। तमापृ॑णा वसुपते॒ वसू॑नाम्। इ॒मङ्कामं॑ मन्दया॒ गोभि॒रश्वै। च॒न्द्रव॑ता॒ राध॑सा प॒प्रथ॑श्च। सु॒व॒र्यवो॑ म॒तिभि॒स्तुभ्यं॒ विप्रा। इन्द्रा॑य॒ वाह॑ कुशि॒कासो॑ अक्रन्। इन्द्र॑स्य॒ नु वी॒र्या॑णि॒ प्रवो॑चम्। यानि॑ च॒कार॑ प्रथ॒मानि॑ व॒ज्री॥१२॥

%2.5.4.2
अह॒न्नहि॒मन्व॒पस्त॑तर्द। प्रव॒क्षणा॑ अभिन॒त्पर्व॑तानाम्। अह॒न्नहिं॒ पर्व॑ते शिश्रिया॒णम्। त्वष्टाऽस्मै॒ वज्र स्व॒र्य॑न्ततक्ष। वा॒श्रा इ॑व धे॒नव॒ स्यन्द॑मानाः। अञ्ज॑ समु॒द्रमव॑ जग्मु॒राप॑। वृ॒षा॒यमा॑णोऽवृणीत॒ सोमम्। त्रिक॑द्रुकेष्वपिबत्सु॒तस्य॑। आ साय॑कं म॒घवा॑ दत्त॒ वज्रम्। अह॑न्नेनं प्रथम॒जा मही॑नाम्॥१३॥

%2.5.4.3
यदिन्द्राह॑न्प्रथम॒जा मही॑नाम्। आन्मा॒यिना॒ममि॑ना॒ प्रोत मा॒याः। आत्सूर्यं॑ ज॒नय॒न्द्यामु॒षासम्। ता॒दीक्ना॒ शत्रू॒न्न किला॑विवित्से। अह॑न्वृ॒त्रं वृ॑त्र॒तरं॒ व्यसम्। इन्द्रो॒ वज्रे॑ण मह॒ता व॒धेन॑। स्कन्धासीव॒ कुलि॑शेना॒विवृ॑क्णा। अहि॑ शयत उप॒पृक्पृ॑थि॒व्याम्। अ॒यो॒ध्येव दु॒र्मद॒ आ हि जु॒ह्वे। म॒हा॒वी॒रन्तु॑विबा॒धमृ॑जी॒षम्॥१४॥

%2.5.4.4
नाता॑रीरस्य॒ समृ॑तिं व॒धानाम्। स रु॒जाना पिपिष॒ इन्द्र॑शत्रुः। विश्वो॒ विहा॑या अर॒तिः। वसु॑र्दधे॒ हस्ते॒ दक्षि॑णे। त॒रणि॒र्न शि॑श्रथत्। श्र॒व॒स्य॑या॒ न शि॑श्रथत्। विश्व॑स्मा॒ इदि॑षुध्य॒से। दे॒व॒त्रा ह॒व्यमूहि॑षे। विश्व॑स्मा॒ इत्सु॒कृते॒ वार॑मृण्वति। अ॒ग्निर्द्वारा॒ व्यृ॑ण्वति॥१५॥

%2.5.4.5
उदु॒ज्जिहा॑नो अ॒भि काम॑मी॒रय\sn{}। प्र॒पृ॒ञ्चन्विश्वा॒ भुव॑नानि पू॒र्वथा। आ के॒तुना॒ सुष॑मिद्धो॒ यजि॑ष्ठः। काम॑न्नो अग्ने अ॒भिह॑र्य दि॒ग्भ्यः। जु॒षा॒णो ह॒व्यम॒मृते॑षु दू॒ढ्य॑। आ नो॑ र॒यिं ब॑हु॒लाङ्गोम॑ती॒मिषम्। नि धे॑हि॒ यक्ष॑द॒मृते॑षु॒ भूष\sn{}। अश्वि॑ना य॒ज्ञमाग॑तम्। दा॒शुष॒ पुरु॑दससा। पू॒षा र॑क्षतु नो र॒यिम्॥१६॥

%2.5.4.6
इ॒मं य॒ज्ञम॒श्विना॑ व॒र्धय॑न्ता। इ॒मौ र॒यिं यज॑मानाय धत्तम्। इ॒मौ प॒शून्र॑क्षतां वि॒श्वतो॑ नः। पू॒षा न॑ पातु॒ सद॒मप्र॑यच्छन्। प्रते॑ म॒हे स॑रस्वति। सुभ॑गे॒ वाजि॑नीवति। स॒त्य॒वाचे॑ भरे म॒तिम्। इ॒दन्नो॑ ह॒व्यङ्घृ॒तव॑त्सरस्वति। स॒त्य॒वाचे॒ प्रभ॑रेमा ह॒वीषि॑। इ॒मानि॑ ते दुरि॒ता सौभ॑गानि। तेभि॑र्व॒य सु॒भगा॑सः स्याम॥१७॥\anuvakamend[व॒ज्र्यही॑नामृजी॒षं व्यृ॑ण्वति रक्षतु नो र॒यि सौभ॑गा॒न्येकं च]

%2.5.5.1
य॒ज्ञो रा॒यो य॒ज्ञ ई॑शे॒ वसू॑नाम्। य॒ज्ञः स॒स्याना॑मु॒त सु॑क्षिती॒नाम्। य॒ज्ञ इ॒ष्टः पू॒र्वचि॑त्तिन्दधातु। य॒ज्ञो ब्र॑ह्म॒ण्वा अप्ये॑तु दे॒वान्। अ॒यं य॒ज्ञो व॑र्धता॒ङ्गोभि॒रश्वै। इ॒यं वेदि॑ स्वप॒त्या सु॒वीरा। इ॒दं ब॒र्॒हिरति॑ ब॒र्॒हीष्य॒न्या। इ॒मं य॒ज्ञं विश्वे॑ अवन्तु दे॒वाः। भग॑ ए॒व भग॑वा अस्तु देवाः। तेन॑ व॒यं भग॑वन्तः स्याम॥१८॥

%2.5.5.2
तन्त्वा॑ भग॒ सर्व॒ इज्जो॑हवीमि। स नो॑ भग पुरए॒ता भ॑वे॒ह। भग॒ प्रणे॑त॒र्भग॒ सत्य॑राधः। भगे॒मान्धिय॒मुद॑व॒ दद॑न्नः। भग॒ प्र णो॑ जनय॒ गोभि॒रश्वै। भग॒ प्र नृभि॑र्नृ॒वन्त॑ स्याम। शश्व॑ती॒ समा॒ उप॑यन्ति लो॒काः। शश्व॑ती॒ समा॒ उप॑य॒न्त्याप॑। इ॒ष्टं पू॒र्त शश्व॑तीना॒ समा॑ना शाश्व॒तेन॑। ह॒विषे॒ष्ट्वाऽन॒न्तं लो॒कं पर॒मा रु॑रोह ॥१९॥

%2.5.5.3
इ॒यमे॒व सा या प्र॑थ॒मा व्यौच्छ॑त्। सा रू॒पाणि॑ कुरुते॒ पञ्च॑ दे॒वी। द्वे स्वसा॑रौ वयत॒स्तन्त्र॑मे॒तत्। स॒ना॒तनं॒ वित॑त॒ षण्म॑यूखम्। अवा॒न्यास्तन्तून्कि॒रतो॑ ध॒त्तो अ॒न्यान्। नाव॑पृ॒ज्याते॒ न ग॑माते॒ अन्तम्। आ वो॑ यन्तूदवा॒हासो॑ अ॒द्य। वृष्टिं॒ ये विश्वे॑ म॒रुतो॑ जु॒नन्ति॑। अ॒यय्योँ अ॒ग्निर्म॑रुत॒ समि॑द्धः। ए॒तं जु॑षध्वङ्कवयो युवानः॥२०॥

%2.5.5.4
धा॒रा॒व॒रा म॒रुतो॑ धृ॒ष्णुवो॑जसः। मृ॒गा न भी॒मास्त॑वि॒षेभि॑रू॒र्मिभि॑। अ॒ग्नयो॒ न शु॑शुचा॒ना ऋ॑जी॒षिण॑। भ्रुमि॒न्धम॑न्त॒ उप॒ गा अ॑वृण्वत। वि च॑क्रमे॒ त्रिर्दे॒वः। आ वे॒धस॒न्नील॑पृष्ठं बृ॒हन्तम्। बृह॒स्पति॒ सद॑ने सादयध्वम्। सा॒दद्यो॑नि॒न्दम॒ आ दी॑दि॒वासम्। हिर॑ण्यवर्णमरु॒ष स॑पेम। स हि शुचि॑ श॒तप॑त्र॒ स शु॒न्ध्यूः ॥२१॥

%2.5.5.5
हिर॑ण्यवाशीरिषि॒रः सु॑व॒र्॒षाः। बृह॒स्पति॒ स स्वा॑वे॒श ऋ॒ष्वाः। पू॒रू सखि॑भ्य आसु॒तिङ्क॑रिष्ठः। पूष॒ स्तव॑ व्र॒ते व॒यम्। नरि॑ष्येम क॒दाच॒न। स्तो॒तार॑स्त इ॒ह स्म॑सि। यास्ते॑ पूष॒न्ना वो॑ अ॒न्तः स॑मु॒द्रे। हि॒र॒ण्ययी॑र॒न्तरि॑क्षे॒ चर॑न्ति। याभि॑र्यासि दू॒त्या सूर्य॑स्य। कामे॑न कृ॒तश्रव॑ इ॒च्छमा॑नः॥२२॥

%2.5.5.6
अर॑ण्या॒न्यर॑ण्यान्य॒सौ। या प्रेव॒ नश्य॑सि। क॒था ग्राम॒न्न पृ॑च्छसि। न त्वा॒भीरि॑व विन्दती ३। वृ॒षा॒र॒वाय॒ वद॑ते। यदु॒पाव॑ति चिच्चि॒कः। आ॒घा॒टीभि॑रिव धा॒वय\sn{}। अ॒र॒ण्या॒निर्म॑हीयते। उ॒त गाव॑ इवादन्। उ॒तो वेश्मे॑व दृश्यते॥२३॥

%2.5.5.7
उ॒तो अ॑रण्या॒निः सा॒यम्। श॒क॒टीरि॑व सर्जति। गाम॒ङ्गैष॒ आ ह्व॑यति। दार्व॒ङ्गैष॒ उपा॑वधीत्। वस॑न्नरण्या॒न्या सा॒यम्। अक्रु॑क्ष॒दिति॑ मन्यते। न वा अ॑रण्या॒निर्\mbox{}ह॑न्ति। अ॒न्यश्चेन्नाभि॒गच्छ॑ति। स्वा॒दोः फल॑स्य ज॒ग्ध्वा। यत्र॒ कामं॒ नि प॑द्यते। आञ्ज॑नगन्धी सुर॒भीम्। ब॒ह्व॒न्नामकृ॑षीवलाम्। प्राहं मृ॒गाणां मा॒तरम्। अ॒र॒ण्या॒नीम॑शसिषम्॥२४॥\anuvakamend[स्या॒म॒ रु॒रो॒ह॒ यु॒वा॒न॒ शु॒न्ध्यूरि॒च्छमा॑नो दृश्यते॒ निप॑द्यते च॒त्वारि॑ च]

%2.5.6.1
वार्त्र॑हत्याय॒ शव॑से। पृ॒त॒ना॒साह्या॑य च। इन्द्र॒ त्वा व॑र्तयामसि। सु॒ब्रह्मा॑णं वी॒रव॑न्तं बृ॒हन्तम्। उ॒रुं ग॑भी॒रं पृ॒थुबु॑ध्नमिन्द्र। श्रु॒तर्\mbox{}षि॑मु॒ग्रम॑भिमाति॒षाहम्। अ॒स्मभ्यं॑ चि॒त्रं वृष॑ण र॒यिं दा। क्षे॒त्रि॒यै त्वा॒ निर्\mbox{}ऋ॑त्यै त्वा। द्रु॒हो मु॑ञ्चामि॒ वरु॑णस्य॒ पाशात्। अ॒ना॒गसं॒ ब्रह्म॑णे त्वा करोमि॥२५॥

%2.5.6.2
शि॒वे ते॒ द्यावा॑पृथि॒वी उ॒भे इ॒मे। शं ते॑ अ॒ग्निः स॒हाद्भिर॑स्तु। शं द्यावा॑पृथि॒वी स॒हौष॑धीभिः। शम॒न्तरि॑क्ष स॒ह वाते॑न ते। शं ते॒ चत॑स्रः प्र॒दिशो॑ भवन्तु। या दैवी॒श्चत॑स्रः प्र॒दिश॑। वात॑पत्नीर॒भि सूर्यो॑ विच॒ष्टे। तासान्त्वा ज॒रस॒ आ द॑धामि। प्र यक्ष्म॑ एतु॒ निर्\mbox{}ऋ॑तिं परा॒चैः। अमो॑चि॒ यक्ष्माद्दुरि॒तादव॑र्त्यै॥२६॥

%2.5.6.3
द्रु॒हः पाशा॒न्निर्\mbox{}ऋ॑त्यै॒ चोद॑मोचि। अहा॒ अव॑र्ति॒मवि॑दत्स्यो॒नम्। अप्य॑भूद्भ॒द्रे सु॑कृ॒तस्य॑ लो॒के। सूर्य॑मृ॒तं तम॑सो॒ ग्राह्या॒ यत्। दे॒वा अमु॑ञ्च॒न्नसृ॑ज॒न्व्ये॑नसः। ए॒वम॒हमि॒मं क्षेत्रि॒याज्जा॑मिश॒सात्। द्रु॒हो मु॑ञ्चामि॒ वरु॑णस्य॒ पाशात्। बृह॑स्पते यु॒वमिन्द्र॑श्च॒ वस्व॑। दि॒व्यस्ये॑शाथे उ॒त पार्थि॑वस्य। ध॒त्त र॒यि स्तु॑व॒ते की॒रये॑चित्॥२७॥

%2.5.6.4
यू॒यं पा॑त स्व॒स्तिभि॒ सदा॑ नः। दे॒वा॒युध॒मिन्द्र॒मा जोहु॑वानाः। वि॒श्वा॒वृध॑म॒भि ये रक्ष॑माणाः। येन॑ ह॒ता दी॒र्घमध्वा॑न॒माय\sn{}। अ॒न॒न्तमर्थ॒मनि॑वर्त्स्यमानाः। यत्ते॑ सुजाते हि॒मव॑त्सु भेष॒जम्। म॒यो॒भूः शन्त॑मा॒ यद्धृ॒दोसि॑। ततो॑ नो देहि सीबले। अ॒दो गि॒रिभ्यो॒ अधि॒ यत्प्र॒धाव॑सि। स॒शोभ॑माना क॒न्ये॑व शुभ्रे॥२८॥

%2.5.6.5
तां त्वा॒ मुद्ग॑ला ह॒विषा॑ वर्धयन्ति। सा न॑ सीबले र॒यिमा भा॑जये॒ह। पूर्वं॑ देवा॒ अप॑रेणानु॒पश्यं॒ जन्म॑भिः। जन्मा॒न्यव॑रै॒ परा॑णि। वेदा॑नि देवा अ॒यम॒स्मीति॒ माम्। अ॒ह हि॒त्वा शरी॑रं ज॒रस॑ प॒रस्तात्। प्रा॒णा॒पा॒नौ चक्षु॒ श्रोत्रम्। वाचं॒ मन॑सि॒ सम्भृ॑ताम्। हि॒त्वा शरी॑रं ज॒रस॑ प॒रस्तात्। आ भूति॒म्भूतिं॑ व॒यम॑श्ञवामहै। इ॒मा ए॒व ता उ॒षसो॒ याः प्र॑थ॒मा व्यौच्छ\sn{}। ता दे॒व्य॑ कुर्वते॒ पञ्च॑रू॒पा। शश्व॑ती॒र्नाव॑पृज्यन्ति। न ग॑म॒न्त्यन्तम्॥२९॥\anuvakamend[क॒रो॒म्यव॑र्त्यै चिच्छुभ्रेऽश्ञवामहै च॒त्वारि॑ च]

%2.5.7.1
वसू॑नां॒ त्वाऽधी॑तेन। रु॒द्राणा॑मू॒र्म्या। आ॒दि॒त्यानां॒ तेज॑सा। विश्वे॑षां दे॒वानां॒ क्रतु॑ना। म॒रुता॒मेम्ना॑ जुहोमि॒ स्वाहा। अ॒भिभू॑तिर॒हमाग॑मम्। इन्द्र॑सखा स्वा॒युध॑। आस्वाशा॑सु दु॒ष्षह॑। इ॒दं वर्चो॑ अ॒ग्निना॑ द॒त्तमागात्। यशो॒ भर्ग॒ सह॒ ओजो॒ बलं॑ च॥३०॥

%2.5.7.2
दी॒र्घा॒यु॒त्वाय॑ श॒तशा॑रदाय। प्रति॑गृभ्णामि मह॒ते वी॒र्या॑य। आयु॑रसि वि॒श्वायु॑रसि। स॒र्वायु॑रसि॒ सर्व॒मायु॑रसि। सर्व॑म्म॒ आयु॑र्भूयात्। सर्व॒मायु॑र्गेषम्। भूर्भुव॒ सुव॑। अ॒ग्निर्धर्मे॑णान्ना॒दः। मृ॒त्युर्धर्मे॒णान्न॑पतिः। ब्रह्म॑ क्ष॒त्र स्वाहा॥३१॥

%2.5.7.3
प्र॒जाप॑तिः प्रणे॒ता। बृह॒स्पति॑ पुरए॒ता। य॒मः पन्था। च॒न्द्रमा पुनर॒सुः स्वाहा। अ॒ग्निर॑न्ना॒दोऽन्न॑पतिः। अ॒न्नाद्य॑म॒स्मिन् य॒ज्ञे यज॑मानाय ददातु॒ स्वाहा। सोमो॒ राजा॒ राज॑पतिः। रा॒ज्यम॒स्मिन् य॒ज्ञे यज॑मानाय ददातु॒ स्वाहा। वरु॑णः स॒म्राट्त्स॒म्राट्प॑तिः। साम्राज्यम॒स्मिन् य॒ज्ञे यज॑मानाय ददातु॒ स्वाहा॥३२॥

%2.5.7.4
मि॒त्रः क्ष॒त्रं क्ष॒त्रप॑तिः। क्ष॒त्रम॒स्मिन् य॒ज्ञे यज॑मानाय ददातु॒ स्वाहा। इन्द्रो॒ बलं॒ बल॑पतिः। बल॑म॒स्मिन् य॒ज्ञे यज॑मानाय ददातु॒ स्वाहा। बृह॒स्पति॒र्ब्रह्म॒ ब्रह्म॑पतिः। ब्रह्मा॒स्मिन् य॒ज्ञे यज॑मानाय ददातु॒ स्वाहा। स॒वि॒ता रा॒ष्ट्र रा॒ष्ट्रप॑तिः। रा॒ष्ट्रम॒स्मिन् य॒ज्ञे यज॑मानाय ददातु॒ स्वाहा। पू॒षा वि॒शां विट्प॑तिः। विश॑म॒स्मिन् य॒ज्ञे यज॑मानाय ददातु॒ स्वाहा। सर॑स्वती॒ पुष्टि॒ पुष्टि॑पत्नी। पुष्टि॑म॒स्मिन् य॒ज्ञे यज॑मानाय ददातु॒ स्वाहा। त्वष्टा॑ पशू॒नां मि॑थु॒नाना रूप॒कृद्रू॒पप॑तिः। रु॒पेणा॒स्मिन् य॒ज्ञे यज॑मानाय प॒शून्द॑दातु॒ स्वाहा॥३३॥\anuvakamend[च॒ स्वाहा॒ साम्राज्यम॒स्मिन् य॒ज्ञे यज॑मानाय ददातु॒ स्वाहा॒ विश॑म॒स्मिन् य॒ज्ञे यज॑मानाय ददातु॒ स्वाहा॑ च॒त्वारि॑ च (अ॒ग्निः सोमो॒ वरु॑णो मि॒त्र इन्द्रो॒ बृह॒स्पति॑ सवि॒ता पू॒षा सर॑स्वती॒ त्वष्टा॒ दश॑ ॥ )]

%2.5.8.1
स ईं पाहि॒ य ऋ॑जी॒षी तरु॑त्रः। यः शिप्र॑वान्वृष॒भो यो म॑ती॒नाम्। यो गोत्र॒भिद्व॑ज्र॒भृद्यो ह॑रि॒ष्ठाः। स इ॑न्द्र चि॒त्रा अ॒भि तृ॑न्धि॒ वाजान्॑। आ ते॒ शुष्मो॑ वृष॒भ ए॑तु प॒श्चात्। ओत्त॒राद॑ध॒रागा पु॒रस्तात्। आ वि॒श्वतो॑ अ॒भिसमेत्व॒र्वाङ्। इन्द्र॑ द्यु॒म्न सुव॑र्वद्धेह्य॒स्मे। प्रोष्व॑स्मै पुरोर॒थम्। इन्द्रा॑य शू॒षम॑र्चत॥३४॥

%2.5.8.2
अ॒भीके॑ चिदु लोक॒कृत्। स॒ङ्गे स॒मत्सु॑ वृत्र॒हा। अ॒स्माकं॑ बोधि चोदि॒ता। नभ॑न्तामन्य॒केषाम्। ज्या॒का अधि॒ धन्व॑सु। इन्द्रं॑ व॒य शु॑ना॒सीरम्। अ॒स्मिन् य॒ज्ञे ह॑वामहे। आ वाजै॒रुप॑ नो गमत्। इन्द्रा॑य॒ शुना॒सीरा॑य। स्रु॒चा जु॑हुत नो ह॒विः॥३५॥

%2.5.8.3
जु॒षतां॒ प्रति॒ मेधि॑रः। प्र ह॒व्यानि॑ घृ॒तव॑न्त्यस्मै। हर्य॑श्वाय भरता स॒जोषा। इन्द्र॒र्तुभि॒र्ब्रह्म॑णा वावृधा॒नः। शु॒ना॒सी॒री ह॒विरि॒दं जु॑षस्व। वय॑ सुप॒र्णा उप॑सेदु॒रिन्द्रम्। प्रि॒यमे॑धा॒ ऋष॑यो॒ नाध॑मानाः। अप॑ ध्वा॒न्तमूर्णु॒हि पू॒र्धि चक्षु॑। मु॒मु॒ग्ध्य॑स्मान्नि॒धये॑ऽव ब॒द्धान्। बृ॒हदिन्द्रा॑य गायत॥३६॥

%2.5.8.4
मरु॑तो वृत्र॒हन्त॑मम्। येन॒ ज्योति॒रज॑नयन्नृता॒वृध॑। दे॒वं दे॒वाय॒ जागृ॑वि। कामि॒हैका॒ क इ॒मे प॑त॒ङ्गाः। मा॒न्था॒लाः कुलि॒परि॑मापतन्ति। अना॑वृतैना॒न्प्रध॑मन्तु दे॒वाः। सौप॑र्णं॒ चक्षु॑स्त॒नुवा॑ विदेय। ए॒वा व॑न्दस्व॒ वरु॑णं बृ॒हन्तम्। न॒म॒स्याधीर॑म॒मृत॑स्य गो॒पाम्। स न॒ शर्म॑ त्रि॒वरू॑थं॒ वियसत्॥३७॥

%2.5.8.5
यू॒यं पा॑त स्व॒स्तिभि॒ सदा॑ नः। नाके॑ सुप॒र्णमुप॒ यत्पत॑न्तम्। हृ॒दा वेन॑न्तो अ॒भ्यच॑क्षत त्वा। हिर॑ण्यपक्षं॒ वरु॑णस्य दू॒तम्। य॒मस्य॒ योनौ॑ शकु॒नं भु॑र॒ण्युम्। शं नो॑ दे॒वीर॒भिष्ट॑ये। आपो॑ भवन्तु पी॒तये। शय्योँर॒भि स्र॑वन्तु नः। ईशा॑ना॒ वार्या॑णाम्। क्षय॑न्तीश्चर्\mbox{}षणी॒नाम्॥३८॥

%2.5.8.6
अ॒पो या॑चामि भेष॒जम्। अ॒प्सु मे॒ सोमो॑ अब्रवीत्। अ॒न्तर्विश्वा॑नि भेष॒जा। अ॒ग्निं च॑ वि॒श्वश॑म्भुवम्। आप॑श्च वि॒श्वभे॑षजीः। यद॒प्सु ते॑ सरस्वति। गोष्वश्वे॑षु॒ यन्मधु॑। तेन॑ मे वाजिनीवति। मुख॑मङ्ग्धि सरस्वति। या सर॑स्वती वैशम्भ॒ल्या॥३९॥

%2.5.8.7
तस्यां मे रास्व। तस्यास्ते भक्षीय। तस्यास्ते भूयिष्ठ॒भाजो॑ भूयास्म। अ॒हं त्वद॑स्मि॒ मद॑सि॒ त्वमे॒तत्। ममा॑सि॒ योनि॒स्तव॒ योनि॑रस्मि। ममै॒व सन्वह॑ ह॒व्यान्य॑ग्ने। पु॒त्रः पि॒त्रे लो॑क॒कृज्जा॑तवेदः। इ॒हैव सन्तत्र॒ सन्तं॑ त्वाऽग्ने। प्रा॒णेन॑ वा॒चा मन॑सा बिभर्मि। ति॒रो मा॒ सन्त॒मायु॒र्मा प्रहा॑सीत्॥४०॥

%2.5.8.8
ज्योति॑षा त्वा वैश्वान॒रेणोप॑तिष्ठे। अ॒यं ते॒ योनि॑र्\mbox{}ऋ॒त्विय॑। यतो॑ जा॒तो अरो॑चथाः। तं जा॒नन्न॑ग्न॒ आरो॑ह। अथा॑ नो वर्धया र॒यिम्। या ते॑ अग्ने य॒ज्ञिया॑ त॒नूस्तयेह्यारो॑हा॒त्माऽऽत्मानम्। अच्छा॒ वसू॑नि कृ॒ण्वन्न॒स्मे नर्या॑ पु॒रूणि॑। य॒ज्ञो भू॒त्वा य॒ज्ञमा सी॑द॒ स्वां योनिम्। जात॑वेदो॒ भुव॒ आ जाय॑मान॒ सक्ष॑य॒ एहि॑। उ॒पाव॑रोह जातवेद॒ पुन॒स्त्वम्॥४१॥

%2.5.8.9
दे॒वेभ्यो॑ ह॒व्यं व॑ह नः प्रजा॒नन्। आयु॑ प्र॒जा र॒यिम॒स्मासु॑ धेहि। अज॑स्रो दीदिहि नो दुरो॒णे। तमिन्द्रं॑ जोहवीमि म॒घवा॑नमु॒ग्रम्। स॒त्रा दधा॑न॒मप्र॑तिष्कुत॒ शवासि। महि॑ष्ठो गी॒र्भिरा च॑ य॒ज्ञियो॑ऽव॒वर्त॑त्। रा॒ये नो॒ विश्वा॑ सु॒पथा॑ कृणोतु व॒ज्री। त्रिक॑द्रुकेषु महि॒षो यवा॑शिरं तुवि॒शुष्म॑स्तृ॒पत्। सोम॑मपिब॒द्विष्णु॑ना सु॒तं यथाऽव॑शत्। स ईं ममाद॒ महि॒ कर्म॒ कर्त॑वे म॒हामु॒रुम्॥४२॥

%2.5.8.10
सैन सश्चद्दे॒वं दे॒वः स॒त्यमिन्दु स॒त्य इन्द्र॑। वि॒दद्यती॑ स॒रमा॑ रु॒ग्णमद्रे। महि॒ पाथ॑ पू॒र्व्य स॒द्ध्रिय॑क्कः। अग्रं॑ नयत्सु॒पद्यक्ष॑राणाम्। अच्छा॒ रवं॑ प्रथ॒मा जा॑न॒तीगात्। वि॒दद्गव्य स॒रमा॑ दृ॒ढमू॒र्वम्। येना॒नुकं॒ मानु॑षी॒ भोज॑ते॒ विट्। आ ये विश्वा स्वप॒त्यानि॑ च॒क्रुः। कृ॒ण्वा॒नासो॑ अमृत॒त्वाय॑ गा॒तुम्। त्वं नृभि॑र्नृपते दे॒वहू॑तौ ॥४३॥

%2.5.8.11
भूरी॑णि वृ॒त्वा ह॑र्यश्व हसि। त्वन्निद॑स्यु॒ञ्चुमु॑रिम्। धुनिं॒ चास्वा॑पयो द॒भीत॑ये सु॒हन्तु॑। ए॒वा पा॑हि प्र॒त्नथा॒ मन्द॑तु त्वा। श्रु॒धि ब्रह्म॑ वावृधस्वो॒त गी॒र्भिः। आ॒विः सुर्यं॑ कृणु॒हि पी॒पिही॒षः। ज॒हि शत्रू र॒भि गा इ॑न्द्र तृन्धि। अग्ने॒ बाध॑स्व॒ वि मृधो॑ नुदस्व। अपामी॑वा॒ अप॒ रक्षासि सेध। अ॒स्मात्स॑मु॒द्राद्बृ॑ह॒तो दि॒वो न॑॥४४॥

%2.5.8.12
अ॒पां भू॒मान॒मुप॑ नः सृजे॒ह। यज्ञ॒ प्रति॑तिष्ठ सुम॒तौ सु॒शेवा॒ आ त्वा। वसू॑नि पुरु॒धा वि॑शन्तु। दी॒र्घमायु॒र्यज॑मानाय कृ॒ण्वन्। अधा॒मृते॑न जरि॒तार॑मङ्ग्धि। इन्द्र॑ शु॒नाव॒द्वित॑नोति॒ सीरम्। सं॒व॒त्स॒रस्य॑ प्रति॒माण॑मे॒तत्। अ॒र्कस्य॒ ज्योति॒स्तदिदा॑स॒ ज्येष्ठम्। सं॒व॒त्स॒र शु॒नव॒त्सीर॑मे॒तत्। इन्द्र॑स्य॒ राध॒ प्रय॑तं पु॒रु त्मना। तद॑र्करू॒पं वि॒मिमा॑नमेति। द्वाद॑शारे॒ प्रति॑तिष्ठ॒तीद्वृषा। अ॒श्वा॒यन्तो॑ ग॒व्यन्तो॑ वा॒जय॑न्तः। हवा॑महे॒ त्वोप॑गन्त॒वा उ॑। आ॒भूष॑न्तस्त्वा सुम॒तौ नवा॑याम्। व॒यमि॑न्द्र त्वा शु॒न हु॑वेम॥४५॥\anuvakamend[अ॒र्च॒त॒ ह॒विर्गा॑यत यसच्चर्\mbox{}षणी॒नां वै॑शम्भ॒ल्या हा॑सी॒त्त्वमु॒रुं दे॒वहू॑तौ न॒स्त्मना॒ षट्च॑]




\prashnaend{प्रा॒ण उ॒देहि॒ पुन॒रा नो॑ भर य॒ज्ञो रा॒यो वार्त्र॑हत्याय॒ वसू॑ना॒ स ईं पाह्य॒ष्टौ॥८॥}{प्रा॒णो र॑क्ष॒त्यगृ॑भीता धाराव॒रा म॒रुतो॑ दीर्घायु॒त्वाय॒ ज्योति॑षा त्वा॒ पञ्च॑चत्वारिशत्॥४५॥}{प्रा॒णः शु॒न हु॑वेम॥}{हरि॑ ओम्॥}{इति श्रीकृष्णयजुर्वेदीयतैत्तिरीयब्राह्मणे द्वितीयाष्टके पञ्चमः प्रपाठकः समाप्तः॥}
\clearpage
