\sect{प्रथमः प्रश्नः}
\setcounter{anuvakam}{0}
\dnsub{तैत्तिरीयब्राह्मणे द्वितीयाष्टके प्रथमः प्रपाठकः}

%2.1.1.1
अङ्गि॑रसो॒ वै स॒त्रमा॑सत। तेषां॒ पृश़्ञि॑र्घर्म॒धुगा॑सीत्। सर्जी॒षेणा॑जीवत्। तेऽब्रुवन्। कस्मै॒ नु स॒त्रमास्महे। येऽस्या ओष॑धी॒र्न ज॒नया॑म॒ इति॑। ते दि॒वो वृष्टि॑मसृजन्त। याव॑न्तः स्तो॒का अ॒वाप॑द्यन्त। ताव॑ती॒रोष॑धयोऽजायन्त। ता जा॒ताः पि॒तरो॑ वि॒षेणा॑लिम्पन्॥१॥

%2.1.1.2
तासाञ्ज॒ग्ध्वा रुप्य॒न्त्यैत्। तेऽब्रुवन्। क इ॒दमि॒त्थम॑क॒रिति॑। व॒यं भा॑ग॒धेय॑मि॒च्छमा॑ना॒ इति॑ पि॒तरोऽब्रुवन्। किव्वोँ॑ भाग॒धेय॒मिति॑। अ॒ग्नि॒हो॒त्र ए॒व नोऽप्य॒स्त्वित्य॑ब्रुवन्। तेभ्य॑ ए॒तद्भा॑ग॒धेयं॒ प्राय॑च्छन्। यद्धु॒त्वा नि॒मार्ष्टि॑। ततो॒ वै त ओष॑धीरस्वदयन्। य ए॒वं वेद॑॥२॥

%2.1.1.3
स्वद॑न्तेऽस्मा॒ ओष॑धयः। ते व॒त्समु॒पावा॑सृजन्। इ॒दन्नो॑ ह॒व्यं प्रदा॑प॒येति॑। सोऽब्रवी॒द्वरं॑ वृणै। दश॑ मा॒ रात्रीर्जा॒तन्न दो॑हन्। आ॒स॒ङ्ग॒वं मा॒त्रा स॒ह च॑रा॒णीति॑। तस्माद्व॒त्सञ्जा॒तन्दश॒ रात्री॒र्न दु॑हन्ति। आ॒स॒ङ्ग॒वं मा॒त्रा स॒ह च॑रति। वारे॑वृत॒ ह्य॑स्य। तस्माद्व॒त्स ससृष्टध॒य रु॒द्रो घातु॑कः। अति॒ हि स॒न्धान्धय॑ति॥३॥\anuvakamend[अ॒लि॒म्प॒न्वेद॒ घातु॑क॒ एकं च]

%2.1.2.1
प्र॒जाप॑तिर॒ग्निम॑सृजत। तं प्र॒जा अन्व॑सृज्यन्त। तम॑भा॒ग उपास्त। सोऽस्य प्र॒जाभि॒रपाक्रामत्। तम॑व॒रुरु॑त्समा॒नोऽन्वैत्। तम॑व॒रुध॒न्नाश॑क्नोत्। स तपो॑ऽतप्यत। सोऽग्निरुपा॑रम॒ताता॑पि॒ वै स्य प्र॒जाप॑ति॒रिति॑। स र॒राटा॒दुद॑मृष्ट॥४॥

%2.1.2.2
तद्घृ॒तम॑भवत्। तस्मा॒द्यस्य॑ दक्षिण॒तः केशा॒ उन्मृ॑ष्टाः। ताञ्ज्येष्ठल॒क्ष्मी प्रा॑जाप॒त्येत्या॑हुः। यद्र॒राटा॑दु॒दमृ॑ष्ट। तस्माद्र॒राटे॒ केशा॒ न स॑न्ति। तद॒ग्नौ प्रागृ॑ह्णात्। तद्व्य॑चिकित्सत्। जु॒हवा॒नी ३ मा हौ॒षा ३ मिति॑। तद्वि॑चिकि॒त्सायै॒ जन्म॑। य ए॒वं वि॒द्वान् वि॑चि॒कित्स॑ति॥५॥

%2.1.2.3
वसी॑य ए॒व चे॑तयते। तं वाग॒भ्य॑वदज्जु॒हुधीति॑। सोऽब्रवीत्। कस्त्वम॒सीति॑। स्वैव ते॒ वागित्य॑ब्रवीत्। सो॑ऽजुहो॒त्स्वाहेति॑। तत्स्वा॑हाका॒रस्य॒ जन्म॑। य ए॒वस्वा॑हाका॒रस्य॒ जन्म॒ वेद॑। क॒रोति॑ स्वाहाका॒रेण॑ वी॒र्यम्। यस्यै॒वं वि॒दुष॑ स्वाहाका॒रेण॒ जुह्व॑ति॥६॥

%2.1.2.4
भोगा॑यै॒वास्य॑ हु॒तं भ॑वति। तस्या॒ आहु॑त्यै॒ पुरु॑षमसृजत। द्वि॒तीय॑मजुहोत्। सोऽश्व॑मसृजत। तृ॒तीय॑मजुहोत्। स गाम॑सृजत। च॒तु॒र्थम॑जुहोत्। सोऽवि॑मसृजत। प॒ञ्च॒मम॑जुहोत्। सो॑ऽजाम॑सृजत॥७॥

%2.1.2.5
सोऽग्निर॑बिभेत्। आहु॑तीभि॒र्वै माऽऽप्नो॒तीति॑। स प्र॒जाप॑तिं॒ पुन॒ प्रावि॑शत्। तं प्र॒जाप॑तिरब्रवीत्। जाय॒स्वेति॑। सोऽब्रवीत्। किं भा॑ग॒धेय॑म॒भि ज॑निष्य॒ इति॑। तुभ्य॑मे॒वेद हू॑याता॒ इत्य॑ब्रवीत्। स ए॒तद्भा॑ग॒धेय॑म॒भ्य॑जायत। यद॑ग्निहो॒त्रम्॥८॥

%2.1.2.6
तस्मा॑दग्निहो॒त्रमु॑च्यते। तद्धू॒यमा॑नमादि॒त्योऽब्रवीत्। मा हौ॑षीः। उ॒भयो॒र्वै ना॑वे॒तदिति॑। सोऽग्निर॑ब्रवीत्। क॒थन्नौ॑ होष्य॒न्तीति॑। सा॒यमे॒व तुभ्यं॑ जु॒हुव\sn{}। प्रा॒तर्मह्य॒मित्य॑ब्रवीत्। तस्मा॑द॒ग्नये॑ सा॒य हू॑यते। सूर्या॑य प्रा॒तः॥९॥

%2.1.2.7
आ॒ग्ने॒यी वै रात्रि॑। ऐ॒न्द्रमह॑। यदनु॑दिते॒ सूर्ये प्रा॒तर्जु॑हु॒यात्। उ॒भय॑मे॒वाग्ने॒य स्यात्। उदि॑ते॒ सूर्ये प्रा॒तर्जु॑होति। तथा॒ग्नये॑ सा॒य हू॑यते। सूर्या॑य प्रा॒तः। रात्रिं॒ वा अनु॑ प्र॒जाः प्र जा॑यन्ते। अह्ना॒ प्रति॑ तिष्ठन्ति। यत्सा॒यं जु॒होति॑॥१०॥

%2.1.2.8
प्रैव तेन॑ जायते। उदि॑ते॒ सूर्ये प्रा॒तर्जु॑होति। प्रत्ये॒व तेन॑ तिष्ठति। प्र॒जाप॑तिरकामयत॒ प्रजा॑ये॒येति॑। स ए॒तद॑ग्निहो॒त्रं मि॑थु॒नम॑पश्यत्। तदुदि॑ते॒ सूर्ये॑ऽजुहोत्। यजु॑षा॒ऽन्यत्। तू॒ष्णीम॒न्यत्। ततो॒ वै स प्राजा॑यत। यस्यै॒वं वि॒दुष॒ उदि॑ते॒ सूर्येऽग्निहो॒त्रं जुह्व॑ति॥११॥

%2.1.2.9
प्रैव जा॑यते। अथो॒ यथा॒ दिवा प्रजा॒नन्नेति॑। ता॒दृगे॒व तत्। अथो॒ खल्वा॑हुः। यस्य॒ वै द्वौ पुण्यौ॑ गृ॒हे वस॑तः। यस्तयो॑र॒न्य रा॒धय॑त्य॒न्यन्न। उ॒भौ वाव स तावृ॑च्छ॒तीति॑। अ॒ग्निं वावादि॒त्यः सा॒यं प्र वि॑शति। तस्मा॑द॒ग्निर्दू॒रान्नक्त॑न्ददृशे। उ॒भे हि तेज॑सी सं॒ पद्ये॑ते॥१२॥

%2.1.2.10
उ॒द्यन्तं॒ वावादि॒त्यम॒ग्निरनु॑ स॒मारो॑हति। तस्माद्धू॒म ए॒वाग्नेर्दिवा॑ ददृशे। यद॒ग्नये॑ सा॒यं जु॑हु॒यात्। आ सूर्या॑य वृश्च्येत। यत्सूर्या॑य प्रा॒तर्जु॑हु॒यात्। आऽग्नये॑ वृश्च्येत। दे॒वताभ्यः स॒मद॑न्दध्यात्। अ॒ग्निर्ज्योति॒र्ज्योति॒ सूर्य॒ स्वाहेत्ये॒व सा॒य हो॑त॒व्यम्। सूर्यो॒ ज्योति॒र्ज्योति॑र॒ग्निः स्वाहेति॑ प्रा॒तः। तथो॒भाभ्या सा॒य हू॑यते॥१३॥

%2.1.2.11
उ॒भाभ्यां प्रा॒तः। न दे॒वताभ्यः स॒मद॑न्दधाति। अ॒ग्निर्ज्योति॒रित्या॑ह। अ॒ग्निर्वै रे॑तो॒धाः। प्र॒जा ज्योति॒रित्या॑ह। प्र॒जा ए॒वास्मै॒ प्र ज॑नयति। सूर्यो॒ ज्योति॒रित्या॑ह। प्र॒जास्वे॒व प्रजा॑तासु॒ रेतो॑ दधाति। ज्योति॑र॒ग्निः स्वाहेत्या॑ह। प्र॒जा ए॒व प्रजा॑ता अ॒स्यां प्रति॑ष्ठापयति॥१४॥

%2.1.2.12
तू॒ष्णीमुत्त॑रा॒माहु॑तिं जुहोति। मि॒थु॒न॒त्वाय॒ प्रजात्यै। यदुदि॑ते॒ सूर्ये प्रा॒तर्जु॑हु॒यात्। यथाऽति॑थये॒ प्रद्रु॑ताय शू॒न्याया॑वस॒थाया॑हा॒र्य हर॑न्ति। ता॒दृगे॒व तत्। क्वाह॒ तत॒स्तद्भव॒तीत्या॑हुः। यत्स न वेद॑। यस्मै॒ तद्धर॒न्तीति॑। तस्मा॒द्यदौ॑ष॒सं जु॒होति॑। तदे॒व सं॑प्र॒ति। अथो॒ यथा॒ प्रार्थ॑मौष॒सं प॑रि॒वेवेष्टि। ता॒दृगे॒व तत्॥१५॥\anuvakamend[अ॒मृ॒ष्ट॒ वि॒चि॒कित्स॑ति॒ जुह्व॑त्य॒जाम॑सृजताग्निहो॒त्र सूर्या॑य प्रा॒तर्जु॒होति॒ जुह्व॑ति सं॒पद्ये॑ते हूयते स्थापयति संप्र॒ति द्वे च॑]

%2.1.3.1
रु॒द्रो वा ए॒षः। यद॒ग्निः। पत्नी स्था॒ली। यन्मध्ये॒ऽग्नेर॑धि॒श्रयेत्। रु॒द्राय॒ पत्नी॒मपि॑ दध्यात्। प्र॒मायु॑का स्यात्। उदी॒चोऽङ्गा॑रान्नि॒रूह्याधि॑ श्रयति। पत्नि॑यै गोपी॒थाय॑। व्य॑न्तान्करोति। तथा॒ पत्न्यप्र॑मायुका भवति॥१६॥

%2.1.3.2
घ॒र्मो वा ए॒षोऽशान्तः। अह॑रह॒ प्र वृ॑ज्यते। यद॑ग्निहो॒त्रम्। प्रति॑षिञ्चेत्प॒शुका॑मस्य। शा॒न्तमि॑व॒ हि प॑श॒व्यम्। न प्रति॑षिञ्चेद्ब्रह्मवर्च॒सका॑मस्य। समि॑द्धमिव॒ हि ब्र॑ह्मवर्च॒सम्। अथो॒ खलु॑। प्र॒ति॒षिच्य॑मे॒व। यत्प्र॑तिषि॒ञ्चति॑॥१७॥

%2.1.3.3
तत्प॑श॒व्यम्। यज्जु॒होति॑। तद्ब्र॑ह्मवर्च॒सि। उ॒भय॑मे॒वाक॑। प्रच्यु॑तं॒ वा ए॒तद॒स्माल्लो॒कात्। अग॑तन्देवलो॒कम्। यच्छृ॒त ह॒विरन॑भिघारितम्। अ॒भि द्यो॑तयति। अ॒भ्ये॑वैन॑द्घारयति। अथो॑ देव॒त्रैवैन॑द्गमयति॥१८॥

%2.1.3.4
पर्य॑ग्नि करोति। रक्ष॑सा॒मप॑हत्यै। त्रिः पर्य॑ग्नि करोति। त्र्या॑वृ॒द्धि य॒ज्ञः। अथो॑ मेध्य॒त्वाय॑। यत्प्रा॒चीन॑मुद्वा॒सयेत्। यज॑मान शु॒चाऽर्प॑येत्। यद्द॑क्षि॒णा। पि॒तृ॒दे॒व॒त्य स्यात्। यत्प्र॒त्यक्॥१९॥

%2.1.3.5
पत्नी शु॒चाऽर्प॑येत्। उ॒दी॒चीन॒मुद्वा॑सयति। ए॒षा वै दे॑वमनु॒ष्याणा शा॒न्ता दिक्। तामे॒वैन॒दनूद्वा॑सयति॒ शान्त्यै। वर्त्म॑ करोति। य॒ज्ञस्य॒ सन्त॑त्यै। निष्ट॑पति। उपै॒व तत्स्तृ॑णाति। च॒तुरुन्न॑यति। चतु॑ष्पादः प॒शव॑॥२०॥

%2.1.3.6
प॒शूने॒वाव॑रुन्धे। सर्वान्पू॒र्णानुन्न॑यति। सर्वे॒ हि पुण्या॑ रा॒द्धाः। अ॒नूच॒ उन्न॑यति। प्र॒जाया॑ अनूचीन॒त्वाय॑। अ॒नूच्ये॒वास्य॑ प्र॒जाऽर्धु॑का भवति। संमृ॑शति॒ व्यावृ॑त्त्यै। नाहोष्य॒न्नुप॑ सादयेत्। यदहोष्यन्नुपसा॒दयेत्। यथा॒ऽन्यस्मा॑ उपनि॒धाय॑॥२१॥

%2.1.3.7
अ॒न्यस्मै प्र॒यच्छ॑ति। ता॒दृगे॒व तत्। आऽस्मै॑ वृश्च्येत। यदे॒व गार्\mbox{}ह॑पत्येऽधि॒ श्रय॑ति। तेन॒ गार्\mbox{}ह॑पत्यं प्रीणाति। अ॒ग्निर॑बिभेत्। आहु॑तयो॒ माऽत्येष्य॒न्तीति॑। स ए॒ता स॒मिध॑मपश्यत्। तामाऽध॑त्त। ततो॒ वा अ॒ग्नावाहु॑तयोऽध्रियन्त॥२२॥

%2.1.3.8
यदे॑न स॒मय॑च्छत्। तत्स॒मिध॑ समि॒त्त्वम्। स॒मिध॒मा द॑धाति। समे॒वैनं॑ यच्छति। आहु॑तीना॒न्धृत्यै। अथो॑ अग्निहो॒त्रमे॒वेध्मव॑त्करोति। आहु॑तीनां॒ प्रति॑ष्ठित्यै। ब्र॒ह्म॒वा॒दिनो॑ वदन्ति। यदेका स॒मिध॑मा॒धाय॒ द्वे आहु॑ती जु॒होति॑। अथ॒ कस्या स॒मिधि॑ द्वि॒तीया॒माहु॑तिं जुहो॒तीति॑॥२३॥

%2.1.3.9
यद्द्वे स॒मिधा॑वा द॒ध्यात्। भ्रातृ॑व्यमस्मै जनयेत्। एका स॒मिध॑मा॒धाय॑। यजु॑षा॒ऽन्यामाहु॑तिं जुहोति। उ॒भे ए॒व स॒मिद्व॑ती॒ आहु॑ती जुहोति। नास्मै॒ भ्रातृ॑व्यञ्जनयति। आदीप्तायां जुहोति। समि॑द्धमिव॒ हि ब्र॑ह्मवर्च॒सम्। अथो॒ यथाऽति॑थिं॒ ज्योति॑ष्कृ॒त्वा प॑रि॒ वेवेष्टि। ता॒दृगे॒व तत्। च॒तुरुन्न॑यति। द्विर्जु॑होति। तस्माद्द्वि॒पाच्चतु॑ष्पादमत्ति। अथो द्वि॒पद्ये॒व चतु॑ष्पद॒ प्रति॑ ष्ठापयति॥२४॥\anuvakamend[भ॒व॒ति॒ प्र॒ति॒षि॒ञ्चति॑ गमयति प्र॒त्यक्प॒शव॑ उपनि॒धायाध्रिय॒न्तेति॒ तच्च॒त्वारि॑ च]

%2.1.4.1
उ॒त्त॒राव॑तीँ॒व्वै दे॒वा आहु॑ति॒मजु॑हवुः। अवा॑ची॒मसु॑राः। ततो॑ दे॒वा अभ॑वन्। पराऽसु॑राः। यङ्का॒मये॑त॒ वसी॑यान्त्स्या॒दिति॑। कनी॑य॒स्तस्य॒ पूर्व हु॒त्वा। उत्त॑रं॒ भूयो॑ जुहुयात्। ए॒षा वा उ॑त्त॒राव॒त्याहु॑तिः। तान्दे॒वा अ॑जुहवुः। तत॒स्ते॑ऽभवन्॥२५॥

%2.1.4.2
यस्यै॒वं जुह्व॑ति। भव॑त्ये॒व। यङ्का॒मये॑त॒ पापी॑यान्त्स्या॒दिति॑। भूय॒स्तस्य॒ पूर्व हु॒त्वा। उत्त॑र॒ङ्कनी॑यो जुहुयात्। ए॒षा वा अवा॒च्याहु॑तिः। तामसु॑रा अजुहवुः। तत॒स्ते परा॑ऽभवन्। यस्यै॒वं जुह्व॑ति। परै॒व भ॑वति॥२६॥

%2.1.4.3
हु॒त्वोप॑ सादय॒त्यजा॑मित्वाय। अथो॒ व्यावृ॑त्त्यै। गार्\mbox{}ह॑पत्यं॒ प्रतीक्षते। अन॑नुध्यायिनमे॒वैनं॑ करोति। अ॒ग्नि॒हो॒त्रस्य॒ वै स्था॒णुर॑स्ति। तं य ऋ॒च्छेत्। य॒ज्ञ॒स्था॒णुमृ॑च्छेत्। ए॒ष वा अ॑ग्निहो॒त्रस्य॑ स्था॒णुः। यत्पूर्वाऽऽहु॑तिः। तां यदुत्त॑रया॒ऽभि जु॑हु॒यात्॥२७॥

%2.1.4.4
य॒ज्ञ॒स्था॒णुमृ॑च्छेत्। अ॒ति॒हाय॒ पूर्वा॒माहु॑तिं जुहोति। य॒ज्ञ॒स्था॒णुमे॒व परि॑ वृणक्ति। अथो॒ भ्रातृ॑व्यमे॒वाप्त्वाऽति॑ क्रामति। अ॒वा॒चीन सा॒यमुप॑मार्ष्टि। रेत॑ ए॒व तद्द॑धाति। ऊ॒र्ध्वं प्रा॒तः। प्र ज॑नयत्ये॒व तत्। ब्र॒ह्म॒वा॒दिनो॑ वदन्ति। च॒तुरुन्न॑यति॥२८॥

%2.1.4.5
द्विर्जु॑होति। अथ॒ क्व॑ द्वे आहु॑ती भवत॒ इति॑। अ॒ग्नौ वैश्वान॒र इति॑ ब्रूयात्। ए॒ष वा अ॒ग्निर्वैश्वान॒रः। यद्ब्राह्म॒णः। हु॒त्वा द्विः प्राश्ञा॑ति। अ॒ग्नावे॒व वैश्वान॒रे द्वे आहु॑ती जुहोति। द्विर्जु॒होति॑। द्विर्निमार्ष्टि। द्विः प्राश्ञा॑ति॥२९॥

%2.1.4.6
षट्त्संप॑द्यन्ते। षड्वा ऋ॒तव॑। ऋ॒तूने॒व प्री॑णाति। ब्र॒ह्म॒वा॒दिनो॑ वदन्ति। किं॒ दे॒व॒त्य॑मग्निहो॒त्रमिति॑। वै॒श्व॒दे॒वमिति॑ ब्रूयात्। यद्यजु॑षा जु॒होति॑। तदैन्द्रा॒ग्नम्। यत्तू॒ष्णीम्। तत्प्रा॑जाप॒त्यम्॥३०॥

%2.1.4.7
यन्नि॒मार्ष्टि॑। तदोष॑धीनाम्। यद्द्वि॒तीयम्। तत्पि॑तृ॒णाम्। यत्प्राश्ञा॑ति। तद्गर्भा॑णाम्। तस्मा॒द्गर्भा॒ अन॑श्ञन्तो वर्धन्ते। यदा॒चाम॑ति। तन्म॑नु॒ष्या॑णाम्। उद॑ङ्पर्या॒वृत्याचा॑मति॥३१॥

%2.1.4.8
आ॒त्मनो॑ गोपी॒थाय॑। निर्णे॑नेक्ति॒ शुद्ध्यै। निष्ट॑पति स्व॒गाकृ॑त्यै। उद्दि॑शति। स॒प्त॒र्॒षीने॒व प्री॑णाति। द॒क्षि॒णा प॒र्याव॑र्तते। स्वमे॒व वी॒र्य॑मनु॑ प॒र्याव॑र्तते। तस्मा॒द्दक्षि॒णोऽर्ध॑ आ॒त्मनो॑ वी॒र्या॑वत्तरः। अथो॑ आदि॒त्यस्यै॒वावृत॒मनु॑ प॒र्याव॑र्तते। हु॒त्वोप॒ समि॑न्धे॥३२॥

%2.1.4.9
ब्र॒ह्म॒व॒र्च॒सस्य॒ समि॑द्ध्यै। न ब॒र्॒हिरनु॒ प्र ह॑रेत्। असस्थितो॒ वा ए॒ष य॒ज्ञः। यद॑ग्निहो॒त्रम्। यद॑नु प्र॒हरेत्। य॒ज्ञं विच्छि॑न्द्यात्। तस्मा॒न्नानु॑ प्र॒हृत्यम्। य॒ज्ञस्य॒ सन्त॑त्यै। अ॒पो नि न॑यति। अ॒व॒भृ॒थस्यै॒व रू॒पम॑कः॥३३॥\anuvakamend[अ॒भ॒व॒न्भ॒व॒ति॒ जु॒हु॒यान्न॑यति मार्ष्टि॒ द्विः प्राश्ञा॑ति प्राजाप॒त्यमाचा॑मतीन्धेऽकः]

%2.1.5.1
ब्र॒ह्म॒वा॒दिनो॑ वदन्ति। अ॒ग्नि॒हो॒त्रप्रा॑यणा य॒ज्ञाः। किंप्रा॑यणमग्निहो॒त्रमिति॑। व॒त्सो वा अ॑ग्निहो॒त्रस्य॒ प्राय॑णम्। अ॒ग्नि॒हो॒त्रं य॒ज्ञानाम्। तस्य॑ पृथि॒वी सद॑। अ॒न्तरि॑क्ष॒माग्नीद्ध्रम्। द्यौर्\mbox{}ह॑वि॒र्धानम्। दि॒व्या आप॒ प्रोक्ष॑णयः। ओष॑धयो ब॒र्॒हिः॥३४॥

%2.1.5.2
वन॒स्पत॑य इ॒ध्मः। दिश॑ परि॒धय॑। आ॒दि॒त्यो यूप॑। यज॑मानः प॒शुः। स॒मु॒द्रो॑ऽवभृ॒थः। सं॒व॒त्स॒रः स्व॑गाका॒रः। तस्मा॒दाहि॑ताग्ने॒ सर्व॑मे॒व ब॑र्\mbox{}हि॒ष्य॑न्द॒त्तं भ॑वति। यत्सा॒यं जु॒होति॑। रात्रि॑मे॒व तेन॑ दक्षि॒ण्यां कुरुते। यत्प्रा॒तः॥३५॥

%2.1.5.3
अह॑रे॒व तेन॑ दक्षि॒ण्यं॑ कुरुते। यत्ततो॒ ददा॑ति। सा दक्षि॑णा। याव॑न्तो॒ वै दे॒वा अहु॑त॒माद\sn{}। ते परा॑ऽभवन्। त ए॒तद॑ग्निहो॒त्र सर्व॑स्यै॒व स॑मव॒दाया॑जुहवुः। तस्मा॑दाहुः। अ॒ग्नि॒हो॒त्रं वै दे॒वा गृ॒हाणां॒ निष्कृ॑तिमपश्य॒न्निति॑। यत्सा॒यं जु॒होति॑। रात्रि॑या ए॒व तद्धु॒ताद्या॑य॥३६॥

%2.1.5.4
यज॑मान॒स्याप॑राभावाय। यत्प्रा॒तः। अह्न॑ ए॒व तद्धु॒ताद्या॑य। यज॑मान॒स्याप॑राभावाय। यत्ततो॒ऽश़्ञाति॑। हु॒तमे॒व तत्। द्वयो॒ पय॑सा जुहुयात्प॒शुका॑मस्य। ए॒तद्वा अ॑ग्निहो॒त्रं मि॑थु॒नम्। य ए॒वं वेद॑। प्र प्र॒जया॑ प॒शुभि॑र्मिथु॒नैर्जा॑यते॥३७॥

%2.1.5.5
इ॒मामे॒व पूर्व॑या दु॒हे। अ॒मूमुत्त॑रया। अ॒धि॒श्रित्योत्त॑र॒मा न॑यति। योना॑वे॒व तद्रेत॑ सिञ्चति प्र॒जन॑ने। आज्ये॑न जुहुया॒त्तेज॑स्कामस्य। तेजो॒ वा आज्यम्। ते॒ज॒स्व्ये॑व भ॑वति। पय॑सा प॒शुका॑मस्य। ए॒तद्वै प॑शू॒ना रू॒पम्। रू॒पेणै॒वास्मै॑ प॒शूनव॑रुन्धे॥३८॥

%2.1.5.6
प॒शु॒माने॒व भ॑वति। द॒ध्नेन्द्रि॒यका॑मस्य। इ॒न्द्रि॒यं वै दधि॑। इ॒न्द्रि॒या॒व्ये॑व भ॑वति। य॒वा॒ग्वा ग्राम॑कामस्यौष॒धा वै म॑नु॒ष्या। भा॒ग॒धेये॑नै॒वास्मै॑ सजा॒तानव॑ रुन्धे। ग्रा॒म्ये॑व भ॑वति। अय॑ज्ञो॒ वा ए॒षः। यो॑ऽसा॒मा॥३९॥

%2.1.5.7
च॒तुरुन्न॑यति। चतु॑रक्षर रथन्त॒रम्। र॒थ॒न्त॒रस्यै॒ष वर्ण॑। उ॒परी॑व हरति। अ॒न्तरि॑क्षं वामदे॒व्यम्। वा॒म॒दे॒व्यस्यै॒ष वर्ण॑। द्विर्जु॑होति। द्व्य॑क्षरं बृ॒हत्। बृ॒ह॒त ए॒ष वर्ण॑। अ॒ग्नि॒हो॒त्रमे॒व तत्साम॑न्वत्करोति॥४०॥

%2.1.5.8
यो वा अ॑ग्निहो॒त्रस्यो॑प॒सदो॒ वेद॑। उपै॑नमुप॒सदो॑ नमन्ति। वि॒न्दत॑ उपस॒त्तारम्। उ॒न्नीयोप॑ सादयति। पृ॒थि॒वीमे॒व प्री॑णाति। हो॒ष्यन्नुप॑सादयति। अ॒न्तरि॑क्षमे॒व प्री॑णाति। हु॒त्वोप॑ सादयति। दिव॑मे॒व प्री॑णाति। ए॒ता वा अ॑ग्निहो॒त्रस्यो॑प॒सद॑॥४१॥

%2.1.5.9
य ए॒वं वेद॑। उपै॑नमुप॒सदो॑ नमन्ति। वि॒न्दत॑ उपस॒त्तारम्। यो वा अ॑ग्निहो॒त्रस्याश्रा॑वितं प्र॒त्याश्रा॑वित॒ होता॑रं ब्र॒ह्माणं॑ वषट्का॒रं वेद॑। तस्य॒ त्वे॑व हु॒तम्। प्रा॒णो वा अ॑ग्निहो॒त्रस्याश्रा॑वितम्। अ॒पा॒नः प्र॒त्याश्रा॑वितम्। मनो॒ होता। चक्षु॑र्ब्र॒ह्मा। नि॒मे॒षो व॑षट्का॒रः॥४२॥

%2.1.5.10
य ए॒वं वेद॑। तस्य॒ त्वे॑व हु॒तम्। सा॒यं॒ यावा॑नश्च॒ वै दे॒वाः प्रा॑त॒र्यावा॑णश्चाग्निहो॒त्रिणो॑ गृ॒हमाग॑च्छन्ति। तान् यन्न त॒र्पयेत्। प्र॒जयाऽस्य प॒शुभि॒र्वि ति॑ष्ठेरन्। यत्त॒र्पयेत्। तृ॒प्ता ए॑नं प्र॒जया॑ प॒शुभि॑स्तर्पयेयुः। स॒जूर्दे॒वैः सा॒यं याव॑भि॒रिति॑ सा॒य संमृ॑शति। स॒जूर्दे॒वैः प्रा॒तर्याव॑भि॒रिति॑ प्रा॒तः। ये चै॒व दे॒वाः सा॑यं॒ यावा॑नो॒ ये च॑ प्रात॒र्यावा॑णः॥४३॥

%2.1.5.11
ताने॒वोभयास्तर्पयति। त ए॑नन्तृ॒प्ताः प्र॒जया॑ प॒शुभि॑स्तर्पयन्ति। अ॒रु॒णो ह॑ स्मा॒हौप॑वेशिः। अ॒ग्नि॒हो॒त्र ए॒वाह सा॒यंप्रा॑त॒र्वज्रं॒ भ्रातृ॑व्येभ्य॒ प्र ह॑रामि। तस्मा॒न्मत्पापी॑यासो॒ भ्रातृ॑व्या॒ इति॑। च॒तुरुन्न॑यति। द्विर्जु॑होति। स॒मित्स॑प्त॒मी। स॒प्तप॑दा॒ शक्व॑री। शा॒क्व॒रो वज्र॑। अ॒ग्नि॒हो॒त्र ए॒व तत्सा॒यंप्रा॑त॒र्वज्रं॒ यज॑मानो॒ भ्रातृ॑व्याय॒ प्र ह॑रति। भव॑त्या॒त्मना। पराऽस्य॒ भ्रातृ॑व्यो भवति॥४४॥\anuvakamend[ब॒र्॒हिः प्रा॒तर्\mbox{}हु॒ताद्या॑य जायते रुन्धेऽसा॒मा क॑रोत्ये॒ता वा अ॑ग्निहो॒त्रस्यो॑प॒सदो॑ वषट्का॒रश्च॑ प्रात॒र्यावा॑णो॒ वज्र॒स्त्रीणि॑ च]

%2.1.6.1
प्र॒जाप॑तिरकामयतात्म॒न्वन्मे॑ जाये॒तेति॑। सो॑ऽजुहोत्। तस्मात्म॒न्वद॑जायत। अ॒ग्निर्वा॒युरा॑दि॒त्यः। तेऽब्रुवन्। प्र॒जाप॑तिरहौषीदात्म॒न्वन्मे॑ जाये॒तेति॑। तस्य॑ व॒यम॑जनिष्महि। जाय॑तान्न आत्म॒न्वदिति॒ ते॑ऽजुहवुः। प्रा॒णाना॑म॒ग्निः। त॒नुवै॑ वा॒युः॥४५॥

%2.1.6.2
चक्षु॑ष आदि॒त्यः। तेषा हु॒ताद॑जायत॒ गौरे॒व। तस्यै॒ पय॑सि॒ व्याय॑च्छन्त। मम॑ हु॒ताद॑जनि॒ ममेति॑। ते प्र॒जाप॑तिं प्र॒श्ञमा॑यन्। स आ॑दि॒त्योऽग्निम॑ब्रवीत्। य॒त॒रो नौ॒ जयात्। तन्नौ॑ स॒हास॒दिति॑। कस्यै कोऽहौ॑षी॒दिति॑ प्र॒जाप॑तिरब्रवी॒त्कस्यै क॒ इति॑। प्रा॒णाना॑म॒हमित्य॒ग्निः॥४६॥

%2.1.6.3
त॒नुवा॑ अ॒हमिति॑ वा॒युः। चक्षु॑षो॒ऽहमित्या॑दि॒त्यः। य ए॒व प्रा॒णाना॒महौ॑षीत्। तस्य॑ हु॒ताद॑ज॒नीति॑। अ॒ग्नेर्\mbox{}हु॒ताद॑ज॒नीति॑। तद॑ग्निहो॒त्रस्याग्निहोत्र॒त्वम्। गौर्वा अ॑ग्निहो॒त्रम्। य ए॒वं वेद॒ गौर॑ग्निहो॒त्रमिति॑। प्रा॒णा॒पा॒नाभ्या॑मे॒वाग्नि सम॑र्धयति। अव्य॑र्धुकः प्राणापा॒नाभ्यां भवति॥४७॥

%2.1.6.4
य ए॒वं वेद॑। तौ वा॒युर॑ब्रवीत्। अनु॒ मा भ॑जत॒मिति॑। यदे॒व गार्\mbox{}ह॑पत्येऽधि॒श्रित्या॑हव॒नीय॑म॒भ्यु॑द्द्रवान्॑। तेन॒ त्वां प्री॑णा॒नित्य॑ब्रूताम्। तस्मा॒द्यद्गार्\mbox{}ह॑पत्येऽधि॒श्रित्या॑हव॒नीय॑म॒भ्यु॑द्द्रव॑ति। वा॒युमे॒व तेन॑ प्रीणाति। प्र॒जाप॑तिर्दे॒वता सृ॒जमा॑नः। अ॒ग्निमे॒व दे॒वता॑नां प्रथ॒मम॑सृजत। सोऽन्यदा॑ल॒म्भ्य॑मवि॑त्वा॥४८॥

%2.1.6.5
प्र॒जाप॑तिम॒भि प॒र्याव॑र्तत। स मृ॒त्योर॑बिभेत्। सो॑ऽमुमा॑दि॒त्यमा॒त्मनो॒ निर॑मिमीत। त हु॒त्वा पराङ्प॒र्याव॑र्तत। ततो॒ वै स मृ॒त्युमपा॑जयत्। अप॑ मृ॒त्युं ज॑यति। य ए॒वं वेद॑। तस्मा॒द्यस्यै॒वं वि॒दुष॑। उ॒तैका॒हमु॒त द्व्य॒हन्न जुह्व॑ति। हु॒तमे॒वास्य॑ भवति। अ॒सौ ह्या॑दि॒त्योऽग्निहो॒त्रम्॥४९॥\anuvakamend[त॒नुवै॑ वा॒युर॒ग्निर्भ॑व॒त्यवि॑त्वा भव॒त्येकं च]

%2.1.7.1
रौ॒द्रङ्गवि॑। वा॒य॒व्य॑मुप॑सृष्टम्। आ॒श्वि॒नन्दु॒ह्यमा॑नम्। सौ॒म्यन्दु॒ग्धम्। वा॒रु॒णमधि॑ श्रितम्। वै॒श्व॒दे॒वा भि॒न्दव॑। पौ॒ष्णमुद॑न्तम्। सा॒र॒स्व॒तं वि॒ष्यन्द॑मानम्। मै॒त्र शर॑। धा॒तुरुद्वा॑सितम्। बृह॒स्पते॒रुन्नी॑तम्। स॒वि॒तुः प्र क्रान्तम्। द्या॒वा॒पृ॒थि॒व्य ह्रि॒यमा॑णम्। ऐ॒न्द्रा॒ग्नमुप॑सन्नम्। अ॒ग्नेः पूर्वाऽऽहु॑तिः। प्र॒जाप॑ते॒रुत्त॑रा। ऐ॒न्द्र हु॒तम्॥५०॥\anuvakamend[उद्वा॑सित स॒प्त च॑]

%2.1.8.1
द॒क्षि॒ण॒त उप॑ सृजति। पि॒तृ॒लो॒कमे॒व तेन॑ जयति। प्राची॒मा व॑र्तयति। दे॒व॒लो॒कमे॒व तेन॑ जयति। उदी॑चीमा॒वृत्य॑ दोग्धि। म॒नु॒ष्य॒लो॒कमे॒व तेन॑ जयति। पूर्वौ॑ दुह्याज्ज्ये॒ष्ठस्य॑ ज्यैष्ठिने॒यस्य॑। यो वा॑ ग॒तश्री॒ स्यात्। अप॑रौ दुह्यात्कनि॒ष्ठस्य॑ कानिष्ठिने॒यस्य॑। यो वा॒ बुभू॑षेत्॥५१॥

%2.1.8.2
न सं मृ॑शति। पा॒प॒व॒स्य॒सस्य॒ व्यावृ॑त्त्यै। वा॒य॒व्यं॑ वा ए॒तदुप॑सृष्टम्। आ॒श्वि॒नन्दु॒ह्यमा॑नम्। मै॒त्रन्दु॒ग्धम्। अ॒र्य॒म्ण उ॑द्वा॒स्यमा॑नम्। त्वा॒ष्ट्रमु॑न्नी॒यमा॑नम्। बृह॒स्पते॒रुन्नी॑तम्। स॒वि॒तुः प्रक्रान्तम्। द्या॒वा॒पृ॒थि॒व्य ह्रि॒यमा॑णम्॥५२॥

%2.1.8.3
ऐ॒न्द्रा॒ग्नमुप॑ सादितम्। सर्वाभ्यो॒ वा ए॒ष दे॒वताभ्यो जुहोति। योऽग्निहो॒त्रं जु॒होति॑। यथा॒ खलु॒ वै धे॒नुन्ती॒र्थे त॒र्पय॑ति। ए॒वम॑ग्निहो॒त्री यज॑मानन्तर्पयति। तृप्य॑ति प्र॒जया॑ प॒शुभि॑। प्र सु॑व॒र्गं लो॒कं जा॑नाति। पश्य॑ति पु॒त्रम्। पश्य॑ति॒ पौत्रम्। प्र प्र॒जया॑ प॒शुभि॑र्मिथु॒नैर्जा॑यते। यस्यै॒वं वि॒दुषोऽग्निहो॒त्रं जुह्व॑ति। य उ॑ चैनदे॒वं वेद॑॥५३॥\anuvakamend[बुभू॑षेद्ध्रि॒यमा॑णञ्जायते॒ द्वे च॑]

%2.1.9.1
त्रयो॒ वै प्रै॑यमे॒धा आ॑सन्। तेषा॒न्त्रिरेकोऽग्निहो॒त्रम॑जुहोत्। द्विरेक॑। स॒कृदेक॑। तेषां॒ यस्त्रिरजु॑होत्। स ऋ॒चाऽजु॑होत्। यो द्विः। स यजु॑षा। यः स॒कृत्। स तू॒ष्णीम्॥५४॥

%2.1.9.2
यश्च॒ यजु॒षाऽजु॑हो॒द्यश्च॑ तू॒ष्णीम्। तावु॒भावार्ध्नुताम्। तस्मा॒द्यजु॒षाऽऽहु॑ति॒ पूर्वा॑ होत॒व्या। तू॒ष्णीमुत्त॑रा। उ॒भे ए॒वर्धी अव॑रुन्धे। अ॒ग्निर्ज्योति॒र्ज्योति॑र॒ग्निः स्वाहेति॑ सा॒यं जु॑होति। रेत॑ ए॒व तद्द॑धाति। सूर्यो॒ ज्योति॒र्ज्योति॒ः! सूर्य॒ स्वाहेति॑ प्रा॒तः। रेत॑ ए॒व हि॒तं प्र ज॑नयति। रेतो॒ वा ए॒तस्य॑ हि॒तन्न प्र जा॑यते॥५५॥

%2.1.9.3
यस्याग्निहो॒त्रमहु॑त॒ सूर्यो॒ऽभ्यु॑देति॑। यद्यन्ते॒ स्यात्। उ॒न्नीय॒ प्राङु॒दाद्र॑वेत्। स उ॑प॒साद्यातमि॑तोरासीत। स य॒दा ताम्येत्। अथ॒ भूः स्वाहेति॑ जुहुयात्। प्र॒जाप॑ति॒र्वै भू॒तः। तमे॒वोपा॑सरत्। स ए॒वैन॒न्तत॒ उन्न॑यति। नार्ति॒मार्च्छ॑ति॒ यज॑मानः॥५६॥\anuvakamend[तू॒ष्णीञ्जा॑यते॒ यज॑मानः]

%2.1.10.1
यद॒ग्निमु॒द्धर॑ति। वस॑व॒स्तर्ह्य॒ग्निः। तस्मि॒न्॒ यस्य॒ तथा॑विधे॒ जुह्व॑ति। वसु॑ष्वे॒वास्याग्निहो॒त्र हु॒तं भ॑वति। निहि॑तो धूपा॒यञ्छे॑ते। रु॒द्रास्तर्ह्य॒ग्निः। तस्मि॒न्॒ यस्य॒ तथा॑विधे॒ जुह्व॑ति। रु॒द्रे॒ष्वे॒वास्याग्निहो॒त्र हु॒तं भ॑वति। प्र॒थ॒ममि॒ध्मम॒र्चिरा ल॑भते। आ॒दि॒त्यास्तर्ह्य॒ग्निः॥५७॥

%2.1.10.2
तस्मि॒न्॒ यस्य॒ तथा॑विधे॒ जुह्व॑ति। आ॒दि॒त्येष्वे॒वास्याग्निहो॒त्र हु॒तं भ॑वति। सर्व॑ ए॒व स॑र्व॒श इ॒ध्म आदीप्तो भवति। विश्वे॑ दे॒वास्तर्ह्य॒ग्निः। तस्मि॒न्॒ यस्य॒ तथा॑विधे॒ जुह्व॑ति। विश्वेष्वे॒वास्य॑ दे॒वेष्व॑ग्निहो॒त्र हु॒तं भ॑वति। नि॒त॒राम॒र्चिरु॒पावै॑ति लोहि॒नीके॑व भवति। इन्द्र॒स्तर्ह्य॒ग्निः। तस्मि॒न्॒ यस्य॒ तथा॑विधे॒ जुह्व॑ति। इन्द्र॑ ए॒वास्याग्निहो॒त्र हु॒तं भ॑वति॥५८॥

%2.1.10.3
अङ्गा॑रा भवन्ति। तेभ्योऽङ्गा॑रेभ्यो॒ऽर्चिरुदे॑ति। प्र॒जाप॑ति॒स्तर्ह्य॒ग्निः। तस्मि॒न्॒ यस्य॒ तथा॑विधे॒ जुह्व॑ति। प्र॒जाप॑तावे॒वास्याग्निहो॒त्र हु॒तं भ॑वति। शरोऽङ्गा॑रा॒ अध्यू॑हन्ते। ब्रह्म॒ तर्ह्य॒ग्निः। तस्मि॒न्॒ यस्य॒ तथा॑विधे॒ जुह्व॑ति। ब्रह्म॑न्ने॒वास्याग्निहो॒त्र हु॒तं भ॑वति। वसु॑षु रु॒द्रेष्वा॑दि॒त्येषु॒ विश्वे॑षु दे॒वेषु॑। इन्द्रे प्र॒जाप॑तौ॒ ब्रह्म\sn{}। अप॑रिवर्गमे॒वास्यै॒तासु॑ दे॒वता॑सु हु॒तं भ॑वति। यस्यै॒वं वि॒दुषोऽग्निहो॒त्रं जुह्व॑ति। य उ॑ चैनदे॒वं वेद॑॥५९॥\anuvakamend[आ॒दि॒त्यास्तर्ह्य॒ग्निरिन्द्र॑ ए॒वास्याग्निहो॒त्र हु॒तं भ॑वति दे॒वेषु॑ च॒त्वारि॑ च (यद॒ग्निन्निहि॑तः प्रथ॒म सर्व॑ ए॒व नि॑त॒रामङ्गा॑रा॒ शरोऽङ्गा॑रा॒ ब्रह्म॒ वसु॑ष्व॒ष्टौ ॥ )]

%2.1.11.1
ऋ॒तन्त्वा॑ स॒त्येन॒ परि॑षिञ्चा॒मीति॑ सा॒यं परि॑षिञ्चति। स॒त्यन्त्व॒र्तेन॒ परि॑षिञ्चा॒मीति॑ प्रा॒तः। अ॒ग्निर्वा ऋ॒तम्। अ॒सावा॑दि॒त्यः स॒त्यम्। अ॒ग्निमे॒व तदा॑दि॒त्येन॑ सा॒यं परि॑षिञ्चति। अ॒ग्निना॑ऽऽदि॒त्यं प्रा॒तः सः। याव॑दहोरा॒त्रे भव॑तः। ताव॑दस्य लो॒कस्य॑। नार्ति॒र्न रिष्टि॑। नान्तो॒ न प॑र्य॒न्तोऽस्ति। यस्यै॒वं वि॒दुषोऽग्निहो॒त्रं जुह्व॑ति। य उ॑चैनदे॒वं वेद॑॥६०॥\anuvakamend[अ॒स्ति॒ द्वे च॑]




\prashnaend{अङ्गि॑रसः प्र॒जाप॑तिर॒ग्नि रु॒द्र उ॑त्त॒राव॑तीं ब्रह्मवा॒दिनोऽग्निहो॒त्रप्रा॑यणा य॒ज्ञाः प्र॒जाप॑तिरकामयतात्म॒न्वद्रौ॒द्रङ्गवि॑ दक्षिण॒तस्त्रयो॒ वै यद॒ग्निमृ॒तन्त्वा॑ स॒त्येनैका॑दश॥११॥}{अङ्गि॑रस॒ प्रैव तेन॑ प॒शूने॒व यन्नि॒मार्ष्टि॒ यो वा अ॑ग्निहो॒त्रस्यो॑प॒सदो॑ दक्षिण॒तष्ष॒ष्टिः॥६०॥}{अङ्गि॑रसो॒ य उ॑चैनदे॒वं वेद॑॥}{हरि॑ ओम्॥}{इति श्रीकृष्णयजुर्वेदीयतैत्तिरीयब्राह्मणे द्वितीयाष्टके प्रथमः प्रपाठकः समाप्तः॥}
\clearpage
