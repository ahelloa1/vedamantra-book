\chapt{अष्टकम् १}
\sect{प्रथमः प्रश्नः}
\setcounter{anuvakam}{0}
\dnsub{तैत्तिरीयब्राह्मणे प्रथमाष्टके प्रथमः प्रपाठकः}

%1.1.1.1
ब्रह्म॒ सन्ध॑त्तं॒ तन्मे॑ जिन्वतम्। क्ष॒त्र सन्ध॑त्तं॒ तन्मे॑ जिन्वतम्। इष॒ सन्ध॑त्तं॒ तां मे॑ जिन्वतम्। ऊर्ज॒ सन्ध॑त्तं॒ तां मे॑ जिन्वतम्। र॒यि सन्ध॑त्तं॒ तां मे॑ जिन्वतम्। पुष्टि॒ सन्ध॑त्तं॒ तां मे॑ जिन्वतम्। प्र॒जा सन्ध॑त्तं॒ तां मे॑ जिन्वतम्। प॒शून्त्सन्ध॑त्तं॒ तान्मे॑ जिन्वतम्। स्तु॒तो॑ऽसि॒ जन॑धाः। दे॒वास्त्वा॑ शुक्र॒पाः प्रण॑यन्तु॥१॥

%1.1.1.2
सु॒वीरा प्र॒जाः प्र॑ज॒नय॒न्परी॑हि। शु॒क्रः शु॒क्रशो॑चिषा। स्तु॒तो॑ऽसि॒ जन॑धाः। दे॒वास्त्वा॑ मन्थि॒पाः प्रण॑यन्तु। सु॒प्र॒जाः प्र॒जाः प्र॑ज॒नय॒न्परी॑हि। म॒न्थी म॒न्थिशो॑चिषा। स॒ञ्ज॒ग्मा॒नौ दि॒व आपृ॑थि॒व्यायु॑। सन्ध॑त्तं॒ तन्मे॑ जिन्वतम्। प्रा॒ण सन्ध॑त्तं॒ तं मे॑ जिन्वतम्। अ॒पा॒न सन्ध॑त्तं॒ तं मे॑ जिन्वतम्॥२॥

%1.1.1.3
व्या॒न सन्ध॑त्तं॒ तं मे॑ जिन्वतम्। चक्षु॒ सन्ध॑त्तं॒ तन्मे॑ जिन्वतम्। श्रोत्र॒ सन्ध॑त्तं॒ तन्मे॑ जिन्वतम्। मन॒ सन्ध॑त्तं॒ तन्मे॑ जिन्वतम्। वाच॒ सन्ध॑त्तं॒ तां मे॑ जिन्वतम्। आयु॑ स्थ॒ आयु॑र्मे धत्तम्। आयु॑र्य॒ज्ञाय॑ धत्तम्। आयु॑र्य॒ज्ञप॑तये धत्तम्। प्रा॒णः स्थ॑ प्रा॒णं मे॑ धत्तम्। प्रा॒णं य॒ज्ञाय॑ धत्तम्॥३॥

%1.1.1.4
प्रा॒णं य॒ज्ञप॑तये धत्तम्। चक्षु॑ स्थ॒श्चक्षु॑र्मे धत्तम्। चक्षु॑र्य॒ज्ञाय॑ धत्तम्। चक्षु॑र्य॒ज्ञप॑तये धत्तम्। श्रोत्र स्थ॒ श्रोत्रं॑ मे धत्तम्। श्रोत्रं॑ य॒ज्ञाय॑ धत्तम्। श्रोत्रं॑ य॒ज्ञप॑तये धत्तम्। तौ दे॑वौ शुक्रामन्थिनौ। क॒ल्पय॑तं॒ दैवी॒र्विश॑। क॒ल्पय॑तं॒ मानु॑षीः॥४॥

%1.1.1.5
इष॒मूर्ज॑म॒स्मासु॑ धत्तम्। प्रा॒णान्प॒शुषु॑। प्र॒जां मयि॑ च॒ यज॑माने च। निर॑स्त॒ शण्ड॑। निर॑स्तो॒ मर्क॑। अप॑नुत्तौ॒ शण्डा॒मर्कौ॑ स॒हामुना। शु॒क्रस्य॑ स॒मिद॑सि। म॒न्थिन॑ स॒मिद॑सि। स प्र॑थ॒मः संकृ॑तिर्वि॒श्वक॑र्मा। स प्र॑थ॒मो मि॒त्रो वरु॑णो अ॒ग्निः। स प्र॑थ॒मो बृह॒स्पति॑श्चिकि॒त्वान्। तस्मा॒ इन्द्रा॑य सु॒तमा जु॑होमि॥५॥\anuvakamend[न॒य॒न्त्व॒पा॒न सन्ध॑त्तं॒ तं मे॑ जिन्वतं प्रा॒णं य॒ज्ञाय॑ धत्तं॒ मानु॑षीर॒ग्निर्द्वे च॑॥ (ब्रह्म॑ क्ष॒त्रं तदिष॒मूर्ज र॒यिं पुष्टिं॑ प्र॒जां तां प॒शून्तान्त्सन्ध॑त्तं॒ तत्प्रा॒णम॑पा॒नं व्या॒नं तं चक्षु॒ श्रोत्रं॒ मन॒स्तद्वाचं॒ ताम्। इ॒षादि॒पञ्च॑के॒ वाचं॒ तां मे॑ प॒शून्त्सन्ध॑त्तं॒ तान्मे प्रा॒णादि॒त्रित॑ये॒ तं मे॒ऽन्यत्र॒ तन्मे)]

%1.1.2.1
कृत्ति॑कास्व॒ग्निमाद॑धीत। ए॒तद्वा अ॒ग्नेर्नक्ष॑त्रम्। यत्कृत्ति॑काः। स्वाया॑मै॒वैनं॑ दे॒वता॑यामा॒धाय॑। ब्र॒ह्म॒व॒र्च॒सी भ॑वति। मुखं॒ वा ए॒तन्नक्ष॑त्राणाम्। यत्कृत्ति॑काः। यः कृत्ति॑कास्व॒ग्निमा॑ध॒त्ते। मुख्य॑ ए॒व भ॑वति। अथो॒ खलु॑॥६॥

%1.1.2.2
अ॒ग्नि॒न॒क्ष॒त्रमित्यप॑चायन्ति। गृ॒हान् ह॒ दाहु॑को भवति। प्र॒जाप॑ती रोहि॒ण्याम॒ग्निम॑सृजत। तं दे॒वा रो॑हि॒ण्यामाद॑धत। ततो॒ वै ते सर्वा॒न्रोहा॑नरोहन्। तद्रो॑हि॒ण्यै रो॑हिणि॒त्वम्। यो रो॑हि॒ण्याम॒ग्निमा॑ध॒त्ते। ऋ॒ध्नोत्ये॒व। सर्वा॒न्रोहान्रोहति। दे॒वा वै भ॒द्राः सन्तो॒ऽग्निमाधि॑त्सन्त॥७॥

%1.1.2.3
तेषा॒मना॑हितो॒ऽग्निरासीत्। अथैभ्यो वा॒मं वस्वपाक्रामत्। ते पुन॑र्वस्वो॒राद॑धत। ततो॒ वै तान् वा॒मं वसू॒पाव॑र्तत। यः पु॒राऽभ॒द्रः सन्पापी॑या॒न्त्स्यात्। स पुन॑र्वस्वोर॒ग्निमाद॑धीत। पुन॑रे॒वैनं॑ वा॒मं वसू॒पाव॑र्तते। भ॒द्रो भ॑वति। यः का॒मये॑त॒ दानका॑मा मे प्र॒जाः स्यु॒रिति॑। स पूर्व॑यो॒ फल्गु॑न्योर॒ग्निमाद॑धीत॥८॥

%1.1.2.4
अ॒र्य॒म्णो वा ए॒तन्नक्ष॑त्रम्। यत्पूर्वे॒ फल्गु॑नी। अ॒र्य॒मेति॒ तमा॑हु॒र्यो ददा॑ति। दान॑कामा अस्मै प्र॒जा भ॑वन्ति। यः का॒मये॑त भ॒गी स्या॒मिति॑। स उत्त॑रयो॒ फल्गु॑न्योर॒ग्निमाद॑धीत। भग॑स्य॒ वा ए॒तन्नक्ष॑त्रम्। यदुत्त॑रे॒ फल्गु॑नी। भ॒ग्ये॑व भ॑वति। का॒ल॒क॒ञ्जा वै नामासु॑रा आसन्॥९॥

%1.1.2.5
ते सु॑व॒र्गाय॑ लो॒काया॒ग्निम॑चिन्वत। पुरु॑ष॒ इष्ट॑का॒मुपा॑दधा॒त्पुरु॑ष॒ इष्ट॑काम्। स इन्द्रो ब्राह्म॒णो ब्रुवा॑ण॒ इष्ट॑का॒मुपा॑धत्त। ए॒षा मे॑ चि॒त्रा नामेति॑। ते सु॑व॒र्गं लो॒कमा प्रारो॑हन्। स इन्द्र॒ इष्ट॑का॒मावृ॑हत्। तेऽवा॑कीर्यन्त। ये॑ऽवाकीर्यन्त। त ऊर्णा॒वभ॑योऽभवन्। द्वावुद॑पतताम्॥१०॥

%1.1.2.6
तौ दि॒व्यौ श्वाना॑वभवताम्। यो भ्रातृ॑व्यवा॒न्त्स्यात्। स चि॒त्राया॑म॒ग्निमाद॑धीत। अ॒व॒कीर्यै॒व भ्रातृ॑व्यान्। ओजो॒ बल॑मिन्द्रि॒यं वी॒र्य॑मा॒त्मन्ध॑त्ते। व॒सन्ता ब्राह्म॒णोऽग्निमाद॑धीत। व॒स॒न्तो वै ब्राह्म॒णस्य॒र्तुः। स्व ए॒वैन॑मृ॒तावा॒धाय॑। ब्र॒ह्म॒व॒र्च॒सी भ॑वति। मुखं॒ वा ए॒तदृ॑तू॒नाम्॥११॥

%1.1.2.7
यद्व॑स॒न्तः। यो व॒सन्ता॒ऽग्निमा॑ध॒त्ते। मुख्य॑ ए॒व भ॑वति। अथो॒ योनि॑मन्तमे॒वैनं॒ प्रजा॑त॒माध॑त्ते। ग्री॒ष्मे रा॑ज॒न्य॑ आद॑धीत। ग्री॒ष्मो वै रा॑ज॒न्य॑स्य॒र्तुः। स्व ए॒वैन॑मृ॒तावा॒धाय॑। इ॒न्द्रि॒या॒वी भ॑वति। श॒रदि॒ वैश्य॒ आद॑धीत। श॒रद्वै वैश्य॑स्य॒र्तुः॥१२॥

%1.1.2.8
स्व ए॒वैन॑मृ॒तावा॒धाय॑। प॒शु॒मान्भ॑वति। न पूर्व॑यो॒ फल्गु॑न्योर॒ग्निमाद॑धीत। ए॒षा वै ज॑घ॒न्या॑ रात्रि॑ संवत्स॒रस्य॑। यत्पूर्वे॒ फल्गु॑नी। पृ॒ष्टि॒त ए॒व सं॑वत्स॒रस्या॒ग्निमा॒धाय॑। पापी॑यान्भवति। उत्त॑रयो॒रा द॑धीत। ए॒षा वै प्र॑थ॒मा रात्रि॑ संवत्स॒रस्य॑। यदुत्त॑रे॒ फल्गु॑नी। मु॒ख॒त ए॒व सं॑वत्स॒रस्या॒ग्निमा॒धाय॑। वसी॑यान्भवति। अथो॒ खलु॑। य॒दैवैनं॑ य॒ज्ञ उ॑प॒नमेत्। अथाद॑धीत। सैवास्यर्द्धि॑॥१३॥\anuvakamend[खल्वा॑धित्सन्त॒ फल्गु॑न्योर॒ग्निमाद॑धीतासन्नपततामृतू॒नां वैश्य॑स्य॒र्तुरुत्त॑रे॒ फल्गु॑नी॒ षट्च॑]

%1.1.3.1
उद्ध॑न्ति। यदे॒वास्या॑ अमे॒ध्यम्। तदप॑हन्ति। अ॒पोऽवोक्षति॒ शान्त्यै। सिक॑ता॒ निव॑पति। ए॒तद्वा अ॒ग्नेर्वैश्वान॒रस्य॑ रू॒पम्। रू॒पेणै॒व वैश्वान॒रमव॑रुन्धे। ऊषां॒ निव॑पति। पुष्टि॒र्वा ए॒षा प्र॒जन॑नम्। यदूषा ॥१४॥

%1.1.3.2
पुष्ट्या॑मे॒व प्र॒जन॑ने॒ऽग्निमाध॑त्ते। अथो॑ सं॒ज्ञान॑ ए॒व। सं॒ज्ञान॒ ह्ये॑तत्प॑शू॒नाम्। यदूषा। द्यावा॑पृथि॒वी स॒हास्ताम्। ते वि॑य॒ती अ॑ब्रूताम्। अस्त्वे॒व नौ॑ स॒ह य॒ज्ञिय॒मिति॑। यद॒मुष्या॑ य॒ज्ञिय॒मासीत्। तद॒स्याम॑दधात्। त ऊषा॑ अभवन्॥१५॥

%1.1.3.3
यद॒स्या य॒ज्ञिय॒मासीत्। तद॒मुष्या॑मदधात्। तद॒दश्च॒न्द्रम॑सि कृ॒ष्णम्। ऊषान्नि॒वप॑न्न॒दो ध्या॑येत्। द्यावा॑पृथि॒व्योरे॒व य॒ज्ञिये॒ऽग्निमाध॑त्ते। अ॒ग्निर्दे॒वेभ्यो॒ निला॑यत। आ॒खू रू॒पं कृ॒त्वा। स पृ॑थि॒वीं प्रावि॑शत्। स ऊ॒तीः कु॑र्वा॒णः पृ॑थि॒वीमनु॒ सम॑चरत्। तदा॑खुकरी॒षम॑भवत्॥१६॥

%1.1.3.4
यदा॑खुकरी॒ष सं॑भा॒रो भव॑ति। यदे॒वास्य॒ तत्र॒ न्य॑क्तम्। तदे॒वाव॑रुन्धे। ऊर्जं॒ वा ए॒त रसं॑ पृथि॒व्या उ॑प॒दीका॒ उद्दि॑हन्ति। यद्व॒ल्मीकम्। यद्व॑ल्मीकव॒पा सं॑भा॒रो भव॑ति। ऊर्ज॑मे॒व रसं॑ पृथि॒व्या अव॑रुन्धे। अथो॒ श्रोत्र॑मे॒व। श्रोत्र॒ ह्ये॑तत्पृ॑थि॒व्याः। यद्व॒ल्मीक॑॥१७॥

%1.1.3.5
अब॑धिरो भवति। य ए॒वं वेद॑। प्र॒जाप॑तिः प्र॒जा अ॑सृजत। तासा॒मन्न॒मुपाक्षीयत। ताभ्य॒ सूद॒मुप॒प्राभि॑नत्। ततो॒ वै तासा॒मन्नं॒ नाक्षी॑यत। यस्य॒ सूद॑ सम्भा॒रो भव॑ति। नास्य॑ गृ॒हेऽन्नं॑ क्षीयते। आपो॒ वा इ॒दमग्रे॑ सलि॒लमा॑सीत्। तेन॑ प्र॒जाप॑तिरश्राम्यत्॥१८॥

%1.1.3.6
क॒थमि॒द स्या॒दिति॑। सो॑ऽपश्यत्पुष्करप॒र्णं तिष्ठ॑त्। सो॑ऽमन्यत। अस्ति॒ वै तत्। यस्मि॑न्नि॒दमधि॒ तिष्ठ॒तीति॑। स व॑रा॒हो रू॒पं कृ॒त्वोप॒ न्य॑मज्जत्। स पृ॑थि॒वीम॒ध आर्च्छत्। तस्या॑ उप॒हत्योद॑मज्जत्। तत्पु॑ष्करप॒र्णेऽप्रथयत्। यदप्र॑थयत्॥१९॥

%1.1.3.7
तत्पृ॑थि॒व्यै पृ॑थिवि॒त्वम्। अभू॒द्वा इ॒दमिति॑। तद्भूम्यै॑ भूमि॒त्वम्। तां दिशोऽनु॒ वात॒ सम॑वहत्। ता शर्क॑राभिरदृहत्। शं वै नो॑ऽभू॒दिति॑। तच्छर्क॑राणा शर्कर॒त्वम्। यद्व॑रा॒हवि॑हत सम्भा॒रो भव॑ति। अ॒स्यामे॒वाछ॑म्बट्कारम॒ग्निमाध॑त्ते। शर्क॑रा भवन्ति॒ धृत्यै॥२०॥

%1.1.3.8
अथो॑ शं॒त्वाय॑। सरे॑ता अ॒ग्निरा॒धेय॒ इत्या॑हुः। आपो॒ वरु॑णस्य॒ पत्न॑य आसन्। ता अ॒ग्निर॒भ्य॑ध्यायत्। ताः सम॑भवत्। तस्य॒ रेत॒ परा॑ऽपतत्। तद्धिर॑ण्यमभवत्। यद्धिर॑ण्यमु॒पास्य॑ति। सरे॑तसमे॒वाग्निमाध॑त्ते। पुरु॑ष॒ इन्न्वै स्वाद्रेत॑सो बीभत्सत॒ इत्या॑हुः॥२१॥

%1.1.3.9
उ॒त्त॒र॒त उपास्य॒त्यबी॑भत्सायै। अति॒ प्रय॑च्छति। आर्ति॑मे॒वाति॒ प्रय॑च्छति। अ॒ग्निर्दे॒वेभ्यो॒ निला॑यत। अश्वो॑ रू॒पं कृ॒त्वा। सोऽश्व॒त्थे सं॑वत्स॒रम॑तिष्ठत्। तद॑श्व॒त्थस्याश्वत्थ॒त्वम्। यदाश्व॑त्थः सम्भा॒रो भव॑ति। यदे॒वास्य॒ तत्र॒ न्य॑क्तम्। तदे॒वाव॑रुन्धे॥२२॥

%1.1.3.10
दे॒वा वा ऊर्जं॒ व्य॑भजन्त। तत॑ उदु॒म्बर॒ उद॑तिष्ठत्। ऊर्ग्वा उ॑दु॒म्बर॑। यदौदु॑म्बरः सम्भा॒रो भव॑ति। ऊर्ज॑मे॒वाव॑रुन्धे। तृ॒तीय॑स्यामि॒तो दि॒वि सोम॑ आसीत्। तं गा॑य॒त्र्याऽह॑रत्। तस्य॑ प॒र्णम॑च्छिद्यत। तत्प॒र्णो॑ऽभवत्। तत्प॒र्णस्य॑ पर्ण॒त्वम्॥२३॥

%1.1.3.11
यस्य॑ पर्ण॒मय॑ सम्भा॒रो भव॑ति। सो॒म॒पी॒थमे॒वाव॑रुन्धे। दे॒वा वै ब्रह्म॑न्नवदन्त। तत्प॒र्ण उपा॑शृणोत्। सु॒श्रवा॒ वै नाम॑। यत्प॑र्ण॒मय॑ सम्भा॒रो भव॑ति। ब्र॒ह्म॒व॒र्च॒समे॒वाव॑ रुन्धे। प्र॒जाप॑तिर॒ग्निम॑सृजत। सो॑ऽबिभे॒त्प्र मा॑ धक्ष्य॒तीति॑। त श॒म्या॑ऽशमयत्॥२४॥

%1.1.3.12
तच्छ॒म्यै॑ शमि॒त्वम्। यच्छ॑मी॒मय॑ सम्भा॒रो भव॑ति। शान्त्या॒ अप्र॑दाहाय। अ॒ग्नेः सृ॒ष्टस्य॑ य॒तः। विक॑ङ्कतं॒ भा आर्च्छत्। यद्वैक॑ङ्कतः सम्भा॒रो भव॑ति। भा ए॒वाव॑ रुन्धे। सहृ॑दयो॒ऽग्निरा॒धेय॒ इत्या॑हुः। म॒रुतो॒ऽद्भिर॒ग्निम॑तमयन्। तस्य॑ ता॒न्तस्य॒ हृद॑य॒माच्छि॑न्दन्। साऽशनि॑रभवत्। यद॒शनि॑हतस्य वृ॒क्षस्य॑ सम्भा॒रो भव॑ति। सहृ॑दयमे॒वाग्निमा ध॑त्ते॥२५॥\anuvakamend[ऊषा॑ अभवन्नभवद्व॒ल्मीकोऽश्राम्य॒दप्र॑थय॒द्धृत्यै॑ बीभत्सत॒ इत्या॑हू रुन्धे पर्ण॒त्वम॑शमयदच्छिन्द॒ स्त्रीणि॑ च]

%1.1.4.1
द्वा॒द॒शसु॑ विक्रा॒मेष्व॒ग्निमा द॑धीत। द्वाद॑श॒ मासा संवत्स॒रः। सं॒व॒त्स॒रादे॒वैन॑मव॒रुद्ध्या ध॑त्ते। यद्द्वा॑द॒शसु॑ विक्रा॒मेष्वा॒ दधी॑त। परि॑मित॒मव॑ रुन्धीत। चक्षु॑र्निमित॒ आद॑धीत। इय॒द्द्वाद॑श विक्रा॒मा (३) इति॑। परि॑मितं चै॒वाप॑रिमितं॒ चाव॑ रुन्धे। अनृ॑तं॒ वै वा॒चा व॑दति। अनृ॑तं॒ मन॑सा ध्यायति॥२६॥

%1.1.4.2
चक्षु॒र्वै स॒त्यम्। अद्रा(३)गित्या॑ह। अद॑र्\mbox{}श॒मिति॑। तत्स॒त्यम्। यश्चक्षु॑र्निमिते॒ऽग्निमा॑ध॒त्ते। स॒त्य ए॒वैन॒मा ध॑त्ते। तस्मा॒दाहि॑ताग्नि॒र्नानृ॑तं वदेत्। नास्य॑ ब्राह्म॒णोऽनाश्वान्गृ॒हे व॑सेत्। स॒त्ये ह्य॑स्या॒ग्निराहि॑तः। आ॒ग्ने॒यी वै रात्रि॑॥२७॥

%1.1.4.3
आ॒ग्ने॒याः प॒शव॑। ऐ॒न्द्रमह॑। नक्तं॒ गार्\mbox{}ह॑पत्य॒मा द॑धाति। प॒शूने॒वाव॑ रुन्धे। दिवा॑ऽऽहव॒नीयम्। इ॒न्द्रि॒यमे॒वाव॑ रुन्धे। अ॒र्धोदि॑ते॒ सूर्य॑ आहव॒नीय॒मा द॑धाति। ए॒तस्मि॒न्वै लो॒के प्र॒जाप॑तिः प्र॒जा अ॑सृजत। प्र॒जा ए॒व तद्यज॑मानः सृजते। अथो॑ भू॒तं चै॒व भ॑वि॒ष्यच्चाव॑ रुन्धे॥२८॥

%1.1.4.4
इडा॒ वै मा॑न॒वी य॑ज्ञानूका॒शिन्या॑सीत्। साऽशृ॑णोत्। असु॑रा अ॒ग्निमाद॑धत॒ इति॑। तद॑गच्छत्। त आ॑हव॒नीय॒मग्र॒ आद॑धत। अथ॒ गार्\mbox{}ह॑पत्यम्। अथान्वाहार्य॒पच॑नम्। साऽब्र॑वीत्। प्र॒तीच्ये॑षा॒ श्रीर॑गात्। भ॒द्रा भू॒त्वा परा॑ भविष्य॒न्तीति॑॥२९॥

%1.1.4.5
यस्यै॒वम॒ग्निरा॑धी॒यते। प्र॒तीच्य॑स्य॒ श्रीरे॑ति। भ॒द्रो भू॒त्वा परा॑भवति। साऽशृ॑णोत्। दे॒वा अ॒ग्निमाद॑धत॒ इति॑। तद॑गच्छत्। तेऽन्वाहार्य॒पच॑न॒मग्र॒ आद॑धत। अथ॒ गार्\mbox{}ह॑पत्यम्। अथा॑हव॒नीयम्। साऽब्र॑वीत्॥३०॥

%1.1.4.6
प्राच्ये॑षा॒ श्रीर॑गात्। भ॒द्रा भू॒त्वा सु॑व॒र्गंल्लो॒कमेष्यन्ति। प्र॒जां तु न वेत्स्यन्त॒ इति॑। यस्यै॒वम॒ग्निरा॑धी॒यते। प्राच्य॑स्य॒ श्रीरे॑ति। भ॒द्रो भू॒त्वा सु॑व॒र्गं लो॒कमे॑ति। प्र॒जां तु न वि॑न्दते। साऽब्र॑वी॒दिडा॒ मनुम्। तथा॒ वा अ॒हं तवा॒ग्निमाधास्यामि। यथा॒ प्र प्र॒जया॑ प॒शुभि॑र्मिथु॒नैर्ज॑नि॒ष्यसे॥३१॥

%1.1.4.7
प्रत्य॒स्मिल्लोँ॒के स्था॒स्यसि॑। अ॒भि सु॑व॒र्गं लो॒कं जे॒ष्यसीति॑। गार्\mbox{}ह॑पत्य॒मग्र॒ आद॑धात्। गार्\mbox{}ह॑पत्यं॒ वा अनु॑ प्र॒जाः प॒शव॒ प्रजा॑यन्ते। गार्\mbox{}ह॑पत्येनै॒वास्मै प्र॒जां प॒शून्प्राज॑नयत्। अथान्वाहार्य॒पच॑नम्। ति॒र्यङ्ङि॑व॒ वा अ॒यं लो॒कः। अ॒स्मिन्नै॒व तेन॑ लो॒के प्रत्य॑तिष्ठत्। अथा॑हव॒नीयम्। तेनै॒व सु॑व॒र्गं लो॒कम॒भ्य॑जयत्॥३२॥

%1.1.4.8
यस्यै॒वम॒ग्निरा॑धी॒यते। प्र प्र॒जया॑ प॒शुभि॑र्मिथु॒नैर्जा॑यते। प्रत्य॒स्मिल्लोँ॒के ति॑ष्ठति। अ॒भि सु॑व॒र्गं लो॒कं ज॑यति। यस्य॒ वा अय॑थादेवतम॒ग्निरा॑धी॒यते। आ दे॒वताभ्यो वृश्च्यते। पापी॑यान्भवति। यस्य॑ यथादेव॒तम्। न दे॒वताभ्य॒ आवृ॑श्च्यते। वसी॑यान्भवति॥३३॥ भृगू॑णां॒ त्वाऽङ्गि॑रसां व्रतपते व्र॒तेनाद॑धा॒मीति॑ भृग्वङ्गि॒रसा॒माद॑ध्यात्। आ॒दि॒त्यानां त्वा दे॒वानां व्रतपते व्र॒तेनाद॑धा॒मीत्य॒न्यासां॒ ब्राह्म॑णीनां प्र॒जानाम्। वरु॑णस्य त्वा॒ राज्ञो व्रतपते व्र॒तेनाद॑धा॒मीति॒ राज्ञ॑। इन्द्र॑स्य त्वेन्द्रि॒येण॑ व्रतपते व्र॒तेनाद॑धा॒मीति॑ राज॒न्य॑स्य। मनोस्त्वा ग्राम॒ण्यो व्रतपते व्र॒तेनाद॑धा॒मिति॒ वैश्य॑स्य। ऋ॒भू॒णां त्वा॑ दे॒वानां व्रतपते व्र॒तेनाद॑धा॒मीति॑ रथका॒रस्य॑। य॒था॒दे॒व॒तम॒ग्निराधी॑यते। न दे॒वताभ्य॒ आवृ॑श्च्यते। वसी॑यान्भवति॥३४॥\anuvakamend[ध्या॒य॒ति॒ वै रात्रि॒श्चाव॑रुन्धे भविष्य॒न्तीत्य॑ब्रवीज्जनि॒ष्यसे॑ऽजय॒द्वसी॑यान्भवति॒ नव॑ च]

%1.1.5.1
प्र॒जाप॑तिर्वा॒चः स॒त्यम॑पश्यत्। तेना॒ग्निमाध॑त्त। तेन॒ वै स आर्ध्नोत्। भूर्भुव॒ सुव॒रित्या॑ह। ए॒तद्वै वा॒चः स॒त्यम्। य ए॒तेना॒ग्निमा॑ध॒त्ते। ऋ॒ध्नोत्ये॒व। अथो॑ स॒त्यप्रा॑शूरे॒व भ॑वति। अथो॒ य ए॒वं वि॒द्वान॑भि॒चर॑ति। स्तृ॒णु॒त ए॒वैनम्॥३५॥

%1.1.5.2
भूरित्या॑ह। प्र॒जा ए॒व तद्यज॑मानः सृजते। भुव॒ इत्या॑ह। अ॒स्मिन्ने॒व लो॒के प्रति॑तिष्ठति। सुव॒रित्या॑ह। सु॒व॒र्ग ए॒व लो॒के प्रति॑तिष्ठति। त्रि॒भिर॒क्षरै॒र्गार्\mbox{}ह॑पत्य॒मा द॑धाति। त्रय॑ इ॒मे लो॒काः। ए॒ष्वे॑वैनं॑ लो॒केषु॒ प्रति॑ष्ठित॒माध॑त्ते। सर्वै प॒ञ्चभि॑राहव॒नीयम्॥३६॥

%1.1.5.3
सु॒व॒र्गाय॒ वा ए॒ष लो॒कायाधी॑यते। यदा॑हव॒नीय॑। सु॒व॒र्ग ए॒वास्मै॑ लो॒के वा॒चः स॒त्य सर्व॑माप्नोति। त्रि॒भिर्गार्\mbox{}ह॑पत्य॒मा द॑धाति। प॒ञ्चभि॑राहव॒नीयम्। अ॒ष्टौ संप॑द्यन्ते। अ॒ष्टाक्ष॑रा गाय॒त्री। गा॒य॒त्रोऽग्निः। यावा॑ने॒वाग्निः। तमाध॑त्ते॥३७॥

%1.1.5.4
प्र॒जाप॑तिः प्र॒जा अ॑सृजत। ता अ॑स्मात्सृ॒ष्टाः परा॑चीरायन्। ताभ्यो॒ ज्योति॒रुद॑गृह्णात्। तं ज्योति॒ पश्य॑न्तीः प्र॒जा अ॒भि स॒माव॑र्तन्त। उ॒परी॑वा॒ग्निमुद्गृ॑ह्णीयादु॒द्धर\sn{}। ज्योति॑रे॒व पश्य॑न्तीः प्र॒जा यज॑मानम॒भि स॒माव॑र्तन्ते। प्र॒जाप॑ते॒रक्ष्य॑श्वयत्। तत्परा॑ऽपतत्। तदश्वो॑ऽभवत्। तदश्व॑स्याश्व॒त्वम्॥३८॥

%1.1.5.5
ए॒ष वै प्र॒जाप॑तिः। यद॒ग्निः। प्रा॒जा॒प॒त्योऽश्व॑। यदश्वं॑ पु॒रस्ता॒न्नय॑ति। स्वमे॒व चक्षु॒ पश्य॑न्प्र॒जाप॑ति॒रनूदे॑ति। व॒ज्री वा ए॒षः। यदश्व॑। यदश्वं॑ पु॒रस्ता॒न्नय॑ति। जा॒ताने॒व भ्रातृ॑व्या॒न्प्रणु॑दते। पुन॒रा व॑र्तयति॥३९॥

%1.1.5.6
ज॒नि॒ष्यमा॑णाने॒व प्रति॑नुदते। न्या॑हव॒नीयो॒ गार्\mbox{}ह॑पत्य\-मकामयत। निगार्\mbox{}ह॑पत्य आहव॒नीयम्। तौ वि॒भाजं॒ नाश॑क्नोत्। सोऽश्व॑ पूर्व॒वाड्भू॒त्वा। प्राञ्चं॒ पूर्व॒मुद॑वहत्। तत्पूर्व॒वाह॑ पूर्ववा॒ट्त्वम्। यदश्वं॑ पु॒रस्ता॒न्नय॑ति। विभ॑क्तिरे॒वैन॑यो॒ सा। अथो॒ नाना॑वीर्यावे॒वैनौ॑ कुरुते॥४०॥

%1.1.5.7
यदु॒पर्यु॑परि॒ शिरो॒ हरेत्। प्रा॒णान्‌विच्छि॑न्द्यात्। अ॒धो॑ऽध॒ शिरो॑ हरति। प्रा॒णानां गोपी॒थाय॑। इय॒त्यग्रे॑ हरति। अथेय॒त्यथेय॑ति। त्रय॑ इ॒मे लो॒काः। ए॒ष्वे॑वैनं॑ लो॒केषु॒ प्रति॑ष्ठित॒माध॑त्ते। प्र॒जाप॑तिर॒ग्निम॑सृजत। सो॑ऽबिभे॒त्प्र मा॑ धक्ष्य॒तीति॑॥४१॥

%1.1.5.8
तस्य॑ त्रे॒धा म॑हि॒मानं॒ व्यौ॑हत्। शान्त्या॒ अप्र॑दाहाय। यत्रे॒धाऽग्निरा॑धी॒यते। म॒हि॒मान॑मे॒वास्य॒ तद्व्यू॑हति। शान्त्या॒ अप्र॑दाहाय। पुन॒रा व॑र्तयति। म॒हि॒मान॑मे॒वास्य॒ संद॑धाति। प॒शुर्वा ए॒षः। यदश्व॑। ए॒ष रु॒द्रः॥४२॥

%1.1.5.9
यद॒ग्निः। यदश्व॑स्य प॒देऽग्निमा॑द॒ध्यात्। रु॒द्राय॑ प॒शूनपि॑दध्यात्। अ॒प॒शुर्यज॑मानः स्यात्। यन्नाक्र॒मयेत्। अन॑वरुद्धा अस्य प॒शव॑ स्युः। पा॒र्श्व॒त आक्र॑मयेत्। यथाऽऽहि॑तस्या॒ग्नेरङ्गा॑रा अभ्यव॒वर्ते॑रन्। अव॑रुद्धा अस्य प॒शवो॒ भव॑न्ति। न रु॒द्रायापि॑दधाति॥४३॥

%1.1.5.10
त्रीणि॑ ह॒वीषि॒ निर्व॑पति। वि॒राज॑ ए॒व विक्रान्तं॒ यज॑मा॒नोऽनु॒ विक्र॑मते। अ॒ग्नये॒ पव॑मानाय। अ॒ग्नये॑ पाव॒काय॑। अ॒ग्नये॒ शुच॑ये। यद॒ग्नये॒ पव॑मानाय नि॒र्वप॑ति। पु॒नात्ये॒वैनम्। यद॒ग्नये॑ पाव॒काय॑। पू॒त ए॒वास्मि॑न्न॒न्नाद्यं॑ दधाति। यद॒ग्नये॒ शुच॑ये। ब्र॒ह्म॒व॒र्च॒समे॒वास्मि॑न्नु॒परि॑ष्टाद्दधाति॥४४॥\anuvakamend[ए॒न॒मा॒ह॒व॒नीयं॑ धत्तेऽश्व॒त्वं व॑र्तयति कुरुत॒ इति॑ रु॒द्रो द॑धाति॒ य॒दग्नये॒ शुच॑य॒ एकं॑ च]

%1.1.6.1
दे॒वा॒सु॒राः संय॑त्ता आसन्। ते दे॒वा वि॑ज॒यमु॑प॒यन्त॑। अ॒ग्नौ वा॒मं वसु॒ सं न्य॑दधत। इ॒दमु॑ नो भविष्यति। यदि॑ नो जे॒ष्यन्तीति॑। तद॒ग्निर्नोत्सह॑मशक्नोत्। तत् त्रे॒धा विन्य॑दधात्। प॒शुषु॒ तृती॑यम्। अ॒प्सु तृती॑यम्। आ॒दि॒त्ये तृती॑यम्॥४५॥

%1.1.6.2
तद्दे॒वा वि॒जित्य॑। पुन॒रवा॑रुरुत्सन्त। तेऽग्नये॒ पव॑मानाय पुरो॒डाश॑म॒ष्टाक॑पालं॒ निर॑वपन्। प॒शवो॒ वा अ॒ग्निः पव॑मानः। यदे॒व प॒शुष्वासीत्। तत्तेनावा॑रुन्धत। तेऽग्नये॑ पाव॒काय॑। आपो॒ वा अ॒ग्निः पा॑व॒कः। यदे॒वाप्स्वासीत्। तत्तेनावा॑रुन्धत॥४६॥

%1.1.6.3
तेऽग्नये॒ शुच॑ये। अ॒सौ वा आ॑दि॒त्योऽग्निः शुचि॑। यदे॒वादि॒त्य आसीत्। तत्तेनावा॑रुन्धत। ब्र॒ह्म॒वा॒दिनो॑ वदन्ति। त॒नुवो॒ वावैता अ॑ग्न्या॒धेय॑स्य। आ॒ग्ने॒यो वा अ॒ष्टाक॑पालोऽग्न्या॒धेय॒मिति॑। यत्तन्नि॒र्वपेत्। नैतानि॑। यथा॒ऽऽत्मा स्यात्॥४७॥

%1.1.6.4
नाङ्गा॑नि। ता॒दृगे॒व तत्। यदे॒तानि॑ नि॒र्वपेत्। न तम्। यथाऽङ्गा॑नि॒ स्युः। नात्मा। ता॒दृगे॒व तत्। उ॒भया॑नि स॒ह नि॒रुप्या॑णि। य॒ज्ञस्य॑ सात्म॒त्वाय॑। उ॒भयं॒ वा ए॒तस्येन्द्रि॒यं वी॒र्य॑माप्यते॥४८॥

%1.1.6.5
योऽग्निमा॑ध॒त्ते। ऐ॒न्द्रा॒ग्नमेका॑दशकपाल॒मनु॒ निर्व॑पेत्। आ॒दि॒त्यं च॒रुम्। इ॒न्द्रा॒ग्नी वै दे॒वाना॒मया॑तमायामानौ। ये ए॒व दे॒वते॒ अया॑तयाम्नी। ताभ्या॑मे॒वास्मा॑ इन्द्रि॒यं वी॒र्य॑मव॑ रुन्धे। आ॒दि॒त्यो भ॑वति। इ॒यं वा अदि॑तिः। अ॒स्यामे॒व प्रति॑तिष्ठति। धे॒न्वै वा ए॒तद्रेत॑॥४९॥

%1.1.6.6
यदाज्यम्। अ॒न॒डुह॑स्तण्डु॒लाः। मि॒थु॒नमे॒वाव॑रुन्धे। घृ॒ते भ॑वति। य॒ज्ञस्यालूक्षान्तत्वाय। च॒त्वार॑ आर्\mbox{}षे॒याः प्राश्ञ॑न्ति। दि॒शामे॒व ज्योति॑षि जुहोति। प॒शवो॒ वा ए॒तानि॑ ह॒वीषि॑। ए॒ष रु॒द्रः। यद॒ग्निः॥५०॥

%1.1.6.7
यत्स॒द्य ए॒तानि॑ ह॒वीषि॑ नि॒र्वपेत्। रु॒द्राय॑ प॒शूनपि॑ दध्यात्। अ॒प॒शुर्यज॑मानः स्यात्। यन्नानु॑नि॒र्वपेत्। अन॑वरुद्धा अस्य प॒शव॑ स्युः। द्वा॒द॒शसु॒ रात्री॒ष्वनु॒ निर्व॑पेत्। सं॒व॒त्स॒रप्र॑तिमा॒ वै द्वाद॑श॒ रात्र॑यः। सं॒व॒त्स॒रेणै॒वास्मै॑ रु॒द्र श॑मयि॒त्वा। प॒शूनव॑रुन्धे। यदेक॑मेकमे॒तानि॑ ह॒वीषि॑ नि॒र्वपेत्॥५१॥

%1.1.6.8
यथा॒ त्रीण्या॒वप॑नानि पू॒रयेत्। ता॒दृक्तत्। न प्र॒जन॑न॒\-मुच्छिषेत्। एकं॑ नि॒रुप्य॑। उत्त॑रे॒ सम॑स्येत्। तृ॒तीय॑मे॒वास्मै॑ लो॒कमुच्छिषति प्र॒जन॑नाय। तं प्र॒जया॑ प॒शुभि॒रनु॒ प्रजा॑यते। अथो॑ य॒ज्ञस्यै॒वैषाऽभिक्रान्तिः। र॒थ॒च॒क्रं प्रव॑र्तयति। म॒नु॒ष्य॒र॒थेनै॒व दे॑वर॒थं प्र॒त्यव॑रोहति॥५२॥

%1.1.6.9
ब्र॒ह्म॒वा॒दिनो॑ वदन्ति। हो॒त॒व्य॑मग्निहो॒त्राँ(३) न हो॑त॒व्या(३) मिति॑। यद्यजु॑षा जुहु॒यात्। अय॑थापूर्व॒माहु॑ती जुहुयात्। यन्न जु॑हु॒यात्। अ॒ग्निः परा॑ भवेत्। तू॒ष्णीमे॒व हो॑त॒व्यम्। य॒था॒पू॒र्वमाहु॑ती जु॒होति॑। नाग्निः परा॑भवति। अ॒ग्नीधे॑ ददाति॥५३॥

%1.1.6.10
अ॒ग्निमु॑खाने॒वर्तून्प्री॑णाति। उ॒प॒बर्\mbox{}ह॑णं ददाति। रू॒पाणा॒मव॑\-रुद्ध्यै। अश्वं॑ ब्र॒ह्मणे। इ॒न्द्रि॒यमे॒वाव॑रुन्धे। धे॒नु होत्रे। आ॒शिष॑ ए॒वाव॑रुन्धे। अ॒न॒ड्वाह॑मध्व॒र्यवे। वह्नि॒र्वा अ॑न॒ड्वान्। वह्नि॑रध्व॒र्युः॥५४॥

%1.1.6.11
वह्नि॑नै॒व वह्नि॑ य॒ज्ञस्याव॑रुन्धे। मि॒थु॒नौ गावौ॑ ददाति। मि॒थु॒नस्याव॑रुद्ध्यै। वासो॑ ददाति। स॒र्व॒दे॒व॒त्यं॑ वै वास॑। सर्वा॑ ए॒व दे॒वता प्रीणाति। आ द्वा॑द॒शभ्यो॑ ददाति। द्वाद॑श॒ मासा संवत्स॒रः। सं॒व॒त्स॒र ए॒व प्रति॑तिष्ठति। काम॑मू॒र्ध्वं देयम्। अप॑रिमित॒स्याव॑रुद्ध्यै॥५५॥\anuvakamend[आ॒दि॒त्ये तृती॑यम॒प्स्वासी॒त्तत्तेनावा॑रुन्धत॒ स्यादाप्यते॒ रेतो॒ऽग्निरेक॑मेकमे॒तानि॑ ह॒वीषि॑ नि॒र्वपेत्प्र॒त्यव॑रोहति ददात्यध्व॒र्युर्देय॒मेकं॑ च]

%1.1.7.1
घ॒र्मः शिर॒स्तद॒यम॒ग्निः। संप्रि॑यः प॒शुभि॑र्भुवत्। छ॒र्दिस्तो॒काय॒ तन॑याय यच्छ। वात॑ प्रा॒णस्तद॒यम॒ग्निः। संप्रि॑यः प॒शुभि॑र्भुवत्। स्व॒दि॒तं तो॒काय॒ तन॑याय पि॒तुं प॑च। प्राची॒मनु॑ प्र॒दिशं॒ प्रेहि॑ वि॒द्वान्। अ॒ग्नेर॑ग्ने पु॒रो अ॑ग्निर्भवे॒ह। विश्वा॒ आशा॒ दीद्या॑नो॒ विभा॑हि। ऊर्जं॑ नो धेहि द्वि॒पदे॒ चतु॑ष्पदे॥५६॥

%1.1.7.2
अ॒र्कश्चक्षु॒स्तद॒सौ सूर्य॒स्तद॒यम॒ग्निः। संप्रि॑यः प॒शुभि॑र्भुवत्। यत्ते॑ शुक्र शु॒क्रं वर्च॑ शु॒क्रा त॒नूः। शु॒क्रं ज्योति॒रज॑स्रम्। तेन॑ मे दीदिहि॒ तेन॒ त्वाऽऽद॑धे। अ॒ग्निनाऽग्ने॒ ब्रह्म॑णा। आ॒न॒शे व्या॑नशे॒ सर्व॒मायु॒र्व्या॑नशे। ये ते॑ अग्ने शि॒वे त॒नुवौ। वि॒राट्च॑ स्व॒राट्च॑। ते मावि॑शतां॒ ते मा॑ जिन्वताम्॥५७॥

%1.1.7.3
ये ते॑ अग्ने शि॒वे त॒नुवौ। सं॒राट्चा॑भि॒भूश्च॑। ते मावि॑शतां॒ ते मा॑ जिन्वताम्। ये ते॑ अग्ने शि॒वे त॒नुवौ। वि॒भूश्च॑ परि॒भूश्च॑। ते मा वि॑शतां॒ ते मा॑ जिन्वताम्। ये ते॑ अग्ने शि॒वे त॒नुवौ। प्र॒भ्वी च॒ प्रभू॑तिश्च। ते मा वि॑शतां॒ ते मा॑ जिन्वताम्। यास्ते॑ अग्ने शि॒वास्त॒नुव॑। ताभि॒स्त्वाऽऽद॑धे। यास्ते॑ अग्ने घो॒रास्त॒नुव॑। ताभि॑र॒मुं ग॑च्छ॥५८॥\anuvakamend[चतु॑ष्पदे जिन्वतां त॒नुव॒स्त्रीणि॑ च]

%1.1.8.1
इ॒मे वा ए॒ते लो॒का अ॒ग्नय॑। ते यदव्या॑वृत्ता आधी॒येर\sn{}। शो॒चये॑यु॒र्यज॑मानम्। घ॒र्मः शिर॒ इति॒ गार्\mbox{}ह॑पत्य॒मा द॑धाति। वात॑ प्रा॒ण इत्य॑न्वाहार्य॒पच॑नम्। अ॒र्कश्चक्षु॒रित्या॑हव॒नीयम्। तेनै॒वैना॒न्व्याव॑र्तयति। तथा॒ न शो॑चयन्ति॒ यज॑मानम्। र॒थ॒न्त॒रम॒भिगा॑यते॒ गार्\mbox{}ह॑पत्य आधी॒यमा॑ने। राथ॑न्तरो॒ वा अ॒यं लो॒कः॥५९॥

%1.1.8.2
अ॒स्मिन्ने॒वैनं॑ लो॒के प्रति॑ष्ठित॒मा ध॑त्ते। वा॒म॒दे॒व्यम॒भिगा॑यत उद्ध्रि॒यमा॑णे। अ॒न्तरि॑क्षं॒ वै वा॑मदे॒व्यम्। अ॒न्तरि॑क्ष ए॒वैनं॒ प्रति॑ष्ठित॒माध॑त्ते। अथो॒ शान्ति॒र्वै वा॑मदे॒व्यम्। शा॒न्तमे॒वैनं॑ पश॒व्य॑मुद्ध॑रते। बृ॒हद॒भिगा॑यत आहव॒नीय॑ आधी॒यमा॑ने। बार्\mbox{}ह॑तो॒ वा अ॒सौ लो॒कः। अ॒मुष्मि॑न्ने॒वैनं॑ लो॒के प्रति॑ष्ठित॒माध॑त्ते। प्र॒जाप॑तिर॒ग्निम॑सृजत॥६०॥

%1.1.8.3
सोऽश्वो॒ऽवारो॑ भू॒त्वा परा॑ङैत्। तं वा॑रव॒न्तीये॑नावारयत। तद्वा॑रव॒न्तीय॑स्य वारवन्तीय॒त्वम्। श्यै॒तेन॑ श्ये॒ती अ॑कुरुत। तच्छ्यै॒तस्य॑ श्यैत॒त्वम्। यद्वा॑रव॒न्तीय॑मभि॒ गाय॑ते। वा॒र॒यि॒त्वैवैनं॒ प्रति॑ष्ठित॒मा ध॑त्ते। श्यै॒तेन॑ श्ये॒ती कु॑रुते। घ॒र्मः शिर॒ इति॒ गार्\mbox{}ह॑पत्य॒माद॑धाति। सशी॑र्\mbox{}षाणमे॒वैन॒मा ध॑त्ते॥६१॥

%1.1.8.4
उपै॑न॒मुत्त॑रो य॒ज्ञो न॑मति। रु॒द्रो वा ए॒षः। यद॒ग्निः। स आ॑धी॒यमा॑न ईश्व॒रो यज॑मानस्य प॒शून् हिसि॑तोः। संप्रि॑यः प॒शुभि॑र्भुव॒दित्या॑ह। प॒शुभि॑रे॒वैन॒ संप्रि॑यं करोति। प॒शू॒नामहिसायै। छ॒र्दिस्तो॒काय॒ तन॑याय य॒च्छेत्या॑ह। आ॒शिष॑मे॒वैतामा शास्ते। वात॑ प्रा॒ण इत्य॑न्वाहार्य॒पच॑नम्॥६२॥

%1.1.8.5
सप्रा॑णमे॒वैन॒मा ध॑त्ते। स्व॒दि॒तं तो॒काय॒ तन॑याय पि॒तुं प॒चेत्या॑ह। अन्न॑मे॒वास्मै स्वदयति। प्राची॒मनु॑ प्र॒दिशं॒ प्रेहि॑ वि॒द्वानित्या॑ह। विभ॑क्तिरे॒वैन॑यो॒ सा। अथो॒ नाना॑वीर्यावे॒वैनौ॑ कुरुते। ऊर्जं॑ नो धेहि द्वि॒पदे॒ चतु॑ष्पद॒ इत्या॑ह। आ॒शिष॑मे॒वैतामा शास्ते। अ॒र्कश्चक्षु॒रित्या॑हव॒नीयम्। अ॒र्को वै दे॒वाना॒मन्नम्॥६३॥

%1.1.8.6
अन्न॑मे॒वाव॑ रुन्धे। तेन॑ मे दीदि॒हीत्या॑ह। समि॑न्ध ए॒वैनम्। आ॒न॒शे व्या॑नश॒ इति॒ त्रिरुदि॑ङ्गयति। त्रय॑ इ॒मे लो॒काः। ए॒ष्वे॑वैनं॑ लो॒केषु॒ प्रति॑ष्ठित॒मा ध॑त्ते। तत्तथा॒ न का॒र्यम्। वीङ्गि॑त॒मप्र॑तिष्ठित॒मा द॑धीत। उ॒द्धृत्यै॒वाधाया॑भि॒मन्त्रिय॑। अवीङ्गितमे॒वैनं॒ प्रति॑ष्ठित॒माध॑त्ते। वि॒राट्च॑ स्व॒राट्च॒ यास्ते॑ अग्ने शि॒वास्त॒नुव॒स्ताभि॒स्त्वाऽऽद॑ध॒ इत्या॑ह। ए॒ता वा अ॒ग्नेः शि॒वास्त॒नुव॑। ताभि॑रे॒वैन॒ सम॑र्धयति। यास्ते॑ अग्ने घो॒रास्त॒नुव॒स्ताभि॑र॒मुं ग॒च्छेति॑ ब्रूया॒द्यं द्वि॒ष्यात्। ताभि॑रे॒वैनं॒ परा॑भावयति॥६४॥\anuvakamend[लो॒को॑ऽसृजतैन॒माध॑त्तेऽन्वाहार्य॒पच॑नं दे॒वाना॒मन्न॑मेनं॒ प्रति॑ष्ठित॒माध॑त्ते॒ पञ्च॑ च]

%1.1.9.1
श॒मी॒ग॒र्भाद॒ग्निं म॑न्थति। ए॒षा वा अ॒ग्नेर्य॒ज्ञिया॑ त॒नूः। तामे॒वास्मै॑ जनयति। अदि॑तिः पु॒त्रका॑मा। सा॒ध्येभ्यो॑ दे॒वेभ्यो ब्रह्मौद॒नम॑पचत्। तस्या॑ उ॒च्छेष॑णमददुः। तत्प्राश्ञात्। सा रेतो॑ऽधत्त। तस्यै॑ धा॒ता चार्य॒मा चा॑जायेताम्। सा द्वि॒तीय॑मपचत्॥६५॥

%1.1.9.2
तस्या॑ उ॒च्छेष॑णमददुः। तत्प्राश्ञात्। सा रेतो॑ऽधत्त। तस्यै॑ मि॒त्रश्च॒ वरु॑णश्चाजायेताम्। सा तृ॒तीय॑मपचत्। तस्या॑ उ॒च्छेष॑णमददुः। तत्प्राश्ञात्। सा रेतो॑ऽधत्त। तस्या॒ अश॑श्च॒ भग॑श्चाजायेताम्। सा च॑तु॒र्थम॑पचत्॥६६॥

%1.1.9.3
तस्या॑ उ॒च्छेष॑णमददुः। तत्प्राश्ञात्। सा रेतो॑ऽधत्त। तस्या॒ इन्द्र॑श्च॒ विव॑स्वाश्चाजायेताम्। ब्र॒ह्मौ॒द॒नं प॑चति। रेत॑ ए॒व तद्द॑धाति। प्राश्ञ॑न्ति ब्राह्म॒णा ओ॑द॒नम्। यदाज्य॑मु॒च्छिष्य॑ते। तेन॑ स॒मिधो॒ऽभ्यज्या द॑धाति। उ॒च्छेष॑णा॒द्वा अदि॑ती॒ रेतो॑ऽधत्त॥६७॥

%1.1.9.4
उ॒च्छेष॑णादे॒व तद्रेतो॑ धत्ते। अस्थि॒ वा ए॒तत्। यत्स॒मिध॑। ए॒तद्रेत॑। यदाज्यम्। यदाज्ये॑न स॒मिधो॒ऽभ्यज्या॒दधा॑ति। अस्थ्ये॒व तद्रेत॑सि दधाति। ति॒स्र आद॑धाति मिथुन॒त्वाय॑। इय॑तीर्भवन्ति। प्र॒जाप॑तिना यज्ञमु॒खेन॒ सम्मि॑ताः॥६८॥

%1.1.9.5
इय॑तीर्भवन्ति। य॒ज्ञ॒प॒रुषा॒ सम्मि॑ताः। इय॑तीर्भवन्ति। ए॒ताव॒द्वै पुरु॑षे वी॒र्यम्। वी॒र्य॑संमिताः। आ॒र्द्रा भ॑वन्ति। आ॒र्द्रमि॑व॒ हि रेत॑ सि॒च्यते। चित्रि॑यस्याश्व॒त्थस्याद॑धाति। चि॒त्रमे॒व भ॑वति। घृ॒तव॑तीभि॒रा द॑धाति॥६९॥

%1.1.9.6
ए॒तद्वा अ॒ग्नेः प्रि॒यं धाम॑। यद्घृ॒तम्। प्रि॒येणै॒वैनं॒ धाम्ना॒ सम॑र्धयति। अथो॒ तेज॑सा। गा॒य॒त्रीभि॑र्ब्राह्म॒णस्याद॑ध्यात्। गा॒य॒त्रछ॑न्दा॒ वै ब्राह्म॒णः। स्वस्य॒ छन्द॑सः प्रत्ययन॒स्त्वाय॑। त्रि॒ष्टुग्भी॑ राज॒न्य॑स्य। त्रि॒ष्टुप्छ॑न्दा॒ वै रा॑ज॒न्य॑। स्वस्य॒ छन्द॑सः प्रत्ययन॒स्त्वाय॑॥७०॥

%1.1.9.7
जग॑तीभि॒र्वैश्य॑स्य। जग॑तीछन्दा॒ वै वैश्य॑। स्वस्य॒ छन्द॑सः प्रत्ययन॒स्त्वाय॑। त सं॑वत्स॒रं गो॑पायेत्। सं॒व॒त्स॒र हि रेतो॑ हि॒तं वर्ध॑ते। यद्ये॑न संवत्स॒रे नोप॒नमेत्। स॒मिध॒ पुन॒राद॑ध्यात्। रेत॑ ए॒व तद्धि॒तं वर्ध॑मानमेति। न मा॒सम॑श्ञीयात्। न स्त्रिय॒मुपे॑यात्॥७१॥

%1.1.9.8
यन्मा॒सम॑श्ञी॒यात्। यत्स्त्रिय॑मुपे॒यात्। निर्वीर्यः स्यात्। नैन॑म॒ग्निरुप॑नमेत्। श्व आ॑धा॒स्यमा॑नो ब्रह्मौद॒नं प॑चति। आ॒दि॒त्या वा इ॒त उ॑त्त॒माः सु॑व॒र्गं लो॒कमा॑यन्। ते वा इ॒तो यन्तं॒ प्रति॑नुदन्ते। ए॒ते खलु॒ वावादि॒त्याः। यद्ब्राह्म॒णाः। तैरे॒व स॒न्त्वं ग॑च्छति॥७२॥

%1.1.9.9
नैनं॒ प्रति॑नुदन्ते। ब्र॒ह्म॒वा॒दिनो॑ वदन्ति। क्वा॑ सः। अ॒ग्निः का॒र्य॑। योऽस्मै प्र॒जां प॒शून्प्र॑ज॒नय॒तीति॑। शल्कै॒स्तारात्रि॑म॒ग्निमि॑न्धीत। तस्मि॑न्नुपव्यु॒षम॒रणी॒ निष्ट॑पेत्। यथ॑र्\mbox{}ष॒भाय॑ वाशि॒ता न्या॑विच्छा॒यति॑। ता॒दृगे॒व तत्। अ॒पो॒दूह्य॒ भस्मा॒ग्निं म॑न्थति॥७३॥

%1.1.9.10
सैव साऽग्नेः सन्त॑तिः। तं म॑थि॒त्वा प्राञ्च॒मुद्ध॑रति। सं॒व॒त्स॒रमे॒व तद्रेतो॑ हि॒तं प्रज॑नयति। अना॑हित॒स्तस्या॒ग्निरित्या॑हुः। यः स॒मिधोऽना॑धाया॒ग्निमा॑ध॒त्त इति॑। ताः सं॑वत्स॒रे पु॒रस्ता॒दाद॑ध्यात्। सं॒व॒त्सरादे॒वैन॑मव॒रुध्याध॑त्ते। यदि॑ संवत्स॒रेऽनाद॒ध्यात्। द्वा॒द॒श्यां पु॒रस्ता॒दाद॑ध्यात्। सं॒व॒त्स॒रप्र॑तिमा॒ वै द्वाद॑श॒ रात्र॑यः। सं॒व॒त्स॒रमे॒वास्याहि॑ता भवन्ति। यदि॑ द्वाद॒श्यां नाद॒ध्यात्। त्र्य॒हे पु॒रस्ता॒दाद॑ध्यात्। आहि॑ता ए॒वास्य॑ भवन्ति॥७४॥\anuvakamend[द्वि॒तीय॑मपचच्चतु॒र्थम॑पच॒ददि॑ती॒ रेतो॑ऽधत्त॒ सम्मि॑ता घृ॒तव॑तीभि॒राद॑धाति राज॒न्य॑ स्वस्य॒ छन्द॑सः प्रत्ययन॒स्त्वाये॑याद्गच्छति मन्थति॒ रात्र॑यश्च॒त्वारि॑ च]

%1.1.10.1
प्र॒जाप॑तिः प्र॒जा अ॑सृजत। स रि॑रिचा॒नो॑ऽमन्यत। स तपो॑ऽतप्यत। स आ॒त्मन्वी॒र्य॑मपश्यत्। तद॑वर्धत। तद॑स्मा॒त्सह॑सो॒र्ध्वम॑सृज्यत। सा वि॒राड॑भवत्। तां दे॑वासु॒रा व्य॑गृह्णत। सोऽब्रवीत्प्र॒जाप॑तिः। मम॒ वा ए॒षा॥७५॥

%1.1.10.2
दोहा॑ ए॒व यु॒ष्माक॒मिति॑। सा तत॒ प्राच्युद॑क्रामत्। तत्प्र॒जाप॑ति॒ पर्य॑गृह्णात्। अथ॑र्व पि॒तुं मे॑ गोपा॒येति॑। सा द्वि॒तीय॒मुद॑क्रामत्। तत्प्र॒जाप॑ति॒ पर्य॑गृह्णात्। नर्य॑ प्र॒जां मे॑ गोपा॒येति॑। सा तृ॒तीय॒मुद॑क्रामत्। तत्प्र॒जाप॑ति॒ पर्य॑गृह्णात्। शस्य॑ प॒शून्मे॑ गोपा॒येति॑॥७६॥

%1.1.10.3
सा च॑तु॒र्थमुद॑क्रामत्। तत्प्र॒जाप॑ति॒ पर्य॑गृह्णात्। सप्र॑थ स॒भां मे॑ गोपा॒येति॑। सा प॑ञ्च॒ममुद॑क्रामत्। तत्प्र॒जाप॑ति॒ पर्य॑गृह्णात्। अहे॑ बुध्निय॒ मन्त्रं॑ मे गोपा॒येति॑। अ॒ग्नीन् वाव सा तान्व्य॑क्रमत। तान्प्र॒जाप॑ति॒ पर्य॑गृह्णात्। अथो॑ प॒ङ्क्तिमे॒व। प॒ङ्क्तिर्वा ए॒षा ब्राह्म॒णे प्रवि॑ष्टा॥७७॥

%1.1.10.4
तामा॒त्मनोऽधि॒ निर्मि॑मीते। यद॒ग्निरा॑धी॒यते। तस्मा॑दे॒ताव॑न्तो॒ऽग्नय॒ आधी॑यन्ते। पाङ्क्तं॒ वा इ॒द सर्वम्। पाङ्क्ते॑नै॒व पाङ्क्त स्पृणोति। अथ॑र्व पि॒तुं मे॑ गोपा॒येत्या॑ह। अन्न॑मे॒वैतेन॑ स्पृणोति। नर्य॑ प्र॒जां मे॑ गोपा॒येत्या॑ह। प्र॒जामे॒वैतेन॑ स्पृणोति। शस्य॑ प॒शून्मे॑ गोपा॒येत्या॑ह॥७८॥

%1.1.10.5
प॒शूने॒वैतेन॑ स्पृणोति। सप्र॑थ स॒भां मे॑ गोपा॒येत्या॑ह। स॒भामे॒वैतेनेन्द्रि॒य स्पृ॑णोति। अहे॑ बुध्निय॒ मन्त्रं॑ मे गोपा॒येत्या॑ह। मन्त्र॑मे॒वैतेन॒ श्रिय स्पृणोति। यदा॑न्वाहार्य॒पच॑नेऽन्वाहा॒र्यं॑ पच॑न्ति। तेन॒ सोऽस्या॒भीष्ट॑ प्री॒तः। यद्गार्\mbox{}ह॑पत्य॒ आज्य॑मधि॒श्रय॑न्ति॒ संपत्नीर्या॒जय॑न्ति। तेन॒ सोऽस्या॒भीष्ट॑ प्री॒तः। यदा॑हव॒नीये॒ जुह्व॑ति॥७९॥

%1.1.10.6
तेन॒ सोऽस्या॒भीष्ट॑ प्री॒तः। यत्स॒भायां वि॒जय॑न्ते। तेन॒ सोऽस्या॒भीष्ट॑ प्री॒तः। यदा॑वस॒थेऽन्न॒ हर॑न्ति। तेन॒ सोऽस्या॒भीष्ट॑ प्री॒तः। तथाऽस्य॒ सर्वे प्री॒ता अ॒भीष्टा॒ आधी॑यन्ते। प्र॒व॒स॒थमे॒ष्यन्ने॒वमुप॑तिष्ठे॒तैक॑मेकम्। यथा ब्राह्म॒णाय॑ गृहेवा॒सिने॑ परि॒दाय॑ गृ॒हानेति॑। ता॒दृगे॒व तत्। पुन॑रा॒गत्योप॑तिष्ठते। सा भा॑गेयमे॒वैषां॒ तत्। सा तत॑ ऊ॒र्ध्वारो॑हत्। सा रो॑हि॒ण्य॑भवत्। तद्रो॑हि॒ण्यै रो॑हिणि॒त्वम्। रो॒हि॒ण्याम॒ग्निमाद॑धीत। स्व ए॒वैनं॒ योनौ॒ प्रति॑ष्ठित॒माध॑त्ते। ऋ॒ध्नोत्ये॑नेन॥८०॥\anuvakamend[ए॒षा प॒शून्मे॑ गोपा॒येति॒ प्रवि॑ष्टा प॒शून्मे॑ गोपा॒येत्या॑ह॒ जुह्व॑ति तिष्ठते स॒प्त च॑]





\prashnaend{ब्रह्म॒ संध॑त्तं॒ कृत्ति॑का॒सूद्ध॑न्ति द्वाद॒शसु॑ प्र॒जाप॑तिर्वा॒चो दे॑वासु॒रास्तद॒ग्निर्नेद्घ॒र्मश्शिर॑ इ॒मे वै श॑मीग॒र्भात्प्र॒जाप॑ति॒ स रि॑रिचा॒नः स तप॒ स आ॒त्मन्वी॒र्यं॑दश॑॥१०॥}{ब्रह्म॒ सन्ध॑त्तं॒ तौ दि॒व्यावथो॑ शं॒त्वाय॒ प्राच्ये॑षां॒ यदु॒पर्यु॑परि॒ यत्स॒द्यः सोऽश्वो॒ऽवारो॑ भू॒त्वा जग॑तीभि॒रशी॑तिः॥८०॥}{ब्रह्म॒ सन्ध॑त्तमृ॒ध्नोत्ये॑नेन॥}{हरि॑ ओम्॥}{इति श्रीकृष्णयजुर्वेदीयतैत्तिरीयब्राह्मणे प्रथमाष्टके प्रथमः प्रपाठकः समाप्तः॥}
\clearpage
\sect{द्वितीयः प्रश्नः}
\setcounter{anuvakam}{0}
\dnsub{तैत्तिरीयब्राह्मणे प्रथमाष्टके द्वितीयः प्रपाठकः}

%1.2.1.1
उ॒द्ध॒न्यमा॑नम॒स्या अ॑मे॒ध्यम्। अप॑ पा॒प्मानं॒ यज॑मानस्य हन्तु। शि॒वा न॑ सन्तु प्र॒दिश॒श्चत॑स्रः। शं नो॑ मा॒ता पृ॑थि॒वी तोक॑साता। शं नो॑ दे॒वीर॒भिष्ट॑ये। आपो॑ भवन्तु पी॒तये। शंयोर॒भि स्र॑वन्तु नः। वै॒श्वा॒न॒रस्य॑ रू॒पम्। पृ॒थि॒व्यां प॑रि॒स्रसा। स्यो॒नमा वि॑शन्तु नः॥१॥

%1.2.1.2
यदि॒दं दि॒वो यद॒दः पृ॑थि॒व्याः। सं॒ज॒ज्ञा॒ने रोद॑सी सम्बभू॒वतु॑। ऊषान्कृ॒ष्णम॑वतु कृ॒ष्णमूषा। इ॒होभयोर्य॒ज्ञिय॒माग॑मिष्ठाः। ऊ॒तीः कु॑र्वा॒णो यत्पृ॑थि॒वीमच॑रः। गु॒हा॒कार॑माखुरू॒पं प्र॒तीत्य॑। तत्ते॒ न्य॑क्तमि॒ह स॒म्भर॑न्तः। श॒तं जी॑वेम श॒रद॒ सवी॑राः। ऊर्जं॑ पृथि॒व्या रस॑मा॒भर॑न्तः। श॒तं जी॑वेम श॒रद॑ पुरू॒चीः॥२॥

%1.2.1.3
व॒म्रीभि॒रनु॑वित्तं॒ गुहा॑सु। श्रोत्रं॑ त उ॒र्व्यब॑धिरा भवामः। प्र॒जाप॑तिसृष्टानां प्र॒जानाम्। क्षु॒धोऽप॑हत्यै सुवि॒तं नो॑ अस्तु। उप॒ प्रभि॑न्न॒मिष॒मूर्जं॑ प्र॒जाभ्य॑। सूदं॑ गृ॒हेभ्यो॒ रस॒माभ॑रामि। यस्य॑ रू॒पं बिभ्र॑दि॒मामवि॑न्दत्। गुहा॒ प्रवि॑ष्टा सरि॒रस्य॒ मध्ये। तस्ये॒दं विह॑तमा॒भर॑न्तः। अछ॑म्बट्कारम॒स्यां वि॑धेम॥३॥

%1.2.1.4
यत्प॒र्यप॑श्यत्सरि॒रस्य॒ मध्ये। उ॒र्वीमप॑श्य॒ज्जग॑तः प्रति॒ष्ठाम्। तत्पुष्क॑रस्या॒यत॑ना॒द्धि जा॒तम्। प॒र्णं पृ॑थि॒व्याः प्रथ॑न हरामि। याभि॒रदृह॒ज्जग॑तः प्रति॒ष्ठाम्। उ॒र्वीमि॒मां वि॑श्वज॒नस्य॑ भ॒र्त्रीम्। ता न॑ शि॒वाः शर्क॑राः सन्तु॒ सर्वा। अ॒ग्ने रेत॑श्च॒न्द्र हिर॑ण्यम्। अ॒द्भ्यः सम्भू॑तम॒मृतं॑ प्र॒जासु॑। तत्स॒म्भर॑न्नुत्तर॒तो नि॒धाय॑॥४॥

%1.2.1.5
अ॒ति॒प्र॒यच्छं॒ दुरि॑तिं तरेयम्। अश्वो॑ रू॒पं कृ॒त्वा यद॑श्व॒त्थेऽति॑ष्ठः। सं॒व॒त्स॒रं दे॒वेभ्यो॑ नि॒लाय॑। तत्ते॒ न्य॑क्तमि॒ह स॒म्भर॑न्तः। श॒तं जी॑वेम श॒रद॒ सवी॑राः। ऊ॒र्जः पृ॑थि॒व्या अध्युत्थि॑तोऽसि। वन॑स्पते श॒तव॑ल्‌शो॒ विरो॑ह। त्वया॑ व॒यमिष॒मूर्जं॒ मद॑न्तः। रा॒यस्पोषे॑ण॒ समि॒षा म॑देम। गा॒य॒त्रि॒या ह्रि॒यमा॑णस्य॒ यत्ते॥५॥

%1.2.1.6
प॒र्णमप॑तत्तृ॒तीय॑स्यै दि॒वोऽधि॑। सो॑ऽयं प॒र्णः सो॑मप॒र्णाद्धि जा॒तः। ततो॑ हरामि सोमपी॒थस्याव॑रुद्ध्यै। दे॒वानां ब्रह्मवा॒दं वद॑तां॒ यत्। उ॒पाशृ॑णोः सु॒श्रवा॒ वै श्रु॒तो॑ऽसि। ततो॒ मामावि॑शतु ब्रह्मवर्च॒सम्। तत्स॒म्भर॒ स्तदव॑रुन्धीय सा॒क्षात्। यया॑ ते सृ॒ष्टस्या॒ग्नेः। हे॒तिमश॑मयत्प्र॒जाप॑तिः। तामि॒मामप्र॑दाहाय॥६॥

%1.2.1.7
श॒मी शान्त्यै॑ हराम्य॒हम्। यत्ते॑ सृ॒ष्टस्य॑ य॒तः। विक॑ङ्कतं॒ भा आर्च्छज्जातवेदः। तया॑ भा॒सा सम्मि॑तः। उ॒रुं नो॑ लो॒कमनु॒ प्रभा॑हि। यत्ते॑ ता॒न्तस्य॒ हृद॑य॒माच्छि॑न्दञ्जातवेदः। म॒रुतो॒ऽद्भिस्त॑मयि॒त्वा। ए॒तत्ते॒ तद॑श॒नेः सम्भ॑रामि। सात्मा॑ अग्ने॒ सहृ॑दयो भवे॒ह। चित्रि॑यादश्व॒त्थात्सम्भृ॑ता बृह॒त्य॑॥७॥

%1.2.1.8
शरी॑रम॒भि सस्कृ॑ताः स्थ। प्र॒जाप॑तिना यज्ञमु॒खेन॒ सम्मि॑ताः। ति॒स्रस्त्रि॒वृद्भि॑र्मिथु॒नाः प्रजात्यै। अ॒श्व॒त्थाद्ध॑व्य\-वा॒हाद्धि जा॒ताम्। अ॒ग्नेस्त॒नूं य॒ज्ञिया॒ सम्भ॑रामि। शा॒न्तयो॑नि शमीग॒र्भम्। अ॒ग्नये॒ प्रज॑नयि॒तवे। यो अ॑श्व॒त्थः श॑मीग॒र्भः। आ॒रु॒रोह॒ त्वे सचा। तं ते॑ हरामि॒ ब्रह्म॑णा॥८॥

%1.2.1.9
य॒ज्ञियै के॒तुभि॑ स॒ह। यं त्वा॑ स॒मभ॑रञ्जातवेदः। य॒था॒श॒री॒रं भू॒तेषु॒ न्य॑क्तम्। स सम्भृ॑तः सीद शि॒वः प्र॒जाभ्य॑। उ॒रुं नो॑ लो॒कमनु॑नेषि वि॒द्वान्। प्रवे॒धसे॑ क॒वये॒ मेध्या॑य। वचो॑ व॒न्दारु॑ वृष॒भाय॒ वृष्णे। यतो॑ भ॒यमभ॑यं॒ तन्नो॑ अस्तु। अव॑ दे॒वान् य॑जे॒हेड्यान्॑। स॒मिधा॒ऽग्निं दु॑वस्यत॥९॥

%1.2.1.10
घृ॒तैर्बो॑धय॒ताति॑थिम्। आऽस्मि॑न् ह॒व्या जु॑होतन। उप॑ त्वाऽग्ने ह॒विष्म॑तीः। घृ॒ताचीर्यन्तु हर्यत। जु॒षस्व॑ स॒मिधो॒ मम॑। तं त्वा॑ स॒मिद्भि॑रङ्गिरः। घृ॒तेन॑ वर्धयामसि। बृ॒हच्छो॑चा यविष्ठ्य। स॒मि॒ध्यमा॑नः प्रथ॒मो नु धर्म॑। सम॒क्तुभि॑रज्यते वि॒श्ववा॑रः॥१०॥

%1.2.1.11
शो॒चिष्के॑शो घृ॒तनि॑र्णिक्पाव॒कः। सु॒य॒ज्ञो अ॒ग्निर्य॒जथा॑य दे॒वान्। घृ॒तप्र॑तीको घृ॒तयो॑निर॒ग्निः। घृ॒तैः समि॑द्धो घृ॒तम॒स्यान्नम्। घृ॒त॒प्रुष॑स्त्वा स॒रितो॑ वहन्ति। घृ॒तं पिबन्त्सु॒यजा॑ यक्षि दे॒वान्। आ॒यु॒र्दा अ॑ग्ने ह॒विषो॑ जुषा॒णः। घृ॒तप्र॑तीको घृ॒तयो॑निरेधि। घृ॒तं पी॒त्वा मधु॒ चारु॒ गव्यम्। पि॒तेव॑ पु॒त्रम॒भिर॑क्षतादि॒मम्॥११॥

%1.2.1.12
त्वाम॑ग्ने समिधा॒नं य॑विष्ठ। दे॒वा दू॒तं च॑क्रिरे हव्य॒वाहम्। उ॒रु॒ज्रय॑सं घृ॒तयो॑नि॒माहु॑तम्। त्वे॒षं चक्षु॑र्दधिरे चोद॒यन्व॑ति। त्वाम॑ग्ने प्र॒दिव॒ आहु॑तं घृ॒तेन॑। सु॒म्ना॒यव॑ सुष॒मिधा॒ समी॑धिरे। स वा॑वृधा॒न ओष॑धीभिरुक्षि॒तः। उ॒रु ज्रयासि॒ पार्थि॑वा॒ विति॑ष्ठसे। घृ॒तप्र॑तीकं व ऋ॒तस्य॑ धूर्॒षदम्। अ॒ग्निं मि॒त्रं न स॑मिधा॒न ऋ॑ञ्जते॥१२॥

%1.2.1.13
इन्धा॑नो अ॒क्रो वि॒दथे॑षु॒ दीद्य॑त्। शु॒क्रव॑र्णा॒मुदु॑ नो यसते॒ धियम्। प्र॒जा अ॑ग्ने॒ संवा॑सय। आशाश्च प॒शुभि॑ स॒ह। रा॒ष्ट्राण्य॑स्मा॒ आधे॑हि। यान्यासन्त्सवि॒तुः स॒वे। म॒ही वि॒श्पत्नी॒ सद॑ने ऋ॒तस्य॑। अ॒र्वाची॒ एतं॑ धरुणे रयी॒णाम्। अ॒न्तर्व॑त्नी॒ जन्यं॑ जा॒तवे॑दसम्। अ॒ध्व॒राणां जनयथः पुरो॒गाम्॥१३॥

%1.2.1.14
आरो॑हतं द॒शत॒ शक्व॑री॒र्मम॑। ऋ॒तेनाग्न॒ आयु॑षा॒ वर्च॑सा स॒ह। ज्योग्जीव॑न्त॒ उत्त॑रामुत्तरा॒ समाम्। दर्\mbox{}श॑म॒हं पू॒र्णमा॑सं य॒ज्ञं यथा॒ यजै। ऋत्वि॑यवती स्थो अ॒ग्निरे॑तसौ। गर्भं॑ दधाथां॒ ते वा॑म॒हं द॑दे। तत्स॒त्यं यद्वी॒रं बि॑भृथः। वी॒रं ज॑नयि॒ष्यथ॑। ते मत्प्रा॒तः प्रज॑निष्येथे। ते मा॒ प्रजा॑ते॒ प्रज॑नयि॒ष्यथ॑॥१४॥

%1.2.1.15
प्र॒जया॑ प॒शुभि॑र्ब्रह्मवर्च॒सेन॑ सुव॒र्गे लो॒के। अनृ॑तात्स॒त्यमुपै॑मि। मा॒नु॒षाद्देव्य॒मुपै॑मि। दैवीं॒ वाचं॑ यच्छामि। शल्कै॑र॒ग्निमि॑न्धा॒नः। उ॒भौ लो॒कौ स॑नेम॒हम्। उ॒भयोर्लो॒कयोर्॑ ऋ॒ध्वा। अति॑ मृ॒त्युं त॑राम्य॒हम्। जात॑वेदो॒ भुव॑नस्य॒ रेत॑। इ॒ह सि॑ञ्च॒ तप॑सो॒ यज्ज॑नि॒ष्यते॥१५॥

%1.2.1.16
अ॒ग्निम॑श्व॒त्थादधि॑ हव्य॒वाहम्। श॒मी॒ग॒र्भाज्ज॒नय॒न्॒ यो म॑यो॒भूः। अ॒यं ते॒ योनि॑र्‌ऋ॒त्विय॑। यतो॑ जा॒तो अरो॑चथाः। तं जा॒नन्न॑ग्न॒ आरो॑ह। अथा॑ नो वर्धया र॒यिम्। अपे॑त॒ वीत॒ वि च॑ सर्प॒तात॑। येऽत्र॒ स्थ पु॑रा॒णा ये च॒ नूत॑नाः। अदा॑दि॒दं य॒मो॑ऽव॒सानं॑ पृथि॒व्याः। अक्र॑न्नि॒मं पि॒तरो॑ लो॒कम॑स्मै॥१६॥

%1.2.1.17
अ॒ग्नेर्भस्मास्य॒ग्नेः पुरी॑षमसि। सं॒ज्ञान॑मसि काम॒धर॑णम्। मयि॑ ते काम॒धर॑णं भूयात्। संव॑ सृजामि॒ हृद॑यानि। ससृ॑ष्टं॒ मनो॑ अस्तु वः। संसृ॑ष्टः प्रा॒णो अ॑स्तु वः। सं या व॑ प्रि॒यास्त॒नुव॑। सं प्रि॒या हृद॑यानि वः। आ॒त्मा वो॑ अस्तु॒ संप्रि॑यः। संप्रि॑यास्त॒नुवो॒ मम॑॥१७॥

%1.2.1.18
कल्पे॑तां॒ द्यावा॑पृथि॒वी। कल्प॑न्ता॒माप॒ ओष॑धीः। कल्प॑न्ताम॒ग्नय॒ पृथ॑क्। मम॒ ज्यैष्ठ्या॑य॒ सव्र॑ताः। येऽग्नय॒ सम॑नसः। अ॒न्त॒रा द्यावा॑पृथि॒वी। वास॑न्तिकावृ॒तू अ॒भि कल्प॑मानाः। इन्द्र॑मिव दे॒वा अ॒भि सं वि॑शन्तु। दि॒वस्त्वा॑ वी॒र्ये॑ण। पृ॒थि॒व्यै म॑हि॒म्ना॥१८॥

%1.2.1.19
अ॒न्तरि॑क्षस्य॒ पोषे॑ण। स॒र्वप॑शु॒माद॑धे। अजी॑जनन्न॒मृतं॒ मर्त्या॑सः। अ॒स्रे॒माणं॑ त॒रणिं॑ वी॒डुज॑म्भम्। दश॒ स्वसा॑रो अ॒ग्रुव॑ समी॒चीः। पुमासं जा॒तम॒भि सर॑भन्ताम्। प्र॒जाप॑तेस्त्वा प्रा॒णेनाभि॒ प्राणि॑मि। पू॒ष्णः पोषे॑ण॒ मह्यम्। दी॒र्घा॒यु॒त्वाय॑ श॒तशा॑रदाय। श॒त श॒रद्भ्य॒ आयु॑षे॒ वर्च॑से॥१९॥

%1.2.1.20
जी॒वात्वै पुण्या॑य। अ॒हं त्वद॑स्मि॒ मद॑सि॒ त्वमे॒तत्। ममा॑सि॒ योनि॒स्तव॒ योनि॑रस्मि। ममै॒व सन्वह॑ ह॒व्यान्य॑ग्ने। पु॒त्रः पि॒त्रे लो॑क॒कृज्जा॑तवेदः। प्रा॒णे त्वा॒ऽमृत॒माद॑धामि। अ॒न्ना॒दम॒न्नाद्या॑य। गो॒प्तारं॒ गुप्त्यै। सु॒गा॒र्॒ह॒प॒त्यो वि॒दह॒न्नरा॑तीः। उ॒षस॒ श्रेय॑सीः श्रेयसी॒र्दध॑त्॥२०॥

%1.2.1.21
अग्ने॑ स॒पत्ना अप॒ बाध॑मानः। रा॒यस्पोष॒मिष॒मूर्ज॑म॒स्मासु॑ धेहि। इ॒मा उ॒ मामुप॑तिष्ठन्तु॒ राय॑। आ॒भिः प्र॒जाभि॑रि॒ह संव॑सेय। इ॒हो इडा॑ तिष्ठतु विश्वरू॒पी। मध्ये॒ वसोर्दीदिहि जातवेदः। ओज॑से॒ बला॑य॒ त्वोद्य॑च्छे। वृष॑णे॒ शुष्मा॒यायु॑षे॒ वर्च॑से। स॒प॒त्न॒तूर॑सि वृत्र॒तूः। यस्ते॑ दे॒वेषु॑ महि॒मा सु॑व॒र्गः॥२१॥

%1.2.1.22
यस्त॑ आ॒त्मा प॒शुषु॒ प्रवि॑ष्टः। पुष्टि॒र्या ते॑ मनु॒ष्ये॑षु पप्र॒थे। तया॑ नो अग्ने जु॒षमा॑ण॒ एहि॑। दि॒वः पृ॑थि॒व्याः पर्य॒न्तिरि॑क्षात्। वातात्प॒शुभ्यो॒ अध्योष॑धीभ्यः। यत्र॑ यत्र जातवेदः सम्ब॒भूथ॑। ततो॑ नो अग्ने जु॒षमा॑ण॒ एहि॑। प्राची॒मनु॑ प्र॒दिशं॒ प्रेहि॑ वि॒द्वान्। अ॒ग्नेर॑ग्ने पु॒रोअ॑ग्निर्भवे॒ह। विश्वा॒ आशा॒ दीद्या॑नो॒ वि भा॑हि॥२२॥

%1.2.1.23
ऊर्जं॑ नो धेहि द्वि॒पदे॒ चतु॑ष्पदे। अन्व॒ग्निरु॒षसा॒मग्र॑मख्यत्। अन्वहा॑नि प्रथ॒मो जा॒तवे॑दाः। अनु॒ सूर्य॑स्य पुरु॒त्रा च॑ र॒श्मीन्। अनु॒ द्यावा॑पृथि॒वी आत॑तान। विक्र॑मस्व म॒हा अ॑सि। वे॒दि॒षन्मानु॑षेभ्यः। त्रि॒षु लो॒केषु॑ जागृहि। यदि॒दं दि॒वो यद॒दः पृ॑थि॒व्याः। सं॒वि॒दा॒ने रोद॑सी सं बभू॒वतु॑॥२३॥

%1.2.1.24
तयो पृ॒ष्ठे सी॑दतु जा॒तवे॑दाः। श॒म्भूः प्र॒जाभ्य॑स्त॒नुवे स्यो॒नः। प्रा॒णं त्वा॒ऽमृत॒ आ द॑धामि। अ॒न्ना॒दम॒न्नाद्या॑य। गो॒प्तारं॒ गुप्त्यै। यत्ते॑ शुक्र शु॒क्रं वर्च॑ शु॒क्रा त॒नूः। शु॒क्रं ज्योति॒रज॑स्रम्। तेन॑ मे दीदिहि॒ तेन॒ त्वाऽऽद॑धे। अ॒ग्निनाऽग्ने॒ ब्रह्म॑णा। आ॒न॒शे व्या॑नशे॒ सर्व॒मायु॒र्व्या॑नशे॥२४॥

%1.2.1.25
नर्य॑ प्र॒जां मे॑ गोपाय। अ॒मृ॒त॒त्वाय॑ जी॒वसे। जा॒तां ज॑नि॒ष्यमा॑णां च। अ॒मृते॑ स॒त्ये प्रति॑ष्ठिताम्। अथ॑र्व पि॒तुं मे॑ गोपाय। रस॒मन्न॑मि॒हायु॑षे। अद॑ब्धा॒योऽशी॑ततनो। अवि॑षन्नः पि॒तुं कृ॑णु। शस्य॑ प॒शून्मे॑ गोपाय। द्वि॒पादो॒ ये चतु॑ष्पदः॥२५॥

%1.2.1.26
अ॒ष्टाश॑फाश्च॒ य इ॒हाग्ने। ये चैक॑शफा आशु॒गाः। सप्र॑थ स॒भां मे॑ गोपाय। ये च॒ सभ्या सभा॒सद॑। तानि॑न्द्रि॒याव॑तः कुरु। सर्व॒मायु॒रुपा॑सताम्। अहे॑ बुध्निय॒ मन्त्रं॑ मे गोपाय। यमृष॑यस्त्रैवि॒दा वि॒दुः। ऋच॒ सामा॑नि॒ यजूषि। सा हि श्रीर॒मृता॑ स॒ताम्॥२६॥

%1.2.1.27
चतु॑ शिखण्डा युव॒तिः सु॒पेशा। घृ॒तप्र॑तीका॒ भुव॑नस्य॒ मध्ये। म॒र्मृ॒ज्यमा॑ना मह॒ते सौभ॑गाय। मह्यं॑ धुक्ष्व॒ यज॑मानाय॒ कामान्॑। इ॒हैव सन्तत्र॑ स॒तो वो॑ अग्नयः। प्रा॒णेन॑ वा॒चा मन॑सा बिभर्मि। ति॒रो मा॒ सन्त॒मायु॒र्मा प्रहा॑सीत्। ज्योति॑षा वो वैश्वान॒रेणोप॑तिष्ठे। प॒ञ्च॒धाऽग्नीन्व्य॑क्रामत्। वि॒राट्त्सृ॒ष्टा प्र॒जाप॑तेः। ऊ॒र्ध्वाऽऽरो॑हद्रोहि॒णी। योनि॑र॒ग्नेः प्रति॑ष्ठितिः॥२७॥\anuvakamend[वि॒श॒न्तु॒ न॒ पु॒रू॒चीर्वि॑धेम नि॒धाय॒ यत्तेऽप्र॑दाहाय बृह॒त्यो ब्रह्म॑णा दुवस्यत वि॒श्ववा॑र इ॒ममृ॑ञ्जते पुरो॒गां प्रज॑नयि॒ष्यथो॑ जनि॒ष्यतेऽस्मै॒ मम॑ महि॒म्ना वर्च॑से॒ दध॑त्सुव॒र्गो भा॑हि सम्बभू॒वतु॒रायु॒र्व्या॑नशे॒ चतु॑ष्सदः स॒तां प्र॒जाप॑ते॒र्द्वे च॑]

%1.2.2.1
नवै॒तान्यहा॑नि भवन्ति। नव॒ वै सु॑व॒र्गा लो॒काः। यदे॒तान्यहान्युप॒यन्ति॑। न॒वस्वे॒व तत्सु॑व॒र्गेषु॑ लो॒केषु॑ स॒त्रिण॑ प्रति॒तिष्ठ॑न्तो यन्ति। अ॒ग्नि॒ष्टो॒माः पर॑ सामानः का॒र्या॑ इत्या॑हुः। अ॒ग्नि॒ष्टो॒मसं॑मितः सुव॒र्गो लो॒क इति॑। द्वाद॑शाग्निष्टो॒मस्य॑ स्तो॒त्राणि॑। द्वाद॑श॒ मासा संवत्स॒रः। तत्तन्न सूर्क्ष्यम्। उ॒क्थ्या॑ ए॒व स॑प्तद॒शाः पर॑ सामानः का॒र्या॥२८॥

%1.2.2.2
प॒शवो॒ वा उ॒क्थानि॑। प॒शू॒नामव॑रुद्ध्यै। वि॒श्व॒जि॒द॒भि॒जिता॑\-वग्निष्टो॒मौ। उ॒क्थ्या सप्तद॒शाः पर॑ समानः। ते सस्तु॑ता वि॒राज॑म॒भि सम्प॑द्यन्ते। द्वे चर्चा॒वति॑रिच्येते। एक॑या॒ गौरति॑रिक्तः। एक॒याऽऽयु॑रू॒नः। सु॒व॒र्गो वै लो॒को ज्योति॑। ऊर्ग्वि॒राट्॥२९॥

%1.2.2.3
सु॒व॒र्गमे॒व तेन॑ लो॒कम॒भि ज॑यन्ति। यत्पर॒ राथ॑न्तरम्। तत्प्र॑थ॒मेऽह॑न्का॒र्यम्। बृ॒हद्द्वि॒तीये। वै॒रू॒पं तृ॒तीये। वै॒रा॒जं च॑तु॒र्थे। शा॒क्व॒रं प॑ञ्च॒मे। रै॒व॒त ष॒ष्ठे। तदु॑ पृ॒ष्ठेभ्यो॒ नय॑न्ति। स॒न्तन॑य ए॒ते ग्रहा॑ गृह्यन्ते॥३०॥

%1.2.2.4
अ॒ति॒ग्रा॒ह्या पर॑ सामसु। इ॒माने॒वैतैर्लो॒कान्त्संत॑न्वन्ति। मि॒थु॒ना ए॒ते ग्रहा॑ गृह्यन्ते। अ॒ति॒ग्रा॒ह्या पर॑ सामसु। मि॒थु॒नमे॒व तैर्यज॑माना॒ अव॑रुन्धते। बृ॒हत्पृ॒ष्ठं भ॑वति। बृ॒हद्वै सु॑व॒र्गो लो॒कः। बृ॒ह॒तैव सु॑व॒र्गं लो॒कं य॑न्ति। त्र॒य॒स्त्रि॒शिनाम॒ साम॑। माध्य॑न्दिने॒ पव॑माने भवति॥३१॥

%1.2.2.5
त्रय॑स्त्रिश॒द्वै दे॒वता। दे॒वता॑ ए॒वाव॑रुन्धते। ये वा इ॒तः पराञ्च संवत्स॒रमु॑प॒यन्ति॑। न है॑नं॒ ते स्व॒स्ति सम॑श्ञुवते। अथ॒ ये॑ऽमुतो॒ऽर्वाञ्च॑मुप॒यन्ति॑। ते है॑न स्व॒स्ति सम॑श्ञुवते। ए॒तद्वा अ॒मुतो॒ऽर्वाञ्च॒मुप॑यन्ति। यदे॒वम्। यो ह॒ खलु॒ वाव प्र॒जाप॑तिः। स उ॑वे॒वेन्द्र॑। तदु॑ दे॒वेभ्यो॒ नय॑न्ति॥३२॥\anuvakamend[का॒र्या॑ वि॒राड्गृ॑ह्यन्ते॒ पव॑माने भव॒तीन्द्र॒ एकं॑ च]

%1.2.3.1
संत॑ति॒र्वा ए॒ते ग्रहा। यत्पर॑ सामानः। वि॒षू॒वान्दि॑वाकी॒र्त्यम्। यथा॒ शाला॑यै॒ पक्ष॑सी। ए॒व सं॑वत्स॒रस्य॒ पक्ष॑सी। यदे॒तेन गृ॒ह्येर\sn{}। विषू॑ची संवत्स॒रस्य॒ पक्ष॑सी॒ व्यव॑स्रसेयाताम्। आर्ति॒मार्च्छे॑युः। यदे॒ते गृ॒ह्यन्ते। यथा॒ शाला॑यै॒ प॑क्षसी मध्य॒मं व॒शम॒भि स॑मा॒यच्छ॑ति॥३३॥

%1.2.3.2
ए॒व सं॑वत्स॒रस्य॒ पक्ष॑सी दिवाकी॒र्त्य॑म॒भि सं त॑न्वन्ति। नार्ति॒मार्च्छ॑न्ति। ए॒क॒वि॒शमह॑र्भवति। शु॒क्राग्रा॒ ग्रहा॑ गृह्यन्ते। प्रत्युत्त॑ब्ध्यै सय॒त्वाय॑। सौ॒र्य॑ ए॒तदह॑ प॒शुराल॑भ्यते। सौ॒र्यो॑ऽतिग्रा॒ह्यो॑ गृह्यते। अह॑रे॒व रू॒पेण॒ सम॑र्धयन्ति। अथो॒ अह्न॑ ए॒वैष ब॒लिर्ह्रि॑यते। स॒प्तैतदह॑रतिग्रा॒ह्या॑ गृह्यन्ते॥३४॥

%1.2.3.3
स॒प्त वै शी॑र्\mbox{}ष॒ण्या प्रा॒णाः। अ॒सावा॑दि॒त्यः शिर॑ प्र॒जानाम्। शी॒र्॒षन्ने॒व प्र॒जानां प्रा॒णान्द॑धाति। तस्मात्स॒प्त शी॒र्॒षन्प्रा॒णाः। इन्द्रो॑ वृ॒त्र ह॒त्वा। असु॑रान्परा॒भाव्य॑। स इ॒माल्लोँ॒कान॒भ्य॑जयत्। तस्या॒सौ लो॒कोऽन॑भिजित आसीत्। तं वि॒श्वक॑र्मा भू॒त्वाऽभ्य॑जयत्। यद्वैश्वकर्म॒णो गृ॒ह्यते॥३५॥

%1.2.3.4
सु॒व॒र्गस्य॑ लो॒कस्या॒भिजि॑त्यै। प्र वा ए॒तेऽस्माल्लो॒काच्च्य॑वन्ते। ये वैश्वकर्म॒णं गृ॒ह्णते। आ॒दि॒त्यः श्वो गृ॑ह्यते। इ॒यं वा अदि॑तिः। अ॒स्यामे॒व प्रति॑ तिष्ठन्ति। अ॒न्योन्यो गृह्येते। विश्वान्ये॒वान्येन॒ कर्मा॑णि कुर्वा॒णा य॑न्ति। अ॒स्याम॒न्येन॒ प्रति॑ तिष्ठन्ति। तावाऽप॑रा॒र्धात्सं॑वत्स॒रस्या॒न्योन्यो गृह्येते। तावु॒भौ स॒ह म॑हाव्र॒ते गृ॑ह्येते। य॒ज्ञस्यै॒वान्तं॑ ग॒त्वा। उ॒भयोर्लो॒कयो॒ प्रति॑तिष्ठन्ति। अ॒र्क्य॑मु॒क्थं भ॑वति। अ॒न्नाद्य॒स्याव॑रुध्यै॥३६॥\anuvakamend[स॒मा॒यच्छ॑त्यतिग्रा॒ह्या॑ गृह्यन्ते गृ॒ह्यते॑ संवत्स॒रस्या॒न्योन्यो गृह्येते॒ पञ्च॑ च]

%1.2.4.1
ए॒क॒वि॒श ए॒ष भ॑वति। ए॒तेन॒ वै दे॒वा ए॑कवि॒शेन॑। आ॒दि॒त्यमि॒त उ॑त्त॒म सु॑व॒र्गं लो॒कमारो॑हयन्। स वा ए॒ष इ॒त ए॑कवि॒शः। तस्य॒ दशा॒वस्ता॒दहा॑नि। दश॑ प॒रस्तात्। स वा ए॒ष वि॒राज्यु॑भ॒यत॒ प्रति॑ष्ठितः। वि॒राजि॒ हि वा ए॒ष उ॑भ॒यत॒ प्रति॑ष्ठितः। तस्मा॑दन्त॒रेमौ लो॒कौ यन्। सर्वे॑षु सुव॒र्गेषु॑ लो॒केष्व॑भि॒तप॑न्नेति॥३७॥

%1.2.4.2
दे॒वा वा आ॑दि॒त्यस्य॑ सुव॒र्गस्य॑ लो॒कस्य॑। परा॑चोऽतिपा॒दाद॑बिभयुः। तं छन्दो॑भिरदृह॒न्धृत्यै। दे॒वा वा आ॑दि॒त्यस्य॑ सुव॒र्गस्य॑ लो॒कस्य॑। अवा॑चोऽवपा॒दाद॑बिभयुः। तं प॒ञ्चभी॑ र॒श्मिभि॒रुद॑वयन्। तस्मा॑देकवि॒शेऽह॒न्पञ्च॑ दिवाकी॒र्त्या॑नि क्रियन्ते। र॒श्मयो॒ वै दि॑वाकी॒र्त्या॑नि। ये गा॑य॒त्रे। ते गा॑य॒त्रीषूत्त॑रयो॒ पव॑मानयोः॥३८॥

%1.2.4.3
म॒हादि॑वाकीर्त्य॒ होतु॑ पृ॒ष्ठम्। वि॒क॒र्णं ब्र॑ह्मसा॒मम्। भा॒सोऽग्निष्टो॒मः। अथै॒तानि॒ परा॑णि। परै॒र्वै दे॒वा आ॑दि॒त्य सु॑व॒र्गं लो॒कम॑पारयन्। यदपा॑रयन्। तत्परा॑णां पर॒त्वम्। पा॒रय॑न्त्येनं॒ परा॑णि। य ए॒वं वेद॑। अथै॒तानि॒ स्परा॑णि। स्परै॒र्वै दे॒वा आ॑दि॒त्य सु॑व॒र्गं लो॒कम॑स्पारयन्। यदस्पा॑रयन्। तत्स्परा॑णा स्पर॒त्वम्। स्पा॒रय॑न्त्यैन॒ स्परा॑णि। य ए॒वं वेद॑॥३९॥\anuvakamend[ए॒ति॒ पव॑मानयो॒ स्परा॑णि॒ पञ्च॑ च]

%1.2.5.1
अप्र॑तिष्ठां॒ वा ए॒ते ग॑च्छन्ति। येषा संवत्स॒रेऽना॒प्तेऽथ॑। ए॒का॒द॒शिन्या॒प्यते। वै॒ष्ण॒वं वा॑म॒नमा ल॑भन्ते। य॒ज्ञो वै विष्णु॑। य॒ज्ञमे॒वाल॑भन्ते॒ प्रति॑ष्ठित्यै। ऐ॒न्द्रा॒ग्नमाल॑भन्ते। इ॒न्द्रा॒ग्नी वै दे॒वाना॒मया॑तयामानौ। ये ए॒व दे॒वते॒ अया॑तयाम्नी। ते ए॒वाल॑भन्ते॥४०॥

%1.2.5.2
वै॒श्व॒दे॒वमाल॑भन्ते। दे॒वता॑ ए॒वाव॑रुन्धते। द्या॒वा॒पृ॒थिव्यां धे॒नुमाल॑भन्ते। द्यावा॑पृथि॒व्योरे॒व प्रति॑ तिष्ठन्ति। वा॒य॒व्यं॑ व॒त्समाल॑भन्ते। वा॒युरे॒वैभ्यो॑ यथाऽऽयत॒नाद्दे॒वता॒ अव॑रुन्धे। आ॒दि॒त्यामविं॑ व॒शामाल॑भन्ते। इ॒यं वा अदि॑तिः। अ॒स्यामे॒व प्रति॑ तिष्ठन्ति। मै॒त्रा॒व॒रु॒णीमाल॑भन्ते॥४१॥

%1.2.5.3
मि॒त्रेणै॒व य॒ज्ञस्य॒ स्वि॑ष्ट शमयन्ति। वरु॑णेन॒ दुरि॑ष्टम्। प्रा॒जा॒प॒त्यं तू॑प॒रं म॑हाव्र॒त आल॑भन्ते। प्रा॒जा॒प॒त्यो॑ऽतिग्रा॒ह्यो॑ गृह्यते। अह॑रे॒व रू॒पेण॒ सम॑र्धयन्ति। अथो॒ अह्न॑ ए॒वैष ब॒लिर्ह्रि॑यते। आ॒ग्ने॒यमा ल॑भन्ते॒ प्रति॒ प्रज्ञात्यै। अ॒ज॒पे॒त्वान् वा ए॒ते पूर्वै॒र्मासै॒रव॑ रुन्धते। यदे॒ते ग॒व्याः प॒शव॑ आल॒भ्यन्ते। उ॒भये॑षां पशू॒नामव॑रुद्ध्यै॥४२॥

%1.2.5.4
यदति॑रिक्तामेकाद॒शिनी॑मा॒लभे॑रन्। अप्रि॑यं॒ भ्रातृ॑व्यम॒भ्यति॑\-रिच्येत। यद्द्वौ द्वौ॑ प॒शू स॒मस्ये॑युः। कनी॑य॒ आयु॑ कुर्वीरन्। यदे॒ते ब्राह्म॑णवन्तः प॒शव॑ आल॒भ्यन्ते। नाप्रि॑यं॒ भ्रातृ॑व्यम॒भ्य॑ति॒रिच्य॑ते। न कनी॑य॒ आयु॑ कुर्वते॥४३॥\anuvakamend[ते ए॒वाल॑भन्ते मैत्रावरु॒णीमाल॑भ॒न्तेऽव॑रुद्ध्यै स॒प्त च॑]

%1.2.6.1
प्र॒जाप॑तिः प्र॒जाः सृ॒ष्ट्वा वृ॒त्तो॑ऽशयत्। तं दे॒वा भू॒ताना॒ रसं॒ तेज॑ स॒म्भृत्य॑। तेनै॑नमभिषज्यन्। म॒हान॑वव॒र्तीति॑। तन्म॑हाव्र॒तस्य॑ महाव्रत॒त्वम्। म॒हद्व्र॒तमिति॑। तन्म॑हाव्र॒तस्य॑ महाव्रत॒त्वम्। म॒ह॒तो व्र॒तमिति॑। तन्म॑हाव्र॒तस्य॑ महाव्रत॒त्वम्। प॒ञ्च॒वि॒शः स्तोमो॑ भवति॥४४॥

%1.2.6.2
चतु॑र्विशत्यर्धमासः संवत्स॒रः। यद्वा ए॒तस्मिन्त्संवत्स॒रेऽधि॒ प्राजा॑यत। तदन्नं॑ पञ्चवि॒शम॑भवत्। म॒ध्य॒तः क्रि॑यते। म॒ध्य॒तो ह्यन्न॑मशि॒तं धि॒नोति॑। अथो॑ मध्य॒त ए॒व प्र॒जाना॒मूर्ग्धी॑यते। अथ॒ यद्वा इ॒दम॑न्त॒तः क्रि॒यते। तस्मा॑दुद॒न्ते प्र॒जाः समे॑धन्ते। अ॒न्त॒तः क्रि॑यते प्र॒जन॑नायै॒व। त्रि॒वृच्छिरो॑ भवति॥४५॥

%1.2.6.3
त्रे॒धा॒वि॒हि॒त हि शिर॑। लोम॑ छ॒वीरस्थि॑। परा॑चा स्तुवन्ति। तस्मा॒त्तत्स॒दृगे॒व। न मेद्य॒तोऽनु॑ मेद्यति। न कृश्य॒तोऽनु॑ कृश्यति। प॒ञ्च॒द॒शोऽन्यः प॒क्षो भ॑वति। स॒प्त॒द॒शोऽन्यः। तस्मा॒द्वयास्यन्यत॒रम॒र्धम॒भि प॒र्याव॑र्तन्ते। अ॒न्य॒त॒रतो॒ हि तद्गरी॑यः क्रि॒यते॥४६॥

%1.2.6.4
प॒ञ्च॒वि॒श आ॒त्मा भ॑वति। तस्मान्मध्य॒तः प॒शवो॒ वरि॑ष्ठाः। ए॒क॒वि॒शं पुच्छम्। द्वि॒पदा॑सु स्तुवन्ति॒ प्रति॑ष्ठित्यै। सर्वे॑ण स॒ह स्तु॑वन्ति। सर्वे॑ण॒ ह्यात्मनाऽऽत्म॒न्वी। स॒होत्पत॑न्ति। एकै॑का॒मुच्छिषन्ति। आ॒त्मन्न् ह्यङ्गा॑नि ब॒द्धानि॑। न वा ए॒तेन॒ सर्व॒ पुरु॑षः॥४७॥

%1.2.6.5
यदि॒तइ॑तो॒ लोमा॑नि द॒तो न॒खान्। प॒रि॒माद॑ क्रियन्ते। तान्ये॒व तेन॒ प्रत्यु॑प्यन्ते। औदु॑म्बर॒स्तल्पो॑ भवति। ऊर्ग्वा अन्न॑मुदु॒म्बर॑। ऊ॒र्ज ए॒वान्नाद्य॒स्याव॑रुध्यै। यस्य॑ तल्प॒सद्य॒मन॑भिजित॒ स्यात्। स दे॒वाना॒ साम्य॑क्षे। त॒ल्प॒सद्य॑म॒भिज॑या॒नीति॒ तल्प॑मा॒रुह्योद्गा॑येत्। त॒ल्प॒सद्य॑मे॒वाभि ज॑यति॥४८॥

%1.2.6.6
यस्य॑ तल्प॒सद्य॑म॒भिजि॑त॒ स्यात्। स दे॒वाना॒ साम्य॑क्षे। त॒ल्प॒सद्यं॒ मा परा॑जे॒षीति॒ तल्प॑मा॒रुह्योद्गा॑येत्। न त॑ल्प॒सद्यं॒ परा॑जयते। प्ले॒ङ्खे शसति। महो॒ वै प्ले॒ङ्खः। मह॑स ए॒वान्नाद्य॒स्याव॑रुद्ध्यै। दे॒वा॒सु॒राः संय॑त्ता आसन्। त आ॑दि॒त्ये व्याय॑च्छन्त। तं दे॒वाः सम॑जयन्॥४९॥

%1.2.6.7
ब्रा॒ह्म॒णश्च॑ शू॒द्रश्च॑ चर्मक॒र्ते व्याय॑च्छेते। दैव्यो॒ वै वर्णो ब्राह्म॒णः। अ॒सु॒र्य॑ शू॒द्रः। इ॒मे॑ऽरात्सुरि॒मे सु॑भू॒तम॑क्र॒न्नित्य॑न्यत॒रो ब्रू॑यात्। इ॒म उ॑द्वासीका॒रिण॑ इ॒मे दु॑र्भू॒तम॑क्र॒न्नित्य॑न्यत॒रः। यदे॒वैषा सुकृ॒तं या राद्धि॑। तद॑न्यत॒रो॑ऽभि श्री॑णाति। यदे॒वैषां दुष्कृ॒तं याऽराद्धिः। तद॑न्यत॒रोऽप॑ हन्ति। ब्रा॒ह्म॒णः सं ज॑यति। अ॒मुमे॒वादि॒त्यं भ्रातृ॑व्यस्य॒ संवि॑न्दन्ते॥५०॥\anuvakamend[भ॒व॒ति॒ भ॒व॒ति॒ क्रि॒यते॒ पुरु॑षो जयत्यजयञ्जय॒त्येकं॑ च]




\prashnaend{उ॒द्ध॒न्यमा॑नं॒ नवै॒तानि॒ सन्त॑तिरेकवि॒श ए॒षोऽप्र॑तिष्ठां प्र॒जाप॑तिर्वृ॒त्तष्षट्॥६॥}{उ॒द्ध॒न्यमा॑न शो॒चिष्के॒शोऽग्ने॑ स॒पत्ना॑नतिग्रा॒ह्या॑ वैश्वदे॒वमाल॑भन्ते पञ्चा॒शत्॥५०॥}{उद्ध॒न्यमा॑न॒ संवि॑न्दन्ते॥}{हरि॑ ओम्॥}{इति श्रीकृष्णयजुर्वेदीयतैत्तिरीयब्राह्मणे प्रथमाष्टके द्वितीयः प्रपाठकः समाप्तः॥}
\clearpage
\sect{तृतीयः प्रश्नः}
\setcounter{anuvakam}{0}
\dnsub{तैत्तिरीयब्राह्मणे प्रथमाष्टके तृतीयः प्रपाठकः}

%1.3.1.1
दे॒वा॒सु॒राः संय॑त्ता आसन्। ते दे॒वा वि॑ज॒यमु॑प॒यन्त॑। अ॒ग्नीषोम॑योस्तेज॒स्विनीस्त॒नूः संन्य॑दधत। इ॒दमु॑ नो भविष्यति। यदि॑ नो जे॒ष्यन्तीति॑। तेना॒ग्नीषोमा॒वपाक्रामताम्। ते दे॒वा वि॒जित्य॑। अ॒ग्नीषोमा॒वन्वैच्छन्। तेऽग्निमन्व॑\-विन्दन्नृ॒तुषूत्स॑न्नम्। तस्य॒ विभ॑क्तीभिस्तेज॒स्विनीस्त॒नू\-रवा॑रुन्धत॥१॥

%1.3.1.2
ते सोम॒मन्व॑विन्दन्। तम॑घ्नन्। तस्य॑ यथाऽभि॒ज्ञायं॑ त॒नूर्व्य॑गृह्णत। ते ग्रहा॑ अभवन्। तद्ग्रहा॑णां ग्रह॒त्वम्। यस्यै॒वं वि॒दुषो॒ ग्रहा॑ गृ॒ह्यन्ते। तस्य॒ त्वे॑व गृ॑ही॒ताः। नानाऽऽग्नेयं पुनरा॒धेये॑ कुर्यात्। यदनाग्नेयं पुनरा॒धेये॑ कु॒र्यात्। व्यृ॑द्धमे॒व तत्॥२॥

%1.3.1.3
अनाग्नेयं॒ वा ए॒तत्क्रि॑यते। यत्स॒मिध॒स्तनू॒नपा॑तमि॒डो ब॒र्\mbox{}हिर्य॑जति। उ॒भावाग्ने॒यावाज्य॑भागौ स्याताम्। अनाज्यभागौ भवत॒ इत्या॑हुः। यदु॒भावाग्ने॒याव॒न्वञ्चा॒विति॑। अ॒ग्नये॒ पव॑माना॒योत्त॑रः स्यात्। यत्पव॑मानाय। तेनाज्य॑भागः। तेन॑ सौ॒म्यः। बुध॑न्वत्याग्ने॒यस्याज्य॑भागस्य पुरोऽनुवा॒क्या॑ भवति॥३॥

%1.3.1.4
यथा॑ सु॒प्तं बो॒धय॑ति। ता॒दृगे॒व तत्। अ॒ग्निन्य॑क्ताः पत्नीसंया॒जाना॒मृच॑ स्युः। तेनाग्ने॒य सर्वं॑ भवति। ए॒क॒धा॒ ते॑ज॒स्विनीं दे॒वता॒मुपै॒तीत्या॑हुः। सैन॑मीश्व॒रा प्र॒दह॒ इति॑। नेति॑ ब्रूयात्। प्र॒जन॑नं॒ वा अ॒ग्निः। प्र॒जन॑नमे॒वोपै॒तीति॑। कृ॒तय॑जु॒ सम्भृ॑तसम्भार॒ इत्या॑हुः॥४॥

%1.3.1.5
न स॒म्भृत्या सम्भा॒राः। न यजु॑ का॒र्य॑मिति॑। अथो॒ खलु॑। स॒म्भृत्या॑ ए॒व सं॑भा॒राः। का॒र्यं॑ यजु॑। पु॒न॒रा॒धेय॑स्य॒ समृ॑द्ध्यै। तेनो॑पा॒शु प्रच॑रति। एष्य॑ इव॒ वा ए॒षः। यत्पु॑नरा॒धेय॑। यथो॑पा॒शु न॒ष्टमि॒च्छति॑॥५॥

%1.3.1.6
ता॒दृगे॒व तत्। उ॒च्चैः स्वि॑ष्ट॒कृत॒मुत्सृ॑जति। यथा॑ न॒ष्टं वि॒त्त्वा प्राहा॒यमिति॑। ता॒दृगे॒व तत्। ए॒क॒धा ते॑ज॒स्विनीं दे॒वता॒मुपै॒तीत्या॑हुः। सैन॑मीश्व॒रा प्र॒दह॒ इति॑। तत्तथा॒ नोपै॑ति। प्र॒या॒जा॒नू॒या॒जेष्वे॒व विभ॑क्तीः कुर्यात्। य॒था॒पू॒र्वमाज्य॑भागौ॒ स्याताम्। ए॒वं प॑त्नीसंया॒जाः॥६॥

%1.3.1.7
तद्वैश्वान॒रव॑त्प्र॒जन॑नवत्तर॒मुपै॒तीति॑। तदा॑हुः। व्यृ॑द्धं॒ वा ए॒तत्। अनाग्नेयं॒ वा ए॒तत्क्रि॑यत॒ इति॑। नेति॑ ब्रूयात्। अ॒ग्निं प्र॑थ॒मं विभ॑क्तीनां यजति। अ॒ग्निमु॑त्त॒मं प॑त्नीसंया॒जानाम्। तेनाग्ने॒यम्। तेन॒ समृ॑द्धं क्रियत॒ इति॑॥७॥\anuvakamend[अ॒रु॒न्ध॒तै॒व तद्भ॑वति॒ सम्भृ॑तसम्भार॒ इत्या॑हुरि॒च्छति॑ पत्नीसंया॒जा नव॑ च]

%1.3.2.1
दे॒वा वै यथा॒दर्\mbox{}शं॑ य॒ज्ञानाह॑रन्त। योऽग्निष्टो॒मम्। य उ॒क्थ्यम्। यो॑ऽतिरा॒त्रम्। ते स॒हैव सर्वे॑ वाज॒पेय॑मपश्यन्। ते। अ॒न्योऽन्यस्मै॒ नाति॑ष्ठन्त। अ॒हम॒नेन॑ यजा॒ इति॑। तेऽब्रुवन्। आ॒जिम॒स्य धा॑वा॒मेति॑॥८॥

%1.3.2.2
तस्मि॑न्ना॒जिम॑धावन्। तं बृह॒स्पति॒रुद॑जयत्। तेना॑यजत। स स्वाराज्यमगच्छत्। तमिन्द्रोऽब्रवीत्। माम॒नेन॑ याज॒येति॑। तेनेन्द्र॑मयाजयत्। सोऽग्रं॑ दे॒वता॑नां॒ पर्यैत्। अग॑च्छ॒त्स्वाराज्यम्। अति॑ष्ठन्तास्मै॒ ज्यैष्ठ्या॑य॥९॥

%1.3.2.3
य ए॒वं वि॒द्वान् वा॑ज॒पेये॑न॒ यज॑ते। गच्छ॑ति॒ स्वाराज्यम्। अग्र समा॒नानां॒ पर्ये॑ति। तिष्ठ॑न्तेऽस्मै॒ ज्यैष्ठ्या॑य। स वा ए॒ष ब्राह्म॒णस्य॑ चै॒व रा॑ज॒न्य॑स्य च य॒ज्ञः। तं वा ए॒तं वा॑ज॒पेय॒ इत्या॑हुः। वा॒जाप्यो॒ वा ए॒षः। वाज॒ ह्ये॑तेन॑ दे॒वा ऐप्स\sn{}। सोमो॒ वै वा॑ज॒पेय॑। यो वै सोमं॑ वाज॒पेयं॒ वेद॑॥१०॥

%1.3.2.4
वा॒ज्ये॑वैनं॑ पी॒त्वा भ॑वति। आऽस्य॑ वा॒जी जा॑यते। अन्नं॒ वै वा॑ज॒पेय॑। य ए॒वं वेद॑। अत्यन्नम्। आऽस्यान्ना॒दो जा॑यते। ब्रह्म॒ वै वा॑ज॒पेय॑। य ए॒वं वेद॑। अत्ति॒ ब्रह्म॒णाऽन्नम्। आऽस्य॑ ब्र॒ह्मा जा॑यते ॥११॥

%1.3.2.5
वाग्वै वाज॑स्य प्रस॒वः। य ए॒वं वेद॑। क॒रोति॑ वा॒चा वी॒र्यम्। ऐनं॑ वा॒चा ग॑च्छति। अपि॑वतीं॒ वाचं॑ वदति। प्र॒जाप॑तिर्दे॒वेभ्यो॑ य॒ज्ञान्व्यादि॑शत्। स आ॒त्मन्वा॑ज॒पेय॑मधत्त। तं दे॒वा अ॑ब्रुवन्। ए॒ष वाव य॒ज्ञः। यद्वा॑ज॒पेय॑॥१२॥

%1.3.2.6
अप्ये॒व नोऽत्रा॒स्त्विति॑। तेभ्य॑ ए॒ता उज्जि॑ती॒ प्राय॑च्छत्। ता वा ए॒ता उज्जि॑तयो॒ व्याख्या॑यन्ते। य॒ज्ञस्य॑ सर्व॒त्वाय॑। दे॒वता॑ना॒मनि॑र्भागाय। दे॒वा वै ब्रह्म॑ण॒श्चान्न॑स्य च॒ शम॑ल॒मपाघ्नन्। यद्ब्रह्म॑ण॒ शम॑ल॒मासीत्। सा गाथा॑ नाराश॒स्य॑भवत्। यदन्न॑स्य। सा सुरा॥१३॥

%1.3.2.7
तस्मा॒द्गाय॑तश्च म॒त्तस्य॑ च॒ न प्र॑ति॒गृह्यम्। यत्प्र॑तिगृह्णी॒यात्। शम॑लं॒ प्रति॑गृह्णीयात्। सर्वा॒ वा ए॒तस्य॒ वाचोऽव॑रुद्धाः। यो वा॑जपेयया॒जी। या पृ॑थि॒व्यां याऽग्नौ या र॑थन्त॒रे। याऽन्तरि॑क्षे॒ या वा॒यौ या वा॑मदे॒व्ये। या दि॒वि याऽऽदि॒त्ये या बृ॑ह॒ति। याऽप्सु यौष॑धीषु॒ या वन॒स्पति॑षु। तस्माद्वाजपेयया॒ज्यार्त्वि॑जीनः। सर्वा॒ ह्य॑स्य॒ वाचोऽव॑रुद्धाः॥१४॥\anuvakamend[धा॒वा॒मेति॒ ज्यैष्ठ्या॑य॒ वेद॑ ब्र॒ह्मा जा॑यते वाज॒पेय॒ सुराऽऽर्त्वि॑जीन॒ एकं॑ च]

%1.3.3.1
दे॒वा वै यद॒न्यैर्ग्रहैर्य॒ज्ञस्य॒ नावारु॑न्धत। तद॑तिग्रा॒ह्यै॑रति॒\-गृह्या\-वा॑रुन्धत। तद॑तिग्र॒ह्या॑णामतिग्राह्य॒त्वम्। यद॑तिग्रा॒ह्या॑ गृ॒ह्यन्ते। यदे॒वान्यैर्ग्रहैर्य॒ज्ञस्य॒ नाव॑रु॒न्धे। तदे॒व तैर॑ति॒गृह्या\-व॑रुन्धे। पञ्च॑ गृह्यन्ते। पाङ्क्तो॑ य॒ज्ञः। यावा॑ने॒व य॒ज्ञः। तमा॒प्त्वाऽव॑रुन्धे॥१५॥

%1.3.3.2
सर्व॑ ऐ॒न्द्रा भ॑वन्ति। ए॒क॒धैव यज॑मान इन्द्रि॒यं द॑धति। स॒प्तद॑श प्राजाप॒त्या ग्रहा॑ गृह्यन्ते। स॒प्त॒द॒शः प्र॒जाप॑तिः। प्र॒जाप॑ते॒राप्त्यै। एक॑य॒र्चा गृ॑ह्णाति। ए॒क॒धैव यज॑माने वी॒र्यं॑ दधाति। सो॒म॒ग्र॒हाश्च॑ सुराग्र॒हाश्च॑ गृह्णाति। ए॒तद्वै दे॒वानां पर॒ममन्नम्। यत्सोम॑॥१६॥

%1.3.3.3
ए॒तन्म॑नु॒ष्या॑णाम्। यत्सुरा। प॒र॒मेणै॒वास्मा॑ अ॒न्नाद्ये॒नाव॑र\-म॒न्नाद्य॒मव॑रुन्धे। सो॒म॒ग्र॒हान्गृ॑ह्णाति। ब्रह्म॑णो॒ वा ए॒तत्तेज॑। यत्सोम॑। ब्रह्म॑ण ए॒व तेज॑सा॒ तेजो॒ यज॑माने दधाति। सु॒रा॒ग्र॒हान्गृ॑ह्णाति। अन्न॑स्य॒ वा ए॒तच्छम॑लम्। यत्सुरा॥१७॥

%1.3.3.4
अन्न॑स्यै॒व शम॑लेन॒ शम॑लं॒ यज॑माना॒दप॑हन्ति। सो॒म॒ग्र॒हाश्च॑ सुराग्र॒हाश्च॑ गृह्णाति। पुमा॒न्॒ वै सोम॑। स्त्री सुरा। तन्मि॑थु॒नम्। मि॒थु॒नमे॒वास्य॒ तद्य॒ज्ञे क॑रोति प्र॒जन॑नाय। आ॒त्मान॑मे॒व सो॑मग्र॒हैः स्पृ॑णोति। जा॒या सु॑राग्र॒हैः। तस्माद्वाजपेयया॒ज्य॑मुष्मि॑ल्लोँ॒के स्त्रिय॒ सम्भ॑वति। वा॒ज॒पेया॑भिजित॒ ह्य॑स्य॥१८॥

%1.3.3.5
पूर्वे॑ सोमग्र॒हा गृ॑ह्यन्ते। अप॑रे सुराग्र॒हाः। पु॒रो॒ऽक्ष सो॑मग्र॒हान्त्सा॑दयति। प॒श्चा॒द॒क्ष सु॑राग्र॒हान्। पा॒प॒व॒स्य॒सस्य॒ विधृ॑त्यै। ए॒ष वै यज॑मानः। यत्सोम॑। अन्न॒ सुरा। सो॒म॒ग्र॒हाश्च॑ सुराग्र॒हाश्च॒ व्यति॑षजति। अ॒न्नाद्ये॑नै॒वैनं॒ व्यति॑षजति॥१९॥

%1.3.3.6
सं॒पृच॑ स्थ॒ सं मा॑ भ॒द्रेण॑ पृ॒ङ्क्तेत्या॑ह। अन्नं॒ वै भ॒द्रम्। अ॒न्नाद्ये॑नै॒वैन॒ ससृ॑जति। अन्न॑स्य॒ वा ए॒तच्छम॑लम्। यत्सुरा। पा॒प्मेव॒ खलु॒ वै शम॑लम्। पा॒प्मना॒ वा ए॑नमे॒तच्छम॑लेन॒ व्यति॑षजति। यत्सो॑मग्र॒हाश्च॑ सुराग्र॒हाश्च॑ व्यति॒षज॑ति। वि॒पृच॑ स्थ॒ वि मा॑ पा॒प्मना॑ पृ॒ङ्क्तेत्या॑ह। पा॒प्मनै॒वैन॒ शम॑लेन॒ व्याव॑र्तयति॥२०॥

%1.3.3.7
तस्माद्वाजपेयया॒जी पू॒तो मेध्यो॑ दक्षि॒ण्य॑। प्राङुद्द्र॑वति सोमग्र॒हैः। अ॒मुमे॒व तैर्लो॒कम॒भिज॑यति। प्र॒त्यङ्ख्सु॑राग्र॒हैः। इ॒ममे॒व तैर्लो॒कम॒भिज॑यति। प्रति॑ष्ठन्ति सोमग्र॒हैः। याव॑दे॒व स॒त्यम्। तेन॑ सूयते। वा॒ज॒सृद्भ्य॑ सुराग्र॒हान् ह॑रन्ति। अनृ॑तेनै॒व विश॒ ससृ॑जति। हि॒र॒ण्य॒पा॒त्रं मधो पू॒र्णं द॑दाति। म॒ध॒व्यो॑ऽसा॒नीति॑। ए॒क॒धा ब्र॒ह्मण॒ उप॑ हरति। ए॒क॒धैव यज॑मान॒ आयु॒स्तेजो॑ दधाति॥२१॥\anuvakamend[आ॒प्त्वाऽव॑रुन्धे॒ सोम॒ शम॑लं॒ यत्सुरा॒ ह्य॑स्यैनं॒ व्यति॑षजति॒ व्याव॑र्तयति सृजति च॒त्वारि॑ च]

%1.3.4.1
ब्र॒ह्म॒वा॒दिनो॑ वदन्ति। नाग्नि॑ष्टो॒मो नोक्थ्य॑। न षो॑ड॒शी नाति॑रा॒त्रः। अथ॒ कस्माद्वाज॒पेये॒ सर्वे॑ यज्ञक्र॒तवोऽव॑रुध्यन्त॒ इति॑। प॒शुभि॒रिति॑ ब्रूयात्। आ॒ग्ने॒यं प॒शुमाल॑भते। अ॒ग्नि॒ष्टो॒ममे॒व तेनाव॑रुन्धे। ऐ॒न्द्रा॒ग्नेनो॒क्थ्यम्। ऐ॒न्द्रेण॑ षोड॒शिन॑ स्तो॒त्रम्। सा॒र॒स्व॒त्याऽति॑रा॒त्रम्॥२२॥

%1.3.4.2
मा॒रु॒त्या बृ॑ह॒तः स्तो॒त्रम्। ए॒ताव॑न्तो॒ वै य॑ज्ञक्र॒तव॑। तान्प॒शुभि॑रे॒वाव॑रुन्धे। आ॒त्मान॑मे॒व स्पृ॑णोत्यग्निष्टो॒मेन॑। प्रा॒णा॒पा॒नावु॒क्थ्ये॑न। वी॒र्य षोड॒शिन॑ स्तो॒त्रेण॑। वाच॑मतिरा॒त्रेण॑। प्र॒जां बृ॑ह॒तः स्तो॒त्रेण॑। इ॒ममे॒व लो॒कम॒भिज॑यत्यग्निष्टो॒मेन॑। अ॒न्तरि॑क्षमु॒क्थ्ये॑न॥२३॥

%1.3.4.3
सु॒व॒र्गं लो॒क षो॑ड॒शिन॑ स्तो॒त्रेण॑। दे॒व॒याना॑ने॒व प॒थ आरो॑हत्यतिरा॒त्रेण॑। नाक रोहति बृह॒तः स्तो॒त्रेण॑। तेज॑ ए॒वात्मन्ध॑त्त आग्ने॒येन॑ प॒शुना। ओजो॒ बल॑मैन्द्रा॒ग्नेन॑। इ॒न्द्रि॒यमै॒न्द्रेण॑। वाच सारस्व॒त्या। उ॒भावे॒व दे॑वलो॒कं च॑ मनुष्यलो॒कं चा॒भिज॑यति मारु॒त्या व॒शया। स॒प्तद॑श प्राजाप॒त्यान्प॒शूनाल॑भते। स॒प्त॒द॒शः प्र॒जाप॑तिः॥२४॥

%1.3.4.4
प्र॒जाप॑ते॒राप्त्यै। श्या॒मा एक॑रूपा भवन्ति। ए॒वमि॑व॒ हि प्र॒जाप॑ति॒ समृ॑द्ध्यै। तान्पर्य॑ग्निकृता॒नुत्सृ॑जति। म॒रुतो॑ य॒ज्ञम॑जिघासन्प्र॒जाप॑तेः। तेभ्य॑ ए॒तां मा॑रु॒तीं व॒शामाल॑भत। तयै॒वैना॑नशमयत्। मा॒रु॒त्या प्र॒चर्य॑। ए॒तान्त्संज्ञ॑पयेत्। म॒रुत॑ ए॒व श॑मयि॒त्वा॥२५॥

%1.3.4.5
ए॒तैः प्रच॑रति। य॒ज्ञस्याघा॑ताय। ए॒क॒धा व॒पा जु॑होति। ए॒क॒दे॒व॒त्या॑ हि। ए॒ते। अथो॑ एक॒धैव यज॑माने वी॒र्यं॑ दधाति। नै॒वा॒रेण॑ स॒प्तद॑शशरावेणै॒तर्\mbox{}हि॒ प्रच॑रति। ए॒तत्पु॑रोडाशा॒ ह्ये॑ते। अथो॑ पशू॒नामे॒व छि॒द्रमपि॑दधाति। सा॒र॒स्व॒त्योत्त॒मया॒ प्रच॑रति। वाग्वै सर॑स्वती। तस्मात्प्रा॒णानां॒ वागु॑त्त॒मा। अथो प्र॒जाप॑तावे॒व य॒ज्ञं प्रति॑ष्ठापयति। प्र॒जाप॑ति॒र्‌हि वाक्। अप॑न्नदती भवति। तस्मान्मनु॒ष्या सर्वां॒ वाचं॑ वदन्ति॥२६॥\anuvakamend[अ॒ति॒रा॒त्रम॒न्तरि॑क्षमु॒क्थ्ये॑न प्र॒जाप॑तिः शमयि॒त्वोत्त॒मया॒ प्रच॑रति॒ षट् च॑]

%1.3.5.1
सा॒वि॒त्रं जु॑होति॒ कर्म॑णः कर्मणः पु॒रस्तात्। कस्तद्वे॒देत्या॑हुः। यद्वा॑ज॒पेय॑स्य॒ पूर्वं॒ यदप॑र॒मिति॑। स॒वि॒तृप्र॑सूत ए॒व य॑थापू॒र्वं कर्मा॑णि करोति। सव॑नेसवने जुहोति। आ॒क्रम॑णमे॒व तत्सेतुं॒ यज॑मानः कुरुते। सु॒व॒र्गस्य॑ लो॒कस्य॒ सम॑ष्ट्यै। वा॒चस्पति॒र्वाच॑म॒द्य स्व॑दाति न॒ इत्या॑ह। वाग्वै दे॒वानां पु॒राऽन्न॑मासीत्। वाच॑मे॒वास्मा॒ अन्न स्वदयति॥२७॥

%1.3.5.2
इन्द्र॑स्य॒ वज्रो॑ऽसि॒ वार्त्र॑घ्न॒ इति॒ रथ॑मु॒पाव॑हरति॒ विजि॑त्यै। वाज॑स्य॒ नु प्र॑स॒वे मा॒तरं॑ म॒हीमित्या॑ह। यच्चै॒वेयम्। यच्चा॒स्यामधि॑। तदे॒वाव॑रुन्धे। अथो॒ तस्मि॑न्ने॒वोभये॒ऽभिषि॑च्यते। अ॒प्स्व॑न्तर॒मृत॑म॒प्सु भे॑ष॒जमित्यश्वान्पल्पूलयति। अ॒प्सु वा अश्व॑स्य॒ तृती॑यं॒ प्रवि॑ष्टम्। तद॑नु॒वेन॒न्वव॑प्लवते। यद॒प्सु प॑ल्पू॒लय॑ति॥२८॥

%1.3.5.3
यदे॒वास्या॒प्सु प्रवि॑ष्टम्। तदे॒वाव॑रुन्धे। ब॒हु वा अश्वो॑ऽमे॒ध्यमुप॑गच्छति। यद॒प्सु प॑ल्पू॒लय॑ति। मेध्या॑ने॒वै\-नान्करोति। वा॒युर्वा त्वा॒ मनु॑र्वा॒ त्वेत्या॑ह। ए॒ता वा ए॒तं दे॒वता॒ अग्रे॒ अश्व॑मयुञ्जन्। ताभि॑रे॒वैनान्॑ युनक्ति। स॒वस्योज्जि॑त्यै। यजु॑षा युनक्ति॒ व्यावृ॑त्त्यै॥२९॥

%1.3.5.4
अपान्नपादाशुहेम॒न्निति॒ सम्मार्ष्टि। मेध्या॑ने॒वैनान्करोति। अथो॒ स्तौत्ये॒वैना॑ना॒जि स॑रिष्य॒तः। वि॒ष्णु॒क्र॒मान्क्र॑मते। विष्णु॑रे॒व भू॒त्वेमाल्लोँ॒कान॒भिज॑यति। वै॒श्व॒दे॒वो वै रथ॑। अ॒ङ्कौ न्य॒ङ्काव॒भितो॒ रथं॒ यावित्या॑ह। या ए॒व दे॒वता॒ रथे॒ प्रवि॑ष्टाः। ताभ्य॑ ए॒व नम॑स्करोति। आ॒त्मनोऽनार्त्यै। अश॑मरथम्भावुकोऽस्य॒ रथो॑ भवति। य ए॒वं वेद॑॥३०॥\anuvakamend[स्व॒द॒य॒ति॒ प॒ल्पू॒लय॑ति॒ व्यावृ॑त्त्या॒ अनार्त्यै॒ द्वे च॑]

%1.3.6.1
दे॒वस्या॒ह स॑वि॒तुः प्र॑स॒वे बृह॒स्पति॑ना वाज॒जिता॒ वाजं॑ जेष॒मित्या॑ह। स॒वि॒तृप्र॑सूत ए॒व ब्रह्म॑णा॒ वाज॒मुज्ज॑यति। दे॒वस्या॒ह स॑वि॒तुः प्र॑स॒वे बृह॒स्पति॑ना वाज॒जिता॒ वर्\mbox{}षि॑ष्ठं॒ नाक रुहेय॒मित्या॑ह। स॒वि॒तृप्र॑सूत ए॒व ब्रह्म॑णा॒ वर्\mbox{}षि॑ष्ठं॒ नाक रोहति। चात्वा॑ले रथच॒क्रं निमि॑त रोहति। अतो॒ वा अङ्गि॑रस उत्त॒माः सु॑व॒र्गं लो॒कमा॑यन्। सा॒क्षादे॒व यज॑मानः सुव॒र्गं लो॒कमे॑ति। आवेष्टयति। वज्रो॒ वै रथ॑। वज्रे॑णै॒व दिशो॒ऽभिज॑यति॥३१॥

%1.3.6.2
वा॒जिना॒ साम॑ गायते। अन्नं॒ वै वाज॑। अन्न॑मे॒वाव॑रुन्धे। वा॒चो वर्ष्म॑ दे॒वेभ्योऽपाक्रामत्। तद्वन॒स्पती॒न्प्रावि॑शत्। सैषा वाग्वन॒स्पति॑षु वदति। या दु॑न्दु॒भौ। तस्माद्दुन्दु॒भिः सर्वा॒ वाचोऽति॑वदति। दु॒न्दु॒भीन्त्स॒माघ्न॑न्ति। प॒र॒मा वा ए॒षा वाक्॥३२॥

%1.3.6.3
या दु॑न्दु॒भौ। प॒र॒मयै॒व वा॒चाऽव॑रां॒ वाच॒म॑वरुन्धे। अथो॑ वा॒च ए॒व वर्ष्म॒ यज॑मा॒नोऽव॑रुन्धे। इन्द्रा॑य॒ वाचं॑ वद॒तेन्द्रं॒ वाजं॑ जापय॒तेन्द्रो॒ वाज॑मजयि॒दित्या॑ह। ए॒ष वा ए॒तर्\mbox{}हीन्द्र॑। यो यज॑ते। यज॑मान ए॒व वाज॒मुज्ज॑यति। स॒प्तद॑श प्रव्या॒धाना॒जिं धा॑वन्ति। स॒प्त॒द॒श स्तो॒त्रं भ॑वति। स॒प्तद॑शसप्तदश दीयन्ते॥३३॥

%1.3.6.4
स॒प्त॒द॒शः प्र॒जाप॑तिः। प्र॒जा॑पते॒राप्त्यै। अर्वा॑ऽसि॒ सप्ति॑रसि वा॒ज्य॑सीत्या॑ह। अ॒ग्निर्वा अर्वा। वा॒युः सप्ति॑। आ॒दि॒त्यो वा॒जी। ए॒ताभि॑रे॒वास्मै॑ दे॒वता॑भिर्देवर॒थं यु॑नक्ति। प्र॒ष्टि॒वा॒हिनं॑ युनक्ति। प्र॒ष्टि॒वा॒ही वै दे॑वर॒थः। दे॒व॒र॒थमे॒वास्मै॑ युनक्ति॥३४॥

%1.3.6.5
वाजि॑नो॒ वाजं॑ धावत॒ काष्ठां गच्छ॒तेत्या॑ह। सु॒व॒र्गो वै लो॒कः काष्ठा। सु॒व॒र्गमे॒व लो॒कं य॑न्ति। सु॒व॒र्गं वा ए॒ते लो॒कं य॑न्ति। य आ॒जिं धाव॑न्ति। प्राञ्चो॑ धावन्ति। प्राङि॑व॒ हि सु॑व॒र्गो लो॒कः। च॒त॒सृभि॒रनु॑ मन्त्रयते। च॒त्वारि॒ छन्दासि। छन्दो॑भिरे॒वैनान्त्सुव॒र्गं लो॒कं ग॑मयति॥३५॥

%1.3.6.6
प्र वा ए॒तेऽस्माल्लो॒काच्च्य॑वन्ते। य आ॒जिं धाव॑न्ति। उद॑ञ्च॒ आव॑र्तन्ते। अ॒स्मादे॒व तेन॑ लो॒कान्नय॑न्ति। र॒थ॒वि॒मो॒च॒नीयं॑ जुहोति॒ प्रति॑ष्ठित्त्यै। आ मा॒ वाज॑स्य प्रस॒वो ज॑गम्या॒दित्या॑ह। अन्नं॒ वै वाज॑। अन्न॑मे॒वाव॑रुन्धे। य॒था॒लो॒कं वा ए॒त उज्ज॑यन्ति। य आ॒जिं धाव॑न्ति॥३६॥

%1.3.6.7
कृ॒ष्णलं॑कृष्णलं वाज॒सृद्भ्य॒ प्रय॑च्छति। यमे॒व ते वाजं॑ लो॒कमु॒ज्जय॑न्ति। तं प॑रि॒क्रीयाव॑रुन्धे। ए॒क॒धा ब्र॒ह्मण॒ उप॑हरति। ए॒क॒धैव यज॑माने वी॒र्यं॑ दधाति। दे॒वा वा ओष॑धीष्वा॒जिम॑युः। ता बृह॒स्पति॒रुद॑जयत्। स नी॒वारा॒न्निर॑वृणीत। तन्नी॒वारा॑णां नीवार॒त्वम्। नै॒वा॒रश्च॒रुर्भ॑वति॥३७॥

%1.3.6.8
ए॒तद्वै दे॒वानां पर॒ममन्नम्। यन्नी॒वारा। प॒र॒मेणै॒वास्मा॑ अ॒न्नाद्ये॒नाव॑रम॒न्नाद्य॒मव॑रुन्धे। स॒प्तद॑शशरावो भवति। स॒प्त॒द॒शः प्र॒जाप॑तिः। प्र॒जाप॑ते॒राप्त्ये। क्षी॒रे भ॑वति। रुच॑मे॒वास्मि॑न्दधाति। स॒र्पिष्वान्भवति मेध्य॒त्वाय॑। बा॒र्॒ह॒स्प॒त्यो वा ए॒ष दे॒वत॑या॥३८॥

%1.3.6.9
यो वा॑ज॒पेये॑न॒ यज॑ते। बा॒र्॒ह॒स्प॒त्य ए॒ष च॒रुः। अश्वान्त्सरिष्य॒तः स॒स्रुष॒श्चाव॑ घ्रापयति। यमे॒व ते वाजं॑ लो॒कमु॒ज्जय॑न्ति। तमे॒वाव॑रुन्धे। अजी॑जिपत वनस्पतय॒ इन्द्रं॒ वाजं॒ विमु॑च्यध्व॒मिति॑ दुन्दु॒भीन् विमु॑ञ्चति। यमे॒व ते वाजं॑ लो॒कमि॑न्द्रि॒यं दु॑न्दु॒भय॑ उ॒ज्जय॑न्ति। तमे॒वाव॑रुन्धे॥३९॥\anuvakamend[अ॒भिज॑यति॒ वा ए॒षा वाग्दी॑यन्तेऽस्मै युनक्ति गमयति॒ य आ॒जिं धाव॑न्ति भवति दे॒वत॑या॒ऽष्टौ च॑]

%1.3.7.1
ता॒र्प्यं यज॑मानं॒ परि॑धापयति। य॒ज्ञो वै ता॒र्प्यम्। य॒ज्ञेनै॒वैन॒ सम॑र्धयति। द॒र्भ॒मयं॒ परि॑धापयति। प॒वित्रं॒ वै द॒र्भाः। पु॒नात्ये॒वैनम्। वाजं॒ वा ए॒षोऽव॑रुरुत्सते। यो वा॑ज॒पये॑न॒ यज॑ते। ओष॑धय॒ खलु॒ वै वाज॑। यद्द॑र्भ॒मयं॑ परिधा॒पय॑ति॥४०॥

%1.3.7.2
वाज॒स्याव॑रुद्ध्यै। जाय॒ एहि॒ सुवो॒ रोहा॒वेत्या॑ह। पत्नि॑या ए॒वैष य॒ज्ञस्यान्वार॒म्भोऽन॑वच्छित्त्यै। स॒प्तद॑शारत्नि॒र्यूपो॑ भवति। स॒प्त॒द॒शः प्र॒जाप॑तिः। प्र॒जाप॑ते॒राप्त्यै। तू॒प॒रश्चतु॑रश्रिर्भवति। गौ॒धू॒मं च॒षालम्। न वा ए॒ते व्री॒हयो॒ न यवा। यद्गो॒धूमा॥४१॥

%1.3.7.3
ए॒वमि॑व॒ हि प्र॒जाप॑ति॒ समृ॑द्ध्यै। अथो॑ अ॒मुमे॒वास्मै॑ लो॒कमन्न॑वन्तं करोति। वासो॑भिर्वेष्टयति। ए॒ष वै यज॑मानः। यद्यूप॑। स॒र्व॒दे॒व॒त्यं॑ वास॑। सर्वा॑भिरे॒वैनं॑ दे॒वता॑भि॒ सम॑र्धयति। अथो॑ आ॒क्रम॑णमे॒व तत्सेतुं॒ यज॑मानः कुरुते। सु॒व॒र्गस्य॑ लो॒कस्य॒ सम॑ष्ट्यै। द्वाद॑श वाजप्रस॒वीया॑नि जुहोति॥४२॥

%1.3.7.4
द्वाद॑श॒ मासा संवत्स॒रः। सं॒व॒त्स॒रमे॒व प्री॑णाति। अथो॑ संवत्स॒रमे॒वास्मा॒ उप॑दधाति। सु॒व॒र्गस्य॑ लो॒कस्य॒ सम॑ष्ट्यै। द॒शभि॒ कल्पै॑ रोहति। नव॒ वै पुरु॑षे प्रा॒णाः। नाभि॑र्दश॒मी। प्रा॒णाने॒व य॑थास्था॒नं क॑ल्पयि॒त्वा। सु॒व॒र्गं लो॒कमे॑ति। ए॒ताव॒द्वै पुरु॑षस्य॒ स्वम्॥४३॥

%1.3.7.5
याव॑त्प्रा॒णाः। याव॑दे॒वास्यास्ति॑। तेन॑ स॒ह सु॑व॒र्गं लो॒कमे॑ति। सुव॑र्दे॒वा अ॑ग॒न्मेत्या॑ह। सु॒व॒र्गमे॒व लो॒कमे॑ति। अ॒मृता॑ अभू॒मेत्या॑ह। अ॒मृत॑मिव॒ हि सु॑व॒र्गो लो॒कः। प्र॒जाप॑तेः प्र॒जा अ॑भू॒मेत्या॑ह। प्रा॒जा॒प॒त्यो वा अ॒यं लो॒कः। अ॒स्मादे॒व तेन॑ लो॒कान्नैति॑॥४४॥

%1.3.7.6
सम॒हं प्र॒जया॒ सं मया प्र॒जेत्या॑ह। आ॒शिष॑मे॒वैतामा शास्ते। आ॒स॒पु॒टैर्घ्न॑न्ति। अन्नं॒ वा इ॒यम्। अ॒न्नाद्ये॑नै॒वैन॒ सम॑र्धयन्ति। ऊषैर्घ्नन्ति। ए॒ते हि सा॒क्षादन्नम्। यदूषा। सा॒क्षादे॒वैन॑म॒न्नाद्ये॑न॒ सम॑र्धयन्ति। पु॒रस्तात्प्र॒त्यञ्चं घ्नन्ति॥४५॥

%1.3.7.7
पु॒रस्ता॒द्धि प्र॑ती॒चीन॒मन्न॑म॒द्यते। शी॒र्॒ष॒तो घ्न॑न्ति। शी॒र्॒ष॒तो ह्यन्न॑म॒द्यते। दि॒ग्भ्यो घ्न॑न्ति। दि॒ग्भ्य ए॒वास्मा॑ अ॒न्नाद्य॒मव॑रुन्धते। ई॒श्व॒रो वा ए॒ष पराङ्प्र॒दघ॑। यो यूप॒ रोह॑ति। हिर॑ण्यम॒ध्यव॑रोहति। अ॒मृतं॒ वै हिर॑ण्यम्। अ॒मृत सुव॒र्गो लो॒कः॥४६॥

%1.3.7.8
अ॒मृत॑ ए॒व सु॑व॒र्गे लो॒के प्रति॑तिष्ठति। श॒तमा॑नं भवति। श॒तायु॒ पुरु॑षः श॒तेन्द्रि॑यः। आयु॑ष्ये॒वेन्द्रि॒ये प्रति॑ तिष्ठति। पुष्ट्यै॒ वा ए॒तद्रू॒पम्। यद॒जा। त्रिः सं॑वत्स॒रस्या॒न्यान्प॒शून्परि॒ प्रजा॑यते। ब॒स्ता॒जि॒नम॒ध्यव॑ रोहति। पुष्ट्या॑मे॒व प्र॒जन॑ने॒ प्रति॑तिष्ठति॥४७॥\anuvakamend[प॒रि॒धा॒पय॑ति गो॒धूमा॑ जुहोति॒ स्वं नैति॑ प्र॒त्यञ्चं घ्नन्ति लो॒को नव॑ च]

%1.3.8.1
स॒प्तान्न॑हो॒माञ्जु॑होति। स॒प्त वा अन्ना॑नि। याव॑न्त्ये॒वान्ना॑नि। तान्ये॒वाव॑रुन्धे। स॒प्त ग्रा॒म्या ओष॑धयः। स॒प्तार॒ण्याः। उ॒भयी॑षा॒मव॑रुद्ध्यै। अन्न॑स्यान्नस्य जुहोति। अन्न॑स्यान्न॒स्या\-व॑रुद्ध्यै। यद्वा॑जपेयया॒ज्यन॑वरुद्धस्याश्नी॒यात्॥४८॥

%1.3.8.2
अव॑रुद्धेन॒ व्यृ॑द्ध्येत। सर्व॑स्य समव॒दाय॑ जुहोति। अन॑वरुद्ध॒स्याव॑रुद्ध्यै। औदु॑म्बरेण स्रु॒वेण॑ जुहोति। ऊर्ग्वा अन्न॑मुदु॒म्बर॑। ऊ॒र्ज ए॒वान्नाद्य॒स्याव॑रुद्ध्यै। दे॒वस्य॑ त्वा सवि॒तुः प्र॑स॒व इत्या॑ह। स॒वि॒तृप्र॑सूत ए॒वैनं॒ ब्रह्म॑णा दे॒वता॑भिर॒भिषि॑ञ्चति। अन्न॑स्यान्नस्या॒भिषि॑ञ्चति। अन्न॑स्यान्न॒स्याव॑रुद्ध्यै॥४९॥

%1.3.8.3
पु॒रस्तात्प्र॒त्यञ्च॑म॒भिषि॑ञ्चति। पु॒रस्ता॒द्धि प्र॑ती॒चीन॒मन्न॑म॒द्यते। शी॒र्॒ष॒तो॑ऽभिषि॑ञ्चति। शी॒र्॒ष॒तो ह्यन्न॑म॒द्यते। आ मुखा॑द॒न्वव॑स्रावयति। मु॒ख॒त ए॒वास्मा॑ अ॒न्नाद्यं॑ दधाति। अ॒ग्नेस्त्वा॒ साम्राज्येना॒भिषि॑ञ्चा॒मीत्या॑ह। ए॒ष वा अ॒ग्नेः स॒वः। तेनै॒वैन॑म॒भिषि॑ञ्चति। इन्द्र॑स्य त्वा॒ साम्राज्येना॒भिषि॑ञ्चा॒मीत्या॑ह॥५०॥

%1.3.8.4
इ॒न्द्रि॒यमे॒वास्मि॑न्ने॒तेन॑ दधाति। बृह॒स्पतेस्त्वा॒ साम्राज्येना॒भि\-षि॑ञ्चा॒मीत्या॑ह। ब्रह्म॒ वै दे॒वानां॒ बृह॒स्पति॑। ब्रह्म॑णै॒वैन॑म॒भि\-षि॑ञ्चति। सो॒म॒ग्र॒हाश्चा॑वदानी॒यानि॑ च॒र्त्विग्भ्य॒ उप॑हरन्ति। अ॒मुमे॒व तैर्लो॒कमन्न॑वन्तं करोति। सु॒रा॒ग्र॒हाश्चा॑नवदानी॒यानि॑ च वाज॒सृद्भ्य॑। इ॒ममे॒व तैर्लो॒कमन्न॑वन्तं करोति। अथो॑ उ॒भयीष्वे॒वाभिषि॑च्यते। वि॒मा॒थं कु॑र्वते वाज॒सृत॑॥५१॥

%1.3.8.5
इ॒न्द्रि॒यस्याव॑रुद्ध्यै। अनि॑रुक्ताभिः प्रातः सव॒ने स्तु॑वते। अनि॑रुक्तः प्र॒जाप॑तिः। प्र॒जाप॑ते॒राप्त्यै। वाज॑वतीभि॒र्माध्य॑न्दिने। अन्नं॒ वै वाज॑। अन्न॑मे॒वाव॑रुन्धे। शि॒पि॒वि॒ष्टव॑तीभिस्तृतीय\-सव॒ने। य॒ज्ञो वै विष्णु॑। प॒शव॒ शिपि॑। य॒ज्ञ ए॒व प॒शुषु॒ प्रति॑ तिष्ठति। बृ॒हदन्त्यं॑ भवति। अन्त॑मे॒वैन श्रि॒यै ग॑मयति॥५२॥\anuvakamend[अ॒श्नी॒यादन्न॑स्यान्न॒स्याव॑रुद्ध्या॒ इन्द्र॑स्य त्वा॒ साम्राज्येना॒भिषि॑ञ्चा॒मीत्या॑ह वाज॒सृत॒ शिपि॒स्त्रीणि॑ च]

%1.3.9.1
नृ॒षदं॒ त्वेत्या॑ह। प्र॒जा वै नॄन्। प्र॒जाना॑मे॒वैतेन॑ सूयते। द्रु॒षद॒मित्या॑ह। वन॒स्पत॑यो॒ वै द्रु। वन॒स्पती॑नामे॒वैतेन॑ सूयते। भु॒व॒न॒सद॒मित्या॑ह। य॒दा वै वसी॑या॒न्भव॑ति। भुव॑नमग॒न्निति॒ वै तमा॑हुः। भुव॑नमे॒वैतेन॑ गच्छति॥५३॥

%1.3.9.2
अ॒प्सु॒षदं॑ त्वा घृत॒सद॒मित्या॑ह। अ॒पामे॒वैतेन॑ घृ॒तस्य॑ सूयते। व्यो॒म॒सद॒मित्या॑ह। य॒दा वै वसी॑या॒न्भव॑ति। व्यो॑माग॒न्निति॒ वै तमा॑हुः। व्यो॑मै॒वैते॑न गच्छति। पृ॒थि॒वि॒षदं॑ त्वाऽन्तरिक्ष॒सद॒मित्या॑ह। ए॒षामे॒वैतेन॑ लो॒काना सूयते। तस्माद्वाजपेयया॒जी न कंच॒न प्र॒त्यव॑रोहति। अपी॑व॒ हि दे॒वता॑ना सू॒यते॥५४॥

%1.3.9.3
ना॒क॒सद॒मित्या॑ह। य॒दा वै वसी॑या॒न्भव॑ति। नाक॑मग॒न्निति॒ वै तमा॑हुः। नाक॑मे॒वैतेन॑ गच्छति। ये ग्रहा पञ्चज॒नीना॒ इत्या॑ह। प॒ञ्च॒ज॒नाना॑मे॒वैतेन॑ सूयते। अ॒पा रस॒मुद्व॑यस॒मित्या॑ह। अ॒पामे॒वैतेन॒ रस॑स्य सूयते। सूर्य॑रश्मि स॒माभृ॑त॒मित्या॑ह सशुक्र॒त्वाय॑॥५५॥\anuvakamend[ग॒च्छ॒ति॒ सू॒यते॒ नव॑ च]

%1.3.10.1
इन्द्रो॑ वृ॒त्र ह॒त्वा। असु॑रान्परा॒भाव्य॑। सो॑ऽमावा॒स्यां प्रत्याग॑च्छत्। ते पि॒तर॑ पूर्वे॒द्युराग॑च्छन्। पि॒तॄन् य॒ज्ञो॑ऽगच्छत्। तं दे॒वाः पुन॑रयाचन्त। तमेभ्यो॒ न पुन॑रददुः। तेऽब्रुव॒न्वरं॑ वृणामहै। अथ॑ व॒ पुन॑र्दास्यामः। अ॒स्मभ्य॑मे॒व पूर्वे॒द्युः क्रि॑याता॒ इति॑॥५६॥

%1.3.10.2
तमेभ्य॒ पुन॑रददुः। तस्मात्पि॒तृभ्य॑ पूर्वे॒द्युः क्रि॑यते। यत्पि॒तृभ्य॑ पूर्वे॒द्युः क॒रोति॑। पि॒तृभ्य॑ ए॒व तद्य॒ज्ञं नि॒ष्क्रीय॒ यज॑मान॒ प्रत॑नुते। सोमा॑य पि॒तृपी॑ताय स्व॒धा नम॒ इत्या॑ह। पि॒तुरे॒वाधि॑ सोमपी॒थमव॑रुन्धे। न हि पि॒ता प्र॒मीय॑माण॒ आहै॒ष सो॑मपी॒थ इति॑। इ॒न्द्रि॒यं वै सो॑मपी॒थः। इ॒न्द्रि॒यमे॒व सो॑मपी॒थमव॑ रुन्धे। तेनेन्द्रि॒येण॑ द्वि॒तीयां जा॒याम॒भ्य॑श्नुते॥५७॥

%1.3.10.3
ए॒तद्वै ब्राह्म॑णं पु॒रा वा॑जवश्रव॒सा वि॒दाम॑क्रन्। तस्मा॒त्ते द्वेद्वे॑ जा॒ये अ॒भ्याक्षत। य ए॒वं वेद॑। अ॒भि द्वि॒तीयां॑ जा॒याम॑श्नुते। अ॒ग्नये॑ कव्य॒वाह॑नाय स्व॒धा नम॒ इत्या॑ह। य ए॒व पि॑तृ॒णाम॒ग्निः। तं प्री॑णाति। ति॒स्र आहु॑तीर्जुहोति। त्रिर्निद॑धाति। षट्त्संप॑द्यन्ते॥५८॥

%1.3.10.4
षड्वा ऋ॒तव॑। ऋ॒तूने॒व प्री॑णाति। तू॒ष्णीं मेक्ष॑ण॒माद॑धाति। अस्ति॑ वा॒ हि ष॒ष्ठ ऋ॒तुर्न वा। दे॒वान् वै पि॒तॄन्प्री॒तान्। म॒नु॒ष्या पि॒तरोऽनु॒ प्रपि॑पते। ति॒स्र आहु॑तीर्जुहोति। त्रिर्निद॑धाति। षट्त्संप॑द्यन्ते। षड्वा ऋ॒तव॑॥५९॥

%1.3.10.5
ऋ॒तव॒ खलु॒ वै दे॒वाः पि॒तर॑। ऋ॒तूने॒व दे॒वान्पि॒तॄन्प्री॑णाति। तान्प्री॒तान्। म॒नु॒ष्या पि॒तरोऽनु॒ प्रपि॑पते। स॒कृ॒दा॒च्छि॒न्नं ब॒र्\mbox{}हिर्भ॑वति। स॒कृदि॑व॒ हि पि॒तर॑। त्रिर्निद॑धाति। तृ॒तीये॒ वा इ॒तो लो॒के पि॒तर॑। ताने॒व प्री॑णाति। परा॒ङाव॑र्तते॥६०॥

%1.3.10.6
ह्लीका॒ हि पि॒तर॑। ओष्मणो व्या॒वृत॒ उपास्ते। ऊ॒ष्मभा॑गा॒ हि पि॒तर॑। ब्र॒ह्म॒वा॒दिनो॑ वदन्ति। प्राश्या (३) न्न प्राश्या (३) मिति॑। यत्प्राश्नी॒यात्। जन्य॒मन्न॑मद्यात्। प्र॒मायु॑कः स्यात्। यन्न प्राश्नी॒यात्। अह॑विः स्यात्॥६१॥

%1.3.10.7
पि॒तृभ्य॒ आवृ॑श्च्येत। अ॒व॒घ्रेय॑मे॒व। तन्नेव॒ प्राशि॑तं॒ नेवाप्रा॑शितम्। वी॒रं वा॒ वै पि॒तर॑ प्र॒यन्तो॒ हर॑न्ति। वी॒रं वा॑ ददति। द॒शां छि॑नत्ति। हर॑णभागा॒ हि पि॒तर॑। पि॒तॄने॒व नि॒रव॑दयते। उत्त॑र॒ आयु॑षि॒ लोम॑ छिन्दीत। पि॒तृ॒णा ह्ये॑तर्\mbox{}हि॒ नेदी॑यः॥६२॥

%1.3.10.8
नम॑स्करोति। न॒म॒स्का॒रो हि पि॑तृ॒णाम्। नमो॑ वः पितरो॒ रसा॑य। नमो॑ वः पितर॒ शुष्मा॑य। नमो॑ वः पितरो जी॒वाय॑। नमो॑ वः पितरः स्व॒धायै। नमो॑ वः पितरो म॒न्यवे। नमो॑ वः पितरो घो॒राय॑। पित॑रो॒ नमो॑ वः। य ए॒तस्मि॑ल्लोँ॒के स्थ॥६३॥

%1.3.10.9
यु॒ष्मास्तेऽनु॑। येऽस्मिल्लोँ॒के। मां तेऽनु॑। य ए॒तस्मि॑ल्लोँ॒के स्थ। यू॒यं तेषां॒ वसि॑ष्ठा भूयास्त। येऽस्मिल्लोँ॒के। अ॒हं तेषां॒ वसि॑ष्ठो भूयास॒मित्या॑ह। वसि॑ष्ठः समा॒नानां भवति। य ए॒वं वि॒द्वान्पि॒तृभ्य॑ क॒रोति॑। ए॒ष वै म॑नु॒ष्या॑णां य॒ज्ञः॥६४॥

%1.3.10.10
दे॒वानां॒ वा इत॑रे य॒ज्ञाः। तेन॒ वा ए॒तत्पि॑तृलो॒के च॑रति। यत्पि॒तृभ्य॑ क॒रोति॑। स ईश्व॒रः प्रमे॑तोः। प्रा॒जा॒प॒त्यय॒र्चा पुन॒रैति॑। य॒ज्ञो वै प्र॒जाप॑तिः। य॒ज्ञेनै॒व स॒ह पुन॒रैति॑। न प्र॒मायु॑को भवति। पि॒तृ॒लो॒के वा ए॒तद्यज॑मानश्चरति। यत्पि॒तृभ्य॑ क॒रोति॑। स ईश्व॒र आर्ति॒मार्तो। प्र॒जाप॑ति॒स्त्वावैनं॒ तत॒ उन्ने॑तुमर्\mbox{}ह॒तीत्या॑हुः। यत्प्रा॑जाप॒त्यय॒र्चा पुन॒रैति॑। प्र॒जाप॑तिरे॒वैनं॒ तत॒ उन्न॑यति। नार्ति॒मार्च्छ॑ति॒ यज॑मानः॥६५॥\anuvakamend[इत्य॑श्नुते पद्यन्ते पद्यन्ते॒ षड्वा ऋ॒तवो॑ वर्त॒तेऽह॑विः स्या॒न्नेदी॑य॒ स्थ य॒ज्ञो यज॑मानश्चरति॒ यत्पि॒तृभ्य॑ क॒रोति॒ पञ्च॑ च]




\prashnaend{दे॒वा॒सु॒रा अ॒ग्नीषोम॑योर्दे॒वा वै यथा॒दर्\mbox{}शं॑ दे॒वा वै यद॒न्यैर्ग्रहै॑र्ब्रह्मवा॒दिनो॒ नाग्नि॑ष्टो॒मो न सा॑वि॒त्रं दे॒वस्या॒हं ता॒र्प्य स॒प्तान्न॑हो॒मान्नृ॒षदं॒ त्वेन्द्रो॑ वृ॒त्र ह॒त्वा दश॑॥१०॥}{दे॒वा॒सु॒रा वा॒ज्ये॑वैनं॒ तस्माद्वाजपेयया॒जी दे॒वस्या॒हं वाज॒स्याव॑रुद्ध्या इन्द्रि॒यमे॒वास्मि॒न्॒ ह्लीका॒ हि पि॒तर॒ पञ्च॑षष्टिः॥६५॥}{दे॒वा॒सु॒रा यज॑मानः॥}{हरि॑ ओम्॥}{इति श्रीकृष्णयजुर्वेदीयतैत्तिरीयब्राह्मणे प्रथमाष्टके तृतीयः प्रपाठकः समाप्तः॥}
\clearpage
\sect{चतुर्थः प्रश्नः}
\setcounter{anuvakam}{0}
\dnsub{तैत्तिरीयब्राह्मणे प्रथमाष्टके चतुर्थः प्रपाठकः}

%1.4.1.1
उ॒भये॒ वा ए॒ते प्र॒जाप॑ते॒रध्य॑सृज्यन्त। दे॒वाश्चासु॑राश्च। तान्न व्य॑जानात्। इ॒मेऽन्य इ॒मेऽन्य इति॑। स दे॒वान॒शून॑करोत्। तान॒भ्य॑षुणोत्। तान्प॒वित्रे॑णापुनात्। तान्प॒रस्तात्प॒वित्र॑स्य॒ व्य॑गृह्णात्। ते ग्रहा॑ अभवन्। तद्ग्रहा॑णां ग्रह॒त्वम्॥१॥

%1.4.1.2
दे॒वता॒ वा ए॒ता यज॑मानस्य गृ॒हे गृ॑ह्यन्ते। यद्ग्रहा। वि॒दुरे॑नं दे॒वाः। यस्यै॒वं वि॒दुष॑ ए॒ते ग्रहा॑ गृ॒ह्यन्ते। ए॒षा वै सोम॒स्याहु॑तिः। यदु॑पा॒शुः। सोमे॑न दे॒वास्त॑र्पया॒णीति॒ खलु॒ वै सोमे॑न यजते। यदु॑पा॒शुं जु॒होति॑। सोमे॑नै॒व तद्दे॒वास्त॑र्पयति। यद्ग्रहां जु॒होति॑॥२॥

%1.4.1.3
दे॒वा ए॒व तद्दे॒वान्ग॑च्छन्ति। यच्च॑म॒सां जु॒होति॑। तेनै॒वानु॑रूपेण॒ यज॑मानः सुव॒र्गं लो॒कमे॑ति। किं न्वे॑तदग्र॑ आसी॒दित्या॑हुः। यत्पात्रा॒णीति॑। इ॒यं वा ए॒तदग्र॑ आसीत्। मृ॒न्मया॑नि॒ वा ए॒तान्या॑सन्। तैर्दे॒वा न व्या॒वृत॑मगच्छन्। त ए॒तानि॑ दारु॒मया॑णि॒ पात्राण्यपश्यन्। तान्य॑कुर्वत॥३॥

%1.4.1.4
तैर्वै ते व्या॒वृत॑मगच्छन्। यद्दा॑रु॒मया॑णि॒ पात्रा॑णि॒ भव॑न्ति। व्या॒वृत॑मे॒व तैर्यज॑मानो गच्छति। यानि॑ दारु॒मया॑णि॒ पात्रा॑णि॒ भव॑न्ति। अ॒मुमे॒व तैर्लो॒कम॒भिज॑यति। यानि॑ मृ॒न्मया॑नि। इ॒ममे॒व तैर्लो॒कम॒भिज॑यति। ब्र॒ह्म॒वा॒दिनो॑ वदन्ति। काश्चत॑स्रः स्था॒लीर्वा॑य॒व्या सोम॒ग्रह॑णी॒रिति॑। दे॒वा वै पृश्ञि॑मदुह्रन्॥४॥

%1.4.1.5
तस्या॑ ए॒ते स्तना॑ आसन्। इ॒यं वै पृश्ञि॑। तामा॑दि॒त्या आ॑दित्यस्था॒ल्या चतु॑ष्पदः प॒शून॑दुह्रन्। यदा॑दित्यस्था॒ली भव॑ति। चतु॑ष्पद ए॒व तया॑ प॒शून् यज॑मान इ॒मां दु॑हे। तामिन्द्र॑ उक्थ्यस्था॒ल्येन्द्रि॒यम॑दुहत्। यदु॑क्थ्यस्था॒ली भव॑ति। इ॒न्द्रि॒यमे॒व तया॒ यज॑मान इ॒मां दु॑हे। तां विश्वे॑ दे॒वा आग्रयणस्था॒ल्योर्ज॑मदुह्रन्। यदाग्रयणस्था॒ली भव॑ति॥५॥

%1.4.1.6
ऊर्ज॑मे॒व तया॒ यज॑मान इ॒मां दु॑हे। तां म॑नु॒ष्या ध्रुवस्था॒ल्याऽऽयु॑रदुह्रन्। यद्ध्रु॑वस्था॒ली भव॑ति। आयु॑रे॒व तया॒ यज॑मान इ॒मां दु॑हे। स्था॒ल्या गृ॒ह्णाति॑। वा॒य॒व्ये॑न जुहोति। तस्मा॑द॒न्येन॒ पात्रे॑ण प॒शून्दु॒हन्ति॑। अ॒न्येन॒ प्रति॑गृह्णन्ति। अथो व्या॒वृत॑मे॒व तद्यज॑मानो गच्छति॥६॥\anuvakamend[ग्र॒ह॒त्वं ग्रहां जु॒होत्य॑कुर्वतादुह्रन्नाग्रयणस्था॒ली भव॑ति॒ नव॑ च]

%1.4.2.1
यु॒व सु॒राम॑मश्विना। नमु॑चावासु॒रे सचा। वि॒पि॒पा॒ना शु॑भस्पती। इन्द्रं॒ कर्म॑ स्वावतम्। पु॒त्रमि॑व पि॒तरा॑व॒श्विनो॒भा। इन्द्राव॑तं॒ कर्म॑णा द॒सना॑भिः। यत्सु॒रामं॒ व्यपि॑ब॒ शची॑भिः। सर॑स्वती त्वा मघवन्नभीष्णात्। अहाव्यग्ने ह॒विरा॒स्ये॑ते। स्रु॒चीव॑ घृ॒तं च॒मू इ॑व॒ सोम॑॥७॥

%1.4.2.2
वा॒ज॒सनि र॒यिम॒स्मे सु॒वीरम्। प्र॒श॒स्तं धे॑हि य॒शसं॑ बृ॒हन्तम्। यस्मि॒न्नश्वा॑स ऋष॒भास॑ उ॒क्षण॑। व॒शा मे॒षा अ॑वसृ॒ष्टास॒ आहु॑ताः। की॒ला॒ल॒पे सोम॑पृष्ठाय वे॒धसे। हृ॒दा म॒तिं ज॑नय॒ चारु॑म॒ग्नये। नाना॒ हि वां दे॒वहि॑त॒ सदो॑ मि॒तम्। मा ससृ॑क्षाथां पर॒मे व्यो॑मन्। सुरा॒ त्वमसि॑ शु॒ष्मिणी॒ सोम॑ ए॒षः। मा मा॑ हिसी॒ स्वां योनि॑मावि॒शन्॥८॥

%1.4.2.3
यदत्र॑ शि॒ष्ट र॒सिन॑ सु॒तस्य॑। यदिन्द्रो॒ अपि॑ब॒च्छची॑भिः। अ॒हं तद॑स्य॒ मन॑सा शि॒वेन॑। सोम॒ राजा॑नमि॒ह भ॑क्षयामि। द्वे स्रु॒ती अ॑शृणवं पितृ॒णाम्। अ॒हं दे॒वाना॑मु॒त मर्त्या॑नाम्। ताभ्या॑मि॒दं विश्वं॒ भुव॑न॒ समे॑ति। अ॒न्त॒रा पूर्व॒मप॑रं च के॒तुम्। यस्ते॑ देव वरुण गाय॒त्रछ॑न्दा॒ पाश॑। तं त॑ ए॒तेनाव॑ यजे॥९॥

%1.4.2.4
यस्ते॑ देव वरुण त्रि॒ष्टुप्छ॑न्दा॒ पाश॑। तं त॑ ए॒तेनाव॑ यजे। यस्ते॑ देव वरुण॒ जग॑तीछन्दा॒ पाश॑। तं त॑ ए॒तेनाव॑ यजे। सोमो॒ वा ए॒तस्य॑ रा॒ज्यमाद॑त्ते। यो राजा॒ सन्रा॒ज्यो वा॒ सोमे॑न॒ यज॑ते। दे॒व॒सु॒वामे॒तानि॑ ह॒वीषि॑ भवन्ति। ए॒ताव॑न्तो॒ वै दे॒वाना स॒वाः। त ए॒वास्मै॑ स॒वान्प्रय॑च्छन्ति। त ए॑नं॒ पुन॑ सुवन्ते रा॒ज्याय॑। दे॒व॒सू राजा॑ भवति॥१०॥\anuvakamend[सोम॑ आवि॒शन् य॑जे रा॒ज्यायैकं॑ च]

%1.4.3.1
उद॑स्थाद्दे॒व्यदि॑तिर्विश्वरू॒पी। आयु॑र्य॒ज्ञप॑तावधात्। इन्द्रा॑य कृण्व॒ती भा॒गम्। मि॒त्राय॒ वरु॑णाय च। इ॒यं वा अ॑ग्निहो॒त्री। इ॒यं वा ए॒तस्य॒ निषी॑दति। यस्याग्निहो॒त्री नि॒षीद॑ति। तामुत्था॑पयेत्। उद॑स्थाद्दे॒व्यदि॑ति॒रिति॑। इ॒यं वै दे॒व्यदि॑तिः॥११॥

%1.4.3.2
इ॒मामे॒वास्मा॒ उत्था॑पयति। आयु॑र्य॒ज्ञप॑तावधा॒दित्या॑ह। आयु॑रे॒वास्मि॑न्दधाति। इन्द्रा॑य कृण्व॒ती भा॒गं मि॒त्राय॒ वरु॑णाय॒ चेत्या॑ह। य॒था॒य॒जुरे॒वैतत्। अव॑र्तिं॒ वा ए॒षैतस्य॑ पा॒प्मानं॑ प्रति॒ख्याय॒ निषी॑दति। यस्याग्निहो॒त्र्युप॑सृष्टा नि॒षीद॑ति। तां दु॒ग्ध्वा ब्राह्म॒णाय॑ दद्यात्। यस्यान्नं॒ नाद्यात्। अव॑र्तिमे॒वास्मि॑न्पा॒प्मानं॒ प्रति॑मुञ्चति॥१२॥

%1.4.3.3
दु॒ग्ध्वा द॑दाति। न ह्यदृ॑ष्टा॒ दक्षि॑णा दी॒यते। पृ॒थि॒वीं वा ए॒तस्य॒ पय॒ प्रवि॑शति। यस्याग्निहो॒त्रं दु॒ह्यमा॑न॒ स्कन्द॑ति। यद॒द्य दु॒ग्धं पृ॑थि॒वीमस॑क्त। यदोष॑धीर॒प्यस॑र॒द्यदाप॑। पयो॑ गृ॒हेषु॒ पयो॑ अघ्नि॒यासु॑। पयो॑ व॒त्सेषु॒ पयो॑ अस्तु॒ तन्मयीत्या॑ह। पय॑ ए॒वात्मन्गृ॒हेषु॑ प॒शुषु॑ धत्ते। अ॒प उप॑सृजति॥१३॥

%1.4.3.4
अ॒द्भिरे॒वैन॑दाप्नोति। यो वै य॒ज्ञस्यार्ते॒ नानार्त स सृ॒जति॑। उ॒भे वै ते तर्ह्यार्च्छ॑तः। आर्च्छ॑ति॒ खलु॒ वा ए॒तद॑ग्निहो॒त्रम्। यद्दु॒ह्यमा॑न॒ स्कन्द॑ति। यद॑भिदु॒ह्यात्। आर्ते॒ नानार्तं य॒ज्ञस्य॒ ससृ॑जेत्। तदे॒व या॒दृक्की॒दृक्च॑ होत॒व्यम्। अथा॒न्यां दु॒ग्ध्वा पुन॑र्\mbox{}होत॒व्यम्। अनार्तेनै॒वार्तं॑ य॒ज्ञस्य॒ निष्क॑रोति॥१४॥

%1.4.3.5
यद्युद्द्रु॑तस्य॒ स्कन्देत्। यत्ततोऽहु॑त्वा॒ पुन॑रे॒यात्। य॒ज्ञं विच्छि॑न्द्यात्। यत्र॒ स्कन्देत्। तन्नि॒षद्य॒ पुन॑र्गृह्णीयात्। यत्रै॒व स्कन्द॑ति। तत॑ ए॒वैन॒त्पुन॑र्गृह्णाति। तदे॒व या॒दृक्की॒दृक्च॑ होत॒व्यम्। अथा॒न्यां दु॒ग्ध्वा पुन॑र्\mbox{}होत॒व्यम्। अनार्तेनै॒वार्तं॑ य॒ज्ञस्य॒ निष्क॑रोति॥१५॥

%1.4.3.6
वि वा ए॒तस्य॑ य॒ज्ञश्छि॑द्यते। यस्याग्निहो॒त्रे॑ऽधिश्रि॑ते॒ श्वाऽन्त॒रा धाव॑ति। रु॒द्रः खलु॒ वा ए॒षः। यद॒ग्निः। यद्गाम॑न्वत्या व॒र्तयेत्। रु॒द्राय॑ प॒शूनपि॑ दध्यात्। अ॒प॒शुर्यज॑मानः स्यात्। यद॒पोऽन्वतिषि॒ञ्चेत्। अ॒ना॒द्यम॒ग्नेराप॑। अ॒ना॒द्यमाभ्या॒मपि॑ दध्यात्। गार्\mbox{}ह॑पत्या॒द्भस्मा॒दाय॑। इ॒दं विष्णु॒र्विच॑क्रम॒ इति॑ वैष्ण॒व्यर्चाऽऽह॑व॒नीयाद्ध्व॒सय॒न्नुद्र॑वेत्। य॒ज्ञो वै विष्णु॑। य॒ज्ञेनै॒व य॒ज्ञ संत॑नोति। भस्म॑ना प॒दमपि॑ वपति॒ शान्त्यै॥१६॥\anuvakamend[वै दे॒व्यदि॑तिर्मुञ्चति सृजति करोति करोत्याभ्या॒मपि॑ दध्या॒त् पञ्च॑ च]

%1.4.4.1
नि वा ए॒तस्या॑हव॒नीयो॒ गार्\mbox{}ह॑पत्यं कामयते। निगार्\mbox{}ह॑पत्य आहव॒नीयम्। यस्या॒ग्निमनु॑द्धृत॒ सूर्यो॒ऽभि नि॒म्रोच॑ति। द॒र्भेण॒ हिर॑ण्यं प्र॒बद्ध्य॑ पु॒रस्ताद्धरेत्। अथा॒ग्निम्। अथाग्निहो॒त्रम्। यद्धिर॑ण्यं पु॒रस्ता॒द्धर॑ति। ज्योति॒र्वै हिर॑ण्यम्। ज्योति॑रे॒वैनं॒ पश्य॒न्नुद्ध॑रति। यद॒ग्निं पूर्व॒ हर॒त्यथाग्निहो॒त्रम्॥१७॥

%1.4.4.2
भा॒ग॒धेये॑नै॒वैनं॒ प्रण॑यति। ब्रा॒ह्म॒ण आ॑र्\mbox{}षे॒य उद्ध॑रेत्। ब्रा॒ह्म॒णो वै सर्वा॑ दे॒वता। सर्वा॑भिरे॒वैनं॑ दे॒वता॑भि॒रुद्ध॑रति। अ॒ग्नि॒हो॒त्रमु॑प॒साद्यातमि॑तोरासीत। व्र॒तमे॒व ह॒तमनु॑ म्रियते। अन्तं॒ वा ए॒ष आ॒त्मनो॑ गच्छति। यस्ताम्य॑ति। अन्त॑मे॒ष य॒ज्ञस्य॑ गच्छति। यस्या॒ग्निमनु॑द्धृ॒त सूर्यो॒ऽभि नि॒म्रोच॑ति॥१८॥

%1.4.4.3
पुन॑ स॒मन्य॑ जुहोति। अन्ते॑नै॒वान्तं॑ य॒ज्ञस्य॒ निष्क॑रोति। वरु॑णो॒ वा ए॒तस्य॑ य॒ज्ञं गृ॑ह्णाति। यस्या॒ग्निमनु॑द्धृत॒ सूर्यो॒ऽभि नि॒म्रोच॑ति। वा॒रु॒णं च॒रुं निर्व॑पेत्। तेनै॒व य॒ज्ञं निष्क्री॑णीते। नि वा ए॒तस्या॑हव॒नीयो॒ गार्\mbox{}ह॑पत्यं कामयते। नि गार्\mbox{}ह॑पत्य आहव॒नीयम्। यस्या॒ग्निमनु॑द्धृत॒ सूर्यो॒ऽभ्यु॑देति॑। च॒तु॒र्गृ॒ही॒तमाज्यं॑ पु॒रस्ताद्धरेत्॥१९॥

%1.4.4.4
अथा॒ग्निम्। अथाग्निहो॒त्रम्। यदाज्यं॑ पु॒रस्ता॒द्धर॑ति। ए॒तद्वा अ॒ग्नेः प्रि॒यं धाम॑। यदाज्यम्। प्रि॒येणै॒वैनं॒ धाम्ना॒ सम॑र्धयति। यद॒ग्निं पूर्व॒ हर॒त्यथाग्निहो॒त्रम्। भा॒ग॒धेये॑नै॒वैनं॒ प्रण॑यति। ब्रा॒ह्म॒ण आ॑र्\mbox{}षे॒य उद्ध॑रेत्। ब्रा॒ह्म॒णो वै सर्वा॑ दे॒वता॥२०॥

%1.4.4.5
सर्वा॑भिरे॒वैनं॑ दे॒वता॑भि॒रुद्ध॑रति। परा॑ची॒ वा ए॒तस्मै व्यु॒च्छन्ती॒ व्यु॑च्छति। यस्या॒ग्निमनु॑द्धृत॒ सूर्यो॒ऽभ्यु॑देति॑। उ॒षाः के॒तुना॑ जुषताम्। य॒ज्ञं दे॒वेभि॑रिन्वि॒तम्। दे॒वेभ्यो॒ मधु॑मत्तम॒ स्वाहेति॑ प्र॒त्यङ्नि॒षद्याज्ये॑न जुहुयात्। प्र॒तीची॑मे॒वास्मै॒ विवा॑सयति। अ॒ग्नि॒हो॒त्रमु॑प॒साद्यातमि॑तोरासीत। व्र॒तमे॒व ह॒तमनु॑ म्रियते। अन्तं॒ वा ए॒ष आ॒त्मनो॑ गच्छति॥२१॥

%1.4.4.6
यस्ताम्य॑ति। अन्त॑मे॒ष य॒ज्ञस्य॑ गच्छति। यस्या॒ग्निमनु॑द्धृत॒ सूर्यो॒ऽभ्यु॑देति॑। पुन॑ स॒मन्य॑ जुहोति। अन्ते॑नै॒वान्तं॑ य॒ज्ञस्य॒ निष्क॑रोति। मि॒त्रो वा ए॒तस्य॑ य॒ज्ञं गृ॑ह्णाति। यस्या॒ग्निमनु॑द्धृत॒ सूर्यो॒ऽभ्यु॑देति॑। मै॒त्रं च॒रुं निर्व॑पेत्। तेनै॒व य॒ज्ञं निष्क्री॑णीते। यस्या॑हव॒नीयेऽ नु॑द्वाते॒ गार्\mbox{}ह॑पत्य उ॒द्वायेत् ॥२२॥

%1.4.4.7
यदा॑हव॒नीय॒मनु॑द्वाप्य॒ गार्\mbox{}ह॑पत्यं॒ मन्थेत्। विच्छि॑न्द्यात्। भ्रातृ॑व्यमस्मै जनयेत्। यद्वै य॒ज्ञस्य॑ वास्त॒व्यं॑ क्रि॒यते। तदनु॑ रु॒द्रोऽव॑चरति। यत्पूर्व॑मन्वव॒स्येत्। वा॒स्त॒व्य॑म॒ग्निमुपा॑सीत। रु॒द्रोऽस्य प॒शून्घातु॑कः स्यात्। आ॒ह॒व॒नीय॑मु॒द्वाप्य॑। गार्\mbox{}ह॑पत्यं मन्थेत्॥२३॥

%1.4.4.8
इ॒तः प्र॑थ॒मं ज॑ज्ञे अ॒ग्निः। स्वाद्योने॒रधि॑ जा॒तवे॑दाः। स गा॑यत्रि॒या त्रि॒ष्टुभा॒ जग॑त्या। दे॒वेभ्यो॑ ह॒व्यं व॑हतु प्रजा॒नन्निति॑। छन्दो॑भिरे॒वैन॒ स्वाद्योने॒ प्रज॑नयति। गार्\mbox{}ह॑पत्यं मन्थति। गार्\mbox{}ह॑पत्यं॒ वा अन्वाहि॑ताग्नेः प॒शव॒ उप॑ तिष्ठन्ते। स यदु॒द्वाय॑ति। तदनु॑ प॒शवोऽप॑ क्रामन्ति। इ॒षे र॒य्यै र॑मस्व॥२४॥

%1.4.4.9
सह॑से द्यु॒म्नाय॑। ऊ॒र्जेऽपत्या॒येत्या॑ह। प॒शवो॒ वै र॒यिः। प॒शूने॒वास्मै॑ रमयति। सा॒र॒स्व॒तौ त्वोत्सौ॒ समि॑न्धाता॒मित्या॑ह। ऋ॒ख्सा॒मे वै सा॑रस्व॒तावुत्सौ। ऋ॒ख्सा॒माभ्या॑मे॒वैन॒ समि॑न्धे। स॒म्राड॑सि वि॒राड॒सीत्या॑ह। र॒थ॒न्त॒रं वै स॒म्राट्। बृ॒हद्वि॒राट्। 2५॥

%1.4.4.10
ताभ्या॑मे॒वैन॒ समि॑न्धे। वज्रो॒ वै च॒क्रम्। वज्रो॒ वा ए॒तस्य॑ य॒ज्ञं विच्छि॑नत्ति। यस्यानो॑ वा॒ रथो॑ वाऽन्त॒राऽग्नी याति॑। आ॒ह॒व॒नीय॑मु॒द्वाप्य॑। गार्\mbox{}ह॑पत्या॒दुद्ध॑रेत्। यद॑ग्ने॒ पूर्वं॒ प्रभृ॑तं प॒द हि ते। सूर्य॑स्य र॒श्मीनन्वा॑त॒तान॑। तत्र॑ रयि॒ष्ठामनु॒ सं भ॑रै॒तम्। सं न॑ सृज सुम॒त्या वाज॑व॒त्येति॑॥२६॥

%1.4.4.11
पूर्वे॑णै॒वास्य॑ य॒ज्ञेन॑ य॒ज्ञमनु॒ संत॑नोति। त्वम॑ग्ने स॒प्रथा॑ अ॒सीत्या॑ह। अ॒ग्निः सर्वा॑ दे॒वता। दे॒वता॑भिरे॒व य॒ज्ञ संत॑नोति। अ॒ग्नये॑ पथि॒कृते॑ पुरो॒डाश॑म॒ष्टाक॑पालं॒ निर्व॑पेत्। अ॒ग्निमे॒व प॑थि॒कृत॒ स्वेन॑ भाग॒धेये॒नोप॑धावति। स ए॒वैनं॑ य॒ज्ञियं॒ पन्था॒मपि॑ नयति। अ॒न॒ड्वान्दक्षि॑णा। व॒ही ह्ये॑ष समृ॑द्ध्यै॥२७॥\anuvakamend[हर॒त्यथाग्निहो॒त्रं नि॒म्रोच॑ति हरेद्दे॒वता॑ गच्छत्यु॒द्वायेन्मन्थेद्रमस्व बृ॒हद्वि॒राडिति॒ नव॑ च (नि वै पूर्वं॒ त्रीणि॑ नि॒म्रोच॑ति द॒र्भेण॒ यद्धिर॑ण्यमग्निहो॒त्रं पुन॒र्वरु॑णो वारु॒णं नि वा ए॒तस्या॒भ्यु॑देति॑ चतुर्गृही॒तमाज्यं॒ यदाज्यं॒ पराच्यु॒षाः पुन॑र्मि॒त्रो मै॒त्रं यस्या॑हव॒नीयेऽनु॑द्वाते॒ गार्\mbox{}ह॑पत्यो॒ यद्वै म॑न्थे॒दुद्ध॑रेत् ॥ )]

%1.4.5.1
यस्य॑ प्रातः सव॒ने सोमो॑ऽति॒रिच्य॑ते। माध्य॑न्दिन॒ सव॑नं का॒मय॑मानो॒ऽभ्यति॑रिच्यते। गौर्ध॑यति म॒रुता॒मिति॒ धय॑द्वतीषु कुर्वन्ति। हि॒नस्ति॒ वै स॒न्ध्यधी॑तम्। स॒न्धीव॒ खलु॒ वा ए॒तत्। यत्सव॑नस्याति॒रिच्य॑ते। यद्धय॑द्वतीषु कु॒र्वन्ति॑। स॒न्धेः शान्त्यै। गा॒य॒त्र साम॑ भवति पञ्चद॒शः स्तोम॑। तेनै॒व प्रा॑तः सव॒नान्नय॑न्ति॥२८॥

%1.4.5.2
म॒रुत्व॑तीषु कुर्वन्ति। तेनै॒व माध्य॑न्दिना॒त्सव॑ना॒न्नय॑न्ति। होतु॑श्चम॒समनून्न॑यन्ते। होताऽनु॑ शसति। म॒ध्य॒त ए॒व य॒ज्ञ स॒माद॑धाति। यस्य॒ माध्य॑न्दिने॒ सव॑ने॒ सोमो॑ऽति॒रिच्य॑ते। आ॒दि॒त्यं तृ॑तीयसव॒नं का॒मय॑मानो॒ऽभ्यति॑रिच्यते। गौ॒रि॒वी॒त साम॑ भवति। अति॑रिक्तं॒ वै गौ॑रिवी॒तम्। अति॑रिक्तं॒ यत्सव॑नस्याति॒रिच्य॑ते॥२९॥

%1.4.5.3
अति॑रिक्तस्य॒ शान्त्यै। बण्म॒हा अ॑सि सू॒र्येति॑ कुर्वन्ति। यस्यै॒वादि॒त्यस्य॒ सव॑नस्य॒ कामे॑नाति॒रिच्य॑ते। तेनै॒वैनं॒ कामे॑न॒ सम॑र्धयन्ति। गौ॒रि॒वी॒त साम॑ भवति। तेनै॒व माध्य॑न्दिना॒त्सव॑ना॒न्नय॑न्ति। स॒प्त॒द॒शः स्तोम॑। तेनै॒व तृ॑तीयसव॒नान्नय॑न्ति। होतु॑श्चम॒समनून्न॑यन्ते। होताऽनु॑ शसति॥३०॥

%1.4.5.4
म॒ध्य॒त ए॒व य॒ज्ञ स॒माद॑धाति। यस्य॑ तृतीयसव॒ने सोमो॑ऽति॒रिच्ये॑त। उ॒क्थ्यं॑ कुर्वीत। यस्यो॒क्थ्ये॑ऽति॒रिच्ये॑त। अ॒ति॒रा॒त्रं कु॑र्वीत। यस्या॑तिरा॒त्रे॑ऽति॒रिच्य॑ते। तत्त्वै दु॑ष्प्रज्ञा॒नम्। यज॑मानं॒ वा ए॒तत्प॒शव॑ आ॒साह्य॑यन्ति। बृ॒हत्साम॑ भवति। बृ॒हद्वा इ॒माल्लोँ॒कान्दा॑धार। बार्\mbox{}ह॑ताः प॒शव॑। बृ॒ह॒तैवास्मै॑ प॒शून्दा॑धार। शि॒पि॒वि॒ष्टव॑तीषु कुर्वन्ति। शि॒पि॒वि॒ष्टो वै दे॒वानां पु॒ष्टम्। पुष्ट्यै॒वैन॒ सम॑र्धयन्ति। होतु॑श्चम॒समनून्न॑यन्ते। होताऽनु॑शसति। म॒ध्य॒त ए॒व य॒ज्ञ स॒माद॑धाति॥३१॥\anuvakamend[य॒न्ति॒ सव॑नस्याति॒रिच्य॑ते शसति दाधारा॒ष्टौ च॑]

%1.4.6.1
एकै॑को॒ वै ज॒नता॑या॒मिन्द्र॑। एकं॒ वा ए॒ताविन्द्र॑म॒भि ससु॑नुतः। यौ द्वौ स सुनु॒तः। प्र॒जाप॑ति॒र्वा ए॒ष विता॑यते। यद्य॒ज्ञः। तस्य॒ ग्रावा॑णो॒ दन्ता। अ॒न्य॒त॒रं वा ए॒ते ससुन्व॒तोर्निर्ब॑प्सति। पूर्वे॑णोप॒सृत्या॑ दे॒वता॒ इत्या॑हुः। पू॒र्वो॒प॒सृ॒तस्य॒ वै श्रेयान्भवति। एति॑व॒न्त्याज्या॑नि भवन्त्य॒भिजि॑त्यै॥३२॥

%1.4.6.2
म॒रुत्व॑तीः प्रति॒पद॑। म॒रुतो॒ वै दे॒वाना॒मप॑राजितमा॒यत॑नम्। दे॒वाना॑मे॒वाप॑राजित आ॒यत॑ने यतते। उ॒भे बृ॑हद्रथन्त॒रे भ॑वतः। इ॒यं वाव र॑थन्त॒रम्। अ॒सौ बृ॒हत्। आ॒भ्यामे॒वैन॑म॒न्तरे॑ति। वा॒चश्च॒ मन॑सश्च। प्रा॒णाच्चा॑पा॒नाच्च॑। दि॒वश्च॑ पृथि॒व्याश्च॑॥३३॥

%1.4.6.3
सर्व॑स्माद्वि॒त्ताद्वेद्यात्। अ॒भि॒व॒र्तो ब्र॑ह्मसा॒मं भ॑वति। सु॒व॒र्गस्य॑ लो॒कस्या॒भिवृ॑त्त्यै। अ॒भि॒जिद्भ॑वति। सु॒व॒र्गस्य॑ लो॒कस्या॒भिजि॑त्यै। वि॒श्व॒जिद्भ॑वति। विश्व॑स्य॒ जित्यै। यस्य॒ भूयासो यज्ञक्र॒तव॒ इत्या॑हुः। स दे॒वता॑ वृङ्क्त॒ इति॑। यद्य॑ग्निष्टो॒मः सोम॑ प॒रस्ता॒त्स्यात्॥३४॥

%1.4.6.4
उ॒क्थ्यं॑ कुर्वीत। यद्यु॒क्थ॑ स्यात्। अ॒ति॒रा॒त्रं कु॑र्वीत। य॒ज्ञ॒क्र॒तुभि॑रे॒वास्य॑ दे॒वता॑ वृङ्क्ते। यो वै छन्दो॑भिरभि॒भव॑ति। स ससुन्व॒तोर॒भिभ॑वति। सं॒वे॒शाय॑ त्वोपवे॒शाय॒ त्वेत्या॑ह। छन्दासि॒ वै सं॑वे॒श उ॑पवे॒शः। छन्दो॑भिरे॒वास्य॒ छन्दास्य॒भिभ॑वति। इ॒ष्टर्गो॒ वा ऋ॒त्विजा॑मध्व॒र्युः॥३५॥

%1.4.6.5
इ॒ष्टर्ग॒ खलु॒ वै पूर्वो॒ऽर्ष्टुः क्षी॑यते। प्राणा॑पानौ मृ॒त्योर्मा॑ पात॒मित्या॑ह। प्रा॒णा॒पा॒नयो॑रे॒व श्र॑यते। प्राणा॑पानौ॒ मा मा॑हासिष्ट॒मित्या॑ह। नैनं॑ पु॒राऽऽयु॑षः प्राणापा॒नौ ज॑हितः। आर्तिं॒ वा ए॒ते निय॑न्ति। येषां दीक्षि॒तानां प्र॒मीय॑ते। तं यद॑व॒वर्जे॑युः। क्रू॒र॒कृता॑मिवैषां लो॒कः स्यात्। आह॑र द॒हेति॑ ब्रूयात्॥३६॥

%1.4.6.6
तं द॑क्षिण॒तो वेद्यै॑ नि॒धाय॑। स॒र्प॒रा॒ज्ञिया॑ ऋ॒ग्भिः स्तु॑युः। इ॒यं वै सर्प॑तो॒ राज्ञी। अ॒स्या ए॒वैनं॒ परि॑ददति। व्यृ॑द्धं॒ तदित्या॑हुः। यत्स्तु॒तमन॑नुशस्त॒मिति॑। होता प्रथ॒मः प्रा॑चीनावी॒ती मार्जा॒लीयं॒ परी॑यात्। या॒मीर॑नुब्रु॒वन्। स॒र्प॒रा॒ज्ञीनां कीर्तयेत्। उ॒भयो॑रे॒वैनं॑ लो॒कयो॒ परि॑ददति॥३७॥

%1.4.6.7
अथो॑ धु॒वन्त्ये॒वैनम्। अथो॒ न्ये॑वास्मै ह्नुवते। त्रिः परि॑यन्ति। त्रय॑ इ॒मे लो॒काः। ए॒भ्य ए॒वैनं॑ लो॒केभ्यो॑ धुवते। त्रिः पुन॒ परि॑यन्ति। षट्त्संप॑द्यन्ते। षड्वा ऋ॒तव॑। ऋ॒तुभि॑रे॒वैनं॑ धुवते। अग्न॒ आयूषि पवस॒ इति॑ प्रति॒पदं॑ कुर्वीरन्। र॒थ॒न्त॒रसा॑मैषा॒ सोम॑ स्यात्। आयु॑रे॒वात्मन्द॑धते। अथो॑ पा॒प्मान॑मे॒व वि॒जह॑तो यन्ति॥३८॥\anuvakamend[अ॒भिजि॑त्यै पृथि॒व्याश्च॒ स्याद॑ध्व॒र्युर्ब्रू॑याल्लो॒कयो॒ परि॑ददति कुर्वीर॒स्त्रीणि॑ च]

%1.4.7.1
अ॒सु॒र्यं॑ वा ए॒तस्मा॒द्वर्णं॑ कृ॒त्वा। प॒शवो॑ वी॒र्य॑मप॑ क्रामन्ति। यस्य॒ यूपो॑ वि॒रोह॑ति। त्वा॒ष्ट्रं ब॑हुरू॒पमाल॑भेत। त्वष्टा॒ वै रू॒पाणा॑मीशे। य ए॒व रू॒पाणा॒मीशे। सोऽस्मिन्प॒शून् वी॒र्यं॑ यच्छति। नास्मात्प॒शवो॑ वी॒र्य॑मप॑ क्रामन्ति। आर्तिं॒ वा ए॒ते निय॑न्ति। येषां दीक्षि॒ताना॑म॒ग्निरु॒द्वाय॑ति॥३९॥

%1.4.7.2
यदा॑हव॒नीय॑ उ॒द्वायेत्। यत्तं मन्थेत्। विच्छि॑न्द्यात्। भ्रातृ॑व्यमस्मै जनयेत्। यदा॑हव॒नीय॑ उ॒द्वायेत्। आग्नीद्ध्रा॒दुद्ध॑रेत्। यदाग्नीद्ध्र उ॒द्वायेत्। गार्\mbox{}ह॑पत्या॒दुद्ध॑रेत्। यद्गार्\mbox{}ह॑पत्य उ॒द्वायेत्। अत॑ ए॒व पुन॑र्मन्थेत्॥४०॥

%1.4.7.3
अत्र॒ वाव स निल॑यते। यत्र॒ खलु॒ वै निली॑नमुत्त॒मं पश्य॑न्ति। तदे॑नमिच्छन्ति। यस्मा॒द्दारो॑रु॒द्वायेत्। तस्या॒रणी॑ कुर्यात्। क्रु॒मु॒कमपि॑ कुर्यात्। ए॒षा वा अ॒ग्नेः प्रि॒या त॒नूः। यत्क्रु॑मु॒कः। प्रि॒ययै॒वैन॑न्त॒नुवा॒ सम॑र्धयति। गार्\mbox{}ह॑पत्यं मन्थति॥४१॥

%1.4.7.4
गार्\mbox{}ह॑पत्यो॒ वा अ॒ग्नेर्योनि॑। स्वादे॒वैनं॒ योनेर्जनयति। नास्मै॒ भ्रातृ॑व्यञ्जनयति। यस्य॒ सोम॑ उप॒दस्येत्। सु॒वर्ण॒ हिर॑ण्यन्द्वे॒धा वि॒च्छिद्य॑। ऋ॒जी॒षेऽन्यदा॑धूनु॒यात्। जु॒हु॒याद॒न्यत्। सोम॑मे॒वाभि॑षु॒णोति॑। सोमं॑ जुहोति। सोम॑स्य॒ वा अ॑भिषू॒यमा॑णस्य प्रि॒या त॒नूरुद॑क्रामत्॥४२॥

%1.4.7.5
तत्सु॒वर्ण॒ हिर॑ण्यमभवत्। यत्सु॒वर्ण॒ हिर॑ण्यं कु॒र्वन्ति॑। प्रि॒ययै॒वैन॑न्त॒नुवा॒ सम॑र्धयन्ति। यस्याक्री॑त॒ सोम॑मप॒हरे॑युः। क्री॒णी॒यादे॒व। सैव तत॒ प्राय॑श्चित्तिः। यस्य॑ क्री॒तम॑प॒हरे॑युः। आ॒दा॒राश्च॑ फाल्गु॒नानि॑ चा॒भिषु॑णुयात्। गा॒य॒त्री य सोम॒माह॑रत्। तस्य॒ योऽशुः प॒राऽप॑तत्॥४३॥

%1.4.7.6
त आ॑दा॒रा अ॑भवन्। इन्द्रो॑ वृ॒त्रम॑हन्। तस्य॑ व॒ल्कः परा॑ऽपतत्। तानि॑ फाल्गु॒नान्य॑भवन्। प॒शवो॒ वै फाल्गु॒नानि॑। प॒शव॒ सोमो॒ राजा। यदा॑दा॒राश्च॑ फाल्गु॒नानि॑ चाभिषु॒णोति॑। सोम॑मे॒व राजा॑नम॒भिषु॑णोति। शृ॒तेन॑ प्रातः सव॒ने श्री॑णीयात्। द॒ध्ना म॒ध्यन्दि॑ने॥४४॥

%1.4.7.7
नी॒त॒मि॒श्रेण॑ तृतीयसव॒ने। अ॒ग्नि॒ष्टो॒मः सोम॑ स्याद्रथन्त॒रसा॑मा। य ए॒वर्त्विजो॑ वृ॒ताः स्युः। त ए॑नं याजयेयुः। एका॒ङ्गान्दक्षि॑णान्दद्या॒त्तेभ्य॑ ए॒व। पुन॒ सोमं॑ क्रीणीयात्। य॒ज्ञेनै॒व तद्य॒ज्ञमि॑च्छति। सैव तत॒ प्राय॑श्चित्तिः। सर्वाभ्यो॒ वा ए॒ष दे॒वताभ्य॒ सर्वेभ्यः पृ॒ष्ठेभ्य॑ आ॒त्मान॒मागु॑रते। यः स॒त्राया॑गु॒रते। ए॒तावा॒न्खलु॒ वै पुरु॑षः। याव॑दस्य वि॒त्तम्। स॒र्व॒वे॒द॒सेन॑ यजेत। सर्व॑पृष्ठोऽस्य॒ सोम॑ स्यात्। सर्वाभ्य ए॒व दे॒वताभ्य॒ सर्वेभ्यः पृ॒ष्ठेभ्य॑ आ॒त्मानं॒ निष्क्री॑णीते॥४५॥\anuvakamend[उ॒द्वाय॑ति मन्थेन्मन्थत्यक्रामत्प॒राऽप॑तन्म॒ध्यन्दि॑न आगु॒रते॒ पञ्च॑ च]

%1.4.8.1
पव॑मान॒ सुव॒र्जन॑। प॒वित्रे॑ण॒ विच॑र्\mbox{}षणिः। यः पोता॒ स पु॑नातु मा। पु॒नन्तु॑ मा देवज॒नाः। पु॒नन्तु॒ मन॑वो धि॒या। पु॒नन्तु॒ विश्व॑ आ॒यव॑। जात॑वेदः प॒वित्र॑वत्। प॒वित्रे॑ण पुनाहि मा। शू॒क्रेण॑ देव॒ दीद्य॑त्। अग्ने॒ क्रत्वा॒ क्रतू॒ रनु॑॥४६॥

%1.4.8.2
यत्ते॑ प॒वित्र॑म॒र्चिषि॑। अग्ने॒ वित॑तमन्त॒रा। ब्रह्म॒ तेन॑ पुनीमहे। उ॒भाभ्यान्देव सवितः। प॒वित्रे॑ण स॒वेन॑ च। इ॒दं ब्रह्म॑ पुनीमहे। वै॒श्व॒दे॒वी पु॑न॒ती दे॒व्यागात्। यस्यै॑ ब॒ह्वीस्त॒नुवो॑ वी॒तपृ॑ष्ठाः। तया॒ मद॑न्तः सध॒माद्ये॑षु। व॒य स्या॑म॒ पत॑यो रयी॒णाम्॥४७॥

%1.4.8.3
वै॒श्वा॒न॒रो र॒श्मिभि॑र्मा पुनातु। वात॑ प्रा॒णेने॑षि॒रो म॑यो॒भूः। द्यावा॑पृथि॒वी पय॑सा॒ पयो॑भिः। ऋ॒ताव॑री य॒ज्ञिये॑ मा पुनीताम्। बृ॒हद्भि॑ सवित॒स्तृभि॑। वर्\mbox{}षि॑ष्ठैर्देव॒ मन्म॑भिः। अग्ने॒ दक्षै पुनाहि मा। येन॑ दे॒वा अपु॑नत। येनापो॑ दि॒व्यङ्कश॑। तेन॑ दि॒व्येन॒ ब्रह्म॑णा॥४८॥

%1.4.8.4
इ॒दं ब्रह्म॑ पुनीमहे। यः पा॑वमा॒नीर॒ध्येति॑। ऋषि॑भि॒ सम्भृ॑त॒ रसम्। सर्व॒ स पू॒तम॑श्ञाति। स्व॒दि॒तं मा॑त॒रिश्व॑ना। पा॒व॒मा॒नीर्यो अ॒ध्येति॑। ऋषि॑भि॒ सम्भृ॑त॒ रसम्। तस्मै॒ सर॑स्वती दुहे। क्षी॒र स॒र्पिर्मधू॑द॒कम्। पा॒व॒मा॒नीः स्व॒स्त्यय॑नीः॥४९॥

%1.4.8.5
सु॒दुघा॒ हि पय॑स्वतीः। ऋषि॑भि॒ सम्भृ॑तो॒ रस॑। ब्रा॒ह्म॒णेष्व॒मृत हि॒तम्। पा॒व॒मानीर्दि॑शन्तु नः। इ॒मं लो॒कमथो॑ अ॒मुम्। कामा॒न्त्सम॑र्धयन्तु नः। दे॒वीर्दे॒वैः स॒माभृ॑ताः। पा॒व॒मा॒नीः स्व॒स्त्यय॑नीः। सु॒दुघा॒ हि घृ॑त॒श्चुत॑। ऋषि॑भि॒ सम्भृ॑तो॒ रस॑॥५०॥

%1.4.8.6
ब्रा॒ह्म॒णेष्व॒मृत हि॒तम्। येन॑ दे॒वाः प॒वित्रे॑ण। आ॒त्मानं॑ पु॒नते॒ सदा। तेन॑ स॒हस्र॑धारेण। पा॒व॒मा॒न्यः पु॑नन्तु मा। प्रा॒जा॒प॒त्यं प॒वित्रम्। श॒तोद्या॑म हिर॒ण्मयम्। तेन॑ ब्रह्म॒विदो॑ व॒यम्। पू॒तं ब्रह्म॑ पुनीमहे। इन्द्र॑ सुनी॒ती स॒ह मा॑ पुनातु। सोम॑ स्व॒स्त्या वरु॑णः स॒मीच्या। य॒मो राजा प्रमृ॒णाभि॑ पुनातु मा। जा॒तवे॑दा मो॒र्जय॑न्त्या पुनातु॥५१॥\anuvakamend[अनु॑ रयी॒णां ब्रह्म॑णा स्व॒स्त्यय॑नीः सु॒दुघा॒ हि घृ॑त॒श्चुत॒ ऋषि॑भि॒ सम्भृ॑तो॒ रस॑ पुनातु॒ त्रीणि॑ च]

%1.4.9.1
प्र॒जा वै स॒त्रमा॑सत॒ तप॒स्तप्य॑माना॒ अजु॑ह्वतीः। दे॒वा अ॑पश्यञ्चम॒सङ्घृ॒तस्य॑ पू॒र्ण स्व॒धाम्। तमुपोद॑तिष्ठ॒न्तम॑जुहवुः। तेनार्धमा॒स ऊर्ज॒मवा॑रुन्धत। तस्मा॑दर्धमा॒से दे॒वा इ॑ज्यन्ते। पि॒तरो॑ऽपश्यञ्चम॒सङ्घृ॒तस्य॑ पू॒र्ण स्व॒धाम्। तमुपोद॑तिष्ठ॒न्तम॑जुहवुः। तेन॑ मा॒स्यूर्ज॒मवा॑रुन्धत। तस्मान्मा॒सि पि॒तृभ्य॑ क्रियते। म॒नु॒ष्या॑ अपश्यञ्चम॒सङ्घृ॒तस्य॑ पू॒र्ण स्व॒धाम्॥५२॥

%1.4.9.2
तमुपोद॑तिष्ठ॒न्तम॑जुहवुः। तेन॑ द्व॒यीमूर्ज॒मवा॑रुन्धत। तस्मा॒द्द्विरह्नो॑ मनु॒ष्येभ्य॒ उप॑ह्रियते। प्रा॒तश्च॑ सा॒यं च॑। प॒शवो॑ऽपश्यञ्चम॒सङ्घृ॒तस्य॑ पू॒र्ण स्व॒धाम्। तमुपोद॑तिष्ठ॒न्त\-म॑जुहवुः। तेन॑ त्र॒यीमूर्ज॒मवा॑रुन्धत। तस्मा॒त्रिरह्न॑ प॒शव॒ प्रेर॑ते। प्रा॒तः स॑ङ्ग॒वे सा॒यम्। असु॑रा अपश्यञ्चम॒सङ्घृ॒तस्य॑ पू॒र्ण स्व॒धाम्॥५३॥

%1.4.9.3
तमुपोद॑तिष्ठ॒न्तम॑जुहवुः। तेन॑ संवत्स॒र ऊर्ज॒मवा॑रुन्धत। ते दे॒वा अ॑मन्यन्त। अ॒मी वा इ॒दम॑भूवन्। यद्व॒य स्म इति॑। त ए॒तानि॑ चातुर्मा॒स्यान्य॑पश्यन्। तानि॒ निर॑वपन्। तैरे॒वैषा॒न्तामूर्ज॑मवृञ्जत। ततो॑ दे॒वा अभ॑वन्। पराऽसु॑राः॥५४॥

%1.4.9.4
यद्यज॑ते। यामे॒व दे॒वा ऊर्ज॑म॒वारु॑न्धत। तान्तेनाव॑रुन्धे। यत्पि॒तृभ्य॑ क॒रोति॑। यामे॒व पि॒तर॒ ऊर्ज॑म॒वारु॑न्धत। तान्तेनाव॑रुन्धे। यदा॑वस॒थेऽन्न॒ हर॑न्ति। यामे॒व म॑नु॒ष्या॑ ऊर्ज॑म॒वारु॑न्धत। तान्तेनाव॑रुन्धे। यद्दक्षि॑णा॒न्ददा॑ति॥५५॥

%1.4.9.5
यामे॒व प॒शव॒ ऊर्ज॑म॒वारु॑न्धत। तान्तेनाव॑रुन्धे। यच्चा॑तुर्मा॒स्यैर्यज॑ते। यामे॒वासु॑रा॒ ऊर्ज॑म॒वारु॑न्धत। तान्तेनाव॑रुन्धे। भव॑त्या॒त्मना। परास्य॒ भ्रातृ॑व्यो भवति। वि॒राजो॒ वा ए॒षा विक्रान्तिः। यच्चा॑तुर्मा॒स्यानि॑। वै॒श्व॒दे॒वेना॒स्मिल्लोँ॒के प्रत्य॑तिष्ठत्। व॒रु॒ण॒प्र॒घा॒सैर॒न्तरि॑क्षे। सा॒क॒मे॒धैर॒मुष्मि॑ल्लोँ॒के। ए॒ष ह॒ त्वावैतत्सर्वं॑ भवति। य ए॒वं वि॒द्वाश्चा॑तुर्मा॒स्यैर्यज॑ते॥५६॥\anuvakamend[म॒नु॒ष्या॑ अपश्यञ्चम॒सङ्घृ॒तस्य॑ पू॒र्ण स्व॒धामसु॑रा अपश्यञ्चम॒सङ्घृ॒तस्य॑ पू॒र्ण स्व॒धामसु॑रा॒ ददात्यतिष्ठच्च॒त्वारि॑ च]

%1.4.10.1
अ॒ग्निर्वाव सं॑वत्स॒रः। आ॒दि॒त्यः प॑रिवत्स॒रः। च॒न्द्रमा॑ इदावत्स॒रः। वा॒युर॑नुवत्स॒रः। यद्वैश्वदे॒वेन॒ यज॑ते। अ॒ग्निमे॒व तत्सं॑वत्स॒रमाप्नोति। तस्माद्वैश्वदे॒वेन॒ यज॑मानः। सं॒व॒त्स॒रीणा स्व॒स्तिमाशास्त॒ इत्याशा॑सीत। यद्व॑रुण\-प्रघा॒सैर्यज॑ते। आ॒दि॒त्यमे॒व तत्प॑रिवत्स॒रमाप्नोति॥५७॥

%1.4.10.2
तस्माद्वरुणप्रघा॒सैर्यज॑मानः। प॒रि॒व॒त्स॒रीणा स्व॒स्तिमाशास्त॒ इत्याशा॑सीत। यत्सा॑कमे॒धैर्यज॑ते। च॒न्द्रम॑समे॒व तदि॑दावत्स॒र\-माप्नोति। तस्मात्साकमे॒धैर्यज॑मानः। इ॒दा॒व॒त्स॒रीणा स्व॒स्तिमाशास्त॒ इत्याशा॑सीत। यत्पि॑तृय॒ज्ञेन॒ यज॑ते। दे॒वाने॒व तद॒न्वव॑स्यति। अथ॒वा अ॑स्य वा॒युश्चा॑नुवत्स॒रश्चाप्री॑ता॒\-वुच्छि॑ष्येते। यच्छु॑नासी॒रीये॑ण॒ यज॑ते॥५८॥

%1.4.10.3
वा॒युमे॒व तद॑नुवत्स॒रमाप्नोति। तस्माच्छुनासी॒रीये॑ण॒ यज॑मानः। अ॒नु॒व॒त्स॒रीणा स्व॒स्तिमाशास्त॒ इत्याशा॑सीत। सं॒व॒त्स॒रं वा ए॒ष ईप्स॒तीत्या॑हुः। यश्चा॑तुर्मा॒स्यैर्यज॑त॒ इति॑। ए॒ष ह॒ त्वै सं॑वत्स॒रमाप्नोति। य ए॒वं वि॒द्वाश्चा॑तुर्मा॒स्यैर्यज॑ते। विश्वे॑ दे॒वाः सम॑यजन्त। तेऽग्निमे॒वाय॑जन्त। त ए॒तं लो॒कम॑जयन्॥५९॥

%1.4.10.4
यस्मि॑न्न॒ग्निः। यद्वैश्वदे॒वेन॒ यज॑ते। ए॒तमे॒व लो॒कं ज॑यति। यस्मि॑न्न॒ग्निः। अ॒ग्नेरे॒व सायु॑ज्य॒मुपै॑ति। य॒दा वैश्वदे॒वेन॒ यज॑ते। अथ॑ संवत्स॒रस्य॑ गृ॒हप॑तिमाप्नोति। य॒दा सं॑वत्स॒रस्य॑ गृ॒हप॑तिमा॒प्नोति॑। अथ॑ सहस्रया॒जिन॑माप्नोति। य॒दा स॑हस्रया॒जिन॑मा॒प्नोति॑॥६०॥

%1.4.10.5
अथ॑ गृहमे॒धिन॑माप्नोति। य॒दा गृ॑हमे॒धिन॑मा॒प्नोति॑। अथा॒ग्निर्भ॑वति। य॒दाग्निर्भव॑ति। अथ॒ गौर्भ॑वति। ए॒षा वै वैश्वदे॒वस्य॒ मात्रा। ए॒तद्वा ए॒तेषा॑मव॒मम्। अतो॑तो॒ वा उत्त॑राणि॒ श्रेयासि भवन्ति। यद्विश्वे॑ दे॒वाः स॒मय॑जन्त। तद्वैश्वदे॒वस्य॑ वैश्वदेव॒त्वम्॥६१॥

%1.4.10.6
अथा॑दि॒त्यो वरु॑ण॒ रा॑जानं वरुणप्रघा॒सैर॑यजत। स ए॒तं लो॒कम॑जयत्। यस्मि॑न्नादि॒त्यः। यद्व॑रुणप्रघा॒सैर्यज॑ते। ए॒तमे॒व लो॒कं ज॑यति। यस्मि॑न्नादि॒त्यः। आ॒दि॒त्यस्यै॒व सायु॑ज्य॒मुपै॑ति। यदा॑दि॒त्यो वरु॑ण॒ राजा॑नं वरुणप्रघा॒सै\-रय॑जत। तद्व॑रुणप्रघा॒सानां वरुणप्रघास॒त्वम्। अथ॒ सोमो॒ राजा॒ छन्दासि साकमे॒धैर॑यजत॥६२॥

%1.4.10.7
स ए॒तं लो॒कम॑जयत्। यस्मिश्च॒न्द्रमा॑ वि॒भाति॑। यत्सा॑कमे॒धैर्यज॑ते। ए॒तमे॒व लो॒कं ज॑यति। यस्मिश्च॒न्द्रमा॑ वि॒भाति॑। च॒न्द्रम॑स ए॒व सायु॑ज्य॒मुपै॑ति। सोमो॒ वै च॒न्द्रमा। ए॒ष ह॒ त्वै सा॒क्षात्सोमं॑ भक्षयति। य ए॒वं वि॒द्वान्त्सा॑कमे॒धैर्यज॑ते। यत्सोम॑श्च॒ राजा॒ छन्दा॑सि च स॒मैध॑न्त॥६३॥

%1.4.10.8
तत्सा॑कमे॒धाना साकमेध॒त्वम्। अथ॒र्तव॑ पि॒तर॑ प्र॒जाप॑तिं पि॒तरं॑ पितृय॒ज्ञेना॑यजन्त। त ए॒तं लो॒कम॑जयन्। यस्मि॑न्नृ॒तव॑। यत्पि॑तृय॒ज्ञेन॒ यज॑ते। ए॒तमे॒व लो॒कं ज॑यति। यस्मि॑न्नृ॒तव॑। ऋ॒तू॒नामे॒व सायु॑ज्य॒मुपै॑ति। यदृ॒तव॑ पि॒तर॑ प्र॒जाप॑तिं पि॒तरं॑ पितृय॒ज्ञेनाय॑जन्त। तत्पि॑तृय॒ज्ञस्य॑ पितृयज्ञ॒त्वम्॥६४॥

%1.4.10.9
अथौष॑धय इ॒मं दे॒वन्त्र्य॑म्बकैरयजन्त॒ प्रथे॑म॒हीति॑। ततो॒ वै ता अ॑प्रथन्त। य ए॒वं वि॒द्वास्त्र्य॑म्बकै॒र्यज॑ते। प्रथ॑ते प्र॒जया॑ प॒शुभि॑। अथ॑ वा॒युः प॑रमे॒ष्ठिन शुनासी॒रीये॑णायजत। स ए॒तं लो॒कम॑जयत्। यस्मि॑न्वा॒युः। यच्छु॑नासी॒रीये॑ण॒ यज॑ते। ए॒तमे॒व लो॒कं ज॑यति। यस्मि॑न्वा॒युः॥६५॥

%1.4.10.10
वा॒योरे॒व सायु॑ज्य॒मुपै॑ति। ब्र॒ह्म॒वा॒दिनो॑ वदन्ति। प्र चा॑तुर्मास्यया॒जी मी॑य॒ता (३) न प्रमी॑य॒ता (३) इति॑। जीव॒न्वा ए॒ष ऋ॒तूनप्ये॑ति। यदि॑ व॒सन्ता प्र॒मीय॑ते। व॒स॒न्तो भ॑वति। यदि॑ ग्री॒ष्मे ग्री॒ष्मः। यदि॑ व॒र्॒षासु॑ व॒र्॒षाः। यदि॑ श॒रदि॑ श॒रत्। यदि॒ हेम॑न् हेम॒न्तः। ऋ॒तुर्भू॒त्वा सं॑वत्स॒रमप्ये॑ति। सं॒व॒त्स॒रः प्र॒जाप॑तिः। प्र॒जाप॑ति॒र्वावैषः॥६६॥\anuvakamend[प॒रि॒व॒त्स॒रमाप्नोति शुनासी॒रीये॑ण॒ यज॑तेऽजयन्त्सहस्रया॒जिन॑मा॒प्नोति॑ वैश्वदेव॒त्व सा॑कमे॒धैर॑यजत स॒मैध॑न्त पितृयज्ञ॒त्वं ज॑यति॒ यस्मि॑न्वा॒युर्\mbox{}हे॑म॒न्तस्त्रीणि॑ च]






\prashnaend{उ॒भये॑ यु॒व सु॒राम॒मुद॑स्था॒न्नि वै यस्य॑ प्रातः सव॒न एकै॑कोऽसु॒र्यं॑ पव॑मानः प्र॒जा वै स॒त्रमा॑सता॒ग्निर्वाव सं॑वत्स॒रो दश॑॥१०॥}{उ॒भये॒ वा उद॑स्था॒त्सर्वा॑भिर्मध्य॒तोऽत्र॒ वाव ब्राह्म॒णेष्वथ॑ गृहमे॒धिन॒ षट्थ्ष॑ष्टिः॥६६॥}{उ॒भये॒ वा वैषः॥}{हरि॑ ओम्॥}{इति श्रीकृष्णयजुर्वेदीयतैत्तिरीयब्राह्मणे प्रथमाष्टके चतुर्थः प्रपाठकः समाप्तः॥}
\clearpage
\sect{पञ्चमः प्रश्नः}
\setcounter{anuvakam}{0}
\dnsub{तैत्तिरीयब्राह्मणे प्रथमाष्टके पञ्चमः प्रपाठकः}

%1.5.1.1
अ॒ग्नेः कृत्ति॑काः। शु॒क्रं प॒रस्ता॒ज्ज्योति॑र॒वस्तात्। प्र॒जाप॑ते रोहि॒णी। आप॑ प॒रस्ता॒दोष॑धयो॒ऽवस्तात्। सोम॑स्येन्व॒का वित॑तानि। प॒रस्ता॒द्वय॑न्तो॒ऽवस्तात्। रु॒द्रस्य॑ बा॒हू। मृ॒ग॒यव॑ प॒रस्ताद्विक्षा॒रो॑ऽवस्तात्। आदि॑त्यै॒ पुन॑र्वसू। वात॑ प॒रस्ता॑दा॒र्द्रम॒वस्तात्॥१॥

%1.5.1.2
बृह॒स्पतेस्ति॒ष्य॑। जुह्व॑तः प॒रस्ता॒द्यज॑माना अ॒वस्तात्। स॒र्पाणा॑माश्रे॒षाः। अ॒भ्या॒गच्छ॑न्तः प॒रस्ता॑दभ्या॒नृत्य॑न्तो॒\-ऽवस्तात्। पि॒तृ॒णां म॒घाः। रु॒दन्त॑ प॒रस्ता॑दपभ्र॒शो॑ऽवस्तात्। अ॒र्य॒म्णः पूर्वे॒ फल्गु॑नी। जा॒या प॒रस्ता॑दृष॒भो॑ऽवस्तात्। भग॒स्योत्त॑रे। व॒ह॒तव॑ प॒रस्ता॒द्वह॑माना अ॒वस्तात्॥२॥

%1.5.1.3
दे॒वस्य॑ सवि॒तुर्\mbox{}हस्त॑। प्र॒स॒वः प॒रस्तात्स॒निर॒वस्तात्। इन्द्र॑स्य चि॒त्रा। ऋ॒तं प॒रस्तात्स॒त्यम॒वस्तात्। वा॒योर्निष्ट्या व्र॒तति॑। प॒रस्ता॒दसि॑द्धिर॒वस्तात्। इ॒न्द्रा॒ग्नि॒योर्विशा॑खे। यु॒गानि॑ प॒रस्तात्कृ॒षमा॑णा अ॒वस्तात्। मि॒त्रस्या॑नूरा॒धाः। अ॒भ्या॒रोह॑त्प॒रस्ता॑द॒भ्यारू॑ढम॒वस्तात्॥३॥

%1.5.1.4
इन्द्र॑स्य रोहि॒णी। शृ॒णत्प॒रस्तात्प्रतिशृ॒णद॒वस्तात्। निर्\mbox{}ऋ॑त्यै मूल॒वर्\mbox{}ह॑णी। प्र॒ति॒भ॒ञ्जन्त॑ प॒रस्तात्प्रतिशृ॒णन्तो॒ऽवस्तात्। अ॒पां पूर्वा॑ अषा॒ढाः। वर्च॑ प॒रस्ता॒त्समि॑तिर॒वस्तात्। विश्वे॑षां दे॒वाना॒मुत्त॑राः। अ॒भि॒जय॑त्प॒रस्ता॑द॒भिजि॑तम॒वस्तात्। विष्णो श्रो॒णा पृ॒च्छमा॑नाः। प॒रस्ता॒त्पन्था॑ अ॒वस्तात्॥४॥

%1.5.1.5
वसू॑ना॒ श्रवि॑ष्ठाः। भू॒तं प॒रस्ता॒द्भूति॑र॒वस्तात्। इन्द्र॑स्य श॒तभि॑षक्। वि॒श्वव्य॑चाः प॒रस्ताद्वि॒श्वक्षि॑तिर॒वस्तात्। अ॒जस्यैक॑पद॒ पूर्वे प्रोष्ठप॒दाः। वै॒श्वा॒न॒रं प॒रस्ताद्वैश्वावस॒वम॒\-वस्तात्। अहेर्बु॒ध्निय॒स्योत्त॑रे। अ॒भि॒षि॒ञ्चन्त॑ प॒रस्ता॑दभि\-षु॒ण्वन्तो॒ऽवस्तात्। पू॒ष्णो रे॒वती। गाव॑ प॒रस्ताद्व॒त्सा अ॒वस्तात्। अ॒श्विनो॑रश्व॒युजौ। ग्राम॑ प॒रस्ता॒त्सेना॒ऽवस्तात्। य॒मस्या॑प॒भर॑णीः। अ॒प॒कर्\mbox{}ष॑न्तः प॒रस्ता॑दप॒वह॑न्तो॒ऽवस्तात्। पू॒र्णा प॒श्चाद्यत्ते॑ दे॒वा अद॑धुः॥५॥\anuvakamend[आ॒र्द्रम॒वस्ता॒द्वह॑माना अ॒वस्ता॑द॒भ्यारू॑ढम॒वस्ता॒त्पन्था॑ अ॒वस्ताद्व॒त्सा अ॒वस्ता॒त्पञ्च॑ च]

%1.5.2.1
यत्पुण्यं॒ नक्ष॑त्रम्। तद्बट्कु॑र्वीतोपव्यु॒षम्। य॒दा वै सूर्य॑ उ॒देति॑। अथ॒ नक्ष॑त्रं॒ नैति॑। याव॑ति॒ तत्र॒ सूर्यो॒ गच्छेत्। यत्र॑ जघ॒न्यं॑ पश्येत्। ताव॑ति कुर्वीत यत्का॒री स्यात्। पु॒ण्या॒ह ए॒व कु॑रुते। ए॒व ह॒ वै य॒ज्ञेषुं॑ च श॒तद्यु॑म्नं च मा॒त्स्यो नि॑रवसाय॒यां च॑कार॥६॥

%1.5.2.2
यो वै न॑क्ष॒त्रियं॑ प्र॒जाप॑तिं॒ वेद॑। उ॒भयो॑रेनं लो॒कयोर्विदुः। हस्त॑ ए॒वास्य॒ हस्त॑। चि॒त्रा शिर॑। निष्ट्या॒ हृद॑यम्। ऊ॒रू विशा॑खे। प्र॒ति॒ष्ठाऽनू॑रा॒धाः। ए॒ष वै न॑क्ष॒त्रिय॑ प्र॒जाप॑तिः। य ए॒वं वेद॑। उ॒भयो॑रेनं लो॒कयोर्विदुः॥७॥

%1.5.2.3
अ॒स्मिश्चा॒मुष्मिश्च। यां का॒मये॑त दुहि॒तरं॑ प्रि॒या स्या॒दिति॑। तां निष्ट्या॑यां दद्यात्। प्रि॒यैव भ॑वति। नेव॒ तु पुन॒राग॑च्छति। अ॒भि॒जिन्नाम॒ नक्ष॑त्रम्। उ॒परि॑ष्टादषा॒ढानाम्। अ॒वस्ताच्छ्रो॒णायै। दे॒वा॒सु॒राः संय॑त्ता आसन्। ते दे॒वास्तस्मि॒न्नक्ष॑त्रे॒ऽभ्य॑जयन्॥८॥

%1.5.2.4
यद॒भ्यज॑यन्। तद॑भि॒जितो॑ऽभिजि॒त्त्वम्। यं का॒मये॑तानपज॒य्यं ज॑ये॒दिति॑। तमे॒तस्मि॒न्नक्ष॑त्रे यातयेत्। अ॒न॒प॒ज॒य्यमे॒व ज॑यति। पा॒पप॑राजितमिव॒ तु। प्र॒जाप॑तिः प॒शून॑सृजत। ते नक्ष॑त्रं नक्षत्र॒मुपा॑तिष्ठन्त। ते स॒माव॑न्त ए॒वाभ॑वन्। ते रे॒वती॒मुपा॑तिष्ठन्त॥९॥

%1.5.2.5
ते रे॒वत्यां॒ प्राभ॑वन्। तस्माद्रे॒वत्यां पशू॒नां कु॑र्वीत। यत्किं चार्वा॒चीन॒ सोमात्। प्रैव भ॑वन्ति। स॒लि॒लं वा इ॒दम॑न्त॒रासीत्। यदत॑रन्। तत्तार॑काणां तारक॒त्वम्। यो वा इ॒ह यज॑ते। अ॒मु स लो॒कं न॑क्षते। तन्नक्ष॑त्राणां नक्षत्र॒त्वम्॥१०॥

%1.5.2.6
दे॒व॒गृ॒हा वै नक्ष॑त्राणि। य ए॒वं वेद॑। गृ॒ह्ये॑व भ॑वति। यानि॒ वा इ॒मानि॑ पृथि॒व्याश्चि॒त्राणि॑। तानि॒ नक्ष॑त्राणि। तस्मा॑दश्ली॒लना॑मश्चि॒त्रे। नाव॑स्ये॒न्न य॑जेत। यथा॑ पापा॒हे कु॑रु॒ते। ता॒दृगे॒व तत्। दे॒व॒न॒क्ष॒त्राणि॒ वा अ॒न्यानि॑॥११॥

%1.5.2.7
य॒म॒न॒क्ष॒त्राण्य॒न्यानि॑। कृत्ति॑काः प्रथ॒मम्। विशा॑खे उत्त॒मम्। तानि॑ देवनक्ष॒त्राणि॑। अ॒नू॒रा॒धाः प्र॑थ॒मम्। अ॒प॒भर॑णीरुत्त॒मम्। तानि॑ यमनक्ष॒त्राणि॑। यानि॑ देवनक्ष॒त्राणि॑। तानि॒ दक्षि॑णेन॒ परि॑यन्ति। यानि॑ यमनक्ष॒त्राणि॑॥१२॥

%1.5.2.8
तान्युत्त॑रेण। अन्वे॑षामरा॒त्स्मेति॑। तद॑नूरा॒धाः। ज्ये॒ष्ठमे॑षाम\-वधि॒ष्मेति॑। तज्ज्येष्ठ॒घ्नी। मूल॑मेषामवृक्षा॒मेति॑। तन्मू॑ल॒वर्\mbox{}ह॑णी। यन्नास॑हन्त। तद॑षा॒ढाः। यदश्लो॑णत्॥१३॥

%1.5.2.9
तच्छ्रो॒णा। यदशृ॑णोत्। तच्छ्रवि॑ष्ठाः। यच्छ॒तमभि॑षज्यन्। तच्छ॒तभि॑षक्। प्रो॒ष्ठ॒प॒देषूद॑यच्छन्त। रे॒वत्या॑मरवन्त। अ॒श्व॒युजो॑रयुञ्जत। अ॒प॒भर॑णी॒ष्वपा॑वहन्। तानि॒ वा ए॒तानि॑ यमनक्ष॒त्राणि॑। यान्ये॒व दे॑वनक्ष॒त्राणि॑। तेषु॑ कुर्वीत यत्का॒री स्यात्। पु॒ण्या॒ह ए॒व कु॑रुते॥१४॥\anuvakamend[च॒का॒रै॒वं वेदो॒भयो॑रेनं लो॒कयोर्विदुरजयन्रे॒वती॒मुपा॑तिष्ठन्त नक्षत्र॒त्वम॒न्यानि॒ यानि॑ यमनक्ष॒त्राण्यश्लो॑णद्यमनक्ष॒त्राणि॒ त्रीणि॑ च]

%1.5.3.1
दे॒वस्य॑ सवि॒तुः प्रा॒तः प्र॑स॒वः प्रा॒णः। वरु॑णस्य सा॒यमा॑स॒वो॑ऽपा॒नः। यत्प्र॑ती॒चीनं॑ प्रात॒स्तनात्। प्रा॒चीन सङ्ग॒वात्। ततो॑ दे॒वा अ॑ग्निष्टो॒मं निर॑मिमत। तत्तदात्त॑वीर्यं निर्मा॒र्गः। मि॒त्रस्य॑ सङ्ग॒वः। तत्पुण्यं॑ तेज॒स्व्यह॑। तस्मा॒त्तर्\mbox{}हि॑ प॒शव॑ स॒माय॑न्ति। यत्प्र॑ती॒चीन सङ्ग॒वात्॥१५॥

%1.5.3.2
प्रा॒चीनं॑ म॒ध्यं दि॑नात्। ततो॑ दे॒वा उ॒क्थ्यं॑ निर॑मिमत। तत्तदात्त॑वीर्यं निर्मा॒र्गः। बृह॒स्पतेर्म॒ध्यं दि॑नः। तत्पुण्यं॑ तेज॒स्व्यह॑। तस्मा॒त्तर्\mbox{}हि॒ तेक्ष्णि॑ष्ठं तपति। यत्प्र॑ती॒चीनं॑ म॒ध्यं दि॑नात्। प्रा॒चीन॑मपरा॒ह्णात्। ततो॑ दे॒वाः षो॑ड॒शिनं॒ निर॑मिमत। तत्तदात्त॑वीर्यं निर्मा॒र्गः॥१६॥

%1.5.3.3
भग॑स्यापरा॒ह्णः। तत्पुण्यं॑ तेज॒स्व्यह॑। तस्मा॑दपरा॒ह्णे कु॑मा॒र्यो॑ भग॑मि॒च्छमा॑नाश्चरन्ति। यत्प्र॑ती॒चीन॑मपरा॒ह्णात्। प्रा॒चीन सा॒यात्। ततो॑ दे॒वा अ॑तिरा॒त्रं निर॑मिमत। तत्तदात्त॑वीर्यं निर्मा॒र्गः। वरु॑णस्य सा॒यम्। तत्पुण्यं॑ तेज॒स्व्यह॑। तस्मा॒त्तर्\mbox{}हि॒ नानृ॑तं वदेत्॥१७॥

%1.5.3.4
ब्रा॒ह्म॒णो वा अ॑ष्टावि॒शो नक्ष॑त्राणाम्। स॒मा॒नस्याह्न॒ पञ्च॒ पुण्या॑नि॒ नक्ष॑त्राणि। च॒त्वार्य॑श्ली॒लानि॑। तानि॒ नव॑। यच्च॑ प॒रस्ता॒न्नक्ष॑त्राणां॒ यच्चा॒वस्तात्। तान्येका॑दश। ब्रा॒ह्म॒णो द्वा॑द॒शः। य ए॒वं वि॒द्वान्त्सं॑वत्स॒रं व्र॒तं चर॑ति। सं॒व॒त्स॒रेणै॒वास्य॑ व्र॒तं गु॒प्तं भ॑वति। स॒मा॒नस्याह्न॒ पञ्च॒ पुण्या॑नि॒ नक्ष॑त्राणि। च॒त्वार्य॑श्ली॒लानि॑। तानि॒ नव॑। आ॒ग्ने॒यी रात्रि॑। ऐ॒न्द्रमह॑। तान्येका॑दश। आ॒दि॒त्यो द्वा॑द॒शः। य ए॒वं वि॒द्वान्त्सं॑वत्स॒रं व्र॒तं चर॑ति। सं॒व॒त्स॒रेणै॒वास्य॑ व्र॒तं गु॒प्तं भ॑वति॥१८॥\anuvakamend[स॒ङ्ग॒वाथ्षो॑ड॒शिनं॒ निर॑मिमत॒ तत्तदात्त॑वीर्यं निर्मा॒र्गो व॑देद्भवति समा॒नस्याह्न॒ पञ्च॒ पुण्या॑नि॒ नक्ष॑त्राण्य॒ष्टौ च॑]

%1.5.4.1
ब्र॒ह्म॒वा॒दिनो॑ वदन्ति। कति॒ पात्रा॑णि य॒ज्ञं व॑ह॒न्तीति॑। त्रयो॑द॒शेति॑ ब्रूयात्। स यद्ब्रू॒यात्। कस्तानि॒ निर॑मिमी॒तेति॑। प्र॒जाप॑ति॒रिति॑ ब्रूयात्। स यद्ब्रू॒यात्। कुत॒स्तानि॒ निर॑मिमी॒तेति॑। आ॒त्मन॒ इति॑। प्रा॒णा॒पा॒नाभ्या॑मे॒वोपा\-श्वन्तर्या॒मौ निर॑मिमीत॥१९॥

%1.5.4.2
व्या॒नादु॑पाशु॒सव॑नम्। वा॒च ऐन्द्रवाय॒वम्। द॒क्ष॒क्र॒तुभ्यां मैत्रावरु॒णम्। श्रोत्रा॑दाश्वि॒नम्। चक्षु॑षः शु॒क्राम॒न्थिनौ। आ॒त्मन॑ आग्रय॒णम्। अङ्गेभ्य उ॒क्थ्यम्। आयु॑षो ध्रु॒वम्। प्र॒ति॒ष्ठाया॑ ऋतुपा॒त्रे। य॒ज्ञं वाव तं प्र॒जाप॑ति॒र्निर॑मिमीत। स निर्मि॑तो॒ नाद्ध्रि॑यत॒ सम॑व्लीयत। स ए॒तान्प्र॒जाप॑तिरपिवा॒पान॑पश्यत्। तां निर॑वपत्। तैर्वै स य॒ज्ञमप्य॑वपत्। यद॑पिवा॒पा भव॑न्ति। य॒ज्ञस्य॒ धृत्या॒ असंव्लयाय॥२०॥\anuvakamend[उ॒पा॒श्व॒न्त॒र्या॒मौ निर॑मिमीतामिमीत॒ षट्च॑]

%1.5.5.1
ऋ॒तमे॒व प॑रमे॒ष्ठि। ऋ॒तं नात्ये॑ति॒ किञ्च॒न। ऋ॒ते स॑मु॒द्र आहि॑तः। ऋ॒ते भूमि॑रि॒यश्रि॒ता। अ॒ग्निस्ति॒ग्मेन॑ शो॒चिषा। तप॒ आक्रान्तमु॒ष्णिहा। शिर॒स्तप॒स्याहि॑तम्। वै॒श्वा॒न॒रस्य॒ तेज॑सा। ऋ॒तेनास्य॒ नि व॑र्तये। स॒त्येन॒ परि॑ वर्तये। तप॑सा॒ऽस्यानु॑ वर्तये। शि॒वेना॒स्योप॑ वर्तये। श॒ग्मेनास्या॒भि व॑र्तये। तदृ॒तं तत्स॒त्यम्। तद्व्र॒तं तच्छ॑केयम्। तेन॑ शकेयं॒ तेन॑ राध्यासम्॥२१॥

%1.5.5.2
यद्घ॒र्मः प॒र्यव॑र्तयत्। अन्तान्पृथि॒व्या दि॒वः। अ॒ग्निरीशा॑न॒ ओज॑सा। वरु॑णो धी॒तिभि॑ स॒ह। इन्द्रो॑ म॒रुद्भि॒ सखि॑भिः स॒ह। अ॒ग्निस्ति॒ग्मेन॑ शो॒चिषा। तप॒ आक्रान्तमु॒ष्णिहा। शिर॒स्तप॒स्याहि॑तम्। वै॒श्वा॒न॒रस्य॒ तेज॑सा। ऋ॒तेनास्य॒ नि व॑र्तये। स॒त्येन॒ परि॑ वर्तये। तप॑सा॒ऽस्यानु॑ वर्तये। शि॒वेना॒स्योप॑ वर्तये। श॒ग्मेनास्या॒भि व॑र्तये। तदृ॒तं तत्स॒त्यम्। तद्व्र॒तं तच्छ॑केयम्। तेन॑ शकेयं॒ तेन॑ राध्यासम्॥२२॥

%1.5.5.3
यो अ॒स्याः पृ॑थि॒व्यास्त्व॒चि। नि॒व॒र्तय॒त्योष॑धीः। अ॒ग्निरीशा॑न॒ ओज॑सा। वरु॑णो धी॒तिभि॑ स॒ह। इन्द्रो॑ म॒रुद्भि॒ सखि॑भिः स॒ह। अ॒ग्निस्ति॒ग्मेन॑ शो॒चिषा। तप॒ आक्रान्तमु॒ष्णिहा। शिर॒स्तप॒स्याहि॑तम्। वै॒श्वा॒न॒रस्य॒ तेज॑सा। ऋ॒तेनास्य॒ नि व॑र्तये। स॒त्येन॒ परि॑ वर्तये। तप॑सा॒ऽस्यानु॑ वर्तये। शि॒वेना॒स्योप॑ वर्तये। श॒ग्मेनास्या॒भि व॑र्तये। तदृ॒तं तत्स॒त्यम्। तद्व्र॒तं तच्छ॑केयम्। तेन॑ शकेयं॒ तेन॑ राध्यासम्॥२३॥

%1.5.5.4
एकं॒ मास॒मुद॑सृजत्। प॒र॒मे॒ष्ठी प्र॒जाभ्य॑। तेनाभ्यो॒ मह॒ आव॑हत्। अ॒मृतं॒ मर्त्याभ्यः। प्र॒जामनु॒ प्र जा॑यसे। तदु॑ ते मर्त्या॒मृतम्। येन॒ मासा॑ अर्धमा॒साः। ऋ॒तव॑ परिवत्स॒राः। येन॒ ते ते प्रजापते। ई॒जा॒नस्य॒ न्यव॑र्तयन्। तेना॒हम॒स्य ब्रह्म॑णा। निव॑र्तयामि जी॒वसे। अ॒ग्निस्ति॒ग्मेन॑ शो॒चिषा। तप॒ आक्रान्तमु॒ष्णिहा। शिर॒स्तप॒स्याहि॑तम्। वै॒श्वा॒न॒रस्य॒ तेज॑सा। ऋ॒तेनास्य॒ नि व॑र्तये। स॒त्येन॒ परि॑ वर्तये। तप॑सा॒ऽस्यानु॑ वर्तये। शि॒वेना॒स्योप॑ वर्तये। श॒ग्मेनास्या॒भि व॑र्तये। तदृ॒तं तत्स॒त्यम्। तद्व्र॒तं तच्छ॑केयम्। तेन॑ शकेयं॒ तेन॑ राध्यासम्॥२४॥\anuvakamend[(परि॑वर्तये स॒हाभिव॑र्तय उ॒ष्णिहा॑ राध्यास॒न्न्यव॑र्तय॒न्नुप॑वर्तये च॒त्वारि॑ च)। ऋ॒तमे॒व षोड॑श। यद्घ॒र्मो यो अ॒स्याः सप्तद॑शसप्तदश। एकं॒ मासं॒ चतु॑र्विशतिः।]

%1.5.6.1
दे॒वा वै यद्य॒ज्ञेऽकु॑र्वत। तदसु॑रा अकुर्वत। तेऽसु॑रा ऊ॒र्ध्वं पृ॒ष्ठेभ्यो॒ नाप॑श्यन्। ते केशा॒नग्रे॑ऽवपन्त। अथ॒ श्मश्रू॑णि। अथो॑पप॒क्षौ। तत॒स्तेऽवाञ्च आयन्। परा॑ऽभवन्। यस्यै॒वं वप॑न्ति। अवा॑ङेति॥२५॥

%1.5.6.2
अथो॒ परै॒व भ॑वति। अथ॑ दे॒वा ऊ॒र्ध्वं पृ॒ष्ठेभ्यो॑ऽपश्यन्। त उ॑पप॒क्षावग्रे॑ऽवपन्त। अथ॒ श्मश्रू॑णि। अथ॒ केशान्॑। तत॒स्ते॑ऽभवन्। सु॒व॒र्गं लो॒कमा॑यन्। यस्यै॒वं वप॑न्ति। भव॑त्या॒त्मना। अथो॑ सुव॒र्गं लो॒कमे॑ति॥२६॥

%1.5.6.3
अथै॒तन्मनु॑र्व॒प्त्रे मि॑थु॒नम॑पश्यत्। स श्मश्रू॒ण्यग्रे॑ऽवपत। अथो॑पप॒क्षौ। अथ॒ केशान्॑। ततो॒ वै स प्राजा॑यत प्र॒जया॑ प॒शुभि॑। यस्यै॒वं वप॑न्ति। प्र प्र॒जया॑ प॒शुभि॑र्मिथु॒नैर्जा॑यते। दे॒वा॒सु॒राः संय॑त्ता आसन्। ते सं॑वत्स॒रे व्याय॑च्छन्त। तान्दे॒वाश्चा॑तुर्मा॒स्यैरे॒वाभि प्रायु॑ञ्जत॥२७॥

%1.5.6.4
वै॒श्व॒दे॒वेन॑ च॒तुरो॑ मा॒सो॑ऽवृञ्ज॒तेन्द्र॑राजानः। ताञ्छी॒र्॒षं नि चाव॑र्तयन्त॒ परि॑ च। व॒रु॒ण॒प्र॒घा॒सैश्च॒तुरो॑ मा॒सो॑ऽवृञ्जत॒ वरु॑णराजानः। ताञ्छी॒र्॒षं नि चाव॑र्तयन्त॒ परि॑ च। सा॒क॒मे॒धैश्च॒तुरो॑ मा॒सो॑ऽवृञ्जत॒ सोम॑राजानः। ताञ्छी॒र्॒षं नि चाव॑र्तयन्त॒ परि॑ च। या सं॑वत्स॒र उ॑पजी॒वाऽऽसीत्। तामे॑षामवृञ्जत। ततो॑ दे॒वा अभ॑वन्। पराऽसु॑राः॥२८॥

%1.5.6.5
य ए॒वं वि॒द्वाश्चा॑तुर्मा॒स्यैर्यज॑ते। भ्रातृ॑व्यस्यै॒व मा॒सो वृ॒क्त्वा। शी॒र्॒षं नि च॑ व॒र्तय॑ते॒ परि॑ च। यैषा सं॑वत्स॒र उ॑पजी॒वा। वृ॒ङ्क्ते तां भ्रातृ॑व्यस्य। क्षु॒धाऽस्य॒ भ्रातृ॑व्य॒ परा॑ भवति। लो॒हि॒ता॒य॒सेन॒ नि व॑र्तयते। यद्वा इ॒माम॒ग्निर्\mbox{}ऋ॒तावाग॑ते निव॒र्तय॑ति। ए॒तदे॒वैना रू॒पं कृ॒त्वा निव॑र्तयति। सा तत॒ श्वश्वो॒ भूय॑सी॒ भव॑न्त्येति॥२९॥

%1.5.6.6
प्र जा॑यते। य ए॒वं वि॒द्वाल्लोँ॑हिताय॒सेन॑ निव॒र्तय॑ते। ए॒तदे॒व रू॒पं कृ॒त्वा नि व॑र्तयते। स तत॒ श्वश्वो॒ भूया॒न्भव॑न्नेति। प्रैव जा॑यते। त्रे॒ण्या श॑ल॒ल्या नि व॑र्तयेत। त्रीणि॑त्रीणि॒ वै दे॒वाना॑मृ॒द्धानि॑। त्रीणि॒ छन्दासि। त्रीणि॒ सव॑नानि। त्रय॑ इ॒मे लो॒काः॥३०॥

%1.5.6.7
ऋ॒ध्यामे॒व तद्वी॒र्य॑ ए॒षु लो॒केषु॒ प्रति॑ तिष्ठति। यच्चा॑तुर्मास्यया॒ज्यात्मनो॒ नाव॒द्येत्। दे॒वेभ्य॒ आवृ॑श्च्येत। च॒तृ॒षुच॑तृषु॒ मासे॑षु॒ नि व॑र्तयेत। प॒रोक्ष॑मे॒व तद्दे॒वेभ्य॑ आ॒त्मनोऽव॑द्य॒त्यनाव्रस्काय। दे॒वानां॒ वा ए॒ष आनी॑तः। यश्चा॑तुर्मास्यया॒जी। य ए॒वं वि॒द्वान्नि च॑ व॒र्तय॑ते॒ परि॑ च। दे॒वता॑ ए॒वाप्ये॑ति। नास्य॑ रु॒द्रः प्र॒जां प॒शून॒भि म॑न्यते॥३१॥\anuvakamend[ए॒त्ये॒त्य॒यु॒ञ्ज॒तासु॑रा एति लो॒का म॑न्यते]

%1.5.7.1
आयु॑षः प्रा॒ण सन्त॑नु। प्रा॒णाद॑पा॒न सन्त॑नु। अ॒पा॒नाद्व्या॒न सन्त॑नु। व्या॒नाच्चक्षु॒ सन्त॑नु। चक्षु॑ष॒ श्रोत्र॒ सन्त॑नु। श्रोत्रा॒न्मन॒ सन्त॑नु। मन॑सो॒ वाच॒ सन्त॑नु। वा॒च आ॒त्मान॒ सन्त॑नु। आ॒त्मन॑ पृथि॒वी सन्त॑नु। पृ॒थि॒व्या अ॒न्तरि॑क्ष॒ सन्त॑नु। अ॒न्तरि॑क्षा॒द्दिव॒ सन्त॑नु। दिव॒ सुव॒ सन्त॑नु॥३२॥\anuvakamend[अ॒न्तरि॑क्ष॒ सन्त॑नु॒ द्वे च॑]

%1.5.8.1
इन्द्रो॑ दधी॒चो अ॒स्थभि॑। वृ॒त्राण्यप्र॑तिष्कुतः। ज॒घान॑ नव॒तीर्नव॑। इ॒च्छन्नश्व॑स्य॒ यच्छिर॑। पर्व॑ते॒ष्वप॑श्रितम्। तद्वि॑दच्छर्य॒णाव॑ति। अत्राह॒ गोरम॑न्वत। नाम॒ त्वष्टु॑रपी॒च्यम्। इ॒त्था च॒न्द्रम॑सो गृ॒हे। इन्द्र॒मिद्गा॒थिनो॑ बृ॒हत्॥३३॥

%1.5.8.2
इन्द्र॑म॒र्केभि॑र॒र्किण॑। इन्द्रं॒ वाणी॑रनूषत। इन्द्र॒ इद्धर्यो॒ सचा। सम्मि॑श्ल॒ आव॑चो॒ युजा। इन्द्रो॑ व॒ज्री हि॑र॒ण्यय॑। इन्द्रो॑ दी॒र्घाय॒ चक्ष॑से। आ सूर्य रोहयद्दि॒वि। वि गोभि॒रद्रि॑मैरयत्। इन्द्र॒ वाजे॑षु नो अव। स॒हस्र॑प्रधनेषु च॥३४॥

%1.5.8.3
उ॒ग्र उ॒ग्राभि॑रू॒तिभि॑। तमिन्द्रं॑ वाजयामसि। म॒हे वृ॒त्राय॒ हन्त॑वे। स वृषा॑ वृष॒भो भु॑वत्। इन्द्र॒ स दाम॑ने कृ॒तः। ओजि॑ष्ठ॒ स बले॑ हि॒तः। द्यु॒म्नी श्लो॒की स सौ॒म्य॑। गि॒रा वज्रो॒ न सम्भृ॑तः। सब॑लो॒ अन॑पच्युतः। व॒व॒क्षुरु॒ग्रो अस्तृ॑तः॥३५॥\anuvakamend[बृ॒हच्चास्तृ॑तः]

%1.5.9.1
दे॒वा॒सु॒राः संय॑त्ता आसन्। स प्र॒जाप॑ति॒रिन्द्रं॑ ज्ये॒ष्ठं पु॒त्रमप॒ न्य॑धत्त। नेदे॑न॒मसु॑रा॒ बली॑यासोऽहन॒न्निति॑। प्र॒ह्रादो॑ ह॒ वै का॑याध॒वः। वि॒रोच॑न॒ स्वं पु॒त्रमप॒ न्य॑धत्त। नेदे॑नं दे॒वा अ॑हन॒न्निति॑। ते दे॒वाः प्र॒जाप॑तिमुपस॒मेत्यो॑चुः। नारा॒जक॑स्य यु॒द्धम॑स्ति। इन्द्र॒मन्वि॑च्छा॒मेति॑। तं॒ य॑ज्ञक्र॒तुभि॒रन्वैच्छन्॥३६॥

%1.5.9.2
तं॒ य॑ज्ञक्र॒तुभि॒र्नान्व॑विन्दन्। तमिष्टि॑भि॒रन्वैच्छन्। तमिष्टि॑भि॒रन्व॑विन्दन्। तदिष्टी॑नामिष्टि॒त्वम्। एष्ट॑यो ह॒ वै नाम॑। ता इष्ट॑य॒ इत्याच॑क्षते प॒रोक्षे॑ण। प॒रोक्ष॑प्रिया इव॒ हि दे॒वाः। तस्मा॑ ए॒तमाग्नावैष्ण॒वमेका॑दशकपालं दीक्ष॒णीयं॒ निर॑वपन्। तद॑प॒द्रुत्या॑तन्वत। तान्प॑त्नीसंया॒जान्त॒ उपा॑नयन्॥३७॥

%1.5.9.3
ते तद॑न्तमे॒व कृ॒त्वोद॑द्रवन्। ते प्रा॑य॒णीय॑म॒भि स॒मारो॑हन्। तद॑प॒द्रुत्या॑तन्वत। ताञ्छ॒य्य्वँ॑न्त॒ उपा॑नयन्। ते तद॑न्तमे॒व कृ॒त्वोद॑द्रवन्। त आ॑ति॒थ्यम॒भि स॒मारो॑हन्। तद॑प॒द्रुत्या॑तन्वत। तानिडान्त॒ उपा॑नयन्। ते तद॑न्तमे॒व कृ॒त्वोद॑द्रवन्। तस्मा॑दे॒ता ए॒तद॑न्ता॒ इष्ट॑य॒ सन्ति॑ष्ठन्ते॥३८॥

%1.5.9.4
ए॒व हि दे॒वा अकु॑र्वत। इति॑ दे॒वा अ॑कुर्वत। इत्यु॒ वै म॑नु॒ष्या कुर्वते। ते दे॒वा ऊ॑चुः। यद्वा इ॒दमु॒च्चैर्य॒ज्ञेन॒ चरा॑म। तन्नोऽसु॑राः पा॒प्माऽनु॑विन्दन्ति। उ॒पा॒शू॑प॒सदा॑ चराम। तथा॒ नोऽसु॑राः पा॒प्मा नानु॑वेत्स्य॒न्तीति॑। त उ॑पा॒शू॑प॒सद॑मतन्वत। ति॒स्र ए॒व सा॑मिधे॒नीर॒नूच्य॑॥३९॥

%1.5.9.5
स्रु॒वेणा॑घा॒रमा॒घार्य॑। ति॒स्रः परा॑ची॒राहु॑तीर्\mbox{}हु॒त्वा। स्रु॒वेणो॑प॒सदं॑ जुह॒वां च॑क्रुः। उ॒ग्रं वचो॒ अपा॑वधीन्त्वे॒षं वचो॒ अपा॑वधी॒स्वाहेति॑। अ॒श॒न॒या॒पि॒पा॒से ह॒ वा उ॒ग्रं वच॑। एन॑श्च॒ वैर॑हत्यं च त्वे॒षं वच॑। ए॒त ह॒ वाव तच्च॑तुर्धाविहि॒तं पा॒प्मानं॑ दे॒वा अप॑जघ्निरे। तथो॑ ए॒वैतदे॑वं॒विद्यज॑मानः। ति॒स्र ए॒व सा॑मिधे॒नीर॒नूच्य॑। स्रु॒वेणा॑घा॒रमा॒घार्य॑॥४०॥

%1.5.9.6
ति॒स्रः परा॑ची॒राहु॑तीर्\mbox{}हु॒त्वा। स्रु॒वेणो॑प॒सदं॑ जुहोति। उ॒ग्रं वचो॒ अपा॑वधीन्त्वे॒षं वचो॒ अपा॑वधी॒ स्वाहेति॑। अ॒श॒न॒या॒पि॒पा॒से ह॒ वा उ॒ग्रं वच॑। एन॑श्च॒ वैर॑हत्यं च त्वे॒षं वच॑। ए॒तमे॒व तच्च॑तुर्धाविहि॒तं पा॒प्मानं॒ यज॑मा॒नोऽप॑ हते। ते॑ऽभि॒नीयै॒वाह॑ प॒शुमाऽल॑भन्त। अह्न॑ ए॒व तद्दे॒वा अव॑र्तिं पा॒प्मानं॑ मृ॒त्युमप॑जघ्निरे। तेना॑भि॒नीये॑व॒ रात्रे॒ प्राच॑रन्। रात्रि॑या ए॒व तद्दे॒वा अव॑र्तिं पा॒प्मानं॑ मृ॒त्युमप॑जघ्निरे॥४१॥

%1.5.9.7
तस्मा॑दभि॒नीयै॒वाह॑ प॒शुमा ल॑भेत। अह्न॑ ए॒व तद्यज॑मा॒नोऽव॑र्तिं पा॒प्मानं॒ भ्रातृ॑व्या॒नप॑ नुदते। तेना॑भि॒नीये॑व॒ रात्रे॒ प्रच॑रेत्। रात्रि॑या ए॒व तद्यज॑मा॒नोऽव॑र्तिं पा॒प्मानं॒ भ्रातृ॑व्या॒नप॑ नुदते। स ए॒ष उ॑पवस॒थीयेऽह॑न्द्विदेव॒त्य॑ प॒शुरा ल॑भ्यते। द्व॒यं वा अ॒स्मिल्लोँ॒के यज॑मानः। अस्थि॑ च मा॒सं च॑। अस्थि॑ चै॒व तेन॑ मा॒सं च॒ यज॑मान॒ सस्कु॑रुते। ता वा ए॒ताः पञ्च॑ दे॒वता। अ॒ग्नीषोमा॑व॒ग्निर्मि॒त्रावरु॑णौ॥४२॥

%1.5.9.8
प॒ञ्च॒प॒ञ्ची वै यज॑मानः। त्वङ्मा॒स स्नावाऽस्थि॑ म॒ज्जा। ए॒तमे॒व तत्प॑ञ्चधाविहि॒तमा॒त्मानं॑ वरुणपा॒शान्मु॑ञ्चति। भे॒ष॒जता॑यै निर्वरुण॒त्वाय॑। त स॒प्तभि॒श्छन्दो॑भिः प्रा॒तर॑ह्वयन्। तस्मात्स॒प्त च॑तुरुत्त॒राणि॒ छन्दासि प्रातरनुवा॒केऽनूच्यन्ते। तमे॒तयो॑पस॒मेत्योपा॑सीदन्। उपास्मै गायता नर॒ इति॑। तस्मा॑दे॒तया॑ बहिष्पवमा॒न उ॑प॒सद्य॑॥४३॥\anuvakamend[ऐ॒च्छ॒न्न॒न॒य॒स्ति॒ष्ठ॒न्ते॒ऽनूच्या॒नूच्य॑ स्रु॒वेणा॑घा॒रमा॒घार्य॒ रात्रि॑या ए॒व तद्दे॒वा अव॑र्तिं पा॒प्मानं॑ मृ॒त्युमप॑जिघ्निरे मि॒त्रावरु॑णौ॒ नव॑ च (दे॒वा यज॑मानो दे॒वा दे॒वा यज॑मानो॒ यज॑मानोऽलभन्त॒ प्राच॑रल्लँभेत॒ प्रच॑रे॒दाल॑भ॒न्ताल॑भेत मृ॒त्युमप॑जघ्निरे॒ भ्रातृ॑व्यान्॥)]

%1.5.10.1
स स॑मु॒द्र उ॑त्तर॒तः प्राज्व॑लद्भूम्य॒न्तेन॑। ए॒ष वाव स स॑मु॒द्रः। यच्चात्वा॑लः। ए॒ष उ॑वे॒व स भूम्य॒न्तः। यद्वेद्य॒न्तः। तदे॒तत्त्रि॑श॒लन्त्रि॑पूरु॒षम्। तस्मा॒त्तं त्रि॑वित॒ तं ख॑नन्ति। स सु॑वर्णरज॒ताभ्यां कु॒शीभ्यां॒ परि॑गृहीत आसीत्। तं यद॒स्या अध्य॒जन॑यन्। तस्मा॑दादि॒त्यः॥४४॥

%1.5.10.2
अथ॒ यत्सु॑वर्णरज॒ताभ्यां कु॒शीभ्यां॒ परि॑गृहीत॒ आसीत्। साऽस्य॑ कौशि॒कता। तं त्रि॒वृता॒ऽभि प्रास्तु॑वत। तं त्रि॒वृताऽद॑दत। तं त्रि॒वृताऽह॑रन्। याव॑ती त्रि॒वृतो॒ मात्रा। तं प॑ञ्चद॒शेना॒भि प्रास्तु॑वत। तं प॑ञ्चद॒शेनाद॑दत। तं प॑ञ्चद॒शेनाह॑रन्। याव॑ती पञ्चद॒शस्य॒ मात्रा॥४५॥

%1.5.10.3
त स॑प्तद॒शेना॒भि प्रास्तु॑वत। त स॑प्तद॒शेनाद॑दत। त स॑प्तद॒शेनाह॑रन्। याव॑ती सप्तद॒शस्य॒ मात्रा। तस्य॑ सप्तद॒शेन॑ ह्रि॒यमा॑णस्य॒ तेजो॒ हरो॑ऽपतत्। तमे॑कवि॒शेना॒भि प्रास्तु॑वत। तमे॑कवि॒शेनाद॑दत। तमे॑कवि॒शेनाह॑रन्। याव॑त्येकवि॒शस्य॒ मात्रा। ते यत्त्रि॒वृता स्तु॒वते॥४६॥

%1.5.10.4
त्रि॒वृतै॒व तद्यज॑मान॒माद॑दते। तन्त्रि॒वृतै॒व ह॑रन्ति। याव॑ती त्रि॒वृतो॒ मात्रा। अ॒ग्निर्वै त्रि॒वृत्। याव॒द्वा अ॒ग्नेर्दह॑तो धू॒म उ॒देत्यानु॒ व्येति॑। ताव॑ती त्रि॒वृतो॒ मात्रा। अ॒ग्नेरे॒वैनं॒ तत्। मात्रा॒ सायु॑ज्य सलो॒कतां गमयन्ति। अथ॒ यत्प॑ञ्चद॒शेन॑ स्तु॒वते। प॒ञ्च॒द॒शेनै॒व तद्यज॑मान॒माद॑दते॥४७॥

%1.5.10.5
तं प॑ञ्चद॒शेनै॒व ह॑रन्ति। याव॑ती पञ्चद॒शस्य॒ मात्रा। च॒न्द्रमा॒ वै प॑ञ्चद॒शः। ए॒ष हि प॑ञ्चद॒श्याम॑पक्षी॒यते। प॒ञ्च॒द॒श्यामा॑पू॒र्यते। च॒न्द्रम॑स ए॒वैनं॒ तत्। मात्रा॒ सायु॑ज्य सलो॒कतां गमयन्ति। अथ॒ यत्स॑प्तद॒शेन॑ स्तु॒वते। स॒प्त॒द॒शेनै॒व तद्यज॑मान॒माद॑दते। त स॑प्तद॒शेनै॒व ह॑रन्ति॥४८॥

%1.5.10.6
याव॑ती सप्तद॒शस्य॒ मात्रा। प्र॒जाप॑ति॒र्वै स॑प्तद॒शः। प्र॒जाप॑तेरे॒वैनं॒ तत्। मात्रा॒ सायु॑ज्य सलो॒कतां गमयन्ति। अथ॒ यदे॑कवि॒शेन॑ स्तु॒वते। ए॒क॒वि॒शेनै॒व तद्यज॑मान॒माद॑दते। तमे॑कवि॒शेनै॒व ह॑रन्ति। याव॑त्येकवि॒शस्य॒ मात्रा। अ॒सौ वा आ॑दि॒त्य ए॑कवि॒शः। आ॒दि॒त्यस्यै॒वैनं॒ तत्॥४९॥

%1.5.10.7
मात्रा॒ सायु॑ज्य सलो॒कतां गमयन्ति। ते कु॒श्यौ। व्य॑घ्नन्। ते अ॑होरा॒त्रे अ॑भवताम्। अह॑रे॒व सु॒वर्णा॑ऽभवत्। र॒ज॒ता रात्रि॑। स यदा॑दि॒त्य उ॒देति॑। ए॒तामे॒व तत्सु॒वर्णां कु॒शीमनु॒ समे॑ति। अथ॒ यद॑स्त॒मेति॑। ए॒तामे॒व तद्र॑ज॒तां कु॒शीमनु॒संवि॑शति। प्र॒ह्रादो॑ ह॒ वै का॑याध॒वः। वि॒रोच॑न॒ स्वं पु॒त्रमुदास्यत्। स प्र॑द॒रो॑ऽभवत्। तस्मात्प्रद॒रादु॑द॒कं नाचा॑मेत्॥५०॥\anuvakamend[आ॒दि॒त्यः प॑ञ्चद॒शस्य॒ मात्रा स्तु॒वते॑ पञ्चद॒शेनै॒व तद्यज॑मान॒माद॑दते सप्तद॒शेनै॒व ह॑रन्त्यादि॒त्यस्यै॒वैनं॒ तद्वि॑शति च॒त्वारि॑ च]

%1.5.11.1
ये वै च॒त्वार॒ स्तोमा। कृ॒तन्तत्। अथ॒ ये पञ्च॑। कलि॒ सः। तस्मा॒च्चतु॑ष्टोमः। तच्चतु॑ष्टोमस्य चतुष्टोम॒त्वम्। तदा॑हुः। क॒त॒मानि॒ तानि॒ ज्योतीषि। य ए॒तस्य॒ स्तोमा॒ इति॑। त्रि॒वृत्प॑ञ्चद॒शः स॑प्तद॒श ए॑कवि॒शः॥५१॥

%1.5.11.2
ए॒तानि॒ वाव तानि॒ ज्योतीषि। य ए॒तस्य॒ स्तोमा। सोऽब्रवीत्। स॒प्त॒द॒शेन॑ ह्रि॒यमा॑णो॒ व्य॑लेशिषि। भि॒षज्य॑त॒ मेति॑। तम॒श्विनौ॑ धा॒नाभि॑रभिषज्यताम्। पू॒षा क॑र॒म्भेण॑। भार॑ती परिवा॒पेण॑। मि॒त्रावरु॑णौ पय॒स्य॑या। तदा॑हुः॥५२॥

%1.5.11.3
यद॒श्विभ्यान्धा॒नाः। पू॒ष्णः क॑र॒म्भः। भार॑त्यै परिवा॒पः। मि॒त्रावरु॑णयोः पय॒स्याऽथ॑। कस्मा॑दे॒तेषा ह॒विषा॒मिन्द्र॑मे॒व य॑ज॒न्तीति॑। ए॒ता ह्ये॑नं दे॒वता॒ इति॑ ब्रूयात्। ए॒तैर्\mbox{}ह॒विर्भि॒\-रभि॑षज्य॒स्तस्मा॒दिति॑। तं वस॑वो॒ऽष्टाक॑पालेन प्रातः सव॒ने॑ऽभिषज्यन्। रु॒द्रा एका॑दशकपालेन॒ माध्य॑न्दिने॒ सव॑ने। विश्वे॑ दे॒वा द्वाद॑शकपालेन तृतीयसव॒ने॥५३॥

%1.5.11.4
स यद॒ष्टाक॑पालान्प्रातः सव॒ने कु॒र्यात्। एका॑दशकपाला॒न्माध्यं॑ दिने॒ सव॑ने। द्वाद॑शकपालास्तृतीयसव॒ने। विलो॑म॒ तद्य॒ज्ञस्य॑ क्रियेत। एका॑दशकपालाने॒व प्रा॑तः सव॒ने कु॑र्यात्। एका॑दशकपाला॒न्माध्य॑न्दिने॒ सव॑ने। एका॑दशकपालास्तृतीयसव॒ने। य॒ज्ञस्य॑ सलोम॒त्वाय॑। तदा॑हुः। यद्वसू॑नां प्रातः सव॒नम्। रु॒द्राणां॒ माध्य॑न्दिन॒ सव॑नम्। विश्वे॑षां दे॒वानां तृतीयसव॒नम्। अथ॒ कस्मा॑दे॒तेषा ह॒विषा॒मिन्द्र॑मे॒व य॑ज॒न्तीति॑। ए॒ता ह्ये॑नं दे॒वता॒ इति॑ ब्रूयात्। ए॒तैर्\mbox{}ह॒विर्भि॒रभि॑षज्य॒ स्तस्मा॒दिति॑॥५४॥\anuvakamend[ए॒क॒वि॒श आ॑हुस्तृतीयसव॒ने प्रा॑तः सव॒नं पञ्च॑ च]

%1.5.12.1
तस्यावा॑चोऽवपा॒दाद॑बिभयुः। तमे॒तेषु॑ स॒प्तसु॒ छन्द॑ स्वश्रयन्। यदश्र॑यन्। तच्छ्रा॑य॒न्तीय॑स्य श्रायन्तीय॒त्वम्। यदवा॑रयन्। तद्वा॑रव॒न्तीय॑स्य वारवन्तीय॒त्वम्। तस्यावा॑च ए॒वाव॑पा॒दाद॑बिभयुः। तस्मा॑ ए॒तानि॑ स॒प्त च॑तुरुत्त॒राणि॒ छन्दा॒स्युपा॑दधुः। तेषा॒मति॒ त्रीण्य॑रिच्यन्त। न त्रीण्युद॑भवन्॥५५॥

%1.5.12.2
स बृ॑ह॒तीमे॒वास्पृ॑शत्। द्वाभ्या॑म॒क्षराभ्याम्। अ॒हो॒रा॒त्राभ्या॑मे॒व। तदा॑हुः। क॒त॒मा सा दे॒वाक्ष॑रा बृह॒ती। यस्या॒न्तत्प्र॒त्यति॑ष्ठत्। द्वाद॑श पौर्णमा॒स्य॑। द्वाद॒शाष्ट॑काः। द्वाद॑शामावा॒स्या। ए॒षा वाव सा दे॒वाक्ष॑रा बृह॒ती॥५६॥

%1.5.12.3
यस्या॒न्तत्प्र॒त्यति॑ष्ठ॒दिति॑। यानि॑ च॒ छन्दास्य॒त्यरि॑च्यन्त। यानि॑ च॒ नोदभ॑वन्। तानि॒ निर्वीर्याणि ही॒नान्य॑मन्यन्त। साऽब्र॑वीद्बृह॒ती। मामे॒व भू॒त्वा। मामुप॒ सश्र॑य॒तेति॑। च॒तुर्भि॑र॒क्षरै॑रनु॒ष्टुग्बृ॑ह॒तीन्नोद॑भवत्। च॒तुर्भि॑र॒क्षरै प॒ङ्क्तिर्बृ॑ह॒ती\-मत्य॑रिच्यत। तस्या॑मे॒तानि॑ च॒त्वार्य॒क्षराण्यप॒च्छिद्या॑\-दधात्॥५७॥

%1.5.12.4
ते बृ॑ह॒ती ए॒व भू॒त्वा। बृ॒ह॒तीमुप॒ सम॑श्रयताम्। अ॒ष्टा॒भि\-र॒क्षरै॑रु॒ष्णिग्बृ॑ह॒तीन्नोद॑भवत्। अ॒ष्टा॒भि\-र॒क्षरैस्त्रि॒ष्टुग्बृ॑ह॒ती\-मत्य॑\-रिच्यत। तस्या॑मे॒तान्य॒ष्टाव॒क्षराण्यप॒च्छिद्या॑\-दधात्। ते बृ॑ह॒ती ए॒व भू॒त्वा। बृ॒ह॒तीमुप॒ सम॑श्रयताम्। द्वा॒द॒शभि॑र॒क्षरैर्गाय॒त्री बृ॑ह॒तीन्नोद॑भवत्। द्वा॒द॒शभि॑र॒क्षरै॒र्जग॑ती बृह॒तीमत्य॑रिच्यत। तस्या॑मे॒तानि॒ द्वाद॑शा॒क्षराण्यप॒च्छिद्या॑\-दधात्॥५८॥

%1.5.12.5
ते बृ॑ह॒ती ए॒व भू॒त्वा। बृ॒ह॒तीमुप॒ सम॑श्रयताम्। सोऽब्रवीत्प्र॒जाप॑तिः। छन्दासि॒ रथो॑ मे भवत। यु॒ष्माभि॑र॒हमे॒तमध्वा॑न॒मनु॒ सञ्च॑रा॒णीति॑। तस्य॑ गाय॒त्री च॒ जग॑ती च प॒क्षाव॑भवताम्। उ॒ष्णिक्च॑ त्रि॒ष्टुप्च॒ प्रष्ट्यौ। अ॒नु॒ष्टुप्च॑ प॒ङ्क्तिश्च॒ धुर्यौ। बृ॒ह॒त्ये॑वोद्धिर॑भवत्। स ए॒तञ्छ॑न्दोर॒थमा॒स्थाय॑। ए॒तमध्वा॑न॒मनु॒ सम॑चरत्। ए॒त ह॒ वै छ॑न्दोर॒थमा॒स्थाय॑। ए॒तमध्वा॑न॒मनु॒ सञ्च॑रति। येनै॒ष ए॒तत्स॒ञ्चर॑ति। य ए॒वं वि॒द्वान्त्सोमे॑न॒ यज॑ते। य उ॑ चैनमे॒वं वेद॑॥५९॥\anuvakamend[अ॒भ॒व॒न्वाव सा दे॒वाक्ष॑रा बृह॒त्य॑दधा॒द्द्वाद॑शा॒क्षराण्यप॒च्छिद्या॑दधादा॒स्थाय॒ षट्च॑]






\prashnaend{अ॒ग्नेः कृत्ति॑का॒ यत्पुण्यं॑ दे॒वस्य॑ सवि॒तुर्ब्र॑ह्मवा॒दिन॒ कत्यृ॒तमे॒व दे॒वा वा आयु॑षः प्रा॒णमिन्द्रो॑ दधी॒चो दे॑वासु॒राः स प्र॒जाप॑ति॒ स स॑मु॒द्रो ये वै च॒त्वार॒स्तस्यावा॑चो॒ द्वाद॑श॥१२॥}{अ॒ग्नेः कृत्ति॑का देवगृ॒हा ऋ॒तमे॒वर्ध्यामे॒व ति॒स्रः परा॑ची॒र्ये वै च॒त्वारो॒ नव॑पञ्चा॒शत्॥५९॥}{अ॒ग्नेः कृत्ति॑का॒ य उ॑ चैनमे॒वं वेद॑॥}{हरि॑ ओम्॥}{इति श्रीकृष्णयजुर्वेदीयतैत्तिरीयब्राह्मणे प्रथमाष्टके पञ्चमः प्रपाठकः समाप्तः॥}
\clearpage
\sect{षष्ठमः प्रश्नः}
\setcounter{anuvakam}{0}
\dnsub{तैत्तिरीयब्राह्मणे प्रथमाष्टके षष्ठः प्रपाठकः}

%1.6.1.1
अनु॑मत्यै पुरो॒डाश॑म॒ष्टाक॑पालं॒ निर्व॑पति। ये प्र॒त्यञ्च॒ शम्या॑या अव॒शीय॑न्ते। तन्नैर्॑ऋ॒तमेक॑कपालम्। इ॒यं वा अनु॑मतिः। इ॒यं निर्\mbox{}ऋ॑तिः। नै॒र्॒ऋ॒तेन॒ पूर्वे॑ण॒ प्रच॑रति। पा॒प्मान॑मे॒व निर्\mbox{}ऋ॑तिं॒ पूर्वां नि॒रव॑दयते। एक॑कपालो भवति। ए॒क॒धैव निर्\mbox{}ऋ॑तिं नि॒रव॑दयते। यदहु॑त्वा॒ गार्\mbox{}ह॑पत्य ई॒युः॥१॥

%1.6.1.2
रु॒द्रो भू॒त्वाऽग्निर॑नू॒त्थाय॑। अ॒ध्व॒र्युं च॒ यज॑मानं च हन्यात्। वीहि॒ स्वाहाऽऽहु॑तिं जुषा॒ण इत्या॑ह। आहु॑त्यै॒वैन शमयति। नार्ति॒मार्च्छ॑त्यध्व॒र्युर्न यज॑मानः। ए॒को॒ल्मु॒केन॑ यन्ति। तद्धि निर्\mbox{}ऋ॑त्यै भाग॒धेयम्। इ॒मान्दिशं॑ यन्ति। ए॒षा वै निर्\mbox{}ऋ॑त्यै॒ दिक्। स्वाया॑मे॒व दि॒शि निर्\mbox{}ऋ॑तिं नि॒रव॑दयते॥२॥

%1.6.1.3
स्वकृ॑त॒ इरि॑णे जुहोति प्रद॒रे वा। ए॒तद्वै निर्\mbox{}ऋ॑त्या आ॒यत॑नम्। स्व ए॒वायत॑ने॒ निर्\mbox{}ऋ॑तिं नि॒रव॑दयते। ए॒ष ते॑ निर्‌ऋते भा॒ग इत्या॑ह। निर्दि॑शत्ये॒वैनाम्। भूते॑ ह॒विष्म॑त्य॒सीत्या॑ह। भूति॑मे॒वोपाव॑र्तते। मु॒ञ्चेममह॑स॒ इत्या॑ह। अह॑स ए॒वैनं॑ मुञ्चति। अ॒ङ्गु॒ष्ठाभ्यां जुहोति॥३॥

%1.6.1.4
अ॒न्त॒त ए॒व निर्\mbox{}ऋ॑तिं नि॒रव॑दयते। कृ॒ष्णं वास॑ कृ॒ष्णतू॑ष॒न्दक्षि॑णा। ए॒तद्वै निर्\mbox{}ऋ॑त्यै रू॒पम्। रू॒पेणै॒व निर्\mbox{}ऋ॑तिं नि॒रव॑दयते। अप्र॑तीक्ष॒माय॑न्ति। निर्\mbox{}ऋ॑त्या अ॒न्तर्\mbox{}हि॑त्यै। स्वाहा॒ नमो॒ य इ॒दञ्च॒कारेति॒ पुन॒रेत्य॒ गार्\mbox{}ह॑पत्ये जुहोति। आहु॑त्यै॒व न॑म॒स्यन्तो॒ गार्\mbox{}ह॑पत्यमु॒पाव॑र्तन्ते। आ॒नु॒म॒तेन॒ प्रच॑रति। इ॒यं वा अनु॑मतिः॥४॥

%1.6.1.5
इ॒यमे॒वास्मै॑ रा॒ज्यमनु॑ मन्यते। धे॒नुर्दक्षि॑णा। इ॒मामे॒व धे॒नुं कु॑रुते। आ॒दि॒त्यञ्च॒रुं निर्व॑पति। उ॒भयीष्वे॒व प्र॒जास्व॒भिषि॑च्यते। दैवी॑षु च॒ मानु॑षीषु च। वरो॒ दक्षि॑णा। वरो॒ हि रा॒ज्य समृ॑द्ध्यै। आ॒ग्ना॒वै॒ष्ण॒वमेका॑दशकपालं॒ निर्व॑पति। अ॒ग्निः सर्वा॑ दे॒वता॥५॥

%1.6.1.6
विष्णु॑र्य॒ज्ञः। दे॒वताश्चै॒व य॒ज्ञञ्चाव॑ रुन्धे। वा॒म॒नो व॒ही दक्षि॑णा। यद्व॒ही। तेनाग्ने॒यः। यद्वा॑म॒नः। तेन॑ वैष्ण॒वः समृ॑द्ध्यै। अ॒ग्नी॒षो॒मीय॒मेका॑दशकपालं॒ निर्व॑पति। अ॒ग्नीषोमाभ्यां॒ वा इन्द्रो॑ वृ॒त्रम॑ह॒न्निति॑। यद॑ग्नीषो॒मीय॒मेका॑दशकपालं नि॒र्वप॑ति॥६॥

%1.6.1.7
वार्त्र॑घ्नमे॒व विजि॑त्यै। हिर॑ण्य॒न्दक्षि॑णा॒ समृ॑द्ध्यै। इन्द्रो॑ वृ॒त्र ह॒त्वा। दे॒वता॑भिश्चेन्द्रि॒येण॑ च॒ व्यार्ध्यत। स ए॒तमैन्द्रा॒ग्नमेका॑दशकपालमपश्यत्। तन्निर॑वपत्। तेन॒ वै स दे॒वताश्चेन्द्रि॒यञ्चावा॑रुन्ध। यदैन्द्रा॒ग्नमेका॑दशकपालं नि॒र्वप॑ति। दे॒वताश्चै॒व तेनेन्द्रि॒यं च॒ यज॑मा॒नोऽव॑रुन्धे। ऋ॒ष॒भो व॒ही दक्षि॑णा॥७॥

%1.6.1.8
यद्व॒ही। तेनाग्ने॒यः। यदृ॑ष॒भः। तेनै॒न्द्रः समृ॑द्ध्यै। आ॒ग्ने॒यम॒ष्टाक॑पालं॒ निर्व॑पति। ऐ॒न्द्रन्दधि॑। यदाग्ने॒यो भव॑ति। अ॒ग्निर्वै य॑ज्ञमु॒खम्। य॒ज्ञ॒मु॒खमे॒वर्द्धिं॑ पु॒रस्ताद्धत्ते। यदै॒न्द्रन्दधि॑॥८॥

%1.6.1.9
इ॒न्द्रि॒यमे॒वाव॑रुन्धे। ऋ॒ष॒भो व॒ही दक्षि॑णा। यद्व॒ही। तेनाग्ने॒यः। यदृ॑ष॒भः। तेनै॒न्द्रः समृ॑द्ध्यै। याव॑ती॒र्वै प्र॒जा ओष॑धीना॒महु॑ताना॒माश्ञ\sn{}। ताः परा॑ऽभवन्। आ॒ग्र॒य॒णं भ॑वति हु॒ताद्या॑य। यज॑मान॒स्याप॑राभावाय॥९॥

%1.6.1.10
दे॒वा वा ओष॑धीष्वा॒जिम॑युः। ता इ॑न्द्रा॒ग्नी उद॑जयताम्। तावे॒तमैन्द्रा॒ग्नन्द्वाद॑शकपालं॒ निर॑वृणाताम्। यदैन्द्रा॒ग्नो भव॒त्युज्जि॑त्यै। द्वाद॑शकपालो भवति। द्वाद॑श॒ मासा संवत्स॒रः। सं॒व॒त्स॒रेणै॒वास्मा॒ अन्न॒मव॑रुन्धे। वै॒श्व॒दे॒वश्च॒रुर्भ॑वति। वै॒श्व॒दे॒वं वा अन्नम्। अन्न॑मे॒वास्मै स्वदयति॥१०॥

%1.6.1.11
प्र॒थ॒म॒जो व॒त्सो दक्षि॑णा॒ समृ॑द्ध्यै। सौ॒म्य श्या॑मा॒कं च॒रुं निर्व॑पति। सोमो॒ वा अ॑कृष्टप॒च्यस्य॒ राजा। अ॒कृ॒ष्ट॒प॒च्यमे॒वास्मै स्वदयति। वासो॒ दक्षि॑णा। सौ॒म्य हि दे॒वत॑या॒ वास॒ समृ॑द्ध्यै। सर॑स्वत्यै च॒रुं निर्व॑पति। सर॑स्वते च॒रुम्। मि॒थु॒नमे॒वाव॑ रुन्धे। मि॒थु॒नौ गावौ॒ दक्षि॑णा॒ समृ॑द्ध्यै। एति॒ वा ए॒ष य॑ज्ञमु॒खादृध्या। योऽग्नेर्दे॒वता॑या॒ एति॑। अ॒ष्टावे॒तानि॑ ह॒वीषि॑ भवन्ति। अ॒ष्टाक्ष॑रा गाय॒त्री। गा॒य॒त्रोऽग्निः। तेनै॒व य॑ज्ञमु॒खादृध्या॑ अ॒ग्नेर्दे॒वता॑यै॒ नैति॑॥११॥\anuvakamend[ई॒युर्नि॒रव॑दयतेऽङ्गु॒ष्ठाभ्यां जुहो॒त्यनु॑मतिर्दे॒वता॑ नि॒र्वप॑ति व॒ही दक्षि॑णा॒ यदै॒न्द्रन्दध्यप॑राभावाय स्वदयति॒ गावौ॒ दक्षि॑णा॒ समृ॑द्ध्यै॒ षट्च॑]

%1.6.2.1
वै॒श्व॒दे॒वेन॒ वै प्र॒जाप॑तिः प्र॒जा अ॑सृजत। ताः सृ॒ष्टा न प्राजा॑यन्त। सोऽग्निर॑कामयत। अ॒हमि॒माः प्रज॑नयेय॒मिति॑। स प्र॒जाप॑तये॒ शुच॑मदधात्। सो॑ऽशोचत्प्र॒जामि॒च्छमा॑नः। तस्मा॒द्यञ्च॑ प्र॒जा भु॒नक्ति॒ यं च॒ न। तावु॒भौ शो॑चतः प्र॒जामि॒च्छमा॑नौ। तास्व॒ग्निमप्य॑सृजत्। ता अ॒ग्निरध्यैत्॥१२॥

%1.6.2.2
सोमो॒ रेतो॑ऽदधात्। स॒वि॒ता प्राज॑नयत्। सर॑स्वती॒ वाच॑मदधात्। पू॒षाऽपो॑षयत्। ते वा ए॒ते त्रिः सं॑वत्स॒रस्य॒ प्रयु॑ज्यन्ते। ये दे॒वाः पुष्टि॑पतयः। सं॒व॒त्स॒रो वै प्र॒जाप॑तिः। सं॒व॒त्स॒रेणै॒वास्मै प्र॒जाः प्राज॑नयत्। ताः प्र॒जा जा॒ता म॒रुतोऽघ्नन्। अ॒स्मानपि॒ न प्रायु॑क्ष॒तेति॑॥१३॥

%1.6.2.3
स ए॒तं प्र॒जाप॑तिर्मारु॒त स॒प्तक॑पालमपश्यत्। तन्निर॑वपत्। ततो॒ वै प्र॒जाभ्यो॑ऽकल्पत। यन्मा॑रु॒तो नि॑रु॒प्यते। य॒ज्ञस्य॒ कॢप्त्यै। प्र॒जाना॒मघा॑ताय। स॒प्तक॑पालो भवति। स॒प्तग॑णा॒ वै म॒रुत॑। ग॒ण॒श ए॒वास्मै॒ विश॑ङ्कल्पयति। स प्र॒जाप॑तिरशोचत्॥१४॥

%1.6.2.4
याः पूर्वा प्र॒जा असृ॑क्षि। म॒रुत॒स्ता अ॑वधिषुः। क॒थमप॑राः सृजे॒येति॑। तस्य॒ शुष्म॑ आ॒ण्डं भू॒तं निर॑वर्तत। तद्व्युद॑हरत्। तद॑पोषयत्। तत्प्राजा॑यत। आ॒ण्डस्य॒ वा ए॒तद्रू॒पम्। यदा॒मिक्षा। यद्व्यु॒द्धर॑ति॥१५॥

%1.6.2.5
प्र॒जा ए॒व तद्यज॑मानः पोषयति। वै॒श्व॒दे॒व्या॑मिक्षा॑ भवति। वै॒श्व॒दे॒व्यो॑ वै प्र॒जाः। प्र॒जा ए॒वास्मै॒ प्रज॑नयति। वाजि॑न॒मान॑यति। प्र॒जास्वे॒व प्रजा॑तासु॒ रेतो॑ दधाति। द्या॒वा॒पृ॒थि॒व्य॑ एक॑कपालो भवति। प्र॒जा ए॒व प्रजा॑ता॒ द्यावा॑पृथि॒वीभ्या॑मुभ॒यत॒ परि॑ गृह्णाति। दे॒वा॒सु॒राः संय॑त्ता आसन्। सोऽग्निर॑ब्रवीत्॥१६॥

%1.6.2.6
मामग्रे॑ यजत। मया॒ मुखे॒नासु॑राञ्जेष्य॒थेति॑। मां द्वि॒तीय॒मिति॒ सोमोऽब्रवीत्। मया॒ राज्ञा॑ जेष्य॒थेति॑। मान्तृ॒तीय॒मिति॑ सवि॒ता। मया॒ प्रसू॑ता जेष्य॒थेति॑। माञ्च॑तु॒र्थीमिति॒ सर॑स्वती। इ॒न्द्रि॒यं वो॒ऽहं धास्या॒मीति॑। मां प॑ञ्च॒ममिति॑ पू॒षा। मया प्रति॒ष्ठया॑ जेष्य॒थेति॑॥१७॥

%1.6.2.7
तेऽग्निना॒ मुखे॒नासु॑रानजयन्। सोमे॑न॒ राज्ञा। स॒वि॒त्रा प्रसू॑ताः। सर॑स्वतीन्द्रि॒यम॑दधात्। पू॒षा प्र॑ति॒ष्ठाऽऽसीत्। ततो॒ वै दे॒वा व्य॑जयन्त। यदे॒तानि॑ ह॒वीषि॑ निरु॒प्यन्ते॒ विजि॑त्यै। नोत्त॑रवे॒दिमुप॑वपति। प॒शवो॒ वा उ॑त्तरवे॒दिः। अजा॑ता इव॒ ह्ये॑तर्\mbox{}हि॑ प॒शव॑॥१८॥\anuvakamend[ऐ॒दित्य॑शोचद्व्यु॒द्धर॑त्यब्रवीत्प्रति॒ष्ठया॑ जेष्य॒थेत्ये॒तर्\mbox{}हि॑ प॒शव॑]

%1.6.3.1
त्रि॒वृद्ब॒र्॒हिर्भ॑वति। मा॒ता पि॒ता पु॒त्रः। तदे॒व तन्मि॑थु॒नम्। उल्ब॒ङ्गर्भो॑ ज॒रायु॑। तदे॒व तन्मि॑थु॒नम्। त्रे॒धा ब॒र्॒हिः सन्न॑द्धं भवति। त्रय॑ इ॒मे लो॒काः। ए॒ष्वे॑व लो॒केषु॒ प्रति॑ तिष्ठति। ए॒क॒धा पुन॒ सन्न॑द्धं भवति। एक॑ इव॒ ह्य॑यं लो॒कः॥१९॥

%1.6.3.2
अ॒स्मिन्ने॒व तेन॑ लो॒के प्रति॑तिष्ठति। प्र॒सुवो॑ भवन्ति। प्र॒थ॒म॒जामे॒व पुष्टि॒मव॑रुन्धे। प्र॒थ॒म॒जो व॒त्सो दक्षि॑णा॒ समृ॑द्ध्यै। पृ॒ष॒दा॒ज्यं गृ॑ह्णाति। प॒शवो॒ वै पृ॑षदा॒ज्यम्। प॒शूने॒वाव॑ रुन्धे। प॒ञ्च॒गृ॒ही॒तं भ॑वति। पाङ्क्ता॒ हि प॒शव॑। ब॒हु॒रू॒पं भ॑वति॥२०॥

%1.6.3.3
ब॒हु॒रू॒पा हि प॒शव॒ समृ॑द्ध्यै। अ॒ग्निं म॑न्थन्ति। अ॒ग्निमु॑खा॒ वै प्र॒जाप॑तिः प्र॒जा अ॑सृजत। यद॒ग्निं मन्थ॑न्ति। अ॒ग्निमु॑खा ए॒व तत्प्र॒जा यज॑मानः सृजते। नव॑ प्रया॒जा इ॑ज्यन्ते। नवा॑नूया॒जाः। अ॒ष्टौ ह॒वीषि॑। द्वावा॑घा॒रौ। द्वावाज्य॑भागौ॥२१॥

%1.6.3.4
त्रि॒शत्सम्प॑द्यन्ते। त्रि॒शद॑क्षरा वि॒राट्। अन्नं॑ वि॒राट्। वि॒राजै॒वान्नाद्य॒मव॑रुन्धे। यज॑मानो॒ वा एक॑कपालः। तेज॒ आज्यम्। यदेक॑कपाल॒ आज्य॑मा॒नय॑ति। यज॑मानमे॒व तेज॑सा॒ सम॑र्धयति। यज॑मानो॒ वा एक॑कपालः। प॒शव॒ आज्यम्॥२२॥

%1.6.3.5
यदेक॑कपाल॒ आज्य॑मा॒नय॑ति। यज॑मानमे॒व प॒शुभि॒ सम॑र्धयति। यदल्प॑मा॒नयेत्। अल्पा॑ एनं प॒शवो॑ भु॒ञ्जन्त॒ उप॑तिष्ठेरन्। यद्ब॒ह्वा॑नयेत्। ब॒हव॑ एनं प॒शवोऽभु॑ञ्जन्त॒ उप॑तिष्ठेरन्। ब॒ह्वा॑नीया॒विः पृ॑ष्ठं कुर्यात्। ब॒हव॑ ए॒वैनं॑ प॒शवो॑ भु॒ञ्जन्त॒ उप॑तिष्ठन्ते। यज॑मानो॒ वा एक॑कपालः। यदेक॑कपालस्याव॒द्येत्॥२३॥

%1.6.3.6
यज॑मान॒स्याव॑द्येत्। उद्वा॒ माद्ये॒द्यज॑मानः। प्र वा॑ मीयेत। स॒कृदे॒व हो॑त॒व्य॑। स॒कृदि॑व॒ हि सु॑व॒र्गो लो॒कः। हु॒त्वाऽभि जु॑होति। यज॑मानमे॒व सु॑व॒र्गं लो॒कं ग॑मयि॒त्वा। तेज॑सा॒ सम॑र्धयति। यज॑मानो॒ वा एक॑कपालः। सु॒व॒र्गो लो॒क आ॑हव॒नीय॑॥२४॥

%1.6.3.7
यदेक॑कपालमाहव॒नीये॑ जु॒होति॑। यज॑मानमे॒व सु॑व॒र्गं लो॒कं ग॑मयति। यद्धस्ते॑न जुहु॒यात्। सु॒व॒र्गाल्लो॒काद्यज॑मान॒मव॑विध्येत्। स्रु॒चा जु॑होति। सु॒व॒र्गस्य॑ लो॒कस्य॒ सम॑ष्ट्यै। यत्प्राङ्पद्ये॑त। दे॒व॒लो॒कम॒भिज॑येत्। यद्द॑क्षि॒णा पि॑तृलो॒कम्। यत्प्र॒त्यक्॥२५॥

%1.6.3.8
रक्षासि य॒ज्ञ ह॑न्युः। यदुदङ्ङ्॑। म॒नु॒ष्य॒लो॒कम॒भिज॑येत्। प्रति॑ष्ठितो होत॒व्य॑। एक॑कपालं॒ वै प्र॑ति॒तिष्ठ॑न्त॒न्द्यावा॑पृथि॒वी अनु॒ प्रति॑तिष्ठतः। द्यावा॑पृथि॒वी ऋ॒तव॑। ऋ॒तून् य॒ज्ञः। य॒ज्ञं यज॑मानः। यज॑मानं प्र॒जाः। तस्मा॒त्प्रति॑ष्ठितो होत॒व्य॑॥२६॥

%1.6.3.9
वा॒जिनो॑ यजति। अ॒ग्निर्वा॒युः सूर्य॑। ते वै वा॒जिन॑। ताने॒व तद्य॑जति। अथो॒ खल्वा॑हुः। छन्दासि॒ वै वा॒जिन॒ इति॑। तान्ये॒व तद्य॑जति। ऋ॒ख्सा॒मे वा इन्द्र॑स्य॒ हरी॑ सोम॒पानौ। तयो परि॒धय॑ आ॒धानम्। वाजि॑नं भाग॒धेयम्॥२७॥

%1.6.3.10
यदप्र॑हृत्य परि॒धीं जु॑हु॒यात्। अ॒न्तरा॑धानाभ्याङ्घा॒सं प्रय॑च्छेत्। प्र॒हृत्य॑ परि॒धीं जु॑होति। निरा॑धानाभ्यामे॒व घा॒सं प्रय॑च्छति। ब॒र्॒हिषि॑ विषि॒ञ्चन्वाजि॑न॒मा न॑यति। प्र॒जा वै ब॒र्॒हिः। रेतो॒ वाजि॑नम्। प्र॒जास्वे॒व रेतो॑ दधाति। स॒मु॒प॒हूय॑ भक्षयन्ति। ए॒तत्सो॑मपीथा॒ ह्ये॑ते। अथो॑ आ॒त्मन्ने॒व रेतो॑ दधते। यज॑मान उत्त॒मो भ॑क्षयति। प॒शवो॒ वै वाजि॑नम्। यज॑मान ए॒व प॒शून्प्रति॑ष्ठापयन्ति॥२८॥\anuvakamend[लो॒को ब॑हुरू॒पं भ॑व॒त्याज्य॑भागौ प॒शव॒ आज्य॑मव॒द्येदा॑हव॒नीय॑ प्र॒त्यक्तस्मा॒त्प्रति॑ष्ठितो होत॒व्यो॑ भाग॒धेय॑मे॒ते च॒त्वारि॑ च]

%1.6.4.1
प्र॒जाप॑तिः सवि॒ता भू॒त्वा प्र॒जा अ॑सृजत। ता ए॑न॒मत्य॑मन्यन्त। ता अ॑स्मा॒दपाक्रामन्। ता वरु॑णो भू॒त्वा प्र॒जा वरु॑णेनाग्राहयत्। ताः प्र॒जा वरु॑णगृहीताः। प्र॒जाप॑तिं॒ पुन॒रुपा॑धावन्ना॒थमि॒च्छमा॑नाः। स ए॒तान्प्र॒जाप॑तिर्वरुणप्रघा॒सान॑पश्यत्। तां निर॑वपत्। तैर्वै स प्र॒जा व॑रुणपा॒शाद॑मुञ्चत्। यद्व॑रुणप्रघा॒सा नि॑रु॒प्यन्ते॥२९॥

%1.6.4.2
प्र॒जाना॒मव॑रुणग्राहाय। तासा॒न्दक्षि॑णो बा॒हुर्न्य॑क्न॒ आसीत्। स॒व्यः प्रसृ॑तः। स ए॒तां द्वि॒तीयान्दक्षिण॒तो वेदि॒मुद॑हन्। ततो॒ वै स प्र॒जाना॒न्दक्षि॑णं बा॒हुं प्रासा॑रयत्। यद्द्वि॒तीयान्दक्षिण॒तो वेदि॑मु॒द्धन्ति॑। प्र॒जाना॑मे॒व तद्यज॑मानो॒ दक्षि॑णं बा॒हुं प्रसा॑रयति। तस्माच्चातुर्मास्यया॒ज्य॑मुष्मि॑ल्लोँ॒क उ॑भ॒याबा॑हुः। य॒ज्ञाभि॑जित॒ ह्य॑स्य। पृ॒थ॒मा॒त्राद्वैदी॒ अस॑म्भिन्ने भवतः॥३०॥

%1.6.4.3
तस्मात्पृथमा॒त्रं व्यसौ। उत्त॑रस्यां॒ वेद्या॑मुत्तरवे॒दिमुप॑ वपति। प॒शवो॒ वा उ॑त्तरवे॒दिः। प॒शूने॒वाव॑रुन्धे। अथो॑ यज्ञप॒रुषोऽन॑न्तरित्यै। ए॒तद्ब्राह्मणान्ये॒व पञ्च॑ ह॒वीषि॑। अथै॒ष ऐन्द्रा॒ग्नो भ॑वति। प्रा॒णा॒पा॒नौ वा ए॒तौ दे॒वानाम्। यदि॑न्द्रा॒ग्नी। यदैन्द्रा॒ग्नो भव॑ति॥३१॥

%1.6.4.4
प्रा॒णा॒पा॒नावे॒वाव॑ रुन्धे। ओजो॒ बलं॒ वा ए॒तौ दै॒वानाम्। यदि॑न्द्रा॒ग्नी। यदैन्द्रा॒ग्नो भव॑ति। ओजो॒ बल॑मे॒वाव॑ रुन्धे। मा॒रु॒त्या॑मिक्षा॑ भवति। वा॒रु॒ण्या॑मिक्षा। मे॒षी च॑ मे॒षश्च॑ भवतः। मि॒थु॒ना ए॒व प्र॒जा व॑रुणपा॒शान्मु॑ञ्चति। लो॒म॒शौ भ॑वतो मेध्य॒त्वाय॑॥३२॥

%1.6.4.5
श॒मी॒प॒र्णान्युप॑ वपति। घा॒समे॒वाभ्या॒मपि॑ यच्छति। प्र॒जाप॑तिम॒न्नाद्य॒न्नोपा॑नमत्। स ए॒तेन॑ श॒तेध्मे॑न ह॒विषा॒ऽन्नाद्य॒मवा॑रुन्ध। यत्प॑रः श॒तानि॑ शमीप॒र्णानि॒ भव॑न्ति। अ॒न्नाद्य॒स्याव॑रुद्ध्यै। सौ॒म्यानि॒ वै क॒रीरा॑णि। सौ॒म्या खलु॒ वा आहु॑तिर्दि॒वो वृष्टि॑ञ्च्यावयति। यत्क॒रीरा॑णि॒ भव॑न्ति। सौ॒म्ययै॒वाहु॑त्या दि॒वो वृष्टि॒मव॑रुन्धे। का॒य एक॑कपालो भवति। प्र॒जानाङ्क॒न्त्वाय॑। प्र॒ति॒पू॒रु॒षङ्क॑रम्भपा॒त्राणि॑ भवन्ति। जा॒ता ए॒व प्र॒जा व॑रुणपा॒शान्मु॑ञ्चति। एक॒मति॑रिक्तम्। ज॒नि॒ष्यमा॑णा ए॒व प्र॒जा व॑रुणपा॒शान्मु॑ञ्चति॥३३॥\anuvakamend[नि॒रु॒प्यन्ते॑ भवतो॒ भव॑ति मेध्य॒त्वाय॑ रुन्धे॒ षट्च॑]

%1.6.5.1
उत्त॑रस्यां॒ वेद्या॑म॒न्यानि॑ ह॒वीषि॑ सादयति। दक्षि॑णायां मारु॒तीम्। अ॒प॒धु॒रमे॒वैना॑ युनक्ति। अथो॒ ओज॑ ए॒वासा॒मव॑ हरति। तस्मा॒द्ब्रह्म॑णश्च क्ष॒त्राच्च॒ विशोऽन्यतोऽपक्र॒मिणी। मा॒रु॒त्या पूर्व॑या॒ प्रच॑रति। अनृ॑तमे॒वाव॑ यजते। वा॒रु॒ण्योत्त॑रया। अ॒न्त॒त ए॒व वरु॑ण॒मव॑ यजते। यदे॒वाध्व॒र्युः क॒रोति॑॥३४॥

%1.6.5.2
तत्प्र॑तिप्रस्था॒ता क॑रोति। तस्मा॒द्यच्छ्रेयान्क॒रोति॑। तत्पापी॑यान्करोति। पत्नीं वाचयति। मेध्या॑मे॒वैनां करोति। अथो॒ तप॑ ए॒वैना॒मुप॑ नयति। यज्जा॒र सन्त॒न्न प्र॑ब्रू॒यात्। प्रि॒यं ज्ञा॒ति रु॑न्ध्यात्। अ॒सौ मे॑ जा॒र इति॒ निर्दि॑शेत्। नि॒र्दिश्यै॒वैनं॑ वरुणपा॒शेन॑ ग्राहयति॥३५॥

%1.6.5.3
प्र॒घा॒स्यान्॑ हवामह॒ इति॒ पत्नी॑मु॒दान॑यति। अह्व॑तै॒वैनाम्। यत्पत्नी॑ पुरोनुवा॒क्या॑मनुब्रू॒यात्। निर्वीर्यो॒ यज॑मानः स्यात्। यज॑मा॒नोऽन्वा॑ह। आ॒त्मन्ने॒व वी॒र्य॑न्धत्ते। उ॒भौ या॒ज्या सवीर्य॒त्वाय॑। यद्ग्रामे॒ यदर॑ण्य॒ इत्या॑ह। य॒थो॒दि॒तमे॒व वरु॑ण॒मव॑ यजते। य॒ज॒मा॒न॒दे॒व॒त्यो॑ वा आ॑हव॒नीय॑॥३६॥

%1.6.5.4
भ्रा॒तृ॒व्य॒दे॒व॒त्यो॑ दक्षि॑णः। यदा॑हव॒नीये॑ जुहु॒यात्। यज॑मानं वरुणपा॒शेन॑ ग्राहयेत्। दक्षि॑णे॒ऽग्नौ जु॑होति। भ्रातृ॑व्यमे॒व व॑रुणपा॒शेन॑ ग्राहयति। शूर्पे॑ण जुहोति। अन्य॑मे॒व वरु॑ण॒मव॑ यजते। शी॒र्॒षन्न॑धि नि॒धाय॑ जुहोति। शी॒र्\mbox{}ष॒त ए॒व वरु॑ण॒मव॑ यजते। प्र॒त्यङ्तिष्ठं॑ जुहोति॥३७॥

%1.6.5.5
प्र॒त्यङ्ङे॒व व॑रुणपा॒शान्निर्मु॑च्यते। अक्र॒न्कर्म॑ कर्म॒कृत॒ इत्या॑ह। दे॒वाऽनृ॒णं नि॑रव॒दाय॑। अ॒नृ॒णा गृ॒हानुप॒ प्रेतेति॒ वावैतदा॑ह। वरु॑णगृहीतं॒ वा ए॒तद्य॒ज्ञस्य॑। यद्यजु॑षा गृही॒तस्या॑ति॒रिच्य॑ते। तुषाश्च निष्का॒सश्च॑। तुषैश्च निष्का॒सेन॑ चावभृ॒थमवै॑ति। वरु॑णगृहीतेनै॒व वरु॑ण॒मव॑यजते। अ॒पो॑ऽवभृ॒थमवै॑ति॥३८॥

%1.6.5.6
अ॒प्सु वै वरु॑णः। सा॒क्षादे॒व वरु॑ण॒मव॑यजते। प्रति॑युतो॒ वरु॑णस्य॒ पाश॒ इत्या॑ह। व॒रु॒ण॒पा॒शादे॒व निर्मु॑च्यते। अप्र॑तीक्ष॒मा य॑न्ति। वरु॑णस्या॒न्तर्\mbox{}हि॑त्यै। एधोऽष्येधिषी॒महीत्या॑ह। स॒मिधै॒वाग्निन्न॑म॒स्यन्त॑ उ॒पाय॑न्ति। तेजो॑ऽसि॒ तेजो॒ मयि॑ धे॒हीत्या॑ह। तेज॑ ए॒वात्मन्ध॑त्ते॥३९॥\anuvakamend[क॒रोति॑ ग्राहयत्याहव॒नीय॒स्तिष्ठं॑ जुहोत्य॒पो॑ऽवभृ॒थमवै॑ति धत्ते]

%1.6.6.1
दे॒वा॒सु॒राः संय॑त्ता आसन्। सोऽग्निर॑ब्रवीत्। ममे॒यमनी॑कवती त॒नूः। तां प्री॑णीत। अथासु॑रान॒भि भ॑विष्य॒थेति॑। ते दे॒वा अ॒ग्नयेऽनी॑कवते पुरो॒डाश॑म॒ष्टाक॑पालं॒ निर॑वपन्। सोऽग्निरनी॑कवा॒न्त्स्वेन॑ भाग॒धेये॑न प्री॒तः। च॒तु॒र्धाऽनी॑कान्यजनयत। ततो॑ दे॒वा अभ॑वन्। पराऽसु॑राः॥४०॥

%1.6.6.2
यद॒ग्नयेऽनी॑कवते पुरो॒डाश॑म॒ष्टाक॑पालं नि॒र्वप॑ति। अ॒ग्निमे॒वानी॑कवन्त॒ स्वेन॑ भाग॒धेये॑न प्रीणाति। सोऽग्निरनी॑कवा॒न्त्स्वेन॑ भाग॒धेये॑न प्री॒तः। च॒तु॒र्धाऽनी॑कानि जनयते। अ॒सौ वा आ॑दि॒त्योऽग्निरनी॑कवान्। तस्य॑ र॒श्मयोऽनी॑कानि। सा॒क सूर्ये॑णोद्य॒ता निर्व॑पति। सा॒क्षादे॒वास्मा॒ अनी॑कानि जनयति। तेऽसु॑रा॒ परा॑जिता॒ यन्त॑। द्यावा॑पृथि॒वी उपाश्रयन्॥४१॥

%1.6.6.3
ते दे॒वा म॒रुद्भ्य॑ सान्तप॒नेभ्य॑श्च॒रुं निर॑वपन्। यन्म॒रुद्भ्य॑ सान्तप॒नेभ्य॑श्च॒रुं नि॒र्वप॑ति। द्यावा॑पृथि॒वीभ्या॑मे॒व तदु॑भ॒यतो॒ यज॑मानो॒ भ्रातृ॑व्या॒न्त्सन्त॑पति। म॒ध्यन्दि॑ने॒ निर्व॑पति। तर्\mbox{}हि॒ हि तेक्ष्णि॑ष्ठ॒न्तप॑ति। च॒रुर्भ॑वति। स॒र्वत॑ ए॒वैना॒न्त्सन्त॑पति। ते दे॒वाः श्वो॑विज॒यिन॒ सन्त॑। सर्वा॑सान्दु॒ग्धे गृ॑हमे॒धीयं॑ च॒रुं निर॑वपन्॥४२॥

%1.6.6.4
आशि॑ता ए॒वाद्योप॑वसाम। कस्य॒ वाऽहे॒दम्। कस्य॑ वा॒ श्वो भ॑वि॒तेति॑। स शृ॒तो॑ऽभवत्। तस्याहु॑तस्य॒ नाश्ञ\sn{}। न हि दे॒वा अहु॑तस्या॒श्ञन्ति॑। तेऽब्रुवन्। कस्मा॑ इ॒म होष्याम॒ इति॑। म॒रुद्भ्यो॑ गृहमे॒धिभ्य॒ इत्य॑ब्रुवन्। तं म॒रुद्भ्यो॑ गृहमे॒धिभ्यो॑ऽजुहवुः॥४३॥

%1.6.6.5
ततो॑ दे॒वा अभ॑वन्। पराऽसु॑राः। यस्यै॒वं वि॒दुषो॑ म॒रुद्भ्यो॑ गृहमे॒धिभ्यो॑ गृ॒हे जुह्व॑ति। भव॑त्या॒त्मना। पराऽस्य॒ भ्रातृ॑व्यो भवति। यद्वै य॒ज्ञस्य॑ पाक॒त्रा क्रि॒यते। प॒श॒व्य॑न्तत्। पा॒क॒त्रा वा ए॒तत्क्रि॑यते। यन्नेध्माब॒र्॒हिर्भव॑ति। न सा॑मिधे॒नीर॒न्वाह॑॥४४॥

%1.6.6.6
न प्र॑या॒जा इ॒ज्यन्ते। नानू॑या॒जाः। य ए॒वं वेद॑। प॒शु॒मान्भ॑वति। आज्य॑भागौ यजति। य॒ज्ञस्यै॒व चक्षु॑षी॒ नान्तरे॑ति। म॒रुतो॑ गृहमे॒धिनो॑ यजति। भा॒ग॒धेये॑नै॒वैना॒न्त्सम॑र्धयति। अ॒ग्निस्वि॑ष्ट॒कृतं॑ यजति॒ प्रति॑ष्ठित्यै। इडान्तो भवति। प॒शवो॒ वा इडा। प॒शुष्वे॒वोपरि॑ष्टा॒त्प्रति॑तिष्ठति॥४५॥\anuvakamend[असु॑रा अश्रयन्गृहमे॒धीयं॑ च॒रुं निर॑वपन्नजुहवुर॒न्वाहेडान्तो भवति॒ द्वे च॑]

%1.6.7.1
यत्पत्नी॑ गृहमे॒धीय॑स्याश्ञी॒यात्। गृ॒ह॒मे॒ध्ये॑व स्यात्। वि त्व॑स्य य॒ज्ञ ऋ॑ध्येत। यन्नाश्ञी॒यात्। अगृ॑हमेधी स्यात्। नास्य॑ य॒ज्ञो व्यृ॑द्ध्येत। प्रति॑वेशं पचेयुः। तस्याश्ञीयात्। गृ॒ह॒मे॒ध्ये॑व भ॑वति। नास्य॑ य॒ज्ञो व्यृ॑द्ध्यते॥४६॥

%1.6.7.2
ते दे॒वा गृ॑हमे॒धीये॑ने॒ष्ट्वा। आशि॑ता अभवन्। आञ्ज॑ता॒भ्य॑ञ्जत। अनु॑ व॒त्सान॑वासयन्। तेभ्योऽसु॑रा॒ क्षुधं॒ प्राहि॑ण्वन्। सा दे॒वेषु॑ लो॒कमवि॑त्वा। असु॑रा॒न्पुन॑रगच्छत्। गृ॒ह॒मे॒धीये॑ने॒ष्ट्वा। आशि॑ता भवन्ति। आञ्ज॑ते॒ऽभ्य॑ञ्जते॥४७॥

%1.6.7.3
अनु॑ व॒त्सान् वा॑सयन्ति। भ्रातृ॑व्यायै॒व तद्यज॑मान॒ क्षुधं॒ प्रहि॑णोति। ते दे॒वा गृ॑हमे॒धीये॑ने॒ष्ट्वा। इन्द्रा॑य निष्का॒सन्न्य॑दधुः। अ॒स्माने॒व श्व इन्द्रो॒ निहि॑तभाग उपावर्ति॒तेति॑। तानिन्द्रो॒ निहि॑तभाग उ॒पाव॑र्तत। गृ॒ह॒मे॒धीये॑ने॒ष्ट्वा। इन्द्रा॑य निष्का॒सं निद॑ध्यात्। इन्द्र॑ ए॒वैनं॒ निहि॑तभाग उ॒पाव॑र्तते। गार्\mbox{}ह॑पत्ये जुहोति॥४८॥

%1.6.7.4
भा॒ग॒धेये॑नै॒वैन॒ सम॑र्धयति। ऋ॒ष॒भमाह्व॑यति। व॒ष॒ट्का॒र ए॒वास्य॒ सः। अथो॑ इन्द्रि॒यमे॒व तद्वी॒र्यं॑ यज॑मानो॒ भ्रातृव्य॑स्य वृङ्क्ते। इन्द्रो॑ वृ॒त्र ह॒त्वा। परां परा॒वत॑मगच्छत्। अपा॑राध॒मिति॒ मन्य॑मानः। सोऽब्रवीत्। क इ॒दं वे॑दिष्य॒तीति॑। तेऽब्रुवन्म॒रुतो॒ वरं॑ वृणामहै॥४९॥

%1.6.7.5
अथ॑ व॒यं वे॑दाम। अ॒स्मभ्य॑मे॒व प्र॑थ॒म ह॒विर्निरु॑प्याता॒ इति॑। त ए॑न॒मध्य॑क्रीडन्। तत्क्री॒डिनाङ्क्रीडि॒त्वम्। यन्म॒रुद्भ्य॑ क्री॒डिभ्य॑ प्रथ॒म ह॒विर्नि॑रु॒प्यते॒ विजि॑त्यै। सा॒क सूर्ये॑णोद्य॒ता निर्व॑पति। ए॒तस्मि॒न्वै लो॒क इन्द्रो॑ वृ॒त्रम॑ह॒न्त्समृ॑द्ध्यै। ए॒तद्ब्राह्मणान्ये॒व पञ्च॑ ह॒वीषि॑। ए॒तद्ब्राह्मण ऐन्द्रा॒ग्नः। अथै॒ष ऐ॒न्द्रश्च॒रुर्भ॑वति॥५०॥

%1.6.7.6
उ॒द्धारं वा ए॒तमिन्द्र॒ उद॑हरत। वृ॒त्र ह॒त्वा। अ॒न्यासु॑ दे॒वता॒स्वधि॑। यदे॒ष ऐ॒न्द्रश्च॒रुर्भव॑ति। उ॒द्धा॒रमे॒व तं यज॑मान॒ उद्ध॑रते। अ॒न्यासु॑ प्र॒जास्वधि॑। वै॒श्व॒क॒र्म॒ण एक॑कपालो भवति। विश्वान्ये॒व तेन॒ कर्मा॑णि॒ यज॑मा॒नोऽव॑रुन्धे॥५१॥\anuvakamend[ऋ॒द्ध्य॒ते॒ऽभ्य॑ञ्जते जुहोति वृणामहै भवत्य॒ष्टौ च॑]

%1.6.8.1
वै॒श्व॒दे॒वेन॒ वै प्र॒जाप॑तिः प्र॒जा अ॑सृजत। ता व॑रुणप्रघा॒सैर्व॑रुणपा॒शाद॑मुञ्चत्। सा॒क॒मे॒धैः प्रत्य॑स्थापयत्। त्र्य॑म्बकै रु॒द्रं नि॒रवा॑दयत। पि॒तृ॒य॒ज्ञेन॑ सुव॒र्गं लो॒कम॑गमयत्। यद्वैश्वदे॒वेन॒ यज॑ते। प्र॒जा ए॒व तद्यज॑मानः सृजते। ता व॑रुणप्रघा॒सैर्व॑रुणपा॒शान्मु॑ञ्चति। सा॒क॒मे॒धैः प्रति॑ष्ठापयति। त्र्य॑म्बकै रु॒द्रं नि॒रव॑दयते॥५२॥

%1.6.8.2
पि॒तृ॒य॒ज्ञेन॑ सुव॒र्गं लो॒कं ग॑मयति। द॒क्षि॒ण॒तः प्रा॑चीनावी॒ती निर्व॑पति। द॒क्षि॒णावृ॒द्धि पि॑तृ॒णाम्। अना॑दृत्य॒ तत्। उ॒त्त॒र॒त ए॒वोप॒वीय॒ निर्व॑पेत्। उ॒भये॒ हि दे॒वाश्च॑ पि॒तर॑श्चे॒ज्यन्ते। अथो॒ यदे॒व द॑क्षिणा॒र्धे॑ऽधि॒ श्रय॑ति। तेन॑ दक्षि॒णावृ॑त्। सोमा॑य पितृ॒मते॑ पुरो॒डाश॒ षट्क॑पालं॒ निर्व॑पति। सं॒व॒त्स॒रो वै सोम॑ पितृ॒मान्॥५३॥

%1.6.8.3
सं॒व॒त्स॒रमे॒व प्री॑णाति। पि॒तृभ्यो॑ बर्\mbox{}हि॒षद्भ्यो॑ धा॒नाः। मासा॒ वै पि॒तरो॑ बर्\mbox{}हि॒षद॑। मासा॑ने॒व प्री॑णाति। यस्मि॒न्वा ऋ॒तौ पुरु॑षः प्र॒मीय॑ते। सोऽस्या॒मुष्मि॑ल्लोँ॒के भ॑वति। ब॒हु॒रू॒पा धा॒ना भ॑वन्ति। अ॒हो॒रा॒त्राणा॑म॒भिजि॑त्यै। पि॒तृभ्योऽग्निष्वा॒त्तेभ्यो॑ म॒न्थम्। अ॒र्ध॒मा॒सा वै पि॒तरोऽग्निष्वा॒त्ताः॥५४॥

%1.6.8.4
अ॒र्ध॒मा॒साने॒व प्री॑णाति। अ॒भि॒वा॒न्या॑यै दु॒ग्धे भ॑वति। सा हि पि॑तृदेव॒त्य॑न्दु॒हे। यत्पू॒र्णम्। तन्म॑नु॒ष्या॑णाम्। उ॒प॒र्य॒र्धो दे॒वानाम्। अ॒र्धः पि॑तृ॒णाम्। अ॒र्ध उप॑मन्थति। अ॒र्धो हि पि॑तृ॒णाम्। एक॒योप॑मन्थति॥५५॥

%1.6.8.5
एका॒ हि पि॑तृ॒णाम्। द॒क्षि॒णोप॑मन्थति। द॒क्षि॒णावृ॒द्धि पि॑तृ॒णाम्। अना॑र॒भ्योप॑मन्थति। तद्धि पि॒तॄन्गच्छ॑ति। इ॒मान्दिशं॒ वेदि॒मुद्ध॑न्ति। उ॒भये॒ हि दे॒वाश्च॑ पि॒तर॑श्चे॒ज्यन्ते। चतु॑ स्रक्तिर्भवति। सर्वा॒ ह्यनु॒ दिश॑ पि॒तर॑। अखा॑ता भवति॥५६॥

%1.6.8.6
खा॒ता हि दे॒वानाम्। म॒ध्य॒तोऽग्निराधी॑यते। अ॒न्त॒तो हि दे॒वाना॑माधी॒यते। वर्\mbox{}षी॑यानि॒ध्म इ॒ध्माद्भ॑वति॒ व्यावृ॑त्त्यै। परि॑श्रयति। अ॒न्तर्\mbox{}हि॑तो॒ हि पि॑तृलो॒को म॑नुष्यलो॒कात्। यत्परु॑षि दि॒नम्। तद्दे॒वानाम्। यद॑न्त॒रा। तन्म॑नु॒ष्या॑णाम्॥५७॥

%1.6.8.7
यत्समू॑लम्। तत्पि॑तृ॒णाम्। समू॑लं ब॒र्॒हिर्भ॑वति॒ व्यावृ॑त्त्यै। द॒क्षि॒णा स्तृ॑णाति। द॒क्षि॒णावृ॒द्धि पि॑तृ॒णाम्। त्रिः पर्ये॑ति। तृ॒तीये॒ वा इ॒तो लो॒के पि॒तर॑। ताने॒व प्री॑णाति। त्रिः पुन॒ पर्ये॑ति। षट्त्सं प॑द्यन्ते॥५८॥

%1.6.8.8
षड्वा ऋ॒तव॑। ऋ॒तूने॒व प्री॑णाति। यत्प्र॑स्त॒रं यजु॑षा गृह्णी॒यात्। प्र॒मायु॑को॒ यज॑मानः स्यात्। यन्न गृ॑ह्णी॒यात्। अ॒ना॒य॒त॒नः स्यात्। तू॒ष्णीमे॒व न्य॑स्येत्। न प्र॒मायु॑को॒ भव॑ति। नाना॑यत॒नः। यत्रीन्प॑रि॒धीन्प॑रिद॒ध्यात्॥५९॥

%1.6.8.9
मृ॒त्युना॒ यज॑मानं॒ परि॑गृह्णीयात्। यन्न प॑रिद॒ध्यात्। रक्षासि य॒ज्ञ ह॑न्युः। द्वौ प॑रि॒धी परि॑दधाति। रक्ष॑सा॒मप॑हत्यै। अथो॑ मृ॒त्योरे॒व यज॑मान॒मुत्सृ॑जति। यत्रीणि॑ त्रीणि ह॒वीष्यु॑दा॒हरे॑युः। त्रय॑स्त्रय एषा सा॒कं प्रमी॑येरन्। एकै॑कमनू॒चीनान्यु॒दाह॑रन्ति। एकै॑क ए॒वैषा॑म॒न्वञ्च॒ प्रमी॑यते। क॒शिपु॑ कशिप॒व्या॑य। उ॒प॒बर्\mbox{}ह॑णमुपबर्\mbox{}ह॒ण्या॑य। आञ्ज॑नमाञ्ज॒न्या॑य। अ॒भ्यञ्ज॑नमभ्यञ्ज॒न्या॑य। य॒था॒भा॒गमे॒वैनान्प्रीणाति॥६०॥\anuvakamend[नि॒रव॑दयते पितृ॒मान॑ग्निष्वा॒त्ता एक॒योप॑ मन्थ॒त्यखा॑ता भवति मनु॒ष्या॑णां पद्यन्ते परिद॒ध्यान्मी॑यते॒ पञ्च॑ च]

%1.6.9.1
अ॒ग्नये॑ दे॒वेभ्य॑ पि॒तृभ्य॑ समि॒ध्यमा॑ना॒यानु॑ ब्रू॒हीत्या॑ह। उ॒भये॒ हि दै॒वाश्च॑ पि॒तर॑श्चे॒ज्यन्ते। एका॒मन्वा॑ह। एका॒ हि पि॑तृ॒णाम्। त्रिरन्वा॑ह। त्रिर्\mbox{}हि दे॒वानाम्। आ॒घा॒रावाघा॑रयति। य॒ज्ञ॒प॒रुषो॒रन॑न्तरित्यै। नार्\mbox{}षे॒यं वृ॑णीते। न होता॑रम्॥६१॥

%1.6.9.2
यदा॑र्\mbox{}षे॒यं वृ॑णी॒त। यद्धोता॑रम्। प्र॒मायु॑को॒ यज॑मानः स्यात्। प्र॒मायु॑को॒ होता। तस्मा॒न्न वृ॑णीते। यज॑मानस्य॒ होतु॑र्गोपी॒थाय॑। अप॑ बर्\mbox{}हिषः प्रया॒जान् य॑जति। प्र॒जा वै ब॒र्॒हिः। प्र॒जा ए॒व मृ॒त्योरुत्सृ॑जति। आज्य॑भागौ यजति॥६२॥

%1.6.9.3
य॒ज्ञस्यै॒व चक्षु॑षी॒ नान्तरे॑ति। प्रा॒ची॒ना॒वी॒ती सोमं॑ यजति। पि॒तृ॒दे॒व॒त्या॑ हि। ए॒षाऽऽहु॑तिः। पञ्च॒कृत्वोऽव॑ द्यति। पञ्च॒ ह्ये॑ता दे॒वता। द्वे पु॑रोऽनुवा॒क्ये। या॒ज्या॑ दे॒वता॑ वषट्का॒रः। ता ए॒व प्री॑णाति। सन्त॑त॒मव॑ द्यति॥६३॥

%1.6.9.4
ऋ॒तू॒ना सन्त॑त्यै। प्रैवैभ्य॒ पूर्व॑या पुरोऽनुवा॒क्य॑याऽऽह। प्रण॑यति द्वि॒तीय॑या। ग॒मय॑ति या॒ज्य॑या। तृ॒तीये॒ वा इ॒तो लो॒के पि॒तर॑। अह्न॑ ए॒वैना॒न्पूर्व॑या पुरोऽनुवा॒क्य॑या॒ऽत्यान॑यति। रात्रि॑यै द्वि॒तीय॑या। ऐवैनान्॑ या॒ज्य॑या गमयति। द॒क्षि॒ण॒तो॑ऽव॒दाय॑। उद॒ङ्ङति॑ क्रामति॒ व्यावृ॑त्त्यै॥६४॥

%1.6.9.5
आ स्व॒धेत्याश्रा॑वयति। अस्तु॑ स्व॒धेति॑ प्र॒त्याश्रा॑वयति। स्व॒धा नम॒ इति॒ वष॑ट्करोति। स्व॒धा॒का॒रो हि पि॑तृ॒णाम्। सोम॒मग्रे॑ यजति। सोम॑प्रयाजा॒ हि पि॒तर॑। सोमं॑ पितृ॒मन्तं॑ यजति। सं॒व॒त्स॒रो वै सोम॑ पितृ॒मान्। सं॒व॒त्स॒रमे॒व तद्य॑जति। पि॒तॄन्ब॑र्\mbox{}हि॒षदो॑ यजति॥६५॥

%1.6.9.6
ये वै यज्वा॑नः। ते पि॒तरो॑ बर्\mbox{}हि॒षद॑। ताने॒व तद्य॑जति। पि॒तॄन॑ग्निष्वा॒त्तान् य॑जति। ये वा अय॑ज्वानो गृहमे॒धिन॑। ते पि॒तरोऽग्निष्वा॒त्ताः। ताने॒व तद्य॑जति। अ॒ग्निङ्क॑व्य॒वाह॑नं यजति। य ए॒व पि॑तृ॒णाम॒ग्निः। तमे॒व तद्य॑जति॥६६॥

%1.6.9.7
अथो॒ यथा॒ऽग्नि स्वि॑ष्ट॒कृतं॒ यज॑ति। ता॒दृगे॒व तत्। ए॒तत्ते॑ तत॒ ये च॒ त्वामन्विति॑ ति॒सृषु॑ स्र॒क्तीषु॒ निद॑धाति। तस्मा॒दा तृ॒तीया॒त्पुरु॑षा॒न्नाम॒ न गृ॑ह्णन्ति। ए॒ताव॑न्तो॒ हीज्यन्ते। अत्र॑ पितरो यथाभा॒गं म॑न्दध्व॒मित्या॑ह। ह्लीका॒ हि पि॒तर॑। उद॑ञ्चो॒ निष्क्रा॑मन्ति। ए॒षा वै म॑नु॒ष्या॑णा॒न्दिक्। स्वामे॒व तद्दिश॒मनु॒ निष्क्रा॑मन्ति॥६७॥

%1.6.9.8
आ॒ह॒व॒नीय॒मुप॑तिष्ठन्ते। न्ये॑वास्मै॒ तद्ध्नु॑वते। यत्स॒त्या॑हव॒नीये। अथा॒न्यत्र॒ चर॑न्ति। आतमि॑तो॒रुप॑तिष्ठन्ते। अ॒ग्निमे॒वोप॑द्र॒ष्टारं॑ कृ॒त्वा। पि॒तॄन्नि॒रव॑दयन्ते। अन्तं॒ वा ए॒ते प्रा॒णानां गच्छन्ति। य आतमि॑तोरुप॒ तिष्ठ॑न्ते। सु॒स॒न्दृश॑न्त्वा व॒यमित्या॑ह॥६८॥

%1.6.9.9
प्रा॒णो वै सु॑स॒न्दृक्। प्रा॒णमे॒वात्मन्द॑धते। योजा॒ न्वि॑न्द्र ते॒ हरी॒ इत्या॑ह। प्रा॒णमे॒व पुन॑रयुक्त। अक्ष॒न्नमी॑मदन्त॒ हीति॒ गार्\mbox{}ह॑पत्य॒मुप॑तिष्ठन्ते। अक्ष॒न्नमी॑मद॒न्ताथ॒ त्वोप॑तिष्ठामह॒ इति॒ वावैतदा॑ह। अमी॑मदन्त पि॒तर॑ सो॒म्या इत्य॒भि प्रप॑द्यन्ते। अमी॑मदन्त पि॒तरोऽथ॑ त्वा॒ऽभि प्रप॑द्यामह॒ इति॒ वावैतदा॑ह। अ॒पः परि॑षिञ्चति। मा॒र्जय॑त्ये॒वैनान्॑॥६९॥

%1.6.9.10
अथो॑ त॒र्पय॑त्ये॒व। तृप्य॑ति प्र॒जया॑ प॒शुभि॑। य ए॒वं वेद॑। अप॑ बर्\mbox{}हिषावनूया॒जौ य॑जति। प्र॒जा वै ब॒र्॒हिः। प्र॒जा ए॒व मृ॒त्योरुत्सृ॑जति। च॒तुर॑ प्रया॒जान् य॑जति। द्वाव॑नूया॒जौ। षट्त्सं प॑द्यन्ते। षड्वा ऋ॒तव॑। ऋ॒तूने॒व प्री॑णाति। न पत्न्यन्वास्ते। न संया॑जयन्ति। यत्पत्न्य॒न्वासी॑त। यत्सं॑या॒जये॑युः। प्र॒मायु॑का स्यात्। तस्मा॒न्नान्वास्ते। न संया॑जयन्ति। पत्नि॑यै गोपी॒थाय॑॥७०॥\anuvakamend[होता॑र॒माज्य॑भागौ यजति॒ सन्त॑त॒मव॑द्यति॒ व्यावृ॑त्त्यै बर्\mbox{}हि॒षदो॑ यजति॒ तमे॒व तद्य॑ज॒त्यनु॒ निष्क्रा॑मन्त्याहैनानृ॒तवो॒ नव॑ च]

%1.6.10.1
प्र॒ति॒पू॒रु॒षमेक॑कपालां॒ निर्व॑पति। जा॒ता ए॒व प्र॒जा रु॒द्रान्नि॒रव॑दयते। एक॒मति॑रिक्तम्। ज॒नि॒ष्यमा॑णा ए॒व प्र॒जा रु॒द्रान्नि॒रव॑दयते। एक॑कपाला भवन्ति। ए॒क॒धैव रु॒द्रन्नि॒रव॑दयते। नाभिघा॑रयति। यद॑भिघा॒रयेत्। अ॒न्त॒र॒व॒चा॒रिण रु॒द्रं कु॑र्यात्। ए॒को॒ल्मु॒केन॑ यन्ति॥७१॥

%1.6.10.2
तद्धि रु॒द्रस्य॑ भाग॒धेयम्। इ॒मान्दिशं॑ यन्ति। ए॒षा वै रु॒द्रस्य॒ दिक्। स्वाया॑मे॒व दि॒शि रु॒द्रन्नि॒रव॑दयते। रु॒द्रो वा अ॑प॒शुका॑या॒ आहु॑त्यै॒ नाति॑ष्ठत। अ॒सौ ते॑ प॒शुरिति॒ निर्दि॑शे॒द्यं द्वि॒ष्यात्। यमे॒व द्वेष्टि॑। तम॑स्मै प॒शुं निर्दि॑शति। यदि॒ न द्वि॒ष्यात्। आ॒खुस्ते॑ प॒शुरिति॑ ब्रूयात्॥७२॥

%1.6.10.3
न ग्रा॒म्यान्प॒शून् हि॒नस्ति॑। नार॒ण्यान्। च॒तु॒ष्प॒थे जु॑होति। ए॒ष वा अ॑ग्नी॒नां पड्बी॑शो॒ नाम॑। अ॒ग्नि॒वत्ये॒व जु॑होति। म॒ध्य॒मेन॑ प॒र्णेन॑ जुहोति। स्रुग्घ्ये॑षा। अथो॒ खलु॑। अ॒न्त॒मेनै॒व हो॑त॒व्यम्। अ॒न्त॒त ए॒व रु॒द्रं नि॒रव॑दयते॥७३॥

%1.6.10.4
ए॒ष ते॑ रुद्र भा॒गः स॒ह स्वस्राऽम्बि॑क॒येत्या॑ह। श॒रद्वा अ॒स्याम्बि॑का॒ स्वसा। तया॒ वा ए॒ष हि॑नस्ति। य हि॒नस्ति॑। तयै॒वैन स॒ह श॑मयति। भे॒ष॒जङ्गव॒ इत्या॑ह। याव॑न्त ए॒व ग्रा॒म्याः प॒शव॑। तेभ्यो॑ भेष॒जं क॑रोति। अवाम्ब रु॒द्रम॑दिम॒हीत्या॑ह। आ॒शिष॑मे॒वैतामा शास्ते॥७४॥

%1.6.10.5
त्र्य॑म्बकं यजामह॒ इत्या॑ह। मृ॒त्योर्मु॑क्षीय॒ माऽमृता॒दिति॒ वावैतदा॑ह। उत्कि॑रन्ति। भग॑स्य लीप्सन्ते। मूते॑कृ॒त्वाऽऽस॑जन्ति। यथा॒ जनं॑ य॒ते॑ऽव॒सं क॒रोति॑। ता॒दृगे॒व तत्। ए॒ष ते॑ रुद्र भा॒ग इत्या॑ह नि॒रव॑त्त्यै। अप्र॑तीक्ष॒मा य॑न्ति। अ॒पः परि॑षिञ्चति। रु॒द्रस्या॒न्तर्\mbox{}हि॑त्यै। प्र वा ए॒तेऽस्माल्लो॒काच्च्य॑वन्ते। ये त्र्य॑म्बकै॒श्चर॑न्ति। आ॒दि॒त्यञ्च॒रुं पुन॒रेत्य॒ निर्व॑पति। इ॒यं वा अदि॑तिः। अ॒स्यामे॒व प्रति॑ तिष्ठन्ति॥७५॥\anuvakamend[य॒न्ति॒ ब्रू॒या॒न्नि॒रव॑दयते शास्ते सिञ्चति॒ षट्च॑]




\prashnaend{अनु॑मत्यै वैश्वदे॒वेन॒ ताः सृ॒ष्टास्त्रि॒वृत्प्र॒जाप॑तिः सवि॒तोत्त॑रस्यान्देवासु॒राः सोऽग्निर्यत्पत्नी॑ वैश्वदे॒वेन॒ ता व॑रुणप्रघा॒सैर॒ग्नये॑ दे॒वेभ्य॑ प्रतिपूरु॒षन्दश॑॥१०॥}{अनु॑मत्यै प्रथम॒जो व॒त्सो ब॑हुरू॒पा हि प॒शव॒स्तस्मात्पृथमा॒त्रं यद॒ग्नयेऽनी॑कवत उद्धा॒रं वा अ॒ग्नये॑ दे॒वेभ्य॑ प्रतिपूरु॒षं पञ्च॑सप्ततिः॥७५॥}{अनु॑मत्यै॒ प्रति॑तिष्ठन्ति॥}{हरि॑ ओम्॥}{इति श्रीकृष्णयजुर्वेदीयतैत्तिरीयब्राह्मणे प्रथमाष्टके षष्ठः प्रपाठकः समाप्तः॥}
\clearpage
\sect{सप्तमः प्रश्नः}
\setcounter{anuvakam}{0}
\dnsub{तैत्तिरीयब्राह्मणे प्रथमाष्टके सप्तमः प्रपाठकः}

%सं॒व॒त्स॒        सं॒व॒त्स॒
%1.7.1.1
ए॒तद्ब्राह्मणान्ये॒व पञ्च॑ ह॒वीषि॑। अथेन्द्रा॑य॒ शुना॒सीरा॑य पुरो॒डाश॒न्द्वाद॑शकपालं॒ निर्व॑पति। सं॒व॒त्स॒रो वा इन्द्रा॒शुना॒सीर॑। सं॒व॒त्स॒रेणै॒वास्मा॒ अन्न॒मव॑ रुन्धे। वा॒य॒व्यं॑ पयो॑ भवति। वा॒युर्वै वृष्ट्यै प्रदापयि॒ता। स ए॒वास्मै॒ वृष्टिं॒ प्रदा॑पयति। सौ॒र्य॑ एक॑कपालो भवति। सूर्ये॑ण॒ वा अ॒मुष्मिँ॑ल्लो॒के वृष्टि॑र्धृ॒ता। स ए॒वास्मै॒ वृष्टिं॒ निय॑च्छति॥१॥

%1.7.1.2
द्वा॒द॒श॒ग॒व सीर॒न्दक्षि॑णा॒ समृ॑द्ध्यै। दे॒वा॒सु॒राः संय॑त्ता आसन्। ते दे॒वा अ॒ग्निम॑ब्रुवन्। त्वया॑ वी॒रेणासु॑रान॒भिभ॑वा॒मेति॑। सोऽब्रवीत्। त्रे॒धाऽहमा॒त्मानं॒ विक॑रिष्य॒ इति॑। स त्रे॒धाऽऽत्मानं॒ व्य॑कुरुत। अ॒ग्निन्तृती॑यम्। रु॒द्रन्तृती॑यम्। वरु॑ण॒न्तृती॑यम्॥२॥

%1.7.1.3
सोऽब्रवीत्। क इ॒दन्तु॒रीय॒मिति॑। अ॒हमितीन्द्रोऽब्रवीत्। सन्तु सृ॑जावहा॒ इति॑। तौ सम॑सृजेताम्। स इन्द्र॑स्तु॒रीय॑मभवत्। यदिन्द्र॑स्तु॒रीय॒मभ॑वत्। तदि॑न्द्रतुरी॒यस्येन्द्रतुरीय॒त्वम्। ततो॒ वै दे॒वा व्य॑जयन्त। यदि॑न्द्रतुरी॒यं नि॑रु॒प्यते॒ विजि॑त्यै॥३॥

%1.7.1.4
व॒हिनी॑ धे॒नुर्दक्षि॑णा। यद्व॒हिनी। तेनाग्ने॒यी। यद्गौः। तेन॑ रौ॒द्री। यद्धे॒नुः। तेनै॒न्द्री। यत्स्त्री स॒ती दा॒न्ता। तेन॑ वारु॒णी समृ॑द्ध्यै। प्र॒जाप॑तिर्य॒ज्ञम॑सृजत॥४॥

%1.7.1.5
त सृ॒ष्ट रक्षास्यजिघासन्। स ए॒ताः प्र॒जाप॑तिरा॒त्मनो॑ दे॒वता॒ निर॑मिमीत। ताभि॒र्वै स दि॒ग्भ्यो रक्षासि॒ प्राणु॑दत। यत्प॑ञ्चाव॒त्तीयं॑ जु॒होति॑। दि॒ग्भ्य ए॒व तद्यज॑मानो॒ रक्षासि॒ प्रणु॑दते। समू॑ढ॒ रक्ष॒ सन्द॑ग्ध॒ रक्ष॒ इत्या॑ह। रक्षास्ये॒व सन्द॑हति। अ॒ग्नये॑ रक्षो॒घ्ने स्वाहेत्या॑ह। दे॒वताभ्य ए॒व वि॑जिग्या॒नाभ्यो॑ भाग॒धेयं॑ करोति। प्र॒ष्टि॒वा॒ही रथो॒ दक्षि॑णा॒ समृ॑द्ध्यै॥५॥

%1.7.1.6
इन्द्रो॑ वृ॒त्र ह॒त्वा। असु॑रान्परा॒भाव्य॑। नमु॑चिमासु॒रन्नाल॑भत। तश॒च्या॑ऽगृह्णात्। तौ सम॑लभेताम्। सोऽस्माद॒भिशु॑नतरोऽभवत्। सोऽब्रवीत्। स॒न्धा सन्द॑धावहै। अथ॒ त्वाऽव॑ स्रक्ष्यामि। न मा॒ शुष्के॑ण॒ नार्द्रेण॑ हनः॥६॥

%1.7.1.7
न दिवा॒ न नक्त॒मिति॑। स ए॒तम॒पां फेन॑मसिञ्चत्। न वा ए॒ष शुष्को॒ नार्द्रो व्यु॑ष्टाऽऽसीत्। अनु॑दित॒ सूर्य॑। न वा ए॒तद्दिवा॒ न नक्तम्। तस्यै॒तस्मि॑ल्लोँ॒के। अ॒पां फेने॑न॒ शिर॒ उद॑वर्तयत्। तदे॑न॒मन्व॑वर्तत। मित्र॑द्रु॒गति॑ ॥७॥

%1.7.1.8
स ए॒तान॑पामा॒र्गान॑जनयत्। तान॑जुहोत्। तैर्वै स रक्षा॒स्यपा॑हत। यद॑पामार्गहो॒मो भव॑ति। रक्ष॑सा॒मप॑हत्यै। ए॒को॒ल्मु॒केन॑ यन्ति। तद्धि रक्ष॑सां भाग॒धेयम्। इ॒मान्दिशं॑ यन्ति। ए॒षा वै रक्ष॑सा॒न्दिक्। स्वाया॑मे॒व दि॒शि रक्षासि हन्ति॥८॥

%1.7.1.9
स्वकृ॑त॒ इरि॑णे जुहोति प्रद॒रे वा। ए॒तद्वै रक्ष॑सामा॒यतन॑म्। स्व ए॒वायत॑ने॒ रक्षासि हन्ति। प॒र्ण॒मये॑न स्रु॒वेण॑ जुहोति। ब्रह्म॒ वै प॒र्णः। ब्रह्म॑णै॒व रक्षासि हन्ति। दे॒वस्य॑ त्वा सवि॒तुः प्र॑स॒व इत्या॑ह। स॒वि॒तृप्र॑सूत ए॒व रक्षासि हन्ति। ह॒त रक्षोऽव॑धिष्म॒ रक्ष॒ इत्या॑ह। रक्ष॑सा॒ स्तृत्यै। यद्वस्ते॒ तद्दक्षि॑णा नि॒रव॑त्यै। अप्र॑तीक्ष॒माय॑न्ति। रक्ष॑साम॒न्तर्‌हि॑त्यै॥९॥\anuvakamend[य॒च्छ॒ति॒ वरु॑ण॒न्तृती॑यं॒ विजि॑त्या असृजत॒ समृ॑द्ध्यै हनो॒ मित्र॑द्रु॒गिति॑ हन्ति॒ स्तृत्यै॒ त्रीणि॑ च]

%1.7.2.1
धा॒त्रे पु॑रो॒डाश॒न्द्वाद॑शकपालं॒ निर्व॑पति। सं॒व॒त्स॒रो वै धा॒ता। सं॒व॒त्स॒रेणै॒वास्मै प्र॒जाः प्रज॑नयति। अन्वे॒वास्मा॒ अनु॑मतिर्मन्यते। रा॒ते रा॒का। प्र सि॑नीवा॒ली ज॑नयति। प्र॒जास्वे॒व प्रजा॑तासु कु॒ह्वा॑ वाच॑न्दधाति। मि॒थु॒नौ गावौ॒ दक्षि॑णा॒ समृ॑द्ध्यै। आ॒ग्ना॒वै॒ष्ण॒वमेका॑दशकपालं॒ निर्व॑पति। ऐ॒न्द्रा॒वै॒ष्ण॒वमेका॑दशकपालम्॥१०॥

%1.7.2.2
वै॒ष्ण॒वन्त्रि॑कपा॒लम्। वी॒र्यं॑ वा अ॒ग्निः। वी॒र्य॑मिन्द्र॑। वी॒र्यं॑ विष्णु॑। प्र॒जा ए॒व प्रजा॑ता वी॒र्ये प्रति॑ष्ठापयति। तस्मात्प्र॒जा वी॒र्या॑वतीः। वा॒म॒न ऋ॑ष॒भो व॒ही दक्षि॑णा। यद्व॒ही। तेनाग्ने॒यः। यदृ॑ष॒भः॥११॥

%1.7.2.3
तेनै॒न्द्रः। यद्वा॑म॒नः। तेन॑ वैष्ण॒वः समृ॑द्ध्यै। अ॒ग्नी॒षो॒मीय॒मेका॑दशकपालं॒ निर्व॑पति। इ॒न्द्रा॒सो॒मीय॒मेका॑दशकपालम्। सौ॒म्यञ्च॒रुम्। सोमो॒ वै रे॑तो॒धाः। अ॒ग्निः प्र॒जानां प्रजनयि॒ता। वृ॒द्धाना॒मिन्द्र॑ प्रदापयि॒ता। सोम॑ ए॒वास्मै॒ रेतो॒ दधा॑ति॥१२॥

%1.7.2.4
अ॒ग्निः प्र॒जां प्रज॑नयति। वृ॒द्धामिन्द्र॒ प्रय॑च्छति। ब॒भ्रुर्दक्षि॑णा॒ समृ॑द्ध्यै। सो॒मा॒पौ॒ष्णञ्च॒रुन्निर्व॑पति। ऐ॒न्द्रा॒पौ॒ष्णञ्च॒रुम्। सोमो॒ वै रे॑तो॒धाः। पू॒षा प॑शू॒नां प्र॑जनयि॒ता। वृ॒द्धाना॒मिन्द्र॑ प्रदापयि॒ता। सोम॑ ए॒वास्मै॒ रेतो॒ दधा॑ति। पू॒षा प॒शून्प्रज॑नयति॥१३॥

%1.7.2.5
वृ॒द्धानिन्द्र॒ प्रय॑च्छति। पौ॒ष्णश्च॒रुर्भ॑वति। इ॒यं वै पू॒षा। अ॒स्यामे॒व प्रति॑तिष्ठति। श्या॒मो दक्षि॑णा॒ समृ॑द्ध्यै। ब॒हु वै पुरु॑षो मे॒ध्यमुप॑गच्छति। वै॒श्वा॒न॒रन्द्वाद॑शकपालं॒ निर्व॑पति। सं॒व॒त्स॒रो वा अ॒ग्निर्वैश्वान॒रः। सं॒व॒त्स॒रेणै॒वैन स्वदयति। हिर॑ण्य॒न्दक्षि॑णा॥१४॥

%1.7.2.6
प॒वित्रं॒ वै हिर॑ण्यम्। पु॒नात्ये॒वैनम्। ब॒हु वै रा॑ज॒न्योऽनृ॑तं करोति। उप॑ जा॒म्यै हर॑ते। जि॒नाति॑ ब्राह्म॒णम्। वद॒त्यनृ॑तम्। अनृ॑ते॒ खलु॒ वै क्रि॒यमा॑णे॒ वरु॑णो गृह्णाति। वा॒रु॒णं य॑व॒मयं॑ च॒रुन्निर्व॑पति। व॒रु॒ण॒पा॒शादे॒वैनं॑ मुञ्चति। अश्वो॒ दक्षि॑णा। वा॒रु॒णो हि दे॒वत॒याऽश्व॒ समृ॑द्ध्यै॥१५॥\anuvakamend[ऐ॒न्द्रा॒वै॒ष्ण॒वमेका॑दशकपालं॒ यदृ॑ष॒भो दधा॑ति पू॒षा प॒शून्प्रज॑नयति॒ हिर॑ण्य॒न्दक्षि॑णा॒ दक्षि॒णैकं च]

%1.7.3.1
र॒त्निना॑मे॒तानि॑ ह॒वीषि॑ भवन्ति। ए॒ते वै रा॒ष्ट्रस्य॑ प्रदा॒तार॑। ए॒ते॑ऽपादा॒तार॑। य ए॒व रा॒ष्ट्रस्य॑ प्रदा॒तार॑। ये॑ऽपादा॒तार॑। त ए॒वास्मै॑ रा॒ष्ट्रं प्रय॑च्छन्ति। रा॒ष्ट्रमे॒व भ॑वति। यत्स॑मा॒हृत्य॑ नि॒र्वपेत्। अर॑त्निनः स्युः। य॒था॒य॒थन्निर्व॑पति रत्नि॒त्वाय॑॥१६॥

%1.7.3.2
यत्स॒द्यो नि॒र्वपेत्। याव॑ती॒मेके॑न ह॒विषा॒ऽऽशिष॑मव रु॒न्धे। ताव॑ती॒मव॑रुन्धीत। अ॒न्व॒हन्निर्व॑पति। भूय॑सीमे॒वाशिष॒मव॑ रुन्धे। भूय॑सो यज्ञक्र॒तूनुपै॑ति। बा॒र्॒ह॒स्प॒त्यञ्च॒रुन्निर्व॑पति ब्र॒ह्मणो॑ गृ॒हे। मु॒ख॒त ए॒वास्मै॒ ब्रह्म॒ सश्य॑ति। अथो॒ ब्रह्म॑न्ने॒व क्ष॒त्रम॒न्वार॑म्भयति। शि॒ति॒पृ॒ष्ठो दक्षि॑णा॒ समृ॑द्ध्यै॥१७॥

%1.7.3.3
ऐ॒न्द्रमेका॑दशकपाल राज॒न्य॑स्य गृ॒हे। इ॒न्द्रि॒यमे॒वाव॑ रुन्धे। ऋ॒ष॒भो दक्षि॑णा॒ समृ॑द्ध्यै। आ॒दि॒त्यञ्च॒रुं महि॑ष्यै गृ॒हे। इ॒यं वा अदि॑तिः। अ॒स्यामे॒व प्रति॑तिष्ठति। धे॒नुर्दक्षि॑णा॒ समृ॑द्ध्यै। भगा॑य च॒रुं वा॒वाता॑यै गृ॒हे। भग॑मे॒वास्मि॑न्दधाति। विचि॑त्तगर्भा पष्ठौ॒ही दक्षि॑णा॒ समृ॑द्ध्यै॥१८॥

%1.7.3.4
नै॒र्॒ऋ॒तञ्च॒रुं प॑रिवृ॒क्त्यै॑ गृ॒हे कृ॒ष्णानां व्रीही॒णान्न॒खनि॑र्भिन्नम्। पा॒प्मान॑मे॒व निर्\mbox{}ऋ॑तिन्नि॒रव॑दयते। कृ॒ष्णा कू॒टा दक्षि॑णा॒ समृ॑द्ध्यै। आ॒ग्ने॒यम॒ष्टाक॑पाल सेना॒न्यो॑ गृ॒हे। सेना॑मे॒वास्य॒ सश्य॑ति। हिर॑ण्य॒न्दक्षि॑णा॒ समृ॑द्ध्यै। वा॒रु॒णन्दश॑कपाल सू॒तस्य॑ गृ॒हे। व॒रु॒ण॒स॒वमे॒वाव॑ रुन्धे। म॒हानि॑रष्टो॒ दक्षि॑णा॒ समृ॑द्ध्यै। मा॒रु॒त स॒प्तक॑पालङ्ग्राम॒ण्यो॑ गृ॒हे॥१९॥

%1.7.3.5
अन्नं॒ वै म॒रुत॑। अन्न॑मे॒वाव॑ रुन्धे। पृश्ञि॒र्दक्षि॑णा॒ समृ॑द्ध्यै। सा॒वि॒त्रन्द्वाद॑शकपालङ्क्ष॒त्तुर्गृ॒हे प्रसूत्यै। उ॒प॒ध्व॒स्तो दक्षि॑णा॒ समृ॑द्ध्यै। आ॒श्वि॒नन्द्वि॑कपा॒ल स॑ङ्ग्रही॒तुर्गृ॒हे। अ॒श्विनौ॒ वै दे॒वानां भि॒षजौ। ताभ्या॑मे॒वास्मै॑ भेष॒जं क॑रोति। स॒वा॒त्यौ॑ दक्षि॑णा॒ समृ॑द्ध्यै। पौ॒ष्णञ्च॒रुं भा॑गदु॒घस्य॑ गृ॒हे॥२०॥

%1.7.3.6
अन्नं॒ वै पू॒षा। अन्न॑मे॒वाव॑ रुन्धे। श्या॒मो दक्षि॑णा॒ समृ॑द्ध्यै। रौ॒द्रङ्गा॑वीधु॒कञ्च॒रुम॑क्षावा॒पस्य॑ गृ॒हे। अ॒न्त॒त ए॒व रु॒द्रन्नि॒रव॑दयते। श॒बल॒ उद्वा॑रो॒ दक्षि॑णा॒ समृ॑द्ध्यै। द्वाद॑शै॒तानि॑ ह॒वीषि॑ भवन्ति। द्वाद॑श॒ मासा संवत्स॒रः। सं॒व॒त्स॒रेणै॒वास्मै॑ रा॒ष्ट्रमव॑रुन्धे। रा॒ष्ट्रमे॒व भ॑वति॥२१॥

%1.7.3.7
यन्न प्र॑ति नि॒र्वपेत्। र॒त्निन॑ आ॒शिषोऽव॑रुन्धीरन्। प्रति॒निर्व॑पति। इन्द्रा॑य सु॒त्राम्णे॑ पुरो॒डाश॒मेका॑दशकपालम्। इन्द्रा॑याहो॒मुचे। आ॒शिष॑ ए॒वाव॑रुन्धे। अ॒यन्नो॒ राजा॑ वृत्र॒हा राजा॑ भू॒त्वा वृ॒त्रं व॑ध्या॒दित्या॑ह। आ॒शिष॑मे॒वैतामा शास्ते। मै॒त्रा॒बा॒र्॒ह॒स्प॒त्यं भ॑वति। श्वे॒तायै श्वे॒तव॑त्सायै दु॒ग्धे॥२२॥

%1.7.3.8
बा॒र्॒ह॒स्प॒त्ये मै॒त्रमपि॑ दधाति। ब्रह्म॑ चै॒वास्मै क्ष॒त्रं च॑ स॒मीची॑ दधाति। अथो॒ ब्रह्म॑न्ने॒व क्ष॒त्रं प्रति॑ष्ठापयति। बा॒र्॒ह॒स्प॒त्येन॒ पूर्वे॑ण॒ प्रच॑रति। मु॒ख॒त ए॒वास्मै॒ ब्रह्म॒ सश्य॑ति। अथो॒ ब्रह्म॑न्ने॒व क्ष॒त्रम॒न्वार॑म्भयति। स्व॒यं॒ कृ॒ता वेदि॑र्भवति। स्व॒य॒न्दि॒नं ब॒र्॒हिः। स्व॒यं॒ कृ॒त इ॒ध्मः। अन॑भिजितस्या॒भिजि॑त्यै। तस्मा॒द्राज्ञा॒मर॑ण्यम॒भिजि॑तम्। सैव श्वे॒ता श्वे॒तव॑त्सा॒ दक्षि॑णा॒ समृ॑द्ध्यै॥२३॥\anuvakamend[र॒त्नि॒त्वाय॒ समृ॑द्ध्यै पष्ठौ॒ही दक्षि॑णा॒ समृ॑द्ध्यै ग्राम॒ण्यो॑ गृ॒हे भा॑गदु॒घस्य॑ गृ॒हे भ॑वति दु॒ग्धे॑ऽभिजि॑त्यै॒ द्वे च॑]
 
%1.7.4.1
दे॒व॒सु॒वामे॒तानि॑ ह॒वीषि॑ भवन्ति। ए॒ताव॑न्तो॒ वै दे॒वाना स॒वाः। त ए॒वास्मै॑ स॒वान्प्रय॑च्छन्ति। त ए॑न सुवन्ते। अ॒ग्निरे॒वैनं॑ गृ॒हप॑तीना सुवते। सोमो॒ वन॒स्पती॑नाम्। रु॒द्रः प॑शू॒नाम्। बृह॒स्पति॑र्वा॒चाम्। इन्द्रो ज्ये॒ष्ठानाम्। मि॒त्रः स॒त्यानाम्॥२४॥

%1.7.4.2
वरु॑णो॒ धर्म॑पतीनाम्। ए॒तदे॒व सर्वं॑ भवति। स॒वि॒ता त्वा प्रस॒वाना सुवता॒मिति॒ हस्त॑ङ्गृह्णाति॒ प्रसूत्यै। ये दे॑वा देव॒ सुव॒ स्थेत्या॑ह। य॒था॒य॒जुरे॒वैतत्। म॒ह॒ते क्ष॒त्राय॑ मह॒त आधि॑पत्याय मह॒ते जान॑राज्या॒येत्या॑ह। आ॒शिष॑मे॒वैतामा शास्ते। ए॒ष वो॑ भरता॒ राजा॒ सोमो॒ऽस्माकं॑ ब्राह्म॒णाना॒ राजेत्या॑ह। तस्मा॒त्सोम॑राजानो ब्राह्म॒णाः। प्रति॒ त्यन्नाम॑ रा॒ज्यम॑धा॒यीत्या॑ह॥२५॥

%1.7.4.3
रा॒ज्यमे॒वास्मि॒न्प्रति॑दधाति। स्वान्त॒नुवं॒ वरु॑णो अशिश्रे॒दित्या॑ह। व॒रु॒ण॒स॒वमे॒वाव॑रुन्धे। शुचेर्मि॒त्रस्य॒ व्रत्या॑ अभू॒मेत्या॑ह। शुचि॑मे॒वैनं॒ व्रत्यं॑ करोति। अम॑न्महि मह॒त ऋ॒तस्य॒ नामेत्या॑ह। म॒नु॒त ए॒वैनम्। सर्वे॒ व्राता॒ वरु॑णस्याभूव॒न्नित्या॑ह। सर्व॑व्रातमे॒वैनं॑ करोति। वि मि॒त्र एवै॒ररा॑तिमतारी॒दित्या॑ह॥२६॥

%1.7.4.4
अरा॑तिमै॒वैन॑न्तारयति। असू॑षुदन्त य॒ज्ञिया॑ ऋ॒तेनेत्या॑ह। स्व॒दय॑त्ये॒वैनम्। व्यु॑ त्रि॒तो ज॑रि॒माण॑न्न आन॒डित्या॑ह। आयु॑रे॒वास्मि॑न्दधाति। द्वाभ्यां॒ विमृ॑ष्टे। द्वि॒पाद्यज॑मान॒ प्रति॑ष्ठित्यै। अ॒ग्नी॒षो॒मीय॑स्य॒ चैका॑दशकपालस्य देवसु॒वां च॑ ह॒विषा॑म॒ग्नये स्विष्ट॒कृते॑ स॒मव॑द्यति। दे॒वता॑भिरे॒वैन॑मुभ॒यत॒ परि॑गृह्णाति। वि॒ष्णु॒क्र॒मान्क्र॑मते। विष्णु॑रे॒व भू॒त्वेमाल्लोँ॒कान॒भिज॑यति॥२७॥\anuvakamend[स॒त्याना॑मधा॒यीत्या॑हातारी॒दित्या॑ह क्रमत॒ एकं च]

%1.7.5.1
अ॒र्थेत॒ स्थेति॑ जुहोति। आहु॑त्यै॒वैना॑ नि॒ष्क्रीय॑ गृह्णाति। अथो॑ ह॒विष्कृ॑तानामे॒वाभिघृ॑तानाङ्गृह्णाति। वह॑न्तीनाङ्गृह्णाति। ए॒ता वा अ॒पा रा॒ष्ट्रम्। रा॒ष्ट्रमे॒वास्मै॑ गृह्णाति। अथो॒ श्रिय॑मे॒वैन॑म॒भिव॑हन्ति। अ॒पां पति॑र॒सीत्या॑ह। मि॒थु॒नमे॒वाक॑। वृषाऽस्यू॒र्मिरित्या॑ह॥२८॥

%1.7.5.2
ऊ॒र्मि॒मन्त॑मे॒वैनं॑ करोति। वृ॒ष॒से॒नो॑ऽसीत्या॑ह। सेना॑मे॒वास्य॒ सश्य॑ति। व्र॒ज॒क्षित॒ स्थेत्या॑ह। ए॒ता वा अ॒पां विश॑। विश॑मे॒वास्मै॒ पर्यू॑हति। म॒रुता॒मोज॒ स्थेत्या॑ह। अन्नं॒ वै म॒रुत॑। अन्न॑मे॒वाव॑रुन्धे। सूर्य॑वर्चस॒ स्थेत्या॑ह॥२९॥

%1.7.5.3
रा॒ष्ट्रमे॒व व॑र्च॒स्व्य॑कः। सूर्य॑त्वचस॒ स्थेत्या॑ह। स॒त्यं वा ए॒तत्। यद्वर्\mbox{}ष॑ति। अनृ॑तं॒ यदा॒तप॑ति॒ वर्\mbox{}ष॑ति। स॒त्या॒नृ॒ते ए॒वाव॑रुन्धे। नैन सत्यानृ॒ते उ॑दि॒ते हिस्तः। य ए॒वं वेद॑। मान्दा॒ स्थेत्या॑ह। रा॒ष्ट्रमे॒व ब्र॑ह्मवर्च॒स्य॑कः॥३०॥

%1.7.5.4
वाशा॒ स्थेत्या॑ह। रा॒ष्ट्रमे॒व व॒श्य॑कः। शक्व॑री॒ स्थेत्या॑ह। प॒शवो॒ वै शक्व॑रीः। प॒शूने॒वाव॑रुन्धे। वि॒श्व॒भृत॒ स्थेत्या॑ह। रा॒ष्ट्रमे॒व प॑य॒स्व्य॑कः। ज॒न॒भृत॒ स्थेत्या॑ह। रा॒ष्ट्रमे॒वेन्द्रि॑या॒व्य॑कः। अ॒ग्नेस्ते॑ज॒स्या स्थेत्या॑ह ॥३१॥

%1.7.5.5
रा॒ष्ट्रमे॒व ते॑ज॒स्व्य॑कः। अ॒पामोष॑धीना॒ रस॒ स्थेत्या॑ह। रा॒ष्ट्रमे॒व म॑ध॒व्य॑मकः। सा॒र॒स्व॒तङ्ग्रह॑ङ्गृह्णाति। ए॒षा वा अ॒पाम्पृ॒ष्ठम्। यत्सर॑स्वती। पृ॒ष्ठमे॒वैन समा॒नानां करोति। षो॒ड॒शभि॑र्गृह्णाति। षोड॑शकलो॒ वै पुरु॑षः। यावा॑ने॒व पुरु॑षः। तस्मि॑न्वी॒र्यं॑ दधाति। षो॒ड॒शभि॑र्जु॒होति॑ षोड॒शभि॑र्गृह्णाति। द्वात्रिश॒त्संप॑द्यन्ते। द्वात्रिशदक्षराऽनु॒ष्टुक्। वाग॑नु॒ष्टुप्सर्वा॑णि॒ छन्दासि। वा॒चैवैन॒ सर्वे॑भि॒श्छन्दो॑भिर॒भिषि॑ञ्चति॥३२॥\anuvakamend[ऊ॒र्मिरित्या॑ह॒ सूर्य॑वर्चस॒ स्थेत्या॑ह ब्रह्मवर्च॒स्य॑कस्तेज॒स्या स्थेत्या॑है॒व पुरु॑ष॒ष्षट् च॑]

%1.7.6.1
देवी॑राप॒ सं मधु॑मती॒र्मधु॑मतीभिः सृज्यध्व॒मित्या॑ह। ब्रह्म॑णै॒वैना॒ स सृ॑जति। अना॑धृष्टाः सीद॒तेत्या॑ह। ब्रह्म॑णै॒वैना सादयति। अ॒न्त॒रा होतु॑श्च॒ धिष्णि॑यं ब्राह्मणाच्छ॒सिन॑श्च सादयति। आ॒ग्ने॒यो वै होता। ऐ॒न्द्रो ब्राह्मणाच्छ॒सी। तेज॑सा चै॒वेन्द्रि॒येण॑ चोभ॒यतो॑ रा॒ष्ट्रं परि॑गृह्णाति। हिर॑ण्ये॒नोत्पु॑नाति। आहु॑त्यै॒ हि प॒वित्राभ्यामुत्पु॒नन्ति॒ व्यावृ॑त्त्यै॥३३॥

%1.7.6.2
श॒तमा॑नं भवति। श॒तायु॒ पुरु॑षः श॒तेन्द्रि॑यः। आयु॑ष्ये॒वेन्द्रि॒ये प्रति॑तिष्ठति। अनि॑भृष्टम॒सीत्या॑ह। अनि॑भृष्ट॒ ह्ये॑तत्। वा॒चो बन्धु॒रित्या॑ह। वा॒चो ह्ये॑ष बन्धु॑। त॒पो॒जा इत्या॑ह। त॒पो॒जा ह्ये॑तत्। सोम॑स्य दा॒त्रम॒सीत्या॑ह॥३४॥

%1.7.6.3
सोम॑स्य॒ ह्ये॑तद्दा॒त्रम्। शु॒क्रा व॑ शु॒क्रेणोत्पु॑ना॒मीत्या॑ह। शु॒क्रा ह्याप॑। शु॒क्र हिर॑ण्यम्। च॒न्द्राश्च॒न्द्रेणेत्या॑ह। च॒न्द्रा ह्याप॑। च॒न्द्र हिर॑ण्यम्। अ॒मृता॑ अ॒मृते॒नेत्या॑ह। अ॒मृता॒ ह्याप॑। अ॒मृत॒ हिर॑ण्यम्॥३५॥

%1.7.6.4
स्वाहा॑ राज॒सूया॒येत्या॑ह। रा॒ज॒सूया॑य॒ ह्ये॑ना उत्पु॒नाति॑। स॒ध॒मादो द्यु॒म्निनी॒रूर्ज॑ ए॒ता इति॑ वारु॒ण्यर्चा गृ॑ह्णाति। व॒रु॒ण॒स॒वमे॒वाव॑रुन्धे। एक॑या गृह्णाति। ए॒क॒धैव यज॑माने वी॒र्यं॑ दधाति। क्ष॒त्रस्योल्ब॑मसि क्ष॒त्रस्य॒ योनि॑र॒सीति॑ ता॒र्प्यञ्चो॒ष्णीषं॑ च॒ प्रय॑च्छति सयोनि॒त्वाय॑। एक॑शतेन दर्भपुञ्जी॒लैः प॑वयति। श॒तायु॒र्वै पुरु॑षः श॒तवीर्यः। आ॒त्मैक॑श॒तः॥३६॥

%1.7.6.5
यावा॑ने॒व पुरु॑षः। तस्मि॑न्वी॒र्यं॑ दधाति। दध्या॑शयति। इ॒न्द्रि॒यमे॒वाव॑ रुन्धे। उ॒दु॒म्बर॑माशयति। अ॒न्नाद्य॒स्याव॑रुद्ध्यै। शष्पाण्याशयति। सुरा॑बलिमे॒वैनं॑ करोति। आ॒विद॑ ए॒ता भ॑वन्ति। आ॒विद॑मे॒वैन॑ङ्गमयन्ति॥३७॥

%1.7.6.6
अ॒ग्निरे॒वैन॒ङ्गार्\mbox{}ह॑पत्येनावति। इन्द्र॑ इन्द्रि॒येण॑। पू॒षा प॒शुभि॑। मि॒त्रावरु॑णौ प्राणापा॒नाभ्याम्। इन्द्रो॑ वृ॒त्राय॒ वज्र॒मुद॑यच्छत्। स दिव॑मलिखत्। सोऽर्य॒म्णः पन्था॑ अभवत्। स आवि॑न्ने॒ द्यावा॑पृथि॒वी धृ॒तव्र॑ते॒ इति॒ द्यावा॑पृथि॒वी उपा॑धावत्। स आ॒भ्यामे॒व प्रसू॑त॒ इन्द्रो॑ वृ॒त्राय॒ वज्रं॒ प्राह॑रत्। आवि॑न्ने॒ द्यावा॑पृथि॒वी धृ॒तव्र॑ते॒ इति॒ यदाह॑॥३८॥

%1.7.6.7
आ॒भ्यामे॒व प्रसू॑तो॒ यज॑मानो॒ वज्रं॒ भ्रातृ॑व्याय॒ प्रह॑रति। आवि॑न्ना दे॒व्यदि॑तिर्विश्वरू॒पीत्या॑ह। इ॒यं वै दे॒व्यदि॑तिर्विश्वरू॒पी। अ॒स्यामे॒व प्रति॑ तिष्ठति। आवि॑न्नो॒ऽयम॒सावा॑मुष्याय॒णोऽस्यां वि॒श्य॑स्मिन्रा॒ष्ट्र इत्या॑ह। वि॒शैवैन रा॒ष्ट्रेण॒ सम॑र्धयति। म॒ह॒ते क्ष॒त्राय॑ मह॒त आधि॑पत्याय मह॒ते जान॑राज्या॒येत्या॑ह। आ॒शिष॑मे॒वैतामा शास्ते। ए॒ष वो॑ भरता॒ राजा॒ सोमो॒ऽस्माकं॑ ब्राह्म॒णाना॒ राजेत्या॑ह। तस्मा॒त्सोम॑राजानो ब्राह्म॒णाः॥३९॥

%1.7.6.8
इन्द्र॑स्य॒ वज्रो॑ऽसि॒ वार्त्र॑घ्न॒ इति॒ धनु॒ प्रय॑च्छति॒ विजि॑त्यै। श॒त्रु॒बाध॑ना॒ स्थेतीषून्॑। शत्रू॑ने॒वास्य॑ बाधन्ते। पा॒त मा प्र॒त्यञ्चं॑ पा॒त मा॑ ति॒र्यञ्च॑म॒न्वञ्चं॑ मा पा॒तेत्या॑ह। ति॒स्रो वै श॑र॒व्या। प्र॒तीची॑ ति॒रश्च्य॒नूची। ताभ्य॑ ए॒वैनं॑ पान्ति। दि॒ग्भ्यो मा॑ पा॒तेत्या॑ह। दि॒ग्भ्य ए॒वैनं॑ पान्ति। विश्वाभ्यो मा ना॒ष्ट्राभ्य॑ पा॒तेत्या॑ह। अप॑रिमितादे॒वैनं॑ पान्ति। हिर॑ण्यवर्णावु॒षसां विरो॒क इति॑ त्रि॒ष्टुभा॑ बा॒हू उद्गृ॑ह्णाति। इ॒न्द्रि॒यं वै वी॒र्य॑न्त्रि॒ष्टुक्। इ॒न्द्रि॒यमे॒व वी॒र्य॑मु॒परि॑ष्टादा॒त्मन्ध॑त्ते॥४०॥\anuvakamend[व्यावृ॑त्त्यै दा॒त्रम॒सीत्या॑हा॒मृत॒ हिर॑ण्यमेकश॒तो ग॑मय॒न्त्याह॑ ब्राह्म॒णा ना॒ष्ट्राभ्य॑ पा॒तेत्या॑ह च॒त्वारि॑ च]

%1.7.7.1
दिशो॒ व्यास्था॑पयति। दि॒शाम॒भिजि॑त्त्यै। यद॑नु प्र॒क्रामेत्। अ॒भि दिशो॑ जयेत्। उत्तु माद्येत्। मन॒साऽनु॒ प्रक्रा॑मति। अ॒भि दिशो॑ जयति। नोन्माद्यति। स॒मिध॒मा ति॒ष्ठेत्या॑ह। तेज॑ ए॒वाव॑रुन्धे॥४१॥

%1.7.7.2
उ॒ग्रामा ति॒ष्ठेत्या॑ह। इ॒न्द्रि॒यमे॒वाव॑रुन्धे। वि॒राज॒माति॒ष्ठेत्या॑ह। अ॒न्नाद्य॑मे॒वाव॑रुन्धे। उदी॑ची॒मा ति॒ष्ठेत्या॑ह। प॒शूने॒वाव॑रुन्धे। ऊ॒र्ध्वामाति॒ष्ठेत्या॑ह। सु॒व॒र्गमे॒व लो॒कम॒भिज॑यति। अनूज्जि॑हीते। सु॒व॒र्गस्य॑ लो॒कस्य॒ सम॑ष्ट्यै॥४२॥

%1.7.7.3
मा॒रु॒त ए॒ष भ॑वति। अन्नं॒ वै म॒रुत॑। अन्न॑मे॒वाव॑रुन्धे। एक॑विशतिकपालो भवति॒ प्रति॑ष्ठित्यै। यो॑ऽरण्येऽनुवा॒क्यो॑ ग॒णः। तं म॑ध्य॒त उप॑दधाति। ग्रा॒म्यैरे॒व प॒शुभि॑रार॒ण्यान्प॒शून्परि॑ गृह्णाति। तस्माद्ग्रा॒म्यैः प॒शुभि॑रार॒ण्याः प॒शव॒ परि॑गृहीताः। पृथि॑र्वै॒न्यः। अ॒भ्य॑षिच्यत॥४३॥

%1.7.7.4
स रा॒ष्ट्रन्नाभ॑वत्। स ए॒तानि॑ पा॒र्थान्य॑पश्यत्। तान्य॑जुहोत्। तैर्वै स रा॒ष्ट्रम॑भवत्। यत्पा॒र्थानि॑ जु॒होति॑। रा॒ष्ट्रमे॒व भ॑वति। बा॒र्॒ह॒स्प॒त्यं पूर्वे॑षामुत्त॒मं भ॑वति। ऐ॒न्द्रमुत्त॑रेषां प्रथ॒मम्। ब्रह्म॑ चै॒वास्मै क्ष॒त्रं च॑ स॒मीची॑ दधाति। अथो॒ ब्रह्म॑न्ने॒व क्ष॒त्रं प्रति॑ष्ठापयति॥४४॥

%1.7.7.5
षट्पु॒रस्ता॑दभिषे॒कस्य॑ जुहोति। षडु॒परि॑ष्टात्। द्वाद॑श॒ संप॑द्यन्ते। द्वाद॑श॒ मासा संवत्स॒रः। सं॒व॒त्स॒रः खलु॒ वै दै॒वानां॒ पूः। दे॒वाना॑मे॒व पुरं॑ मध्य॒तो व्यव॑सर्पति। तस्य॒ न कुत॑श्च॒नोपाव्या॒धो भ॑वति। भू॒ताना॒मवेष्टीर्जुहोति। अत्रात्र॒ वै मृ॒त्युर्जा॑यते। यत्र॑यत्रै॒व मृ॒त्युर्जाय॑ते। तत॑ ए॒वैन॒मव॑यजते। तस्माद्राज॒सूये॑नेजा॒नो नाभिच॑रित॒वै। प्र॒त्यगे॑नमभिचा॒रः स्तृ॑णुते॥४५॥\anuvakamend[रु॒न्धे॒ सम॑ष्ट्या असिच्यत स्थापयति॒ जाय॑ते॒ पञ्च॑ च]

%1.7.8.1
सोम॑स्य॒ त्विषि॑रसि॒ तवे॑व मे॒ त्विषि॑र्भूया॒दिति॑ शार्दूलच॒र्मोप॑स्तृणाति। यैव सोमे॒ त्विषि॑। या शार्दू॒ले। तामे॒वाव॑रुन्धे। मृ॒त्योर्वा ए॒ष वर्ण॑। यच्छार्दू॒लः। अ॒मृत॒ हिर॑ण्यम्। अ॒मृत॑मसि मृ॒त्योर्मा॑ पा॒हीति॒ हिर॑ण्य॒मुपास्यति। अ॒मृत॑मे॒व मृ॒त्योर॒न्तर्ध॑त्ते। श॒तमा॑नं भवति॥४६॥

%1.7.8.2
श॒तायु॒ पुरु॑षः श॒तेन्द्रि॑यः। आयु॑ष्ये॒वेन्द्रि॒ये प्रति॑तिष्ठति। दि॒द्योन्मा॑ पा॒हीत्यु॒परि॑ष्टा॒दधि॒ निद॑धाति। उ॒भ॒यत॑ ए॒वास्मै॒ शर्म॑ दधाति। अवेष्टा दन्द॒शूका॒ इति॑ क्ली॒ब सीसे॑न विध्यति। द॒न्द॒शूका॑ने॒वाव॑यजते। तस्मात्क्ली॒बन्द॑न्द॒शूका॒ दशु॑काः। निर॑स्त॒न्नमु॑चे॒ शिर॒ इति॑ लोहिताय॒सन्निर॑स्यति। पा॒प्मान॑मे॒व नमु॑चिन्नि॒रव॑दयते। प्रा॒णा आ॒त्मन॒ पूर्वे॑ऽभि॒षिच्या॒ इत्या॑हुः॥४७॥

%1.7.8.3
सोमो॒ राजा॒ वरु॑णः। दै॒वा ध॑र्म॒सुव॑श्च॒ ये। ते ते॒ वाच सुवन्ता॒न्ते ते प्रा॒ण सु॑वन्ता॒मित्या॑ह। प्रा॒णाने॒वात्मन॒ पूर्वा॑न॒भिषि॑ञ्चति। यद्ब्रू॒यात्। अ॒ग्नेस्त्वा॒ तेज॑सा॒ऽभिषि॑ञ्चा॒मीति॑। ते॒ज॒स्व्ये॑व स्यात्। दु॒श्चर्मा॒ तु भ॑वेत्। सोम॑स्य त्वा द्यु॒म्नेना॒भिषि॑ञ्चा॒मीत्या॑ह। सौ॒म्यो वै दे॒वत॑या॒ पुरु॑षः॥४८॥

%1.7.8.4
स्वयै॒वैनं॑ दे॒वत॑या॒ऽभिषि॑ञ्चति। अ॒ग्नेस्तेज॒सेत्या॑ह। तेज॑ ए॒वास्मि॑न्दधाति। सूर्य॑स्य॒ वर्च॒सेत्या॑ह। वर्च॑ ए॒वास्मि॑न्दधाति। इन्द्र॑स्येन्द्रि॒येणेत्या॑ह। इ॒न्द्रि॒यमे॒वास्मि॑न्दधाति। मि॒त्रावरु॑णयोर्वी॒र्ये॑णेत्या॑ह। वी॒र्य॑मे॒वास्मि॑न्दधाति। म॒रुता॒मोज॒सेत्या॑ह॥४९॥

%1.7.8.5
ओज॑ ए॒वास्मि॑न्दधाति। क्ष॒त्राणाङ्क्ष॒त्रप॑तिर॒सीत्या॑ह। क्ष॒त्राणा॑मे॒वैनं॑ क्ष॒त्रप॑तिं करोति। अति॑ दि॒वस्पा॒हीत्या॑ह। अत्य॒न्यान्पा॒हीति॒ वावैतदा॑ह। स॒माव॑वृत्रन्नध॒रागुदी॑ची॒रित्या॑ह। रा॒ष्ट्रमे॒वास्मि॑न्ध्रु॒वम॑कः। उ॒च्छेष॑णेन जुहोति। उ॒च्छेष॑णभागो॒ वै रु॒द्रः। भा॒ग॒धेये॑नै॒व रु॒द्रन्नि॒रव॑दयते॥५०॥

%1.7.8.6
उद॑ङ्प॒रेत्याग्नीद्ध्रे जुहोति। ए॒षा वै रु॒द्रस्य॒ दिक्। स्वाया॑मे॒व दि॒शि रु॒द्रन्नि॒रव॑दयते। रुद्र॒ यत्ते॒ क्रयी॒ पर॒न्नामेत्या॑ह। यद्वा अ॑स्य॒ क्रयी॒ पर॒न्नाम॑। तेन॒ वा ए॒ष हि॑नस्ति। य हि॒नस्ति॑। तेनै॒वैन स॒ह श॑मयति। तस्मै॑ हु॒तम॑सि य॒मेष्ट॑म॒सीत्या॑ह। य॒मादे॒वास्य॑ मृ॒त्युमव॑यजते॥५१॥

%1.7.8.7
प्रजा॑पते॒ न त्वदे॒तान्य॒न्य इति॒ तस्यै॑ गृ॒हे जु॑हुयात्। याङ्का॒मये॑त रा॒ष्ट्रम॑स्यै प्र॒जा स्या॒दिति॑। रा॒ष्ट्रमे॒वास्यै प्र॒जा भ॑वति। प॒र्ण॒मये॑नाध्व॒र्युर॒भिषि॑ञ्चति। ब्र॒ह्म॒व॒र्च॒समे॒वास्मि॒न्त्विषि॑न्दधाति। औदु॑म्बरेण राज॒न्य॑। ऊर्ज॑मे॒वास्मि॑न्न॒न्नाद्य॑न्दधाति। आश्व॑त्थेन॒ वैश्य॑। विश॑मे॒वास्मि॒न्पुष्टि॑न्दधाति। नैय॑ग्रोधेन॒ जन्य॑। मि॒त्राण्ये॒वास्मै॑ कल्पयति। अथो॒ प्रति॑ष्ठित्यै॥५२॥\anuvakamend[भ॒व॒त्या॒हु॒ पुरु॑ष॒ ओज॒सेत्या॑ह नि॒रव॑दयते यजते॒ जन्यो॒ द्वे च॑]

%1.7.9.1
इन्द्र॑स्य॒ वज्रो॑ऽसि॒ वार्त्र॑घ्न॒ इति॒ रथ॑मु॒पाव॑हरति॒ विजि॑त्यै। मि॒त्रावरु॑णयोस्त्वा प्रशा॒स्त्रोः प्र॒शिषा॑ युन॒ज्मीत्या॑ह। ब्रह्म॑णै॒वैनं॑ दे॒वताभ्याय्युँनक्ति। प्र॒ष्टि॒वा॒हिन॑य्युँनक्ति। प्र॒ष्टि॒वा॒ही वै दे॑वर॒थः। दे॒व॒र॒थमे॒वास्मै॑ युनक्ति। त्रयोऽश्वा॑ भवन्ति। रथ॑श्चतु॒र्थः। द्वौ स॑व्येष्ठसार॒थी। षट्त्सं प॑द्यन्ते॥५३॥

%1.7.9.2
षड्वा ऋ॒तव॑। ऋ॒तुभि॑रे॒वैन॑य्युँनक्ति। वि॒ष्णु॒क्र॒मान्क्र॑मते। विष्णु॑रे॒व भू॒त्वेमाल्लोँ॒कान॒भिज॑यति। यः क्ष॒त्रिय॒ प्रति॑हितः। सोऽन्वार॑भते। रा॒ष्ट्रमे॒व भ॑वति। त्रि॒ष्टुभा॒ऽन्वार॑भते। इ॒न्द्रि॒यं वै त्रि॒ष्टुक्। इ॒न्द्रि॒यमे॒व यज॑माने दधाति॥५४॥

%1.7.9.3
म॒रुतां प्रस॒वे जे॑ष॒मित्या॑ह। म॒रुद्भि॑रे॒व प्रसू॑त॒ उज्ज॑यति। आ॒प्तं मन॒ इत्या॑ह। यदे॒व मन॒सैप्सीत्। तदा॑पत्। रा॒ज॒न्यं॑ जिनाति। अनाक्रान्त ए॒वाक्र॑मते। वि वा ए॒ष इ॑न्द्रि॒येण॑ वी॒र्ये॑णर्ध्यते। यो रा॑ज॒न्यं॑ जि॒नाति॑। सम॒हमि॑न्द्रि॒येण॑ वी॒र्ये॑णेत्या॑ह॥५५॥

%1.7.9.4
इ॒न्द्रि॒यमे॒व वी॒र्य॑मा॒त्मन्ध॑त्ते। प॒शू॒नां म॒न्युर॑सि॒ तवे॑व मे म॒न्युर्भू॑या॒दिति॒ वारा॑ही उपा॒नहा॒वुप॑ मुञ्चते। प॒शू॒नां वा ए॒ष म॒न्युः। यद्व॑रा॒हः। तेनै॒व प॑शू॒नां म॒न्युमा॒त्मन्ध॑त्ते। अ॒भि वा इ॒य सु॑षुवा॒णङ्का॑मयते। तस्येश्व॒रेन्द्रि॒यं वी॒र्य॑मादा॑तोः। वारा॑ही उपा॒नहा॒वुप॑मुञ्चते। अ॒स्या ए॒वान्तर्ध॑त्ते। इ॒न्द्रि॒यस्य॑ वी॒र्य॑स्यानात्यै॥५६॥

%1.7.9.5
नमो॑ मा॒त्रे पृ॑थि॒व्या इत्या॒हाहिसायै। इय॑द॒स्यायु॑र॒स्यायु॑र्मे धे॒हीत्या॑ह। आयु॑रे॒वात्मन्ध॑त्ते। ऊर्ग॒स्यूर्जं॑ मे धे॒हीत्या॑ह। ऊर्ज॑मे॒वात्मन्ध॑त्ते। युङ्ङ॑सि॒ वर्चो॑सि॒ वर्चो॒ मयि॑ धे॒हीत्या॑ह। वर्च॑ ए॒वात्मन्ध॑त्ते। ए॒क॒धा ब्र॒ह्मण॒ उप॑हरति। ए॒क॒धैव यज॑मान॒ आयु॒रूर्जं॒ वर्चो॑ दधाति। र॒थ॒वि॒मो॒च॒नीया॑ जुहोति॒ प्रति॑ष्ठित्यै॥५७॥

%1.7.9.6
त्रयोऽश्वा॑ भवन्ति। रथ॑श्चतु॒र्थः। तस्माच्च॒तुर्जु॑होति। यदु॒भौ स॒हाव॒तिष्ठे॑ताम्। स॒मा॒नं लो॒कमि॑याताम्। स॒ह स॑ङ्ग्रही॒त्रा र॑थ॒वाह॑ने॒ रथ॒माद॑धाति। सु॒व॒र्गादे॒वैनं॑ लो॒काद॒न्तर्द॑धाति। ह॒सः शु॑चि॒षदित्याद॑धाति। ब्रह्म॑णै॒वैन॑मुपाव॒हर॑ति। ब्रह्म॒णाऽऽद॑धाति। अति॑च्छन्द॒साऽऽद॑धाति। अति॑च्छन्दा॒ वै सर्वा॑णि॒ छन्दासि। सर्वे॑भिरे॒वैन॒ञ्छन्दो॑भि॒राद॑धाति। वर्ष्म॒ वा ए॒षा छन्द॑साम्। यदति॑च्छन्दाः। यदति॑च्छन्दसा॒ दधा॑ति। वर्ष्मै॒वैन समा॒नानां करोति॥५८॥\anuvakamend[प॒द्य॒न्ते॒ द॒धा॒ति॒ वी॒र्ये॑णेत्या॒हानात्यै॒ प्रति॑ष्ठित्यै॒ ब्रह्म॒णाऽऽद॑धाति स॒प्त च॑]

%1.7.10.1
मि॒त्रो॑ऽसि॒ वरु॑णो॒ऽसीत्या॑ह। मै॒त्रं वा अह॑। वा॒रु॒णी रात्रि॑। अ॒हो॒रा॒त्राभ्या॑मे॒वैन॑मु॒पाव॑हरति। मि॒त्रो॑ऽसि॒ वरु॑णो॒ऽसीत्या॑ह। मै॒त्रो वै दक्षि॑णः। वा॒रु॒णः स॒व्यः। वै॒श्व॒दे॒व्या॑मिक्षा। स्वमे॒वैनौ॑ भाग॒धेय॑मु॒पाव॑हरति। सम॒हं विश्वैर्दे॒वैरित्या॑ह॥५९॥

%1.7.10.2
वै॒श्व॒दे॒व्यो॑ वै प्र॒जाः। ता ए॒वाद्या कुरुते। क्ष॒त्रस्य॒ नाभि॑रसि क्ष॒त्रस्य॒ योनि॑र॒सीत्य॑धीवा॒समास्तृ॑णाति सयोनि॒त्वाय॑। स्यो॒नामा सी॑द सु॒षदा॒मा सी॒देत्या॑ह। य॒था॒य॒जुरे॒वैतत्। मा त्वा॑ हिसी॒न्मा मा॑ हिसी॒दित्या॒हाहिसायै। निष॑साद धृ॒तव्र॑तो॒ वरु॑णः प॒स्त्यास्वा साम्राज्याय सु॒क्रतु॒रित्या॑ह। साम्राज्यमे॒वैन सु॒क्रतुं॑ करोति। ब्रह्मा(३)न्त्व रा॑जन्ब्र॒ह्माऽसि॑ सवि॒ताऽसि॑ स॒त्यस॑व॒ इत्या॑ह। स॒वि॒तार॑मे॒वैन स॒त्यस॑वं करोति॥६०॥

%1.7.10.3
ब्रह्मा(३)न्त्व रा॑जन्ब्र॒ह्माऽसीन्द्रो॑ऽसि स॒त्यौजा॒ इत्या॑ह। इन्द्र॑मे॒वैन स॒त्यौज॑सं करोति। ब्रह्मा(३)न्त्व रा॑जन्ब्र॒ह्माऽसि॑ मि॒त्रो॑ऽसि सु॒शेव॒ इत्या॑ह। मि॒त्रमे॒वैन सु॒शेवं॑ करोति। ब्रह्मा(३)न्त्व रा॑जन्ब्र॒ह्मासि॒ वरु॑णोऽसि स॒त्यध॒र्मेत्या॑ह। वरु॑णमे॒वैन स॒त्यध॑र्माणं करोति। स॒वि॒ताऽसि॑ स॒त्यस॑व॒ इत्या॑ह। गा॒य॒त्रीमे॒वैतेना॑भि॒ व्याह॑रति। इन्द्रो॑ऽसि स॒त्यौजा॒ इत्या॑ह। त्रि॒ष्टुभ॑मे॒वैतेना॑भि॒ व्याह॑रति॥६१॥

%1.7.10.4
मि॒त्रो॑ऽसि सु॒शेव॒ इत्या॑ह। जग॑तीमे॒वैतेना॑भि॒ व्याह॑रति। स॒त्यमे॒ता दे॒वता। स॒त्यमे॒तानि॒ छन्दासि। स॒त्यमे॒वाव॑रुन्धे। वरु॑णोऽसि स॒त्यध॒र्मेत्या॑ह। अ॒नु॒ष्टुभ॑मे॒वैतेना॑भि॒ व्याह॑रति। स॒त्या॒नृ॒ते वा अ॑नु॒ष्टुप्। स॒त्या॒नृ॒ते वरु॑णः। स॒त्या॒नृ॒ते ए॒वाव॑रुन्धे॥६२॥

%1.7.10.5
नैन सत्यानृ॒ते उ॑दि॒ते हिस्तः। य ए॒वं वेद॑। इन्द्र॑स्य॒ वज्रो॑ऽसि॒ वार्त्र॑घ्न॒ इति॒ स्प्यं प्रय॑च्छति। वज्रो॒ वै स्प्यः। वज्रे॑णै॒वास्मा॑ अवरप॒र र॑न्धयति। ए॒व हि तच्छ्रेय॑। यद॑स्मा ए॒ते रध्ये॑युः। दिशो॒ऽभ्य॑य राजा॑ऽभू॒दिति॒ पञ्चा॒क्षान्प्रय॑च्छति। ए॒ते वै सर्वेऽया। अप॑राजायिनमे॒वैनं॑ करोति॥६३॥

%1.7.10.6
ओ॒द॒नमुद्ब्रु॑वते। प॒र॒मे॒ष्ठी वा ए॒षः। यदो॑द॒नः। प॒र॒मामे॒वैन॒ श्रिय॑ङ्गमयति। सुश्लो॒काँ (४) सुम॑ङ्ग॒लाँ (४) सत्य॑रा॒जा (३) नित्या॑ह। आ॒शिष॑मे॒वैतामा शास्ते। शौ॒न॒ शे॒पमाख्या॑पयते। व॒रु॒ण॒पा॒शादे॒वैनं॑ मुञ्चति। प॒र॒ श॒तं भ॑वति। श॒तायु॒ पुरु॑षः श॒तेन्द्रि॑यः। आयु॑ष्ये॒वेन्द्रि॒ये प्रति॑ तिष्ठति। मा॒रु॒तस्य॒ चैक॑विशतिकपालस्य वैश्वदे॒व्यै चा॒मिक्षा॑या अ॒ग्नये स्विष्ट॒कृते॑ स॒मव॑द्यति। दे॒वता॑भिरे॒वैन॑मुभ॒यत॒ परि॑ गृह्णाति। अ॒पान्नप्त्रे॒ स्वाहो॒र्जो नप्त्रे॒ स्वाहा॒ऽग्नये॑ गृ॒हप॑तये॒ स्वाहेति॑ ति॒स्र आहु॑तीर्जुहोति। त्रय॑ इ॒मे लो॒काः। ए॒ष्वे॑व लो॒केषु॒ प्रति॑ तिष्ठति॥६४॥\anuvakamend[दे॒वैरित्या॑ह स॒त्यस॑वं करोति त्रि॒ष्टुभ॑मे॒वैतेना॑भि॒ व्याह॑रति सत्यानृ॒ते ए॒वाव॑रुन्धे करोति श॒तेन्द्रि॑य॒ष्षट् च॑]




\prashnaend{ए॒तद्ब्राह्मणानि धा॒त्रे र॒त्निनान्देवसु॒वाम॒र्थेतो॒ देवी॒र्दिश॒ सोम॒स्येन्द्र॑स्य मि॒त्रो दश॑॥१०॥}{ए॒तद्ब्राह्मणानि वैष्ण॒वन्त्रि॑कपा॒लमन्नं॒ वै पू॒षा वाशा॒ स्थेत्या॑ह॒ दिशो॒ व्यास्था॑पय॒त्युद॑ङ्प॒रेत्य॒ ब्रह्मा ३ न्त्व रा॑ज॒श्चतु॑ष्षष्टिः॥६४॥}{ए॒तद्ब्राह्मणानि॒ प्रति॑तिष्ठति॥}{हरि॑ ओम्॥}{इति श्रीकृष्णयजुर्वेदीयतैत्तिरीयब्राह्मणे प्रथमाष्टके सप्तमः प्रपाठकः समाप्तः॥}
\clearpage
\sect{अष्टमः प्रश्नः}
\setcounter{anuvakam}{0}
\dnsub{तैत्तिरीयब्राह्मणे प्रथमाष्टके अष्टमः प्रपाठकः}

%1.8.1.1
वरु॑णस्य सुषुवा॒णस्य॑ दश॒धेन्द्रि॒यं वी॒र्यं॑ परा॑ऽपतत्। तत्स॒सृद्भि॒रनु॒ सम॑सर्पत्। तत्स॒सृपा ससृ॒त्त्वम्। अ॒ग्निना॑ दे॒वेन॑ प्रथ॒मेऽह॒न्ननु॒ प्रायु॑ङ्क्त। सर॑स्वत्या वा॒चा द्वि॒तीये। स॒वि॒त्रा प्र॑स॒वेन॑ तृ॒तीये। पू॒ष्णा प॒शुभि॑श्चतु॒र्थे। बृह॒स्पति॑ना॒ ब्रह्म॑णा पञ्च॒मे। इन्द्रे॑ण दे॒वेन॑ ष॒ष्ठे। वरु॑णेन॒ स्वया॑ दे॒वत॑या सप्त॒मे॥१॥

%1.8.1.2
सोमे॑न॒ राज्ञाऽष्ट॒मे। त्वष्ट्रा॑ रू॒पेण॑ नव॒मे। विष्णु॑ना य॒ज्ञेनाप्नोत्। यत्स॒सृपो॒ भव॑न्ति। इ॒न्द्रि॒यमे॒व तद्वी॒र्यं॑ यज॑मान आप्नोति। पूर्वा॑पूर्वा॒ वेदि॑र्भवति। इ॒न्द्रि॒यस्य॑ वी॒र्य॑स्याव॑रुद्ध्यै। पु॒रस्ता॑दुप॒सदा सौ॒म्येन॒ प्रच॑रति। सोमो॒ वै रे॑तो॒धाः। रेत॑ ए॒व तद्द॑धाति। अ॒न्त॒रा त्वा॒ष्ट्रेण॑। रेत॑ ए॒व हि॒तन्त्वष्टा॑ रू॒पाणि॒ विक॑रोति। उ॒परि॑ष्टाद्वैष्ण॒वेन॑। य॒ज्ञो वै विष्णु॑। य॒ज्ञ ए॒वान्त॒तः प्रति॑ तिष्ठति॥२॥\anuvakamend[स॒प्त॒मे द॑धाति॒ पञ्च॑ च]

%1.8.2.1
जा॒मि वा ए॒तत्कु॑र्वन्ति। यत्स॒द्यो दी॒क्षय॑न्ति स॒द्यः सोमं॑ क्री॒णन्ति॑। पु॒ण्ड॒रि॒स्र॒जां प्रय॑च्छ॒त्यजा॑मित्वाय। अङ्गि॑रसः सुव॒र्गं लो॒कं यन्त॑। अ॒प्सु दीक्षात॒पसी॒ प्रावे॑शयन्। तत्पु॒ण्डरी॑कमभवत्। यत्पु॑ण्डरिस्र॒जां प्र॒यच्छ॑ति। सा॒क्षादे॒व दीक्षात॒पसी॒ अव॑रुन्धे। द॒शभि॑र्वत्सत॒रैः सोमं॑ क्रीणाति। दशाक्षरा वि॒राट्॥३॥

%1.8.2.2
अन्नं॑ वि॒राट्। वि॒राजै॒वान्नाद्य॒मव॑ रुन्धे। मु॒ष्क॒रा भ॑वन्ति सेन्द्र॒त्वाय॑। द॒श॒पेयो॑ भवति। अ॒न्नाद्य॒स्याव॑रुद्ध्यै। श॒तं ब्राह्म॒णाः पि॑बन्ति। श॒तायु॒ पुरु॑षः श॒तेन्द्रि॑यः। आयु॑ष्ये॒वेन्द्रि॒ये प्रति॑तिष्ठति। स॒प्त॒द॒श स्तो॒त्रं भ॑वति। स॒प्त॒द॒शः प्र॒जाप॑तिः॥४॥

%1.8.2.3
प्र॒जाप॑ते॒राप्त्यै। प्रा॒का॒शाव॑ध्व॒र्यवे॑ ददाति। प्र॒का॒शमे॒वैन॑ङ्गमयति। स्रज॑मुद्गा॒त्रे। व्ये॑वास्मै॑ वासयति। रु॒क्म होत्रे। आ॒दि॒त्यमे॒वास्मा॒ उन्न॑यति। अश्वं॑ प्रस्तोतृप्रतिह॒र्तृभ्याम्। प्रा॒जा॒प॒त्यो वा अश्व॑। प्र॒जाप॑ते॒राप्त्यै॥५॥

%1.8.2.4
द्वाद॑श पष्ठौ॒हीर्ब्र॒ह्मणे। आयु॑रे॒वाव॑रुन्धे। व॒शां मैत्रावरु॒णाय॑। रा॒ष्ट्रमे॒व व॒श्य॑कः। ऋ॒ष॒भं ब्राह्मणाच्छ॒सिने। रा॒ष्ट्रमे॒वेन्द्रि॑या॒व्य॑कः। वास॑सी नेष्टापो॒तृभ्याम्। प॒वित्रे॑ ए॒वास्यै॒ते। स्थूरि॑ यवाचि॒तम॑च्छावा॒काय॑। अ॒न्त॒त ए॒व वरु॑ण॒मव॑ यजते॥६॥

%1.8.2.5
अ॒नड्वाह॑म॒ग्नीधे। वह्नि॒र्वा अ॑न॒ड्वान्। वह्नि॑र॒ग्नीत्। वह्नि॑नै॒व वह्नि॑ य॒ज्ञस्याव॑रुन्धे। इन्द्र॑स्य सुषुवा॒णस्य॑ त्रे॒धेन्द्रि॒यं वी॒र्यं॑ परा॑ऽपतत्। भृगु॒स्तृती॑यमभवत्। श्रा॒य॒न्तीय॒न्तृती॑यम्। सर॑स्वती॒ तृती॑यम्। भा॒र्ग॒वो होता॑ भवति। श्रा॒य॒न्तीयं॑ ब्रह्मसा॒मं भ॑वति। वा॒र॒व॒न्तीय॑मग्निष्टोमसा॒मम्। सा॒र॒स्व॒तीर॒पो गृ॑ह्णाति। इ॒न्द्रि॒यस्य॑ वी॒र्य॑स्याव॑रुद्ध्यै। श्रा॒य॒न्तीयं॑ ब्रह्मसा॒मं भ॑वति। इ॒न्द्रि॒यमे॒वास्मि॑न्वी॒र्य श्रयति। वा॒र॒व॒न्तीय॑मग्निष्टोमसा॒मम्। इ॒न्द्रि॒यमे॒वास्मि॑न्वी॒र्यं॑ वारयति॥७॥\anuvakamend[वि॒राट्प्र॒जाप॑ति॒रश्व॑ प्र॒जाप॑ते॒राप्त्यै॑ यजते ब्रह्मसा॒मं भ॑वति स॒प्त च॑]

%1.8.3.1
ई॒श्व॒रो वा ए॒ष दिशोऽनून्म॑दितोः। यन्दिशोऽनु॑ व्यास्था॒पय॑न्ति। दि॒शामवेष्टयो भवन्ति। दि॒क्ष्वे॑व प्रति॑ तिष्ठ॒त्यनु॑न्मादाय। पञ्च॑ दे॒वता॑ यजति। पञ्च॒ दिश॑। दि॒क्ष्वे॑व प्रति॑ तिष्ठति। ह॒विषो॑हविष इ॒ष्ट्वा बा॑र्\mbox{}हस्प॒त्यम॒भिघा॑रयति। य॒ज॒मा॒न॒दे॒व॒त्यो॑ वै बृह॒स्पति॑। यज॑मानमे॒व तेज॑सा॒ सम॑र्धयति॥८॥

%1.8.3.2
आ॒दि॒त्यां म॒ल्॒हाङ्ग॒र्भिणी॒मा ल॑भते। मा॒रु॒तीं पृश्ञिं॑ पष्ठौ॒हीम्। विशं॑ चैवास्मै॑ रा॒ष्ट्रं च॑ स॒मीची॑ दधाति। आ॒दि॒त्यया॒ पूर्व॑या॒ प्रच॑रति। मा॒रु॒त्योत्त॑रया। रा॒ष्ट्र ए॒व विश॒मनु॑बध्नाति। उ॒च्चैरा॑दि॒त्याया॒ आश्रा॑वयति। उ॒पा॒शु मा॑रु॒त्यै। तस्माद्रा॒ष्ट्रं विश॒मति॑वदति। ग॒र्भिण्या॑दि॒त्या भ॑वति॥९॥

%1.8.3.3
इ॒न्द्रि॒यं वै गर्भ॑। रा॒ष्ट्रमे॒वेन्द्रि॑या॒व्य॑कः। अ॒ग॒र्भा मा॑रु॒ती। विड्वै म॒रुत॑। विश॑मे॒व निरि॑न्द्रियामकः। दे॒वा॒सु॒राः संय॑त्ता आसन्। ते दे॒वा अ॒श्विनो पू॒षन्वा॒चः स॒त्य स॑न्नि॒धाय॑। अनृ॑ते॒नासु॑रान॒भ्य॑भवन्। तेऽश्विभ्यां पू॒ष्णे पु॑रो॒डाश॒न्द्वाद॑शकपालं॒ निर॑वपन्। ततो॒ वै ते वा॒चः स॒त्यमवा॑रुन्धत॥१०॥

%1.8.3.4
यद॒श्विभ्यां पू॒ष्णे पु॑रो॒डाश॒न्द्वाद॑शकपालन्नि॒र्वप॑ति। अनृ॑तेनै॒व भ्रातृ॑व्यानभि॒भूय॑। वा॒चः स॒त्यमव॑रुन्धे। सर॑स्वते सत्य॒वाचे॑ च॒रुम्। पूर्व॑मे॒वोदि॒तम्। उत्त॑रेणा॒भि गृ॑णाति। स॒वि॒त्रे स॒त्यप्र॑सवाय पुरो॒डाश॒न्द्वाद॑शकपालं॒ प्रसूत्यै। दू॒तान्प्रहि॑णोति। आ॒विद॑ ए॒ता भ॑वन्ति। आ॒विद॑मे॒वैन॑ङ्गमयन्ति। अथो॑ दू॒तेभ्य॑ ए॒व न छि॑द्यते। ति॒सृ॒ध॒न्व शु॑ष्कदृ॒तिर्दक्षि॑णा॒ समृ॑द्ध्यै॥११॥\anuvakamend[अ॒र्ध॒य॒ति॒ भ॒व॒त्य॒रु॒न्ध॒त॒ ग॒म॒य॒न्ति॒ द्वे च॑]

%1.8.4.1
आ॒ग्ने॒यम॒ष्टाक॑पालं॒ निर्व॑पति। तस्मा॒च्छिशि॑रे कुरुपञ्चा॒लाः प्राञ्चो॑ यान्ति। सौ॒म्यञ्च॒रुम्। तस्माद्वस॒न्तव्व्यँ॑व॒साया॑दयन्ति। सा॒वि॒त्रन्द्वाद॑शकपालम्। तस्मात्पु॒रस्ता॒द्यवा॑ना सवि॒त्रा विरु॑न्धते। बा॒र्॒ह॒स्प॒त्यञ्च॒रुम्। स॒वि॒त्रैव वि॒रुध्य॑। ब्रह्म॑णा॒ यवा॒नाद॑धते। त्वा॒ष्ट्रम॒ष्टाक॑पालम्॥१२॥

%1.8.4.2
रू॒पाण्ये॒व तेन॑ कुर्वते। वै॒श्वा॒न॒रन्द्वाद॑शकपालम्। तस्माज्जघ॒न्ये॑ नैदा॑घे प्र॒त्यञ्च॑ कुरुपञ्चा॒ला यान्ति। सा॒र॒स्व॒तञ्च॒रुन्निर्व॑पति। तस्मात्प्रा॒वृषि॒ सर्वा॒ वाचो॑ वदन्ति। पौ॒ष्णेन॒ व्यव॑स्यन्ति। मै॒त्रेण॑ कृषन्ते। वा॒रु॒णेन॒ विधृ॑ता आसते। क्षै॒त्र॒प॒त्येन॑ पाचयन्ते। आ॒दि॒त्येनाद॑धते॥१३॥

%1.8.4.3
मा॒सिमास्ये॒तानि॑ ह॒वीषि॑ नि॒रुप्या॒णीत्या॑हुः। तेनै॒वर्तून्प्रयु॑ङ्क्त॒ इति॑। अथो॒ खल्वा॑हुः। कः सं॑वत्स॒रञ्जी॑विष्य॒तीति॑। षडे॒व पूर्वे॒द्युर्नि॒रुप्या॑णि। षडु॑त्तरे॒द्युः। तेनै॒वर्तून्प्रयु॑ङ्क्ते। दक्षि॑णो रथवाहनवा॒हः पूर्वे॑षा॒न्दक्षि॑णा। उत्त॑र॒ उत्त॑रेषाम्। सं॒व॒त्स॒रस्यै॒वान्तौ॑ युनक्ति। सु॒व॒र्गस्य॑ लो॒कस्य॒ सम॑ष्ट्यै॥१४॥\anuvakamend[त्वा॒ष्ट्रम॒ष्टाक॑पालन्दधते युन॒क्त्येकं॑ च]

%1.8.5.1
इन्द्र॑स्य सुषुवा॒णस्य॑ दश॒धेन्द्रि॒यं वी॒र्यं॑ परा॑ऽपतत्। स यत्प्र॑थ॒मन्नि॒रष्ठी॑वत्। तत्क्व॑लमभवत्। यद्द्वि॒तीयम्। तद्बद॑रम्। यत्तृ॒तीयम्। तत्क॒र्कन्धु॑। यन्न॒स्तः। स सि॒हः। यदक्ष्यो॥१५॥

%1.8.5.2
स शार्दू॒लः। यत्कर्ण॑योः। स वृक॑। य ऊ॒र्ध्वः। स सोम॑। याऽवा॑ची। सा सुरा। त्र॒याः सक्त॑वो भवन्ति। इ॒न्द्रि॒यस्याव॑रुद्ध्यै। त्र॒याणि॒ लोमा॑नि॥१६॥

%1.8.5.3
त्विषि॑मे॒वाव॑रुन्धे। त्रयो॒ ग्रहा। वी॒र्य॑मे॒वाव॑रुन्धे। नाम्ना॑ दश॒मी। नव॒ वै पुरु॑षे प्रा॒णाः। नाभि॑र्दश॒मी। प्रा॒णा इ॑न्द्रि॒यं वी॒र्यम्। प्रा॒णाने॒वेन्द्रि॒यं वी॒र्यं॑ यज॑मान आ॒त्मन्ध॑त्ते। सीसे॑न क्ली॒बाच्छष्पा॑णि क्रीणाति। न वा ए॒तदयो॒ न हिर॑ण्यम्॥१७॥

%1.8.5.4
यत्सीसम्। न स्त्री न पुमान्॑। यत्क्ली॒बः। न सोमो॒ न सुरा। यत्सौत्राम॒णी समृ॑द्ध्यै। स्वा॒द्वीन्त्वा स्वा॒दुनेत्या॑ह। सोम॑मे॒वैनां करोति। सोमोऽस्य॒श्विभ्यां पच्यस्व॒ सर॑स्वत्यै पच्य॒स्वेन्द्रा॑य सु॒त्राम्णे॑ पच्य॒स्वेत्या॑ह। ए॒ताभ्यो॒ ह्ये॑षा दे॒वताभ्य॒ पच्य॑ते। ति॒स्रः ससृ॑ष्टा वसति॥१८॥

%1.8.5.5
ति॒स्रो हि रात्री क्री॒तः सोमो॒ वस॑ति। पु॒नातु॑ ते परि॒स्रुत॒मिति॒ यजु॑षा पुनाति॒ व्यावृ॑त्त्यै। प॒वित्रे॑ण पुनाति। प॒वित्रे॑ण॒ हि सोमं॑ पु॒नन्ति॑। वारे॑ण॒ शश्व॑ता॒ तनेत्या॑ह। वारे॑ण॒ हि सोमं॑ पु॒नन्ति॑। वा॒युः पू॒तः प॒वित्रे॒णेति॒ नैतया॑ पुनीयात्। व्यृ॑द्धा॒ ह्ये॑षा। अ॒ति॒प॒वि॒तस्यै॒तया॑ पुनी॒यात्। कु॒विद॒ङ्गेत्यनि॑रुक्तया प्राजाप॒त्यया॑ गृह्णाति॥१९॥

%1.8.5.6
अनि॑रुक्तः प्र॒जाप॑तिः। प्र॒जाप॑ते॒राप्त्यै। एक॑य॒र्चा गृ॑ह्णाति। ए॒क॒धैव यज॑माने वी॒र्यं॑ दधाति। आ॒श्वि॒नन्धू॒म्रमाल॑भते। अ॒श्विनौ॒ वै दे॒वानां भि॒षजौ। ताभ्या॑मे॒वास्मै॑ भेष॒जं क॑रोति। सा॒र॒स्व॒तं मे॒षम्। वाग्वै सर॑स्वती। वा॒चैवैनं॑ भिषज्यति। ऐ॒न्द्रमृ॑ष॒भ सेन्द्र॒त्वाय॑॥२०॥\anuvakamend[अक्ष्यो॒र्लोमा॑नि॒ हिर॑ण्यं वसति गृह्णाति भिषज्य॒त्येकं च]

%1.8.6.1
यत्त्रि॒षु यूपेष्वा॒लभे॑त। ब॒हि॒र्धाऽस्मा॑दिन्द्रि॒यं वी॒र्यं॑ दध्यात्। भ्रातृ॑व्यमस्मै जनयेत्। ए॒क॒यू॒प आल॑भते। ए॒क॒धैवास्मि॑न्निन्द्रि॒यं वी॒र्यं॑ दधाति। नास्मै॒ भ्रातृ॑व्यञ्जनयति। नैतेषां पशू॒नां पु॑रो॒डाशा॑ भवन्ति। ग्रह॑पुरोडाशा॒ ह्ये॑ते। यु॒व सु॒राम॑मश्वि॒नेति॑ सर्वदेव॒त्ये॑ याज्यानुवा॒क्ये॑ भवतः। सर्वा॑ ए॒व दे॒वता प्रीणाति॥२१॥

%1.8.6.2
ब्रा॒ह्म॒णं परि॑क्रीणीयादु॒च्छेष॑णस्य पा॒तारम्। ब्रा॒ह्म॒णो ह्याहु॑त्या उ॒च्छेष॑णस्य पा॒ता। यदि॑ ब्राह्म॒णन्न वि॒न्देत्। व॒ल्मी॒क॒व॒पाया॒मव॑ नयेत्। सैव तत॒ प्राय॑श्चित्तिः। यद्वै सौत्राम॒ण्यै व्यृ॑द्धम्। तद॑स्यै॒ समृ॑द्धम्। ना॒ना॒दे॒व॒त्या प॒शव॑श्च पुरो॒डाशाश्च भवन्ति॒ समृ॑द्ध्यै। ऐ॒न्द्रः प॑शू॒नामु॑त्त॒मो भ॑वति। ऐ॒न्द्रः पु॑रो॒डाशा॑नां प्रथ॒मः॥२२॥

%1.8.6.3
इ॒न्द्रि॒ये ए॒वास्मै॑ स॒मीची॑ दधाति। पु॒रस्ता॑दनूया॒जानां पुरो॒डाशै॒ प्रच॑रति। प॒शवो॒ वै पु॑रो॒डाशा। प॒शूने॒वाव॑ रुन्धे। ऐ॒न्द्रमेका॑दशकपालं॒ निर्व॑पति। इ॒न्द्रि॒यमे॒वाव॑ रुन्धे। सा॒वि॒त्रन्द्वाद॑शकपालं॒ प्रसूत्यै। वा॒रु॒णन्दश॑कपालम्। अ॒न्त॒त ए॒व वरु॑ण॒मव॑ यजते। वड॑बा॒ दक्षि॑णा॥२३॥

%1.8.6.4
उ॒त वा ए॒षाऽश्व सू॒ते। उ॒ताऽश्व॑त॒रम्। उ॒त सोम॑ उ॒त सुरा। यत्सौत्राम॒णी समृ॑द्ध्यै। बा॒र्॒ह॒स्प॒त्यं प॒शुञ्च॑तु॒र्थम॑तिपवि॒तस्या ल॑भते। ब्रह्म॒ वै दे॒वानां॒ बृह॒स्पति॑। ब्रह्म॑णै॒व य॒ज्ञस्य॒ व्यृ॑द्ध॒मपि॑ वपति। पु॒रो॒डाश॑वाने॒ष प॒शुर्भ॑वति। न ह्ये॑तस्य॒ ग्रहं॑ गृ॒ह्णन्ति॑। सोम॑प्रतीकाः पितरस्तृप्णु॒तेति॑ शतातृ॒ण्णाया स॒मव॑नयति॥२४॥

%1.8.6.5
श॒तायु॒ पुरु॑षः श॒तेन्द्रि॑यः। आयु॑ष्ये॒वेन्द्रि॒ये प्रति॑तिष्ठति। दक्षि॑णे॒ऽग्नौ जु॑होति। पा॒प॒व॒स्य॒सस्य॒ व्यावृ॑त्त्यै। हिर॑ण्यमन्त॒रा धा॑रयति। पू॒तामे॒वैनां जुहोति। श॒तमा॑नं भवति। श॒तायु॒ पुरु॑षः श॒तेन्द्रि॑यः। आयु॑ष्ये॒वेन्द्रि॒ये प्रति॑तिष्ठति। यत्रै॒व श॑तातृ॒ण्णान्धा॒रय॑ति॥२५॥

%1.8.6.6
तन्निद॑धाति॒ प्रति॑ष्ठित्यै। पि॒तॄन् वा ए॒तस्येन्द्रि॒यं वी॒र्यं॑ गच्छति। य सोमो॑ऽति॒ पव॑ते। पि॒तृ॒णां याज्यानुवा॒क्या॑भि॒रुप॑ तिष्ठते। यदे॒वास्य॑ पि॒तॄनि॑न्द्रि॒यं वी॒र्यं॑ गच्छ॑ति। तदे॒वाव॑ रुन्धे। ति॒सृभि॒रुप॑ तिष्ठते। तृ॒तीये॒ वा इ॒तो लो॒के पि॒तर॑। ताने॒व प्री॑णाति। अथो॒ त्रीणि॒ वै य॒ज्ञस्येन्द्रि॒याणि॑। अ॒ध्व॒र्युर्\mbox{}होता ब्र॒ह्मा। त उप॑तिष्ठन्ते। यान्ये॒व य॒ज्ञस्येन्द्रि॒याणि॑। तैरे॒वास्मै॑ भेष॒जं क॑रोति॥२६॥\anuvakamend[प्री॒णा॒ति॒ प्र॒थ॒मो दक्षि॑णा स॒मव॑नयति धा॒रय॑तीन्द्रि॒याणि॑ च॒त्वारि॑ च]

%1.8.7.1
अ॒ग्नि॒ष्टो॒ममग्र॒ आह॑रति। य॒ज्ञ॒मु॒खं वा अ॑ग्निष्टो॒मः। य॒ज्ञ॒मु॒खमे॒वारभ्य॑ स॒वमा क्र॑मते। अथै॒षो॑ऽभिषेच॒नीय॑श्चतुस्त्रि॒शप॑वमानो भवति। त्रय॑स्त्रिश॒द्वै दे॒वता। ता ए॒वाप्नो॑ति। प्र॒जाप॑तिश्चतुस्त्रि॒शः। तमे॒वाप्नो॑ति। स॒श॒र ए॒ष स्तोमा॑ना॒मय॑थापूर्वम्। यद्विष॑मा॒ स्तोमा॥२७॥

%1.8.7.2
ए॒तावा॒न् वै य॒ज्ञः। यावा॒न्पव॑मानाः। अ॒न्त॒ श्लेष॑ण॒न्त्वा अ॒न्यत्। यत्स॒माः पव॑मानाः। तेनाऽसशरः। तेन॑ यथापू॒र्वम्। आ॒त्मनै॒वाग्नि॑ष्टो॒मेन॒र्ध्नोति॑। आ॒त्मना॒ पुण्यो॑ भवति। प्र॒जा वा उ॒क्थानि॑। प॒शव॑ उ॒क्थानि॑। यदु॒क्थ्यो॑ भव॒त्यनु॒ सन्त॑त्त्यै॥२८॥\anuvakamend[स्तोमा प॒शव॑ उ॒क्थान्येकं च]

%1.8.8.1
उप॑ त्वा जा॒मयो॒ गिर॒ इति॑ प्रति॒पद्भ॑वति। वाग्वै वा॒युः। वा॒च ए॒वैषो॑ऽभिषे॒कः। सर्वा॑सामे॒व प्र॒जाना सूयते। सर्वा॑ एनं प्र॒जा राजेति॑ वदन्ति। ए॒तमु॒ त्यन्दश॒ क्षिप॒ इत्या॑ह। आ॒दि॒त्या वै प्र॒जाः। प्र॒जाना॑मे॒वैतेन॑ सूयते। यन्ति॒ वा ए॒ते य॑ज्ञमु॒खात्। ये स॑म्भा॒र्या॑ अक्र\sn{}॥२९॥

%1.8.8.2
यदाह॒ पव॑स्व वा॒चो अ॑ग्रिय॒ इति॑। तेनै॒व य॑ज्ञमु॒खान्नय॑न्ति। अ॒नु॒ष्टुक्प्र॑थ॒मा भ॑वति। अ॒नु॒ष्टुगु॑त्त॒मा। वाग्वा अ॑नु॒ष्टुक्। वा॒चैव प्र॒यन्ति॑। वा॒चोद्य॑न्ति। उद्व॑तीर्भवन्ति। उद्व॒द्वा अ॑नु॒ष्टुभो॑ रू॒पम्। आनु॑ष्टुभो राज॒न्य॑॥३०॥

%1.8.8.3
तस्मा॒दुद्व॑तीर्भवन्ति। सौ॒र्य॑नु॒ष्टुगु॑त्त॒मा भ॑वति। सु॒व॒र्गस्य॑ लो॒कस्य॒ सन्त॑त्यै। यो वै स॒वादेति॑। नैन स॒व उप॑नमति। यः साम॑भ्य॒ एति॑। पापी॑यान्त्सुषुवा॒णो भ॑वति। ए॒तानि॒ खलु॒ वै सामा॑नि। यत्पृ॒ष्ठानि॑। यत्पृ॒ष्ठानि॒ भव॑न्ति॥३१॥

%1.8.8.4
तैरे॒व स॒वान्नैति॑। यानि॑ देवरा॒जाना॒ सामा॑नि। तैर॒मुष्मि॑ल्लोँ॒क ऋ॑ध्नोति। यानि॑ मनुष्यरा॒जाना॒ सामा॑नि। तैर॒स्मिल्लोँ॒क ऋ॑ध्नोति। उ॒भयो॑रे॒व लो॒कयोर्॑ ऋध्नोति। दे॒व॒लो॒के च॑ मनुष्यलो॒के च॑। ए॒क॒वि॒शो॑ऽभिषेच॒नीय॑स्योत्त॒मो भ॑वति। ए॒क॒वि॒शः के॑शवप॒नीय॑स्य प्रथ॒मः। स॒प्त॒द॒शो द॑श॒पेय॑॥३२॥

%1.8.8.5
विड्वा ए॑कवि॒शः। रा॒ष्ट्र स॑प्तद॒शः। विश॑ ए॒वैतन्म॑ध्य॒तो॑ऽभिषि॑च्यते। तस्मा॒द्वा ए॒ष वि॒शां प्रि॒यः। वि॒शो हि म॑ध्य॒तो॑ऽभिषि॒च्यते। यद्वा ए॑नम॒दो दिशोऽनु॑ व्यास्था॒पय॑न्ति। तत्सु॑व॒र्गल्लो॒कम॒भ्या रो॑हति। यदि॒मल्लोँ॒कन्न प्र॑त्यव॒रोहेत्। अ॒ति॒ज॒नं वे॒यात्। उद्वा॑ माद्येत्। यदे॒ष प्र॑ती॒चीन॑ स्तोमो॒ भव॑ति। इ॒ममे॒व तेन॑ लो॒कं प्र॒त्यव॑रोहति। अथो॑ अ॒स्मिन्ने॒व लो॒के प्रति॑ तिष्ठ॒त्यनु॑न्मादाय॥३३॥\anuvakamend[अक्र॑न्राज॒न्यो॑ भव॑न्ति दश॒पेयो॑ माद्ये॒त्रीणि॑ च]

%1.8.9.1
इ॒यं वै॒ र॑ज॒ता। अ॒सौ हरि॑णी। यद्रु॒क्मौ भव॑तः। आ॒भ्यामे॒वैन॑मुभ॒यत॒ परि॑ गृह्णाति। वरु॑णस्य॒ वा अ॑भिषि॒च्यमा॑न॒स्याप॑। इ॒न्द्रि॒यं वी॒र्य॑न्निर॑घ्नन्। तत्सु॒वर्ण॒ हिर॑ण्यमभवत्। यद्रु॒क्मम॑न्त॒र्दधा॑ति। इ॒न्द्रि॒यस्य॑ वी॒र्य॑स्यानि॑र्घाताय। श॒तमा॑नो भवति श॒तक्ष॑रः। श॒तायु॒ पुरु॑षः श॒तेन्द्रि॑यः। आयु॑ष्ये॒वेन्द्रि॒ये प्रति॑तिष्ठति। आयु॒र्वै हिर॑ण्यम्। आ॒यु॒ष्या॑ ए॒वैन॑म॒भ्यति॑ क्षरन्ति। तेजो॒ वै हिर॑ण्यम्। ते॒ज॒स्या॑ ए॒वैन॑म॒भ्यति॑ क्षरन्ति। वर्चो॒ वै हिर॑ण्यम्। व॒र्च॒स्या॑ ए॒वैन॑म॒भ्यति॑ क्षरन्ति॥३४॥\anuvakamend[श॒तक्ष॑रो॒ऽष्टौ च॑]

%1.8.10.1
अप्र॑तिष्ठितो॒ वा ए॒ष इत्या॑हुः। यो रा॑ज॒सूये॑न॒ यज॑त॒ इति॑। य॒दा वा ए॒ष ए॒तेन॑ द्विरा॒त्रेण॒ यज॑ते। अथ॑ प्रति॒ष्ठा। अथ॑ संवत्स॒रमाप्नोति। याव॑न्ति संवत्स॒रस्या॑होरा॒त्राणि॑। ताव॑तीरे॒तस्य॑ स्तो॒त्रीया। अ॒हो॒रा॒त्रेष्वे॒व प्रति॑ तिष्ठति। अ॒ग्नि॒ष्टो॒मः पूर्व॒मह॑र्भवति। अ॒ति॒रा॒त्र उत्त॑रम्॥३५॥

%1.8.10.2
नानै॒वाहो॑रा॒त्रयो॒ प्रति॑ तिष्ठति। पौ॒र्ण॒मा॒स्यां पूर्व॒मह॑र्भवति। व्य॑ष्टकाया॒मुत्त॑रम्। नानै॒वार्ध॑मा॒सयो॒ प्रति॑तिष्ठति। अ॒मा॒वा॒स्या॑यां॒ पूर्व॒मह॑र्भवति। उद्दृ॑ष्ट॒ उत्त॑रम्। नानै॒व मास॑यो॒ प्रति॑तिष्ठति। अथो॒ खलु॑। ये ए॒व स॑मानप॒क्षे पु॑ण्या॒हे स्याताम्। तयो का॒र्यं॑ प्रति॑ष्ठित्यै॥३६॥

%1.8.10.3
अ॒प॒श॒व्यो द्वि॑रा॒त्र इत्या॑हुः। द्वे ह्ये॑ते छन्द॑सी। गा॒य॒त्रं च॒ त्रैष्टु॑भं च। जग॑तीम॒न्तर्य॑न्ति। न तेन॒ जग॑ती कृ॒तेत्या॑हुः। यदे॑नान्तृतीयसव॒ने कु॒र्वन्तीति॑। य॒दा वा ए॒षाऽहीन॒स्याह॒र्भज॑ते। सा॒ह्नस्य॑ वा॒ सव॑नम्। अथै॒व जग॑ती कृ॒ता। अथ॑ पश॒व्य॑। व्यु॑ष्टि॒र्वा ए॒ष द्वि॑रा॒त्रः। य ए॒वं वि॒द्वान्द्वि॑रा॒त्रेण॒ यज॑ते। व्ये॑वास्मा॑ उच्छति। अथो॒ तम॑ ए॒वाप॑ हते। अ॒ग्नि॒ष्टो॒मम॑न्त॒त आ ह॑रति। अ॒ग्निः सर्वा॑ दे॒वता। दे॒वतास्वे॒व प्रति॑ तिष्ठति॥३७॥\anuvakamend[उत्त॑रं॒ प्रति॑ष्ठित्यै पश॒व्य॑ स॒प्त च॑]
\prashnaend{वरु॑णस्य जा॒मि वा ईश्व॒र आग्ने॒यमिन्द्र॑स्य॒ यत्त्रि॒ष्व॑ग्निष्टो॒ममुप॑ त्वे॒यं वै र॑ज॒ताऽप्र॑तिष्ठितो॒ दश॑॥१०॥}{वरु॑णस्य॒ यद॒श्विभ्यां॒ यत्त्रि॒षु तस्मा॒दुद्व॑तीः स॒प्तत्रिशत्॥३७॥}{वरु॑णस्य॒ प्रति॑तिष्ठति॥}{हरि॑ ओम्॥}{इति श्रीकृष्णयजुर्वेदीयतैत्तिरीयब्राह्मणे प्रथमाष्टके अष्टमः प्रपाठकः समाप्तः॥}
\clearpage
%%% END ASHTAKAM
