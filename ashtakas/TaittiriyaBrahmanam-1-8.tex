\sect{अष्टमः प्रश्नः}
\setcounter{anuvakam}{0}
\dnsub{तैत्तिरीयब्राह्मणे प्रथमाष्टके अष्टमः प्रपाठकः}

%1.8.1.1
वरु॑णस्य सुषुवा॒णस्य॑ दश॒धेन्द्रि॒यं वी॒र्यं॑ परा॑ऽपतत्। तत्स॒सृद्भि॒रनु॒ सम॑सर्पत्। तत्स॒सृपा ससृ॒त्त्वम्। अ॒ग्निना॑ दे॒वेन॑ प्रथ॒मेऽह॒न्ननु॒ प्रायु॑ङ्क्त। सर॑स्वत्या वा॒चा द्वि॒तीये। स॒वि॒त्रा प्र॑स॒वेन॑ तृ॒तीये। पू॒ष्णा प॒शुभि॑श्चतु॒र्थे। बृह॒स्पति॑ना॒ ब्रह्म॑णा पञ्च॒मे। इन्द्रे॑ण दे॒वेन॑ ष॒ष्ठे। वरु॑णेन॒ स्वया॑ दे॒वत॑या सप्त॒मे॥१॥

%1.8.1.2
सोमे॑न॒ राज्ञाऽष्ट॒मे। त्वष्ट्रा॑ रू॒पेण॑ नव॒मे। विष्णु॑ना य॒ज्ञेनाप्नोत्। यत्स॒सृपो॒ भव॑न्ति। इ॒न्द्रि॒यमे॒व तद्वी॒र्यं॑ यज॑मान आप्नोति। पूर्वा॑पूर्वा॒ वेदि॑र्भवति। इ॒न्द्रि॒यस्य॑ वी॒र्य॑स्याव॑रुद्ध्यै। पु॒रस्ता॑दुप॒सदा सौ॒म्येन॒ प्रच॑रति। सोमो॒ वै रे॑तो॒धाः। रेत॑ ए॒व तद्द॑धाति। अ॒न्त॒रा त्वा॒ष्ट्रेण॑। रेत॑ ए॒व हि॒तन्त्वष्टा॑ रू॒पाणि॒ विक॑रोति। उ॒परि॑ष्टाद्वैष्ण॒वेन॑। य॒ज्ञो वै विष्णु॑। य॒ज्ञ ए॒वान्त॒तः प्रति॑ तिष्ठति॥२॥\anuvakamend[स॒प्त॒मे द॑धाति॒ पञ्च॑ च]

%1.8.2.1
जा॒मि वा ए॒तत्कु॑र्वन्ति। यत्स॒द्यो दी॒क्षय॑न्ति स॒द्यः सोमं॑ क्री॒णन्ति॑। पु॒ण्ड॒रि॒स्र॒जां प्रय॑च्छ॒त्यजा॑मित्वाय। अङ्गि॑रसः सुव॒र्गं लो॒कं यन्त॑। अ॒प्सु दीक्षात॒पसी॒ प्रावे॑शयन्। तत्पु॒ण्डरी॑कमभवत्। यत्पु॑ण्डरिस्र॒जां प्र॒यच्छ॑ति। सा॒क्षादे॒व दीक्षात॒पसी॒ अव॑रुन्धे। द॒शभि॑र्वत्सत॒रैः सोमं॑ क्रीणाति। दशाक्षरा वि॒राट्॥३॥

%1.8.2.2
अन्नं॑ वि॒राट्। वि॒राजै॒वान्नाद्य॒मव॑ रुन्धे। मु॒ष्क॒रा भ॑वन्ति सेन्द्र॒त्वाय॑। द॒श॒पेयो॑ भवति। अ॒न्नाद्य॒स्याव॑रुद्ध्यै। श॒तं ब्राह्म॒णाः पि॑बन्ति। श॒तायु॒ पुरु॑षः श॒तेन्द्रि॑यः। आयु॑ष्ये॒वेन्द्रि॒ये प्रति॑तिष्ठति। स॒प्त॒द॒श स्तो॒त्रं भ॑वति। स॒प्त॒द॒शः प्र॒जाप॑तिः॥४॥

%1.8.2.3
प्र॒जाप॑ते॒राप्त्यै। प्रा॒का॒शाव॑ध्व॒र्यवे॑ ददाति। प्र॒का॒शमे॒वैन॑ङ्गमयति। स्रज॑मुद्गा॒त्रे। व्ये॑वास्मै॑ वासयति। रु॒क्म होत्रे। आ॒दि॒त्यमे॒वास्मा॒ उन्न॑यति। अश्वं॑ प्रस्तोतृप्रतिह॒र्तृभ्याम्। प्रा॒जा॒प॒त्यो वा अश्व॑। प्र॒जाप॑ते॒राप्त्यै॥५॥

%1.8.2.4
द्वाद॑श पष्ठौ॒हीर्ब्र॒ह्मणे। आयु॑रे॒वाव॑रुन्धे। व॒शां मैत्रावरु॒णाय॑। रा॒ष्ट्रमे॒व व॒श्य॑कः। ऋ॒ष॒भं ब्राह्मणाच्छ॒सिने। रा॒ष्ट्रमे॒वेन्द्रि॑या॒व्य॑कः। वास॑सी नेष्टापो॒तृभ्याम्। प॒वित्रे॑ ए॒वास्यै॒ते। स्थूरि॑ यवाचि॒तम॑च्छावा॒काय॑। अ॒न्त॒त ए॒व वरु॑ण॒मव॑ यजते॥६॥

%1.8.2.5
अ॒नड्वाह॑म॒ग्नीधे। वह्नि॒र्वा अ॑न॒ड्वान्। वह्नि॑र॒ग्नीत्। वह्नि॑नै॒व वह्नि॑ य॒ज्ञस्याव॑रुन्धे। इन्द्र॑स्य सुषुवा॒णस्य॑ त्रे॒धेन्द्रि॒यं वी॒र्यं॑ परा॑ऽपतत्। भृगु॒स्तृती॑यमभवत्। श्रा॒य॒न्तीय॒न्तृती॑यम्। सर॑स्वती॒ तृती॑यम्। भा॒र्ग॒वो होता॑ भवति। श्रा॒य॒न्तीयं॑ ब्रह्मसा॒मं भ॑वति। वा॒र॒व॒न्तीय॑मग्निष्टोमसा॒मम्। सा॒र॒स्व॒तीर॒पो गृ॑ह्णाति। इ॒न्द्रि॒यस्य॑ वी॒र्य॑स्याव॑रुद्ध्यै। श्रा॒य॒न्तीयं॑ ब्रह्मसा॒मं भ॑वति। इ॒न्द्रि॒यमे॒वास्मि॑न्वी॒र्य श्रयति। वा॒र॒व॒न्तीय॑मग्निष्टोमसा॒मम्। इ॒न्द्रि॒यमे॒वास्मि॑न्वी॒र्यं॑ वारयति॥७॥\anuvakamend[वि॒राट्प्र॒जाप॑ति॒रश्व॑ प्र॒जाप॑ते॒राप्त्यै॑ यजते ब्रह्मसा॒मं भ॑वति स॒प्त च॑]

%1.8.3.1
ई॒श्व॒रो वा ए॒ष दिशोऽनून्म॑दितोः। यन्दिशोऽनु॑ व्यास्था॒पय॑न्ति। दि॒शामवेष्टयो भवन्ति। दि॒क्ष्वे॑व प्रति॑ तिष्ठ॒त्यनु॑न्मादाय। पञ्च॑ दे॒वता॑ यजति। पञ्च॒ दिश॑। दि॒क्ष्वे॑व प्रति॑ तिष्ठति। ह॒विषो॑हविष इ॒ष्ट्वा बा॑र्\mbox{}हस्प॒त्यम॒भिघा॑रयति। य॒ज॒मा॒न॒दे॒व॒त्यो॑ वै बृह॒स्पति॑। यज॑मानमे॒व तेज॑सा॒ सम॑र्धयति॥८॥

%1.8.3.2
आ॒दि॒त्यां म॒ल्॒हाङ्ग॒र्भिणी॒मा ल॑भते। मा॒रु॒तीं पृश्ञिं॑ पष्ठौ॒हीम्। विशं॑ चैवास्मै॑ रा॒ष्ट्रं च॑ स॒मीची॑ दधाति। आ॒दि॒त्यया॒ पूर्व॑या॒ प्रच॑रति। मा॒रु॒त्योत्त॑रया। रा॒ष्ट्र ए॒व विश॒मनु॑बध्नाति। उ॒च्चैरा॑दि॒त्याया॒ आश्रा॑वयति। उ॒पा॒शु मा॑रु॒त्यै। तस्माद्रा॒ष्ट्रं विश॒मति॑वदति। ग॒र्भिण्या॑दि॒त्या भ॑वति॥९॥

%1.8.3.3
इ॒न्द्रि॒यं वै गर्भ॑। रा॒ष्ट्रमे॒वेन्द्रि॑या॒व्य॑कः। अ॒ग॒र्भा मा॑रु॒ती। विड्वै म॒रुत॑। विश॑मे॒व निरि॑न्द्रियामकः। दे॒वा॒सु॒राः संय॑त्ता आसन्। ते दे॒वा अ॒श्विनो पू॒षन्वा॒चः स॒त्य स॑न्नि॒धाय॑। अनृ॑ते॒नासु॑रान॒भ्य॑भवन्। तेऽश्विभ्यां पू॒ष्णे पु॑रो॒डाश॒न्द्वाद॑शकपालं॒ निर॑वपन्। ततो॒ वै ते वा॒चः स॒त्यमवा॑रुन्धत॥१०॥

%1.8.3.4
यद॒श्विभ्यां पू॒ष्णे पु॑रो॒डाश॒न्द्वाद॑शकपालन्नि॒र्वप॑ति। अनृ॑तेनै॒व भ्रातृ॑व्यानभि॒भूय॑। वा॒चः स॒त्यमव॑रुन्धे। सर॑स्वते सत्य॒वाचे॑ च॒रुम्। पूर्व॑मे॒वोदि॒तम्। उत्त॑रेणा॒भि गृ॑णाति। स॒वि॒त्रे स॒त्यप्र॑सवाय पुरो॒डाश॒न्द्वाद॑शकपालं॒ प्रसूत्यै। दू॒तान्प्रहि॑णोति। आ॒विद॑ ए॒ता भ॑वन्ति। आ॒विद॑मे॒वैन॑ङ्गमयन्ति। अथो॑ दू॒तेभ्य॑ ए॒व न छि॑द्यते। ति॒सृ॒ध॒न्व शु॑ष्कदृ॒तिर्दक्षि॑णा॒ समृ॑द्ध्यै॥११॥\anuvakamend[अ॒र्ध॒य॒ति॒ भ॒व॒त्य॒रु॒न्ध॒त॒ ग॒म॒य॒न्ति॒ द्वे च॑]

%1.8.4.1
आ॒ग्ने॒यम॒ष्टाक॑पालं॒ निर्व॑पति। तस्मा॒च्छिशि॑रे कुरुपञ्चा॒लाः प्राञ्चो॑ यान्ति। सौ॒म्यञ्च॒रुम्। तस्माद्वस॒न्तव्व्यँ॑व॒साया॑दयन्ति। सा॒वि॒त्रन्द्वाद॑शकपालम्। तस्मात्पु॒रस्ता॒द्यवा॑ना सवि॒त्रा विरु॑न्धते। बा॒र्॒ह॒स्प॒त्यञ्च॒रुम्। स॒वि॒त्रैव वि॒रुध्य॑। ब्रह्म॑णा॒ यवा॒नाद॑धते। त्वा॒ष्ट्रम॒ष्टाक॑पालम्॥१२॥

%1.8.4.2
रू॒पाण्ये॒व तेन॑ कुर्वते। वै॒श्वा॒न॒रन्द्वाद॑शकपालम्। तस्माज्जघ॒न्ये॑ नैदा॑घे प्र॒त्यञ्च॑ कुरुपञ्चा॒ला यान्ति। सा॒र॒स्व॒तञ्च॒रुन्निर्व॑पति। तस्मात्प्रा॒वृषि॒ सर्वा॒ वाचो॑ वदन्ति। पौ॒ष्णेन॒ व्यव॑स्यन्ति। मै॒त्रेण॑ कृषन्ते। वा॒रु॒णेन॒ विधृ॑ता आसते। क्षै॒त्र॒प॒त्येन॑ पाचयन्ते। आ॒दि॒त्येनाद॑धते॥१३॥

%1.8.4.3
मा॒सिमास्ये॒तानि॑ ह॒वीषि॑ नि॒रुप्या॒णीत्या॑हुः। तेनै॒वर्तून्प्रयु॑ङ्क्त॒ इति॑। अथो॒ खल्वा॑हुः। कः सं॑वत्स॒रञ्जी॑विष्य॒तीति॑। षडे॒व पूर्वे॒द्युर्नि॒रुप्या॑णि। षडु॑त्तरे॒द्युः। तेनै॒वर्तून्प्रयु॑ङ्क्ते। दक्षि॑णो रथवाहनवा॒हः पूर्वे॑षा॒न्दक्षि॑णा। उत्त॑र॒ उत्त॑रेषाम्। सं॒व॒त्स॒रस्यै॒वान्तौ॑ युनक्ति। सु॒व॒र्गस्य॑ लो॒कस्य॒ सम॑ष्ट्यै॥१४॥\anuvakamend[त्वा॒ष्ट्रम॒ष्टाक॑पालन्दधते युन॒क्त्येकं॑ च]

%1.8.5.1
इन्द्र॑स्य सुषुवा॒णस्य॑ दश॒धेन्द्रि॒यं वी॒र्यं॑ परा॑ऽपतत्। स यत्प्र॑थ॒मन्नि॒रष्ठी॑वत्। तत्क्व॑लमभवत्। यद्द्वि॒तीयम्। तद्बद॑रम्। यत्तृ॒तीयम्। तत्क॒र्कन्धु॑। यन्न॒स्तः। स सि॒हः। यदक्ष्यो॥१५॥

%1.8.5.2
स शार्दू॒लः। यत्कर्ण॑योः। स वृक॑। य ऊ॒र्ध्वः। स सोम॑। याऽवा॑ची। सा सुरा। त्र॒याः सक्त॑वो भवन्ति। इ॒न्द्रि॒यस्याव॑रुद्ध्यै। त्र॒याणि॒ लोमा॑नि॥१६॥

%1.8.5.3
त्विषि॑मे॒वाव॑रुन्धे। त्रयो॒ ग्रहा। वी॒र्य॑मे॒वाव॑रुन्धे। नाम्ना॑ दश॒मी। नव॒ वै पुरु॑षे प्रा॒णाः। नाभि॑र्दश॒मी। प्रा॒णा इ॑न्द्रि॒यं वी॒र्यम्। प्रा॒णाने॒वेन्द्रि॒यं वी॒र्यं॑ यज॑मान आ॒त्मन्ध॑त्ते। सीसे॑न क्ली॒बाच्छष्पा॑णि क्रीणाति। न वा ए॒तदयो॒ न हिर॑ण्यम्॥१७॥

%1.8.5.4
यत्सीसम्। न स्त्री न पुमान्॑। यत्क्ली॒बः। न सोमो॒ न सुरा। यत्सौत्राम॒णी समृ॑द्ध्यै। स्वा॒द्वीन्त्वा स्वा॒दुनेत्या॑ह। सोम॑मे॒वैनां करोति। सोमोऽस्य॒श्विभ्यां पच्यस्व॒ सर॑स्वत्यै पच्य॒स्वेन्द्रा॑य सु॒त्राम्णे॑ पच्य॒स्वेत्या॑ह। ए॒ताभ्यो॒ ह्ये॑षा दे॒वताभ्य॒ पच्य॑ते। ति॒स्रः ससृ॑ष्टा वसति॥१८॥

%1.8.5.5
ति॒स्रो हि रात्री क्री॒तः सोमो॒ वस॑ति। पु॒नातु॑ ते परि॒स्रुत॒मिति॒ यजु॑षा पुनाति॒ व्यावृ॑त्त्यै। प॒वित्रे॑ण पुनाति। प॒वित्रे॑ण॒ हि सोमं॑ पु॒नन्ति॑। वारे॑ण॒ शश्व॑ता॒ तनेत्या॑ह। वारे॑ण॒ हि सोमं॑ पु॒नन्ति॑। वा॒युः पू॒तः प॒वित्रे॒णेति॒ नैतया॑ पुनीयात्। व्यृ॑द्धा॒ ह्ये॑षा। अ॒ति॒प॒वि॒तस्यै॒तया॑ पुनी॒यात्। कु॒विद॒ङ्गेत्यनि॑रुक्तया प्राजाप॒त्यया॑ गृह्णाति॥१९॥

%1.8.5.6
अनि॑रुक्तः प्र॒जाप॑तिः। प्र॒जाप॑ते॒राप्त्यै। एक॑य॒र्चा गृ॑ह्णाति। ए॒क॒धैव यज॑माने वी॒र्यं॑ दधाति। आ॒श्वि॒नन्धू॒म्रमाल॑भते। अ॒श्विनौ॒ वै दे॒वानां भि॒षजौ। ताभ्या॑मे॒वास्मै॑ भेष॒जं क॑रोति। सा॒र॒स्व॒तं मे॒षम्। वाग्वै सर॑स्वती। वा॒चैवैनं॑ भिषज्यति। ऐ॒न्द्रमृ॑ष॒भ सेन्द्र॒त्वाय॑॥२०॥\anuvakamend[अक्ष्यो॒र्लोमा॑नि॒ हिर॑ण्यं वसति गृह्णाति भिषज्य॒त्येकं च]

%1.8.6.1
यत्त्रि॒षु यूपेष्वा॒लभे॑त। ब॒हि॒र्धाऽस्मा॑दिन्द्रि॒यं वी॒र्यं॑ दध्यात्। भ्रातृ॑व्यमस्मै जनयेत्। ए॒क॒यू॒प आल॑भते। ए॒क॒धैवास्मि॑न्निन्द्रि॒यं वी॒र्यं॑ दधाति। नास्मै॒ भ्रातृ॑व्यञ्जनयति। नैतेषां पशू॒नां पु॑रो॒डाशा॑ भवन्ति। ग्रह॑पुरोडाशा॒ ह्ये॑ते। यु॒व सु॒राम॑मश्वि॒नेति॑ सर्वदेव॒त्ये॑ याज्यानुवा॒क्ये॑ भवतः। सर्वा॑ ए॒व दे॒वता प्रीणाति॥२१॥

%1.8.6.2
ब्रा॒ह्म॒णं परि॑क्रीणीयादु॒च्छेष॑णस्य पा॒तारम्। ब्रा॒ह्म॒णो ह्याहु॑त्या उ॒च्छेष॑णस्य पा॒ता। यदि॑ ब्राह्म॒णन्न वि॒न्देत्। व॒ल्मी॒क॒व॒पाया॒मव॑ नयेत्। सैव तत॒ प्राय॑श्चित्तिः। यद्वै सौत्राम॒ण्यै व्यृ॑द्धम्। तद॑स्यै॒ समृ॑द्धम्। ना॒ना॒दे॒व॒त्या प॒शव॑श्च पुरो॒डाशाश्च भवन्ति॒ समृ॑द्ध्यै। ऐ॒न्द्रः प॑शू॒नामु॑त्त॒मो भ॑वति। ऐ॒न्द्रः पु॑रो॒डाशा॑नां प्रथ॒मः॥२२॥

%1.8.6.3
इ॒न्द्रि॒ये ए॒वास्मै॑ स॒मीची॑ दधाति। पु॒रस्ता॑दनूया॒जानां पुरो॒डाशै॒ प्रच॑रति। प॒शवो॒ वै पु॑रो॒डाशा। प॒शूने॒वाव॑ रुन्धे। ऐ॒न्द्रमेका॑दशकपालं॒ निर्व॑पति। इ॒न्द्रि॒यमे॒वाव॑ रुन्धे। सा॒वि॒त्रन्द्वाद॑शकपालं॒ प्रसूत्यै। वा॒रु॒णन्दश॑कपालम्। अ॒न्त॒त ए॒व वरु॑ण॒मव॑ यजते। वड॑बा॒ दक्षि॑णा॥२३॥

%1.8.6.4
उ॒त वा ए॒षाऽश्व सू॒ते। उ॒ताऽश्व॑त॒रम्। उ॒त सोम॑ उ॒त सुरा। यत्सौत्राम॒णी समृ॑द्ध्यै। बा॒र्॒ह॒स्प॒त्यं प॒शुञ्च॑तु॒र्थम॑तिपवि॒तस्या ल॑भते। ब्रह्म॒ वै दे॒वानां॒ बृह॒स्पति॑। ब्रह्म॑णै॒व य॒ज्ञस्य॒ व्यृ॑द्ध॒मपि॑ वपति। पु॒रो॒डाश॑वाने॒ष प॒शुर्भ॑वति। न ह्ये॑तस्य॒ ग्रहं॑ गृ॒ह्णन्ति॑। सोम॑प्रतीकाः पितरस्तृप्णु॒तेति॑ शतातृ॒ण्णाया स॒मव॑नयति॥२४॥

%1.8.6.5
श॒तायु॒ पुरु॑षः श॒तेन्द्रि॑यः। आयु॑ष्ये॒वेन्द्रि॒ये प्रति॑तिष्ठति। दक्षि॑णे॒ऽग्नौ जु॑होति। पा॒प॒व॒स्य॒सस्य॒ व्यावृ॑त्त्यै। हिर॑ण्यमन्त॒रा धा॑रयति। पू॒तामे॒वैनां जुहोति। श॒तमा॑नं भवति। श॒तायु॒ पुरु॑षः श॒तेन्द्रि॑यः। आयु॑ष्ये॒वेन्द्रि॒ये प्रति॑तिष्ठति। यत्रै॒व श॑तातृ॒ण्णान्धा॒रय॑ति॥२५॥

%1.8.6.6
तन्निद॑धाति॒ प्रति॑ष्ठित्यै। पि॒तॄन् वा ए॒तस्येन्द्रि॒यं वी॒र्यं॑ गच्छति। य सोमो॑ऽति॒ पव॑ते। पि॒तृ॒णां याज्यानुवा॒क्या॑भि॒रुप॑ तिष्ठते। यदे॒वास्य॑ पि॒तॄनि॑न्द्रि॒यं वी॒र्यं॑ गच्छ॑ति। तदे॒वाव॑ रुन्धे। ति॒सृभि॒रुप॑ तिष्ठते। तृ॒तीये॒ वा इ॒तो लो॒के पि॒तर॑। ताने॒व प्री॑णाति। अथो॒ त्रीणि॒ वै य॒ज्ञस्येन्द्रि॒याणि॑। अ॒ध्व॒र्युर्\mbox{}होता ब्र॒ह्मा। त उप॑तिष्ठन्ते। यान्ये॒व य॒ज्ञस्येन्द्रि॒याणि॑। तैरे॒वास्मै॑ भेष॒जं क॑रोति॥२६॥\anuvakamend[प्री॒णा॒ति॒ प्र॒थ॒मो दक्षि॑णा स॒मव॑नयति धा॒रय॑तीन्द्रि॒याणि॑ च॒त्वारि॑ च]

%1.8.7.1
अ॒ग्नि॒ष्टो॒ममग्र॒ आह॑रति। य॒ज्ञ॒मु॒खं वा अ॑ग्निष्टो॒मः। य॒ज्ञ॒मु॒खमे॒वारभ्य॑ स॒वमा क्र॑मते। अथै॒षो॑ऽभिषेच॒नीय॑श्चतुस्त्रि॒शप॑वमानो भवति। त्रय॑स्त्रिश॒द्वै दे॒वता। ता ए॒वाप्नो॑ति। प्र॒जाप॑तिश्चतुस्त्रि॒शः। तमे॒वाप्नो॑ति। स॒श॒र ए॒ष स्तोमा॑ना॒मय॑थापूर्वम्। यद्विष॑मा॒ स्तोमा॥२७॥

%1.8.7.2
ए॒तावा॒न् वै य॒ज्ञः। यावा॒न्पव॑मानाः। अ॒न्त॒ श्लेष॑ण॒न्त्वा अ॒न्यत्। यत्स॒माः पव॑मानाः। तेनाऽसशरः। तेन॑ यथापू॒र्वम्। आ॒त्मनै॒वाग्नि॑ष्टो॒मेन॒र्ध्नोति॑। आ॒त्मना॒ पुण्यो॑ भवति। प्र॒जा वा उ॒क्थानि॑। प॒शव॑ उ॒क्थानि॑। यदु॒क्थ्यो॑ भव॒त्यनु॒ सन्त॑त्त्यै॥२८॥\anuvakamend[स्तोमा प॒शव॑ उ॒क्थान्येकं च]

%1.8.8.1
उप॑ त्वा जा॒मयो॒ गिर॒ इति॑ प्रति॒पद्भ॑वति। वाग्वै वा॒युः। वा॒च ए॒वैषो॑ऽभिषे॒कः। सर्वा॑सामे॒व प्र॒जाना सूयते। सर्वा॑ एनं प्र॒जा राजेति॑ वदन्ति। ए॒तमु॒ त्यन्दश॒ क्षिप॒ इत्या॑ह। आ॒दि॒त्या वै प्र॒जाः। प्र॒जाना॑मे॒वैतेन॑ सूयते। यन्ति॒ वा ए॒ते य॑ज्ञमु॒खात्। ये स॑म्भा॒र्या॑ अक्र\sn{}॥२९॥

%1.8.8.2
यदाह॒ पव॑स्व वा॒चो अ॑ग्रिय॒ इति॑। तेनै॒व य॑ज्ञमु॒खान्नय॑न्ति। अ॒नु॒ष्टुक्प्र॑थ॒मा भ॑वति। अ॒नु॒ष्टुगु॑त्त॒मा। वाग्वा अ॑नु॒ष्टुक्। वा॒चैव प्र॒यन्ति॑। वा॒चोद्य॑न्ति। उद्व॑तीर्भवन्ति। उद्व॒द्वा अ॑नु॒ष्टुभो॑ रू॒पम्। आनु॑ष्टुभो राज॒न्य॑॥३०॥

%1.8.8.3
तस्मा॒दुद्व॑तीर्भवन्ति। सौ॒र्य॑नु॒ष्टुगु॑त्त॒मा भ॑वति। सु॒व॒र्गस्य॑ लो॒कस्य॒ सन्त॑त्यै। यो वै स॒वादेति॑। नैन स॒व उप॑नमति। यः साम॑भ्य॒ एति॑। पापी॑यान्त्सुषुवा॒णो भ॑वति। ए॒तानि॒ खलु॒ वै सामा॑नि। यत्पृ॒ष्ठानि॑। यत्पृ॒ष्ठानि॒ भव॑न्ति॥३१॥

%1.8.8.4
तैरे॒व स॒वान्नैति॑। यानि॑ देवरा॒जाना॒ सामा॑नि। तैर॒मुष्मि॑ल्लोँ॒क ऋ॑ध्नोति। यानि॑ मनुष्यरा॒जाना॒ सामा॑नि। तैर॒स्मिल्लोँ॒क ऋ॑ध्नोति। उ॒भयो॑रे॒व लो॒कयोर्॑ ऋध्नोति। दे॒व॒लो॒के च॑ मनुष्यलो॒के च॑। ए॒क॒वि॒शो॑ऽभिषेच॒नीय॑स्योत्त॒मो भ॑वति। ए॒क॒वि॒शः के॑शवप॒नीय॑स्य प्रथ॒मः। स॒प्त॒द॒शो द॑श॒पेय॑॥३२॥

%1.8.8.5
विड्वा ए॑कवि॒शः। रा॒ष्ट्र स॑प्तद॒शः। विश॑ ए॒वैतन्म॑ध्य॒तो॑ऽभिषि॑च्यते। तस्मा॒द्वा ए॒ष वि॒शां प्रि॒यः। वि॒शो हि म॑ध्य॒तो॑ऽभिषि॒च्यते। यद्वा ए॑नम॒दो दिशोऽनु॑ व्यास्था॒पय॑न्ति। तत्सु॑व॒र्गल्लो॒कम॒भ्या रो॑हति। यदि॒मल्लोँ॒कन्न प्र॑त्यव॒रोहेत्। अ॒ति॒ज॒नं वे॒यात्। उद्वा॑ माद्येत्। यदे॒ष प्र॑ती॒चीन॑ स्तोमो॒ भव॑ति। इ॒ममे॒व तेन॑ लो॒कं प्र॒त्यव॑रोहति। अथो॑ अ॒स्मिन्ने॒व लो॒के प्रति॑ तिष्ठ॒त्यनु॑न्मादाय॥३३॥\anuvakamend[अक्र॑न्राज॒न्यो॑ भव॑न्ति दश॒पेयो॑ माद्ये॒त्रीणि॑ च]

%1.8.9.1
इ॒यं वै॒ र॑ज॒ता। अ॒सौ हरि॑णी। यद्रु॒क्मौ भव॑तः। आ॒भ्यामे॒वैन॑मुभ॒यत॒ परि॑ गृह्णाति। वरु॑णस्य॒ वा अ॑भिषि॒च्यमा॑न॒स्याप॑। इ॒न्द्रि॒यं वी॒र्य॑न्निर॑घ्नन्। तत्सु॒वर्ण॒ हिर॑ण्यमभवत्। यद्रु॒क्मम॑न्त॒र्दधा॑ति। इ॒न्द्रि॒यस्य॑ वी॒र्य॑स्यानि॑र्घाताय। श॒तमा॑नो भवति श॒तक्ष॑रः। श॒तायु॒ पुरु॑षः श॒तेन्द्रि॑यः। आयु॑ष्ये॒वेन्द्रि॒ये प्रति॑तिष्ठति। आयु॒र्वै हिर॑ण्यम्। आ॒यु॒ष्या॑ ए॒वैन॑म॒भ्यति॑ क्षरन्ति। तेजो॒ वै हिर॑ण्यम्। ते॒ज॒स्या॑ ए॒वैन॑म॒भ्यति॑ क्षरन्ति। वर्चो॒ वै हिर॑ण्यम्। व॒र्च॒स्या॑ ए॒वैन॑म॒भ्यति॑ क्षरन्ति॥३४॥\anuvakamend[श॒तक्ष॑रो॒ऽष्टौ च॑]

%1.8.10.1
अप्र॑तिष्ठितो॒ वा ए॒ष इत्या॑हुः। यो रा॑ज॒सूये॑न॒ यज॑त॒ इति॑। य॒दा वा ए॒ष ए॒तेन॑ द्विरा॒त्रेण॒ यज॑ते। अथ॑ प्रति॒ष्ठा। अथ॑ संवत्स॒रमाप्नोति। याव॑न्ति संवत्स॒रस्या॑होरा॒त्राणि॑। ताव॑तीरे॒तस्य॑ स्तो॒त्रीया। अ॒हो॒रा॒त्रेष्वे॒व प्रति॑ तिष्ठति। अ॒ग्नि॒ष्टो॒मः पूर्व॒मह॑र्भवति। अ॒ति॒रा॒त्र उत्त॑रम्॥३५॥

%1.8.10.2
नानै॒वाहो॑रा॒त्रयो॒ प्रति॑ तिष्ठति। पौ॒र्ण॒मा॒स्यां पूर्व॒मह॑र्भवति। व्य॑ष्टकाया॒मुत्त॑रम्। नानै॒वार्ध॑मा॒सयो॒ प्रति॑तिष्ठति। अ॒मा॒वा॒स्या॑यां॒ पूर्व॒मह॑र्भवति। उद्दृ॑ष्ट॒ उत्त॑रम्। नानै॒व मास॑यो॒ प्रति॑तिष्ठति। अथो॒ खलु॑। ये ए॒व स॑मानप॒क्षे पु॑ण्या॒हे स्याताम्। तयो का॒र्यं॑ प्रति॑ष्ठित्यै॥३६॥

%1.8.10.3
अ॒प॒श॒व्यो द्वि॑रा॒त्र इत्या॑हुः। द्वे ह्ये॑ते छन्द॑सी। गा॒य॒त्रं च॒ त्रैष्टु॑भं च। जग॑तीम॒न्तर्य॑न्ति। न तेन॒ जग॑ती कृ॒तेत्या॑हुः। यदे॑नान्तृतीयसव॒ने कु॒र्वन्तीति॑। य॒दा वा ए॒षाऽहीन॒स्याह॒र्भज॑ते। सा॒ह्नस्य॑ वा॒ सव॑नम्। अथै॒व जग॑ती कृ॒ता। अथ॑ पश॒व्य॑। व्यु॑ष्टि॒र्वा ए॒ष द्वि॑रा॒त्रः। य ए॒वं वि॒द्वान्द्वि॑रा॒त्रेण॒ यज॑ते। व्ये॑वास्मा॑ उच्छति। अथो॒ तम॑ ए॒वाप॑ हते। अ॒ग्नि॒ष्टो॒मम॑न्त॒त आ ह॑रति। अ॒ग्निः सर्वा॑ दे॒वता। दे॒वतास्वे॒व प्रति॑ तिष्ठति॥३७॥\anuvakamend[उत्त॑रं॒ प्रति॑ष्ठित्यै पश॒व्य॑ स॒प्त च॑]
\prashnaend{वरु॑णस्य जा॒मि वा ईश्व॒र आग्ने॒यमिन्द्र॑स्य॒ यत्त्रि॒ष्व॑ग्निष्टो॒ममुप॑ त्वे॒यं वै र॑ज॒ताऽप्र॑तिष्ठितो॒ दश॑॥१०॥}{वरु॑णस्य॒ यद॒श्विभ्यां॒ यत्त्रि॒षु तस्मा॒दुद्व॑तीः स॒प्तत्रिशत्॥३७॥}{वरु॑णस्य॒ प्रति॑तिष्ठति॥}{हरि॑ ओम्॥}{इति श्रीकृष्णयजुर्वेदीयतैत्तिरीयब्राह्मणे प्रथमाष्टके अष्टमः प्रपाठकः समाप्तः॥}
\clearpage
