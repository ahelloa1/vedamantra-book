\sect{अष्टमः प्रश्नः}
\setcounter{anuvakam}{0}
\dnsub{तैत्तिरीयब्राह्मणे तृतीयाष्टके अष्टमः प्रपाठकः}

%3.8.1.1
सा॒ङ्ग्र॒ह॒ण्येष्ट्या॑ यजते। इ॒माञ्ज॒नता॒ सङ्गृ॑ह्णा॒नीति॑। द्वाद॑शारत्नी रश॒ना भ॑वति। द्वाद॑श॒ मासा संवत्स॒रः। सं॒व॒त्स॒रमे॒वाव॑ रुन्धे। मौ॒ञ्जी भ॑वति। ऊर्ग्वै मुञ्जा। ऊर्ज॑मे॒वाव॑ रुन्धे। चि॒त्रा नक्ष॑त्रम्भवति। चि॒त्रं वा ए॒तत्कर्म॑॥१॥

%3.8.1.2
यद॑श्वमे॒धः समृ॑द्ध्यै। पुण्य॑नाम देव॒यज॑नम॒ध्यव॑स्यति। पुण्या॑मे॒व तेन॑ की॒र्तिम॒भि ज॑यति। अप॑दातीनृ॒त्विज॑ स॒माव॑ह॒न्त्या सु॑ब्रह्म॒ण्याया। सु॒व॒र्गस्य॑ लो॒कस्य॒ सम॑ष्ट्यै। के॒श॒श्म॒श्रु व॑पते। न॒खानि॒ नि कृ॑न्तते। द॒तो धा॑वते। स्नाति॑। अह॑तं॒ वास॒ परि॑धत्ते। पा॒प्मनोऽप॑हत्यै। वाचं॑ य॒त्वोप॑ वसति। सु॒व॒र्गस्य॑ लो॒कस्य॒ गुप्त्यै। रात्रिं॑ जाग॒रय॑न्त आसते। सु॒व॒र्गस्य॑ लो॒कस्य॒ सम॑ष्ट्यै॥२॥\anuvakamend[कर्म॑ धत्ते॒ पञ्च॑ च]

%3.8.2.1
चतु॑ष्टय्य॒ आपो॑ भवन्ति। चतु॑ शफो॒ वा अश्व॑ प्राजाप॒त्यः समृ॑द्ध्यै। ता दि॒ग्भ्यः स॒माभृ॑ता भवन्ति। दि॒क्षु वा आप॑। अन्नं॒ वा आप॑। अ॒द्भ्यो वा अन्नं॑ जायते। यदे॒वाद्भ्योऽन्नं॒ जाय॑ते। तदव॑ रुन्धे। तासु॑ ब्रह्मौद॒नं प॑चति। रेत॑ ए॒व तद्द॑धाति॥३॥

%3.8.2.2
चतु॑ शरावो भवति। दि॒क्ष्वे॑व प्रति॑तिष्ठति। उ॒भ॒यतो॑रु॒क्मौ भ॑वतः। उ॒भ॒यत॑ ए॒वास्मि॒न्रुच॑न्दधाति। उद्ध॑रति शृत॒त्वाय॑। स॒र्पिष्वान्भवति मेध्य॒त्वाय॑। च॒त्वार॑ आर्\mbox{}षे॒याः प्राश्ञ॑न्ति। दि॒शामे॒व ज्योति॑षि जुहोति। च॒त्वारि॒ हिर॑ण्यानि ददाति। दि॒शामे॒व ज्योती॒ष्यव॑ रुन्धे॥४॥

%3.8.2.3
यदाज्य॑मु॒च्छिष्य॑ते। तस्मि॑न्रश॒नान्यु॑नत्ति। प्र॒जाप॑ति॒र्वा ओ॑द॒नः। रेत॒ आज्यम्। यदाज्ये॑ रश॒नान्यु॒नत्ति॑। प्र॒जाप॑तिमे॒व रेत॑सा॒ सम॑र्धयति। द॒र्भ॒मयी॑ रश॒ना भ॑वति। ब॒हु वा ए॒ष कु॑च॒रो॑ मे॒ध्यमुप॑गच्छति। यदश्व॑। प॒वित्रं॒ वै द॒र्भाः॥५॥

%3.8.2.4
यद्द॑र्भ॒मयी॑ रश॒ना भव॑ति। पु॒नात्ये॒वैनम्। पू॒तमे॑न॒म्मेध्य॒मा ल॑भते। अश्व॑स्य॒ वा आल॑ब्धस्य महि॒मोद॑क्रामत्। स म॒हर्त्वि॑ज॒ प्रावि॑शत्। तन्म॒हर्त्वि॑जाम्महर्त्वि॒क्त्वम्। यन्म॒हर्त्वि॑जः प्रा॒श्ञन्ति॑। म॒हि॒मान॑मे॒वास्मि॒न्तद्द॑धति। अश्व॑स्य॒ वा आल॑ब्धस्य॒ रेत॒ उद॑क्रामत्। तत्सु॒वर्ण॒ हिर॑ण्यमभवत्। यत्सु॒वर्ण॒ हिर॑ण्य॒न्ददा॑ति। रेत॑ ए॒व तद्द॑धाति। ओ॒द॒ने द॑दाति। रेतो॒ वा ओ॑द॒नः। रेतो॒ हिर॑ण्यम्। रेत॑सै॒वास्मि॒न्रेतो॑ दधाति॥६॥\anuvakamend[द॒धा॒ति॒ रु॒न्धे॒ द॒र्भा अ॑भव॒थ्षट् च॑]

%3.8.3.1
यो वै ब्रह्म॑णे दे॒वेभ्य॑ प्र॒जाप॑त॒येऽप्र॑तिप्रो॒च्याश्वं॒ मेध्यं॑ ब॒ध्नाति॑। आ दे॒वताभ्यो वृश्च्यते। पापी॑यान्भवति। यः प्र॑ति॒प्रोच्य॑। न दे॒वताभ्य॒ आवृ॑श्च्यते। वसी॑यान्भवति। यदाह॑। ब्रह्म॒न्नश्व॒म्मेध्य॑म्भन्त्स्यामि दे॒वेभ्य॑ प्र॒जाप॑तये॒ तेन॑ राध्यास॒मिति॑। ब्रह्म॒ वै ब्र॒ह्मा। ब्रह्म॑ण ए॒व दे॒वेभ्य॑ प्र॒जाप॑तये प्रति॒प्रोच्याश्व॒म्मेध्य॑म्बध्नाति॥७॥

%3.8.3.2
न दे॒वताभ्य॒ आ वृ॑श्च्यते। वसी॑यान्भवति। दे॒वस्य॑ त्वा सवि॒तुः प्र॑स॒व इति॑ रश॒नामाद॑त्ते॒ प्रसूत्यै। अ॒श्विनोर्बा॒हुभ्या॒मित्या॑ह। अ॒श्विनौ॒ हि दे॒वाना॑मध्व॒र्यू आस्ताम्। पू॒ष्णो हस्ताभ्या॒मित्या॑ह॒ यत्यै। व्यृ॑द्धं॒ वा ए॒तद्य॒ज्ञस्य॑। यद॑य॒जुष्के॑ण क्रि॒यते। इ॒माम॑गृभ्णन्रश॒नामृ॒तस्येत्यधि॑ वदति॒ यजु॑ष्कृत्यै। य॒ज्ञस्य॒ समृ॑द्ध्यै॥८॥

%3.8.3.3
तदा॑हुः। द्वाद॑शारत्नी रश॒ना क॑र्त॒व्या ३ त्रयो॑दशार॒त्नी ३ रिति॑। ऋ॒ष॒भो वा ए॒ष ऋ॑तू॒नाम्। यत्सं॑वत्स॒रः। तस्य॑ त्रयोद॒शो मासो॑ वि॒ष्टपम्। ऋ॒ष॒भ ए॒ष य॒ज्ञानाम्। यद॑श्वमे॒धः। यथा॒ वा ऋ॑ष॒भस्य॑ वि॒ष्टपम्। ए॒वमे॒तस्य॑ वि॒ष्टपम्। त्र॒यो॒द॒शम॑र॒त्नि र॑श॒नाया॑मु॒पा द॑धाति॥९॥

%3.8.3.4
यथ॑र्\mbox{}ष॒भस्य॑ वि॒ष्टप सस्क॒रोति॑। ता॒दृगे॒व तत्। पूर्व॒ आयु॑षि वि॒दथे॑षु क॒व्येत्या॑ह। आयु॑रे॒वास्मि॑न्दधाति। तया॑ दे॒वाः सु॒तमा ब॑भूवु॒रित्या॑ह। भूति॑मे॒वोपाव॑र्तते। ऋ॒तस्य॒ सामन्त्स॒रमा॒रप॒न्तीत्या॑ह। स॒त्यं वा ऋ॒तम्। स॒त्येनै॒वैन॑मृ॒तेनार॑भते। अ॒भि॒धा अ॒सीत्या॑ह॥१०॥

%3.8.3.5
तस्मा॑दश्वमेधया॒जी सर्वा॑णि भू॒तान्य॒भि भ॑वति। भुव॑नम॒सीत्या॑ह। भू॒मान॑मे॒वोपै॑ति। य॒न्ताऽसीत्या॑ह। य॒न्तार॑मे॒वैनं॑ करोति। ध॒र्तासीत्या॑ह। ध॒र्तार॑मे॒वैनं॑ करोति। सोऽग्निं वैश्वान॒रमित्या॑ह। अ॒ग्नावे॒वैनं॑ वैश्वान॒रे जु॑होति। सप्र॑थस॒मित्या॑ह ॥। 1१॥

%3.8.3.6
प्र॒जयै॒वैनं॑ प॒शुभि॑ प्रथयति। स्वाहा॑कृत॒ इत्या॑ह। होम॑ ए॒वास्यै॒षः। पृ॒थि॒व्यामित्या॑ह। अ॒स्यामे॒वैनं॒ प्रति॑ष्ठापयति। य॒न्ता राड्य॒न्ताऽसि॒ यम॑नो ध॒र्तासि॑ ध॒रुण॒ इत्या॑ह। रू॒पमे॒वास्यै॒तन्म॑हि॒मान॒व्व्याँच॑ष्टे। कृ॒ष्यै त्वा॒ क्षेमा॑य त्वा र॒य्यै त्वा॒ पोषा॑य॒ त्वेत्या॑ह। आ॒शिष॑मे॒वैतामाशास्ते। स्व॒गा त्वा॑ दे॒वेभ्य॒ इत्या॑ह। दे॒वेभ्य॑ ए॒वैन स्व॒गा क॑रोति। स्वाहा त्वा प्र॒जाप॑तय॒ इत्या॑ह। प्रा॒जा॒प॒त्यो वा अश्व॑। यस्या॑ ए॒व दे॒वता॑या आल॒भ्यते। तयै॒वैन॒ सम॑र्धयति॥१२॥\anuvakamend[ब॒ध्ना॒ति॒ समृ॑द्ध्या उ॒पाद॑धात्य॒सीत्या॑ह॒ सप्र॑थस॒मित्या॑ह दे॒वेभ्य॒ इत्या॑ह॒ पञ्च॑ च]

%3.8.4.1
यः पि॒तुर॑नु॒जाया पु॒त्रः। स पु॒रस्तान्नयति। यो मा॒तुर॑नु॒जाया पु॒त्रः। स प॒श्चान्न॑यति। विष्व॑ञ्चमे॒वास्मात्पा॒प्मानं॒ विवृ॑हतः। यो अर्व॑न्तं॒ जिघासति॒ तम॒भ्य॑मीति॒ वरु॑ण॒ इति॒ श्वानं॑ चतुर॒क्षं प्रसौ॑ति। प॒रो मर्त॑ प॒रः श्वेति॒ शुन॑श्चतुर॒क्षस्य॒ प्रह॑न्ति। श्वेव॒ वै पा॒प्मा भ्रातृ॑व्यः। पा॒प्मान॑मे॒वास्य॒ भ्रातृ॑व्य हन्ति। सै॒ध्र॒कम्मुस॑लम्भवति॥१३॥

%3.8.4.2
कर्म॑कर्मै॒वास्मै॑ साधयति। पौ॒श्च॒ले॒यो ह॑न्ति। पु॒श्च॒ल्वां वै दे॒वाः शुच॒न्न्य॑दधुः। शु॒चैवास्य॒ शुच हन्ति। पा॒प्मा वा ए॒तमीप्स॒तीत्या॑हुः। योऽश्वमे॒धेन॒ यज॑त॒ इति॑। अश्व॑स्याधस्प॒दमुपास्यति। व॒ज्री वा अश्व॑ प्राजाप॒त्यः। वज्रे॑णै॒व पा॒प्मान॒म्भ्रातृ॑व्य॒मव॑ क्रामति। द॒क्षि॒णाऽप॑ प्लावयति॥१४॥

%3.8.4.3
पा॒प्मान॑मे॒वास्मा॒च्छम॑ल॒मप॑ प्लावयति। ऐ॒षी॒क उ॑दू॒हो भ॑वति। आयु॒र्वा इ॒षीका। आयु॑रे॒वास्मि॑न्दधति। अ॒मृतं॒ वा इ॒षीका। अ॒मृत॑मे॒वास्मि॑न्दधति। वे॒त॒स॒शा॒खोप॒सम्ब॑द्धा भवति। अ॒प्सुयो॑नि॒र्वा अश्व॑। अ॒प्सु॒जो वे॑त॒सः। स्वादे॒वैन॒य्योँने॒र्निर्मि॑मीते। पु॒रस्तात्प्र॒त्यञ्च॑म॒भ्युदू॑हति। पु॒रस्ता॑दे॒वास्मि॑न्प्र॒तीच्य॒मृत॑न्दधाति। अ॒हं च॒ त्वं च॑ वृत्रह॒न्निति॑ ब्र॒ह्मा यज॑मानस्य॒ हस्त॑ङ्गृह्णाति। ब्र॒ह्म॒क्ष॒त्रे ए॒व सन्द॑धाति। अ॒भिक्रत्वेन्द्र भू॒रध॒ज्मन्नित्य॑ध्व॒र्युर्यज॑मानं वाचयत्य॒भिजि॑त्यै॥१५॥\anuvakamend[भ॒व॒ति॒ प्ला॒व॒य॒ति॒ मि॒मी॒ते॒ पञ्च॑ च]

%3.8.5.1
च॒त्वार॑ ऋ॒त्विज॒ समु॑क्षन्ति। आ॒भ्य ए॒वैनं॑ चत॒सृभ्यो॑ दि॒ग्भ्यो॑ऽभि समी॑रयन्ति। श॒तेन॑ राजपु॒त्रैः स॒हाध्व॒र्युः। पु॒रस्तात्प्र॒त्यङ्तिष्ठ॒न्प्रोक्ष॑ति। अ॒नेनाश्वे॑न॒ मेध्ये॑ने॒ष्ट्वा। अ॒य राजा॑ वृ॒त्रं व॑ध्या॒दिति॑। रा॒ज्यं वा अ॑ध्व॒र्युः। क्ष॒त्र रा॑जपु॒त्रः। रा॒ज्येनै॒वास्मि॑न्क्ष॒त्रन्द॑धाति। श॒तेना॑ रा॒जभि॑रु॒ग्रैः स॒ह ब्र॒ह्मा॥१६॥

%3.8.5.2
द॒क्षि॒ण॒त उद॒ङ्तिष्ठ॒न्प्रोक्ष॑ति। अ॒नेनाश्वे॑न॒ मेध्ये॑ने॒ष्ट्वा। अ॒य राजाप्रतिधृ॒ष्योऽस्त्विति॑। बलं॒ वै ब्र॒ह्मा। बल॑मरा॒जोग्रः। बले॑नै॒वास्मि॒न्बल॑न्दधाति। श॒तेन॑ सूतग्राम॒णिभि॑ स॒ह होता। प॒श्चात्प्राङ्तिष्ठ॒न्प्रोक्ष॑ति। अ॒नेनाश्वे॑न॒ मेध्ये॑ने॒ष्ट्वा। अ॒य राजा॒ऽस्यै वि॒शः॥१७॥

%3.8.5.3
ब॒हु॒ग्वै ब॑ह्व॒श्वायै॑ बह्वजावि॒कायै। ब॒हु॒व्री॒हि॒य॒वायै॑ बहुमाषति॒लायै। ब॒हु॒हि॒र॒ण्यायै॑ बहुह॒स्तिका॑यै। ब॒हु॒दा॒स॒पू॒रु॒षायै॑ रयि॒मत्यै॒ पुष्टि॑मत्यै। ब॒हु॒रा॒य॒स्पो॒षायै॒ राजा॒स्त्विति॑। भू॒मा वै होता। भू॒मा सू॑तग्राम॒ण्य॑। भू॒म्नैवास्मि॑न्भू॒मान॑न्दधाति। श॒तेन॑ क्षत्तसङ्ग्रही॒तृभि॑ स॒होद्गा॒ता। उ॒त्त॒र॒तो द॑क्षि॒णा तिष्ठ॒न्प्रोक्ष॑ति॥१८॥

%3.8.5.4
अ॒नेनाश्वे॑न॒ मेध्ये॑ने॒ष्ट्वा। अ॒य राजा॒ सर्व॒मायु॑रे॒त्विति॑। आयु॒र्वा उ॑द्गा॒ता। आयु॑ क्षत्तसङ्ग्रही॒तार॑। आयु॑षै॒वास्मि॒न्नायु॑र्दधाति। श॒तश॑तम्भवन्ति। श॒तायु॒ पुरु॑षः श॒तेन्द्रि॑यः। आयु॑ष्ये॒वेन्द्रि॒ये प्रति॑तिष्ठति। च॒तु॒ श॒ता भ॑वन्ति। चत॑स्रो॒ दिश॑। दि॒क्ष्वे॑व प्रति॑ तिष्ठति॥१९॥\anuvakamend[ब्र॒ह्मा वि॒श उ॑क्षति॒ दिश॒ एकं च]

%3.8.6.1
यथा॒ वै ह॒विषो॑ गृही॒तस्य॒ स्कन्द॑ति। ए॒वं वा ए॒तदश्व॑स्य स्कन्दति। यन्नि॒क्तमना॑लब्धमुत्सृ॒जन्ति॑। यत्स्तोक्या॑ अ॒न्वाह॑। स॒र्व॒हुत॑मे॒वैनं॑ करो॒त्यस्क॑न्दाय। अस्क॑न्न॒ हि तत्। यद्धु॒तस्य॒ स्कन्द॑ति। स॒हस्र॒मन्वा॑ह। स॒हस्र॑सम्मितः सुव॒र्गो लो॒कः। सु॒व॒र्गस्य॑ लो॒कस्या॒भिजि॑त्यै॥२०॥

%3.8.6.2
यत्परि॑मिता अनुब्रू॒यात्। परि॑मित॒मव॑ रुन्धीत। अप॑रिमिता॒ अन्वा॑ह। अप॑रिमितः सुव॒र्गो लो॒कः। सु॒व॒र्गस्य॑ लो॒कस्य॒ सम॑ष्ट्यै। स्तोक्या॑ जुहोति। या ए॒व वर्ष्या॒ आप॑। ता अव॑ रुन्धे। अ॒स्यां जु॑होति। इ॒यं वा अ॒ग्निर्वैश्वान॒रः॥२१॥

%3.8.6.3
अ॒स्यामे॒वैना॒ प्रति॑ष्ठापयति। उ॒वाच॑ ह प्र॒जाप॑तिः। स्तोक्या॑सु॒ वा अ॒हम॑श्वमे॒ध सस्था॑पयामि। तेन॒ तत॒ सस्थि॑तेन चरा॒मीति॑। अ॒ग्नये॒ स्वाहेत्या॑ह। अ॒ग्नय॑ ए॒वैनं॑ जुहोति। सोमा॑य॒ स्वाहेत्या॑ह। सोमा॑यै॒वैनं॑ जुहोति। स॒वि॒त्रे स्वाहेत्या॑ह। स॒वि॒त्र ए॒वैनं॑ जुहोति॥२२॥

%3.8.6.4
सर॑स्वत्यै॒ स्वाहेत्या॑ह। सर॑स्वत्या ए॒वैनं॑ जुहोति। पू॒ष्णे स्वाहेत्या॑ह। पू॒ष्ण ए॒वैनं॑ जुहोति। बृह॒स्पत॑ये॒ स्वाहेत्या॑ह। बृह॒स्पत॑य ए॒वैनं॑ जुहोति। अ॒पाम्मोदा॑य॒ स्वाहेत्या॑ह। अ॒द्भ्य ए॒वैनं॑ जुहोति। वा॒यवे॒ स्वाहेत्या॑ह। वा॒यव॑ ए॒वैनं॑ जुहोति॥२३॥

%3.8.6.5
मि॒त्राय॒ स्वाहेत्या॑ह। मि॒त्रायै॒वैनं॑ जुहोति। वरु॑णाय॒ स्वाहेत्या॑ह। वरु॑णायै॒वैनं॑ जुहोति। ए॒ताभ्य॑ ए॒वैनं॑ दे॒वताभ्यो जुहोति। दश॑दश सं॒पादं॑ जुहोति। दशाक्षरा वि॒राट्। अन्नं॑ वि॒राट्। वि॒राजै॒वान्नाद्य॒मव॑ रुन्धे। प्र वा ए॒षोऽस्माल्लो॒काच्च्य॑वते। यः परा॑ची॒राहु॑तीर्जु॒होति॑। पुन॑ पुनरभ्या॒वर्तं॑ जुहोति। अ॒स्मिन्ने॒व लो॒के प्रति॑तिष्ठति। ए॒ता ह वाव सोऽश्वमे॒धस्य॒ सस्थि॑तिमुवा॒चास्क॑न्दाय। अस्क॑न्न॒ हि तत्। यद्य॒ज्ञस्य॒ सस्थि॑तस्य॒ स्कन्द॑ति॥२४॥\anuvakamend[अ॒भिजि॑त्यै वैश्वान॒रः स॑वि॒त्र ए॒वैनं॑ जुहोति वा॒यव॑ ए॒वैनं॑ जुहोति च्यवते॒ षट् च॑]

%3.8.7.1
प्र॒जाप॑तये त्वा॒ जुष्टं॒ प्रोक्षा॒मीति॑ पु॒रस्तात्प्र॒त्यङ्तिष्ठ॒न्प्रोक्ष॑ति। प्र॒जाप॑ति॒र्वै दे॒वाना॑मन्ना॒दो वी॒र्या॑वान्। अ॒न्नाद्य॑मे॒वास्मि॑न्वी॒र्यं॑ दधाति। तस्मा॒दश्व॑ पशू॒नाम॑न्ना॒दो वी॒र्या॑वत्तमः। इ॒न्द्रा॒ग्निभ्या॒न्त्वेति॑ दक्षिण॒तः। इ॒न्द्रा॒ग्नी वै दे॒वाना॒मोजि॑ष्ठौ॒ बलि॑ष्ठौ। ओज॑ ए॒वास्मि॒न्बल॑न्दधाति। तस्मा॒दश्व॑ पशू॒नामोजि॑ष्ठो॒ बलि॑ष्ठः। वा॒यवे॒ त्वेति॑ प॒श्चात्। वा॒युर्वै दे॒वाना॑मा॒शुः सा॑रसा॒रित॑मः॥२५॥

%3.8.7.2
ज॒वमे॒वास्मि॑न्दधाति। तस्मा॒दश्व॑ पशू॒नामा॒शुः सा॑रसा॒रित॑मः। विश्वेभ्यस्त्वा दे॒वेभ्य॒ इत्यु॑त्तर॒तः। विश्वे॒ वै दे॒वा दे॒वानां यश॒स्वित॑माः। यश॑ ए॒वास्मि॑न्दधाति। तस्मा॒दश्व॑ पशू॒नां य॑श॒स्वित॑मः। दे॒वेभ्य॒स्त्वेत्य॒धस्तात्। दे॒वा वै दे॒वाना॒मप॑चिततमाः। अप॑चितिमे॒वास्मि॑न्दधाति। तस्मा॒दश्व॑ पशू॒नामप॑चिततमः॥२६॥

%3.8.7.3
सर्वेभ्यस्त्वा दे॒वेभ्य॒ इत्यु॒परि॑ष्टात्। सर्वे॒ वै दे॒वास्त्विषि॑मन्तो हर॒स्विन॑। त्विषि॑मे॒वास्मि॒न्॒ हरो॑ दधाति। तस्मा॒दश्व॑ पशू॒नान्त्विषि॑मान्‌हर॒स्वित॑मः। दि॒वे त्वा॒ऽन्तरि॑क्षाय त्वा पृथि॒व्यै त्वेत्या॑ह। ए॒भ्य ए॒वैनं॑ लो॒केभ्य॒ प्रोक्ष॑ति। स॒ते त्वाऽस॑ते त्वा॒ऽद्भ्यस्त्वौष॑धीभ्यस्त्वा॒ विश्वेभ्यस्त्वा भू॒तेभ्य॒ इत्या॑ह। तस्मा॑दश्वमेधया॒जिन॒ सर्वा॑णि भू॒तान्युप॑जीवन्ति। ब्र॒ह्म॒वा॒दिनो॑ वदन्ति। यत्प्रा॑जाप॒त्योऽश्व॑। अथ॒ कस्मा॑देनम॒न्याभ्यो॑ दे॒वता॒भ्योऽपि॒ प्रोक्ष॒तीति॑। अश्वे॒ वै सर्वा॑ दे॒वता॑ अ॒न्वाय॑त्ताः। तं यद्विश्वेभ्यस्त्वा भू॒तेभ्य॒ इति॑ प्रो॒क्षति॑। दे॒वता॑ ए॒वास्मि॑न्न॒न्वा या॑तयति। तस्मा॒दश्वे॒ सर्वा॑ दे॒वता॑ अ॒न्वाय॑त्ताः॥२७॥\anuvakamend[सा॒र॒सा॒रित॒मोऽप॑चिततमः प्राजाप॒त्योऽश्व॒ पञ्च॑ च]

%3.8.8.1
यथा॒ वै ह॒विषो॑ गृही॒तस्य॒ स्कन्द॑ति। ए॒वं वा ए॒तदश्व॑स्य स्कन्दति। यत्प्रोक्षि॑त॒मना॑लब्धमुत्सृ॒जन्ति॑। यद॑श्वचरि॒तानि॑ जु॒होति॑। स॒र्व॒हुत॑मे॒वैनं॑ करो॒त्यस्क॑न्दाय। अस्क॑न्न॒ हि तत्। यद्धु॒तस्य॒ स्कन्द॑ति। ई॒ङ्का॒राय॒ स्वाहें कृ॑ताय॒ स्वाहेत्या॑ह। ए॒तानि॒ वा अ॑श्वचरि॒तानि॑। च॒रि॒तैरे॒वैन॒ सम॑र्धयति॥२८॥

%3.8.8.2
तदा॑हुः। अना॑हुतयो॒ वा अ॑श्वचरि॒तानि॑। नैता हो॑त॒व्या॑ इति॑। अथो॒ खल्वा॑हुः। हो॒त॒व्या॑ ए॒व। अत्र॒ वावैवं वि॒द्वान॑श्वमे॒ध सस्था॑पयति। यद॑श्वचरि॒तानि॑ जु॒होति॑। तस्माद्धोत॒व्या॑ इति॑। ब॒हि॒र्धा वा ए॑नमे॒तदा॒यत॑नाद्दधाति। भ्रातृ॑व्यमस्मै जनयति॥२९॥

%3.8.8.3
यस्या॑नायत॒नेऽन्यत्रा॒ग्नेराहु॑तीर्जु॒होति॑। सा॒वि॒त्रि॒या इष्ट्या पु॒रस्तात्स्विष्ट॒कृत॑। आ॒ह॒व॒नीयेऽश्वचरि॒तानि॑ जुहोति। आ॒यत॑न ए॒वास्याहु॑तीर्जुहोति। नास्मै॒ भ्रातृ॑व्यञ्जनयति। तदा॑हुः। य॒ज्ञ॒मु॒खेय॑ज्ञमुखे होत॒व्या। य॒ज्ञस्य॒ कॢप्त्यै। सु॒व॒र्गस्य॑ लो॒कस्यानु॑ख्यात्या॒ इति॑। अथो॒ खल्वा॑हुः॥३०॥

%3.8.8.4
यद्य॑ज्ञमु॒खेय॑ज्ञमुखे जुहु॒यात्। प॒शुभि॒र्यज॑मान॒व्व्यँ॑र्धयेत्। अव॑ सुव॒र्गाल्लो॒कात्प॑द्येत। पापी॑यान्त्स्या॒दिति॑। स॒कृदे॒व हो॑त॒व्या। न यज॑मानं प॒शुभि॒र्व्य॑र्धयति। अ॒भि सु॑व॒र्गं लो॒कं ज॑यति। न पापी॑यान्भवति। अ॒ष्टाच॑त्वारिशतमश्वरू॒पाणि॑ जुहोति। अ॒ष्टाच॑त्वारिशदक्षरा॒ जग॑ती। जाग॒तोऽश्व॑ प्राजाप॒त्यः समृ॑द्ध्यै। ए॒कमति॑रिक्तं जुहोति। तस्मा॒देक॑ प्र॒जास्वर्धु॑कः॥३१॥\anuvakamend[अ॒र्ध॒य॒ति॒ ज॒न॒य॒ति॒ खल्वा॑हु॒र्जग॑ती॒ त्रीणि॑ च]

%3.8.9.1
वि॒भूर्मा॒त्रा प्र॒भूः पि॒त्रेत्या॑ह। इ॒यं वै मा॒ता। अ॒सौ पि॒ता। आ॒भ्यामे॒वैनं॒ परि॑ददाति। अश्वो॑ऽसि॒ हयो॒ऽसीत्या॑ह। शास्त्ये॒वैन॑मे॒तत्। तस्माच्छि॒ष्टाः प्र॒जा जा॑यन्ते। अत्यो॒ऽसीत्या॑ह। तस्मा॒दश्व॒ सर्वान्प॒शूनत्ये॑ति। तस्मा॒दश्व॒ सर्वे॑षां पशू॒ना श्रैष्ठ्यं॑ गच्छति॥३२॥

%3.8.9.2
प्र यश॒ श्रैष्ठ्य॑माप्नोति। य ए॒वं वेद॑। नरो॒ऽस्यर्वा॑ऽसि॒ सप्ति॑रसि वा॒ज्य॑सीत्या॑ह। रू॒पमे॒वास्यै॒तन्म॑हि॒मान॒व्व्याँच॑ष्टे। ययु॒र्नामा॒ऽसीत्या॑ह। ए॒तद्वा अश्व॑स्य प्रि॒यन्ना॑म॒धेयम्। प्रि॒येणै॒वैन॑न्नाम॒धेये॑ना॒भि व॑दति। तस्मा॒दप्या॑मि॒त्रौ स॒ङ्गत्य॑। नाम्ना॒ चेद्ध्वये॑ते। मि॒त्रमे॒व भ॑वतः॥३३॥

%3.8.9.3
आ॒दि॒त्यानां॒ पत्वाऽन्वि॒हीत्या॑ह। आ॒दि॒त्याने॒वैन॑ङ्गमयति। अ॒ग्नये॒ स्वाहा॒ स्वाहेन्द्रा॒ग्निभ्या॒मिति॑ पूर्वहो॒मां जु॑होति। पूर्व॑ ए॒व द्वि॒षन्त॒म्भ्रातृ॑व्य॒मति॑ क्रामति। भूर॑सि भु॒वे त्वा॒ भव्या॑य त्वा भविष्य॒ते त्वेत्युत्सृ॑जति सर्व॒त्वाय॑। देवा॑ आशापाला ए॒तन्दे॒वेभ्योऽश्व॒म्मेधा॑य॒ प्रोक्षि॑तङ्गोपाय॒तेत्या॑ह। श॒तं वै तल्प्या॑ राजपु॒त्रा दे॒वा आ॑शापा॒लाः। तेभ्य॑ ए॒वैनं॒ परि॑ ददाति। ई॒श्व॒रो वा अश्व॒ प्रमु॑क्त॒ परां परा॒वत॒ङ्गन्तो। इ॒ह धृति॒ स्वाहे॒ह विधृ॑ति॒ स्वाहे॒ह रन्ति॒ स्वाहे॒ह रम॑ति॒ स्वाहेति॑ चतृ॒षु प॒त्सु जु॑होति॥३४॥

%3.8.9.4
ए॒ता वा अश्व॑स्य॒ बन्ध॑नम्। ताभि॑रे॒वैन॑म्बध्नाति। तस्मा॒दश्व॒ प्रमु॑क्तो॒ बन्ध॑न॒मा ग॑च्छति। तस्मा॒दश्व॒ प्रमु॑क्तो॒ बन्ध॑न॒न्न ज॑हाति। रा॒ष्ट्रं वा अ॑श्वमे॒धः। रा॒ष्ट्रे खलु॒ वा ए॒ते व्याय॑च्छन्ते। येऽश्व॒म्मेध्य॒ रक्ष॑न्ति। तेषां॒ य उ॒दृचं॒ गच्छ॑न्ति। रा॒ष्ट्रादे॒व ते रा॒ष्ट्रङ्ग॑च्छन्ति। अथ॒ य उ॒दृच॒न्न गच्छ॑न्ति॥३५॥

%3.8.9.5
रा॒ष्ट्रादे॒व ते व्यव॑च्छिद्यन्ते। परा॒ वा ए॒ष सि॑च्यते। यो॑ऽब॒लोऽश्वमे॒धेन॒ यज॑ते। यद॒मित्रा॒ अश्वं॑ वि॒न्देर\sn{}। ह॒न्येतास्य य॒ज्ञः। च॒तु॒ श॒ता र॑क्षन्ति। य॒ज्ञस्याघा॑ताय। अथा॒न्यमा॒नीय॒ प्रोक्षे॑युः। सैव तत॒ प्राय॑श्चित्तिः॥३६॥\anuvakamend[ग॒च्छ॒ति॒ भ॒व॒त॒ प॒त्सु जु॑होति॒ न गच्छ॑न्ति॒ नव॑ च]

%3.8.10.1
प्र॒जाप॑तिरकामयताश्वमे॒धेन॑ यजे॒येति॑। स तपो॑ऽतप्यत। तस्य॑ तेपा॒नस्य॑। स॒प्तात्मनो॑ दे॒वता॒ उद॑क्रामन्। सा दी॒क्षाऽभ॑वत्। स ए॒तानि॑ वैश्वदे॒वान्य॑पश्यत्। तान्य॑जुहोत्। तैर्वै स दी॒क्षामवा॑रुन्ध। यद्वैश्वदे॒वानि॑ जु॒होति॑। दी॒क्षामे॒व तैर्यज॑मा॒नोऽव॑ रुन्धे॥३७॥

%3.8.10.2
स॒प्त जु॑होति। स॒प्त हि ता दे॒वता॑ उ॒दक्रा॑मन्। अ॒न्व॒हं जु॑होति। अ॒न्व॒हमे॒व दी॒क्षामव॑ रुन्धे। त्रीणि॑ वैश्वदे॒वानि॑ जुहोति। च॒त्वार्यौद्ग्रह॒णानि॑। स॒प्त संप॑द्यन्ते। स॒प्त वै शी॑र्‌ष॒ण्या प्रा॒णाः। प्रा॒णा दी॒क्षा। प्रा॒णैरे॒व प्रा॒णान्दी॒क्षामव॑ रुन्धे॥३८॥

%3.8.10.3
एक॑विशतिं वैश्वदे॒वानि॑ जुहोति। एक॑विशति॒र्वै दे॑वलो॒काः। द्वाद॑श॒ मासा॒ पञ्च॒र्तव॑। त्रय॑ इ॒मे लो॒काः। अ॒सावा॑दि॒त्य ए॑कवि॒शः। ए॒ष सु॑व॒र्गो लो॒कः। तद्दैव्यं॑ क्ष॒त्रम्। सा श्रीः। तद्ब्र॒ध्नस्य॑ वि॒ष्टपम्। तत्स्वाराज्यमुच्यते॥३९॥

%3.8.10.4
त्रि॒शत॑मौद्ग्रह॒णानि॑ जुहोति। त्रि॒शद॑क्षरा वि॒राट्। अन्नं॑ वि॒राट्। वि॒राजै॒वान्नाद्य॒मव॑ रुन्धे। त्रे॒धा वि॒भज्य॑ दे॒वतां जुहोति। त्र्या॑वृतो॒ वै दे॒वाः। त्र्या॑वृत इ॒मे लो॒काः। ए॒षां लो॒काना॒माप्त्यै। ए॒षां लो॒कानां॒ कॢप्त्यै। अप॒ वा ए॒तस्मात्प्रा॒णाः क्रा॑मन्ति॥४०॥

%3.8.10.5
यो दी॒क्षाम॑तिरे॒चय॑ति। स॒प्ता॒हं प्रच॑रन्ति। स॒प्त वै शी॑र्\mbox{}ष॒ण्या प्रा॒णाः। प्रा॒णा दी॒क्षा। प्रा॒णैरे॒व प्रा॒णान्दी॒क्षामव॑ रुन्धे। पू॒र्णा॒हु॒तिमु॑त्त॒मां जु॑होति। सर्वं॒ वै पूर्णाहु॒तिः। सर्व॑मे॒वाप्नो॑ति। अथो॑ इ॒यं वै पूर्णाहु॒तिः। अ॒स्यामे॒व प्रति॑ तिष्ठति॥४१॥\anuvakamend[रु॒न्धे॒ प्रा॒णान्दी॒क्षामव॑ रुन्ध उच्यते क्रामन्ति तिष्ठति]

%3.8.11.1
प्र॒जाप॑तिरश्वमे॒धम॑सृजत। त सृ॒ष्टं न किञ्च॒नोद॑यच्छत्। तं वैश्वदे॒वान्ये॒वोद॑यच्छन्। यद्वैश्वदे॒वानि॑ जु॒होति॑। य॒ज्ञस्योद्य॑त्यै। स्वाहा॒ऽऽधिमाधी॑ताय॒ स्वाहा। स्वाहाऽधी॑तं॒ मन॑से॒ स्वाहा। स्वाहा॒ मन॑ प्र॒जाप॑तये॒ स्वाहा। काय॒ स्वाहा॒ कस्मै॒ स्वाहा॑ कत॒मस्मै॒ स्वाहेति॑ प्राजाप॒त्ये मुख्ये॑ भवतः। प्र॒जाप॑तिमुखाभिरे॒वैनं॑ दे॒वता॑भि॒रुद्य॑च्छते॥४२॥

%3.8.11.2
अदि॑त्यै॒ स्वाहाऽदि॑त्यै म॒ह्यै स्वाहाऽदि॑त्यै सुमृडी॒कायै॒ स्वाहेत्या॑ह। इ॒यं वा अदि॑तिः। अ॒स्या ए॒वैनं॑ प्रति॒ष्ठायोद्य॑च्छते। सर॑स्वत्यै॒ स्वाहा॒ सर॑स्वत्यै बृह॒त्यै स्वाहा॒ सर॑स्वत्यै पाव॒कायै॒ स्वाहेत्या॑ह। वाग्वै सर॑स्वती। वा॒चैवैन॒मुद्य॑च्छते। पू॒ष्णे स्वाहा॑ पू॒ष्णे प्र॑प॒थ्या॑य॒ स्वाहा॑ पू॒ष्णे न॒रन्धि॑षाय॒ स्वाहेत्या॑ह। प॒शवो॒ वै पू॒षा। प॒शुभि॑रे॒वैन॒मुद्य॑च्छते। त्वष्ट्रे॒ स्वाहा॒ त्वष्ट्रे॑ तु॒रीपा॑य॒ स्वाहा॒ त्वष्ट्रे॑ पुरु॒रूपा॑य॒ स्वाहेत्या॑ह। त्वष्टा॒ वै प॑शू॒नां मि॑थु॒नाना रूप॒कृत्। रू॒पमे॒व प॒शुषु॑ दधाति। अथो॑ रू॒पैरे॒वैन॒मुद्य॑च्छते। विष्ण॑वे॒ स्वाहा॒ विष्ण॑वे निखुर्य॒पाय॒ स्वाहा॒ विष्ण॑वे निभूय॒पाय॒ स्वाहेत्या॑ह। य॒ज्ञो वै विष्णु॑। य॒ज्ञायै॒वैन॒मुद्य॑च्छते। पू॒र्णा॒हु॒तिमु॑त्त॒मां जु॑होति। प्रत्युत्त॑ब्ध्यै सय॒त्वाय॑॥४३॥\anuvakamend[य॒च्छ॒ते॒ पु॒रु॒रूपा॑य॒ स्वाहेत्या॑हा॒ष्टौ च॑]

%3.8.12.1
सा॒वि॒त्रम॒ष्टाक॑पालं प्रा॒तर्निर्व॑पति। अ॒ष्टाक्ष॑रा गाय॒त्री। गा॒य॒त्रं प्रा॑तः सव॒नम्। प्रा॒त॒ स॒व॒नादे॒वैनं॑ गायत्रि॒याश्छन्द॒सोऽधि॒ निर्मि॑मीते। अथो प्रातः सव॒नमे॒व तेनाप्नोति। गा॒य॒त्रीं छन्द॑। स॒वि॒त्रे प्र॑सवि॒त्र एका॑दशकपालं म॒ध्यन्दि॑ने। एका॑दशाक्षरा त्रि॒ष्टुप्। त्रैष्टु॑भं॒ माध्य॑न्दिन॒ सव॑नम्। माध्य॑न्दिनादे॒वैन॒ सव॑नात्रि॒ष्टुभ॒श्छन्द॒सोऽधि॒ निर्मि॑मीते॥४४॥

%3.8.12.2
अथो॒ माध्य॑न्दिनमे॒व सव॑नं॒ तेनाप्नोति। त्रि॒ष्टुभं॒ छन्द॑। स॒वि॒त्र आ॑सवि॒त्रे द्वाद॑शकपालमपरा॒ह्णे। द्वाद॑शाक्षरा॒ जग॑ती। जाग॑तं तृतीयसव॒नम्। तृ॒ती॒य॒स॒व॒नादे॒वैनं॒ जग॑त्या॒श्छन्द॒सोऽधि॒ निर्मि॑मीते। अथो॑ तृतीयसव॒नमे॒व तेनाप्नोति। जग॑तीं॒ छन्द॑। ई॒श्व॒रो वा अश्व॒ प्रमु॑क्त॒ परां परा॒वतं॒ गन्तो। इ॒ह धृति॒ स्वाहे॒ह विधृ॑ति॒ स्वाहे॒ह रन्ति॒ स्वाहे॒ह रम॑ति॒ स्वाहेति॒ चत॑स्र॒ आहु॑तीर्जुहोति॥४५॥

%3.8.12.3
चत॑स्रो॒ दिश॑। दि॒ग्भिरे॒वैनं॒ परि॑गृह्णाति। आश्व॑त्थो व्र॒जो भ॑वति। प्र॒जाप॑तिर्दे॒वेभ्यो॒ निला॑यत। अश्वो॑ रू॒पं कृ॒त्वा। सोऽश्व॒त्थे सं॑वत्स॒रम॑तिष्ठत्। तद॑श्व॒त्थस्याश्वत्थ॒त्वम्। यदाश्व॑त्थो व्र॒जो भव॑ति। स्व ए॒वैनं॒ योनौ॒ प्रति॑ष्ठापयति॥४६॥\anuvakamend[त्रि॒ष्टुभ॒श्छन्द॒सोऽधि॒ निर्मि॑मीते जुहोति॒ नव॑ च]

%3.8.13.1
आ ब्रह्म॑न्ब्राह्म॒णो ब्र॑ह्मवर्च॒सी जा॑यता॒मित्या॑ह। ब्रा॒ह्म॒ण ए॒व ब्र॑ह्मवर्च॒सं द॑धाति। तस्मात्पु॒रा ब्राह्म॒णो ब्र॑ह्मवर्च॒स्य॑जायत। आऽस्मिन्रा॒ष्ट्रे रा॑ज॒न्य॑ इष॒व्य॑ शूरो॑ महार॒थो जा॑यता॒मित्या॑ह। रा॒ज॒न्य॑ ए॒व शौ॒र्यं म॑हि॒मानं॑ दधाति। तस्मात्पु॒रा रा॑ज॒न्य॑ इष॒व्य॑ शूरो॑ महार॒थो॑ऽजायत। दोग्ध्री॑ धे॒नुरित्या॑ह। धे॒न्वामे॒व पयो॑ दधाति। तस्मात्पु॒रा दोग्ध्री॑ धे॒नुर॑जायत। वोढा॑ऽन॒ड्वानित्या॑ह॥४७॥

%3.8.13.2
अ॒न॒डुह्ये॒व वी॒र्यं॑ दधाति। तस्मात्पु॒रा वोढा॑ऽन॒ड्वान॑जायत। आ॒शुः सप्ति॒रित्या॑ह। अश्व॑ ए॒व ज॒वं द॑धाति। तस्मात्पु॒राऽऽशुरश्वो॑ऽजायत। पुर॑न्धि॒र्योषेत्या॑ह। यो॒षित्ये॒व रू॒पं द॑धाति। तस्मा॒त्स्त्री यु॑व॒तिः प्रि॒या भावु॑का। जि॒ष्णू र॑थे॒ष्ठा इत्या॑ह। आ ह॒ वै तत्र॑ जि॒ष्णू र॑थे॒ष्ठा जा॑यते॥४८॥

%3.8.13.3
यत्रै॒तेन॑ य॒ज्ञेन॒ यज॑न्ते। स॒भेयो॒ युवेत्या॑ह। यो वै पूर्ववय॒सी। स स॒भेयो॒ युवा। तस्मा॒द्युवा॒ पुमान्प्रि॒यो भावु॑कः। आऽस्य यज॑मानस्य वी॒रो जा॑यता॒मित्या॑ह। आ ह॒ वै तत्र॒ यज॑मानस्य वी॒रो जा॑यते। यत्रै॒तेन॑ य॒ज्ञेन॒ यज॑न्ते। नि॒का॒मेनि॑कामे नः प॒र्जन्यो॑ वर्\mbox{}ष॒त्वित्या॑ह। नि॒का॒मेनि॑कामे ह॒ वै तत्र॑ प॒र्जन्यो॑ वर्\mbox{}षति। यत्रै॒तेन॑ य॒ज्ञेन॒ यज॑न्ते। फ॒लिन्यो॑ न॒ ओष॑धयः पच्यन्ता॒मित्या॑ह। फ॒लिन्यो॑ ह॒ वै तत्रौष॑धयः पच्यन्ते। यत्रै॒तेन॑ य॒ज्ञेन॒ यज॑न्ते। यो॒ग॒क्षे॒मो न॑ कल्पता॒मित्या॑ह। कल्प॑ते ह॒ वै तत्र॑ प्र॒जाभ्यो॑ योगक्षे॒मः। यत्रै॒तेन॑ य॒ज्ञेन॒ यज॑न्ते ॥४९॥\anuvakamend[अ॒न॒ड्वानित्या॑ह जायते वर्‌षति स॒प्त च॑]

%3.8.14.1
प्र॒जाप॑तिर्दे॒वेभ्यो॑ य॒ज्ञान्व्यादि॑शत्। स आ॒त्मन्न॑श्वमे॒धम॑धत्त। तं दे॒वा अ॑ब्रुवन्। ए॒ष वाव य॒ज्ञः। यद॑श्वमे॒धः। अप्ये॒व नोत्रा॒स्त्विति॑। तेभ्य॑ ए॒तान॑न्नहो॒मान्प्राय॑च्छत्। तान॑जुहोत्। तैर्वै स दे॒वान॑प्रीणात्। यद॑न्नहो॒मां जु॒होति॑॥५०॥

%3.8.14.2
दे॒वाने॒व तैर्यज॑मानः प्रीणाति। आज्ये॑न जुहोति। अ॒ग्नेर्वा ए॒तद्रू॒पम्। यदाज्यम्। यदाज्ये॑न जु॒होति॑। अ॒ग्निमे॒व तत्प्री॑णाति। मधु॑ना जुहोति। म॒ह॒त्यै वा ए॒तद्दे॒वता॑यै रू॒पम्। यन्मधु॑। यन्मधु॑ना जु॒होति॑॥५१॥

%3.8.14.3
म॒ह॒तीमे॒व तद्दे॒वतां प्रीणाति। त॒ण्डु॒लैर्जु॑होति। वसू॑नां॒ वा ए॒तद्रू॒पम्। यत्त॑ण्डु॒लाः। यत्त॑ण्डु॒लैर्जु॒होति॑। वसू॑ने॒व तत्प्री॑णाति। पृथु॑कैर्जुहोति। रु॒द्राणां॒ वा ए॒तद्रू॒पम्। यत्पृथु॑काः। यत्पृथु॑कैर्जु॒होति॑।॥५२॥

%3.8.14.4
रु॒द्राने॒व तत्प्री॑णाति। ला॒जैर्जु॑होति। आ॒दि॒त्यानां॒ वा ए॒तद्रू॒पम्। यल्ला॒जाः। यल्ला॒जैर्जु॒होति॑। आ॒दि॒त्याने॒व तत्प्री॑णाति। क॒रम्बैर्जुहोति। विश्वे॑षां॒ वा ए॒तद्दे॒वाना रू॒पम्। यत्क॒रम्बा। यत्क॒रम्बैर्जु॒होति॑॥५३॥

%3.8.14.5
विश्वा॑ने॒व तद्दे॒वान्प्री॑णाति। धा॒नाभि॑र्जुहोति। नक्ष॑त्राणां॒ वा ए॒तद्रू॒पम्। यद्धा॒नाः। यद्धा॒नाभि॑र्जु॒होति॑। नक्ष॑त्राण्ये॒व तत्प्री॑णाति। सक्तु॑भिर्जुहोति। प्र॒जाप॑ते॒र्वा ए॒तद्रू॒पम्। यत्सक्त॑वः। यत्सक्तु॑भिर्जु॒होति॑॥५४॥

%3.8.14.6
प्र॒जा॑पतिमे॒व तत्प्री॑णाति। म॒सूस्यैर्जुहोति। सर्वा॑सां॒ वा ए॒तद्दे॒वता॑ना रू॒पम्। यन्म॒सूस्या॑नि। यन्म॒सूस्यैर्जु॒होति॑। सर्वा॑ ए॒व तद्दे॒वता प्रीणाति। प्रि॒य॒ङ्गु॒त॒ण्डु॒लैर्जु॑होति। प्रि॒याङ्गा॑ ह॒ वै नामै॒ते। ए॒तैर्वै दे॒वा अश्व॒स्याङ्गा॑नि॒ सम॑दधुः। यत्प्रि॑यङ्गुतण्डु॒लैर्जु॒होति॑। अश्व॑स्यै॒वाङ्गा॑नि॒ संद॑धाति। दशान्ना॑नि जुहोति। दशाक्षरा वि॒राट्। वि॒राट्कृ॒त्स्नस्या॒न्नाद्य॒स्याव॑रुध्यै॥५५॥\anuvakamend[जु॒होति॒ मधु॑ना जु॒होति॒ पृथु॑कैर्जु॒होति॑ क॒रम्बैर्जु॒होति॒ सक्तु॑भिर्जु॒होति॑ प्रियङ्गुतण्डु॒लैर्जु॒होति॑ च॒त्वारि॑ च (अ॒न्नहो॒मानाज्ये॑ना॒ग्नेर्मधु॑ना तण़्डु॒लैः पृथु॑कैर्ला॒जैः क॒रम्बैर्धा॒नाभि॒ सक्तु॑भिर्म॒सूस्यै प्रियङ्गुतण्डु॒लैर्द॒शान्ना॑नि॒ द्वाद॑श। )]

%3.8.15.1
प्र॒जाप॑तिरश्वमे॒धम॑सृजत। त सृ॒ष्ट रक्षास्यजिघासन्। स ए॒तान्प्र॒जाप॑तिर्न॒क्त हो॒मान॑पश्यत्। तान॑जुहोत्। तैर्वै स य॒ज्ञाद्रक्षा॒स्यपा॑हन्। यन्न॑क्त हो॒मां जु॒होति॑। य॒ज्ञादे॒व तैर्यज॑मानो॒ रक्षा॒स्यप॑ हन्ति। आज्ये॑न जुहोति। वज्रो॒ वा आज्यम्। वज्रे॑णै॒व य॒ज्ञाद्रक्षा॒स्यप॑ हन्ति॥५६॥

%3.8.15.2
आज्य॑स्य प्रति॒पदं॑ करोति। प्रा॒णो वा आज्यम्। मु॒ख॒त ए॒वास्य॑ प्रा॒णं द॑धाति। अ॒न्न॒हो॒माञ्जु॑होति। शरी॑रवदे॒वाव॑ रुन्धे। व्य॒त्यासं॑ जुहोति। उ॒भय॒स्याव॑रुध्यै। नक्तं॑ जुहोति। रक्ष॑सा॒मप॑हत्यै। आज्ये॑नान्त॒तो जु॑होति॥५७॥

%3.8.15.3
प्रा॒णो वा आज्यम्। उ॒भ॒यत॑ ए॒वास्य॑ प्रा॒णं द॑धाति। पु॒रस्ताच्चो॒परि॑ष्टाच्च। एक॑स्मै॒ स्वाहेत्या॑ह। अ॒स्मिन्ने॒व लो॒के प्रति॑तिष्ठति। द्वाभ्या॒ स्वाहेत्या॑ह। अ॒मुष्मि॑न्ने॒व लो॒के प्रति॑ तिष्ठति। उ॒भयो॑रे॒व लो॒कयो॒ प्रति॑ तिष्ठति। अ॒स्मिश्चा॒मुष्मिश्च। श॒ताय॒ स्वाहेत्या॑ह। श॒तायु॒र्वै पुरु॑षः श॒तवीर्यः। आयु॑रे॒व वी॒र्य॑मव॑ रुन्धे। स॒हस्रा॑य॒ स्वाहेत्या॑ह। आयु॒र्वै स॒हस्रम्। आयु॑रे॒वाव॑ रुन्धे। सर्व॑स्मै॒ स्वाहेत्या॑ह। अप॑रिमितमे॒वाव॑ रुन्धे॥५८॥\anuvakamend[ए॒व य॒ज्ञाद्रक्षा॒स्यप॑हन्त्यन्त॒तो जु॑होति श॒ताय॒ स्वाहेत्या॑ह स॒प्त च॑]

%3.8.16.1
प्र॒जाप॑तिं॒ वा ए॒ष ईप्स॒तीत्या॑हुः। योऽश्वमे॒धेन॒ यज॑त॒ इति॑। अथो॑ आहुः। सर्वा॑णि भू॒तानीति॑। एक॑स्मै॒ स्वाहेत्या॑ह। प्र॒जाप॑ति॒र्वा एक॑। तमे॒वाप्नो॑ति। एक॑स्मै॒ स्वाहा॒ द्वाभ्या॒ स्वाहेत्य॑भिपू॒र्वमाहु॑तीर्जुहोति। अ॒भि॒पू॒र्वमे॒व सु॑व॒र्गं लो॒कमे॑ति। ए॒को॒त्त॒रं जु॑होति॥५९॥

%3.8.16.2
ए॒क॒वदे॒व सु॑व॒र्गं लो॒कमे॑ति। सन्त॑तं जुहोति। सु॒व॒र्गस्य॑ लो॒कस्य॒ सन्त॑त्यै। श॒ताय॒ स्वाहेत्या॑ह। श॒तायु॒र्वै पुरु॑षः श॒तवीर्यः। आयु॑रे॒व वी॒र्य॑मव॑रुन्धे। स॒हस्रा॑य॒ स्वाहेत्या॑ह। आयु॒र्वै स॒हस्रम्। आयु॑रे॒वाव॑ रुन्धे। अ॒युता॑य॒ स्वाहा॑ नि॒युता॑य॒ स्वाहा प्र॒युता॑य॒ स्वाहेत्या॑ह॥६०॥

%3.8.16.3
त्रय॑ इ॒मे लो॒काः। इ॒माने॒व लो॒कानव॑ रुन्धे। अर्बु॑दाय॒ स्वाहेत्या॑ह। वाग्वा अर्बु॑दम्। वाच॑मे॒वाव॑ रुन्धे। न्य॑र्बुदाय॒ स्वाहेत्या॑ह। यो वै वा॒चो भू॒मा। तन्न्य॑र्बुदम्। वा॒च ए॒व भू॒मान॒मव॑ रुन्धे। स॒मु॒द्राय॒ स्वाहेत्या॑ह ॥६१॥

%3.8.16.4
स॒मु॒द्रमे॒वाप्नो॑ति। मध्या॑य॒ स्वाहेत्या॑ह। मध्य॑मे॒वाप्नो॑ति। अन्ता॑य॒ स्वाहेत्या॑ह। अन्त॑मे॒वाप्नो॑ति। प॒रा॒र्धाय॒ स्वाहेत्या॑ह। प॒रा॒र्धमे॒वाप्नो॑ति। उ॒षसे॒ स्वाहा॒ व्यु॑ष्ट्यै॒ स्वाहेत्या॑ह। रात्रि॒र्वा उ॒षाः। अह॒र्व्यु॑ष्टिः। अ॒हो॒रा॒त्रे ए॒वाव॑रुन्धे। अथो॑ अहोरा॒त्रयो॑रे॒व प्रति॑तिष्ठति। ता यदु॒भयी॒र्दिवा॑ वा॒ नक्तं॑ वा जुहु॒यात्। अ॒हो॒रा॒त्रे मो॑हयेत्। उ॒षसे॒ स्वाहा॒ व्यु॑ष्ट्यै॒ स्वाहो॑देष्य॒ते स्वाहोद्य॒ते स्वाहेत्यनु॑दिते जुहोति। उदि॑ताय॒ स्वाहा॑ सुव॒र्गाय॒ स्वाहा॑ लो॒काय॒ स्वाहेत्युदि॑ते जुहोति। अ॒हो॒रा॒त्रयो॒रव्य॑तिमोहाय॥६२॥\anuvakamend[ए॒को॒त्त॒रं जु॑होति प्र॒युता॑य॒ स्वाहेत्या॑ह समु॒द्राय॒ स्वाहेत्या॒हाह॒र्व्यु॑ष्टिः स॒प्त च॑]

%3.8.17.1
वि॒भूर्मा॒त्रा प्र॒भूः पि॒त्रेत्य॑श्वना॒मानि॑ जुहोति। उ॒भयो॑रे॒वैनं॑ लो॒कयोर्नाम॒धेयं॑ गमयति। आय॑नाय॒ स्वाहा॒ प्राय॑णाय॒ स्वाहेत्यु॑द्द्रा॒वाञ्जु॑होति। सर्व॑मे॒वैन॒मस्क॑न्न सुव॒र्गं लो॒कं ग॑मयति। अ॒ग्नये॒ स्वाहा॒ सोमा॑य॒ स्वाहेति॑ पूर्वहो॒माञ्जु॑होति। पूर्व॑ ए॒व द्वि॒षन्तं॒ भ्रातृ॑व्य॒मति॑ क्रामति। पृ॒थि॒व्यै स्वाहा॒ऽन्तरि॑क्षाय॒ स्वाहेत्या॑ह। य॒था॒य॒जुरे॒वैतत्। अ॒ग्नये॒ स्वाहा॒ सोमा॑य॒ स्वाहेति॑ पूर्वदी॒क्षा जु॑होति। पूर्व॑ ए॒व द्वि॒षन्तं॒ भ्रातृ॑व्य॒मति॑ क्रामति ॥६३॥

%3.8.17.2
पृ॒थि॒व्यै स्वाहा॒ऽन्तरि॑क्षाय॒ स्वाहेत्ये॑कवि॒शिनीं दी॒क्षां जु॑होति। एक॑विशति॒र्वै दे॑वलो॒काः। द्वाद॑श॒ मासा॒ पञ्च॒र्तव॑। त्रय॑ इ॒मे लो॒काः। अ॒सावा॑दि॒त्य ए॑कवि॒शः। ए॒ष सु॑व॒र्गो लो॒कः। सु॒व॒र्गस्य॑ लो॒कस्य॒ सम॑ष्ट्यै। भुवो॑ दे॒वानां॒ कर्म॒णेत्यृ॑तुदी॒क्षा जु॑होति। ऋ॒तूने॒वास्मै॑ कल्पयति। अ॒ग्नये॒ स्वाहा॑ वा॒यवे॒ स्वाहेति॑ जुहो॒त्यन॑न्तरित्यै॥६४॥

%3.8.17.3
अ॒र्वाङ्य॒ज्ञः संक्रा॑म॒त्वित्याप्तीर्जुहोति। सु॒व॒र्गस्य॑ लो॒कस्याप्त्यै। भू॒तं भव्यं॑ भवि॒ष्यदिति॒ पर्याप्तीर्जुहोति। सु॒व॒र्गस्य॑ लो॒कस्य॒ पर्याप्त्यै। आ मे॑ गृ॒हा भ॑व॒न्त्वित्या॒भूर्जु॑होति। सु॒व॒र्गस्य॑ लो॒कस्याभूत्यै। अ॒ग्निना॒ तपोऽन्व॑भव॒दित्य॑नु॒भूर्जु॑होति। सु॒व॒र्गस्य॑ लो॒कस्यानु॑भूत्यै। स्वाहा॒ऽऽधिमाधी॑ताय॒ स्वाहेति॒ सम॑स्तानि वैश्वदे॒वानि॑ जुहोति। सम॑स्तमे॒व द्वि॒षन्तं॒ भ्रातृ॑व्य॒मति॑ क्रामति॥६५॥

%3.8.17.4
द॒द्भ्यः स्वाहा॒ हनूभ्या॒ स्वाहेत्य॑ङ्गहो॒माञ्जु॑होति। अङ्गे॑अङ्गे॒ वै पुरु॑षस्य पा॒प्मोप॑श्लिष्टः। अङ्गा॑दङ्गादे॒वैनं॑ पा॒प्मन॒स्तेन॑ मुञ्चति। अ॒ञ्ज्ये॒ताय॒ स्वाहा॑ कृ॒ष्णाय॒ स्वाहा श्वे॒ताय॒ स्वाहेत्य॑श्वरू॒पाणि॑ जुहोति। रू॒पैरे॒वैन॒ सम॑र्धयति। ओष॑धीभ्य॒ स्वाहा॒ मूलेभ्य॒ स्वाहेत्यो॑षधिहो॒माञ्जु॑होति। द्व॒य्यो वा ओष॑धयः। पुष्पेभ्यो॒ऽन्याः फलं॑ गृ॒ह्णन्ति॑। मूलेभ्यो॒ऽन्याः। ता ए॒वोभयी॒रव॑ रुन्धे॥६६॥

%3.8.17.5
वन॒स्पति॑भ्य॒ स्वाहेति॑ वनस्पतिहो॒माञ्जु॑होति। आ॒र॒ण्यस्या॒न्नाद्य॒स्याव॑रुध्यै। मे॒षस्त्वा॑ पच॒तैर॑व॒त्वित्यपाव्यानि जुहोति। प्रा॒णा वै दे॒वा अपाव्याः। प्रा॒णाने॒वाव॑ रुन्धे। कूप्याभ्य॒ स्वाहा॒द्भ्यः स्वाहेत्य॒पा होमाञ्जुहोति। अ॒प्सु वा आप॑। अन्नं॒ वा आप॑। अ॒द्भ्यो वा अन्नं॑ जायते। यदे॒वाद्भ्योऽन्नं॒ जाय॑ते। तदव॑ रुन्धे॥६७॥\anuvakamend[पू॒र्व॒दी॒क्षा जु॑होति॒ पूर्व॑ ए॒व द्वि॒षन्तं॒ भ्रातृ॑व्य॒मति॑ क्राम॒त्यन॑न्तरित्यै क्रामति रुन्धे॒ जाय॑त॒ एकं॑ च]

%3.8.18.1
अम्भासि जुहोति। अ॒यं वै लो॒कोऽम्भासि। तस्य॒ वस॒वोऽधि॑पतयः। अ॒ग्निर्ज्योति॑। यदम्भासि जु॒होति॑। इ॒ममे॒व लो॒कमव॑ रुन्धे। वसू॑ना॒ सायु॑ज्यं गच्छति। अ॒ग्निं ज्योति॒रव॑ रुन्धे। नभासि जुहोति। अ॒न्तरि॑क्षं॒ वै नभासि॥६८॥

%3.8.18.2
तस्य॑ रु॒द्रा अधि॑पतयः। वा॒युर्ज्योति॑। यन्नभासि जु॒होति॑। अ॒न्तरि॑क्षमे॒वाव॑ रुन्धे। रु॒द्राणा॒ सायु॑ज्यं गच्छति। वा॒युं ज्योति॒रव॑ रुन्धे। महासि जुहोति। अ॒सौ वै लो॒को महासि। तस्या॑दि॒त्या अधि॑पतयः। सूर्यो॒ ज्योति॑॥६९॥

%3.8.18.3
यन्महासि जु॒होति॑। अ॒मुमे॒व लो॒कमव॑ रुन्धे। आ॒दि॒त्याना॒ सायु॑ज्यं गच्छति। सूर्यं॒ ज्योति॒रव॑ रुन्धे। नमो॒ राज्ञे॒ नमो॒ वरु॑णा॒येति॑ य॒व्यानि॑ जुहोति। अ॒न्नाद्य॒स्याव॑रुध्यै। म॒यो॒भूर्वातो॑ अ॒भि वा॑तू॒स्रा इति॑ ग॒व्यानि॑ जुहोति। प॒शू॒नामव॑रुध्यै। प्रा॒णाय॒ स्वाहा व्या॒नाय॒ स्वाहेति॑ सन्ततिहो॒माञ्जु॑होति। सु॒व॒र्गस्य॑ लो॒कस्य॒ संत॑त्यै ॥७०॥

%3.8.18.4
सि॒ताय॒ स्वाहाऽसि॑ताय॒ स्वाहेति॒ प्रमु॑क्तीर्जुहोति। सु॒व॒र्गस्य॑ लो॒कस्य॒ प्रमु॑क्त्यै। पृ॒थि॒व्यै स्वाहा॒ऽन्तरि॑क्षाय॒ स्वाहेत्या॑ह। य॒था॒य॒जुरे॒वैतत्। द॒त्वते॒ स्वाहा॑ऽद॒न्तका॑य॒ स्वाहेति॑ शरीरहो॒माञ्जु॑होति। पि॒तृ॒लो॒कमे॒व तैर्यज॑मानो॒ऽव॑ रुन्धे। कस्त्वा॑ युनक्ति॒ स त्वा॑ युन॒क्त्विति॑ परि॒धीन् यु॑नक्ति। इ॒मे वै लो॒काः प॑रि॒धय॑। इ॒माने॒वास्मै॑ लो॒कान् यु॑नक्ति। सु॒व॒र्गस्य॑ लो॒कस्य॒ सम॑ष्ट्यै॥७१॥

%3.8.18.5
यः प्रा॑ण॒तो य आत्म॒दा इति॑ महि॒मानौ॑ जुहोति। सु॒व॒र्गो वै लो॒को मह॑। सु॒व॒र्गमे॒व ताभ्यां लो॒कं यज॑मा॒नोऽव॑ रुन्धे। आ ब्रह्म॑न्ब्राह्म॒णो ब्र॑ह्मवर्च॒सी जा॑यता॒मिति॒ सम॑स्तानि ब्रह्मवर्च॒सानि॑ जुहोति। ब्र॒ह्म॒व॒र्चसमे॒व तैर्यज॑मा॒नोऽव॑ रुन्धे। जज्ञि॒ बीज॒मिति॑ जुहो॒त्यन॑न्तरित्यै। अ॒ग्नये॒ सम॑नमत्पृथि॒व्यै सम॑नम॒दिति॑ सन्नतिहो॒माञ्जु॑होति। सु॒व॒र्गस्य॑ लो॒कस्य॒ सन्न॑त्यै। भू॒ताय॒ स्वाहा॑ भविष्य॒ते स्वाहेति॑ भूताभ॒व्यौ होमौ॑ जुहोति। अ॒यं वै लो॒को भू॒तम्॥७२॥

%3.8.18.6
अ॒सौ भ॑वि॒ष्यत्। अ॒नयो॑रे॒व लो॒कयो॒ प्रति॑तिष्ठति। सर्व॒स्याप्त्यै। सर्व॒स्याव॑रुध्यै। यदक्र॑न्दः प्रथ॒मं जाय॑मान॒ इत्य॑श्वस्तो॒मीयं॑ जुहोति। सर्व॒स्याप्त्यै। सर्व॑स्य॒ जित्यै। सर्व॑मे॒व तेनाप्नोति। सर्वं॑ जयति। योऽश्वमे॒धेन॒ यज॑ते॥७३॥

%3.8.18.7
य उ॑ चैनमे॒वं वेद॑। य॒ज्ञ रक्षास्यजिघासन्। स ए॒तान्प्र॒जाप॑तिर्नक्तहो॒मान॑पश्यत्। तान॑जुहोत्। तैर्वै स य॒ज्ञाद्रक्षा॒स्यपा॑हन्। यन्न॑क्तहो॒मां जु॒होति॑। य॒ज्ञादे॒व तैर्यज॑मानो॒ रक्षा॒स्यप॑हन्ति। उ॒षसे॒ स्वाहा॒ व्यु॑ष्ट्यै॒ स्वाहेत्य॑न्त॒तो जु॑होति। सु॒व॒र्गस्य॑ लो॒कस्य॒ सम॑ष्ट्यै॥७४॥\anuvakamend[वै नभासि॒ सूर्यो॒ ज्योति॒ सन्त॑त्यै॒ सम॑ष्ट्यै भू॒तं यज॑ते॒ नव॑ च]

%3.8.19.1
ए॒क॒यू॒पो वै॑काद॒शिनी॑ वा। अ॒न्येषां य॒ज्ञानां॒ यूपा॑ भवन्ति। ए॒क॒वि॒शिन्य॑श्वमे॒धस्य॑। सु॒व॒र्गस्य॑ लो॒कस्या॒भिजि॑त्यै। बै॒ल्॒वो वा॑ खादि॒रो वा॑ पाला॒शो वा। अ॒न्येषां यज्ञक्रतू॒नां यूपा॑ भवन्ति। राज्जु॑दाल॒ एक॑विशत्यरत्निरश्वमे॒धस्य॑। सु॒व॒र्गस्य॑ लो॒कस्य॒ सम॑ष्ट्यै। नान्येषां पशू॒नां ते॑ज॒न्या अ॑व॒द्यन्ति॑। अव॑द्य॒न्त्यश्व॑स्य॥७५॥

%3.8.19.2
पा॒प्मा वै ते॑ज॒नी। पा॒प्मनोऽप॑हत्यै। प्ल॒क्ष॒शा॒खाया॑म॒न्येषां पशू॒नाम॑व॒द्यन्ति॑। वे॒त॒स॒शा॒खाया॒मश्व॑स्य। अ॒प्सुयो॑नि॒र्वा अश्व॑। अ॒प्सु॒जो वे॑त॒सः। स्व ए॒वास्य॒ योना॒वव॑ द्यति। यूपे॑षु ग्रा॒म्यान्प॒शून्नि॑यु॒ञ्जन्ति॑। आ॒रो॒केष्वा॑र॒ण्यान्धा॑रयन्ति। प॒शू॒नां व्यावृ॑त्त्यै। आ ग्रा॒म्यान्प॒शूल्लँभ॑न्ते। प्रार॒ण्यान्त्सृ॑जन्ति। पा॒प्मनोऽप॑हत्यै॥७६॥\anuvakamend[अश्व॑स्य॒ व्यावृ॑त्त्यै॒ त्रीणि॑ च]

%3.8.20.1
राज्जु॑दालमग्नि॒ष्ठं मि॑नोति। भ्रू॒ण॒ह॒त्याया॒ अप॑हत्यै। पौतु॑द्रवाव॒भितो॑ भवतः। पुण्य॑स्य ग॒न्धस्याव॑रुध्यै। भ्रू॒ण॒ह॒त्यामे॒वास्मा॑दप॒हत्य॑। पुण्ये॑न ग॒न्धेनो॑भ॒यत॒ परि॑ गृह्णाति। षड्बै॒ल्॒वा भ॑वन्ति। ब्र॒ह्म॒व॒र्च॒सस्याव॑रुध्यै। षट्खा॑दि॒राः। तेज॒सोऽव॑रुध्यै॥७७॥

%3.8.20.2
षट्पा॑ला॒शाः। सो॒म॒पी॒थस्याव॑रुध्यै। एक॑विशति॒ संप॑द्यन्ते। एक॑विशति॒र्वै दे॑वलो॒काः। द्वाद॑श॒ मासा॒ पञ्च॒र्तव॑। त्रय॑ इ॒मे लो॒काः। अ॒सावा॑दि॒त्य एक॑वि॒शः। ए॒ष सु॑व॒र्गो लो॒कः। सु॒व॒र्गस्य॑ लो॒कस्य॒ सम॑ष्ट्यै। श॒तं प॒शवो॑ भवन्ति॥७८॥

%3.8.20.3
श॒तायु॒ पुरु॑षः श॒तेन्द्रि॑यः। आयु॑ष्ये॒वेन्द्रि॒ये प्रति॑ तिष्ठति। सर्वं॒ वा अ॑श्वमे॒ध्याप्नो॑ति। अप॑रिमिता भवन्ति। अप॑रिमित॒स्याव॑रुध्यै। ब्र॒ह्म॒वा॒दिनो॑ वदन्ति। कस्मात्स॒त्यात्। द॒क्षि॒ण॒तोऽन्येषां पशू॒नाम॑व॒द्यन्ति॑। उ॒त्त॒र॒तोऽश्व॒स्येति॑। वा॒रुणो॒ वा अश्व॑॥७९॥

%3.8.20.4
ए॒षा वै वरु॑णस्य॒ दिक्। स्वाया॑मे॒वास्य॑ दि॒श्यव॑द्यति। यदित॑रेषां पशू॒नाम॑व॒द्यति॑। श॒त॒दे॒व॒त्यं॑ तेनाव॑ रुन्धे। चि॒तेऽग्नावधि॑ वैत॒से कटेऽश्वं॑ चिनोति। अ॒प्सुयो॑नि॒र्वा अश्व॑। अ॒प्सु॒जो वे॑त॒सः। स्व ए॒वैनं॒ योनौ॒ प्रति॑ष्ठापयति। पु॒रस्तात्प्र॒त्यञ्चं॑ तूप॒रं चि॑नोति। प॒श्चात्प्रा॒चीनं॑ गोमृ॒गम्॥८०॥

%3.8.20.5
प्रा॒णा॒पा॒नावे॒वास्मिन्त्स॒म्यञ्चौ॑ दधाति। अश्वं॑ तूप॒रं गो॑मृ॒गमिति॑ सर्व॒हुत॑ ए॒ताञ्जु॑होति। ए॒षां लो॒काना॑म॒भिजि॑त्यै। आ॒त्मना॒ऽभि जु॑होति। सात्मा॑नमे॒वैन॒ सत॑नुं करोति। सात्मा॒ऽमुष्मि॑ल्लोँ॒के भ॑वति। य ए॒वं वेद॑। अथो॒ वसो॑रे॒व धारां॒ तेनाव॑ रुन्धे। इ॒लु॒वर्दा॑य॒ स्वाहा॑ बलि॒वर्दा॑य॒ स्वाहेत्या॑ह। सं॒व॒त्स॒रो वा इ॑लु॒वर्द॑। प॒रि॒व॒त्स॒रो ब॑लि॒वर्द॑। सं॒व॒त्स॒रादे॒व प॑रिवत्स॒रादायु॒रव॑ रुन्धे। आयु॑रे॒वास्मि॑न्दधाति। तस्मा॑दश्वमेधया॒जी ज॒रसा॑ वि॒स्रसा॒मुं लो॒कमे॑ति॥८१॥\anuvakamend[तेज॒सोऽव॑रुध्यै भव॒न्त्यश्वो॑ गोमृ॒गमि॑लु॒वर्द॑श्च॒त्वारि॑ च]

%3.8.21.1
ए॒क॒वि॒शोऽग्निर्भ॑वति। ए॒क॒वि॒शः स्तोम॑। एक॑विशति॒र्यूपा। यथा॒ वा अश्वा॑ वर्\mbox{}ष॒भा वा॒ वृषा॑णः सस्फु॒रेर\sn{}। ए॒वमे॒व तत्स्तोमा॒ सस्फु॑रन्ते। यदे॑कवि॒शाः। ते यत्स॑मृ॒च्छेर\sn{}। ह॒न्येतास्य य॒ज्ञः। द्वा॒द॒श ए॒वाग्निः स्या॒दित्या॑हुः। द्वा॒द॒शः स्तोम॑॥८२॥

%3.8.21.2
एका॑दश॒ यूपा। यद्द्वा॑द॒शोऽग्निर्भव॑ति। द्वाद॑श॒ मासा संवत्स॒रः। सं॒व॒त्स॒रेणै॒वास्मा॒ अन्न॒मव॑ रुन्धे। यद्दश॒ यूपा॒ भव॑न्ति। दशाक्षरा वि॒राट्। अन्नं॑ वि॒राट्। वि॒राजै॒वान्नाद्य॒मव॑ रुन्धे। य ए॑काद॒शः। स्तन॑ ए॒वास्यै॒ सः॥८३॥

%3.8.21.3
दु॒ह ए॒वैनां॒ तेन॑। तदा॑हुः। यद्द्वा॑द॒शोऽग्निः स्याद्द्वाद॒शः स्तोम॒ एका॑दश॒ यूपा। यथा॒ स्थूरि॑णा या॒यात्। ता॒दृक्तत्। ए॒क॒वि॒श ए॒वाग्निः स्या॒दित्या॑हुः। ए॒क॒वि॒शः स्तोम॑। एक॑विशति॒र्यूपा। यथा॒ प्रष्टि॑भि॒र्याति॑। ता॒दृगे॒व तत्॥८४॥

%3.8.21.4
यो वा अ॑श्वमे॒धे ति॒स्रः क॒कुभो॒ वेद॑। क॒कुद्ध॒ राज्ञां भवति। ए॒क॒वि॒शोऽग्निर्भ॑वति। ए॒क॒वि॒शः स्तोम॑। एक॑विशति॒र्यूपा। ए॒ता वा अ॑श्वमे॒धे ति॒स्रः क॒कुभ॑। य ए॒वं वेद॑। क॒कुद्ध॒ राज्ञां भवति। यो वा अश्व॑मे॒धे त्रीणि॑ शी॒र्॒षाणि॒ वेद॑। शिरो॑ ह॒ राज्ञां भवति। ए॒क॒वि॒शोऽग्निर्भ॑वति। ए॒क॒वि॒शः स्तोम॑। एक॑विशति॒र्यूपा। ए॒तानि॒ वा अ॑श्वमे॒धे त्रीणि॑ शी॒र्॒षाणि॑। य ए॒वं वेद॑। शिरो॑ ह॒ राज्ञां भवति॥८५॥\anuvakamend[द्वा॒द॒शः स्तोम॒ स ए॒व तच्छिरो॑ ह॒ राज्ञां भवति॒ षट् च॑]

%3.8.22.1
दे॒वा वा अ॑श्वमे॒धे पव॑माने। सु॒व॒र्गं लो॒कं न प्राजा॑नन्। तमश्व॒ प्राजा॑नात्। यद॑श्वमे॒धेऽश्वे॑न॒ मेध्ये॒नोद॑ञ्चो बहिष्पवमा॒न सर्प॑न्ति। सु॒व॒र्गस्य॑ लो॒कस्य॒ प्रज्ञात्यै। न वै म॑नु॒ष्य॑ सुव॒र्गं लो॒कमञ्ज॑सा वेद। अश्वो॒ वै सु॑व॒र्गं लो॒कमञ्ज॑सा वेद। यदु॑द्गा॒तोद्गायेत्। यथा क्षेत्रज्ञो॒ऽन्येन॑ प॒था प्र॑तिपा॒दयेत्। ता॒दृक्तत्॥८६॥

%3.8.22.2
उ॒द्गा॒तार॑मप॒रुध्य॑। अश्व॑मुद्गी॒थाय॑ वृणीते। यथा क्षेत्र॒ज्ञोऽञ्ज॑सा॒ नय॑ति। ए॒वमे॒वैन॒मश्व॑ सुव॒र्गं लो॒कमञ्ज॑सा नयति। पुच्छ॑म॒न्वा र॑भन्ते। सु॒व॒र्गस्य॑ लो॒कस्य॒ सम॑ष्ट्यै। हिं क॑रोति। सामै॒वाक॑। हिं क॑रोति। उ॒द्गी॒थ ए॒वास्य॒ सः॥८७॥

%3.8.22.3
वड॑बा॒ उप॑ रुन्धन्ति। मि॒थु॒न॒त्वाय॒ प्रजात्यै। अथो॒ यथो॑पगा॒तार॑ उप॒गाय॑न्ति। ता॒दृगे॒व तत्। उद॑गासी॒दश्वो॒ मेध्य॒ इत्या॑ह। प्रा॒जा॒प॒त्यो वा अश्व॑। प्र॒जाप॑तिरुद्गी॒थः। उ॒द्गी॒थमे॒वाव॑ रुन्धे। अथो॑ ऋक्सा॒मयो॑रे॒व प्रति॑ तिष्ठति। हिर॑ण्येनो॒पाक॑रोति। ज्योति॒र्वै हिर॑ण्यम्। ज्योति॑रे॒व मु॑ख॒तो द॑धाति। यज॑माने च प्र॒जासु॑ च। अथो॒ हिर॑ण्यज्योतिरे॒व यज॑मानः सुव॒र्गं लो॒कमे॑ति॥८८॥\anuvakamend[तत्स उ॒पाक॑रोति च॒त्वारि॑ च]

%3.8.23.1
पुरु॑षो॒ वै य॒ज्ञः। य॒ज्ञः प्र॒जाप॑तिः। यदश्वे॑ प॒शून्नि॑यु॒ञ्जन्ति॑। य॒ज्ञादे॒व तद्य॒ज्ञं प्रयु॑ङ्क्ते। अश्वं॑ तूप॒रं गो॑मृ॒गम्। तान॑ग्नि॒ष्ठ आल॑भते। से॒ना॒मु॒खमे॒व तत्सश्य॑ति। तस्माद्राजमु॒खं भी॒ष्मं भावु॑कम्। आ॒ग्ने॒यं कृ॒ष्णग्री॑वं पु॒रस्ताल्ल॒लाटे। पू॒र्वा॒ग्निमे॒व तं कु॑रुते॥८९॥

%3.8.23.2
तस्मात्पूर्वा॒ग्निं पु॒रस्तात्स्थापयन्ति। पौ॒ष्णम॒न्वञ्चम्। अन्नं॒ वै पू॒षा। तस्मात्पूर्वा॒ग्नावा॑हा॒र्य॑मा ह॑रन्ति। ऐ॒न्द्रा॒पौ॒ष्णमु॒परि॑ष्टात्। ऐ॒न्द्रो वै रा॑ज॒न्योऽन्नं॑ पू॒षा। अ॒न्नाद्ये॑नै॒वैन॑मुभ॒यत॒ परि॑ गृह्णाति। तस्माद्राज॒न्योऽन्ना॒दो भावु॑कः। आ॒ग्ने॒यौ कृ॒ष्णग्री॑वौ बाहु॒वोः। बा॒हु॒वोरे॒व वी॒र्यं धत्ते॥९०॥

%3.8.23.3
तस्माद्राज॒न्यो॑ बाहुब॒लीभावु॑कः। त्वा॒ष्ट्रौ लो॑मशस॒क्थौ स॒क्थ्योः। स॒क्थ्योरे॒व वी॒र्यं॑ धत्ते। तस्माद्राज॒न्य॑ ऊरुब॒लीभावु॑कः। शि॒ति॒पृ॒ष्ठौ बा॑र्\mbox{}हस्प॒त्यौ पृ॒ष्ठे। ब्र॒ह्म॒व॒र्च॒समे॒वोपरि॑ष्टाद्धत्ते। अथो॑ क॒वचे॑ ए॒वैते अ॒भित॒ पर्यू॑हते। तस्माद्राज॒न्य॑ सन्न॑द्धो वी॒र्यं॑ करोति। धा॒त्रे पृ॑षोद॒रम॒धस्तात्। प्र॒ति॒ष्ठामे॒वैतां कु॑रुते। अथो॑ इ॒यं वै धा॒ता। अ॒स्यामे॒व प्रति॑ तिष्ठति। सौ॒र्यं ब॒लक्षं॒ पुच्छे। उ॒त्से॒धमे॒व तं कु॑रुते। तम्मा॑दुत्से॒धम्भ॒ये प्र॒जा अ॒भिसश्र॑यन्ति॥९१॥\anuvakamend[कु॒रु॒ते॒ ध॒त्ते॒ कु॒रु॒ते॒ पञ्च॑ च]




\prashnaend{सा॒ङ्ग्र॒ह॒ण्या चतु॑ष्टय्यो॒ यो वै यः पि॒तुश्च॒त्वारो॒ यथा॑ नि॒क्तं प्र॒जाप॑तये त्वा॒ यथा॒ प्रोक्षि॑तं वि॒भूरा॑ह प्र॒जाप॑तिरकामयताश्वमे॒धेन॑ प्र॒जाप॑ति॒र्न किञ्च॒न सा॑वि॒त्रमा ब्रह्म॑न्प्र॒जाप॑तिर्दे॒वैभ्य॑ प्र॒जाप॑ती॒ रक्षासि प्र॒जाप॑तिमीप्सति वि॒भूर॑श्वना॒मान्यम्भास्येकयू॒पो राज्जु॑दालमेकवि॒शो दे॒वाः पुरु॑ष॒स्त्रयो॑विशतिः॥२३॥}{सा॒ङ्ग्रह॒ण्या तस्मा॑दश्वमेधया॒जी यत्परि॑मिता॒ यद्य॑ज्ञमु॒खे यो दी॒क्षान्दे॒वाने॒व त्रय॑ इ॒मे सि॒ताय॑ प्राणापा॒नावे॒वास्मि॒न्तस्माद्राज॒न्य॑ एक॑नवतिः॥९१॥}{सा॒ङ्ग्र॒ह॒ण्या सश्र॑यन्ति॥}{हरि॑ ओम्॥}{इति श्रीकृष्णयजुर्वेदीयतैत्तिरीयब्राह्मणे तृतीयाष्टके अष्टमः प्रपाठकः समाप्तः॥}
\clearpage
