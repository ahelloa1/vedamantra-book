\sect{षष्ठमः प्रश्नः}
\setcounter{anuvakam}{0}
\dnsub{तैत्तिरीयब्राह्मणे तृतीयाष्टके षष्ठः प्रपाठकः}

%3.6.1.1
अ॒ञ्जन्ति॒ त्वाम॑ध्व॒रे दे॑व॒यन्त॑। वन॑स्पते॒ मधु॑ना॒ दैव्ये॑न। यदू॒र्ध्वस्ति॑ष्ठा॒द्द्रवि॑णे॒ह ध॑त्तात्। यद्वा॒ क्षयो॑ मा॒तुर॒स्या उ॒पस्थे। उच्छ्र॑यस्व वनस्पते। वर्ष्म॑न्पृथि॒व्या अधि॑। सुमि॑ती मी॒यमा॑नः। वर्चो॑धा य॒ज्ञवा॑हसे। समि॑द्धस्य॒ श्रय॑माणः पु॒रस्तात्। ब्रह्म॑ वन्वा॒नो अ॒जर सु॒वीरम्॥१॥

%3.6.1.2
आ॒रे अ॒स्मदम॑तिं॒ बाध॑मानः। उच्छ्र॑यस्व मह॒ते सौभ॑गाय। ऊ॒र्ध्व ऊ॒षुण॑ ऊ॒तये। तिष्ठा॑ दे॒वो न स॑वि॒ता। ऊ॒र्ध्वो वाज॑स्य॒ सनि॑ता॒ यद॒ञ्जिभि॑। वा॒घद्भि॑र्वि॒ह्वया॑महे। ऊ॒र्ध्वो न॑ पा॒ह्यह॑सो॒ नि के॒तुना। विश्व॒ सम॒त्रिण॑न्दह। कृ॒धी न॑ ऊ॒र्ध्वां च॒ रथा॑य जी॒वसे। वि॒दा दे॒वेषु॑ नो॒ दुव॑॥२॥

%3.6.1.3
जा॒तो जा॑यते सुदिन॒त्वे अह्नाम्। सम॒र्य आ वि॒दथे॒ वर्ध॑मानः। पु॒नन्ति॒ धीरा॑ अ॒पसो॑ मनी॒षा। दे॒व॒या विप्र॒ उदि॑यर्ति॒ वाचम्। युवा॑ सु॒वासा॒ परि॑वीत॒ आगात्। स उ॒ श्रेयान्भवति॒ जाय॑मानः। तन्धीरा॑सः क॒वय॒ उन्न॑यन्ति। स्वा॒धियो॒ मन॑सा देव॒यन्त॑। पृ॒थु॒पाजा॒ अम॑र्त्यः। घृ॒तनि॑र्णि॒ख्स्वा॑हुतः। अ॒ग्निर्य॒ज्ञस्य॑ हव्य॒वाट्। त स॒बाधो॑ य॒तः स्रु॑चः। इ॒त्था धि॒या य॒ज्ञव॑न्तः। आच॑क्रुर॒ग्निमू॒तये। त्वं वरु॑ण उ॒त मि॒त्रो अ॑ग्ने। त्वां व॑र्धन्ति म॒तिभि॒र्वसि॑ष्ठाः। त्वे वसु॑ सुषण॒नानि॑ सन्तु। यू॒यं पा॑त स्व॒स्तिभि॒ सदा॑ नः॥३॥\anuvakamend[सु॒वीर॒न्दुव॒ स्वा॑हुतो॒ऽष्टौ च॑]

%3.6.2.1
होता॑ यक्षद॒ग्नि स॒मिधा॑ सुष॒मिधा॒ समि॑द्धं॒ नाभा॑ पृथि॒व्याः स॑ङ्ग॒थे वा॒मस्य॑। वर्ष्म॑न्दि॒व इ॒डस्प॒दे वेत्वाज्य॑स्य॒ होत॒र्यज॑। होता॑ यक्ष॒त्तनू॒नपा॑त॒मदि॑ते॒र्गर्भं॒ भुव॑नस्य गो॒पाम्। मध्वा॒द्य दे॒वो दे॒वेभ्यो॑ देव॒यानान्प॒थो अ॑नक्तु॒ वेत्वाज्य॑स्य॒ होत॒र्यज॑। होता॑ यक्ष॒न्नरा॒शसं॑ नृश॒स्त्रं नॄः प्र॑णेत्रम्। गोभि॑र्व॒पावा॒न्त्स्याद्वी॒रैः शक्ती॑वा॒न्रथै प्रथम॒या वा॒ हिर॑ण्यैश्च॒न्द्री वेत्वाज्य॑स्य॒ होत॒र्यज॑। होता॑ यक्षद॒ग्निमि॒ड ई॑डि॒तो दे॒वो दे॒वा आव॑क्षद्दू॒तो ह॑व्य॒वाडमू॑रः। उपे॒मं य॒ज्ञमुपे॒मां दे॒वो दे॒वहू॑तिमवतु॒ वेत्वाज्य॑स्य॒ होत॒र्यज॑। होता॑ यक्षद्ब॒र्‌हिः सु॒ष्टरी॒मोर्णं॑म्रदा अ॒स्मिन् य॒ज्ञे वि च॒ प्र च॑ प्रथता स्वास॒स्थं दे॒वेभ्य॑। एमे॑नद॒द्य वस॑वो रु॒द्रा आ॑दि॒त्याः स॑दन्तु प्रि॒यमिन्द्र॑स्यास्तु॒ वेत्याज्य॑स्य॒ होत॒र्यज॑॥४॥

%3.6.2.2
होता॑ यक्ष॒द्दुर॑ ऋ॒ष्वाः क॑व॒ष्यो को॑षधावनी॒रुदाता॑भी॒र्जिह॑तां॒ विपक्षो॑भिः श्रयन्ताम्। सु॒प्रा॒य॒णा अ॒स्मिन् य॒ज्ञे विश्र॑यन्तामृता॒वृधो॑ वि॒यन्त्वाज्य॑स्य॒ होत॒र्यज॑। होता॑ यक्षदु॒षासा॒नक्ता॑ बृह॒ती सु॒पेश॑सा॒ नॄः पति॑भ्यो॒ योनि॑ङ्कृण्वा॒ने। स॒स्मय॑माने॒ इन्द्रे॑ण दे॒वैरेदं ब॒र्॒हिः सी॑दतां वी॒तामाज्य॑स्य॒ होत॒र्यज॑। होता॑ यक्ष॒द्दैव्या॒ होता॑रा म॒न्द्रा पोता॑रा क॒वी प्रचे॑तसा। स्वि॑ष्टम॒द्यान्यः क॑रदि॒षा स्व॑भिगूर्तम॒न्य ऊ॒र्जा सत॑वसे॒मं य॒ज्ञं दि॒वि दे॒वेषु॑ धत्तां वी॒तामाज्य॑स्य॒ होत॒र्यज॑। होता॑ यक्षत्ति॒स्रो दे॒वीर॒पसा॑म॒पस्त॑मा॒ अच्छि॑द्रम॒द्येदमप॑स्तन्वताम्। दे॒वेभ्यो॑ दे॒वीर्दे॒वमपो॑ वि॒यन्त्वाज्य॑स्य॒ होत॒र्यज॑। होता॑ यक्ष॒त्त्वष्टा॑र॒मचि॑ष्टु॒मपा॑क रेतो॒धां विश्र॑वसं यशो॒धाम्। पु॒रु॒रूप॒मका॑मकर्‌शन सु॒पोष॒ पोषै॒ स्यात्सु॒वीरो॑ वी॒रैर्वेत्वाज्य॑स्य॒ होत॒र्यज॑। होता॑ यक्ष॒द्वन॒स्पति॑मु॒पाव॑स्रक्षद्धि॒यो जो॒ष्टार श॒शम॒न्नर॑। स्वदा॒त्स्वधि॑तिर्\mbox{}ऋतु॒थाद्य दे॒वो दे॒वेभ्यो॑ ह॒व्यावा॒ड्वेत्वाज्य॑स्य॒ होत॒र्यज॑। होता॑ यक्षद॒ग्नि स्वाहाऽऽज्य॑स्य॒ स्वाहा॒ मेद॑स॒ स्वाहा स्तो॒काना॒ स्वाहा॒ स्वाहा॑कृतीना॒ स्वाहा॑ ह॒व्यसूक्तीनाम्। स्वाहा॑ दे॒वा आज्य॒पान्त्स्वाहा॒ऽग्नि हो॒त्राज्जु॑षा॒णा अग्न॒ आज्य॑स्य वियन्तु॒ होत॒र्यज॑॥५॥\anuvakamend[प्रि॒यमिन्द्र॑स्यास्तु॒ वेत्वाज्य॑स्य॒ होत॒र्यज॑ सु॒वीरो॑ वी॒रैर्वेत्वाज्य॑स्य॒ होत॒र्यज॑ च॒त्वारि॑ च (अ॒ग्निन्तनू॒नपा॑त॒न्नरा॒शस॑म॒ग्निमि॒ड ई॑डि॒तो ब॒र्‌हिर्दुर॑ उ॒षासा॒नक्ता॒ दैव्या॑ ति॒स्रस्त्वष्टा॑रं॒ वन॒स्पति॑म॒ग्निम्। पञ्च॒ वेत्वेको॑ वि॒यन्तु॒ द्विर्वी॒तामेको॑ वि॒यन्तु॒ द्विर्वेत्वेको॑ वियन्तु॒ होत॒र्यज॑ ॥ )]

%3.6.3.1
समि॑द्धो अ॒द्य मनु॑षो दुरो॒णे। दे॒वो दे॒वान् य॑जसि जातवेदः। आ च॒ वह॑ मित्रमहश्चिकि॒त्वान्। त्वन्दू॒तः क॒विर॑सि॒ प्रचे॑ताः। तनू॑नपात्प॒थ ऋ॒तस्य॒ यानान्॑। मध्वा॑ सम॒ञ्जन्त्स्व॑दया सुजिह्व। मन्मा॑नि धी॒भिरु॒त य॒ज्ञमृ॒न्धन्। दे॒व॒त्रा च॑ कृणुह्यध्व॒रन्न॑। नरा॒शस॑स्य महि॒मान॑मेषाम्। उप॑ स्तोषाम यज॒तस्य॑ य॒ज्ञैः॥६॥

%3.6.3.2
ते सु॒क्रत॑व॒ शुच॑यो धिय॒न्धाः। स्वद॑न्तु दे॒वा उ॒भया॑नि ह॒व्या। आ॒जुह्वा॑न॒ ईड्यो॒ वन्द्य॑श्च। आयाह्यग्ने॒ वसु॑भिः स॒जोषा। त्वं दे॒वाना॑मसि यह्व॒ होता। स ए॑नान् यक्षीषि॒तो यजी॑यान्। प्रा॒चीनं॑ ब॒र्‌हिः प्र॒दिशा॑ पृथि॒व्याः। वस्तो॑र॒स्या वृ॑ज्यते॒ अग्रे॒ अह्नाम्। व्यु॑ प्रथते वित॒रं वरी॑यः। दे॒वेभ्यो॒ अदि॑तये स्यो॒नम्॥७॥

%3.6.3.3
व्यच॑स्वतीरुर्वि॒या विश्र॑यन्ताम्। पति॑भ्यो॒ न जन॑य॒ शुम्भ॑मानाः। देवीर्द्वारो बृहतीर्विश्वमिन्वाः। दे॒वेभ्यो॑ भवथ सुप्राय॒णाः। आसु॒ष्वय॑न्ती यज॒ते उपा॑के। उ॒षासा॒नक्ता॑ सदतां॒ नि योनौ। दि॒व्ये योष॑णे बृह॒ती सु॑रु॒क्मे। अधि॒ श्रिय शुक्र॒पिश॒न्दधा॑ने। दैव्या॒ होता॑रा प्रथ॒मा सु॒वाचा। मिमा॑ना य॒ज्ञं मनु॑षो॒ यज॑ध्यै॥८॥

%3.6.3.4
प्र॒चो॒दय॑न्ता वि॒दथे॑षु का॒रू। प्रा॒चीनं॒ ज्योति॑ प्र॒दिशा॑ दि॒शन्ता। आ नो॑ य॒ज्ञं भार॑ती॒ तूय॑मेतु। इडा॑ मनु॒ष्वदि॒ह चे॒तय॑न्ती। ति॒स्रो दे॒वीर्ब॒र्‌हिरेद स्यो॒नम्। सर॑स्वती॒ स्वप॑सः सदन्तु। य इ॒मे द्यावा॑पृथि॒वी जनि॑त्री। रू॒पैरपिश॒द्भुव॑नानि॒ विश्वा। तम॒द्य हो॑तरिषि॒तो यजी॑यान्। दे॒वन्त्वष्टा॑रमि॒ह य॑क्षि वि॒द्वान्॥९॥

%3.6.3.5
उ॒पाव॑सृज॒त्मन्या॑ सम॒ञ्जन्। दे॒वानां॒ पाथ॑ ऋतु॒था ह॒वीषि॑। वन॒स्पति॑ शमि॒ता दे॒वो अ॒ग्निः। स्वद॑न्तु ह॒व्यं मधु॑ना घृ॒तेन॑। स॒द्यो जा॒तो व्य॑मिमीत य॒ज्ञम्। अ॒ग्निर्दे॒वाना॑मभवत्पुरो॒गाः। अ॒स्य होतु॑ प्र॒दिश्यृ॒तस्य॑ वा॒चि। स्वाहा॑कृत ह॒विर॑दन्तु दे॒वाः॥१०॥\anuvakamend[य॒ज्ञैः स्यो॒नं यज॑ध्यै वि॒द्वान॒ष्टौ च॑]

%3.6.4.1
अ॒ग्निर्‌होता॑ नो अध्व॒रे। वा॒जी सन्परि॑णीयते। दे॒वो दे॒वेषु॑ य॒ज्ञिय॑। परि॑त्रिवि॒ष्ट्य॑ध्व॒रम्। यात्य॒ग्नी र॒थीरि॑व। आ दे॒वेषु॒ प्रयो॒ दध॑त्। परि॒ वाज॑पतिः क॒विः। अ॒ग्निर्‌ह॒व्यान्य॑क्रमीत्। दध॒द्रत्ना॑नि दा॒शुषे॥११॥\anuvakamend[अ॒ग्निर्‌होता॑ नो॒ नव॑]

%3.6.5.1
अजै॑द॒ग्निः। अस॑न॒द्वाज॒न्नि। दे॒वो दे॒वेभ्यो॑ ह॒व्यावाट्। प्राञ्जो॑भिर्‌हिन्वा॒नः। धेना॑भि॒ कल्प॑मानः। य॒ज्ञस्यायु॑ प्रति॒रन्। उप॒ प्रेष्य॑ होतः। ह॒व्या दे॒वेभ्य॑॥१२॥\anuvakamend[अजै॑द॒ष्टौ]

%3.6.6.1
दैव्या शमितार उ॒त म॑नुष्या॒ आर॑भध्वम्। उप॑नयत॒ मेध्या॒ दुर॑। आ॒शासा॑ना॒ मेध॑पतिभ्यां॒ मेधम्। प्रास्मा॑ अ॒ग्निं भ॑रत। स्तृ॒णी॒त ब॒र्॒हिः। अन्वे॑नं मा॒ता म॑न्यताम्। अनु॑ पि॒ता। अनु॒ भ्राता॒ सग॑र्भ्यः। अनु॒ सखा॒ सयूथ्यः। उ॒दी॒चीना अस्य प॒दो निध॑त्तात्॥१३॥

%3.6.6.2
सूर्य॒ञ्चक्षु॑र्गमयतात्। वातं॑ प्रा॒णम॒न्वव॑सृजतात्। दिश॒ श्रोत्रम्। अ॒न्तरि॑क्ष॒मसुम्। पृ॒थि॒वी शरी॑रम्। ए॒क॒धाऽस्य॒ त्वच॒माच्छ्य॑तात्। पु॒रा नाभ्या॑ अपि॒शसो॑ व॒पामुत्खि॑दतात्। अ॒न्तरे॒वोष्माणं॑ वारयतात्। श्ये॒नम॑स्य॒ वक्ष॑ कृणुतात्। प्र॒शसा॑ बा॒हू॥१४॥

%3.6.6.3
श॒ला दो॒षणी। क॒श्यपे॒वासा। अच्छि॑द्रे॒ श्रोणी। क॒वषो॒रू स्रे॒कप॑र्णाष्ठी॒वन्ता। षड्विशतिरस्य॒ वङ्क्र॑यः। ता अ॑नु॒ष्ठ्योच्च्या॑वयतात्। गात्र॑ङ्गात्रम॒स्यानू॑नङ्कृणुतात्। ऊ॒व॒ध्य॒गो॒हं पार्थि॑वङ्खनतात्। अ॒स्ना रक्ष॒ ससृ॑जतात्। व॒नि॒ष्ठुम॑स्य॒ मा रा॑विष्ट॥१५॥

%3.6.6.4
उरू॑कं॒ मन्य॑मानाः। नेद्व॑स्तो॒के तन॑ये। रवि॑ता॒रव॑च्छमितारः। अध्रि॑गो शमी॒ध्वम्। सु॒शमि॑ शमीध्वम्। श॒मी॒ध्वम॑ध्रिगो। अध्रि॑गु॒श्चापा॑पश्च। उ॒भौ दे॒वाना शमि॒तारौ। तावि॒मं प॒शु श्र॑पयतां प्रवि॒द्वासौ। यथा॑यथाऽस्य॒ श्रप॑ण॒न्तथा॑तथा॥१६॥\anuvakamend[ध॒त्ता॒द्बा॒हू मा रा॑विष्ट॒ तथा॑तथा]

%3.6.7.1
जु॒षस्व॑ स॒प्रथ॑स्तमम्। वचो॑ दे॒वप्स॑रस्तमम्। ह॒व्या जुह्वा॑न आ॒सनि॑। इ॒मन्नो॑ य॒ज्ञम॒मृते॑षु धेहि। इ॒मा ह॒व्या जा॑तवेदो जुषस्व। स्तो॒काना॑मग्ने॒ मेद॑सो घृ॒तस्य॑। होत॒ प्राशा॑न प्रथ॒मो नि॒षद्य॑। घृ॒तव॑न्तः पावक ते। स्तो॒काः श्चो॑तन्ति॒ मेद॑सः। स्वध॑र्मन्दे॒ववी॑तये॥१७॥

%3.6.7.2
श्रेष्ठ॑न्नो धेहि॒ वार्यम्। तुभ्य स्तो॒का घृ॑त॒श्चुत॑। अग्ने॒ विप्रा॑य सन्त्य। ऋषि॒ श्रेष्ठ॒ समि॑ध्यसे। य॒ज्ञस्य॑ प्रावि॒ता भ॑व। तुभ्य श्चोतन्त्यध्रिगो शचीवः। स्तो॒कासो॑ अग्ने॒ मेद॑सो घृ॒तस्य॑। क॒वि॒श॒स्तो बृ॑ह॒ता भा॒नुनागा। ह॒व्या जु॑षस्व मेधिर। ओजि॑ष्ठन्ते मध्य॒तो मेद॒ उद्भृ॑तम्। प्र ते॑ व॒यन्द॑दामहे। श्चोत॑न्ति ते वसो स्तो॒का अधि॑त्व॒चि। प्रति॒ तान्दे॑व॒शोवि॑हि॥१८॥\anuvakamend[दे॒ववी॑तय॒ उद्भृ॑त॒न्त्रीणि॑ च]

%3.6.8.1
आवृ॑त्रहणा वृत्र॒हभि॒ शुष्मै। इन्द्र॑ या॒तन्नमो॑भिरग्ने अ॒र्वाक्। यु॒व राधो॑भि॒रक॑वेभिरिन्द्र। अग्ने॑ अ॒स्मे भ॑वतमुत्त॒मेभि॑। होता॑ यक्षदिन्द्रा॒ग्नी। छाग॑स्य व॒पाया॒ मेद॑सः। जु॒षेता ह॒विः। होत॒र्यज॑। विह्यख्य॒न्मन॑सा॒ वस्य॑ इ॒च्छन्। इन्द्राग्नी ज्ञा॒स उ॒त वा॑ सजा॒तान्॥१९॥

%3.6.8.2
नान्या यु॒वत्प्रम॑तिरस्ति॒ मह्यम्। स वा॒न्धियं॑ वाज॒यन्ती॑मतक्षम्। होता॑ यक्षदिन्द्रा॒ग्नी। पु॒रो॒डाश॑स्य जु॒षेता ह॒विः। होत॒र्यज॑। त्वामी॑डते अजि॒रन्दू॒त्या॑य। ह॒विष्म॑न्त॒ सद॒मिन्मानु॑षासः। यस्य॑ दे॒वैरास॑दो ब॒र्‌हि॒र॑ग्ने। अहान्यस्मै सु॒दिना॑ भवन्तु। होता॑ यक्षद॒ग्निम्। पु॒रो॒डाश॑स्य जु॒षता ह॒विः। होत॒र्यज॑॥२०॥\anuvakamend[स॒जा॒तान॒ग्निन्द्वे च॑]

%3.6.9.1
गी॒र्भिर्विप्र॒ प्रम॑तिमि॒च्छमा॑नः। ईट्टे॑ र॒यिं य॒शसं॑ पूर्व॒भाजम्। इन्द्राग्नी वृत्रहणा सुवज्रा। प्र णो॒ नव्ये॑भिस्तिरतन्दे॒ष्णैः। माच्छेद्म र॒श्मीरिति॒ नाध॑मानाः। पि॒तृ॒णा शक्ती॑रनु॒यच्छ॑मानाः। इ॒न्द्रा॒ग्निभ्या॒ङ्कं वृष॑णो मदन्ति। ताह्यद्री॑ धि॒षणा॑या उ॒पस्थे। अ॒ग्नि सु॑दी॒ति सु॒दृशं॑ गृ॒णन्त॑। न॒म॒स्याम॒स्त्वेड्यं॑ जातवेदः। त्वान्दू॒तम॑र॒ति ह॑व्य॒वाहम्। दे॒वा अ॑कृण्वन्न॒मृत॑स्य॒ नाभिम्॥२१॥\anuvakamend[जा॒त॒वे॒दो॒ द्वे च॑]

%3.6.10.1
त्व ह्य॑ग्ने प्रथ॒मो म॒नोता। अ॒स्या धि॒यो अभ॑वो दस्म॒होता। त्व सीव्वृँषन्नकृणोर्दु॒ष्टरी॑तु। सहो॒ विश्व॑स्मै॒ सह॑से॒ सह॑ध्यै। अधा॒ होता॒ न्य॑सीदो॒ यजी॑यान्। इ॒डस्प॒द इ॒षय॒न्नीड्य॒ सन्। तन्त्वा॒ नर॑ प्रथ॒मन्दे॑व॒यन्त॑। म॒हो रा॒ये चि॒तय॑न्तो॒ अनु॑ग्मन्। वृ॒तेव॒ यन्तं॑ ब॒हुभि॑र्वस॒व्यै। त्वे र॒यिञ्जा॑गृ॒वासो॒ अनु॑ग्मन्॥२२॥

%3.6.10.2
रुश॑न्तम॒ग्निन्द॑र्‌श॒तम्बृ॒हन्तम्। व॒पाव॑न्तं वि॒श्वहा॑ दीदि॒वासम्। प॒दन्दे॒वस्य॒ नम॑सा वि॒यन्त॑। श्र॒व॒स्यव॒ श्रव॑ आप॒न्नमृ॑क्तम्। नामा॑नि चिद्दधिरे य॒ज्ञिया॑नि। भ॒द्रायान्ते रणयन्त॒ सन्दृ॑ष्टौ। त्वां व॑र्धन्ति क्षि॒तय॑ पृथि॒व्याम्। त्व राय॑ उ॒भया॑सो॒ जना॑नाम्। त्वन्त्रा॒ता त॑रणे॒ चेत्यो॑भूः। पि॒ता मा॒ता सद॒मिन्मानु॑षाणाम्॥२३॥

%3.6.10.3
सप॒र्येण्य॒ स प्रि॒यो वि॒क्ष्व॑ग्निः। होता॑ म॒न्द्रो निष॑सादा॒ यजी॑यान्। तन्त्वा॑ व॒यन्दम॒ आ दी॑दि॒वासम्। उप॑ज्ञु॒बाधो॒ नम॑सा सदेम। तन्त्वा॑ व॒य सु॒धियो॒ नव्य॑मग्ने। सु॒म्ना॒यव॑ ईमहे देव॒यन्त॑। त्वं विशो॑ अनयो॒ दीद्या॑नः। दि॒वो अ॑ग्ने बृह॒ता रो॑च॒नेन॑। वि॒शां क॒विं वि॒श्पति॒ शश्व॑तीनाम्। नि॒तोश॑नं वृष॒भं च॑र्‌षणी॒नाम्॥२४॥

%3.6.10.4
प्रेती॑षणि मि॒षय॑न्तं पाव॒कम्। राज॑न्तम॒ग्निं य॑ज॒त र॑यी॒णाम्। सो अ॑ग्न ईजे शश॒मे च॒ मर्त॑। यस्त॒ आन॑ट्त्स॒मिधा॑ ह॒व्यदा॑तिम्। य आहु॑तिं॒ परि॒ वेदा॒ नमो॑भिः। विश्वेत्सवा॒मा द॑धते॒ त्वोत॑। अ॒स्मा उ॑ ते॒ महि॑ म॒हे वि॑धेम। नमो॑भिरग्ने स॒मिधो॒त ह॒व्यैः। वेदी॑सूनो सहसो गी॒र्भिरु॒क्थैः। आ ते भ॒द्राया सुम॒तौ य॑तेम॥२५॥

%3.6.10.5
आ॒ यस्त॒तन्थ॒ रोद॑सी॒ विभा॒सा। श्रवो॑भिश्च श्रव॒स्य॑स्तरु॑त्रः। बृ॒हद्भि॒र्वाजै॒ स्थवि॑रेभिर॒स्मे। रे॒वद्भि॑रग्ने वित॒रं वि भा॑हि। नृ॒वद्व॑सो॒ सद॒मिद्धेह्य॒स्मे। भूरि॑तो॒काय॒ तन॑याय प॒श्वः। पू॒र्वीरिषो॑ बृह॒तीरा॒रे अ॑घाः। अ॒स्मे भ॒द्रा सौश्रव॒सानि॑ सन्तु। पु॒रूण्य॑ग्ने पुरु॒धा त्वा॒या। वसू॑नि राजन्व॒सुता॑ते अश्याम्। पु॒रूणि॒ हि त्वे पु॑रुवार॒ सन्ति॑। अग्ने॒ वसु॑ विध॒ते राज॑नि॒त्वे॥२६॥\anuvakamend[जा॒गृ॒वासो॒ अनु॑ग्म॒न्मानु॑षाणाञ्चर्‌षणी॒नां य॑तेमाश्या॒न्द्वे च॑]

%3.6.11.1
आभ॑रत शिक्षतं वज्रबाहू। अ॒स्मा इ॑न्द्राग्नी अवत॒ शची॑भिः। इ॒मे नु ते र॒श्मय॒ सूर्य॑स्य। येभि॑ सपि॒त्वं पि॒तरो॑ न॒ आय\sn{}। होता॑ यक्षदिन्द्रा॒ग्नी। छाग॑स्य ह॒विष॒ आत्ता॑म॒द्य। म॒ध्य॒तो मेद॒ उद्भृ॑तम्। पु॒रा द्वेषोभ्यः। पु॒रा पौरु॑षेय्या गृ॒भः। घस्तान्नू॒नम्॥२७॥

%3.6.11.2
घा॒से अ॑ज्राणां॒ यव॑सप्रथमानाम्। सु॒मत्क्ष॑राणा श॒तरु॑द्रियाणाम्। अ॒ग्नि॒ष्वा॒त्तानां॒ पीवो॑पवसनानाम्। पा॒र्श्व॒तः श्रो॑णि॒तः शि॑ताम॒त उ॑त्साद॒तः। अङ्गा॑दङ्गा॒दव॑त्तानाम्। कर॑त ए॒वेन्द्रा॒ग्नी। जु॒षेता ह॒विः। होत॒र्यज॑। दे॒वेभ्यो॑ वनस्पते ह॒वीषि॑। हिर॑ण्यपर्ण प्र॒दिव॑स्ते॒ अर्थम्।॥२८॥

%3.6.11.3
प्र॒द॒क्षि॒णिद्र॑श॒नया॑ नि॒यूय॑। ऋ॒तस्य॑ वक्षि प॒थिभी॒ रजि॑ष्ठैः। होता॑ यक्षद्वन॒स्पति॑म॒भिहि। पि॒ष्टत॑मया॒ रभि॑ष्ठया रश॒नयाधि॑त। यत्रेन्द्राग्नि॒योश्छाग॑स्य ह॒विष॑ प्रि॒या धामा॑नि। यत्र॒ वन॒स्पते प्रि॒या पाथासि। यत्र॑ दे॒वाना॑माज्य॒पानां प्रि॒या धामा॑नि। यत्रा॒ग्नेर्‌होतु॑ प्रि॒या धामा॑नि। तत्रै॒तं प्र॒स्तुत्ये॑वोप॒स्तुत्ये॑ वो॒पाव॑स्रक्षत्। रभी॑यासमिव कृ॒त्वी॥२९॥

%3.6.11.4
कर॑दे॒वन्दे॒वो वन॒स्पति॑। जु॒षता ह॒विः। होत॒र्यज॑। पि॒प्री॒हि दे॒वा उ॑श॒तो य॑विष्ठ। वि॒द्वा ऋ॒तूर्\mbox{}ऋ॑तुपते यजे॒ह। ये दैव्या॑ ऋ॒त्विज॒स्तेभि॑रग्ने। त्व होतॄ॑णाम॒स्याय॑जिष्ठः। होता॑ यक्षद॒ग्नि स्वि॑ष्ट॒कृतम्। अया॑ड॒ग्निरि॑न्द्राग्नि॒योश्छाग॑स्य ह॒विष॑ प्रि॒या धामा॑नि। अया॒ड्वन॒स्पते प्रि॒या पाथासि। अयाड्दे॒वाना॑माज्य॒पानां प्रि॒या धामा॑नि। यक्ष॑द॒ग्नेर्‌होतु॑ प्रि॒या धामा॑नि। यक्ष॒त्स्वं म॑हि॒मानम्। आय॑जता॒मेज्या॒ इष॑। कृ॒णोतु॒ सो अ॑ध्व॒रा जा॒तवे॑दाः। जु॒षता ह॒विः। होत॒र्यज॑ ॥३०॥\anuvakamend[नू॒नमर्थं॑ कृ॒त्वी पाथासि स॒प्त च॑]

%3.6.12.1
उपो॑ ह॒ यद्वि॒दथं॑ वा॒जिनो॒ गूः। गी॒र्भिर्विप्रा॒ प्रम॑तिमि॒च्छमा॑नाः। अ॒र्वन्तो॒ न काष्ठा॒न्नक्ष॑माणाः। इ॒न्द्रा॒ग्नी जोहु॑वतो॒ नर॒स्ते। वन॑स्पते रश॒नया॑ऽभि॒धाय॑। पि॒ष्टत॑मया व॒युना॑नि वि॒द्वान्। वह॑ देव॒त्रा दि॑धिषो ह॒वीषि॑। प्र च॑दा॒तार॑म॒मृते॑षु वोचः। अ॒ग्नि स्वि॑ष्ट॒कृतम्। अया॑ड॒ग्निरि॑न्द्राग्नि॒योश्छग॑स्य ह॒विष॑ प्रि॒या धामा॑नि॥३१॥

%3.6.12.2
अया॒ड्वन॒स्पते प्रि॒या पाथासि। अयाड्दे॒वाना॑माज्य॒पानां प्रि॒या धामा॑नि। यक्ष॑द॒ग्नेर्‌होतु॑ प्रि॒या धामा॑नि। यक्ष॒त्स्वं म॑हि॒मानम्। आय॑जता॒मेज्या॒ इष॑। कृ॒णोतु॒ सो अ॑ध्व॒रा जा॒तवे॑दाः। जु॒षता ह॒विः। अग्ने॒ यद॒द्य वि॒शो अ॑ध्वरस्य होतः। पाव॑क शोचे॒ वेष्ट्व हि यज्वा। ऋ॒ता य॑जासि महि॒ना वियद्भूः। ह॒व्या व॑ह यविष्ठ॒ या ते॑ अ॒द्य॥३२॥\anuvakamend[धामा॑नि॒ भूरेकं च]

%3.6.13.1
दे॒वं ब॒र्॒हिः सु॑दे॒वन्दे॒वैः स्यात्सु॒वीरं॑ वी॒रैर्वस्तोर्वृ॒ज्येता॒क्तोः प्रभ्रि॑ये॒तात्य॒न्यान्रा॒या ब॒र्॒हिष्म॑तो मदेम वसु॒वने॑ वसु॒धेय॑स्य वेतु॒ यज॑। दे॒वीर्द्वार॑ सङ्घा॒ते वि॒ड्वीर्याम॑ञ्छिथि॒रा ध्रु॒वा दे॒वहू॑तौ व॒त्स ई॑मेना॒स्तरु॑ण॒ आमि॑मीयात्कुमा॒रो वा॒ नव॑जातो॒ मैना॒ अर्वा॑ रे॒णुक॑काट॒ पृण॑ग्वसु॒वने॑ वसु॒धेय॑स्य वियन्तु यज॑। दे॒वी उ॒षासा॒नक्ताऽद्या॒स्मिन्‌य॒ज्ञे प्र॑य॒त्य॑ह्वेता॒मपि॑ नू॒नन्दैवी॒र्विश॒ प्राया॑सिष्टा॒ सुप्री॑ते॒ सुधि॑ते वसु॒वने॑ वसु॒धेय॑स्य वीतां॒ यज॑। दे॒वी जोष्ट्री॒ वसु॑धिती॒ ययो॑र॒न्याऽघाद्द्वेषासि यू॒यव॒दान्याव॑क्ष॒द्वसु॒ वार्या॑णि॒ यज॑मानाय वसु॒वने॑ वसु॒धेय॑स्य वीतां॒ यज॑। दे॒वी ऊ॒र्जाहु॑ती॒ इष॒मूर्ज॑म॒न्याव॑क्ष॒त्सग्धि॒ सपी॑तिम॒न्या नवे॑न॒ पूर्व॒न्दय॑माना॒ स्याम॑ पुरा॒णेन॒ नव॒न्तामूर्ज॑मू॒र्जाहु॑ती ऊ॒र्जय॑माने अधातां वसु॒वने॑ वसु॒धेय॑स्य वीतां॒ यज॑। दे॒वा दैव्या॒ होता॑रा॒ नेष्टा॑रा॒ पोता॑रा ह॒ताघ॑शसावाभ॒रद्व॑सू वसु॒वने वसु॒धेय॑स्य वीतां॒ यज॑। दे॒वीस्ति॒स्रस्ति॒स्रो दे॒वीरिडा॒ सर॑स्वती॒ भार॑ती॒ द्यां भार॑त्यादि॒त्यैर॑स्पृक्ष॒त्सर॑स्वती॒म रु॒द्रैर्य॒ज्ञमा॑वीदि॒हैवेड॑या॒ वसु॑मत्या सध॒मादं॑ मदेम वसु॒वने॑ वसु॒धेय॑स्य वियन्तु॒ यज॑। दे॒वो नरा॒शस॑स्त्रिशी॒र्॒षा ष॑ड॒क्षः श॒तमिदे॑नशितिपृ॒ष्ठा आद॑धति स॒हस्र॑मीं॒ प्रव॑हन्ति मि॒त्रावरु॒णेद॑स्य हो॒त्रमर्\mbox{}ह॑तो॒ बृह॒स्पति॑ स्तो॒त्रम॒श्विनाऽऽध्व॑र्यवं वसु॒वने॑वसु॒धेयस्य॑ वेतु॒ यज॑। दे॒वो वन॒स्पति॑र्व॒र्॒षप्रा॑वा घृ॒तनि॑र्णि॒ग्द्यामग्रे॒णास्पृ॑क्ष॒दान्तरि॑क्षं॒ मध्ये॑नाप्राः पृथि॒वीमुप॑रेणादृहीद्वसु॒वने॑ वसु॒धेय॑स्य वेतु॒ यज॑। दे॒वं ब॒र्॒हिर्वारि॑तीनान्नि॒धेधा॑ऽसि॒ प्रच्यु॑तीना॒मप्र॑च्युतन्निकाम॒धर॑णं पुरुस्पा॒र्॒हं यश॑स्वदे॒ना ब॒र्॒हिषा॒ऽन्या ब॒र्॒हीष्य॒भि ष्या॑म वसु॒वने॑ वसु॒धेय॑स्य वेतु॒ यज॑। दे॒वो अ॒ग्निः स्वि॑ष्ट॒कृत्सु॒द्रवि॑णा म॒न्द्रः क॒विः स॒त्यम॑न्माऽऽय॒जी होता॒ होतु॑र्॒होतु॒राय॑जीया॒नग्ने॒ यान्दे॒वानया॒ड्या अपि॑प्रे॒र्ये ते॑ हो॒त्रे अम॑त्सत॒ ता स॑स॒नुषी॒ होत्रान्देवङ्ग॒मान्दि॒वि दे॒वेषु॑ य॒ज्ञमेर॑ये॒म स्वि॑ष्ट॒कृच्चाग्ने॒ होताऽभूर्वसु॒वने॑ वसु॒धेय॑स्य नमोवा॒के वीहि॒ यज॑ ॥३३॥\anuvakamend[यजैकं च]

%3.6.14.1
दे॒वं ब॒र्॒हिः। व॒सु॒वने॑ वसु॒धेय॑स्य वेतु। दे॒वीर्द्वार॑। व॒सु॒वने॑ वसु॒धेय॑स्य वियन्तु। दे॒वी उ॒षासा॒नक्ता। व॒सु॒वने॑ वसु॒धेय॑स्य वीताम्। दे॒वी जोष्ट्री। व॒सु॒वने॑ वसु॒धेय॑स्य वीताम्। दे॒वी ऊ॒र्जाहु॑ती। व॒सु॒वने॑ वसु॒धेयस्य॑ वीताम्॥३४॥

%3.6.14.2
दे॒वा दैव्या॒ होता॑रा। व॒सु॒वने॑ वसु॒धेय॑स्य वीताम्। दे॒वीस्ति॒स्रस्ति॒स्रो दे॒वीः। व॒सु॒वने॑ वसु॒धेय॑स्य वियन्तु। दे॒वो नरा॒शस॑। व॒सु॒वने॑ वसु॒धेय॑स्य वेतु। दे॒वो वन॒स्पति॑। व॒सु॒वने॑ वसु॒धेय॑स्य वेतु। दे॒वं ब॒र्॒हिर्वारि॑तीनाम्। व॒सु॒वने॑ वसु॒धेय॑स्य वेतु॥३५॥

%3.6.14.3
दे॒वो अ॒ग्निः स्वि॑ष्ट॒कृत्। सु॒द्रवि॑णा म॒न्द्रः क॒विः। स॒त्यम॑न्माय॒जी होता। होतु॑र्‌होतु॒राय॑जीयान्। अग्ने॒ यान्दे॒वानयाट्। या अपि॑प्रेः। ये ते॑ हो॒त्रे अम॑त्सत। ता स॑स॒नुषी॒ होत्रान्देवङ्ग॒माम्। दि॒वि दे॒वेषु॑ य॒ज्ञमेर॑ये॒मम्। स्वि॒ष्ट॒कृच्चाग्ने॒ होताऽभू। व॒सु॒वने॑ वसु॒धेय॑स्य नमोवा॒के वीहि॑॥३६॥\anuvakamend[वी॒तां॒ वे॒त्वभू॒रेकं च]

%3.6.15.1
अ॒ग्निम॒द्य होता॑रमवृणीता॒यं यज॑मान॒ पच॑न्प॒क्तीः पच॑न्पुरो॒डाशं॑ ब॒ध्नन्नि॑न्द्रा॒ग्निभ्या॒ञ्छाग सूप॒स्था अ॒द्य दे॒वो वन॒स्पति॑रभवदिन्द्रा॒ग्निभ्यां॒ छागे॒नाघ॑स्ता॒न्तं मे॑द॒स्तः प्रति॑पच॒ताग्र॑भीष्टा॒मवी॑वृधेतां पुरो॒डाशे॑न॒ त्वाम॒द्यर्\mbox{}ष॑ आर्\mbox{}षेय ऋषीणान्नपादवृणीता॒यं यज॑मानो ब॒हुभ्य॒ आ सङ्ग॑तेभ्य ए॒ष मे॑ दे॒वेषु॒ वसु॒ वार्या य॑क्ष्यत॒ इति॒ ता या दे॒वा दे॑व॒दाना॒न्यदु॒स्तान्य॑स्मा॒ आ च॒ शास्वा च॑ गुरस्वेषि॒तश्च॑ होत॒रसि॑ भद्र॒वाच्या॑य॒ प्रेषि॑तो॒ मानु॑षः सूक्तवा॒काय॑ सू॒क्ता ब्रू॑हि ॥३७॥\anuvakamend[अ॒ग्निम॒द्यैकम्]






\prashnaend{अ॒ञ्जन्ति॒ होता॑ यक्ष॒त्समि॑द्धो अ॒द्याग्निरजै॒द्दैव्या॑ जु॒षस्वा वृ॑त्रहणा गी॒र्भिस्त्व ह्याभ॑रत॒मुपो॑ह॒ यद्दे॒वं ब॒र्॒हिः सु॑दे॒वन्दे॒वं ब॒र्॒हिर॒ग्निम॒द्य पञ्च॑दश॥१५॥}{अ॒ञ्जन्त्य॒ग्निर्\mbox{}होता॑ नो गी॒र्भिरुपो॑ ह॒ यद्वि॒दथं॑ वा॒जिन॑ स॒प्तत्रिशत्॥३७॥}{अ॒ञ्जन्ति॑ सू॒क्ताब्रू॑हि॥}{हरि॑ ओम्॥}{इति श्रीकृष्णयजुर्वेदीयतैत्तिरीयब्राह्मणे तृतीयाष्टके षष्ठः प्रपाठकः समाप्तः॥}
\clearpage
