\sect{सप्तमः प्रश्नः}
\setcounter{anuvakam}{0}
\dnsub{तैत्तिरीयब्राह्मणे प्रथमाष्टके सप्तमः प्रपाठकः}

%सं॒व॒त्स॒        सं॒व॒त्स॒
%1.7.1.1
ए॒तद्ब्राह्मणान्ये॒व पञ्च॑ ह॒वीषि॑। अथेन्द्रा॑य॒ शुना॒सीरा॑य पुरो॒डाश॒न्द्वाद॑शकपालं॒ निर्व॑पति। सं॒व॒त्स॒रो वा इन्द्रा॒शुना॒सीर॑। सं॒व॒त्स॒रेणै॒वास्मा॒ अन्न॒मव॑ रुन्धे। वा॒य॒व्यं॑ पयो॑ भवति। वा॒युर्वै वृष्ट्यै प्रदापयि॒ता। स ए॒वास्मै॒ वृष्टिं॒ प्रदा॑पयति। सौ॒र्य॑ एक॑कपालो भवति। सूर्ये॑ण॒ वा अ॒मुष्मिँ॑ल्लो॒के वृष्टि॑र्धृ॒ता। स ए॒वास्मै॒ वृष्टिं॒ निय॑च्छति॥१॥

%1.7.1.2
द्वा॒द॒श॒ग॒व सीर॒न्दक्षि॑णा॒ समृ॑द्ध्यै। दे॒वा॒सु॒राः संय॑त्ता आसन्। ते दे॒वा अ॒ग्निम॑ब्रुवन्। त्वया॑ वी॒रेणासु॑रान॒भिभ॑वा॒मेति॑। सोऽब्रवीत्। त्रे॒धाऽहमा॒त्मानं॒ विक॑रिष्य॒ इति॑। स त्रे॒धाऽऽत्मानं॒ व्य॑कुरुत। अ॒ग्निन्तृती॑यम्। रु॒द्रन्तृती॑यम्। वरु॑ण॒न्तृती॑यम्॥२॥

%1.7.1.3
सोऽब्रवीत्। क इ॒दन्तु॒रीय॒मिति॑। अ॒हमितीन्द्रोऽब्रवीत्। सन्तु सृ॑जावहा॒ इति॑। तौ सम॑सृजेताम्। स इन्द्र॑स्तु॒रीय॑मभवत्। यदिन्द्र॑स्तु॒रीय॒मभ॑वत्। तदि॑न्द्रतुरी॒यस्येन्द्रतुरीय॒त्वम्। ततो॒ वै दे॒वा व्य॑जयन्त। यदि॑न्द्रतुरी॒यं नि॑रु॒प्यते॒ विजि॑त्यै॥३॥

%1.7.1.4
व॒हिनी॑ धे॒नुर्दक्षि॑णा। यद्व॒हिनी। तेनाग्ने॒यी। यद्गौः। तेन॑ रौ॒द्री। यद्धे॒नुः। तेनै॒न्द्री। यत्स्त्री स॒ती दा॒न्ता। तेन॑ वारु॒णी समृ॑द्ध्यै। प्र॒जाप॑तिर्य॒ज्ञम॑सृजत॥४॥

%1.7.1.5
त सृ॒ष्ट रक्षास्यजिघासन्। स ए॒ताः प्र॒जाप॑तिरा॒त्मनो॑ दे॒वता॒ निर॑मिमीत। ताभि॒र्वै स दि॒ग्भ्यो रक्षासि॒ प्राणु॑दत। यत्प॑ञ्चाव॒त्तीयं॑ जु॒होति॑। दि॒ग्भ्य ए॒व तद्यज॑मानो॒ रक्षासि॒ प्रणु॑दते। समू॑ढ॒ रक्ष॒ सन्द॑ग्ध॒ रक्ष॒ इत्या॑ह। रक्षास्ये॒व सन्द॑हति। अ॒ग्नये॑ रक्षो॒घ्ने स्वाहेत्या॑ह। दे॒वताभ्य ए॒व वि॑जिग्या॒नाभ्यो॑ भाग॒धेयं॑ करोति। प्र॒ष्टि॒वा॒ही रथो॒ दक्षि॑णा॒ समृ॑द्ध्यै॥५॥

%1.7.1.6
इन्द्रो॑ वृ॒त्र ह॒त्वा। असु॑रान्परा॒भाव्य॑। नमु॑चिमासु॒रन्नाल॑भत। तश॒च्या॑ऽगृह्णात्। तौ सम॑लभेताम्। सोऽस्माद॒भिशु॑नतरोऽभवत्। सोऽब्रवीत्। स॒न्धा सन्द॑धावहै। अथ॒ त्वाऽव॑ स्रक्ष्यामि। न मा॒ शुष्के॑ण॒ नार्द्रेण॑ हनः॥६॥

%1.7.1.7
न दिवा॒ न नक्त॒मिति॑। स ए॒तम॒पां फेन॑मसिञ्चत्। न वा ए॒ष शुष्को॒ नार्द्रो व्यु॑ष्टाऽऽसीत्। अनु॑दित॒ सूर्य॑। न वा ए॒तद्दिवा॒ न नक्तम्। तस्यै॒तस्मि॑ल्लोँ॒के। अ॒पां फेने॑न॒ शिर॒ उद॑वर्तयत्। तदे॑न॒मन्व॑वर्तत। मित्र॑द्रु॒गति॑ ॥७॥

%1.7.1.8
स ए॒तान॑पामा॒र्गान॑जनयत्। तान॑जुहोत्। तैर्वै स रक्षा॒स्यपा॑हत। यद॑पामार्गहो॒मो भव॑ति। रक्ष॑सा॒मप॑हत्यै। ए॒को॒ल्मु॒केन॑ यन्ति। तद्धि रक्ष॑सां भाग॒धेयम्। इ॒मान्दिशं॑ यन्ति। ए॒षा वै रक्ष॑सा॒न्दिक्। स्वाया॑मे॒व दि॒शि रक्षासि हन्ति॥८॥

%1.7.1.9
स्वकृ॑त॒ इरि॑णे जुहोति प्रद॒रे वा। ए॒तद्वै रक्ष॑सामा॒यतन॑म्। स्व ए॒वायत॑ने॒ रक्षासि हन्ति। प॒र्ण॒मये॑न स्रु॒वेण॑ जुहोति। ब्रह्म॒ वै प॒र्णः। ब्रह्म॑णै॒व रक्षासि हन्ति। दे॒वस्य॑ त्वा सवि॒तुः प्र॑स॒व इत्या॑ह। स॒वि॒तृप्र॑सूत ए॒व रक्षासि हन्ति। ह॒त रक्षोऽव॑धिष्म॒ रक्ष॒ इत्या॑ह। रक्ष॑सा॒ स्तृत्यै। यद्वस्ते॒ तद्दक्षि॑णा नि॒रव॑त्यै। अप्र॑तीक्ष॒माय॑न्ति। रक्ष॑साम॒न्तर्‌हि॑त्यै॥९॥\anuvakamend[य॒च्छ॒ति॒ वरु॑ण॒न्तृती॑यं॒ विजि॑त्या असृजत॒ समृ॑द्ध्यै हनो॒ मित्र॑द्रु॒गिति॑ हन्ति॒ स्तृत्यै॒ त्रीणि॑ च]

%1.7.2.1
धा॒त्रे पु॑रो॒डाश॒न्द्वाद॑शकपालं॒ निर्व॑पति। सं॒व॒त्स॒रो वै धा॒ता। सं॒व॒त्स॒रेणै॒वास्मै प्र॒जाः प्रज॑नयति। अन्वे॒वास्मा॒ अनु॑मतिर्मन्यते। रा॒ते रा॒का। प्र सि॑नीवा॒ली ज॑नयति। प्र॒जास्वे॒व प्रजा॑तासु कु॒ह्वा॑ वाच॑न्दधाति। मि॒थु॒नौ गावौ॒ दक्षि॑णा॒ समृ॑द्ध्यै। आ॒ग्ना॒वै॒ष्ण॒वमेका॑दशकपालं॒ निर्व॑पति। ऐ॒न्द्रा॒वै॒ष्ण॒वमेका॑दशकपालम्॥१०॥

%1.7.2.2
वै॒ष्ण॒वन्त्रि॑कपा॒लम्। वी॒र्यं॑ वा अ॒ग्निः। वी॒र्य॑मिन्द्र॑। वी॒र्यं॑ विष्णु॑। प्र॒जा ए॒व प्रजा॑ता वी॒र्ये प्रति॑ष्ठापयति। तस्मात्प्र॒जा वी॒र्या॑वतीः। वा॒म॒न ऋ॑ष॒भो व॒ही दक्षि॑णा। यद्व॒ही। तेनाग्ने॒यः। यदृ॑ष॒भः॥११॥

%1.7.2.3
तेनै॒न्द्रः। यद्वा॑म॒नः। तेन॑ वैष्ण॒वः समृ॑द्ध्यै। अ॒ग्नी॒षो॒मीय॒मेका॑दशकपालं॒ निर्व॑पति। इ॒न्द्रा॒सो॒मीय॒मेका॑दशकपालम्। सौ॒म्यञ्च॒रुम्। सोमो॒ वै रे॑तो॒धाः। अ॒ग्निः प्र॒जानां प्रजनयि॒ता। वृ॒द्धाना॒मिन्द्र॑ प्रदापयि॒ता। सोम॑ ए॒वास्मै॒ रेतो॒ दधा॑ति॥१२॥

%1.7.2.4
अ॒ग्निः प्र॒जां प्रज॑नयति। वृ॒द्धामिन्द्र॒ प्रय॑च्छति। ब॒भ्रुर्दक्षि॑णा॒ समृ॑द्ध्यै। सो॒मा॒पौ॒ष्णञ्च॒रुन्निर्व॑पति। ऐ॒न्द्रा॒पौ॒ष्णञ्च॒रुम्। सोमो॒ वै रे॑तो॒धाः। पू॒षा प॑शू॒नां प्र॑जनयि॒ता। वृ॒द्धाना॒मिन्द्र॑ प्रदापयि॒ता। सोम॑ ए॒वास्मै॒ रेतो॒ दधा॑ति। पू॒षा प॒शून्प्रज॑नयति॥१३॥

%1.7.2.5
वृ॒द्धानिन्द्र॒ प्रय॑च्छति। पौ॒ष्णश्च॒रुर्भ॑वति। इ॒यं वै पू॒षा। अ॒स्यामे॒व प्रति॑तिष्ठति। श्या॒मो दक्षि॑णा॒ समृ॑द्ध्यै। ब॒हु वै पुरु॑षो मे॒ध्यमुप॑गच्छति। वै॒श्वा॒न॒रन्द्वाद॑शकपालं॒ निर्व॑पति। सं॒व॒त्स॒रो वा अ॒ग्निर्वैश्वान॒रः। सं॒व॒त्स॒रेणै॒वैन स्वदयति। हिर॑ण्य॒न्दक्षि॑णा॥१४॥

%1.7.2.6
प॒वित्रं॒ वै हिर॑ण्यम्। पु॒नात्ये॒वैनम्। ब॒हु वै रा॑ज॒न्योऽनृ॑तं करोति। उप॑ जा॒म्यै हर॑ते। जि॒नाति॑ ब्राह्म॒णम्। वद॒त्यनृ॑तम्। अनृ॑ते॒ खलु॒ वै क्रि॒यमा॑णे॒ वरु॑णो गृह्णाति। वा॒रु॒णं य॑व॒मयं॑ च॒रुन्निर्व॑पति। व॒रु॒ण॒पा॒शादे॒वैनं॑ मुञ्चति। अश्वो॒ दक्षि॑णा। वा॒रु॒णो हि दे॒वत॒याऽश्व॒ समृ॑द्ध्यै॥१५॥\anuvakamend[ऐ॒न्द्रा॒वै॒ष्ण॒वमेका॑दशकपालं॒ यदृ॑ष॒भो दधा॑ति पू॒षा प॒शून्प्रज॑नयति॒ हिर॑ण्य॒न्दक्षि॑णा॒ दक्षि॒णैकं च]

%1.7.3.1
र॒त्निना॑मे॒तानि॑ ह॒वीषि॑ भवन्ति। ए॒ते वै रा॒ष्ट्रस्य॑ प्रदा॒तार॑। ए॒ते॑ऽपादा॒तार॑। य ए॒व रा॒ष्ट्रस्य॑ प्रदा॒तार॑। ये॑ऽपादा॒तार॑। त ए॒वास्मै॑ रा॒ष्ट्रं प्रय॑च्छन्ति। रा॒ष्ट्रमे॒व भ॑वति। यत्स॑मा॒हृत्य॑ नि॒र्वपेत्। अर॑त्निनः स्युः। य॒था॒य॒थन्निर्व॑पति रत्नि॒त्वाय॑॥१६॥

%1.7.3.2
यत्स॒द्यो नि॒र्वपेत्। याव॑ती॒मेके॑न ह॒विषा॒ऽऽशिष॑मव रु॒न्धे। ताव॑ती॒मव॑रुन्धीत। अ॒न्व॒हन्निर्व॑पति। भूय॑सीमे॒वाशिष॒मव॑ रुन्धे। भूय॑सो यज्ञक्र॒तूनुपै॑ति। बा॒र्॒ह॒स्प॒त्यञ्च॒रुन्निर्व॑पति ब्र॒ह्मणो॑ गृ॒हे। मु॒ख॒त ए॒वास्मै॒ ब्रह्म॒ सश्य॑ति। अथो॒ ब्रह्म॑न्ने॒व क्ष॒त्रम॒न्वार॑म्भयति। शि॒ति॒पृ॒ष्ठो दक्षि॑णा॒ समृ॑द्ध्यै॥१७॥

%1.7.3.3
ऐ॒न्द्रमेका॑दशकपाल राज॒न्य॑स्य गृ॒हे। इ॒न्द्रि॒यमे॒वाव॑ रुन्धे। ऋ॒ष॒भो दक्षि॑णा॒ समृ॑द्ध्यै। आ॒दि॒त्यञ्च॒रुं महि॑ष्यै गृ॒हे। इ॒यं वा अदि॑तिः। अ॒स्यामे॒व प्रति॑तिष्ठति। धे॒नुर्दक्षि॑णा॒ समृ॑द्ध्यै। भगा॑य च॒रुं वा॒वाता॑यै गृ॒हे। भग॑मे॒वास्मि॑न्दधाति। विचि॑त्तगर्भा पष्ठौ॒ही दक्षि॑णा॒ समृ॑द्ध्यै॥१८॥

%1.7.3.4
नै॒र्॒ऋ॒तञ्च॒रुं प॑रिवृ॒क्त्यै॑ गृ॒हे कृ॒ष्णानां व्रीही॒णान्न॒खनि॑र्भिन्नम्। पा॒प्मान॑मे॒व निर्\mbox{}ऋ॑तिन्नि॒रव॑दयते। कृ॒ष्णा कू॒टा दक्षि॑णा॒ समृ॑द्ध्यै। आ॒ग्ने॒यम॒ष्टाक॑पाल सेना॒न्यो॑ गृ॒हे। सेना॑मे॒वास्य॒ सश्य॑ति। हिर॑ण्य॒न्दक्षि॑णा॒ समृ॑द्ध्यै। वा॒रु॒णन्दश॑कपाल सू॒तस्य॑ गृ॒हे। व॒रु॒ण॒स॒वमे॒वाव॑ रुन्धे। म॒हानि॑रष्टो॒ दक्षि॑णा॒ समृ॑द्ध्यै। मा॒रु॒त स॒प्तक॑पालङ्ग्राम॒ण्यो॑ गृ॒हे॥१९॥

%1.7.3.5
अन्नं॒ वै म॒रुत॑। अन्न॑मे॒वाव॑ रुन्धे। पृश्ञि॒र्दक्षि॑णा॒ समृ॑द्ध्यै। सा॒वि॒त्रन्द्वाद॑शकपालङ्क्ष॒त्तुर्गृ॒हे प्रसूत्यै। उ॒प॒ध्व॒स्तो दक्षि॑णा॒ समृ॑द्ध्यै। आ॒श्वि॒नन्द्वि॑कपा॒ल स॑ङ्ग्रही॒तुर्गृ॒हे। अ॒श्विनौ॒ वै दे॒वानां भि॒षजौ। ताभ्या॑मे॒वास्मै॑ भेष॒जं क॑रोति। स॒वा॒त्यौ॑ दक्षि॑णा॒ समृ॑द्ध्यै। पौ॒ष्णञ्च॒रुं भा॑गदु॒घस्य॑ गृ॒हे॥२०॥

%1.7.3.6
अन्नं॒ वै पू॒षा। अन्न॑मे॒वाव॑ रुन्धे। श्या॒मो दक्षि॑णा॒ समृ॑द्ध्यै। रौ॒द्रङ्गा॑वीधु॒कञ्च॒रुम॑क्षावा॒पस्य॑ गृ॒हे। अ॒न्त॒त ए॒व रु॒द्रन्नि॒रव॑दयते। श॒बल॒ उद्वा॑रो॒ दक्षि॑णा॒ समृ॑द्ध्यै। द्वाद॑शै॒तानि॑ ह॒वीषि॑ भवन्ति। द्वाद॑श॒ मासा संवत्स॒रः। सं॒व॒त्स॒रेणै॒वास्मै॑ रा॒ष्ट्रमव॑रुन्धे। रा॒ष्ट्रमे॒व भ॑वति॥२१॥

%1.7.3.7
यन्न प्र॑ति नि॒र्वपेत्। र॒त्निन॑ आ॒शिषोऽव॑रुन्धीरन्। प्रति॒निर्व॑पति। इन्द्रा॑य सु॒त्राम्णे॑ पुरो॒डाश॒मेका॑दशकपालम्। इन्द्रा॑याहो॒मुचे। आ॒शिष॑ ए॒वाव॑रुन्धे। अ॒यन्नो॒ राजा॑ वृत्र॒हा राजा॑ भू॒त्वा वृ॒त्रं व॑ध्या॒दित्या॑ह। आ॒शिष॑मे॒वैतामा शास्ते। मै॒त्रा॒बा॒र्॒ह॒स्प॒त्यं भ॑वति। श्वे॒तायै श्वे॒तव॑त्सायै दु॒ग्धे॥२२॥

%1.7.3.8
बा॒र्॒ह॒स्प॒त्ये मै॒त्रमपि॑ दधाति। ब्रह्म॑ चै॒वास्मै क्ष॒त्रं च॑ स॒मीची॑ दधाति। अथो॒ ब्रह्म॑न्ने॒व क्ष॒त्रं प्रति॑ष्ठापयति। बा॒र्॒ह॒स्प॒त्येन॒ पूर्वे॑ण॒ प्रच॑रति। मु॒ख॒त ए॒वास्मै॒ ब्रह्म॒ सश्य॑ति। अथो॒ ब्रह्म॑न्ने॒व क्ष॒त्रम॒न्वार॑म्भयति। स्व॒यं॒ कृ॒ता वेदि॑र्भवति। स्व॒य॒न्दि॒नं ब॒र्॒हिः। स्व॒यं॒ कृ॒त इ॒ध्मः। अन॑भिजितस्या॒भिजि॑त्यै। तस्मा॒द्राज्ञा॒मर॑ण्यम॒भिजि॑तम्। सैव श्वे॒ता श्वे॒तव॑त्सा॒ दक्षि॑णा॒ समृ॑द्ध्यै॥२३॥\anuvakamend[र॒त्नि॒त्वाय॒ समृ॑द्ध्यै पष्ठौ॒ही दक्षि॑णा॒ समृ॑द्ध्यै ग्राम॒ण्यो॑ गृ॒हे भा॑गदु॒घस्य॑ गृ॒हे भ॑वति दु॒ग्धे॑ऽभिजि॑त्यै॒ द्वे च॑]
 
%1.7.4.1
दे॒व॒सु॒वामे॒तानि॑ ह॒वीषि॑ भवन्ति। ए॒ताव॑न्तो॒ वै दे॒वाना स॒वाः। त ए॒वास्मै॑ स॒वान्प्रय॑च्छन्ति। त ए॑न सुवन्ते। अ॒ग्निरे॒वैनं॑ गृ॒हप॑तीना सुवते। सोमो॒ वन॒स्पती॑नाम्। रु॒द्रः प॑शू॒नाम्। बृह॒स्पति॑र्वा॒चाम्। इन्द्रो ज्ये॒ष्ठानाम्। मि॒त्रः स॒त्यानाम्॥२४॥

%1.7.4.2
वरु॑णो॒ धर्म॑पतीनाम्। ए॒तदे॒व सर्वं॑ भवति। स॒वि॒ता त्वा प्रस॒वाना सुवता॒मिति॒ हस्त॑ङ्गृह्णाति॒ प्रसूत्यै। ये दे॑वा देव॒ सुव॒ स्थेत्या॑ह। य॒था॒य॒जुरे॒वैतत्। म॒ह॒ते क्ष॒त्राय॑ मह॒त आधि॑पत्याय मह॒ते जान॑राज्या॒येत्या॑ह। आ॒शिष॑मे॒वैतामा शास्ते। ए॒ष वो॑ भरता॒ राजा॒ सोमो॒ऽस्माकं॑ ब्राह्म॒णाना॒ राजेत्या॑ह। तस्मा॒त्सोम॑राजानो ब्राह्म॒णाः। प्रति॒ त्यन्नाम॑ रा॒ज्यम॑धा॒यीत्या॑ह॥२५॥

%1.7.4.3
रा॒ज्यमे॒वास्मि॒न्प्रति॑दधाति। स्वान्त॒नुवं॒ वरु॑णो अशिश्रे॒दित्या॑ह। व॒रु॒ण॒स॒वमे॒वाव॑रुन्धे। शुचेर्मि॒त्रस्य॒ व्रत्या॑ अभू॒मेत्या॑ह। शुचि॑मे॒वैनं॒ व्रत्यं॑ करोति। अम॑न्महि मह॒त ऋ॒तस्य॒ नामेत्या॑ह। म॒नु॒त ए॒वैनम्। सर्वे॒ व्राता॒ वरु॑णस्याभूव॒न्नित्या॑ह। सर्व॑व्रातमे॒वैनं॑ करोति। वि मि॒त्र एवै॒ररा॑तिमतारी॒दित्या॑ह॥२६॥

%1.7.4.4
अरा॑तिमै॒वैन॑न्तारयति। असू॑षुदन्त य॒ज्ञिया॑ ऋ॒तेनेत्या॑ह। स्व॒दय॑त्ये॒वैनम्। व्यु॑ त्रि॒तो ज॑रि॒माण॑न्न आन॒डित्या॑ह। आयु॑रे॒वास्मि॑न्दधाति। द्वाभ्यां॒ विमृ॑ष्टे। द्वि॒पाद्यज॑मान॒ प्रति॑ष्ठित्यै। अ॒ग्नी॒षो॒मीय॑स्य॒ चैका॑दशकपालस्य देवसु॒वां च॑ ह॒विषा॑म॒ग्नये स्विष्ट॒कृते॑ स॒मव॑द्यति। दे॒वता॑भिरे॒वैन॑मुभ॒यत॒ परि॑गृह्णाति। वि॒ष्णु॒क्र॒मान्क्र॑मते। विष्णु॑रे॒व भू॒त्वेमाल्लोँ॒कान॒भिज॑यति॥२७॥\anuvakamend[स॒त्याना॑मधा॒यीत्या॑हातारी॒दित्या॑ह क्रमत॒ एकं च]

%1.7.5.1
अ॒र्थेत॒ स्थेति॑ जुहोति। आहु॑त्यै॒वैना॑ नि॒ष्क्रीय॑ गृह्णाति। अथो॑ ह॒विष्कृ॑तानामे॒वाभिघृ॑तानाङ्गृह्णाति। वह॑न्तीनाङ्गृह्णाति। ए॒ता वा अ॒पा रा॒ष्ट्रम्। रा॒ष्ट्रमे॒वास्मै॑ गृह्णाति। अथो॒ श्रिय॑मे॒वैन॑म॒भिव॑हन्ति। अ॒पां पति॑र॒सीत्या॑ह। मि॒थु॒नमे॒वाक॑। वृषाऽस्यू॒र्मिरित्या॑ह॥२८॥

%1.7.5.2
ऊ॒र्मि॒मन्त॑मे॒वैनं॑ करोति। वृ॒ष॒से॒नो॑ऽसीत्या॑ह। सेना॑मे॒वास्य॒ सश्य॑ति। व्र॒ज॒क्षित॒ स्थेत्या॑ह। ए॒ता वा अ॒पां विश॑। विश॑मे॒वास्मै॒ पर्यू॑हति। म॒रुता॒मोज॒ स्थेत्या॑ह। अन्नं॒ वै म॒रुत॑। अन्न॑मे॒वाव॑रुन्धे। सूर्य॑वर्चस॒ स्थेत्या॑ह॥२९॥

%1.7.5.3
रा॒ष्ट्रमे॒व व॑र्च॒स्व्य॑कः। सूर्य॑त्वचस॒ स्थेत्या॑ह। स॒त्यं वा ए॒तत्। यद्वर्\mbox{}ष॑ति। अनृ॑तं॒ यदा॒तप॑ति॒ वर्\mbox{}ष॑ति। स॒त्या॒नृ॒ते ए॒वाव॑रुन्धे। नैन सत्यानृ॒ते उ॑दि॒ते हिस्तः। य ए॒वं वेद॑। मान्दा॒ स्थेत्या॑ह। रा॒ष्ट्रमे॒व ब्र॑ह्मवर्च॒स्य॑कः॥३०॥

%1.7.5.4
वाशा॒ स्थेत्या॑ह। रा॒ष्ट्रमे॒व व॒श्य॑कः। शक्व॑री॒ स्थेत्या॑ह। प॒शवो॒ वै शक्व॑रीः। प॒शूने॒वाव॑रुन्धे। वि॒श्व॒भृत॒ स्थेत्या॑ह। रा॒ष्ट्रमे॒व प॑य॒स्व्य॑कः। ज॒न॒भृत॒ स्थेत्या॑ह। रा॒ष्ट्रमे॒वेन्द्रि॑या॒व्य॑कः। अ॒ग्नेस्ते॑ज॒स्या स्थेत्या॑ह ॥३१॥

%1.7.5.5
रा॒ष्ट्रमे॒व ते॑ज॒स्व्य॑कः। अ॒पामोष॑धीना॒ रस॒ स्थेत्या॑ह। रा॒ष्ट्रमे॒व म॑ध॒व्य॑मकः। सा॒र॒स्व॒तङ्ग्रह॑ङ्गृह्णाति। ए॒षा वा अ॒पाम्पृ॒ष्ठम्। यत्सर॑स्वती। पृ॒ष्ठमे॒वैन समा॒नानां करोति। षो॒ड॒शभि॑र्गृह्णाति। षोड॑शकलो॒ वै पुरु॑षः। यावा॑ने॒व पुरु॑षः। तस्मि॑न्वी॒र्यं॑ दधाति। षो॒ड॒शभि॑र्जु॒होति॑ षोड॒शभि॑र्गृह्णाति। द्वात्रिश॒त्संप॑द्यन्ते। द्वात्रिशदक्षराऽनु॒ष्टुक्। वाग॑नु॒ष्टुप्सर्वा॑णि॒ छन्दासि। वा॒चैवैन॒ सर्वे॑भि॒श्छन्दो॑भिर॒भिषि॑ञ्चति॥३२॥\anuvakamend[ऊ॒र्मिरित्या॑ह॒ सूर्य॑वर्चस॒ स्थेत्या॑ह ब्रह्मवर्च॒स्य॑कस्तेज॒स्या स्थेत्या॑है॒व पुरु॑ष॒ष्षट् च॑]

%1.7.6.1
देवी॑राप॒ सं मधु॑मती॒र्मधु॑मतीभिः सृज्यध्व॒मित्या॑ह। ब्रह्म॑णै॒वैना॒ स सृ॑जति। अना॑धृष्टाः सीद॒तेत्या॑ह। ब्रह्म॑णै॒वैना सादयति। अ॒न्त॒रा होतु॑श्च॒ धिष्णि॑यं ब्राह्मणाच्छ॒सिन॑श्च सादयति। आ॒ग्ने॒यो वै होता। ऐ॒न्द्रो ब्राह्मणाच्छ॒सी। तेज॑सा चै॒वेन्द्रि॒येण॑ चोभ॒यतो॑ रा॒ष्ट्रं परि॑गृह्णाति। हिर॑ण्ये॒नोत्पु॑नाति। आहु॑त्यै॒ हि प॒वित्राभ्यामुत्पु॒नन्ति॒ व्यावृ॑त्त्यै॥३३॥

%1.7.6.2
श॒तमा॑नं भवति। श॒तायु॒ पुरु॑षः श॒तेन्द्रि॑यः। आयु॑ष्ये॒वेन्द्रि॒ये प्रति॑तिष्ठति। अनि॑भृष्टम॒सीत्या॑ह। अनि॑भृष्ट॒ ह्ये॑तत्। वा॒चो बन्धु॒रित्या॑ह। वा॒चो ह्ये॑ष बन्धु॑। त॒पो॒जा इत्या॑ह। त॒पो॒जा ह्ये॑तत्। सोम॑स्य दा॒त्रम॒सीत्या॑ह॥३४॥

%1.7.6.3
सोम॑स्य॒ ह्ये॑तद्दा॒त्रम्। शु॒क्रा व॑ शु॒क्रेणोत्पु॑ना॒मीत्या॑ह। शु॒क्रा ह्याप॑। शु॒क्र हिर॑ण्यम्। च॒न्द्राश्च॒न्द्रेणेत्या॑ह। च॒न्द्रा ह्याप॑। च॒न्द्र हिर॑ण्यम्। अ॒मृता॑ अ॒मृते॒नेत्या॑ह। अ॒मृता॒ ह्याप॑। अ॒मृत॒ हिर॑ण्यम्॥३५॥

%1.7.6.4
स्वाहा॑ राज॒सूया॒येत्या॑ह। रा॒ज॒सूया॑य॒ ह्ये॑ना उत्पु॒नाति॑। स॒ध॒मादो द्यु॒म्निनी॒रूर्ज॑ ए॒ता इति॑ वारु॒ण्यर्चा गृ॑ह्णाति। व॒रु॒ण॒स॒वमे॒वाव॑रुन्धे। एक॑या गृह्णाति। ए॒क॒धैव यज॑माने वी॒र्यं॑ दधाति। क्ष॒त्रस्योल्ब॑मसि क्ष॒त्रस्य॒ योनि॑र॒सीति॑ ता॒र्प्यञ्चो॒ष्णीषं॑ च॒ प्रय॑च्छति सयोनि॒त्वाय॑। एक॑शतेन दर्भपुञ्जी॒लैः प॑वयति। श॒तायु॒र्वै पुरु॑षः श॒तवीर्यः। आ॒त्मैक॑श॒तः॥३६॥

%1.7.6.5
यावा॑ने॒व पुरु॑षः। तस्मि॑न्वी॒र्यं॑ दधाति। दध्या॑शयति। इ॒न्द्रि॒यमे॒वाव॑ रुन्धे। उ॒दु॒म्बर॑माशयति। अ॒न्नाद्य॒स्याव॑रुद्ध्यै। शष्पाण्याशयति। सुरा॑बलिमे॒वैनं॑ करोति। आ॒विद॑ ए॒ता भ॑वन्ति। आ॒विद॑मे॒वैन॑ङ्गमयन्ति॥३७॥

%1.7.6.6
अ॒ग्निरे॒वैन॒ङ्गार्\mbox{}ह॑पत्येनावति। इन्द्र॑ इन्द्रि॒येण॑। पू॒षा प॒शुभि॑। मि॒त्रावरु॑णौ प्राणापा॒नाभ्याम्। इन्द्रो॑ वृ॒त्राय॒ वज्र॒मुद॑यच्छत्। स दिव॑मलिखत्। सोऽर्य॒म्णः पन्था॑ अभवत्। स आवि॑न्ने॒ द्यावा॑पृथि॒वी धृ॒तव्र॑ते॒ इति॒ द्यावा॑पृथि॒वी उपा॑धावत्। स आ॒भ्यामे॒व प्रसू॑त॒ इन्द्रो॑ वृ॒त्राय॒ वज्रं॒ प्राह॑रत्। आवि॑न्ने॒ द्यावा॑पृथि॒वी धृ॒तव्र॑ते॒ इति॒ यदाह॑॥३८॥

%1.7.6.7
आ॒भ्यामे॒व प्रसू॑तो॒ यज॑मानो॒ वज्रं॒ भ्रातृ॑व्याय॒ प्रह॑रति। आवि॑न्ना दे॒व्यदि॑तिर्विश्वरू॒पीत्या॑ह। इ॒यं वै दे॒व्यदि॑तिर्विश्वरू॒पी। अ॒स्यामे॒व प्रति॑ तिष्ठति। आवि॑न्नो॒ऽयम॒सावा॑मुष्याय॒णोऽस्यां वि॒श्य॑स्मिन्रा॒ष्ट्र इत्या॑ह। वि॒शैवैन रा॒ष्ट्रेण॒ सम॑र्धयति। म॒ह॒ते क्ष॒त्राय॑ मह॒त आधि॑पत्याय मह॒ते जान॑राज्या॒येत्या॑ह। आ॒शिष॑मे॒वैतामा शास्ते। ए॒ष वो॑ भरता॒ राजा॒ सोमो॒ऽस्माकं॑ ब्राह्म॒णाना॒ राजेत्या॑ह। तस्मा॒त्सोम॑राजानो ब्राह्म॒णाः॥३९॥

%1.7.6.8
इन्द्र॑स्य॒ वज्रो॑ऽसि॒ वार्त्र॑घ्न॒ इति॒ धनु॒ प्रय॑च्छति॒ विजि॑त्यै। श॒त्रु॒बाध॑ना॒ स्थेतीषून्॑। शत्रू॑ने॒वास्य॑ बाधन्ते। पा॒त मा प्र॒त्यञ्चं॑ पा॒त मा॑ ति॒र्यञ्च॑म॒न्वञ्चं॑ मा पा॒तेत्या॑ह। ति॒स्रो वै श॑र॒व्या। प्र॒तीची॑ ति॒रश्च्य॒नूची। ताभ्य॑ ए॒वैनं॑ पान्ति। दि॒ग्भ्यो मा॑ पा॒तेत्या॑ह। दि॒ग्भ्य ए॒वैनं॑ पान्ति। विश्वाभ्यो मा ना॒ष्ट्राभ्य॑ पा॒तेत्या॑ह। अप॑रिमितादे॒वैनं॑ पान्ति। हिर॑ण्यवर्णावु॒षसां विरो॒क इति॑ त्रि॒ष्टुभा॑ बा॒हू उद्गृ॑ह्णाति। इ॒न्द्रि॒यं वै वी॒र्य॑न्त्रि॒ष्टुक्। इ॒न्द्रि॒यमे॒व वी॒र्य॑मु॒परि॑ष्टादा॒त्मन्ध॑त्ते॥४०॥\anuvakamend[व्यावृ॑त्त्यै दा॒त्रम॒सीत्या॑हा॒मृत॒ हिर॑ण्यमेकश॒तो ग॑मय॒न्त्याह॑ ब्राह्म॒णा ना॒ष्ट्राभ्य॑ पा॒तेत्या॑ह च॒त्वारि॑ च]

%1.7.7.1
दिशो॒ व्यास्था॑पयति। दि॒शाम॒भिजि॑त्त्यै। यद॑नु प्र॒क्रामेत्। अ॒भि दिशो॑ जयेत्। उत्तु माद्येत्। मन॒साऽनु॒ प्रक्रा॑मति। अ॒भि दिशो॑ जयति। नोन्माद्यति। स॒मिध॒मा ति॒ष्ठेत्या॑ह। तेज॑ ए॒वाव॑रुन्धे॥४१॥

%1.7.7.2
उ॒ग्रामा ति॒ष्ठेत्या॑ह। इ॒न्द्रि॒यमे॒वाव॑रुन्धे। वि॒राज॒माति॒ष्ठेत्या॑ह। अ॒न्नाद्य॑मे॒वाव॑रुन्धे। उदी॑ची॒मा ति॒ष्ठेत्या॑ह। प॒शूने॒वाव॑रुन्धे। ऊ॒र्ध्वामाति॒ष्ठेत्या॑ह। सु॒व॒र्गमे॒व लो॒कम॒भिज॑यति। अनूज्जि॑हीते। सु॒व॒र्गस्य॑ लो॒कस्य॒ सम॑ष्ट्यै॥४२॥

%1.7.7.3
मा॒रु॒त ए॒ष भ॑वति। अन्नं॒ वै म॒रुत॑। अन्न॑मे॒वाव॑रुन्धे। एक॑विशतिकपालो भवति॒ प्रति॑ष्ठित्यै। यो॑ऽरण्येऽनुवा॒क्यो॑ ग॒णः। तं म॑ध्य॒त उप॑दधाति। ग्रा॒म्यैरे॒व प॒शुभि॑रार॒ण्यान्प॒शून्परि॑ गृह्णाति। तस्माद्ग्रा॒म्यैः प॒शुभि॑रार॒ण्याः प॒शव॒ परि॑गृहीताः। पृथि॑र्वै॒न्यः। अ॒भ्य॑षिच्यत॥४३॥

%1.7.7.4
स रा॒ष्ट्रन्नाभ॑वत्। स ए॒तानि॑ पा॒र्थान्य॑पश्यत्। तान्य॑जुहोत्। तैर्वै स रा॒ष्ट्रम॑भवत्। यत्पा॒र्थानि॑ जु॒होति॑। रा॒ष्ट्रमे॒व भ॑वति। बा॒र्॒ह॒स्प॒त्यं पूर्वे॑षामुत्त॒मं भ॑वति। ऐ॒न्द्रमुत्त॑रेषां प्रथ॒मम्। ब्रह्म॑ चै॒वास्मै क्ष॒त्रं च॑ स॒मीची॑ दधाति। अथो॒ ब्रह्म॑न्ने॒व क्ष॒त्रं प्रति॑ष्ठापयति॥४४॥

%1.7.7.5
षट्पु॒रस्ता॑दभिषे॒कस्य॑ जुहोति। षडु॒परि॑ष्टात्। द्वाद॑श॒ संप॑द्यन्ते। द्वाद॑श॒ मासा संवत्स॒रः। सं॒व॒त्स॒रः खलु॒ वै दै॒वानां॒ पूः। दे॒वाना॑मे॒व पुरं॑ मध्य॒तो व्यव॑सर्पति। तस्य॒ न कुत॑श्च॒नोपाव्या॒धो भ॑वति। भू॒ताना॒मवेष्टीर्जुहोति। अत्रात्र॒ वै मृ॒त्युर्जा॑यते। यत्र॑यत्रै॒व मृ॒त्युर्जाय॑ते। तत॑ ए॒वैन॒मव॑यजते। तस्माद्राज॒सूये॑नेजा॒नो नाभिच॑रित॒वै। प्र॒त्यगे॑नमभिचा॒रः स्तृ॑णुते॥४५॥\anuvakamend[रु॒न्धे॒ सम॑ष्ट्या असिच्यत स्थापयति॒ जाय॑ते॒ पञ्च॑ च]

%1.7.8.1
सोम॑स्य॒ त्विषि॑रसि॒ तवे॑व मे॒ त्विषि॑र्भूया॒दिति॑ शार्दूलच॒र्मोप॑स्तृणाति। यैव सोमे॒ त्विषि॑। या शार्दू॒ले। तामे॒वाव॑रुन्धे। मृ॒त्योर्वा ए॒ष वर्ण॑। यच्छार्दू॒लः। अ॒मृत॒ हिर॑ण्यम्। अ॒मृत॑मसि मृ॒त्योर्मा॑ पा॒हीति॒ हिर॑ण्य॒मुपास्यति। अ॒मृत॑मे॒व मृ॒त्योर॒न्तर्ध॑त्ते। श॒तमा॑नं भवति॥४६॥

%1.7.8.2
श॒तायु॒ पुरु॑षः श॒तेन्द्रि॑यः। आयु॑ष्ये॒वेन्द्रि॒ये प्रति॑तिष्ठति। दि॒द्योन्मा॑ पा॒हीत्यु॒परि॑ष्टा॒दधि॒ निद॑धाति। उ॒भ॒यत॑ ए॒वास्मै॒ शर्म॑ दधाति। अवेष्टा दन्द॒शूका॒ इति॑ क्ली॒ब सीसे॑न विध्यति। द॒न्द॒शूका॑ने॒वाव॑यजते। तस्मात्क्ली॒बन्द॑न्द॒शूका॒ दशु॑काः। निर॑स्त॒न्नमु॑चे॒ शिर॒ इति॑ लोहिताय॒सन्निर॑स्यति। पा॒प्मान॑मे॒व नमु॑चिन्नि॒रव॑दयते। प्रा॒णा आ॒त्मन॒ पूर्वे॑ऽभि॒षिच्या॒ इत्या॑हुः॥४७॥

%1.7.8.3
सोमो॒ राजा॒ वरु॑णः। दै॒वा ध॑र्म॒सुव॑श्च॒ ये। ते ते॒ वाच सुवन्ता॒न्ते ते प्रा॒ण सु॑वन्ता॒मित्या॑ह। प्रा॒णाने॒वात्मन॒ पूर्वा॑न॒भिषि॑ञ्चति। यद्ब्रू॒यात्। अ॒ग्नेस्त्वा॒ तेज॑सा॒ऽभिषि॑ञ्चा॒मीति॑। ते॒ज॒स्व्ये॑व स्यात्। दु॒श्चर्मा॒ तु भ॑वेत्। सोम॑स्य त्वा द्यु॒म्नेना॒भिषि॑ञ्चा॒मीत्या॑ह। सौ॒म्यो वै दे॒वत॑या॒ पुरु॑षः॥४८॥

%1.7.8.4
स्वयै॒वैनं॑ दे॒वत॑या॒ऽभिषि॑ञ्चति। अ॒ग्नेस्तेज॒सेत्या॑ह। तेज॑ ए॒वास्मि॑न्दधाति। सूर्य॑स्य॒ वर्च॒सेत्या॑ह। वर्च॑ ए॒वास्मि॑न्दधाति। इन्द्र॑स्येन्द्रि॒येणेत्या॑ह। इ॒न्द्रि॒यमे॒वास्मि॑न्दधाति। मि॒त्रावरु॑णयोर्वी॒र्ये॑णेत्या॑ह। वी॒र्य॑मे॒वास्मि॑न्दधाति। म॒रुता॒मोज॒सेत्या॑ह॥४९॥

%1.7.8.5
ओज॑ ए॒वास्मि॑न्दधाति। क्ष॒त्राणाङ्क्ष॒त्रप॑तिर॒सीत्या॑ह। क्ष॒त्राणा॑मे॒वैनं॑ क्ष॒त्रप॑तिं करोति। अति॑ दि॒वस्पा॒हीत्या॑ह। अत्य॒न्यान्पा॒हीति॒ वावैतदा॑ह। स॒माव॑वृत्रन्नध॒रागुदी॑ची॒रित्या॑ह। रा॒ष्ट्रमे॒वास्मि॑न्ध्रु॒वम॑कः। उ॒च्छेष॑णेन जुहोति। उ॒च्छेष॑णभागो॒ वै रु॒द्रः। भा॒ग॒धेये॑नै॒व रु॒द्रन्नि॒रव॑दयते॥५०॥

%1.7.8.6
उद॑ङ्प॒रेत्याग्नीद्ध्रे जुहोति। ए॒षा वै रु॒द्रस्य॒ दिक्। स्वाया॑मे॒व दि॒शि रु॒द्रन्नि॒रव॑दयते। रुद्र॒ यत्ते॒ क्रयी॒ पर॒न्नामेत्या॑ह। यद्वा अ॑स्य॒ क्रयी॒ पर॒न्नाम॑। तेन॒ वा ए॒ष हि॑नस्ति। य हि॒नस्ति॑। तेनै॒वैन स॒ह श॑मयति। तस्मै॑ हु॒तम॑सि य॒मेष्ट॑म॒सीत्या॑ह। य॒मादे॒वास्य॑ मृ॒त्युमव॑यजते॥५१॥

%1.7.8.7
प्रजा॑पते॒ न त्वदे॒तान्य॒न्य इति॒ तस्यै॑ गृ॒हे जु॑हुयात्। याङ्का॒मये॑त रा॒ष्ट्रम॑स्यै प्र॒जा स्या॒दिति॑। रा॒ष्ट्रमे॒वास्यै प्र॒जा भ॑वति। प॒र्ण॒मये॑नाध्व॒र्युर॒भिषि॑ञ्चति। ब्र॒ह्म॒व॒र्च॒समे॒वास्मि॒न्त्विषि॑न्दधाति। औदु॑म्बरेण राज॒न्य॑। ऊर्ज॑मे॒वास्मि॑न्न॒न्नाद्य॑न्दधाति। आश्व॑त्थेन॒ वैश्य॑। विश॑मे॒वास्मि॒न्पुष्टि॑न्दधाति। नैय॑ग्रोधेन॒ जन्य॑। मि॒त्राण्ये॒वास्मै॑ कल्पयति। अथो॒ प्रति॑ष्ठित्यै॥५२॥\anuvakamend[भ॒व॒त्या॒हु॒ पुरु॑ष॒ ओज॒सेत्या॑ह नि॒रव॑दयते यजते॒ जन्यो॒ द्वे च॑]

%1.7.9.1
इन्द्र॑स्य॒ वज्रो॑ऽसि॒ वार्त्र॑घ्न॒ इति॒ रथ॑मु॒पाव॑हरति॒ विजि॑त्यै। मि॒त्रावरु॑णयोस्त्वा प्रशा॒स्त्रोः प्र॒शिषा॑ युन॒ज्मीत्या॑ह। ब्रह्म॑णै॒वैनं॑ दे॒वताभ्याय्युँनक्ति। प्र॒ष्टि॒वा॒हिन॑य्युँनक्ति। प्र॒ष्टि॒वा॒ही वै दे॑वर॒थः। दे॒व॒र॒थमे॒वास्मै॑ युनक्ति। त्रयोऽश्वा॑ भवन्ति। रथ॑श्चतु॒र्थः। द्वौ स॑व्येष्ठसार॒थी। षट्त्सं प॑द्यन्ते॥५३॥

%1.7.9.2
षड्वा ऋ॒तव॑। ऋ॒तुभि॑रे॒वैन॑य्युँनक्ति। वि॒ष्णु॒क्र॒मान्क्र॑मते। विष्णु॑रे॒व भू॒त्वेमाल्लोँ॒कान॒भिज॑यति। यः क्ष॒त्रिय॒ प्रति॑हितः। सोऽन्वार॑भते। रा॒ष्ट्रमे॒व भ॑वति। त्रि॒ष्टुभा॒ऽन्वार॑भते। इ॒न्द्रि॒यं वै त्रि॒ष्टुक्। इ॒न्द्रि॒यमे॒व यज॑माने दधाति॥५४॥

%1.7.9.3
म॒रुतां प्रस॒वे जे॑ष॒मित्या॑ह। म॒रुद्भि॑रे॒व प्रसू॑त॒ उज्ज॑यति। आ॒प्तं मन॒ इत्या॑ह। यदे॒व मन॒सैप्सीत्। तदा॑पत्। रा॒ज॒न्यं॑ जिनाति। अनाक्रान्त ए॒वाक्र॑मते। वि वा ए॒ष इ॑न्द्रि॒येण॑ वी॒र्ये॑णर्ध्यते। यो रा॑ज॒न्यं॑ जि॒नाति॑। सम॒हमि॑न्द्रि॒येण॑ वी॒र्ये॑णेत्या॑ह॥५५॥

%1.7.9.4
इ॒न्द्रि॒यमे॒व वी॒र्य॑मा॒त्मन्ध॑त्ते। प॒शू॒नां म॒न्युर॑सि॒ तवे॑व मे म॒न्युर्भू॑या॒दिति॒ वारा॑ही उपा॒नहा॒वुप॑ मुञ्चते। प॒शू॒नां वा ए॒ष म॒न्युः। यद्व॑रा॒हः। तेनै॒व प॑शू॒नां म॒न्युमा॒त्मन्ध॑त्ते। अ॒भि वा इ॒य सु॑षुवा॒णङ्का॑मयते। तस्येश्व॒रेन्द्रि॒यं वी॒र्य॑मादा॑तोः। वारा॑ही उपा॒नहा॒वुप॑मुञ्चते। अ॒स्या ए॒वान्तर्ध॑त्ते। इ॒न्द्रि॒यस्य॑ वी॒र्य॑स्यानात्यै॥५६॥

%1.7.9.5
नमो॑ मा॒त्रे पृ॑थि॒व्या इत्या॒हाहिसायै। इय॑द॒स्यायु॑र॒स्यायु॑र्मे धे॒हीत्या॑ह। आयु॑रे॒वात्मन्ध॑त्ते। ऊर्ग॒स्यूर्जं॑ मे धे॒हीत्या॑ह। ऊर्ज॑मे॒वात्मन्ध॑त्ते। युङ्ङ॑सि॒ वर्चो॑सि॒ वर्चो॒ मयि॑ धे॒हीत्या॑ह। वर्च॑ ए॒वात्मन्ध॑त्ते। ए॒क॒धा ब्र॒ह्मण॒ उप॑हरति। ए॒क॒धैव यज॑मान॒ आयु॒रूर्जं॒ वर्चो॑ दधाति। र॒थ॒वि॒मो॒च॒नीया॑ जुहोति॒ प्रति॑ष्ठित्यै॥५७॥

%1.7.9.6
त्रयोऽश्वा॑ भवन्ति। रथ॑श्चतु॒र्थः। तस्माच्च॒तुर्जु॑होति। यदु॒भौ स॒हाव॒तिष्ठे॑ताम्। स॒मा॒नं लो॒कमि॑याताम्। स॒ह स॑ङ्ग्रही॒त्रा र॑थ॒वाह॑ने॒ रथ॒माद॑धाति। सु॒व॒र्गादे॒वैनं॑ लो॒काद॒न्तर्द॑धाति। ह॒सः शु॑चि॒षदित्याद॑धाति। ब्रह्म॑णै॒वैन॑मुपाव॒हर॑ति। ब्रह्म॒णाऽऽद॑धाति। अति॑च्छन्द॒साऽऽद॑धाति। अति॑च्छन्दा॒ वै सर्वा॑णि॒ छन्दासि। सर्वे॑भिरे॒वैन॒ञ्छन्दो॑भि॒राद॑धाति। वर्ष्म॒ वा ए॒षा छन्द॑साम्। यदति॑च्छन्दाः। यदति॑च्छन्दसा॒ दधा॑ति। वर्ष्मै॒वैन समा॒नानां करोति॥५८॥\anuvakamend[प॒द्य॒न्ते॒ द॒धा॒ति॒ वी॒र्ये॑णेत्या॒हानात्यै॒ प्रति॑ष्ठित्यै॒ ब्रह्म॒णाऽऽद॑धाति स॒प्त च॑]

%1.7.10.1
मि॒त्रो॑ऽसि॒ वरु॑णो॒ऽसीत्या॑ह। मै॒त्रं वा अह॑। वा॒रु॒णी रात्रि॑। अ॒हो॒रा॒त्राभ्या॑मे॒वैन॑मु॒पाव॑हरति। मि॒त्रो॑ऽसि॒ वरु॑णो॒ऽसीत्या॑ह। मै॒त्रो वै दक्षि॑णः। वा॒रु॒णः स॒व्यः। वै॒श्व॒दे॒व्या॑मिक्षा। स्वमे॒वैनौ॑ भाग॒धेय॑मु॒पाव॑हरति। सम॒हं विश्वैर्दे॒वैरित्या॑ह॥५९॥

%1.7.10.2
वै॒श्व॒दे॒व्यो॑ वै प्र॒जाः। ता ए॒वाद्या कुरुते। क्ष॒त्रस्य॒ नाभि॑रसि क्ष॒त्रस्य॒ योनि॑र॒सीत्य॑धीवा॒समास्तृ॑णाति सयोनि॒त्वाय॑। स्यो॒नामा सी॑द सु॒षदा॒मा सी॒देत्या॑ह। य॒था॒य॒जुरे॒वैतत्। मा त्वा॑ हिसी॒न्मा मा॑ हिसी॒दित्या॒हाहिसायै। निष॑साद धृ॒तव्र॑तो॒ वरु॑णः प॒स्त्यास्वा साम्राज्याय सु॒क्रतु॒रित्या॑ह। साम्राज्यमे॒वैन सु॒क्रतुं॑ करोति। ब्रह्मा(३)न्त्व रा॑जन्ब्र॒ह्माऽसि॑ सवि॒ताऽसि॑ स॒त्यस॑व॒ इत्या॑ह। स॒वि॒तार॑मे॒वैन स॒त्यस॑वं करोति॥६०॥

%1.7.10.3
ब्रह्मा(३)न्त्व रा॑जन्ब्र॒ह्माऽसीन्द्रो॑ऽसि स॒त्यौजा॒ इत्या॑ह। इन्द्र॑मे॒वैन स॒त्यौज॑सं करोति। ब्रह्मा(३)न्त्व रा॑जन्ब्र॒ह्माऽसि॑ मि॒त्रो॑ऽसि सु॒शेव॒ इत्या॑ह। मि॒त्रमे॒वैन सु॒शेवं॑ करोति। ब्रह्मा(३)न्त्व रा॑जन्ब्र॒ह्मासि॒ वरु॑णोऽसि स॒त्यध॒र्मेत्या॑ह। वरु॑णमे॒वैन स॒त्यध॑र्माणं करोति। स॒वि॒ताऽसि॑ स॒त्यस॑व॒ इत्या॑ह। गा॒य॒त्रीमे॒वैतेना॑भि॒ व्याह॑रति। इन्द्रो॑ऽसि स॒त्यौजा॒ इत्या॑ह। त्रि॒ष्टुभ॑मे॒वैतेना॑भि॒ व्याह॑रति॥६१॥

%1.7.10.4
मि॒त्रो॑ऽसि सु॒शेव॒ इत्या॑ह। जग॑तीमे॒वैतेना॑भि॒ व्याह॑रति। स॒त्यमे॒ता दे॒वता। स॒त्यमे॒तानि॒ छन्दासि। स॒त्यमे॒वाव॑रुन्धे। वरु॑णोऽसि स॒त्यध॒र्मेत्या॑ह। अ॒नु॒ष्टुभ॑मे॒वैतेना॑भि॒ व्याह॑रति। स॒त्या॒नृ॒ते वा अ॑नु॒ष्टुप्। स॒त्या॒नृ॒ते वरु॑णः। स॒त्या॒नृ॒ते ए॒वाव॑रुन्धे॥६२॥

%1.7.10.5
नैन सत्यानृ॒ते उ॑दि॒ते हिस्तः। य ए॒वं वेद॑। इन्द्र॑स्य॒ वज्रो॑ऽसि॒ वार्त्र॑घ्न॒ इति॒ स्प्यं प्रय॑च्छति। वज्रो॒ वै स्प्यः। वज्रे॑णै॒वास्मा॑ अवरप॒र र॑न्धयति। ए॒व हि तच्छ्रेय॑। यद॑स्मा ए॒ते रध्ये॑युः। दिशो॒ऽभ्य॑य राजा॑ऽभू॒दिति॒ पञ्चा॒क्षान्प्रय॑च्छति। ए॒ते वै सर्वेऽया। अप॑राजायिनमे॒वैनं॑ करोति॥६३॥

%1.7.10.6
ओ॒द॒नमुद्ब्रु॑वते। प॒र॒मे॒ष्ठी वा ए॒षः। यदो॑द॒नः। प॒र॒मामे॒वैन॒ श्रिय॑ङ्गमयति। सुश्लो॒काँ (४) सुम॑ङ्ग॒लाँ (४) सत्य॑रा॒जा (३) नित्या॑ह। आ॒शिष॑मे॒वैतामा शास्ते। शौ॒न॒ शे॒पमाख्या॑पयते। व॒रु॒ण॒पा॒शादे॒वैनं॑ मुञ्चति। प॒र॒ श॒तं भ॑वति। श॒तायु॒ पुरु॑षः श॒तेन्द्रि॑यः। आयु॑ष्ये॒वेन्द्रि॒ये प्रति॑ तिष्ठति। मा॒रु॒तस्य॒ चैक॑विशतिकपालस्य वैश्वदे॒व्यै चा॒मिक्षा॑या अ॒ग्नये स्विष्ट॒कृते॑ स॒मव॑द्यति। दे॒वता॑भिरे॒वैन॑मुभ॒यत॒ परि॑ गृह्णाति। अ॒पान्नप्त्रे॒ स्वाहो॒र्जो नप्त्रे॒ स्वाहा॒ऽग्नये॑ गृ॒हप॑तये॒ स्वाहेति॑ ति॒स्र आहु॑तीर्जुहोति। त्रय॑ इ॒मे लो॒काः। ए॒ष्वे॑व लो॒केषु॒ प्रति॑ तिष्ठति॥६४॥\anuvakamend[दे॒वैरित्या॑ह स॒त्यस॑वं करोति त्रि॒ष्टुभ॑मे॒वैतेना॑भि॒ व्याह॑रति सत्यानृ॒ते ए॒वाव॑रुन्धे करोति श॒तेन्द्रि॑य॒ष्षट् च॑]




\prashnaend{ए॒तद्ब्राह्मणानि धा॒त्रे र॒त्निनान्देवसु॒वाम॒र्थेतो॒ देवी॒र्दिश॒ सोम॒स्येन्द्र॑स्य मि॒त्रो दश॑॥१०॥}{ए॒तद्ब्राह्मणानि वैष्ण॒वन्त्रि॑कपा॒लमन्नं॒ वै पू॒षा वाशा॒ स्थेत्या॑ह॒ दिशो॒ व्यास्था॑पय॒त्युद॑ङ्प॒रेत्य॒ ब्रह्मा ३ न्त्व रा॑ज॒श्चतु॑ष्षष्टिः॥६४॥}{ए॒तद्ब्राह्मणानि॒ प्रति॑तिष्ठति॥}{हरि॑ ओम्॥}{इति श्रीकृष्णयजुर्वेदीयतैत्तिरीयब्राह्मणे प्रथमाष्टके सप्तमः प्रपाठकः समाप्तः॥}
\clearpage
